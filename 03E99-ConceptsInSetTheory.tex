\documentclass[12pt]{article}
\usepackage{pmmeta}
\pmcanonicalname{ConceptsInSetTheory}
\pmcreated{2013-03-22 14:13:14}
\pmmodified{2013-03-22 14:13:14}
\pmowner{matte}{1858}
\pmmodifier{matte}{1858}
\pmtitle{concepts in set theory}
\pmrecord{54}{35656}
\pmprivacy{1}
\pmauthor{matte}{1858}
\pmtype{Topic}
\pmcomment{trigger rebuild}
\pmclassification{msc}{03E99}
\pmrelated{Set}
\pmrelated{PolynomialFunction}

\endmetadata

\usepackage{amssymb}
\usepackage{amsmath}
\usepackage{amsfonts}

%\usepackage{fca}

\newcommand{\sR}[0]{\mathbb{R}}
\newcommand{\sC}[0]{\mathbb{C}}
\newcommand{\sN}[0]{\mathbb{N}}
\newcommand{\sZ}[0]{\mathbb{Z}}

%\usepackage{bbm}
 \newcommand{\Z}{\mathbbmss{Z}}
 \newcommand{\C}{\mathbbmss{C}}
 \newcommand{\R}{\mathbbmss{R}}
 \newcommand{\Q}{\mathbbmss{Q}}



\newcommand*{\norm}[1]{\lVert #1 \rVert}
\newcommand*{\abs}[1]{| #1 |}
\begin{document}
\PMlinkescapeword{simple}
\PMlinkescapeword{mean}
The aim of this entry is to present a list of the key objects and 
concepts used in set theory. Each entry in the list links 
(or will link in the future) to the corresponding PlanetMath 
entry where the object is presented in greater detail. 
For convenience, this list also presents the encouraged 
notation to use (at PlanetMath) for these objects.

%The aim of this entry is to present a list of common notation in set theory.  %Where more than one notation is in common use, several options are listed; %PlanetMath authors are encouraged to use the first for consistency.

\begin{itemize}
\item set
\item set axioms
\item Venn diagrams
\item $\emptyset$, the empty set (also $\{\}$ or $\varnothing$),
\item $\{x\}$, singleton,
\item $\{a_1,\,a_2,\,a_3,\,...\}$ ({\em list form}), the set with elements 
$a_1,\,a_2,\,a_3,\,...$,
\item $x \in A$, $x$ is an element of the set $A$,
\item $A\ni x$, $A$ is a set containing $x$,
\item $x \notin A$, $x$ is not an element of the set $A$,
\item $A\cup B$, union of sets $A$ and $B$,
\item $\bigcup_{i\in I} A_i$, union of a family of sets $A_i$ indexed by elements of $I$,
\item $A\coprod B$, disjoint union (or $A\overset{\cdot}{\cup} B$),
\item $A\cap B$, intersection of sets $A$ and $B$,
\item $\bigcap_{i\in I} A_i$, intersection of a family of sets $A_i$, indexed by elements  in $I$,
\item $A\!\smallsetminus\!B$, set difference. An alternative notation for this is $A-B$, which  should be avoided since in the context of vector spaces, $A-B$ is 
used for the set of all elements of the form $a-b$ (see \PMlinkname{Minkowski sum}{MinkowskiSum2}),
\item $A/_\sim$, set of equivalence classes in $A$ determined by an equivalence 
relation $\sim$ in $A$, 
\item $[a]$, equivalence class in $A/_\sim$ generated by $a\in A$,
\item $A^\complement$, set complement of $A$ (where the ambient 
set containing $A$ is understood from context),
\item $A\,\triangle\ B$, symmetric set difference of $A$ and $B$,
\item $A\times B$, Cartesian product of $A$ and $B$,
\item $\prod_{i \in I}A_i$, Cartesian product of the sets $A_i$ (sometimes also {\LARGE $\times$}$_{i \in I}A_i$),
\item $\operatorname{id}_X$, identity mapping $X\to X$,
\item $\mathcal{P}(A)$, power set of $A$ (also $2^{A}$),
\item $^AB$ or $B^A$, the set of functions from $A$ to $B$ (rare outside of logic and set theory),
\item $f\colon A\to B$,\, $f$ is a function having domain $A$ and codomain $B$, 
\item $\operatorname{card}(A)$, cardinality of $A$ (also $\sharp A$ or $\left|A\right|$, which can be confused with the absolute value),
\item $A = B$, $A$ and $B$ are equal (generally as sets; occasionally this notation is used to mean ``$A$ is canonically isomorphic to $B$''),
\item $A \subseteq B$, $A$ is a subset of $B$ (or $A\subset B$, especially in set theory and logic),
\item $A \subsetneq B$, $A$ is a proper subset (that is, $A\subseteq B$ but $A\neq B$; occasionally authors will use $A\subset B$ to mean ``proper subset'', conflicting with the above),
\item $A \supseteq B$, $A$ is a superset of $B$ (with the same caveats as the previous entries).
\item discrete topology
\end{itemize}

\subsubsection*{Set builder notation}
\[
\left\{ x\in A \text{ such that }\textit{condition} \right\}.
\]
When $A$ is obvious it may be omitted.  Other symbols are also sometimes used in place of the words ``such that'', for example
\[
\left\{ x\in A \mid \textit{condition} \right\},
\]
\[
\left\{ x\in A : \textit{condition} \right\},
\]
\[
\{ x\in A \,\vdots \,\,\,condition \}
\]
or
\[
\left\{ x\in A \text{ s.t. } \textit{condition}\right\}.
\]

The reader should take care that if the objects under discussion are not just sets (say, groups or schemes) the operations may not be simple set operations, but rather their analogue in the relevant category.  For example, the product of two groups is usually assigned a group law of a particular form, while the product of two schemes has ``extra'' points beyond those obtained from the Cartesian product of the schemes.  Such conventions will normally be defined along with the category itself, although occasionally they will be an example of a general notion defined the same way in all categories (such as the categorical direct product).
%%%%%
%%%%%
\end{document}
