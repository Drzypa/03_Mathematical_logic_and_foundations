\documentclass[12pt]{article}
\usepackage{pmmeta}
\pmcanonicalname{InductionAxiom}
\pmcreated{2013-03-22 12:56:51}
\pmmodified{2013-03-22 12:56:51}
\pmowner{Henry}{455}
\pmmodifier{Henry}{455}
\pmtitle{induction axiom}
\pmrecord{7}{33306}
\pmprivacy{1}
\pmauthor{Henry}{455}
\pmtype{Definition}
\pmcomment{trigger rebuild}
\pmclassification{msc}{03F35}
\pmsynonym{IND}{InductionAxiom}
\pmsynonym{-IND}{InductionAxiom}
\pmsynonym{axiom of induction}{InductionAxiom}

% this is the default PlanetMath preamble.  as your knowledge
% of TeX increases, you will probably want to edit this, but
% it should be fine as is for beginners.

% almost certainly you want these
\usepackage{amssymb}
\usepackage{amsmath}
\usepackage{amsfonts}

% used for TeXing text within eps files
%\usepackage{psfrag}
% need this for including graphics (\includegraphics)
%\usepackage{graphicx}
% for neatly defining theorems and propositions
%\usepackage{amsthm}
% making logically defined graphics
%%%\usepackage{xypic}

% there are many more packages, add them here as you need them

% define commands here
%\PMlinkescapeword{theory}
\begin{document}
An induction axiom specifies that a theory includes induction, possibly restricted to specific formulas.  IND is the general axiom of induction:
$$\phi(0)\wedge\forall x(\phi(x)\rightarrow\phi(x+1))\rightarrow \forall x\phi(x)\text{ for any formula }\phi$$

If $\phi$ is restricted to some family of formulas $F$ then the axiom is called F-IND, or F induction.  For example the axiom $\Sigma^0_1$-IND is:
$$\phi(0)\wedge\forall x(\phi(x)\rightarrow\phi(x+1))\rightarrow \forall x\phi(x)\text{ where }\phi\text{ is }\Sigma^0_1$$
%%%%%
%%%%%
\end{document}
