\documentclass[12pt]{article}
\usepackage{pmmeta}
\pmcanonicalname{ExampleOfMonadicAlgebra}
\pmcreated{2013-03-22 17:51:55}
\pmmodified{2013-03-22 17:51:55}
\pmowner{CWoo}{3771}
\pmmodifier{CWoo}{3771}
\pmtitle{example of monadic algebra}
\pmrecord{10}{40343}
\pmprivacy{1}
\pmauthor{CWoo}{3771}
\pmtype{Example}
\pmcomment{trigger rebuild}
\pmclassification{msc}{03G15}
\pmdefines{functional monadic algebra}

\endmetadata

\usepackage{amssymb,amscd}
\usepackage{amsmath}
\usepackage{amsfonts}
\usepackage{mathrsfs}

% used for TeXing text within eps files
%\usepackage{psfrag}
% need this for including graphics (\includegraphics)
%\usepackage{graphicx}
% for neatly defining theorems and propositions
\usepackage{amsthm}
% making logically defined graphics
%%\usepackage{xypic}
\usepackage{pst-plot}

% define commands here
\newcommand*{\abs}[1]{\left\lvert #1\right\rvert}
\newtheorem{prop}{Proposition}
\newtheorem{thm}{Theorem}
\newtheorem{ex}{Example}
\newcommand{\real}{\mathbb{R}}
\newcommand{\pdiff}[2]{\frac{\partial #1}{\partial #2}}
\newcommand{\mpdiff}[3]{\frac{\partial^#1 #2}{\partial #3^#1}}
\begin{document}
The canonical example of a monadic algebra is what is known as a \emph{functional monadic algebra}, which is explained in this entry.

Let $A$ be a Boolean algebra and $X$ be a non-empty set.  Then $A^X$, the set of all functions from $X$ into $A$, has a natural Boolean algebraic structure defined as follows:
$$(f\wedge g)(x):=f(x)\wedge g(x),\qquad (f')(x):=f(x)',\qquad 1(x)=1$$
where $f,g:X\to A$ are functions, and $1:X\to A$ is just the constant function mapping everything to $1\in A$ (the abuse of notation here is harmless).

For each $f:X\to A$, let $f(X)\subseteq A$ be the range of $f$.  Let $B$ be the subset of $A^X$ consisting of all functions $f$ such that $\bigvee f(X)$ and $\bigwedge f(X)$ exist, where $\bigvee$ and $\bigwedge$ are the infinite join and infinite meet operations on $A$.  In other words, $$B:=\lbrace f\in A^X\mid \bigvee f(X)\in A\mbox{ and }\bigwedge f(X)\in A\rbrace.$$
\begin{prop} $B$ defined above is a Boolean subalgebra of $A^X$. \end{prop}
\begin{proof}
We need to show that, (1): $1\in B$, (2): for any $f\in B$, $f'\in B$, and (3): for any $f,g\in B$, $f\wedge g\in B$.
\begin{enumerate}
\item $\bigvee 1(X)=\bigvee \lbrace 1\rbrace =1$ and $\bigwedge 1(X)=\bigwedge \lbrace 1\rbrace =1$ so $1\in B$
\item Suppose $f\in B$.  Then $\bigvee f'(X)=\bigvee \lbrace f'(x)\mid x\in X\rbrace = \bigvee \lbrace f(x)'\mid x\in X\rbrace$.  By de Morgan's law on infinite joins, the last expression is $(\bigwedge \lbrace f(x)\mid x\in X\rbrace )'$, which exists.  Dually, $\bigwedge f'(X)$ exists by de Morgan's law on infinite meets.  Therefore, $f'\in B$.
\item Suppose $f,g\in B$.  Then 
\begin{eqnarray*}
\bigwedge (f\wedge g)(X) &=& \bigwedge \lbrace f(x)\wedge g(x)\mid x\in X\rbrace \\ 
&=& \bigwedge \lbrace f(x)\mid x\in X\rbrace \wedge \bigwedge \lbrace g(x) \mid x\in X\rbrace \\ 
&=& \bigwedge f(X)\wedge \bigwedge g(X),
\end{eqnarray*} 
which exists because both $\bigwedge f(X)$ and $\bigwedge g(X)$ do.  In addition, $$\bigvee (f\wedge g)(X)=\bigvee \lbrace f(x)\wedge g(x)\mid x\in X\rbrace =\bigvee f(X)\wedge \bigvee g(X).$$  The last equality stems from the distributive law of infinite meets over finite joins.  Since the last expression exists, $f\wedge g \in B$.
\end{enumerate}
The three conditions are verified and the proof is complete.
\end{proof}
\textbf{Remark}.  Every constant function belongs to $B$.

For each $f\in B$, write $f^{\vee}:=\bigvee f(X)$ and $f^{\wedge}:=\bigwedge f(X)$.  Define two functions $f^{\exists},f^{\forall}\in A^X$ by $$f^{\exists}(x):=f^{\vee}\qquad\mbox{ and }\qquad f^{\forall}(x):=f^{\wedge}.$$
Since these are constant functions, they belong to $B$.

Now, we define operators $\exists ,\forall$ on $B$ by setting $$\exists (f):=f^{\exists}\qquad \mbox { and }\qquad \forall (f): = f^{\forall}.$$
By the remark above, $\exists$ and $\forall$ are well-defined functions on $B$ ($f^{\exists},f^{\forall}\in B$).

\begin{prop} $\exists$ is an existential quantifier operator on $B$ and $\forall$ is its dual. \end{prop}
\begin{proof} The following three conditions need to be verified:
\begin{itemize}
\item $\exists(0)=0$: $\exists(0)(x)=0^{\exists}(x)=0^{\vee}=\bigvee 0(X)=\bigvee 0=0$.
\item $f\le \exists(f)$: $f(x)\le \bigvee f(X)=f^{\vee}=f^{\exists}(x)=\exists(f)(x)$.
\item $\exists(f\wedge \exists(g))=\exists(f)\wedge \exists(g)$: 
\begin{eqnarray*}
\exists(f\wedge \exists(g))(x) &=& \bigvee \big(f\wedge \exists(g)\big)(X) = \bigvee \big(f(X)\wedge \exists(g)(X)\big) \\ &=& \bigvee \big(f(X)\wedge \exists(g)(x)\big) = \bigvee \big(f(X)\wedge \bigvee g(X)\big) \\
&=& \bigvee f(X) \wedge \bigvee g(X) = \exists(f)(x) \wedge \exists(g)(x) = \big(\exists(f)\wedge \exists(g)\big)(x).
\end{eqnarray*}
\end{itemize}
Finally, to see that $\forall$ is the dual of $\exists$, we do the following computations: 
\begin{eqnarray*}
\forall (f)(x) &=& \bigwedge \lbrace f(x)\mid x\in X\rbrace =\bigwedge \lbrace f(x)'' \mid x\in X\rbrace \\ &=& \big( \bigvee \lbrace f(x)'\mid x\in X\rbrace \big)'=\big( \bigvee \lbrace f'(x)\mid x\in X\rbrace \big)'=\big(\exists(f')\big)'(x),
\end{eqnarray*}
completing the proof.
\end{proof}

Based on Propositions 1 and 2, $(B,\exists)$ is a monadic algebra, and is called the \emph{functional monadic algebra} for the pair $(A,X)$.
%%%%%
%%%%%
\end{document}
