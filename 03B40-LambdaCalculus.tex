\documentclass[12pt]{article}
\usepackage{pmmeta}
\pmcanonicalname{LambdaCalculus}
\pmcreated{2013-03-22 12:32:39}
\pmmodified{2013-03-22 12:32:39}
\pmowner{ratboy}{4018}
\pmmodifier{ratboy}{4018}
\pmtitle{lambda calculus}
\pmrecord{9}{32788}
\pmprivacy{1}
\pmauthor{ratboy}{4018}
\pmtype{Definition}
\pmcomment{trigger rebuild}
\pmclassification{msc}{03B40}
\pmrelated{CombinatoryLogic}
\pmrelated{ChurchInteger}
\pmrelated{RussellsParadox}
\pmdefines{pure lambda calculus}
\pmdefines{lambda abstraction}
\pmdefines{lambda expression}

\usepackage{amssymb}
\usepackage{amsmath}
\usepackage{amsfonts}
\DeclareMathOperator{\inc}{inc}
\DeclareMathOperator{\add}{add}
\DeclareMathOperator{\mul}{mul}
\DeclareMathOperator{\Not}{not}
\begin{document}
\emph{Lambda calculus} (often referred to as $\lambda$-calculus) was invented in the 1930s by Alonzo Church, as a form of mathematical logic dealing primarly with functions and the application of functions to their arguments.
In \emph{pure lambda calculus}, there are no constants.  Instead, there are only \emph{lambda abstractions} (which are simply specifications of functions), variables, and applications of functions to functions.  For instance, Church integers are used as a substitute for actual constants representing integers.

A lambda abstraction is typically specified using a \emph{lambda expression}, which might look like the following.

$$
\lambda\;x\;.\;f\;x
$$

The above specifies a function of one argument, that can be \emph{reduced} by applying the function $f$ to its argument (function application is left-associative by default, and parentheses can be used to specify associativity).

The $\lambda$-calculus is equivalent to combinatory logic (though much more concise).  Most functional programming languages are also equivalent to $\lambda$-calculus, to a degree (any imperative features in such languages are, of course, not equivalent).

\subsubsection*{Examples}

We can specify the Church integer $3$ in $\lambda$-calculus as

$$
3 = \lambda f\;x\;.\;f\;(f\;(f\;x))
$$

Suppose we have a function $\inc$, which when given a string representing an integer, returns a new string representing the number following that integer.
Then

$$
3\;\inc\;\texttt{"0"} = \texttt{"3"}
$$

Addition of Church integers in $\lambda$-calculus is

\begin{eqnarray*}
\add & = & \lambda\;x\;y\;.\;(\lambda\;f\;z\;.\;x\;f\;(y\;f\;z)) \\
\add\;2\;3 & = & \lambda\;f\; z\; .\; 2\; f\; (3\; f\; z) \\
        & = & \lambda\; f\; z\; .\; 2\; f\; (f\; (f\; (f\; z))) \\
        & = & \lambda\; f\; z\; .\; f\; (f\; (f\; (f\; (f\; z)))) \\
        & = & 5
\end{eqnarray*}

Multiplication is

\begin{eqnarray*}
\mul & = & \lambda\;x\;y\;.\;(\lambda\;f\;z\;.\;x\;(\lambda\;w\;.\;y\;f\;w)\;z) \\
\mul\;2\;3 & = & \lambda\;f\;z\;.\;2\;(\lambda\;w\;.\;3\;f\;w)\;z \\
& = & \lambda\;f\;z\;.\;2\;(\lambda\;w\;.\;f\;(f\;(f\;w)))\;z \\
& = & \lambda\;f\;z\;.\;f\;(f\;(f\;(f\;(f\;(f\;z))))) \\
& = & 6.
\end{eqnarray*}

\subsubsection*{Russell's Paradox in $\lambda$-calculus}

The $\lambda$-calculus readily admits Russell's Paradox.
Let us define a function $r$ that takes a function $x$ as an argument, and is reduced to the application of the logical function $\Not$ to the application of $x$ to itself.

$$
r\;=\;\lambda\;x\;.\;\Not\;(x\;x)
$$

Now what happens when we apply $r$ to itself?

\begin{eqnarray*}
r\;r & = & \Not\;(r\;r) \\
    & = & \Not\;(\Not\;(r\;r)) \\
    & \vdots &
\end{eqnarray*}

Since we have $\Not\;(r\;r) = (r\;r)$, we have a paradox.
%%%%%
%%%%%
\end{document}
