\documentclass[12pt]{article}
\usepackage{pmmeta}
\pmcanonicalname{KripkeSemantics}
\pmcreated{2013-03-22 19:31:15}
\pmmodified{2013-03-22 19:31:15}
\pmowner{CWoo}{3771}
\pmmodifier{CWoo}{3771}
\pmtitle{Kripke semantics}
\pmrecord{25}{42496}
\pmprivacy{1}
\pmauthor{CWoo}{3771}
\pmtype{Definition}
\pmcomment{trigger rebuild}
\pmclassification{msc}{03B48}
\pmclassification{msc}{03B20}
\pmclassification{msc}{03B45}
\pmdefines{Kripke frame}
\pmdefines{possible world}
\pmdefines{accessibility relation}
\pmdefines{accessible}
\pmdefines{valid}
\pmdefines{sound}
\pmdefines{complete}

\usepackage{amssymb,amscd}
\usepackage{amsmath}
\usepackage{amsfonts}
\usepackage{mathrsfs}

% used for TeXing text within eps files
%\usepackage{psfrag}
% need this for including graphics (\includegraphics)
%\usepackage{graphicx}
% for neatly defining theorems and propositions
\usepackage{amsthm}
% making logically defined graphics
%%\usepackage{xypic}
\usepackage{pst-plot}

% define commands here
\newcommand*{\abs}[1]{\left\lvert #1\right\rvert}
\newtheorem{prop}{Proposition}
\newtheorem{thm}{Theorem}
\newtheorem{ex}{Example}
\newcommand{\real}{\mathbb{R}}
\newcommand{\pdiff}[2]{\frac{\partial #1}{\partial #2}}
\newcommand{\mpdiff}[3]{\frac{\partial^#1 #2}{\partial #3^#1}}

\begin{document}
A \emph{Kripke frame} (or simply a frame) $\mathcal{F}$ is a pair $(W,R)$ where $W$ is a non-empty set whose elements are called \emph{worlds} or \emph{possible worlds}, $R$ is a binary relation on $W$ called the \emph{accessibility relation}.  When $v R w$, we say that $w$ is \emph{accessible} from $v$.  A Kripke frame is said to have property $P$ if $R$ has the property $P$.  For example, a symmetric frame is a frame whose accessibility relation is symmetric.

A \emph{Kripke model} (or simply a model) $M$ for a propositional logical system (classical, intuitionistic, or modal) $\Lambda$ is a pair $(\mathcal{F},V)$, where $\mathcal{F}:=(W,R)$ is a Kripke frame, and $V$ is a function that takes each atomic formula of $\Lambda$ to a subset of $W$.  If $w\in V(p)$, we say that $p$ is \emph{true} at world $w$.  We say that $M$ is a $\Lambda$-model \emph{based on} the frame $\mathcal{F}$ if $M=(\mathcal{F},V)$ is a model for the logic $\Lambda$.

\textbf{Remark}.  Associated with each world $w$, we may also define a Boolean-valued valuation $V_w$ on the set of all wff's of $\Lambda$, so that $V_w(p)=1$ iff $w \in V(p)$.  In this sense, the Kripke semantics can be thought of as a generalization of the truth-value semantics for classical propositional logic.  The truth-value semantics is just a Kripke model based on a frame with one world.  Conversely, given a collection of valuations $\lbrace V_w\mid w\in W\rbrace$, we have model $(\mathcal{F},V)$ where $w\in V(p)$ iff $V_w(p)=1$.

Since the well-formed formulas (wff's) of $\Lambda$ are uniquely readable, $V$ may be inductively extended so it is defined on all wff's.  The following are some examples:
\begin{itemize}
\item in classical propositional logic PL$_c$, $V(A\to B):=V(A)^c\cup V(B)$, where $S^c:=W-S$,
\item in the modal propositional logic $K$, $V(\square A):= V(A)^{\square}$, where $S^{\square}:=\lbrace u \mid\; \uparrow\!\! u \subseteq S \rbrace$, and $\uparrow\!\! u: =\lbrace w \mid u R w\rbrace$, and
\item in intuitionistic propositional logic PL$_i$, $V(A\to B):=(V(A)- V(B))^{\#}$, where $S^{\#}:=(\downarrow\!\! S)^c$, and $\downarrow\!\!S: = \lbrace u \mid u R w, w\in S\rbrace$.
\end{itemize}

Truth at a world can now be defined for wff's: a wff $A$ is \emph{true} at world $w$ if $w\in V(A)$, and we write $$M \models_w A\qquad\mbox{or}\qquad \models_w A$$
if no confusion arises.  If $w\notin V(A)$, we write $M \not\models_w A$.  The three examples above can be now interpreted as:
\begin{itemize}
\item $\models_w A\to B$ means $\models_w A$ implies $\models_w B$ in PL$_c$, 
\item $\models_w \square A$ means for all worlds $v$ with $wRv$, we have $\models_v A$ in $K$, and
\item $\models_w A\to B$ means for all worlds $v$ with $wRv$, $\models_v A$ implies $\models_v B$ in PL$_i$.
\end{itemize}

A wff $A$ is said to be \emph{valid} 
\begin{itemize}
\item in a model $M$ if $A$ in true at all possible worlds $w$ in $M$,
\item in a frame if $A$ is valid in all models $M$ based on $\mathcal{F}$,
\item in a collection $\mathcal{C}$ of frames if $A$ is valid in all frames in $\mathcal{C}$.
\end{itemize}
We denote
$$M\models A, \qquad \mathcal{F} \models A, \qquad \mbox{or} \qquad \mathcal{C} \models A$$
if $A$ is valid in $M,\mathcal{F}$, or $\mathcal{C}$ respectively.

A logic $\Lambda$, equipped with a deductive system, is \emph{sound} in $\mathcal{C}$ if $$\vdash A \qquad \mbox{implies}\qquad \mathcal{C} \models A.$$  Here, $\vdash A$ means that wff $A$ is a theorem deducible from the deductive system of $\Lambda$.   Conversely, if $$\mathcal{C} \models A\qquad \mbox{implies} \qquad \vdash A,$$ we say that $\Lambda$ is \emph{complete} in $\mathcal{C}$.

%%%%%
%%%%%
\end{document}
