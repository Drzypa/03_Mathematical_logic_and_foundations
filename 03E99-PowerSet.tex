\documentclass[12pt]{article}
\usepackage{pmmeta}
\pmcanonicalname{PowerSet}
\pmcreated{2013-03-22 11:43:46}
\pmmodified{2013-03-22 11:43:46}
\pmowner{matte}{1858}
\pmmodifier{matte}{1858}
\pmtitle{power set}
\pmrecord{23}{30136}
\pmprivacy{1}
\pmauthor{matte}{1858}
\pmtype{Definition}
\pmcomment{trigger rebuild}
\pmclassification{msc}{03E99}
\pmclassification{msc}{03E10}
\pmclassification{msc}{37-01}
\pmsynonym{powerset}{PowerSet}
%\pmkeywords{Set}
%\pmkeywords{Power}
%\pmkeywords{Cardinality}
\pmrelated{PowerObject}
\pmrelated{ProofOfGeneralAssociativity}
\pmdefines{finite power set}
\pmdefines{finite powerset}

% this is the default PlanetMath preamble.  as your knowledge
% of TeX increases, you will probably want to edit this, but
% it should be fine as is for beginners.

% almost certainly you want these
\usepackage{amssymb}
\usepackage{amsmath}
\usepackage{amsfonts}

% used for TeXing text within eps files
%\usepackage{psfrag}
% need this for including graphics (\includegraphics)
%\usepackage{graphicx}
% for neatly defining theorems and propositions
%\usepackage{amsthm}
% making logically defined graphics
%%%%%%%%\usepackage{xypic}

% there are many more packages, add them here as you need them

% define commands here

\newcommand{\sR}[0]{\mathbb{R}}
\newcommand{\sC}[0]{\mathbb{C}}
\newcommand{\sN}[0]{\mathbb{N}}
\newcommand{\sZ}[0]{\mathbb{Z}}

\newcommand{\powset}[1]{\mathcal{P}(#1)}
\begin{document}
\PMlinkescapeword{states}
\PMlinkescapeword{property}
{\bf Definition}
If $X$ is a set, then the \emph{power set of $X$}, denoted by  $\powset{X}$, is the
set whose elements are the subsets of $X$. 

\subsubsection*{Properties}
\begin{enumerate}
\item If $X$ is finite, then $|\powset{X}|=2^{|X|}$.
\item The above property also holds when $X$ is not finite. 
For a set $X$, let $|X|$ be the cardinality of $X$. 
Then $|\powset{X}|=2^{|X|}=|2^X|$,
where $2^X$ is the set of all functions from $X$ to $\{0,1\}$.
\item For an arbitrary set $X$, Cantor's theorem states:
a) there is no bijection between $X$ and $\powset{X}$, and
b) the cardinality of $\powset{X}$ is greater than the cardinality of $X$.
\end{enumerate}

\subsubsection*{Example}
Suppose $S=\{a,b\}$. Then $\powset{S}=\{\emptyset, \{a\}, \{b\}, S\}$.
In particular, $|\powset{S}|=2^{|S|}=4$. 

\subsubsection*{Related definition}
If $X$ is a set, then the \emph{finite power set of $X$}, denoted by  $\mathcal{F}(X)$, is the
set whose elements are the {\bf finite} subsets of $X$. 

\subsubsection*{Remark}
Due to the canonical correspondence between elements of $\powset{X}$ and elements of $2^X$, the power set is sometimes also denoted by $2^X$.
%%%%%
%%%%%
%%%%%
%%%%%
%%%%%
%%%%%
%%%%%
\end{document}
