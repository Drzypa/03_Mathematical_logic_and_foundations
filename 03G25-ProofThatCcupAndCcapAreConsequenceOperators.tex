\documentclass[12pt]{article}
\usepackage{pmmeta}
\pmcanonicalname{ProofThatCcupAndCcapAreConsequenceOperators}
\pmcreated{2013-03-22 16:29:41}
\pmmodified{2013-03-22 16:29:41}
\pmowner{rspuzio}{6075}
\pmmodifier{rspuzio}{6075}
\pmtitle{proof that $C_\cup$ and $C_\cap$ are consequence operators}
\pmrecord{21}{38670}
\pmprivacy{1}
\pmauthor{rspuzio}{6075}
\pmtype{Proof}
\pmcomment{trigger rebuild}
\pmclassification{msc}{03G25}
\pmclassification{msc}{03G10}
\pmclassification{msc}{03B22}

% this is the default PlanetMath preamble.  as your knowledge
% of TeX increases, you will probably want to edit this, but
% it should be fine as is for beginners.

% almost certainly you want these
\usepackage{amssymb}
\usepackage{amsmath}
\usepackage{amsfonts}

% used for TeXing text within eps files
%\usepackage{psfrag}
% need this for including graphics (\includegraphics)
%\usepackage{graphicx}
% for neatly defining theorems and propositions
\usepackage{amsthm}
% making logically defined graphics
%%%\usepackage{xypic}

% there are many more packages, add them here as you need them

% define commands here

\newtheorem{definition}{Definition}
\newtheorem{theorem}{Theorem}
\begin{document}
The proof that the operators $C_\cup$ and $C_\cap$ defined in the second 
example of section 3 of the
\PMlinkname{parent entry}{ConsequenceOperator} are consequence operators 
is a relatively straightforward matter of checking that they satisfy the defining properties given there.  For convenience, those definitions are
reproduced here.

\begin{definition}
Given a set $L$ and two elements, $X$ and $Y$, of this set, the 
function $C_\cap (X,Y) \colon \mathcal{P}(L) \to \mathcal{P}(L)$ 
is defined as follows:
 \[
  C_\cap (X,Y) (Z) = 
  \begin{cases} 
   X \cup Z & Y \cap Z \not= \emptyset \\ 
   Z & Y \cap Z = \emptyset 
  \end{cases} 
 \]
\end{definition}

\begin{theorem} 
For every choice of two elements, $X$ and $Y$, of a given set $L$, the 
function $C_\cap (X,Y)$ is a consequence operator. 
\end{theorem}

\begin{proof} ~

\emph{Property 1:}
Since $Z$ is a subset of itself and of $X \cup Z$, it follows that
$Z \subseteq C_\cap (X,Y) (Z)$ in either case.

\emph{Property 2:}
We consider two cases.  If $Y \cap Z = \emptyset$, then $C_\cap (X,Y) 
(Z) = Z$, so 
 \[C_\cap (X,Y) (C_\cap (X,Y) (Z)) = C_\cap (X,Y) (Z).\]
If $Y \cap Z \not= \emptyset$, then 
\begin{eqnarray*}
Y \cap C_\cap (X,Y) (Z) &=&
Y \cap (X \cup Z) \\
&=& (Y \cap X) \cup (Y \cap Z).
\end{eqnarray*}
Again, since $Y \cap Z \not= \emptyset$, we also
have $(Y \cap X) \cup (Y \cap Z) \not= \emptyset$, so
\begin{eqnarray*}
C_\cap (X,Y) (C_\cap (X,Y) (Z)) 
&=& X \cup C_\cap (X,Y) (Z) \\
&=& X \cup (X \cup Z) \\
&=& X \cup Z \\
&=& C_\cap (X,Y) (Z)
\end{eqnarray*}
So, in both cases, we find that 
\[C_\cap (X,Y) (C_\cap (X,Y) (Z)) = C_\cap (X,Y) (Z).\]

\emph{Property 3:}
Suppose that $Z$ and $W$ are subsets of $L$ and that $Z$ is a subset
of $W$.  Then there are three possibilities: 

1. $Y \cap Z = \emptyset$ and $Y \cap W = \emptyset$
 
In this case, we have $C_\cap (X,Y) (Z) = Z$ and
$C_\cap (X,Y) (W) = W$, so $C_\cap (X,Y) (Z) \subseteq C_\cap (X,Y) (W)$.

2. $Y \cap Z = \emptyset$ but $Y \cap W \not= \emptyset$ 

In this case, $C_\cap (X,Y) (Z) = Z$ and $C_\cap (X,Y) (W) = X \cup 
W$.  Since $Z \subseteq W$ implies $Z \subseteq X \cup W$, we have 
$C_\cap (X,Y) (Z) \subseteq C_\cap (X,Y) (W)$. 

3. $Y \cap Z \not= \emptyset$ and $Y \cap W \not= \emptyset$

In this case,
$C_\cap (X,Y) (Z) = X \cup Z$ and $C_\cap (X,Y) (W) = X \cup W$.  Since 
$Z \subseteq W$ implies $X \cup Z \subseteq X \cup W$, we have 
$C_\cap (X,Y) (Z) \subseteq C_\cap (X,Y) (W)$.

\end{proof}

\begin{definition}
Given a set $L$ and two elements, $X$ and $Y$, of this set, the 
function $C_\cup (X,Y) \colon \mathcal{P}(L) \to \mathcal{P}(L)$ 
is defined as follows:
 \[
  C_\cup (X,Y)(Z) = 
  \begin{cases}  
   X \cup Z & Y \cup Z = Z \\ 
   Z & Y \cup Z \not= Z 
  \end{cases}
 \]
\end{definition}

\begin{theorem} 
For every choice of two elements, $X$ and $Y$, of a given set $L$, the 
function $C_\cup (X,Y)$ is a consequence operator.
\end{theorem}

\begin{proof} ~

\emph{Property 1:}
Since $Z$ is a subset of itself and of $X \cup Z$, it follows that
$Z \subseteq C_\cup (X,Y) (Z)$ in either case.

\emph{Property 2:}
We consider two cases. If $C_\cup (X,Y) (Z) = Z$, then
 \[C_\cup (X,Y) (C_\cup (X,Y) (Z)) = C_\cup (X,Y) (Z).\]
If $C_\cup (X,Y) (Z) = X \cup Z$, then we note that, because 
$X \cup (X \cup Z) = X \cup Z$, we must have $C_\cup (X,Y) 
(X \cup Z) = X \cup Z$ whether or not $Y \cup (X \cup Z) = 
X \cup Z$, so 
 \[C_\cup (X,Y) (C_\cup (X,Y) (Z)) = C_\cup (X,Y) (Z).\]

\emph{Property 3:}
Suppose that $Z$ and $W$ are subsets of $L$ and that $Z$ is a subset
of $W$.  Then there are three possibilities: 

1. $Y \cup Z = Z$ and $Y \cup W = W$
 
In this case, we have $C_\cup (X,Y) (Z) = X \cup Z$ and
$C_\cup (X,Y) (W) = X \cup W$.  Since $Z \subseteq W$ implies 
$X \cup Z \subseteq X \cup W$, we have $C_\cup (X,Y) (Z) 
\subseteq C_\cup (X,Y) (W)$.

2. $Y \cup Z \not= Z$ but $Y \cup W = W$ 

In this case, $C_\cup (X,Y) (Z) = Z$ and $C_\cup (X,Y) (W) = X \cup W$. 
Since $Z \subseteq W$ implies $Z \subseteq X \cup W$, we have 
$C_\cup (X,Y) (Z) \subseteq C_\cup (X,Y) (W)$. 

3. $Y \cup Z \not= Z$ and $Y \cup W \not= W$

In this case, $C_\cup (X,Y) (Z) = Z$ and $C_\cup (X,Y) (W) = W$, 
so $C_\cup (X,Y) (Z) \subseteq C_\cup (X,Y) (W)$.
\end{proof}
%%%%%
%%%%%
\end{document}
