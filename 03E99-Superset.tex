\documentclass[12pt]{article}
\usepackage{pmmeta}
\pmcanonicalname{Superset}
\pmcreated{2013-05-24 14:35:12}
\pmmodified{2013-05-24 14:35:12}
\pmowner{yark}{2760}
\pmmodifier{unlord}{1}
\pmtitle{superset}
\pmrecord{13}{32585}
\pmprivacy{1}
\pmauthor{yark}{1}
\pmtype{Definition}
\pmcomment{trigger rebuild}
\pmclassification{msc}{03E99}
\pmrelated{Subset}
\pmrelated{SetTheory}
\pmdefines{proper superset}
\pmdefines{contains}
\pmdefines{contained}

\usepackage{amssymb}
\usepackage{amsmath}
\usepackage{amsfonts}
\begin{document}
\PMlinkescapeword{equivalent}
\PMlinkescapeword{mean}
\PMlinkescapeword{relation}
\PMlinkescapeword{similar}

Given two sets $A$ and $B$, $A$ is a \emph{superset} of $B$ if every element in $B$ is also in $A$.  We denote this relation as $A\supseteq B$.  This is equivalent to saying that $B$ is a subset of $A$, that is $A\supseteq B \Leftrightarrow B\subseteq A$.

Similar rules to those that hold for $\subseteq$ also hold for $\supseteq$.
If $X\supseteq Y$ and $Y\supseteq X$, then $X = Y$.
Every set is a superset of itself, and every set is a superset of the empty set.

We say $A$ is a \emph{proper superset} of $B$ if $A \supseteq B$ and $A \neq B$.  This relation is sometimes denoted by $A \supset B$,
but $A \supset B$ is often used to mean the more general superset relation,
so it should be made explicit when ``proper superset'' is intended,
possibly by using $X\varsupsetneq Y$ or $X\supsetneqq Y$ (or $X\supsetneq Y$ or $X\varsupsetneqq Y$).

One will occasionally see a collection $C$ of subsets of some set $X$ made into a partial order ``by containment''.  Depending on context this can mean defining a partial order where $Y\leq Z$ means $Y \subseteq Z$, or it can mean defining the opposite partial order: $Y\leq Z$ means $Y \supseteq Z$.  This is frequently used when applying Zorn's lemma. 

One will also occasionally see a collection $C$ of subsets of some set $X$ made into a category, usually by defining a single abstract morphism $Y\to Z$ whenever $Y\subseteq Z$ (this being a special case of the general method of treating pre-orders as categories).  This allows a concise definition of presheaves and sheaves, and it is generalized when defining a site.
%%%%%
%%%%%.
\end{document}
