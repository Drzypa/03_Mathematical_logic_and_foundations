\documentclass[12pt]{article}
\usepackage{pmmeta}
\pmcanonicalname{Tautology}
\pmcreated{2013-03-22 15:27:35}
\pmmodified{2013-03-22 15:27:35}
\pmowner{rspuzio}{6075}
\pmmodifier{rspuzio}{6075}
\pmtitle{tautology}
\pmrecord{13}{37310}
\pmprivacy{1}
\pmauthor{rspuzio}{6075}
\pmtype{Definition}
\pmcomment{trigger rebuild}
\pmclassification{msc}{03B05}
\pmclassification{msc}{03B10}
\pmsynonym{tautological}{Tautology}
\pmsynonym{tautologically}{Tautology}
\pmrelated{ContradictoryStatement}

\endmetadata

\usepackage{amssymb}
\usepackage{amsmath}
\usepackage{amsfonts}
\begin{document}
\PMlinkescapeword{column}
\PMlinkescapeword{contain}
\PMlinkescapeword{simple}
\PMlinkescapeword{terms}

A \emph{tautology} is a statement (or form) which is true solely on account of its logical form rather than because of the meaning of the terms employed.

In propositional logic, a tautology is a statement which is true regardless of the truth value of the substatements of which it is composed.  An example would be

It is a sunny day and on a sunny day the rain does not fall so the rain does not fall.

This statement is true no matter what the truth value of the statements of which it is comprised, ``It is a sunny day'' and ``The rain does not fall'' may be.  More generally, any statement of the form

(P and (if P then Q)) implies Q

or, in symbols,

\[ (P \wedge (P \rightarrow Q)) \rightarrow Q \]

is true no matter what statements one may substitute for $P$ and for $Q$.  (They may even contain connectives themselves.  An expression of the sort appearing above in which one can obtain tautologies by substituting arbitrary statements for the variables which appear in the expression is known as a \emph{tautologous form}.) 

To test a statement or form to see if it is a tautology, one may construct a truth table.  If it turns out that one obtains ``T'' in every column, then the statement is a tautology.

In predicate logic, a tautology is a statement which is true no matter what choice one makes for the predicates which appear in that statement.  A simple example would be

If all cats are black then there exists a black cat.

This is of the general form

\[ (\forall x) (A(x) \rightarrow B(x)) \rightarrow (\exists x) (A(x) \rightarrow B(x)) \]

with $A$ being ``is a cat'' and $B$ being ``is black''.

In the notation of Peano, tautology is denoted by ``$\curlyvee$''.  (``v'' 
comes from the Latin ``verum'' = 'true'.)
%%%%%
%%%%%
\end{document}
