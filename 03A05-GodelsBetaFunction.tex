\documentclass[12pt]{article}
\usepackage{pmmeta}
\pmcanonicalname{GodelsBetaFunction}
\pmcreated{2013-03-22 18:34:57}
\pmmodified{2013-03-22 18:34:57}
\pmowner{gribskoff}{21395}
\pmmodifier{gribskoff}{21395}
\pmtitle{G\"{o}del's beta function}
\pmrecord{18}{41307}
\pmprivacy{1}
\pmauthor{gribskoff}{21395}
\pmtype{Definition}
\pmcomment{trigger rebuild}
\pmclassification{msc}{03A05}
\pmclassification{msc}{03-01}
\pmclassification{msc}{11A07}
\pmsynonym{representation of sequences in formal systems}{GodelsBetaFunction}
%\pmkeywords{G\"{o}del's gamma sequence}
%\pmkeywords{congruence}
%\pmkeywords{G\"{o}del's beta sequence}
%\pmkeywords{representation}
%\pmkeywords{chinese remainder theorem}
%\pmkeywords{primitive recursive}
\pmrelated{ChineseRemainderTheorem}
\pmrelated{BeyondFormalism}
\pmrelated{FromHilbertsTenthProblemToGodelsTrichotomy}
\pmrelated{LinearCongruence}
\pmdefines{gamma sequence}

\endmetadata

\usepackage{amssymb}
\usepackage{amsmath}
\usepackage{amsfonts}
\usepackage{amsmath}
\usepackage{graphicx}
%%\usepackage{xypic}
\usepackage{psfrag}
\usepackage{amsthm}
\usepackage[T1]{fontenc}
\usepackage{yfonts}

% define commands here
\begin{document}
G\"{o}del's $\beta$ function is the tool needed to express in a formal theory \textbf{Z} assertions  about finite sequences of natural  numbers. (For a description of \textbf{Z} see the PM entry \PMlinkname{"beyond formalism: G\"{o}del's incompleteness"}{BeyondFormalism}). 

Let 

$$1, \ldots, n.$$

be a sequence of natural numbers. Then there is the factorial $n!$ such that

$$n \cdot (n - 1) \cdot \ldots \cdot 1 = n!.$$

This means that the factorial $n!$ is divided by each of the elements of
 
$$1, \ldots, n$$

so that we can build the derived sequence

$$ {1}\mid{n!}, {2}\mid{n!}, \ldots, {n}\mid{n!}.$$

\textbf{Definition 1} \quad [\textbf{G\"{o}del's $\Gamma$ sequence}] 
         
\begin{center}
\emph{If we denote $n!$ by $l$ then the elements of G\"{o}del's $\Gamma$ sequence are the numbers of the form}

$(k + 1) \cdot l + 1.$
\end{center}

Its general representation is
 
$$\Gamma = 1 \cdot l + 1, 2 \cdot l + 1, \ldots, n \cdot l + 1, (n + 1) \cdot l + 1.$$ 

\textbf{Theorem 1}:

\begin{center}
\emph{The elements of $\Gamma$ are pairwise relatively prime}.
\end{center}

\textbf{Proof}: (\emph{Reductio}) 

\begin{enumerate}
\item We assume the existence of a prime number $p$ such that $p$ divides

$$(j + 1) \cdot l + 1$$ 

and $p$ divides also 

$$(j + k + 1) \cdot l + 1.$$

\item This gives by definition the congruence

$$[(j + k + 1) \cdot l + 1] \equiv [(j + 1) \cdot l + 1] \quad \pmod{p}$$

\item Therefore $p$ divides

$$[(j + k + 1) \cdot l + 1] - [(j + 1) \cdot l + 1].$$

\item Then $p$ divides

$$j \cdot l + k \cdot l + l + 1 - j \cdot l - l - 1,$$

that is, $p$ divides $k \cdot l$.

\item But
  
$$ {p}\mid{k \cdot l} \rightarrow {p}\mid{k} \vee {p}\mid{l}.$$
\end{enumerate}

\textbf{Case I}: \textbf{Assume $ {p}\mid{l}$}

\begin{enumerate}
\item ${p}\mid{l} \rightarrow {p}\mid{l} \cdot (j + 1).$

\item But by hypothesis: $ {p}\mid{(j + 1) \cdot l + 1}.$

\item Therefore: $({p}\nmid{l}).$
\end{enumerate}

\textbf{Case II}: \textbf{Assume $ {p}\mid{k}$} 

\begin{enumerate}
\item $k \le n \le \text{max} (1, \ldots, n).$

\item But $(\forall k) {k}\mid{l}.$

\item $ {p}\mid{k} \wedge {k}\mid{l} \rightarrow  {p}\mid{l}$ 
\end{enumerate}

Case I.3. and Case II.3. are the desired contradiction.

\textbf{Definition 2}: \quad [\textbf{G\"{o}del's $\beta$ function}] 

\begin{center}
\emph{If $m, l,$ and $k$ are natural numbers then the function}

$\beta (m, l, k)$
 
\emph{computes the rest of the division of $m$ by a term} $(k + 1) \cdot l + 1$

\emph{of the $\Gamma$-sequence}.

\emph{Therefore:} $\beta (m, l, k) = R [m, (k + 1) \cdot l + 1].$ 
\end{center}

\textbf{Theorem 2}: \quad [\textbf{Sequences of natural numbers are representable by the $\beta$ function}.]  

$$<a_o, \ldots, a_n> \in \mathbb{N} \rightarrow (\exists m) (\exists l) [a_k = \beta (m, k, l)].$$ 

\textbf{Proof}: 

\begin{enumerate}
\item Let 

$$a_o, \ldots, a_n$$

be a sequence of natural numbers.

\item Then there is a number $l$ such that

$$\Gamma = 1 \cdot l + 1, 2 \cdot l + 1, \ldots, n \cdot l + 1, (n + 1) \cdot l + 1.$$

\item If $l \ge \text{max} (a_o, \ldots, a_n).$

\item Then $a_k < (k + 1) \cdot l + 1.$

\item But by the previous proposition the numbers

$$(k + 1) \ldots l + 1$$

are pairwise relatively prime.

\item This implies that the simultaneous congruences

$$x \equiv a_o [\pmod{1 \cdot l + 1}]$$ 

$$\vdots$$
                                  
$$x \equiv a_n [\pmod{(n + 1) \cdot l + 1}]$$

have a common solution $m$, by the chinese remainder theorem.

\item Therefore 

$$m \equiv a_k [\pmod{(k + 1) \cdot l + 1}].$$

\item This means that

$$a_k = R [m, (k + 1) \cdot l + 1].$$

\item But that is 

$$a_k = \beta (m, l, k).$$
\end{enumerate}

It is easily seen that the $\beta$ function is primitive recursive.

For that we only have to redenominated $m, l$ and $k$ as: 
\[ \begin{array}{ccc}
m & = & x_1 \\
l & = & x_2 \\
k & = & x_3. \\
\end{array} \]

Then $\beta (x_1, x_2, x_3) = R [x_1, (x_3 + 1) \cdot x_2 + 1]$.

But the functions "+", "$\cdot$" e "$R$" are primitive recursive.

Therefore $\beta$ is primitive recursive. 

\begin{thebibliography} {99}
\bibitem{BPHD} Bernays, P., Hilbert, D., \emph{Grundlagen der Mathematik}, 2.Auflage, Berlin, 1968.
\bibitem{KG} G\"{o}del, K., \emph{Collected works}, ed. S. Feferman, Oxford, 1987-2003.
\bibitem{SK} Kleene, S., \emph{Introduction to metamathematics}, North-Holland, Amsterdam, 1964.
\end{thebibliography}

\end{document}
%%%%%
%%%%%
\end{document}
