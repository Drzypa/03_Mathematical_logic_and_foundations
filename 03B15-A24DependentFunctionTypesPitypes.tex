\documentclass[12pt]{article}
\usepackage{pmmeta}
\pmcanonicalname{A24DependentFunctionTypesPitypes}
\pmcreated{2013-11-09 5:28:08}
\pmmodified{2013-11-09 5:28:08}
\pmowner{PMBookProject}{1000683}
\pmmodifier{PMBookProject}{1000683}
\pmtitle{A.2.4 Dependent function types ($\Pi$-types)}
\pmrecord{1}{}
\pmprivacy{1}
\pmauthor{PMBookProject}{1000683}
\pmtype{Feature}
\pmclassification{msc}{03B15}

\endmetadata

\usepackage{xspace}
\usepackage{amssyb}
\usepackage{amsmath}
\usepackage{amsfonts}
\usepackage{amsthm}
\makeatletter
\newcommand{\comp}{\textsc{comp}}
\newcommand{\defeq}{\vcentcolon\equiv}  
\newcommand{\define}[1]{\textbf{#1}}
\newcommand{\elim}{\textsc{elim}}
\newcommand{\form}{\textsc{form}}
\newcommand{\indexsee}[2]{\index{#1|see{#2}}}    
\newcommand{\intro}{\textsc{intro}}
\newcommand{\jdeq}{\equiv}      
\newcommand{\jdeqtp}[4]{#1 \vdash #2 \jdeq #3 : #4}
\def\lam#1{{\lambda}\@lamarg#1:\@endlamarg\@ifnextchar\bgroup{.\,\lam}{.\,}}
\def\@lamarg#1:#2\@endlamarg{\if\relax\detokenize{#2}\relax #1\else\@lamvar{\@lameatcolon#2},#1\@endlamvar\fi}
\def\@lameatcolon#1:{#1}
\def\lamu#1{{\lambda}\@lamuarg#1:\@endlamuarg\@ifnextchar\bgroup{.\,\lamu}{.\,}}
\def\@lamuarg#1:#2\@endlamuarg{#1}
\def\@lamvar#1,#2\@endlamvar{(#2\,{:}\,#1)}
\newcommand{\oftp}[3]{#1 \vdash #2 : #3}
\newcommand{\tmtp}[2]{#1 \mathord{:} #2}
\def\tprd#1{\@tprd{#1}\@ifnextchar\bgroup{\tprd}{}}
\def\@tprd#1{\mathchoice{{\textstyle\prod_{(#1)}}}{\prod_{(#1)}}{\prod_{(#1)}}{\prod_{(#1)}}}
\newcommand{\uniq}{\textsc{uniq}}
\newcommand{\UU}{\ensuremath{\mathcal{U}}\xspace}
\newcommand{\vcentcolon}{:\!\!}
\let\autoref\cref
\makeatother

\begin{document}
\index{type!dependent function}%
\index{type!function}%

In \PMlinkname{\S 1.2}{12functiontypes}, we introduced non-dependent functions $A\to B$ in
order to define a family of types as a function $\lam{x:A} B:A\to\UU_i$, which
then gives rise to a type of dependent functions $\tprd{x:A} B$. But with explicit contexts
we may replace $\lam{x:A} B:A\to\UU_i$ with the judgment
%
\begin{equation*}
  \oftp{\tmtp xA}{B}{\UU_i}.
\end{equation*}
%
Consequently, we may define dependent functions directly, without reference to non-dependent ones. This way we follow the general principle that each type former, with its constants and rules, should be introduced independently of all other type formers.
%
In fact, henceforth each type former is introduced systematically by:
\begin{itemize}
\item a \define{formation rule}, stating when the type former can be applied;\index{formation rule}\index{rule!formation}
\item some \define{introduction rules}, stating how to inhabit the type;\index{introduction rule}\index{rule!introduction}
\item \define{elimination rules}, or an induction principle, stating how to use an
  element of the type;
  \index{induction principle}\index{eliminator}
\item \define{computation rules}, which are judgmental equalities explaining what happens when elimination rules are applied to results of introduction rules;
  \index{computation rule}
  \indexsee{rule!computation}{computation rule}
\item optional \define{uniqueness principles}, which are judgmental equalities explaining how every element of the type is uniquely determined by the results of elimination rules applied to it.
  \index{uniqueness!principle}
  \indexsee{principle!uniqueness}{uniqueness principle}
\end{itemize}
(See also \autoref{rmk:introducing-new-concepts}.)

For the dependent function type these rules are:
%
\begin{mathparpagebreakable}
  \def\premise{\oftp{\Gamma}{A}{\UU_i} \and \oftp{\Gamma,\tmtp xA}{B}{\UU_i}}
  \inferrule*[right=$\Pi$-\form]
    \premise
    {\oftp\Gamma{\tprd{x:A}B}{\UU_i}}
\and
  \inferrule*[right=$\Pi$-\intro]
  {\oftp{\Gamma,\tmtp xA}{b}{B}}
  {\oftp\Gamma{\lam{x:A} b}{\tprd{x:A} B}}
\and
  \inferrule*[right=$\Pi$-\elim]
  {\oftp\Gamma{f}{\tprd{x:A} B} \\ \oftp\Gamma{a}{A}}
  {\oftp\Gamma{f(a)}{B[a/x]}}
\and
  \inferrule*[right=$\Pi$-\comp]
  {\oftp{\Gamma,\tmtp xA}{b}{B} \\ \oftp\Gamma{a}{A}}
  {\jdeqtp\Gamma{(\lam{x:A} b)(a)}{b[a/x]}{B[a/x]}}
\and
  \inferrule*[right=$\Pi$-\uniq]
  {\oftp\Gamma{f}{\tprd{x:A} B}}
  {\jdeqtp\Gamma{f}{(\lamu{x:A}f(x))}{\tprd{x:A} B}}
\end{mathparpagebreakable}

The expression $\lam{x:A} b$ binds free occurrences of $x$ in $b$, as does $\tprd{x:A} B$ for
$B$.

When $x$ does not occur freely in $B$ so that $B$ does not depend on $A$, we obtain as a
special case the ordinary function type $A\to B \defeq \tprd{x:A} B$. We take this as the \emph{definition} of $\to$.

We may abbreviate an expression $\lam{x:A} b$ as $\lamu{x:A} b$, with the understanding
that the omitted type $A$ should be filled in appropriately before type-checking.


\end{document}
