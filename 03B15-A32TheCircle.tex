\documentclass[12pt]{article}
\usepackage{pmmeta}
\pmcanonicalname{A32TheCircle}
\pmcreated{2013-11-09 5:59:05}
\pmmodified{2013-11-09 5:59:05}
\pmowner{PMBookProject}{1000683}
\pmmodifier{PMBookProject}{1000683}
\pmtitle{A.3.2 The circle}
\pmrecord{1}{}
\pmprivacy{1}
\pmauthor{PMBookProject}{1000683}
\pmtype{Feature}
\pmclassification{msc}{03B15}

\endmetadata

\usepackage{xspace}
\usepackage{amssyb}
\usepackage{amsmath}
\usepackage{amsfonts}
\usepackage{amsthm}
\makeatletter
\newcommand{\base}{\ensuremath{\mathsf{base}}\xspace}
\newcommand{\comp}{\textsc{comp}}
\newcommand{\ctx}{\ensuremath{\mathsf{ctx}}}
\newcommand{\dpath}[4]{#3 =^{#1}_{#2} #4}
\newcommand{\elim}{\textsc{elim}}
\newcommand{\form}{\textsc{form}}
\newcommand{\id}[3][]{\ensuremath{#2 =_{#1} #3}\xspace}
\newcommand{\ind}[1]{\mathsf{ind}_{#1}}
\newcommand{\intro}{\textsc{intro}}
\newcommand{\jdeq}{\equiv}      
\newcommand{\jdeqtp}[4]{#1 \vdash #2 \jdeq #3 : #4}
\def\lamu#1{{\lambda}\@lamuarg#1:\@endlamuarg\@ifnextchar\bgroup{.\,\lamu}{.\,}}
\def\@lamuarg#1:#2\@endlamuarg{#1}
\newcommand{\lloop}{\ensuremath{\mathsf{loop}}\xspace}
\newcommand{\mapdep}[2]{\ensuremath{\mapdepfunc{#1}\mathopen{}\left(#2\right)\mathclose{}}\xspace}
\newcommand{\mapdepfunc}[1]{\ensuremath{\mathsf{apd}_{#1}}\xspace} 
\newcommand{\oftp}[3]{#1 \vdash #2 : #3}
\newcommand{\Sn}{\mathbb{S}}
\newcommand{\tmtp}[2]{#1 \mathord{:} #2}
\newcommand{\UU}{\ensuremath{\mathcal{U}}\xspace}
\newcommand{\wfctx}[1]{#1\ \ctx}
\let\apd\mapdep
\let\autoref\cref
\makeatother

\begin{document}
\index{type!circle}%

Here we give an example of a basic higher inductive type; others follow the same
general scheme, albeit with elaborations.

Note that the rules below do not precisely follow the pattern of the ordinary
inductive types in \autoref{sec:syntax-more-formally}: the rules refer to the
notions of transport and functoriality of maps (\PMlinkname{\S 2.2}{22functionsarefunctors}), and the
second computation rule is a propositional, not judgmental, equality. These
differences are discussed in \PMlinkname{\S 6.2}{62inductionprinciplesanddependentpaths}.

\begin{mathparpagebreakable}
  \inferrule*[right=$\Sn^1$-\form]
  {\wfctx\Gamma}
  {\oftp\Gamma{\Sn^1}{\UU_i}}
\and
  \inferrule*[right=$\Sn^1$-\intro${}_1$]
  {\wfctx\Gamma}
  {\oftp\Gamma{\base}{\Sn^1}}
\and
  \inferrule*[right=$\Sn^1$-\intro${}_2$]
  {\wfctx\Gamma}
  {\oftp\Gamma{\lloop}{\id[\Sn^1]{\base}{\base}}}
\and
  \inferrule*[right=$\Sn^1$-\elim]
  {\oftp{\Gamma,\tmtp x{\Sn^1}}{C}{\UU_i} \\
   \oftp{\Gamma}{b}{C[\base/x]} \\
   \oftp{\Gamma}{\ell}{\dpath C \lloop b b} \\
   \oftp\Gamma{p}{\Sn^1}}
  {\oftp\Gamma{\ind{\Sn^1}(x.C,b,\ell,p)}{C[p/x]}}
\and
  \inferrule*[right=$\Sn^1$-\comp${}_1$]
  {\oftp{\Gamma,\tmtp x{\Sn^1}}{C}{\UU_i} \\
   \oftp{\Gamma}{b}{C[\base/x]} \\
   \oftp{\Gamma}{\ell}{\dpath C \lloop b b}}
  {\jdeqtp\Gamma{\ind{\Sn^1}(x.C,b,\ell,\base)}{b}{C[\base/x]}}
\and
  \inferrule*[right=$\Sn^1$-\comp${}_2$]
  {\oftp{\Gamma,\tmtp x{\Sn^1}}{C}{\UU_i} \\
   \oftp{\Gamma}{b}{C[\base/x]} \\
   \oftp{\Gamma}{\ell}{\dpath C \lloop b b}}
  {\oftp\Gamma{\Sn^1\text{-}\mathsf{loopcomp}}
    {\id {\apd{(\lamu{y:\Sn^1} \ind{\Sn^1}(x.C,b,\ell,y))}{\lloop}} {\ell}}}
\end{mathparpagebreakable}
%
In $\ind{\Sn^1}$, $x$ is bound in $C$. The notation ${\dpath C \lloop b b}$ for dependent paths was introduced in \PMlinkname{\S 6.2}{62inductionprinciplesanddependentpaths}.
\index{rules of type theory|)}%


\end{document}
