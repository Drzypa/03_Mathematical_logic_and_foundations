\documentclass[12pt]{article}
\usepackage{pmmeta}
\pmcanonicalname{AxiomOfUnion}
\pmcreated{2013-03-22 13:42:49}
\pmmodified{2013-03-22 13:42:49}
\pmowner{Sabean}{2546}
\pmmodifier{Sabean}{2546}
\pmtitle{axiom of union}
\pmrecord{8}{34394}
\pmprivacy{1}
\pmauthor{Sabean}{2546}
\pmtype{Axiom}
\pmcomment{trigger rebuild}
\pmclassification{msc}{03E30}
\pmsynonym{union}{AxiomOfUnion}

\endmetadata

% this is the default PlanetMath preamble.  as your knowledge
% of TeX increases, you will probably want to edit this, but
% it should be fine as is for beginners.

% almost certainly you want these
\usepackage{amssymb}
\usepackage{amsmath}
\usepackage{amsfonts}

% used for TeXing text within eps files
%\usepackage{psfrag}
% need this for including graphics (\includegraphics)
%\usepackage{graphicx}
% for neatly defining theorems and propositions
%\usepackage{amsthm}
% making logically defined graphics
%%%\usepackage{xypic}

% there are many more packages, add them here as you need them

% define commands here
\begin{document}
For any $X$ there exists a set $Y = \bigcup X$.

The Axiom of Union is an axiom of Zermelo-Fraenkel set theory.  In symbols, it reads
\[
\forall X \exists Y \forall u (u \in Y \leftrightarrow \exists z (z \in X \land u \in z)).
\]

Notice that this means that $Y$ is the set of elements of all elements of $X$.  More succinctly, the union of any set of sets is a set.  By Extensionality, the set $Y$ is unique.  $Y$ is called the \emph{union} of $X$.

In particular, the Axiom of Union, along with the Axiom of Pairing allows us to define
\[
X \cup Y = \bigcup \{ X, Y \},
\]
as well as the triple
\[
\{ a, b, c \} = \{ a, b \} \cup \{ c \}
\]
and therefore the $n$-tuple
\[
\{ a_1, \ldots, a_n \} = \{ a_1 \} \cup \cdots \cup \{ a_n \}
\]
%%%%%
%%%%%
\end{document}
