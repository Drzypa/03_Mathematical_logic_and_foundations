\documentclass[12pt]{article}
\usepackage{pmmeta}
\pmcanonicalname{ExampleOfDefinableType}
\pmcreated{2013-03-22 13:29:43}
\pmmodified{2013-03-22 13:29:43}
\pmowner{aplant}{12431}
\pmmodifier{aplant}{12431}
\pmtitle{example of definable type}
\pmrecord{5}{34070}
\pmprivacy{1}
\pmauthor{aplant}{12431}
\pmtype{Example}
\pmcomment{trigger rebuild}
\pmclassification{msc}{03C07}
%\pmkeywords{dense linear order}
\pmrelated{ExampleOfUniversalStructure}
\pmrelated{DedekindCuts}

\endmetadata

\usepackage{amssymb}
\usepackage{amsmath}
\usepackage{amsfonts}
\begin{document}
Consider $(\mathbf{Q},<)$ as a structure in a language with one binary relation, which we interpret as the order. 
This is a universal, $\aleph_{0}$-categorical structure (see example of universal structure).

The theory of $(\mathbf{Q},<)$ has quantifier elimination, and so is o-minimal. 
Thus a type over the set $\mathbf{Q}$ is determined by the quantifier free formulas over $\mathbf{Q}$, which in turn are determined by the atomic formulas over $\mathbf{Q}$.
An atomic formula in one variable over $B$ is of the form $x<b$ or $x>b$ or $x=b$ for some $b \in B$. 
Thus each 1-type over $\mathbf{Q}$ determines a Dedekind cut over  $\mathbf{Q}$, and conversely a Dedekind cut determines a complete type over $\mathbf{Q}$. 
Let $D(p):=\{a \in \mathbf{Q}:x>a \in p\}$.

\medskip

Thus there are two classes of type over $\mathbf{Q}$. 
\begin{enumerate}\item Ones where $D(p)$ is of the form $(- \infty,a)$  or $(-\infty,a]$ for some $a \in \mathbf{Q}$. It is clear that these are definable from the above discussion.
\item Ones where $D(p)$ has no supremum in $\mathbf{Q}$. These are clearly not definable by o-minimality of $\mathbf{Q}$.
\end{enumerate}
%%%%%
%%%%%
\end{document}
