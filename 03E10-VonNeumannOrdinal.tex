\documentclass[12pt]{article}
\usepackage{pmmeta}
\pmcanonicalname{VonNeumannOrdinal}
\pmcreated{2013-03-22 12:32:37}
\pmmodified{2013-03-22 12:32:37}
\pmowner{Henry}{455}
\pmmodifier{Henry}{455}
\pmtitle{von Neumann ordinal}
\pmrecord{11}{32787}
\pmprivacy{1}
\pmauthor{Henry}{455}
\pmtype{Definition}
\pmcomment{trigger rebuild}
\pmclassification{msc}{03E10}
\pmsynonym{ordinal}{VonNeumannOrdinal}
\pmrelated{VonNeumannInteger}
\pmrelated{ZermeloFraenkelAxioms}
\pmrelated{OrdinalNumber}
\pmdefines{successor ordinal}
\pmdefines{limit ordinal}
\pmdefines{successor}

\usepackage{amssymb}
\usepackage{amsmath}
\usepackage{amsfonts}
\begin{document}
The \emph{\PMlinkescapetext{von Neumann ordinal}} is a method of defining ordinals in set theory.

The von Neumann ordinal $\alpha$ is defined to be the well-ordered set containing the von Neumann ordinals which precede $\alpha$.  The set of finite von Neumann ordinals is known as the von Neumann integers.  Every well-ordered set is isomorphic to a von Neumann ordinal.

They can be constructed by transfinite recursion as follows:

\begin{itemize}
\item The empty set is $0$.

\item Given any ordinal $\alpha$, the ordinal $\alpha+1$ (the \emph{successor} of $\alpha$) is defined to be $\alpha\cup\{\alpha\}$.

\item Given a set $A$ of ordinals, $\bigcup_{a\in A} a$ is an ordinal.

\end{itemize}

If an ordinal is the successor of another ordinal, it is an \emph{successor ordinal}.  If an ordinal is neither $0$ nor a successor ordinal then it is a \emph{limit ordinal}.  The first limit ordinal is named $\omega$.

The class of ordinals is denoted $\mathbf{On}$.

The von Neumann ordinals have the convenient property that if $a<b$ then $a\in b$ and $a\subset b$.
%%%%%
%%%%%
\end{document}
