\documentclass[12pt]{article}
\usepackage{pmmeta}
\pmcanonicalname{ChurchInteger}
\pmcreated{2013-03-22 12:32:31}
\pmmodified{2013-03-22 12:32:31}
\pmowner{mathcam}{2727}
\pmmodifier{mathcam}{2727}
\pmtitle{Church integer}
\pmrecord{8}{32785}
\pmprivacy{1}
\pmauthor{mathcam}{2727}
\pmtype{Definition}
\pmcomment{trigger rebuild}
\pmclassification{msc}{03B40}
\pmclassification{msc}{68N18}
\pmrelated{LambdaCalculus}

\endmetadata

\usepackage{amssymb}
\usepackage{amsmath}
\usepackage{amsfonts}
\usepackage{listings}
\begin{document}
A \emph{Church integer} is a representation of integers as functions, invented by Alonzo Church.  An integer $N$ is represented as a higher-order function, which applies a given function to a given expression $N$ times.

For example, in the programming language Haskell, a function that returns a particular Church integer might be

\begin{align*}
\operatorname{church} 0 &= \ f x \rightarrow x\\
\operatorname{church} n &= c\\
&\operatorname{where}: c f x = c' f (f x)\\
&\hspace{.5in}\operatorname{where}: c' = \operatorname{church} (n - 1)\\
\end{align*}

The transformation from a Church integer to an integer might be

\begin{verbatim}
unchurch n = n (+1) 0
\end{verbatim}

Thus we can generate the integers--the \texttt{(+1)} function would be applied to an initial value of $0$ $n$ times, yielding the ordinary integer $n$.
%%%%%
%%%%%
\end{document}
