\documentclass[12pt]{article}
\usepackage{pmmeta}
\pmcanonicalname{InclusionMapping}
\pmcreated{2013-03-22 13:43:08}
\pmmodified{2013-03-22 13:43:08}
\pmowner{Koro}{127}
\pmmodifier{Koro}{127}
\pmtitle{inclusion mapping}
\pmrecord{9}{34402}
\pmprivacy{1}
\pmauthor{Koro}{127}
\pmtype{Definition}
\pmcomment{trigger rebuild}
\pmclassification{msc}{03E20}
\pmsynonym{inclusion map}{InclusionMapping}
\pmsynonym{inclusion}{InclusionMapping}
\pmrelated{Pullback2}

\endmetadata

% this is the default PlanetMath preamble.  as your knowledge
% of TeX increases, you will probably want to edit this, but
% it should be fine as is for beginners.

% almost certainly you want these
\usepackage{amssymb}
\usepackage{amsmath}
\usepackage{amsfonts}

% used for TeXing text within eps files
%\usepackage{psfrag}
% need this for including graphics (\includegraphics)
%\usepackage{graphicx}
% for neatly defining theorems and propositions
%\usepackage{amsthm}
% making logically defined graphics
%%%\usepackage{xypic}

% there are many more packages, add them here as you need them

% define commands here

\newcommand{\sR}[0]{\mathbb{R}}
\newcommand{\sC}[0]{\mathbb{C}}
\newcommand{\sN}[0]{\mathbb{N}}
\newcommand{\sZ}[0]{\mathbb{Z}}

% The below lines should work as the command
% \renewcommand{\bibname}{References}
% without creating havoc when rendering an entry in 
% the page-image mode.
\makeatletter
\@ifundefined{bibname}{}{\renewcommand{\bibname}{References}}
\makeatother

\newcommand*{\norm}[1]{\lVert #1 \rVert}
\newcommand*{\abs}[1]{| #1 |}
\begin{document}
\PMlinkescapeword{combination}
\PMlinkescapeword{symbol}
{\bf Definition} Let $X$ be a subset of $Y$. Then the {\bf inclusion map}
from $X$ to $Y$ is the mapping
\begin{eqnarray*}
\iota: X&\to& Y \\
       x&\mapsto& x.
\end{eqnarray*}

In other words, the inclusion map is simply a fancy way to say
that every element in $X$ is also an element in $Y$.

To indicate that a mapping is an inclusion mapping, one usually writes
 $\hookrightarrow$ instead of $\to$ when defining or mentioning an
inclusion map. This hooked arrow symbol  $\hookrightarrow$ can be 
seen as combination of the symbols $\subset$ and $\to$.
In the above definition, we have not used this convention.
However, examples of this convention would be:
\begin{itemize}
\item
Let $\iota:X\hookrightarrow Y$ be the inclusion map from $X$ to $Y$.
\item
We have the inclusion $S^n\hookrightarrow \sR^{n+1}$.
\end{itemize}
%%%%%
%%%%%
\end{document}
