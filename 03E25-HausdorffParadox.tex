\documentclass[12pt]{article}
\usepackage{pmmeta}
\pmcanonicalname{HausdorffParadox}
\pmcreated{2013-03-22 15:16:12}
\pmmodified{2013-03-22 15:16:12}
\pmowner{GrafZahl}{9234}
\pmmodifier{GrafZahl}{9234}
\pmtitle{Hausdorff paradox}
\pmrecord{9}{37057}
\pmprivacy{1}
\pmauthor{GrafZahl}{9234}
\pmtype{Theorem}
\pmcomment{trigger rebuild}
\pmclassification{msc}{03E25}
\pmclassification{msc}{51M04}
%\pmkeywords{paradox}
%\pmkeywords{unit ball}
%\pmkeywords{decomposition}
\pmrelated{ChoiceFunction}
\pmrelated{BanachTarskiParadox}
\pmrelated{ProofofBanachTarskiParadox}

\endmetadata

% this is the default PlanetMath preamble.  as your knowledge
% of TeX increases, you will probably want to edit this, but
% it should be fine as is for beginners.

% almost certainly you want these
\usepackage{amssymb}
\usepackage{amsmath}
\usepackage{amsfonts}

% used for TeXing text within eps files
%\usepackage{psfrag}
% need this for including graphics (\includegraphics)
%\usepackage{graphicx}
% for neatly defining theorems and propositions
\usepackage{amsthm}
% making logically defined graphics
%%%\usepackage{xypic}

% there are many more packages, add them here as you need them

% define commands here
\newcommand{\Prod}{\prod\limits}
\newcommand{\Sum}{\sum\limits}
\newcommand{\mbb}{\mathbb}
\newcommand{\mbf}{\mathbf}
\newcommand{\mc}{\mathcal}
\newcommand{\ol}{\overline}

% Math Operators/functions
\DeclareMathOperator{\Frob}{Frob}
\DeclareMathOperator{\cwe}{cwe}
\DeclareMathOperator{\we}{we}
\DeclareMathOperator{\wt}{wt}
\begin{document}
\PMlinkescapeword{constructible}
\PMlinkescapeword{mean}
\PMlinkescapeword{sounds}
\PMlinkescapeword{state}
\PMlinkescapeword{unit}
Let $S^2$ be the unit sphere in the Euclidean space $\mbb{R}^3$. Then
it is possible to take ``half'' and ``a third'' of $S^2$ such that
both of these parts are essentially congruent (we give a formal
version in a minute). This sounds paradoxical:
wouldn't that mean that half of the sphere's area is equal to only a
third? The ``paradox'' resolves itself if one takes into account that
one can choose non-measurable subsets of the sphere which ostensively are ``half'' and a ``third'' of it, using geometric congruence as means of comparison.

Let us now formally state the Theorem.

\newtheorem*{thm}{Theorem}
\begin{thm}[Hausdorff paradox~\cite{H}]
There exists a disjoint \PMlinkescapetext{decomposition} of the unit sphere $S^2$ in the
Euclidean space $\mbb{R}^3$ into four subsets $A,B,C,D$, such that the
following conditions are met:
\begin{enumerate}
\item Any two of the sets $A$, $B$, $C$ and $B\cup C$ are congruent.
\item $D$ is countable.
\end{enumerate}
\end{thm}

A crucial ingredient to the proof is the \PMlinkid{axiom of choice}{310}, so the
sets $A$, $B$ and $C$ are not constructible. The theorem itself is a
crucial ingredient to the proof of the so-called Banach-Tarski
paradox.

\begin{thebibliography}{H}

\bibitem[H]{H} \textsc{F.~Hausdorff}, Bemerkung \"{u}ber den Inhalt von
  Punktmengen, \emph{Math.\ Ann.}\ 75, 428--433, (1915), \texttt{\PMlinkexternal{http://dz-srv1.sub.uni-goettingen.de/sub/digbib/loader?did=D28919}{http://dz-srv1.sub.uni-goettingen.de/sub/digbib/loader?did=D28919}} (in German).

\end{thebibliography}
%%%%%
%%%%%
\end{document}
