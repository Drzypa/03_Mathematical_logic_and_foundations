\documentclass[12pt]{article}
\usepackage{pmmeta}
\pmcanonicalname{ExtensionOfAFunction}
\pmcreated{2013-03-22 17:51:00}
\pmmodified{2013-03-22 17:51:00}
\pmowner{Wkbj79}{1863}
\pmmodifier{Wkbj79}{1863}
\pmtitle{extension of a function}
\pmrecord{6}{40322}
\pmprivacy{1}
\pmauthor{Wkbj79}{1863}
\pmtype{Definition}
\pmcomment{trigger rebuild}
\pmclassification{msc}{03E20}
\pmrelated{RestrictionOfAFunction}
\pmdefines{extension}

\usepackage{amssymb}
\usepackage{amsmath}
\usepackage{amsfonts}
\usepackage{pstricks}
\usepackage{psfrag}
\usepackage{graphicx}
\usepackage{amsthm}
%%\usepackage{xypic}

\begin{document}
Let $f\colon X \to Y$ be a function and $A$ and $B$ be sets such that $X\subseteq A$ and $Y\subseteq B$.  An \emph{extension} of $f$ to $A$ is a function $g\colon A \to B$ such that $f(x)=g(x)$ for all $x\in X$.  Alternatively, $g$ is an extension of $f$ to $A$ if $f$ is the restriction of $g$ to $X$.

Typically, functions are not arbitrarily extended.  Rather, it is usually insisted upon that extensions have certain properties.  Examples include analytic continuations and meromorphic extensions. 
%%%%%
%%%%%
\end{document}
