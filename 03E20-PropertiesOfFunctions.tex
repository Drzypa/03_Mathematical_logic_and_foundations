\documentclass[12pt]{article}
\usepackage{pmmeta}
\pmcanonicalname{PropertiesOfFunctions}
\pmcreated{2013-03-22 14:59:54}
\pmmodified{2013-03-22 14:59:54}
\pmowner{yark}{2760}
\pmmodifier{yark}{2760}
\pmtitle{properties of functions}
\pmrecord{20}{36705}
\pmprivacy{1}
\pmauthor{yark}{2760}
\pmtype{Result}
\pmcomment{trigger rebuild}
\pmclassification{msc}{03E20}
\pmrelated{PropertiesOfAFunction}

\endmetadata

\usepackage{amssymb}
\usepackage{amsmath}
\usepackage{amsfonts}

\def\emptyset{\varnothing}
\begin{document}
Let $f \colon X \to Y$ be a function.
Let $(A_i)_{i \in I}$ be a family of subsets of $X$,
and let $(B_j)_{j \in J}$ be a family of subsets of $Y$,
where $I$ and $J$ are non-empty index sets.

Then, it is easy to prove, directly from definitions, that the following hold:

\begin{itemize}
\item $f(\bigcup \limits_{i \in I}{A_i}) = \bigcup \limits_{i \in I}{f(A_i)}$ (i.e., the image of a union is the union of the images)
\item $f(\bigcap \limits_{i \in I}{A_i}) \subseteq \bigcap \limits_{i \in I}{f(A_i)}$ (i.e., the image of an intersection is contained in the intersection of the images)
\item $A \subseteq f^{-1}(f(A))$ for any $A \subseteq X$ (where $f^{-1}(f(A))$ is the inverse image of $f(A)$)
\item $f(f^{-1}(B)) \subseteq B$ for any $B \subseteq Y$
\item $f^{-1}(Y \setminus B) = X \setminus f^{-1}(B)$ for any $B \subseteq Y$
\item $f^{-1}(\bigcup \limits_{j \in J}{B_j}) = \bigcup \limits_{j \in J}{f^{-1}(B_j)}$ (the inverse image of a union is the union of the inverse images)
\item $f^{-1}(\bigcap \limits_{j \in J}{B_j}) = \bigcap \limits_{j \in J}{f^{-1}(B_j)}$ (the inverse image of an intersection is the intersection of the inverse images)
\item $f(f^{-1}(B)) = B$ for every $B \subseteq Y$ if and only if $f$ is surjective.
\end{itemize}

For more properties related specifically to inverse images, see the \PMlinkname{inverse image}{InverseImage} entry.

Further, the following conditions are equivalent (for more, see the entry on injective functions):
\begin{itemize}
\item $f$ is injective
\item $f(S \cap T) = f(S) \cap f(T)$ for all $S, T \subseteq X$
\item $f^{-1}(f(S)) = S$ for all $S \subseteq X$
\item $f(S) \cap f(T) = \emptyset$ for all $S,T \subseteq X$ such that $S \cap T = \emptyset$
\item $f(S \setminus T) = f(S) \setminus f(T)$ for all $S,T \subseteq X$
\end{itemize}
%%%%%
%%%%%
\end{document}
