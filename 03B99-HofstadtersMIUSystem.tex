\documentclass[12pt]{article}
\usepackage{pmmeta}
\pmcanonicalname{HofstadtersMIUSystem}
\pmcreated{2013-03-22 13:57:48}
\pmmodified{2013-03-22 13:57:48}
\pmowner{Daume}{40}
\pmmodifier{Daume}{40}
\pmtitle{Hofstadter's MIU system}
\pmrecord{8}{34732}
\pmprivacy{1}
\pmauthor{Daume}{40}
\pmtype{Definition}
\pmcomment{trigger rebuild}
\pmclassification{msc}{03B99}
\pmsynonym{MIU system}{HofstadtersMIUSystem}

% this is the default PlanetMath preamble.  as your knowledge
% of TeX increases, you will probably want to edit this, but
% it should be fine as is for beginners.

% almost certainly you want these
\usepackage{amssymb}
\usepackage{amsmath}
\usepackage{amsfonts}

% used for TeXing text within eps files
%\usepackage{psfrag}
% need this for including graphics (\includegraphics)
%\usepackage{graphicx}
% for neatly defining theorems and propositions
%\usepackage{amsthm}
% making logically defined graphics
%%%\usepackage{xypic} 

% there are many more packages, add them here as you need them

% define commands here
\begin{document}
The alphabet of the system contains three symbols $M, I, U$.  The set of theorem is the set of string constructed by the rules and the axiom, is denoted by $\mathcal{T}$ and can be built as follows:
\begin{enumerate}
\item[(axiom)] $MI\in \mathcal{T}$.
\item[(i)] If $xI\in \mathcal{T}$ then $xIU\in \mathcal{T}$.
\item[(ii)] If $Mx \in \mathcal{T}$ then $Mxx\in \mathcal{T}$.
\item[(iii)] In any theorem, $III$ can be replaced by $U$.
\item[(iv)] In any theorem, $UU$ can be omitted.
\end{enumerate}
\textbf{example:}\\
\begin{itemize}
\item Show that $MUII \in \mathcal{T}$\\
\begin{tabular}{ll}
$MI\in \mathcal{T}$ & by axiom\\
$\implies MII\in \mathcal{T}$ & by rule (ii) where $x=I$\\
$\implies MIIII\in \mathcal{T}$ & by rule (ii) where $x=II$\\
$\implies MIIIIIIII\in \mathcal{T}$ & by rule (ii) where $x=IIII$\\
$\implies MIIIIIIIIU\in \mathcal{T}$ & by rule (i) where $x=MIIIIIII$\\
$\implies MIIIIIUU\in \mathcal{T}$ & by rule (iii)\\
$\implies MIIIII\in \mathcal{T}$ & by rule (iv)\\
$\implies MUII\in \mathcal{T}$ & by rule (iii)\\
\end{tabular}
\item Is $MU$ a theorem?\\
No. Why? Because the number of $I$'s of a theorem is never a multiple of 3.  We will show this by structural induction.\\\\
\textit{base case:}  The statement is true for the base case. Since the axiom has one $I$ .  Therefore not a multiple of 3.\\
\textit{induction hypothesis:}  Suppose true for premise of all rule.\\
\textit{induction step:}  By induction hypothesis we assume the premise of each rule to be true and show that the application of the rule keeps the staement true.\\
\textit{Rule 1:}  Applying rule 1 does not add any $I$'s to the formula. Therefore the statement is true for rule 1 by induction hypothesis.\\
\textit{Rule 2:}  Applying rule 2 doubles the amount of $I$'s of the formula but since the initial amount of $I$'s was not a multiple of 3 by induction hypothesis.  Doubling that amount does not make it a multiple of 3 \textit{(i.e. if $n \not\equiv 0 \operatorname{mod} 3$ then $2n \not\equiv 0 \operatorname{mod} 3$)}. Therefore the statement is true for rule 2.\\
\textit{Rule 3:} Applying rule 3 replaces $III$ by $U$.  Since the initial amount of $I$'s was not a multiple of 3 by induction hypothesis. Removing $III$ will not make the number of $I$'s in the formula be a multiple of 3.  Therefore the statement is true for rule 3.\\
\textit{Rule 4:}  Applying rule 4 removes $UU$ and does not change the amount of $I$'s.  Since the initial amount of $I$'s was not a multiple of 3 by induction hypothesis.  Therefore the statement is true for rule 4.\\\\
Therefore all theorems do not have a multiple of 3 $I$'s.
\end{itemize}
\cite{1}
\begin{thebibliography}{1}
\bibitem[HD]{1} Hofstader, R. Douglas: G\"odel, Escher, Bach: an Eternal Golden Braid. Basic Books, Inc., New York, 1979.
\end{thebibliography}
%%%%%
%%%%%
\end{document}
