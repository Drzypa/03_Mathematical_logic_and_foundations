\documentclass[12pt]{article}
\usepackage{pmmeta}
\pmcanonicalname{ElementaryFunctionalArithmetic}
\pmcreated{2013-03-22 12:56:39}
\pmmodified{2013-03-22 12:56:39}
\pmowner{Henry}{455}
\pmmodifier{Henry}{455}
\pmtitle{Elementary Functional Arithmetic}
\pmrecord{5}{33302}
\pmprivacy{1}
\pmauthor{Henry}{455}
\pmtype{Definition}
\pmcomment{trigger rebuild}
\pmclassification{msc}{03F30}
\pmsynonym{EFA}{ElementaryFunctionalArithmetic}
\pmrelated{PeanoArithmetic}

\endmetadata

% this is the default PlanetMath preamble.  as your knowledge
% of TeX increases, you will probably want to edit this, but
% it should be fine as is for beginners.

% almost certainly you want these
\usepackage{amssymb}
\usepackage{amsmath}
\usepackage{amsfonts}

% used for TeXing text within eps files
%\usepackage{psfrag}
% need this for including graphics (\includegraphics)
%\usepackage{graphicx}
% for neatly defining theorems and propositions
%\usepackage{amsthm}
% making logically defined graphics
%%%\usepackage{xypic}

% there are many more packages, add them here as you need them

% define commands here
%\PMlinkescapeword{theory}
\begin{document}
\emph{Elementary Functional Arithmetic}, or EFA, is a weak theory of arithmetic created by removing induction from Peano Arithmetic.  Because it lacks induction, axioms defining exponentiation must be added.

\begin{itemize}
\item $\forall x (x'\neq 0)$ ($0$ is the first number)

\item $\forall x,y (x'=y'\rightarrow x=y)$ (the successor function is one-to-one)

\item $\forall x (x+0=x)$ ($0$ is the additive identity)

\item $\forall x,y(x+y'=(x+y)')$ (addition is the repeated application of the successor function)

\item $\forall x(x\cdot 0=0)$

\item $\forall x,y(x\cdot(y')=x\cdot y+x$ (multiplication is repeated addition)

\item $\forall x(\neg (x<0))$ ($0$ is the smallest number)

\item $\forall x,y(x<y'\leftrightarrow x<y\vee x=y)$

\item $\forall x(x^0=1)$

\item $\forall x(x^{y'}=x^y\cdot x)$

\end{itemize}
%%%%%
%%%%%
\end{document}
