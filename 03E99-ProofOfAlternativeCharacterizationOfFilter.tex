\documentclass[12pt]{article}
\usepackage{pmmeta}
\pmcanonicalname{ProofOfAlternativeCharacterizationOfFilter}
\pmcreated{2013-03-22 14:43:05}
\pmmodified{2013-03-22 14:43:05}
\pmowner{rspuzio}{6075}
\pmmodifier{rspuzio}{6075}
\pmtitle{proof of alternative characterization of filter}
\pmrecord{5}{36340}
\pmprivacy{1}
\pmauthor{rspuzio}{6075}
\pmtype{Proof}
\pmcomment{trigger rebuild}
\pmclassification{msc}{03E99}
\pmclassification{msc}{54A99}

% this is the default PlanetMath preamble.  as your knowledge
% of TeX increases, you will probably want to edit this, but
% it should be fine as is for beginners.

% almost certainly you want these
\usepackage{amssymb}
\usepackage{amsmath}
\usepackage{amsfonts}

% used for TeXing text within eps files
%\usepackage{psfrag}
% need this for including graphics (\includegraphics)
%\usepackage{graphicx}
% for neatly defining theorems and propositions
%\usepackage{amsthm}
% making logically defined graphics
%%%\usepackage{xypic}

% there are many more packages, add them here as you need them

% define commands here
\begin{document}
First, suppose that $\mathbf{F}$ is a filter.  We shall show that, for
any two elements $A$ and $B$ of $\mathbf{F}$, it is the case that $A
\cap B \in \mathbf{F}$ if and only if $A \in \mathbf{F}$ and $B \in
\mathbf{F}$.

By the definition of filter, if $A \in \mathbf{F}$ and $B \in
\mathbf{F}$ then $A \cap B \in \mathbf{F}$.  Since $A \supseteq A \cap
B$ and $\mathbf{F}$ is a filter, $A \cap B \in \mathbf{F}$ implies $A
\in \mathbf{F}$.  Likewise, $A \cap B \in \mathbf{F}$ implies $B \in
\mathbf{F}$.

Next, we shall show that any proper subset $\mathbf{F}$ of the power
set of $X$ such that $A \cap B \in \mathbf{F}$ if and only if $A \in
\mathbf{F}$ and $B \in \mathbf{F}$ is a filter.

If the empty set were to belong to $\mathbf{F}$ then for any $A
\subset X$, we would have $A \cap \emptyset = \emptyset \in
\mathbf{F}$.  This would imply that every subset of $X$ belongs to
$\mathbf{F}$, contrary to our hypothesis that $\mathbf{F}$ is a proper
subset of the power set of $X$.

If $A \subseteq B \subseteq X$ and $A \in \mathbf{F}$, then $A \cap B
= A \in \mathbf{F}$.  By our hypothesis, $B \in \mathbf{F}$.

The third defining property of a filter --- If $A \in \mathbf{F}$ and
$B \in \mathbf{F}$ then $A \cap B \in \mathbf{F}$ --- is part of our
hypothesis.
%%%%%
%%%%%
\end{document}
