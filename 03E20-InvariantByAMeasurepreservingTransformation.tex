\documentclass[12pt]{article}
\usepackage{pmmeta}
\pmcanonicalname{InvariantByAMeasurepreservingTransformation}
\pmcreated{2013-03-22 18:04:15}
\pmmodified{2013-03-22 18:04:15}
\pmowner{asteroid}{17536}
\pmmodifier{asteroid}{17536}
\pmtitle{invariant by a measure-preserving transformation}
\pmrecord{4}{40604}
\pmprivacy{1}
\pmauthor{asteroid}{17536}
\pmtype{Definition}
\pmcomment{trigger rebuild}
\pmclassification{msc}{03E20}
\pmclassification{msc}{28D05}
\pmclassification{msc}{37A05}
%\pmkeywords{invariant}

\endmetadata

% this is the default PlanetMath preamble.  as your knowledge
% of TeX increases, you will probably want to edit this, but
% it should be fine as is for beginners.

% almost certainly you want these
\usepackage{amssymb}
\usepackage{amsmath}
\usepackage{amsfonts}

% used for TeXing text within eps files
%\usepackage{psfrag}
% need this for including graphics (\includegraphics)
%\usepackage{graphicx}
% for neatly defining theorems and propositions
%\usepackage{amsthm}
% making logically defined graphics
%%%\usepackage{xypic}

% there are many more packages, add them here as you need them

% define commands here

\begin{document}
\PMlinkescapephrase{invariant}
\PMlinkescapephrase{properties}
\PMlinkescapephrase{property}

Let $X$ be a set and $T:X \longrightarrow X$ a transformation of $X$.

The notion of invariance by $T$ we are about to describe is stronger than the usual notion of \PMlinkname{invariance}{invariant}, and is especially useful in ergodic theory. Thus, in most applications, $(X, \mathfrak{B}, \mu)$ is a measure space and $T$ is a measure-preserving transformation. Nevertheless, the definition of invariance and its properties are general and do not require any such assumptions.

$\,$

{\bf Definition -} A subset $A \subseteq X$ is said to be \emph{invariant} by $T$, or $T$-\emph{invariant}, if $T^{-1}(A)=A$.

$\,$

The fundamental property of this concept is the following: if $A$ is invariant by $T$, then so is $X \setminus A$.

Thus, when $A$ is invariant by $T$ we obtain by restriction two well-defined transformations
\begin{align*}
T|_A :A \longrightarrow A\\
T|_{X\setminus A} : X\setminus A \longrightarrow X\setminus A
\end{align*}

Hence, the existence of an \PMlinkescapetext{invariant subset} allows one to decompose the set $X$ into two disjoint subsets and study the transformation $T$ in each of these subsets.

{\bf Remark -} When $T$ is a measure-preserving transformation in a measure space $(X, \mathfrak{B}, \mu)$ one usually restricts the notion of invariance to measurable subsets $A \in \mathfrak{B}$.
%%%%%
%%%%%
\end{document}
