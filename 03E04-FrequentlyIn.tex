\documentclass[12pt]{article}
\usepackage{pmmeta}
\pmcanonicalname{FrequentlyIn}
\pmcreated{2013-03-22 17:14:23}
\pmmodified{2013-03-22 17:14:23}
\pmowner{CWoo}{3771}
\pmmodifier{CWoo}{3771}
\pmtitle{frequently in}
\pmrecord{6}{39569}
\pmprivacy{1}
\pmauthor{CWoo}{3771}
\pmtype{Definition}
\pmcomment{trigger rebuild}
\pmclassification{msc}{03E04}
\pmsynonym{clusters at}{FrequentlyIn}
\pmdefines{cluster point of a net}

\endmetadata

\usepackage{amssymb,amscd}
\usepackage{amsmath}
\usepackage{amsfonts}
\usepackage{mathrsfs}

% used for TeXing text within eps files
%\usepackage{psfrag}
% need this for including graphics (\includegraphics)
%\usepackage{graphicx}
% for neatly defining theorems and propositions
\usepackage{amsthm}
% making logically defined graphics
%%\usepackage{xypic}
\usepackage{pst-plot}
\usepackage{psfrag}

% define commands here
\newtheorem{prop}{Proposition}
\newtheorem{thm}{Theorem}
\newtheorem{ex}{Example}
\newcommand{\real}{\mathbb{R}}
\newcommand{\pdiff}[2]{\frac{\partial #1}{\partial #2}}
\newcommand{\mpdiff}[3]{\frac{\partial^#1 #2}{\partial #3^#1}}
\begin{document}
Recall that a net is a function $x$ from a directed set $D$ to a set $X$.  The value of $x$ at $i\in D$ is usually denoted by $x_i$.  Let $A$ be a subset of $X$.  We say that a net $x$ is \emph{frequently in} $A$ if for every $i\in D$, there is a $j\in D$ such that $i\le j$ and $x_j\in A$.

Suppose a net $x$ is frequently in $A\subseteq X$.  Let $E:=\lbrace j\in D\mid x_j\in A\rbrace$.  Then $E$ is a cofinal subset of $D$, for if $i\in D$, then by definition of $A$, there is $i\le j\in D$ such that $x_j\in A$, and therefore $j\in E$.

The notion of ``frequently in'' is related to the notion of ``eventually in'' in the following sense: a net $x$ is eventually in a set $A\subseteq X$ iff it is not frequently in $A^{\complement}$, its complement.  Suppose $x$ is eventually in $A$.  There is $j\in D$ such that $x_k\in A$ for all $k\ge j$, or equivalently, $x_k\in A^{\complement}$ for no $k\ge j$.  The converse is can be argued by tracing the previous statements backwards.

In a topological space $X$, a point $a\in X$ is said to be a \emph{cluster point of a net} $x$ (or, occasionally, $x$ \emph{clusters at} $a$) if $x$ is frequently in every neighborhood of $a$.  In this general definition, a limit point is always a cluster point.  But a cluster point need not be a limit point.  As an example, take the sequence $0,2,0,4,0,6,0,8,\ldots,0,2n,0,\ldots$ has $0$ as a cluster point.  But clearly $0$ is not a limit point, as the sequence diverges in $\mathbb{R}$.
%%%%%
%%%%%
\end{document}
