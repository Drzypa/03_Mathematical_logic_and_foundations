\documentclass[12pt]{article}
\usepackage{pmmeta}
\pmcanonicalname{PropertiesOfSubstitutability}
\pmcreated{2013-03-22 19:35:51}
\pmmodified{2013-03-22 19:35:51}
\pmowner{CWoo}{3771}
\pmmodifier{CWoo}{3771}
\pmtitle{properties of substitutability}
\pmrecord{23}{42588}
\pmprivacy{1}
\pmauthor{CWoo}{3771}
\pmtype{Result}
\pmcomment{trigger rebuild}
\pmclassification{msc}{03B10}
\pmclassification{msc}{03B05}

\endmetadata

\usepackage{amssymb,amscd}
\usepackage{amsmath}
\usepackage{amsfonts}
\usepackage{mathrsfs}
\usepackage{proof}
\usepackage{bussproofs}

% used for TeXing text within eps files
%\usepackage{psfrag}
% need this for including graphics (\includegraphics)
%\usepackage{graphicx}
% for neatly defining theorems and propositions
\usepackage{amsthm}
% making logically defined graphics
%%\usepackage{xypic}
\usepackage{pst-plot}
\usepackage{multicol}
\usepackage{enumerate}
\usepackage{tabls}

% define commands here
\newcommand*{\abs}[1]{\left\lvert #1\right\rvert}
\newtheorem{prop}{Proposition}
\newtheorem{thm}{Theorem}
\newtheorem{lem}{Lemma}
\newtheorem{cor}{Corollary}
\newtheorem{ex}{Example}

\begin{document}
In this entry, we list some basic properties of substitutability in first order logic with respect to commutativity.  

\begin{prop} If $x,y$ are distinct variables, then $$t[s/x][r/y]=t[r/y][s[r/y]/x],$$ provided that $x$ does not occur in $r$. \end{prop}
\begin{proof}
Suppose $x$ and $y$ are distinct variables, and $r,s,t$ are terms.  We do induction on the complexity of $t$.
\begin{enumerate}
\item First, suppose $t$ is a constant symbol.  Then LHS and RHS are both $t$.  
\item Next, suppose $t$ is a variable.  
\begin{itemize}
\item If $t$ is $x$, then LHS $=s[r/y]$, and since $y$ is not $x$, RHS $=x[s[r/y]/x]=s[r/y]$.  
\item If $t$ is $y$, then LHS $=y[r/y]=r$, since $x$ is not $y$, and RHS $=r[s[r/y]/x]=r$, since $x$ does not occur in $r$.  
\item If $t$ is neither $x$ nor $y$, then both sides are $t$.
\end{itemize}
In all three cases, LHS $=$ RHS.  
\item Finally, suppose $t$ is of the form $f(t_1,\ldots,t_n)$.  Then LHS $= f(t_1[s/x],\ldots, t_n[s/x]) [r/y] = f(t_1[s/x][r/y], \ldots, t_n[s/x][r/y])$, which, by induction, is $$f(t_1[r/y][s[r/y]/x], \ldots, t_n[r/y][s[r/y]/x])$$  or $f(t_1[r/y], \ldots, t_n[r/y]) [s[r/y]/x] = f(t_1,\ldots, t_n) [r/y][s[r/y]/x] = $RHS.
\end{enumerate}
\end{proof}
Now, if $s$ is $y$, then $t[y/x][r/y]=t[r/y][r/x]$, and we record the following corollary:
\begin{cor} If $x$ is not in $r$ and $y$ not in $t$, then $t[y/x][r/y]=t[r/x]$. \end{cor}
The only thing we need to show is the case when $x$ is $y$, but this is also clear, as $t[y/x][r/y]=t[x/x][r/x]=t[r/x]$.

With respect to formulas, we have a similar proposition:
\begin{prop} If $x,y$ are distinct variables, then $$A[t/x][s/y]=A[s/y][t[s/y]/x],$$ provided that $x$ does not occur in $s$, and $t$ and $s$ are respectively free for $x$ and $y$ in $A$. \end{prop}
\begin{proof} Suppose $x$ and $y$ are distinct variables, $s,t$ terms, and $A$ a wff.  We do induction on the complexity of $A$.
\begin{enumerate}
\item First, suppose $A$ is atomic.
\begin{itemize}
\item $A$ is of the form $t_1 = t_2$, then LHS is $t_1[t/x][s/y]=t_2[t/x][s/y]$ and we can apply the previous equation to both $t_1$ and $t_2$ to get RHS.  
\item If $A$ is of the form $R(t_1,\ldots,t_n)$, then LHS is $R(t_1[t/x][s/y],\ldots, t_n[t/x][s/y])$, and we again apply the previous equation to each $t_i$ to get RHS.  
\end{itemize}
\item Next, suppose $A$ is of the form $B\to C$.  Then LHS $=B[t/x][s/y] \to C[t/x][s/y]$, and, by induction, is $B[s/y][t[s/y]/x] \to C[s/y][t[s/y]/x] = $ RHS.  
\item Finally, suppose $A$ is of the form $\exists z B$.
\begin{itemize}
\item $x$ is $z$.  Then $A[t/x][s/y]=A[s/y]$, and $A[s/y][t[s/y]/x]=A[s/y]$ since $x$ is bound in $A[s/y]$.
\item $x$ is not $z$.  Then $A[t/x][s/y]=(\exists z B[t/x])[s/y]$.
\begin{itemize}
\item $y$ is $z$.  Then $(\exists z B[t/x])[s/y] = \exists z B[t/x]$.  On the other hand, $A[s/y][t[s/y]/x] = A[t[s/y]/x] = \exists z B[t[s/y]/x]$.  By induction, $t,s$ are free for $x,y$ in $B$, and $B[t/x]=B[t[s/y]/x]$, the result follows.
\item $y$ is not $z$.  Then $(\exists z B[t/x])[s/y] = \exists z B[t/x][s/y]$.  On the other hand, $A[s/y][t[s/y]/x] = (\exists z B[s/y])[t[s/y]/x] = \exists z B[s/y][t[s/y]/x]$ since $x$ is not $z$.  By induction again, $t,s$ are free for $x,y$ in $B$, and $B[t/x]=B[t[s/y]/x]$, the result follows once more.
\end{itemize}
\end{itemize}
\end{enumerate}
\end{proof}
Now, if $t$ is $y$, then $A[y/x][s/y]=A[s/y][s/x]$, and we record the following corollary:
\begin{cor} If $y$ is not free in $A$, and is free for $x$ in $A$, then $A[y/x][s/y]=A[s/x]$. \end{cor}

%%%%%
%%%%%
\end{document}
