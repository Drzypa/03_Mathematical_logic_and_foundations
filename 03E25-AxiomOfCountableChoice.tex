\documentclass[12pt]{article}
\usepackage{pmmeta}
\pmcanonicalname{AxiomOfCountableChoice}
\pmcreated{2013-03-22 14:46:23}
\pmmodified{2013-03-22 14:46:23}
\pmowner{yark}{2760}
\pmmodifier{yark}{2760}
\pmtitle{axiom of countable choice}
\pmrecord{14}{36418}
\pmprivacy{1}
\pmauthor{yark}{2760}
\pmtype{Definition}
\pmcomment{trigger rebuild}
\pmclassification{msc}{03E25}
\pmsynonym{countable axiom of choice}{AxiomOfCountableChoice}
\pmsynonym{countable AC}{AxiomOfCountableChoice}
%\pmkeywords{choice}
\pmdefines{countable choice}

\usepackage{amssymb}
\usepackage{amsmath}
\usepackage{amsfonts}
\begin{document}
\PMlinkescapeword{between}
\PMlinkescapeword{link}
\PMlinkescapeword{moment}
\PMlinkescapeword{states}

The \emph{Axiom of Countable Choice} (CC) is a weak form of the \PMlinkname{Axiom of Choice}{AxiomOfChoice}.
It states that every countable set of nonempty sets has a choice function.

\PMlinkescapetext{ZF+CC} (that is, the Zermelo-Fraenkel axioms together with the Axiom of Countable Choice) suffices to prove that the union of countably many countable sets is countable. It also suffices to prove that every infinite set has a countably infinite subset, and that a set $X$ is infinite if and only if there is a bijection between $X$ and a proper subset of $X$.
%%%%%
%%%%%
\end{document}
