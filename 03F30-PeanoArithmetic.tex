\documentclass[12pt]{article}
\usepackage{pmmeta}
\pmcanonicalname{PeanoArithmetic}
\pmcreated{2013-03-22 12:32:42}
\pmmodified{2013-03-22 12:32:42}
\pmowner{alozano}{2414}
\pmmodifier{alozano}{2414}
\pmtitle{Peano arithmetic}
\pmrecord{8}{32789}
\pmprivacy{1}
\pmauthor{alozano}{2414}
\pmtype{Axiom}
\pmcomment{trigger rebuild}
\pmclassification{msc}{03F30}
\pmrelated{NaturalNumber}
\pmrelated{PressburgerArithmetic}
\pmrelated{ElementaryFunctionalArithmetic}
\pmrelated{PeanoArithmeticFirstOrder}
\pmdefines{Peano's axioms}
\pmdefines{successor}
\pmdefines{axiom of induction}

\endmetadata

\usepackage{amssymb}
\usepackage{amsmath}
\usepackage{amsfonts}
\begin{document}
\emph{Peano's axioms} are a definition of the set of natural numbers, denoted $\mathbb{N}$.  From these axioms \emph{Peano arithmetic} on natural numbers can be derived.

\begin{enumerate}
\item $0\in\mathbb{N}$ (0 is a natural number)
\item For each $x\in\mathbb{N}$, there exists exactly one $x'\in\mathbb{N}$, called the \emph{successor} of $x$
\item $x'\neq 0$ (0 is not the successor of any natural number)
\item $x = y$ if and only if $x' = y'$.
\item (\emph{axiom of induction}) If $M\subseteq\mathbb{N}$ and $0\in M$ and $x\in M$ implies $x'\in M$, then $M = \mathbb{N}$.
\end{enumerate}

%The first axiom gives a starting point for constructing the set (note that often 0 is used instead of 1).  The second axiom provides a means of constructing the rest of the set.

The \emph{successor} of $x$ is sometimes denoted $Sx$ instead of $x'$.  We then have $1 = S0$, $2 = S1 = SS0$, and so on.

Peano arithmetic consists of statements derived via these axioms.  For instance, from these axioms we can define addition and multiplication on natural numbers.
Addition is defined as

\begin{eqnarray*}
x+1 & = & x'\quad\text{for all }x\in\mathbb{N} \\
x+y' & = & (x+y)'\quad\text{for all }x,y\in\mathbb{N}
\end{eqnarray*}

Addition defined in this manner can then be proven to be both associative and commutative.

Multiplication is

\begin{eqnarray*}
x\cdot 1 & = & x\quad\text{for all }x\in\mathbb{N} \\
x\cdot y' & = & x\cdot y + x\quad\text{for all }x,y\in\mathbb{N}
\end{eqnarray*}

This definition of multiplication can also be proven to be both associative and commutative, and it can also be shown to be distributive over addition.
%%%%%
%%%%%
\end{document}
