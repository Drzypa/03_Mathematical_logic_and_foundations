\documentclass[12pt]{article}
\usepackage{pmmeta}
\pmcanonicalname{SimpleFunction}
\pmcreated{2013-03-22 12:21:16}
\pmmodified{2013-03-22 12:21:16}
\pmowner{mps}{409}
\pmmodifier{mps}{409}
\pmtitle{simple function}
\pmrecord{9}{32022}
\pmprivacy{1}
\pmauthor{mps}{409}
\pmtype{Definition}
\pmcomment{trigger rebuild}
\pmclassification{msc}{03-00}
\pmclassification{msc}{26A09}
\pmclassification{msc}{26-00}
\pmclassification{msc}{28-00}
\pmrelated{CharacteristicFunction}
\pmrelated{Integral2}
\pmdefines{step function}

\endmetadata

\usepackage{amssymb}
\usepackage{amsmath}
\usepackage{amsfonts}

% used for TeXing text within eps files
%\usepackage{psfrag}
% need this for including graphics (\includegraphics)
%\usepackage{graphicx}
% for neatly defining theorems and propositions
%\usepackage{amsthm}
% making logically defined graphics
%%%\usepackage{xypic} 

% there are many more packages, add them here as you need them

% define commands here
\newcommand{\md}{d}
\newcommand{\mv}[1]{\mathbf{#1}}	% matrix or vector
\newcommand{\mvt}[1]{\mv{#1}^{\mathrm{T}}}
\newcommand{\mvi}[1]{\mv{#1}^{-1}}
\newcommand{\mderiv}[1]{\frac{\md}{\md {#1}}} %d/dx
\newcommand{\mnthderiv}[2]{\frac{\md^{#2}}{\md {#1}^{#2}}} %d^n/dx
\newcommand{\mpderiv}[1]{\frac{\partial}{\partial {#1}}} %partial d^n/dx
\newcommand{\mnthpderiv}[2]{\frac{\partial^{#2}}{\partial {#1}^{#2}}} %partial d^n/dx
\newcommand{\borel}{\mathfrak{B}}
\newcommand{\integers}{\mathbb{Z}}
\newcommand{\rationals}{\mathbb{Q}}
\newcommand{\reals}{\mathbb{R}}
\newcommand{\complexes}{\mathbb{C}}
\newcommand{\naturals}{\mathbb{N}}
\newcommand{\defined}{:=}
\newcommand{\var}{\mathrm{var}}
\newcommand{\cov}{\mathrm{cov}}
\newcommand{\corr}{\mathrm{corr}}
\newcommand{\set}[1]{\{#1\}}
\newcommand{\bra}[1]{\langle#1 \vert}
\newcommand{\ket}[1]{\vert \hspace{1pt}#1\rangle}
\newcommand{\braket}[2]{\langle #1 \ket{#2}}
\begin{document}
In measure theory, a \emph{simple function} is a function that is a
finite linear combination
\[
h = \sum_{k=1}^n c_k \chi_{A_k}
\]
of characteristic functions, where the $c_k$ are real coefficients and
every $A_k$ is a measurable set with respect to a fixed measure space.

If the measure space is $\mathbb{R}$ and each $A_k$ is an interval,
then the function is called a \emph{step function}.  Thus, every step
function is a simple function.

Simple functions are used in analysis to interpolate between
characteristic functions and measurable functions.  In other words,
characteristic functions are easy to integrate:
\[
\int_E \chi_{A}\,dx = |A|,
\]
while simple functions are not much harder to integrate:
\[
\int_E \sum_{k=1}^n c_k \chi_{A_k}\,dx = \sum_{k=1}^n c_k |A_k|.
\]
To integrate a measurable function, one approximates it from below by
simple functions.  Thus, simple functions can be used to define the
Lebesgue integral over a subset of the measure space.

\PMlinkescapeword{words}
%%%%%
%%%%%
\end{document}
