\documentclass[12pt]{article}
\usepackage{pmmeta}
\pmcanonicalname{PrimitiveRecursiveNumber}
\pmcreated{2013-03-22 19:06:06}
\pmmodified{2013-03-22 19:06:06}
\pmowner{CWoo}{3771}
\pmmodifier{CWoo}{3771}
\pmtitle{primitive recursive number}
\pmrecord{13}{41994}
\pmprivacy{1}
\pmauthor{CWoo}{3771}
\pmtype{Definition}
\pmcomment{trigger rebuild}
\pmclassification{msc}{03D25}
\pmclassification{msc}{68Q05}
\pmrelated{BombellisMethodOfComputingSquareRoots}

\usepackage{amssymb,amscd}
\usepackage{amsmath}
\usepackage{amsfonts}
\usepackage{mathrsfs}

% used for TeXing text within eps files
%\usepackage{psfrag}
% need this for including graphics (\includegraphics)
%\usepackage{graphicx}
% for neatly defining theorems and propositions
\usepackage{amsthm}
% making logically defined graphics
%%\usepackage{xypic}
\usepackage{pst-plot}

% define commands here
\newcommand*{\abs}[1]{\left\lvert #1\right\rvert}
\newtheorem{prop}{Proposition}
\newtheorem{thm}{Theorem}
\newtheorem{ex}{Example}
\newcommand{\real}{\mathbb{R}}
\newcommand{\pdiff}[2]{\frac{\partial #1}{\partial #2}}
\newcommand{\mpdiff}[3]{\frac{\partial^#1 #2}{\partial #3^#1}}
\begin{document}
A special \PMlinkescapetext{type} of computable numbers is so-called the \emph{primitive recursive numbers}.  Informally, these are numbers that can be measured by primitive recursive functions to an arbitrary degree of precision.

\textbf{Definition}.  A non-negative real number $r$ is said to be \emph{primitive recursive} if there is a primitive recursive function $f:\mathbb{N} \to \mathbb{N}$ such that 
\begin{displaymath}
f(n) = \left\{
\begin{array}{ll}
[r] \textrm{  }(\textrm{the integer part of }r), & \textrm{if } n=0, \\
n^\textrm{th}\textrm{ digit of }r\textrm{ when }r\textrm{ is expressed in its decimal representation,} & \textrm{if } n\ne 0.
\end{array}
\right.
\end{displaymath}
A real number $r$ is \emph{primitive recursive} if $|r|$ is, and a complex number $x+yi$ is \emph{primitive recursive} if both $x$ and $y$ are.

Clearly, any integer is primitive recursive.  It is easy to see that all rational numbers are primitive recursive too, as the decimal representation of a rational number is periodic, so if $$r=[r].\overline{a_1\cdots a_k},$$ we can define $f$ so that
\begin{displaymath}
f(n) = \left\{
\begin{array}{ll}
[r] , & \textrm{if } n=0, \\
a_i & \textrm{if } n\ne 0 \mbox{ and } n \equiv i \pmod k.
\end{array}
\right.
\end{displaymath}
Here, we assume that $r$ is non-negative.

In addition, we can show that $\sqrt{n}$ is primitive recursive for any non-negative integer $n$.
\begin{proof}  Suppose $r=\sqrt{n}$.  Write $r$ in its decimal representation $$r=n_0.n_1 n_2 \cdots n_k \cdots$$
Then $n_0=[\sqrt{n}]$.  Multiply $r$ by $10$ to get its decimal representation $$10r = n_0 n_1.n_2 \cdots n_k \cdots $$
Then $10n_0 + n_1 = [10r]=[\sqrt{100 n}]$, so that $n_1 = [\sqrt{100n}]-10n_0$  By induction, we see that
$$n_{k+1} = [\sqrt{100^{k+1} n} ] - 10 (10^k n_0 + 10^{k-1} n_1 + \cdots + n_k).$$
Define $f:\mathbb{N}^2 \to \mathbb{N}$ by $f(n,m)=n_m$.  Then $f(n,0)$ is primitive recursive.  Next, $$f(n,m)= [\sqrt{100^m n} ]- 10 \sum_{i=0}^{m-1} 10^{m-1-i} f(n,i) = h(n,m, \overline{f}(n,m)),$$ where $$h(x,y,z)= [\sqrt{100^x y} ] \dot{-} 10 \sum_{i=0}^{y \dot{-} 1} 10^{y \dot{-} s(i) } (z)_i $$
which is primitive recursive (all of the operations, including the bounded sum are primitive recursive).  Since $f$ is defined by course-of-values recursion via $h$, $f$ is primitive recursive also.
\end{proof}

\textbf{Remark}.  It can be shown that $\pi$ is primitive recursive.  A proof of this can be found in the link below.

\begin{thebibliography}{9}
\bibitem{SS} S. G. Simpson, \PMlinkexternal{Foundations of Mathematics}{http://www.math.psu.edu/simpson/courses/math558/fom.pdf}. (2009).
\end{thebibliography}
%%%%%
%%%%%
\end{document}
