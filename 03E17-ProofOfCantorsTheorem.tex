\documentclass[12pt]{article}
\usepackage{pmmeta}
\pmcanonicalname{ProofOfCantorsTheorem}
\pmcreated{2013-03-22 12:44:55}
\pmmodified{2013-03-22 12:44:55}
\pmowner{Wkbj79}{1863}
\pmmodifier{Wkbj79}{1863}
\pmtitle{proof of Cantor's theorem}
\pmrecord{7}{33053}
\pmprivacy{1}
\pmauthor{Wkbj79}{1863}
\pmtype{Proof}
\pmcomment{trigger rebuild}
\pmclassification{msc}{03E17}
\pmclassification{msc}{03E10}
%\pmkeywords{diagonal argument}

\endmetadata

% this is the default PlanetMath preamble.  as your knowledge
% of TeX increases, you will probably want to edit this, but
% it should be fine as is for beginners.
\def\equiv{\Leftrightarrow}

% almost certainly you want these
\usepackage{amssymb}
\usepackage{amsmath}
\usepackage{amsfonts}

% used for TeXing text within eps files
%\usepackage{psfrag}
% need this for including graphics (\includegraphics)
%\usepackage{graphicx}
% for neatly defining theorems and propositions
%\usepackage{amsthm}
% making logically defined graphics
%%%\usepackage{xypic}

% there are many more packages, add them here as you need them

% define commands here
\def\P{{\mathcal P}}
\def\sse{\subseteq}
\begin{document}
The proof of this theorem is fairly \PMlinkescapetext{simple} using the following construction, which is central to Cantor's diagonal argument.

Consider a function $F\colon X\to \P(X)$ from a set $X$ to its power set. Then we define the set $Z\sse X$ as follows:

\[ Z = \{x\in X \mid x\not\in F(x)\}\]

Suppose that $F$ is a bijection. Then there must exist an $x\in X$ such that $F(x)=Z$. Then we have the following contradiction:

\[ x\in Z \equiv x\not\in F(x) \equiv x\not\in Z\]

Hence, $F$ cannot be a bijection between $X$ and $\P(X)$.
%%%%%
%%%%%
\end{document}
