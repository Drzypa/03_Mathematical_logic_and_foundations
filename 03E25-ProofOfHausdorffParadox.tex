\documentclass[12pt]{article}
\usepackage{pmmeta}
\pmcanonicalname{ProofOfHausdorffParadox}
\pmcreated{2013-03-22 15:16:18}
\pmmodified{2013-03-22 15:16:18}
\pmowner{GrafZahl}{9234}
\pmmodifier{GrafZahl}{9234}
\pmtitle{proof of Hausdorff paradox}
\pmrecord{6}{37059}
\pmprivacy{1}
\pmauthor{GrafZahl}{9234}
\pmtype{Proof}
\pmcomment{trigger rebuild}
\pmclassification{msc}{03E25}
\pmclassification{msc}{51M04}
\pmclassification{msc}{20F05}
%\pmkeywords{decomposition}
\pmrelated{ChoiceFunction}
\pmrelated{BanachTarskiParadox}

\endmetadata

% this is the default PlanetMath preamble.  as your knowledge
% of TeX increases, you will probably want to edit this, but
% it should be fine as is for beginners.

% almost certainly you want these
\usepackage{amssymb}
\usepackage{amsmath}
\usepackage{amsfonts}

% used for TeXing text within eps files
%\usepackage{psfrag}
% need this for including graphics (\includegraphics)
%\usepackage{graphicx}
% for neatly defining theorems and propositions
%\usepackage{amsthm}
% making logically defined graphics
%%%\usepackage{xypic}

% there are many more packages, add them here as you need them

% define commands here
\newcommand{\Bigcup}{\bigcup\limits}
\newcommand{\Prod}{\prod\limits}
\newcommand{\Sum}{\sum\limits}
\newcommand{\mbb}{\mathbb}
\newcommand{\mbf}{\mathbf}
\newcommand{\mc}{\mathcal}
\newcommand{\mmm}[9]{\left(\begin{array}{rrr}#1&#2&#3\\#4&#5&#6\\#7&#8&#9\end{array}\right)}
\newcommand{\ol}{\overline}

% Math Operators/functions
\DeclareMathOperator{\Frob}{Frob}
\DeclareMathOperator{\cwe}{cwe}
\DeclareMathOperator{\we}{we}
\DeclareMathOperator{\wt}{wt}
\begin{document}
\PMlinkescapeword{complete}
\PMlinkescapeword{component}
\PMlinkescapeword{decomposition}
\PMlinkescapeword{meets}
\PMlinkescapeword{relation}
\PMlinkescapeword{relations}
\PMlinkescapeword{restriction}
\PMlinkescapeword{structure}
\PMlinkescapeword{type}
\PMlinkescapeword{unit}
We start by outlining the general ideas, followed by the strict proof.

\subsection*{General approach}

The unit sphere $S^2$ in the Euclidean space $\mbb{R}^3$ is
finite in the sense that it is \PMlinkid{bounded}{4826} and has a finite volume:
$\frac{4}{3}\pi$.

On the other hand, $S^2$ is an infinite point set. As everyone who has
ever visited Hilbert's Hotel knows, it is possible to split an
infinite set, such as the natural numbers, in pieces and all pieces
are, in a sense, equal to the original set, for example if
\begin{equation*}
\mbb{N}_3:=\{n\in\mbb{N}\mid n\text{ is divisible by }3\},
\end{equation*}
then na\"{i}vely $\mbb{N}_3$ has a third of the size of $\mbb{N}$, and
half of the size of $\mbb{N}\setminus\mbb{N}_3$. There exists,
however, a bijection from $\mbb{N}$ to $\mbb{N}\setminus\mbb{N}_3$, so
$\mbb{N}_3$ has, again na\"{i}vely, both half and a third of
the size of $\mbb{N}$.

To do the same thing with $S^2$, bijections won't suffice, however: we
need isometries to establish congruence. We can \emph{almost} reduce
this problem to the case of bijections in the following way. The group
$R$ of rotations (rotations are isometries) in $\mbb{R}^3$ acts on
$S^2$ (say, from the right). Given an countably infinite subgroup $G$
of $R$ which acts freely (or almost so) on $S^2$, take a set of \PMlinkid{orbit}{1517}
representatives $M$ of the action of $G$ on $S^2$ (this requires the
\PMlinkid{axiom of choice}{310}). Then a disjoint decomposition
$G_1,\ldots G_n$ of $G$ into
countable sets acting on $M$ yields a disjoint decomposition of
$S^2$. Doing similar juggling with $G_1,\ldots,G_n$ as we did with
$\mbb{N}$, $\mbb{N}_3$ and $\mbb{N}\setminus\mbb{N}_3$ above should
yield the result, \emph{if} all the $G_1,\ldots G_n$ are related by fixed
isometries, for example if $G_1=G_2\varphi$ for some isometry
$\varphi$, and similar relations for the other pieces. This last
restriction will make the proof possible only in three or more
dimensions, and indeed, Banach and Tarski later showed in~\cite{BT}
that analogous theorems on the line or in the plane do not hold.

\subsection*{The proof}

Let $G$ be the subgroup of rotations of $\mbb{R}^3$ generated by a
half rotation $\varphi$ and a one third rotation $\psi$ around
different axes. We fix an orthonormal basis for the rest
of the proof, so we may identify $\varphi$ and $\psi$ with their
rotation matrices acting from the right on the row vectors of
$\mbb{R}^3$, such that without restriction
\begin{equation*}
\psi=\mmm{\lambda}{\mu}{0}{-\mu}{\lambda}{0}{0}{0}{1},\qquad\varphi=\mmm{-\cos\vartheta}{0}{\sin\vartheta}{0}{-1}{0}{\sin\vartheta}{0}{\cos\vartheta},
\end{equation*}
where $\lambda=\cos\frac{2\pi}{3}=-\frac{1}{2}$,
$\mu=\sin\frac{2\pi}{3}=\frac{1}{2}\sqrt{3}$ and $\vartheta$ is
arbitrary for now.

Since $\varphi^2=1$ and $\psi^3=1$, there exists
for each element $g$ from $G$ other than the unity, $\phi$, $\psi$ or $\psi^2$ a
positive integer $n$ and numbers $m_k\in\{1,2\}$, $1\leq k\leq n$,
such that $g$ can be written in one of the following ways:
\begin{itemize}
\item[($\alpha$)]$\left(\Prod_{k=1}^n\varphi\psi^{m_k}\right)$,
\item[($\beta$)]$\psi^{m_1}\left(\Prod_{k=2}^n\varphi\psi^{m_k}\right)\varphi$,
\item[($\gamma$)]$\left(\Prod_{k=1}^n\varphi\psi^{m_k}\right)\varphi$,
\item[($\delta$)]$\psi^{m_1}\left(\Prod_{k=2}^n\varphi\psi^{m_k}\right)$
  ($n\geq 2$).
\end{itemize}

To have complete control over the structure of $G$, we fix $\vartheta$
in such a way, that $g$ can be written \emph{uniquely} in one of the
ways ($\alpha$)--($\delta$), with fixed $n$ and $m_k$. In other words: we
fix $\vartheta$ such that the unity $1$ cannot be written in
one of the ways ($\alpha$)--($\delta$).

To see how to do that, let us see to where the vector $(0,0,1)$ is
transported by a transformation of type ($\alpha$). It is easily
verified that
\begin{equation*}
\varphi\psi=\mmm{-\lambda\cos\vartheta}{-\mu\cos\vartheta}{\sin\vartheta}{\mu}{-\lambda}{0}{\lambda\sin\vartheta}{\mu\sin\vartheta}{\cos\vartheta}\text{
  and }\varphi\psi^2=\mmm{\lambda\cos\vartheta}{-\mu\cos\vartheta}{\sin\vartheta}{\mu}{\lambda}{0}{-\lambda\sin\vartheta}{\mu\sin\vartheta}{\cos\vartheta},
\end{equation*}
so that
$(0,0,1)\varphi\psi=(\lambda\sin\vartheta,\mu\sin\vartheta,\cos\vartheta)$
and
$(0,0,1)\varphi\psi^2=(-\lambda\sin\vartheta,\mu\sin\vartheta,\cos\vartheta)$.
More generally, let $n$ be a positive integer, $p_1,p_2$ polynomials
of degree $n-1$ and $p_3$ a polynomial of degree $n$, then
\begin{multline*}
(p_1(\cos\vartheta)\sin\vartheta,p_2(\cos\vartheta)\sin\vartheta,p_3(\cos\vartheta))\varphi\psi=\\((\lambda
  p_3(\cos\vartheta)+\mu
  p_2(\cos\vartheta)-\lambda\cos\vartheta
  p_1(\cos\vartheta))\sin\vartheta,(\mu p_3(\cos\vartheta)-\lambda
  p_2(\cos\vartheta)-\mu\cos\vartheta
  p_1(\cos\vartheta))\sin\vartheta,(1-\cos^2\vartheta)p_1(\cos\vartheta)+p_3(\cos\vartheta)),
\end{multline*}
\begin{multline*}
(p_1(\cos\vartheta)\sin\vartheta,p_2(\cos\vartheta)\sin\vartheta,p_3(\cos\vartheta))\varphi\psi^2=\\((-\lambda
  p_3(\cos\vartheta)+\mu
  p_2(\cos\vartheta)+\lambda\cos\vartheta
  p_1(\cos\vartheta))\sin\vartheta,(\mu p_3(\cos\vartheta)+\lambda
  p_2(\cos\vartheta)-\mu\cos\vartheta
  p_1(\cos\vartheta))\sin\vartheta,(1-\cos^2\vartheta)p_1(\cos\vartheta)+p_3(\cos\vartheta)).
\end{multline*}
By induction, it follows that $(0,0,1)$ is transported by a
transformation of type ($\alpha$) to a vector whose third component is
a nonconstant polynomial $p$ in $\cos\vartheta$. If we restrict $\vartheta$ such that
$0\leq\vartheta\leq\pi$, there are only finitely many values for
$\vartheta$ such that $p(\cos\vartheta)=1$. Given all possible
combinations of $n$ and $m_k$, $1\leq k\leq n$, there are in total
only countably many problematic values for $\vartheta$, so we can
easily fix hereby an unproblematic one. Now that the case ($\alpha$)
has been dealt with, so have been the others automatically. For assume
one could write a $1$ of type ($\gamma$), one could convert it to type
($\delta$) by $1=\varphi 1\varphi$, and from type $\delta$ to type
($\alpha$) or type ($\beta$) by $1=\psi^{3-m_1}1\psi^{m_1}$, and
lastly from type ($\beta$) to type ($\alpha$) by $1=\varphi 1\varphi$,
completing the contradiction.

Now that we know the structure of $G$, how does it act on $S^2$?
Certainly not freely, since every rotation other than the unity has
its axis as fixed point set, which in case of the sphere makes two
fixed points per rotation. The product of two rotations is again a
rotation. Furthermore $G$ is finitely generated and so is
countable. So there are only countably many points of $S^2$ where the
action of $G$ fails to be free. Denote the set of these points by $D$,
so that $G$ acts freely on $E:=S^2\setminus D$. The action creates a
partition on $E$ into orbits. The \PMlinkescapetext{axiom of choice} allows us to choose
a set $M$, such that $M$ meets any orbit in precisely one element. We
have then the disjoint union
\begin{equation*}
E=\Bigcup_{g\in G}Mg,
\end{equation*}
where $Mg$ is the image of $M$ under the action of the group element
$g$. We now define three sets $A$, $B$, $C$ to be the smallest sets satisfying:
\begin{itemize}
\item$M1=M\subseteq A$;
\item if $Mg$ is a subset of $A$, $B$ or $C$, then $Mg\varphi$ is a
  subset of $B$, $A$ or $A$, respectively;
\item if $Mg$ is a subset of $A$, $B$ or $C$, then $Mg\psi$ is a
  subset of $B$, $C$ or $A$, respectively;
\item if $Mg$ is a subset of $A$, $B$ or $C$, then $Mg\psi^2$ is a
  subset of $C$, $A$ or $B$, respectively.
\end{itemize}
The sets $A$, $B$ and $C$ are well-defined because we ensured the
uniqueness of the representations ($\alpha$)--($\delta$) above.

The sets $A$, $B$, $C$ and $B\cup C$ are all congruent by virtue of
\begin{equation*}
A\psi=B,\qquad A\psi^2=C\quad\text{and}\quad A\varphi=B\cup C.
\end{equation*}

Since $S^2=A\cup B\cup C\cup D$, a disjoint union, and $D$ is
countable, the theorem is proven.

\begin{thebibliography}{BT}

\bibitem[BT]{BT} \textsc{St.~Banach, A.~Tarski}, Sur la d\'{e}composition
  des ensembles de points en parties respectivement congruentes,
  \emph{Fund.\ math.}\ 6, 244--277, (1924).

\bibitem[H]{H} \textsc{F.~Hausdorff}, Bemerkung \"{u}ber den Inhalt von
  Punktmengen, \emph{Math.\ Ann.}\ 75, 428--433, (1915).

\end{thebibliography}
%%%%%
%%%%%
\end{document}
