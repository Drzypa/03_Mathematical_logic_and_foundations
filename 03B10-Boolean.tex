\documentclass[12pt]{article}
\usepackage{pmmeta}
\pmcanonicalname{Boolean}
\pmcreated{2013-03-22 11:54:43}
\pmmodified{2013-03-22 11:54:43}
\pmowner{akrowne}{2}
\pmmodifier{akrowne}{2}
\pmtitle{Boolean}
\pmrecord{11}{30543}
\pmprivacy{1}
\pmauthor{akrowne}{2}
\pmtype{Definition}
\pmcomment{trigger rebuild}
\pmclassification{msc}{03B10}
\pmclassification{msc}{55Q35}
\pmclassification{msc}{55Q05}
\pmclassification{msc}{20L05}
\pmclassification{msc}{18D05}
\pmclassification{msc}{18-00}

\endmetadata

\usepackage{amssymb}
\usepackage{amsmath}
\usepackage{amsfonts}
\usepackage{graphicx}
%%%%\usepackage{xypic}
\begin{document}
\section{Boolean}

\emph{Boolean} refers to that which can take on the values ``true'' or ``false,'' or that which concerns truth and falsity.  For example ``Boolean variable,'' ``Boolean logic,'' ``Boolean statement,'' etc.

``Boolean'' is named for George Boole, the 19th century mathematician.

\begin{thebibliography}{3}
\bibitem{MactutorBoole} George Boole at MacTutor: \PMlinkexternal{http://www-gap.dcs.st-and.ac.uk/~history/Mathematicians/Boole.html}{http://www-gap.dcs.st-and.ac.uk/~history/Mathematicians/Boole.html}
\bibitem{WikipediaBoole} George Boole at Wikipedia: \PMlinkexternal{http://en.wikipedia.org/wiki/George_Boole}{http://en.wikipedia.org/wiki/George_Boole}
\end{thebibliography}
%%%%%
%%%%%
%%%%%
%%%%%
\end{document}
