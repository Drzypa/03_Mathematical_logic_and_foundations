\documentclass[12pt]{article}
\usepackage{pmmeta}
\pmcanonicalname{AnotherDefinitionOfCofinality}
\pmcreated{2013-03-22 13:52:59}
\pmmodified{2013-03-22 13:52:59}
\pmowner{x_bas}{2940}
\pmmodifier{x_bas}{2940}
\pmtitle{another definition of cofinality}
\pmrecord{9}{34626}
\pmprivacy{1}
\pmauthor{x_bas}{2940}
\pmtype{Definition}
\pmcomment{trigger rebuild}
\pmclassification{msc}{03E04}
%\pmkeywords{supremum approximation cofinality}

\endmetadata

% this is the default PlanetMath preamble.  as your knowledge
% of TeX increases, you will probably want to edit this, but
% it should be fine as is for beginners.

% almost certainly you want these
\usepackage{amssymb}
\usepackage{amsmath}
\usepackage{amsfonts}

% used for TeXing text within eps files
%\usepackage{psfrag}
% need this for including graphics (\includegraphics)
%\usepackage{graphicx}
% for neatly defining theorems and propositions
%\usepackage{amsthm}
% making logically defined graphics
%%%\usepackage{xypic}

% there are many more packages, add them here as you need them

% define commands here
\begin{document}
\PMlinkescapeword{cofinality}
\PMlinkescapeword{parent}
Let $\kappa$ be a limit ordinal (e.g. a cardinal). The {\em cofinality of $\kappa$} $\operatorname{cf}(\kappa)$ could also be defined as:
$$\operatorname{cf}(\kappa)=\inf \{ |U| : U \subseteq \kappa \text{ such that } \sup \; U = \kappa \} $$
($\sup \; U$ is calculated using the natural order of the ordinals). 
The cofinality of a cardinal is always a regular cardinal and hence $\operatorname{cf}(\kappa) = \operatorname{cf}(\operatorname{cf}(\kappa))$.  

This definition is equivalent to the parent definition.
%%%%%
%%%%%
\end{document}
