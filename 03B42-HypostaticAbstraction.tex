\documentclass[12pt]{article}
\usepackage{pmmeta}
\pmcanonicalname{HypostaticAbstraction}
\pmcreated{2013-03-22 17:54:01}
\pmmodified{2013-03-22 17:54:01}
\pmowner{Jon Awbrey}{15246}
\pmmodifier{Jon Awbrey}{15246}
\pmtitle{hypostatic abstraction}
\pmrecord{18}{40391}
\pmprivacy{1}
\pmauthor{Jon Awbrey}{15246}
\pmtype{Definition}
\pmcomment{trigger rebuild}
\pmclassification{msc}{03B42}
\pmclassification{msc}{03B30}
\pmclassification{msc}{03B22}
\pmclassification{msc}{03B15}
\pmclassification{msc}{03A05}
\pmclassification{msc}{00A30}
\pmsynonym{hypostasis}{HypostaticAbstraction}
\pmsynonym{objectification}{HypostaticAbstraction}
\pmsynonym{reification}{HypostaticAbstraction}
\pmsynonym{subjectal abstraction}{HypostaticAbstraction}
\pmrelated{PrescisiveAbstraction}
\pmrelated{ContinuousPredicate}
\pmdefines{abstract object}
\pmdefines{formal object}
\pmdefines{hypostatic object}

% This is the default PlanetMath preamble.
% as your knowledge of TeX increases, you
% will probably want to edit this, but it
% should be fine as is for beginners.

% Almost certainly you want these:

\usepackage{amssymb}
\usepackage{amsmath}
\usepackage{amsfonts}

% Used for TeXing text within EPS files:

\usepackage{psfrag}

% Need this for including graphics (\includegraphics):

\usepackage{graphicx}

% For neatly defining theorems and propositions:

%\usepackage{amsthm}

% Making logically defined graphics:

%%%\usepackage{xypic}

% There are many more packages, add them here as you need them.

% define commands here

\begin{document}
\PMlinkescapephrase{adapted}
\PMlinkescapephrase{Adapted}
\PMlinkescapephrase{contain}
\PMlinkescapephrase{Contain}
\PMlinkescapephrase{contains}
\PMlinkescapephrase{Contains}
\PMlinkescapephrase{occur}
\PMlinkescapephrase{occurs}
\PMlinkescapephrase{occur in}
\PMlinkescapephrase{occur ins}
\PMlinkescapephrase{occurs in}
\PMlinkescapephrase{occurs ins}
\PMlinkescapephrase{trace}
\PMlinkescapephrase{Trace}
\PMlinkescapephrase{variety}
\PMlinkescapephrase{Variety}

\textbf{Hypostatic abstraction} is a formal operation that takes an element of information, as expressed in a proposition $X\ \operatorname{is}\ Y,$ and conceives its information to consist in the relation between that subject and another subject, as expressed in the proposition $X\ \operatorname{has}\ Y\!\operatorname{-ness}.$  The existence of the abstract subject $Y\!\operatorname{-ness}$ consists solely in the truth of those propositions that contain the concrete predicate $Y.$  Hypostatic abstraction is known under many names, for example, \textit{hypostasis}, \textit{objectification}, \textit{reification}, and \textit{subjectal abstraction}.  The object of discussion or thought thus introduced is termed a \textit{hypostatic object}.

The above definition is adapted from one given by Charles Sanders Peirce (CP 4.235).  The main thing about the formal operation of hypostatic abstraction, insofar as it can be observed to operate on formal linguistic expressions, is that it converts an adjective or some part of a predicate into an extra subject, upping the arity of the main predicate in the process.

For example, a typical case of hypostatic abstraction occurs in the transformation from ``honey is sweet" to ``honey possesses sweetness", which transformation can be viewed in the following variety of ways:

\begin{center}\begin{tabular}{c}
\includegraphics[scale=0.8]{HA_Fig_1}
\\[24pt]
\includegraphics[scale=0.8]{HA_Fig_2}
\\[24pt]
\includegraphics[scale=0.8]{HA_Fig_3}
\\[24pt]
\includegraphics[scale=0.8]{HA_Fig_4}
\end{tabular}\end{center}

The grammatical trace of this hypostatic transformation tells of a process that abstracts the adjective ``sweet" from the predicate ``is sweet", decants the higher-arity predicate ``possesses", and precipitates the substantive ``sweetness" in the role of its correlative subject.

\section{References and further reading}

\begin{itemize}
\item
Peirce, Charles Sanders (1902), ``The Simplest Mathematics", CP 4.227--323 in \textit{Collected Papers of Charles Sanders Peirce}, vols. 1--6, Charles Hartshorne and Paul Weiss (eds.), vols. 7--8, Arthur W. Burks (ed.), Harvard University Press, Cambridge, MA, 1931--1935, 1958.  Cited as (CP volume.paragraph).
\item
Zeman, J. Jay (1982), ``Peirce on Abstraction", \textit{The Monist}, 65 (1982), 211--229.  Reprinted, pp. 293--311 in \textit{The Relevance of Charles Peirce}, Eugene Freeman (ed.), Monist Library of Philosophy, La Salle, IL, 1983.  \PMlinkexternal{Online}{http://web.clas.ufl.edu/users/jzeman/peirce_on_abstraction.htm}.
\end{itemize}

%%%%%
%%%%%
\end{document}
