\documentclass[12pt]{article}
\usepackage{pmmeta}
\pmcanonicalname{Consistent}
\pmcreated{2013-03-22 13:00:20}
\pmmodified{2013-03-22 13:00:20}
\pmowner{Henry}{455}
\pmmodifier{Henry}{455}
\pmtitle{consistent}
\pmrecord{6}{33386}
\pmprivacy{1}
\pmauthor{Henry}{455}
\pmtype{Definition}
\pmcomment{trigger rebuild}
\pmclassification{msc}{03B99}
\pmdefines{inconsistent}

\endmetadata

% this is the default PlanetMath preamble.  as your knowledge
% of TeX increases, you will probably want to edit this, but
% it should be fine as is for beginners.

% almost certainly you want these
\usepackage{amssymb}
\usepackage{amsmath}
\usepackage{amsfonts}

% used for TeXing text within eps files
%\usepackage{psfrag}
% need this for including graphics (\includegraphics)
%\usepackage{graphicx}
% for neatly defining theorems and propositions
%\usepackage{amsthm}
% making logically defined graphics
%%%\usepackage{xypic}

% there are many more packages, add them here as you need them

% define commands here
%\PMlinkescapeword{theory}
\begin{document}
If $T$ is a theory of $\mathcal{L}$ then it is consistent iff there is some model $\mathcal{M}$ of $\mathcal{L}$ such that $\mathcal{M}\vDash T$.  If a theory is not consistent then it is inconsistent.

A slightly different definition is sometimes used, that $T$ is consistent iff $T\not\vdash\bot$ (that is, as long as it does not prove a contradiction).  As long as the proof calculus used is sound and complete, these two definitions are equivalent.
%%%%%
%%%%%
\end{document}
