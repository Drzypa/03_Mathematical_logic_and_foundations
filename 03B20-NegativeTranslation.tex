\documentclass[12pt]{article}
\usepackage{pmmeta}
\pmcanonicalname{NegativeTranslation}
\pmcreated{2013-03-22 19:35:48}
\pmmodified{2013-03-22 19:35:48}
\pmowner{CWoo}{3771}
\pmmodifier{CWoo}{3771}
\pmtitle{negative translation}
\pmrecord{9}{42587}
\pmprivacy{1}
\pmauthor{CWoo}{3771}
\pmtype{Definition}
\pmcomment{trigger rebuild}
\pmclassification{msc}{03B20}
\pmclassification{msc}{03F55}
\pmsynonym{double negative translation}{NegativeTranslation}
\pmdefines{Kolmogorov negative translation}
\pmdefines{Godel negative translation}
\pmdefines{Kuroda negative translation}
\pmdefines{Krivine negative translation}

\endmetadata

\usepackage{amssymb,amscd}
\usepackage{amsmath}
\usepackage{amsfonts}
\usepackage{mathrsfs}
\usepackage{proof}
\usepackage{bussproofs}

% used for TeXing text within eps files
%\usepackage{psfrag}
% need this for including graphics (\includegraphics)
%\usepackage{graphicx}
% for neatly defining theorems and propositions
\usepackage{amsthm}
% making logically defined graphics
%%\usepackage{xypic}
\usepackage{pst-plot}
\usepackage{multicol}
\usepackage{enumerate}
\usepackage{tabls}

% define commands here
\newcommand*{\abs}[1]{\left\lvert #1\right\rvert}
\newtheorem{prop}{Proposition}
\newtheorem{thm}{Theorem}
\newtheorem{lem}{Lemma}
\newtheorem{cor}{Corollary}
\newtheorem{ex}{Example}

\begin{document}
It is well-known that classical propositional logic PL$_c$ can be considered as a subsystem of intuitionistic propositional logic PL$_i$ by translating any wff $A$ in PL$_c$ into $\neg \neg A$ in PL$_i$.  According to Glivenko's theorem, $A$ is a theorem of PL$_c$ iff $\neg \neg A$ is a theorem of PL$_i$.  This translation, however, fails to preserve theoremhood in the corresponding predicate logics.  For example, if $A$ is of the form $\exists x B$, then $\vdash_c A$ no longer implies $\vdash_i \neg \neg A$.  A number of translations have been devised to overcome this defect.  They are collective known as \emph{negative translations} or \emph{double negative translations} of classical logic into intuitionistic logic.  Below is a list of the most commonly mentioned negative translations:
\begin{itemize}
\item \emph{Kolmogorov negative translation}:
\begin{multicols}{2}
\begin{enumerate}
\item $p^{\circ}:= \neg \neg p$ with $p$ atomic and $\perp^{\circ}:=\perp$
\item $(A\land  B)^{\circ}:=\neg \neg (A^{\circ}\land B^{\circ})$
\item $(A\lor B)^{\circ}:=\neg \neg (A^{\circ} \lor B^{\circ})$
\item $(A\to B)^{\circ}:=\neg \neg (A^{\circ}\to B^{\circ})$
\item $(\forall x A)^{\circ}:=\neg\neg \forall x A^{\circ}$
\item $(\exists x A)^{\circ}:=\neg\neg \exists x A^{\circ}$
\end{enumerate}
\end{multicols}
\item \emph{God\"el negative translation}:
\begin{multicols}{2}
\begin{enumerate}
\item $p^-:= \neg \neg p$ with $p$ atomic and $\perp^-:=\perp$
\item $(A\land  B)^-:=A^-\land B^-$
\item $(A\lor B)^-:=\neg \neg (A^- \lor B^-)$
\item $(A\to B)^-:=A^-\to B^-$
\item $(\forall x A)^-:=\forall x A^-$
\item $(\exists x A)^-:=\neg\neg \exists x A^-$
\end{enumerate}
\end{multicols}
\item \emph{Kuroda negative translation}:
\begin{multicols}{2}
\begin{enumerate}
\item $p_u:= p$ with $p$ atomic and $\perp_u:=\perp$
\item $(A\land  B)_u:=A_u\land B_u$
\item $(A\lor B)_u:=A_u \lor B_u$
\item $(A\to B)_u:=A_u\to B_u$
\item $(\forall x A)_u:=\forall x \neg\neg A_u$
\item $(\exists x A)_u:=\exists x A_u$
\end{enumerate}
\end{multicols}
And then $A^u:=\neg \neg A_u$.
\item \emph{Krivine negative translation}:
\begin{multicols}{2}
\begin{enumerate}
\item $p_r:= \neg p$ with $p$ atomic and $\perp_r:=\neg \perp$
\item $(A\land  B)_r:=A_r\lor B_r$
\item $(A\lor B)_r:=A_r \land B_r$
\item $(A\to B)_r:=\neg A_r\land B_r$
\item $(\forall x A)_r:=\exists x A_r$
\item $(\exists x A)_r:=\neg \exists x \neg A_r$
\end{enumerate}
\end{multicols}
And then $A^r:=\neg A_r$.
\end{itemize}

\textbf{Remark}.  It can be shown that for any wff $A$: $$\vdash_i A^* \leftrightarrow A^{\#}\qquad\qquad \mbox{and} \qquad\qquad \vdash_c A \quad \mbox{iff} \quad \vdash_i A^*$$ where $*,\# \in \lbrace \circ, -, u, r\rbrace$.

%%%%%
%%%%%
\end{document}
