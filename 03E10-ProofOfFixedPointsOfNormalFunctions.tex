\documentclass[12pt]{article}
\usepackage{pmmeta}
\pmcanonicalname{ProofOfFixedPointsOfNormalFunctions}
\pmcreated{2013-03-22 13:29:01}
\pmmodified{2013-03-22 13:29:01}
\pmowner{Henry}{455}
\pmmodifier{Henry}{455}
\pmtitle{proof of fixed points of normal functions}
\pmrecord{4}{34055}
\pmprivacy{1}
\pmauthor{Henry}{455}
\pmtype{Proof}
\pmcomment{trigger rebuild}
\pmclassification{msc}{03E10}

% this is the default PlanetMath preamble.  as your knowledge
% of TeX increases, you will probably want to edit this, but
% it should be fine as is for beginners.

% almost certainly you want these
\usepackage{amssymb}
\usepackage{amsmath}
\usepackage{amsfonts}

% used for TeXing text within eps files
%\usepackage{psfrag}
% need this for including graphics (\includegraphics)
%\usepackage{graphicx}
% for neatly defining theorems and propositions
%\usepackage{amsthm}
% making logically defined graphics
%%%\usepackage{xypic}

% there are many more packages, add them here as you need them

% define commands here
%\PMlinkescapeword{theory}
\begin{document}
Suppose $f$ is a $\kappa$-normal function and consider any $\alpha<\kappa$ and define a sequence by $\alpha_0=\alpha$ and $\alpha_{n+1}=f(\alpha_n)$.  Let $\alpha_\omega=\sup_{n<\omega}\alpha_n$.  Then, since $f$ is continuous, $$f(\alpha_\omega)=\sup_{n<\omega} f(\alpha_n)=\sup_{n<\omega}\alpha_{n+1}=\alpha_\omega$$
So $\operatorname{Fix}(f)$ is unbounded.

Suppose $N$ is a set of fixed points of $f$ with $|N|<\kappa$.  Then $$f(\sup N)=\sup_{\alpha\in N} f(\alpha)=\sup_{\alpha\in N}\alpha=\sup N$$
so $\sup N$ is also a fixed point of $f$, and therefore $\operatorname{Fix}(f)$ is closed.
%%%%%
%%%%%
\end{document}
