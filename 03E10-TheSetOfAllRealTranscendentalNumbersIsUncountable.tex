\documentclass[12pt]{article}
\usepackage{pmmeta}
\pmcanonicalname{TheSetOfAllRealTranscendentalNumbersIsUncountable}
\pmcreated{2013-03-22 16:08:05}
\pmmodified{2013-03-22 16:08:05}
\pmowner{gilbert_51126}{14238}
\pmmodifier{gilbert_51126}{14238}
\pmtitle{the set of all real transcendental numbers is uncountable}
\pmrecord{11}{38206}
\pmprivacy{1}
\pmauthor{gilbert_51126}{14238}
\pmtype{Theorem}
\pmcomment{trigger rebuild}
\pmclassification{msc}{03E10}

\endmetadata

% this is the default PlanetMath preamble.  as your knowledge
% of TeX increases, you will probably want to edit this, but
% it should be fine as is for beginners.

% almost certainly you want these
\usepackage{amssymb}
\usepackage{amsmath}
\usepackage{amsfonts}

% used for TeXing text within eps files
%\usepackage{psfrag}
% need this for including graphics (\includegraphics)
%\usepackage{graphicx}
% for neatly defining theorems and propositions
%\usepackage{amsthm}
% making logically defined graphics
%%%\usepackage{xypic}

% there are many more packages, add them here as you need them

% define commands here

\begin{document}
\emph{Theorem.}The set of all real transcendental numbers is uncountable.

\emph{Proof.}
Denote $\mathbb{T}$ and $\mathbb{A}$ be the set of real transcendental and real algebraic numbers respectively. Suppose $\mathbb{T}$ is countable. Then the union $\mathbb{T} \cup \mathbb{A} = \mathbb{R}$ is also countable, since $\mathbb{A}$ is also countable, which is a contradiction. Therefore $\mathbb{T}$ must be uncountable. $\Box$ 
%%%%%
%%%%%
\end{document}
