\documentclass[12pt]{article}
\usepackage{pmmeta}
\pmcanonicalname{CardinalArithmetic}
\pmcreated{2013-03-22 14:15:13}
\pmmodified{2013-03-22 14:15:13}
\pmowner{yark}{2760}
\pmmodifier{yark}{2760}
\pmtitle{cardinal arithmetic}
\pmrecord{38}{35701}
\pmprivacy{1}
\pmauthor{yark}{2760}
\pmtype{Topic}
\pmcomment{trigger rebuild}
\pmclassification{msc}{03E10}
\pmrelated{OrdinalArithmetic}
\pmrelated{CardinalNumber}
\pmrelated{CardinalExponentiationUnderGCH}
\pmrelated{CardinalityOfTheContinuum}
\pmdefines{cardinal addition}
\pmdefines{cardinal multiplication}
\pmdefines{cardinal exponentiation}
\pmdefines{sum of cardinals}
\pmdefines{product of cardinals}
\pmdefines{addition}
\pmdefines{multiplication}
\pmdefines{exponentiation}
\pmdefines{sum}
\pmdefines{product}

\usepackage{amsmath}

\begin{document}
\PMlinkescapeword{algebraic}
\PMlinkescapeword{arithmetic}
\PMlinkescapeword{index}
\PMlinkescapeword{inequalities}
\PMlinkescapeword{inequality}
\PMlinkescapeword{product}
\PMlinkescapeword{products}
\PMlinkescapeword{properties}
\PMlinkescapeword{similar}
\PMlinkescapeword{sum}
\PMlinkescapeword{sums}
\PMlinkescapeword{terms}

\section*{Definitions}

Let $\kappa$ and $\lambda$ be cardinal numbers, 
and let $A$ and $B$ be disjoint sets such that $|A|=\kappa$ and $|B|=\lambda$.
(Here $|X|$ denotes the cardinality of a set $X$, 
that is, the unique cardinal number equinumerous with $X$.)
Then we define cardinal addition, cardinal multiplication 
and cardinal exponentiation as follows.
\begin{align*}
\kappa+\lambda&=|A\cup B|. \\
\kappa\lambda&=|A\times B|. \\
\kappa^\lambda&=|A^B|.
\end{align*}
(Here $A^B$ denotes the set of all functions from $B$ to $A$.)
These three operations are well-defined, that is, 
they do not depend on the choice of $A$ and $B$.
Also note that for multiplication and exponentiation $A$ 
and $B$ do not actually need to be disjoint.

We also define addition and multiplication for arbitrary numbers of cardinals.
Suppose $I$ is an index set and $\kappa_i$ is a cardinal for every $i\in I$.
Then $\sum_{i\in I}\kappa_i$ is defined to be 
the cardinality of the union $\bigcup_{i\in I}A_i$,
where the $A_i$ are pairwise disjoint and $|A_i|=\kappa_i$ for each $i\in I$.
Similarly, $\prod_{i\in I}\kappa_i$ is defined to be the cardinality of the 
\PMlinkname{Cartesian product}{GeneralizedCartesianProduct} 
$\prod_{i\in I}B_i$, where $|B_i|=\kappa_i$ for each $i\in I$.

\section*{Properties}

In the following, $\kappa$, $\lambda$, $\mu$ and $\nu$ are arbitrary cardinals,
unless otherwise specified.

Cardinal arithmetic obeys many of the same algebraic laws as real arithmetic.
In particular, the following properties hold.
\begin{align*}
\kappa+\lambda&=\lambda+\kappa.\\
(\kappa+\lambda)+\mu&=\kappa+(\lambda+\mu).\\
\kappa\lambda&=\lambda\kappa.\\
(\kappa\lambda)\mu&=\kappa(\lambda\mu).\\
\kappa(\lambda+\mu)&=\kappa\lambda+\kappa\mu.\\
\kappa^\lambda\kappa^\mu&=\kappa^{\lambda+\mu}.\\
(\kappa^\lambda)^\mu&=\kappa^{\lambda\mu}.\\
\kappa^\mu\lambda^\mu&=(\kappa\lambda)^\mu.
\end{align*}

Some special cases involving $0$ and $1$ are as follows:
\begin{align*}
\kappa+0&=\kappa.\\
0\kappa&=0.\\
\kappa^0&=1.\\
0^\kappa&=0, \text{ for } \kappa>0.\\
1\kappa&=\kappa.\\
\kappa^1&=\kappa.\\
1^\kappa&=1.
\end{align*}

If at least one of $\kappa$ and $\lambda$ is infinite, then the following hold.
\begin{align*}
\kappa+\lambda&=\max(\kappa,\lambda).\\
\kappa\lambda&=\max(\kappa,\lambda), \text{ provided } \kappa\ne0\ne\lambda.
\end{align*}

Also notable is that if $\kappa$ and $\lambda$ are cardinals
with $\lambda$ infinite and $2 \le \kappa \le 2^\lambda$,
then
\begin{align*}
\kappa^\lambda&=2^\lambda.
\end{align*}

Inequalities are also important in cardinal arithmetic.
The most famous is Cantor's theorem
\begin{align*}
\kappa&<2^\kappa.
\end{align*}

If $\mu\le\kappa$ and $\nu\le\lambda$, then
\begin{align*}
\mu+\nu&\le\kappa+\lambda.\\
\mu\nu&\le\kappa\lambda.\\
\mu^\nu&\le\kappa^\lambda, \text{ unless } \mu=\nu=\kappa=0<\lambda.
\end{align*}

Similar inequalities hold for infinite sums and products. 
Let $I$ be an index set, 
and suppose that $\kappa_i$ and $\lambda_i$ are cardinals for every $i\in I$.
If $\kappa_i\le\lambda_i$ for every $i\in I$, then 
\begin{align*}
\sum_{i\in I}\kappa_i&\le\sum_{i\in I}\lambda_i.\\
\prod_{i\in I}\kappa_i&\le\prod_{i\in I}\lambda_i.
\end{align*}
If, moreover, $\kappa_i<\lambda_i$ for all $i\in I$, 
then we have K\"onig's theorem.
\begin{align*}
\sum_{i\in I}\kappa_i&<\,\prod_{i\in I}\lambda_i.
\end{align*}

If $\kappa_i=\kappa$ for every $i$ in the index set $I$, then
\begin{align*}
\sum_{i\in I}\kappa_i&=\kappa|I|.\\
\prod_{i\in I}\kappa_i&=\kappa^{|I|}.
\end{align*}
Thus it is possible to define exponentiation in terms of multiplication,
and multiplication in terms of addition.
%%%%%
%%%%%
\end{document}
