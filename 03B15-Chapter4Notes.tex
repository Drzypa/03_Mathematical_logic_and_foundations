\documentclass[12pt]{article}
\usepackage{pmmeta}
\pmcanonicalname{Chapter4Notes}
\pmcreated{2013-11-18 1:07:49}
\pmmodified{2013-11-18 1:07:49}
\pmowner{PMBookProject}{1000683}
\pmmodifier{rspuzio}{6075}
\pmtitle{Chapter 4 Notes}
\pmrecord{2}{87672}
\pmprivacy{1}
\pmauthor{PMBookProject}{6075}
\pmtype{Feature}
\pmclassification{msc}{03B15}

\usepackage{xspace}
\usepackage{amssyb}
\usepackage{amsmath}
\usepackage{amsfonts}
\usepackage{amsthm}
\makeatletter
\newcommand{\brck}[1]{\trunc{}{#1}}
\def\@dprd#1{\prod_{(#1)}\,}
\def\@dprd@noparens#1{\prod_{#1}\,}
\def\@dsm#1{\sum_{(#1)}\,}
\def\@dsm@noparens#1{\sum_{#1}\,}
\def\@eatprd\prd{\prd@parens}
\def\@eatsm\sm{\sm@parens}
\newcommand{\eqv}[2]{\ensuremath{#1 \simeq #2}\xspace}
\newcommand{\iscontr}{\ensuremath{\mathsf{isContr}}}
\def\prd#1{\@ifnextchar\bgroup{\prd@parens{#1}}{\@ifnextchar\sm{\prd@parens{#1}\@eatsm}{\prd@noparens{#1}}}}
\def\prd@noparens#1{\mathchoice{\@dprd@noparens{#1}}{\@tprd{#1}}{\@tprd{#1}}{\@tprd{#1}}}
\def\prd@parens#1{\@ifnextchar\bgroup  {\mathchoice{\@dprd{#1}}{\@tprd{#1}}{\@tprd{#1}}{\@tprd{#1}}\prd@parens}  {\@ifnextchar\sm    {\mathchoice{\@dprd{#1}}{\@tprd{#1}}{\@tprd{#1}}{\@tprd{#1}}\@eatsm}    {\mathchoice{\@dprd{#1}}{\@tprd{#1}}{\@tprd{#1}}{\@tprd{#1}}}}}
\newcommand{\qinv}{\ensuremath{\mathsf{qinv}}}
\newcommand{\sectionNotes}{\phantomsection\section*{Notes}\addcontentsline{toc}{section}{Notes}\markright{\textsc{\@chapapp{} \thechapter{} Notes}}}
\def\sm#1{\@ifnextchar\bgroup{\sm@parens{#1}}{\@ifnextchar\prd{\sm@parens{#1}\@eatprd}{\sm@noparens{#1}}}}
\def\sm@noparens#1{\mathchoice{\@dsm@noparens{#1}}{\@tsm{#1}}{\@tsm{#1}}{\@tsm{#1}}}
\def\sm@parens#1{\@ifnextchar\bgroup  {\mathchoice{\@dsm{#1}}{\@tsm{#1}}{\@tsm{#1}}{\@tsm{#1}}\sm@parens}  {\@ifnextchar\prd    {\mathchoice{\@dsm{#1}}{\@tsm{#1}}{\@tsm{#1}}{\@tsm{#1}}\@eatprd}    {\mathchoice{\@dsm{#1}}{\@tsm{#1}}{\@tsm{#1}}{\@tsm{#1}}}}}
\def\@tprd#1{\mathchoice{{\textstyle\prod_{(#1)}}}{\prod_{(#1)}}{\prod_{(#1)}}{\prod_{(#1)}}}
\newcommand{\trunc}[2]{\mathopen{}\left\Vert #2\right\Vert_{#1}\mathclose{}}
\def\@tsm#1{\mathchoice{{\textstyle\sum_{(#1)}}}{\sum_{(#1)}}{\sum_{(#1)}}{\sum_{(#1)}}}
\let\autoref\cref
\makeatother

\begin{document}
The fact that the space of continuous maps equipped with quasi-inverses has the wrong homotopy type to be the ``space of homotopy equivalences'' is well-known in algebraic topology.
In that context, the ``space of homotopy equivalences'' $(\eqv AB)$ is usually defined simply as the subspace of the function space $(A\to B)$ consisting of the functions that are homotopy equivalences.
In type theory, this would correspond most closely to $\sm{f:A\to B} \brck{\qinv(f)}$; see \PMlinkexternal{Exercise 3.11}{http://planetmath.org/node/87824}.

The first definition of equivalence given in homotopy type theory was the one that we have called $\iscontr(f)$, which was due to Voevodsky.
The possibility of the other definitions was subsequently observed by various people.
The basic theorems about adjoint equivalences\index{adjoint!equivalence} such as \PMlinkname{Lemma 4.2.2}{42halfadjointequivalences#Thmprelem1},\PMlinkname{Theorem 4.2.3}{42halfadjointequivalences#Thmprethm1} are adaptations of standard facts in higher category theory and homotopy theory.
Using bi-invertibility as a definition of equivalences was suggested by Andr\'e Joyal.

The properties of equivalences discussed in \PMlinkname{\S 4.6}{46surjectionsandembeddings},\PMlinkname{\S 4.7}{47closurepropertiesofequivalences} are well-known in homotopy theory.
Most of them were first proven in type theory by Voevodsky.

The fact that every function is equivalent to a fibration is a standard fact in homotopy theory.
The notion of object classifier
\index{object!classifier}%
\index{classifier!object}%
in $(\infty,1)$-category
\index{.infinity1-category@$(\infty,1)$-category}%
theory (the categorical analogue of \PMlinkname{Theorem 4.8.3}{48theobjectclassifier#Thmprethm1}) is due to Rezk (see~\cite{Rezk05},\cite{lurie:higher-topoi}).

Finally, the fact that univalence implies function extensionality (\PMlinkname{\S 4.9}{49univalenceimpliesfunctionextensionality}) is due to Voevodsky.
Our proof is a simplification of his.

\begin{thebibliography}{99}

\bibitem{lurie:higher-topoi} {Jacob Lurie}, \emph{{Higher topos theory}}  {Princeton University Press},{2009}

\bibitem{Rezk05} {Charles Rezk}. {Toposes and homotopy toposes}. {\url{http://www.math.uiuc.edu/~rezk/homotopy-topos-sketch.pdf}}, 2005.

\end{thebibliography}

\end{document}
