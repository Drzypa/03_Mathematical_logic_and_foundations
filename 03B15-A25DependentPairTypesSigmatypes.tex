\documentclass[12pt]{article}
\usepackage{pmmeta}
\pmcanonicalname{A25DependentPairTypesSigmatypes}
\pmcreated{2013-11-09 5:31:58}
\pmmodified{2013-11-09 5:31:58}
\pmowner{PMBookProject}{1000683}
\pmmodifier{PMBookProject}{1000683}
\pmtitle{A.2.5 Dependent pair types ($Sigma$-types)}
\pmrecord{1}{}
\pmprivacy{1}
\pmauthor{PMBookProject}{1000683}
\pmtype{Feature}
\pmclassification{msc}{03B15}

\endmetadata

\usepackage{xspace}
\usepackage{amssyb}
\usepackage{amsmath}
\usepackage{amsfonts}
\usepackage{amsthm}
\makeatletter
\newcommand{\comp}{\textsc{comp}}
\newcommand{\defeq}{\vcentcolon\equiv}  
\newcommand{\elim}{\textsc{elim}}
\newcommand{\form}{\textsc{form}}
\newcommand{\ind}[1]{\mathsf{ind}_{#1}}
\newcommand{\intro}{\textsc{intro}}
\newcommand{\jdeq}{\equiv}      
\newcommand{\jdeqtp}[4]{#1 \vdash #2 \jdeq #3 : #4}
\newcommand{\oftp}[3]{#1 \vdash #2 : #3}
\def\prdsym{\textstyle\prod}
\def\smsym{\textstyle\sum}
\newcommand{\tmtp}[2]{#1 \mathord{:} #2}
\def\tsm#1{\@tsm{#1}\@ifnextchar\bgroup{\tsm}{}}
\def\@tsm#1{\mathchoice{{\textstyle\sum_{(#1)}}}{\sum_{(#1)}}{\sum_{(#1)}}{\sum_{(#1)}}}
\newcommand{\tup}[2]{(#1,#2)}
\newcommand{\UU}{\ensuremath{\mathcal{U}}\xspace}
\newcommand{\vcentcolon}{:\!\!}
\let\autoref\cref
\makeatother

\begin{document}

\index{type!dependent pair}%
\index{type!product}%

In \PMlinkname{\S 1.6}{16dependentpairtypes}, we needed $\to$ and $\prdsym$ types in order to
define the introduction and elimination rules for $\smsym$; as with $\prdsym$, contexts allow us to state the rules for $\smsym$ independently:
%
\begin{mathparpagebreakable}
  \def\premise{\oftp{\Gamma}{A}{\UU_i} \and \oftp{\Gamma,\tmtp xA}{B}{\UU_i}}
  \inferrule*[right=$\Sigma$-\form]
    \premise
    {\oftp\Gamma{\tsm{x:A} B}{\UU_i}}
  \and
  \inferrule*[right=$\Sigma$-\intro]
    {\oftp{\Gamma, \tmtp x A}{B}{\UU_i} \\
     \oftp\Gamma{a}{A} \\ \oftp\Gamma{b}{B[a/x]}}
    {\oftp\Gamma{\tup ab}{\tsm{x:A} B}}
  \and
  \inferrule*[right=$\Sigma$-\elim]
    {\oftp{\Gamma, \tmtp z {\tsm{x:A} B}}{C}{\UU_i} \\
     \oftp{\Gamma,\tmtp x A,\tmtp y B}{g}{C[\tup x y/z]} \\
     \oftp\Gamma{p}{\tsm{x:A} B}}
    {\oftp\Gamma{\ind{\tsm{x:A} B}(z.C,x.y.g,p)}{C[p/z]}}
  \and
  \inferrule*[right=$\Sigma$-\comp]
    {\oftp{\Gamma, \tmtp z {\tsm{x:A} B}}{C}{\UU_i} \\
     \oftp{\Gamma, \tmtp x A, \tmtp y B}{g}{C[\tup x y/z]} \\\\
     \oftp\Gamma{a'}{A} \\ \oftp\Gamma{b'}{B[a'/x]}}
    {\jdeqtp\Gamma{\ind{\tsm{x:A} B}(z.C,x.y.g,\tup{a'}{b'})}{g[a',b'/x,y]}{C[\tup {a'} {b'}/z]}}
\end{mathparpagebreakable}
%
The expression $\tsm{x:A} B$ binds free occurrences of $x$ in $B$. Furthermore, because
$\ind{\tsm{x:A} B}$ has some arguments with free variables beyond those in $\Gamma$,
we bind (following the variable names above) $z$ in $C$, and $x$ and $y$ in $g$.
These bindings are written as $z.C$ and $x.y.g$, to indicate the names of the bound
variables.
\index{variable!bound}%
In particular, we treat $\ind{\tsm{x:A} B}$ as a primitive,
two of whose arguments contain binders; this is superficially similar to, but
different from, $\ind{\tsm{x:A} B}$ being a function that takes functions as
arguments.

When $B$ does not contain free occurrences of $x$, we obtain as a special case
the cartesian product $A \times B \defeq \tsm{x:A} B$. We take this
as the \emph{definition} of the cartesian product.

Notice that we don't postulate a judgmental uniqueness principle for $\Sigma$-types, even
though we could have; see PMlinkname{Corollary 1}{27sigmatypes#Thmcor1} for a proof of the corresponding
propositional uniqueness principle.


\end{document}
