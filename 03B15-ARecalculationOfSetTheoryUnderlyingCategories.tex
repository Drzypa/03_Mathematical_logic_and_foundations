\documentclass[12pt]{article}
\usepackage{pmmeta}
\pmcanonicalname{ARecalculationOfSetTheoryUnderlyingCategories}
\pmcreated{2013-11-04 21:38:12}
\pmmodified{2013-11-04 21:38:12}
\pmowner{joseph120206}{1000115}
\pmmodifier{joseph120206}{1000115}
\pmtitle{A recalculation of set theory underlying Categories. }
\pmrecord{1}{}
\pmprivacy{1}
\pmauthor{joseph120206}{1000115}
\pmtype{Definition}

\endmetadata

% this is the default PlanetMath preamble.  as your knowledge
% of TeX increases, you will probably want to edit this, but
% it should be fine as is for beginners.

% almost certainly you want these
\usepackage{amssymb}
\usepackage{amsmath}
\usepackage{amsfonts}

% need this for including graphics (\includegraphics)
\usepackage{graphicx}
% for neatly defining theorems and propositions
\usepackage{amsthm}

% making logically defined graphics
%\usepackage{xypic}
% used for TeXing text within eps files
%\usepackage{psfrag}

% there are many more packages, add them here as you need them

% define commands here

\begin{document}
 \documentclass{article}
 \usepackage{amsmath}
 \usepackage{amsfonts}
 \usepackage{amssymb}
\usepackage{amsthm}
 \usepackage{graphicx}
 \newtheorem*{mydef}{Definition}
 \newtheorem{thm}{Theorem}[section]
 \newtheorem{cor}{Corollary}[section]
 \newtheorem{lem}{Lemma}[section]
 \author{oorahb120206}
\today
\title{Continuum of Mathematics}
\maketitle
\section{Sets and Propositions}\\
We will assume the simple tool of a arrow, and the intuition that our mind can associate one object of our imagination with a imagined slight; surely this is not to much of a stretch upon which to build a theory. 
A proposition \textbf{P} is a statement, a propositional function \textbf{P($\mathbf{x}$)} is a statement equipped with a variable $\mathbf{x}$ allowed to vary over a domain of definition (a intuitive notion of those objects we wish to prove in contrast to our proposition.) of our choice. A arrow $\longrightarrow$ is a association of a object $\mathbf{a }$ (called the source of the arrow) with a logic value $\langle \textbf{T}, \textbf{F} \rangle$,(called the target of the arrow) such that \textbf{T} is true and \textbf{F} is false.  For a object $\mathbf{a}$ of the domain of definition we call $\mathbf{P(a)}$ the image of $\mathbf{a}$ under \textbf{P}, and write $\mathbf{P(a)}=\mathbf{T}$ or $ \mathbf{P(a)}=\mathbf{F}$.  The satisfactory data that is generated by the propositional function is called the Range of \textbf{P}. The Range is partitioned into two $ur-sets$ (a collection of logical values with respect to some proposition). We denote the two ur-sets as  \textbf{T(x)} the collection of all truth values, and \textbf{F(x)} as the collection of all false values. We denote the preimage of the ur-sets as $\mathbf{T^{-1}(x)}$ and $\mathbf{F^{-1}(x)}$ and call them sets. 
\begin{mydef}{Set}\\
Let \textbf{S} be a set, \textbf{P} be a proposition, and $\mathbf{a}$ an object, then it follows, 
$$\mathbf{S}= \{\mathbf{a}:\mathbf{P(a)}\}$$
this is read as, the set \textbf{S} equals the collection of those $\mathbf{a}$ such that $\mathbf{P(a)}=\mathbf{T}$ We call the set $\mathbf{F^{-1}(x)}$ the relative complement of \textbf{S}
denoted $\lnot \mathbf{S}$, such that 
$$\lnot \mathbf{S}=\{\mathbf{a}:\lnot \mathbf{P(a)}\}$$
\end{mydef} 
In particular $\lnot \mathbf{S}= \mathbf{F^{-1}(x)}$. Let \textbf{U} denote the domain of definition, then we write \textbf{P}: \textbf{U} $\longrightarrow \langle \textbf{T}, \textbf{F} \rangle$. This brings us to our first axiom. 
\begin{mydef}{Axiom of Extenionality}\\
Two sets are equal if and only it they are compiled of the same data. \footnote{I refrain from using the word contain since this in my mind suggest that the elements of a set are allocated to a specific space, and this is not the definition that I wish to portray. A set is a symbol denoting a satisfaction, that is the objects satisfying some proposition, even though the objects may be scattered through our space. }
\end{mydef}
If we allow the variable in the propositional function to vary over sets then the ur-set $\mathbf{T^{-1}(x)}$ is called a Class. And from this we may derive all the tools we need to accomplish what is called mathematics using the objects in Category Theory. There is the matter of the $\in$ relation, to use this notation as a suggestive argument that the object $\mathbf{a}$ is associated with a truth logic value, is not a improper abuse of notation I would suppose. So if we wrote $\mathbf{a} \in \mathbf{S}$ to suggest that $\mathbf{P(a)}=\mathbf{T}$ this too is not a improper abuse of notation I would suppose.  
\end{document}
\end{document}
