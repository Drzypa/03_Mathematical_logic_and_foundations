\documentclass[12pt]{article}
\usepackage{pmmeta}
\pmcanonicalname{TruthvalueSemanticsForClassicalPropositionalLogic}
\pmcreated{2013-03-22 18:51:20}
\pmmodified{2013-03-22 18:51:20}
\pmowner{CWoo}{3771}
\pmmodifier{CWoo}{3771}
\pmtitle{truth-value semantics for classical propositional logic}
\pmrecord{20}{41665}
\pmprivacy{1}
\pmauthor{CWoo}{3771}
\pmtype{Definition}
\pmcomment{trigger rebuild}
\pmclassification{msc}{03B05}
\pmsynonym{entail}{TruthvalueSemanticsForClassicalPropositionalLogic}
\pmdefines{truth-value semantics}
\pmdefines{valuation}
\pmdefines{interpretation}
\pmdefines{valid}
\pmdefines{invalid}
\pmdefines{satisfiable}

\usepackage{amssymb,amscd}
\usepackage{amsmath}
\usepackage{amsfonts}
\usepackage{mathrsfs}

% used for TeXing text within eps files
%\usepackage{psfrag}
% need this for including graphics (\includegraphics)
%\usepackage{graphicx}
% for neatly defining theorems and propositions
\usepackage{amsthm}
% making logically defined graphics
%%\usepackage{xypic}
\usepackage{pst-plot}

% define commands here
\newcommand*{\abs}[1]{\left\lvert #1\right\rvert}
\newtheorem{prop}{Proposition}
\newtheorem{thm}{Theorem}
\newtheorem{ex}{Example}
\newcommand{\real}{\mathbb{R}}
\newcommand{\pdiff}[2]{\frac{\partial #1}{\partial #2}}
\newcommand{\mpdiff}[3]{\frac{\partial^#1 #2}{\partial #3^#1}}

\begin{document}
In classical propositional logic, an \emph{interpretation} of a well-formed formula (wff) $p$ is an assignment of truth (=1) or falsity (=0) to $p$.  Any interpreted wff is called a proposition.  

An \emph{interpretation} of all wffs over the variable set $V$ is then a Boolean function on $\overline{V}$.  However, one needs to be careful, for we do not want both $p$ and $\neg p$ be interpreted as true simultaneously (at least not in classical propositional logic)!  The proper way to find an interpretation on the wffs is to start from the atoms.

Call any Boolean-valued function $\nu$ on $V$ a \emph{valuation} on $V$.  We want to extend $\nu$ to a Boolean-valued function $\overline{\nu}$ on $\overline{V}$ of all wffs.  The way this is done is similar to the construction of wffs; we build a sequence of functions, starting from $\nu$ on $V_0$, next $\nu_1$ on $V_1$, and so on... Finally, we take the union of all these ``approximations'' to arrive at $\overline{\nu}$.  So how do we go from $\nu$ to $\nu_1$?  We need to interpret $\neg p$ and $p\vee q$ from the valuations of atoms $p$ and $q$.  In other words, we must also interpret logical connectives too.

Before doing this, we define a truth function for each of the logical connectives:
\begin{itemize}
\item for $\neg$, define $f:\lbrace 0,1\rbrace \to \lbrace 0,1\rbrace$ given by $f(x)=1-x$.
\item for $\vee$, define $g:\lbrace 0,1\rbrace^2 \to \lbrace 0,1\rbrace$ given by $g(x,y)=\max(x,y)$.
\end{itemize}
As we are trying to \emph{interpret} $\neg$ (not) and $\vee$ (or), the choices for $f$ and $g$ are clear.  The values $0,1$ are interpreted as the usual integers (so they can be subtracted and ordered, etc...).  Hence $f$ and $g$ make sense.

Next, recall that $V_i$ are sets of wffs built up from wffs in $V_{i-1}$ (see construction of well-formed formulas for more detail).  We define a function $\nu_i: V_i\to \lbrace 0,1\rbrace$ for each $i$, as follows: 
\begin{itemize}
\item $\nu_0:=\nu$
\item suppose $\nu_i$ has been defined, we define $\nu_{i+1}:V_{i+1} \to \lbrace 0,1\rbrace$ given by 
\begin{displaymath}
\nu_{i+1}(p):= \left\{
\begin{array}{ll}
\nu_i(p) & \mbox{if } p\in V_i,\\
f(\nu_i(q)) & \mbox{if } p = \neg q\mbox{ for some }q\in V_i, \\
g(\nu_i(q),\nu_i(r)) & \mbox{if } p = q\vee r \mbox{ for some }q,r\in V_i.
\end{array}
\right.
\end{displaymath}
\end{itemize}

Finally, take $\overline{\nu}$ to be the union of all the approximations: $$\overline{\nu}:=\bigcup_{i=0}^{\infty} \nu_i.$$  Then, by unique readability of wffs, $\overline{\nu}$ is an interpretation on $\overline{V}$.

\textbf{Remark}.  If $\perp$ is included in the language of the logic (as the symbol for falsity), we also require that $\nu_i(\perp)=\overline{\nu}(\perp)=0$.

\textbf{Remark} $\overline{V}$ can be viewed as an inductive set over $V$ with respect to the $\neg$ and $\vee$, viewed as operations on $\overline{V}$.  Furthermore, $\overline{V}$ is freely generated by $V$, since each $V_{i+1}$ can be partitioned into sets $V_i$,  $\lbrace (p \vee q) \mid p,q \in V_i\rbrace$, and $\lbrace (\neg p) \mid p\in V_i\rbrace$, and each partition is non-empty.  As a result, any valuation $\nu$ on $V$ uniquely extends to a valuation $\overline{\nu}$ on $\overline{V}$.

\textbf{Definitions}.  Let $p,q$ be wffs in $\overline{V}$.
\begin{itemize}
\item $p$ is \emph{true} or \emph{satisfiable} for some valuation $\nu$ if $\overline{\nu}(p)=1$ (otherwise, it is \emph{false} for $\nu$).
\item $p$ is true for every valuation $\nu$, then $p$ is said to be \emph{valid} (or \emph{tautologous}).  If $p$ is false for every $\nu$, it is \emph{invalid}.  If $p$ is valid, we write $\models p$.
\item $p$ implies $q$ for a valuation $\nu$ if $\overline{\nu}(p)=1$ implies $\overline{\nu}(q)=1$.  $p$ \emph{semantically implies} if $p$ implies $q$ for every valuation $\nu$, and is denoted by $p \models q$.
\item $p$ is \emph{equivalent} to $q$ for $\nu$ if $\overline{\nu}(p)=\overline{\nu}(q)$.  They are \emph{semantically equivalent} if they are equivalent for every $\nu$, and written $p\equiv q$.
\end{itemize}

Semantical equivalence is an equivalence relation on $\overline{V}$.

The above can be easily generalized to sets of wffs.  Let $T$ be a set of propositions.
\begin{itemize}
\item $T$ is true or satisfiable for $\nu$ if $\overline{\nu}(T)=\lbrace 1\rbrace$ (otherwise, it is false for $\nu$).
\item $T$ is valid if it is true for every $\nu$; it is \emph{invalid} if it is false for every $\nu$.  If $T$ is valid, we write $\models T$.
\item $T$ implies $p$ for $\nu$ if $\overline{\nu}(T)=\lbrace 1\rbrace$ implies $\overline{\nu}(p)=1$.  $T$ semantically implies $p$ if $T$ implies $p$ for every $\nu$, and is denoted by $T\models p$.  
\item $T_1$ implies $T_2$ for $\nu$ if, for every $p\in T_2$, $T_1$ implies $p$ for $\nu$.  $T_1$ semantically implies $T_2$ if $T_1$ implies $T_2$ for every $\nu$, and is denoted by $T_1 \models T_2$.
\item $T_1$ is equivalent to $T_2$ for $\nu$ if for some valuation $\nu$, $T_1$ implies $T_2$ for $\nu$ and $T_2$ implies $T_1$ for $\nu$.  $T_1$ and $T_2$ are semantically equivalent if $T_1 \models T_2$ and $T_2 \models T_1$, written $T_1 \equiv T_2$.
\end{itemize}

Clearly, $\models p$ iff $\varnothing \models p$, and $T\models p$ iff $T\models \lbrace p\rbrace$.

%%%%%
%%%%%
\end{document}
