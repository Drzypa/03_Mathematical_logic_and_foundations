\documentclass[12pt]{article}
\usepackage{pmmeta}
\pmcanonicalname{MaximalityPrinciple}
\pmcreated{2013-03-22 12:26:18}
\pmmodified{2013-03-22 12:26:18}
\pmowner{akrowne}{2}
\pmmodifier{akrowne}{2}
\pmtitle{maximality principle}
\pmrecord{9}{32533}
\pmprivacy{1}
\pmauthor{akrowne}{2}
\pmtype{Theorem}
\pmcomment{trigger rebuild}
\pmclassification{msc}{03E30}
\pmclassification{msc}{03E25}
\pmsynonym{maximal principle}{MaximalityPrinciple}
\pmrelated{ZornsLemma}
\pmrelated{AxiomOfChoice}
\pmrelated{WellOrderingPrinciple}
\pmrelated{TukeysLemma}
\pmrelated{ZermelosPostulate}
\pmrelated{HaudorffsMaximumPrinciple}

\endmetadata

\usepackage{amssymb}
\usepackage{amsmath}
\usepackage{amsfonts}

%\usepackage{psfrag}
%\usepackage{graphicx}
%%%\usepackage{xypic}
\begin{document}
Let $S$ be a collection of sets.  If, for each chain $C \subseteq S$, there exists an $X \in S$ such that every element of $C$ is a subset of $X$, then $S$ contains a maximal element.  This is known as the \emph{maximality principle}.

The maximality principle is equivalent to the axiom of choice.
%%%%%
%%%%%
\end{document}
