\documentclass[12pt]{article}
\usepackage{pmmeta}
\pmcanonicalname{TruthvalueSemanticsForPropositionalLogicIsSound}
\pmcreated{2013-03-22 19:33:01}
\pmmodified{2013-03-22 19:33:01}
\pmowner{CWoo}{3771}
\pmmodifier{CWoo}{3771}
\pmtitle{truth-value semantics for propositional logic is sound}
\pmrecord{12}{42532}
\pmprivacy{1}
\pmauthor{CWoo}{3771}
\pmtype{Definition}
\pmcomment{trigger rebuild}
\pmclassification{msc}{03B05}

\endmetadata

\usepackage{amssymb,amscd}
\usepackage{amsmath}
\usepackage{amsfonts}
\usepackage{mathrsfs}

% used for TeXing text within eps files
%\usepackage{psfrag}
% need this for including graphics (\includegraphics)
%\usepackage{graphicx}
% for neatly defining theorems and propositions
\usepackage{amsthm}
% making logically defined graphics
%%\usepackage{xypic}
\usepackage{pst-plot}

% define commands here
\newcommand*{\abs}[1]{\left\lvert #1\right\rvert}
\newtheorem{prop}{Proposition}
\newtheorem{thm}{Theorem}
\newtheorem{ex}{Example}
\newcommand{\real}{\mathbb{R}}
\newcommand{\pdiff}[2]{\frac{\partial #1}{\partial #2}}
\newcommand{\mpdiff}[3]{\frac{\partial^#1 #2}{\partial #3^#1}}

\begin{document}
The soundness theorem of propositional logic says the following: every theorem is a tautology.  In symbol: $\vdash A$ implies $\models A$ for any wff $A$.

\begin{thm} Propositional logic is sound with respect to truth-value semantics. \end{thm}
\begin{proof}
Basically, we need to show that every axiom is a tautology, and that the inference rule modus ponens preserves truth.  Since theorems are deduced from axioms and by applications of modus ponens, they are tautologies as a result.

Using truth tables, one easily verifies that every axiom is true (under any valuation).

First, let us verify that $(A\to B)\to (\neg B\to \neg A)$ is a tautology.  The corresponding truth table is
\begin{center}
\begin{tabular}{ccccccc}
$A$ & $B$ & $\neg A$ & $\neg B$ & $A\to B$ & $\neg B\to \neg A$ & $(A\to B)\to (\neg B\to \neg A)$ \\
\hline 
T & T & F & F & T & T & T\\
T & F & F & T & F & F & T\\
F & T & T & F & T & T & T\\
F & F & T & T & T & T & T
\end{tabular}
\end{center}
Checking the truth values in the last column confirms that $(A\to B)\to (\neg B\to \neg A)$ is a tautology.

Next, let us check that $(A\to (B\to C))\to ((A\to B)\to (A\to C))$ is a tautology.  This time, we use a ``reduced'' truth table.

\begin{center}
\begin{tabular}{ccccccccccccc}
$(A$ & $\to$ & $(B$ & $\to$ & $C))$ & $\to$ & $((A$ & $\to$ & $B)$ & $\to$ & $(A$ & $\to$ & $C))$ \\
\hline 
T & T & T & T & T & T & T & T & T & T & T & T & T \\
T & F & T & F & F & T & T & T & T & F & T & F & F \\
T & T & F & T & T & T & T & F & F & T & T & T & T \\
T & T & F & T & F & T & T & F & F & T & T & F & F \\
F & T & T & T & T & T & F & T & T & T & F & T & T \\
F & T & T & F & F & T & F & T & T & T & F & T & F \\
F & T & F & T & T & T & F & T & F & T & F & T & T \\
F & T & F & T & F & T & F & T & F & T & F & T & F 
\end{tabular}
\end{center}
Notice that the truth values under the third $\to$ are all $T$, hence $(A\to (B\to C))\to ((A\to B)\to (A\to C))$ is a tautology.

Finally, we check that $A\to (B\to A)$ is a tautology.  This can be done without truth tables.  Let $v$ be a valuation.  We may assume $v(A)=1$, since $v(A\to (B\to A))=1$ otherwise.  If $v(A)=1$, then $v(B\to A)=1$ no matter what $v(B)$ is.  Therefore, $v(A\to (B\to A))=1$, and $A\to (B\to A)$ is a tautology.

Next, we show that modus ponens preserves truths.  In other words, $V(A)=V(A\to B)=1$ imply $V(B)=1$.  But if not, then either $V(A)=0$, or $V(A\to B)=0$.
\end{proof}
The soundness theorem can be used to prove that certain wff's of propositional logic are not theorems.  For example, we show that the schema $A\to (A\land B)$ is not a theorem schema (an instance of it is not a theorem).  Pick two distinct propositional variables $p$ and $q$, and use the truth table:

\begin{center}
\begin{tabular}{ccccc}
$A$ & $\to$ & $(A$ & $\land$ & $B)$\\
\hline 
T & T & T & T & T\\
T & F & T & F & F\\
F & T & F & F & T\\
F & T & F & F & F
\end{tabular}
\end{center}
Since the second column contains an $F$, $p\to (p\land  q)$ is not true, and therefore $\not \vdash A\to (A\land B)$ by the soundness theorem.  As another example, we show that the \emph{disjunction property}
\begin{center} if $\vdash A \lor B$, then $\vdash A$ or $\vdash B$ \end{center}
is not true in classical propositional logic (it is true, however, in intuitionistic logic).  To see this, let $A$ be $p\to q$ and $B$ be $q\to p$, where $p,q$ are propositional variables.  Then $A\lor B$ is an instance of the theorem schema $(C\to D)\lor (D\to C)$.  However, neither $\vdash A$ nor $\vdash B$, as illustrated in the following truth table:
\begin{center}
\begin{tabular}{ccccc}
$p$ & $q$ & $p\to q$ & $q\to p$ & $(p\to q)\lor (q\to p)$ \\
\hline 
T & T & T & T & T\\
T & F & F & T & T\\
F & T & T & F & T\\
F & F & T & T & T
\end{tabular}
\end{center}
Notice that both the third and the fourth columns contain an $F$, and therefore by the soundness theorem, $A$ and $B$ are not theorems.

%%%%%
%%%%%
\end{document}
