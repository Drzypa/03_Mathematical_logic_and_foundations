\documentclass[12pt]{article}
\usepackage{pmmeta}
\pmcanonicalname{ProofOfTheExistenceOfTranscendentalNumbers}
\pmcreated{2013-03-22 13:24:36}
\pmmodified{2013-03-22 13:24:36}
\pmowner{kidburla2003}{1480}
\pmmodifier{kidburla2003}{1480}
\pmtitle{proof of the existence of transcendental numbers}
\pmrecord{8}{33955}
\pmprivacy{1}
\pmauthor{kidburla2003}{1480}
\pmtype{Proof}
\pmcomment{trigger rebuild}
\pmclassification{msc}{03E10}
\pmrelated{AlgebraicNumbersAreCountable}

% this is the default PlanetMath preamble.  as your knowledge
% of TeX increases, you will probably want to edit this, but
% it should be fine as is for beginners.

% almost certainly you want these
\usepackage{amssymb}
\usepackage{amsmath}
\usepackage{amsfonts}

% used for TeXing text within eps files
%\usepackage{psfrag}
% need this for including graphics (\includegraphics)
%\usepackage{graphicx}
% for neatly defining theorems and propositions
%\usepackage{amsthm}
% making logically defined graphics
%%%\usepackage{xypic}

% there are many more packages, add them here as you need them

% define commands here
\begin{document}
Cantor discovered this proof.

\section*{Lemma:}

Consider a natural number $k$. Then the number of algebraic numbers of height $k$ is finite.

\subsection*{Proof:}

To see this, note the sum in the definition of height is positive. Therefore:

$
n \leq k
$

where $n$ is the degree of the polynomial. For a polynomial of degree $n$, there are only $n$ coefficients, and the sum of their moduli is $(k-n)$, and there is only a finite number of ways of doing this (the number of ways is the number of algebraic numbers). For every polynomial with degree less than $n$, there are less ways. So the sum of all of these is also finite, and this is the number of algebraic numbers with height $k$ (with some repetitions). The result follows.

\section*{Proof of the main theorem:}

You can start writing a list of the algebraic numbers because you can put all the ones with height 1, then with height 2, etc, and write them in numerical order within those sets because they are finite sets. This implies that the set of algebraic numbers is countable. However, by diagonalisation, the set of real numbers is uncountable. So there are more real numbers than algebraic numbers; the result follows.
%%%%%
%%%%%
\end{document}
