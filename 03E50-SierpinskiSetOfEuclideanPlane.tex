\documentclass[12pt]{article}
\usepackage{pmmeta}
\pmcanonicalname{SierpinskiSetOfEuclideanPlane}
\pmcreated{2013-05-18 23:13:46}
\pmmodified{2013-05-18 23:13:46}
\pmowner{pahio}{2872}
\pmmodifier{unlord}{1}
\pmtitle{Sierpi\'nski set of Euclidean plane}
\pmrecord{11}{42312}
\pmprivacy{1}
\pmauthor{pahio}{1}
\pmtype{Definition}
\pmcomment{trigger rebuild}
\pmclassification{msc}{03E50}
%\pmkeywords{continuum hypothesis}
\pmrelated{Countable}
\pmdefines{Sierpinski set}

\endmetadata

% this is the default PlanetMath preamble.  as your knowledge
% of TeX increases, you will probably want to edit this, but
% it should be fine as is for beginners.

% almost certainly you want these
\usepackage{amssymb}
\usepackage{amsmath}
\usepackage{amsfonts}

% used for TeXing text within eps files
%\usepackage{psfrag}
% need this for including graphics (\includegraphics)
%\usepackage{graphicx}
% for neatly defining theorems and propositions
 \usepackage{amsthm}
% making logically defined graphics
%%%\usepackage{xypic}

% there are many more packages, add them here as you need them

% define commands here

\theoremstyle{definition}
\newtheorem*{thmplain}{Theorem}

\begin{document}
A subset $S$ of $\mathbb{R}^2$ is called a \emph{Sierpi\'nski set} of the plane, if every line parallel to the $x$-axis intersects $S$ only in countably many points and every line parallel to the $y$-axis avoids $S$ in only countably many points:
$$\{x \in \mathbb{R}\,\vdots\;\, (x,\,y) \in S\} \,\mbox{ is countable for all }y \in \mathbb{R}$$
$$\{y \in \mathbb{R}\,\vdots\;\, (x,\,y) \notin S\} \,\mbox{ is countable for all }x \in \mathbb{R}$$

The existence of Sierpi\'nski sets is \PMlinkname{equivalent}{Equivalent3} with the continuum hypothesis, as is proved in [1].


\begin{thebibliography}{8}
\bibitem{gk}{\sc Gerald Kuba}:\, ``Wie plausibel ist die Kontinuumshypothese?''.\, --\emph{Elemente der Mathematik} \textbf{61} (2006).
\end{thebibliography}

%%%%%
%%%%%.
\end{document}
