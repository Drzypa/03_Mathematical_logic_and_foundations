\documentclass[12pt]{article}
\usepackage{pmmeta}
\pmcanonicalname{BooleanValuedModel}
\pmcreated{2013-03-22 12:51:08}
\pmmodified{2013-03-22 12:51:08}
\pmowner{Henry}{455}
\pmmodifier{Henry}{455}
\pmtitle{Boolean valued model}
\pmrecord{8}{33184}
\pmprivacy{1}
\pmauthor{Henry}{455}
\pmtype{Definition}
\pmcomment{trigger rebuild}
\pmclassification{msc}{03C90}
\pmclassification{msc}{03E40}
%\pmkeywords{model}
%\pmkeywords{Boolean valued}
%\pmkeywords{Boolean-valued}
\pmdefines{Boolean-valued model}

\endmetadata

% this is the default PlanetMath preamble.  as your knowledge
% of TeX increases, you will probably want to edit this, but
% it should be fine as is for beginners.

% almost certainly you want these
\usepackage{amssymb}
\usepackage{amsmath}
\usepackage{amsfonts}

% used for TeXing text within eps files
%\usepackage{psfrag}
% need this for including graphics (\includegraphics)
%\usepackage{graphicx}
% for neatly defining theorems and propositions
%\usepackage{amsthm}
% making logically defined graphics
%%%\usepackage{xypic}

% there are many more packages, add them here as you need them

% define commands here
\begin{document}
A traditional model of a language makes every formula of that language either true or false.  A \emph{Boolean valued model } is a generalization in which formulas take on any value in a Boolean algebra.

Specifically, a Boolean valued model of a signature $\Sigma$ over the language $\mathcal{L}$ is a set $\mathcal{A}$ together with a Boolean algebra $\mathcal{B}$.  Then the objects of the model are the functions $\mathcal{A}^\mathcal{B}=\mathcal{B}\rightarrow\mathcal{A}$.

For any formula $\phi$, we can assign a value $\lVert\phi\rVert$ from the Boolean algebra.  For example, if $\mathcal{L}$ is the language of first order logic, a typical recursive definition of $\lVert\phi\rVert$ might look something like this:
\begin{itemize}

\item $\lVert f=g \rVert=\bigvee_{f(b)=g(b)}b$

\item $\lVert\neg\phi\rVert=\lVert\phi\rVert^\prime$

\item $\lVert\phi\vee\psi\rVert=\lVert\phi\rVert\vee\lVert\psi\rVert$

\item $\lVert\exists x\phi(x)\rVert=\bigvee_{f\in\mathcal{A}^\mathcal{B}} \lVert\phi(f)\rVert$

\end{itemize}
%%%%%
%%%%%
\end{document}
