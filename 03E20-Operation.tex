\documentclass[12pt]{article}
\usepackage{pmmeta}
\pmcanonicalname{Operation}
\pmcreated{2013-03-22 14:57:23}
\pmmodified{2013-03-22 14:57:23}
\pmowner{rspuzio}{6075}
\pmmodifier{rspuzio}{6075}
\pmtitle{operation}
\pmrecord{10}{36653}
\pmprivacy{1}
\pmauthor{rspuzio}{6075}
\pmtype{Definition}
\pmcomment{trigger rebuild}
\pmclassification{msc}{03E20}
\pmrelated{Function}
\pmrelated{Mapping}
\pmrelated{Transformation}
\pmrelated{BinaryOperation}
\pmdefines{closed under}
\pmdefines{arithmetic operation}

\endmetadata

% this is the default PlanetMath preamble.  as your knowledge
% of TeX increases, you will probably want to edit this, but
% it should be fine as is for beginners.

% almost certainly you want these
\usepackage{amssymb}
\usepackage{amsmath}
\usepackage{amsfonts}

% used for TeXing text within eps files
%\usepackage{psfrag}
% need this for including graphics (\includegraphics)
%\usepackage{graphicx}
% for neatly defining theorems and propositions
%\usepackage{amsthm}
% making logically defined graphics
%%%\usepackage{xypic}

% there are many more packages, add them here as you need them

% define commands here
\newtheorem{dfn}{Definition}
\begin{document}
According to the dictionary \textbf{Webster's 1913}, which can be accessed through
\htmladdnormallink{HyperDictionary.com}{http://www.hyperdictionary.com/}, mathematical
meaning of the word \textit{operation} is: ``\textit{some \textbf{transformation} to be
made upon quantities}''. Thus, operation is similar to mapping or function. The most
general mathematical definition of operation can be made as follows:
%
\begin{dfn}
    \textbf{Operation} $\#$ defined on the sets $X_1,X_2,\ldots,X_n$ with values in $X$
    is a mapping from Cartesian product $X_1\times X_2\times
\cdots\times X_n$ to $X$, i.e.
    $$
        \# \colon X_1\times X_2\times\cdots\times X_n \longrightarrow X.
    $$
\end{dfn}
%
\noindent
Result of operation is usually denoted by one of the following notation:
\begin{itemize}
\item $x_1 \# x_2 \# \cdots \# x_n$
\item $\#(x_1,\ldots,x_n)$
\item $(x_1,\ldots,x_n)_\#$
\end{itemize}
%
The following examples show variety of the concept operation used in mathematics.

\vspace{0.5cm}
\noindent
\underline{\textbf{Examples}}
\begin{enumerate}
\item \textit{Arithmetic operations}: \PMlinkname{addition}{Addition}, subtraction, \PMlinkname{multiplication}{Multiplication}, division.
    Their generalization leads to the so-called binary operations, which is a basic concept
    for such algebraic structures as groups and rings.

\item Operations on vectors in the plane ($\mathbb{R}^2$).
    \begin{itemize}
    \item \textit{Multiplication by a scalar}. Generalization leads to vector spaces.
    \item \textit{Scalar product}. Generalization leads to Hilbert spaces.
    \end{itemize}

\item Operations on vectors in the space ($\mathbb{R}^3$).
    \begin{itemize}
    \item \textit{Cross product}. Can be generalized for the vector space of arbitrary
            finite dimension, see vector product in general vector spaces.
    \item \textit{Triple product}.
    \end{itemize}

\item Some operations on functions.
    \begin{itemize}
    \item \textit{Composition}.
    \item \textit{Function inverse}.
    \end{itemize}
\end{enumerate}
\vspace{0.5cm}

In the case when some of the sets $X_i$ are equal to the values set $X$, it is usually said
that operation is defined just on $X$. For such operations, it could be interesting
to consider their action on some subset $U\subset X$. In particular,
if operation on elements from $U$ always gives an element from $U$, it is said that $U$
is \textit{closed under} this operation. Formally it is expressed in the following definition.

\begin{dfn}
    Let operation $\#\colon X_1\times X_2\times\cdots\times X_n \longrightarrow X$ is defined
    on $X$, i.e. there exists $k\geq 1$ and indexes $1\leq j_1 < j_2 < \cdots < j_k\leq n$
    such that $X_{j_1}=X_{j_2}=\cdots=X_{j_k}=X$. For simplicity, let us assume that $j_i=i$.
    A subset $U\subset X$ is said to be \textbf{closed under} operation $\#$ if
    for all $u_1,u_2,\ldots,u_k$ from U and for all $x_j\in X_j\, j>k$ holds:
    $$
        \#(u_1,u_2,\ldots,u_k,x_{k+1},x_{k+2},\ldots,x_n)\in U.
    $$
\end{dfn}
%
\noindent
The next examples illustrates this definition.

\vspace{0.5cm}
\noindent
\underline{\textbf{Examples}}
\begin{enumerate}
\item Vector space $V$ over a field $K$ is a set, on which the following two operations
    are defined:
    \begin{itemize}
    \item multiplication by a scalar:
        $$
            \cdot\colon K\times V\longrightarrow V
        $$

    \item addition
        $$
            +\colon V\times V \longrightarrow V.
        $$
    \end{itemize}
    Of course these operations need to satisfy some properties (for details see the entry vector space).
    A subset $W\subset V$, which is closed under these operations, is called vector subspace.

\item Consider collection of all subsets of the real numbers $\mathbb{R}$, which we denote by $2^\mathbb{R}$.
    On this collection, binary operation intersection of sets is defined:
    $$
        \cap\colon 2^\mathbb{R} \times 2^\mathbb{R} \longrightarrow 2^\mathbb{R}.
    $$
    Collection of sets $\mathfrak{C}\subset 2^\mathbb{R}$:
    $$
        \mathfrak{C}:=\{ [a,b) \colon \, a\leq b \}
    $$
    is closed under this operation.
\end{enumerate}
%%%%%
%%%%%
\end{document}
