\documentclass[12pt]{article}
\usepackage{pmmeta}
\pmcanonicalname{Number}
\pmcreated{2015-11-17 12:09:30}
\pmmodified{2015-11-17 12:09:30}
\pmowner{pahio}{2872}
\pmmodifier{pahio}{2872}
\pmtitle{number}
\pmrecord{10}{40621}
\pmprivacy{1}
\pmauthor{pahio}{2872}
\pmtype{Feature}
\pmcomment{trigger rebuild}
\pmclassification{msc}{03E10}
\pmrelated{ClassificationOfComplexNumbers}
\pmrelated{Fraction}
\pmrelated{CardinalNumber}
\pmrelated{OrdinalNumber}
\pmrelated{RuleOfProduct}

% this is the default PlanetMath preamble.  as your knowledge
% of TeX increases, you will probably want to edit this, but
% it should be fine as is for beginners.

% almost certainly you want these
\usepackage{amssymb}
\usepackage{amsmath}
\usepackage{amsfonts}

% used for TeXing text within eps files
%\usepackage{psfrag}
% need this for including graphics (\includegraphics)
%\usepackage{graphicx}
% for neatly defining theorems and propositions
 \usepackage{amsthm}
% making logically defined graphics
%%%\usepackage{xypic}

% there are many more packages, add them here as you need them

% define commands here

\theoremstyle{definition}
\newtheorem*{thmplain}{Theorem}

\begin{document}
{\em Number} is an abstract concept which is 
\textbf{not defined generally} in mathematics.\, The numbers 
can be used for counting \PMlinkescapetext{objects, comparing 
quantities and expressing order}.

In mathematics one \textbf{can define} different 
\textbf{kinds} of numbers; some of the most common are 
\PMlinkescapetext{the integers, the fractional numbers, the 
real numbers and the complex numbers}.\, There are also many 
special kinds of e.g. \PMlinkescapetext{integers}: odd numbers, 
prime numbers, triangular numbers, Fibonacci numbers, etc.\, 
The \PMlinkname{algebraic integers}{AlgebraicNumberTheory} are 
a special kind of complex numbers, having 
\PMlinkname{similar}{Number} divisibility properties as the 
\PMlinkescapetext{rational integers} but much richer.

Usually the \PMlinkescapetext{word ``number'' refers only to the complex} numbers, which can be thought to be formed by gradually expanding the available system of numbers:
\begin{itemize}
\item natural numbers 1, 2, 3, \ldots
\item integers (added \PMlinkname{0}{Null} and negative integers)
\item rational numbers (added fractional numbers)
\item real numbers (added irrational numbers)
\item complex numbers (added imaginary numbers)
\end{itemize}
A usual applier of the mathematics, e.g. an engineer, probably 
believes that there are no other numbers than the complex 
numbers.\, Some school book may tell that the complex numbers 
form the widest possible \PMlinkname{field}{Field} of numbers.\, However, the mathematicians know that there exist infinitely many extension fields of the field $\mathbb{C}$ of the complex numbers, e.g. the rational function field $\mathbb{C}(X)$ or the formal Laurent series field $\mathbb{C}((X))$.\, That's a different matter if one wants to call {\em numbers} the elements of the last fields.

The field $\mathbb{Q}$ of the rational numbers can be extended 
also in another direction than the real and complex numbers:\, 
the field of \PMlinkname{$p$-adic numbers}{PAdicIntegers} makes a completion of $\mathbb{Q}$ which resembles $\mathbb{R}$ but which is not contained neither in $\mathbb{R}$ \PMlinkescapetext{nor} in $\mathbb{C}$.

%%%%%
%%%%%
\end{document}
