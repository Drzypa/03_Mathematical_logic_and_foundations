\documentclass[12pt]{article}
\usepackage{pmmeta}
\pmcanonicalname{111TheFieldOfRationalNumbers}
\pmcreated{2013-11-06 17:43:55}
\pmmodified{2013-11-06 17:43:55}
\pmowner{PMBookProject}{1000683}
\pmmodifier{PMBookProject}{1000683}
\pmtitle{11.1 The field of rational numbers}
\pmrecord{1}{}
\pmprivacy{1}
\pmauthor{PMBookProject}{1000683}
\pmtype{Feature}
\pmclassification{msc}{03B15}

\usepackage{xspace}
\usepackage{amssyb}
\usepackage{amsmath}
\usepackage{amsfonts}
\usepackage{amsthm}
\newcommand{\defeq}{\vcentcolon\equiv}  
\newcommand{\emptyt}{\ensuremath{\mathbf{0}}\xspace}
\newcommand{\indexdef}[1]{\index{#1|defstyle}}   
\newcommand{\indexsee}[2]{\index{#1|see{#2}}}    
\newcommand{\N}{\ensuremath{\mathbb{N}}\xspace}
\newcommand{\Q}{\ensuremath{\mathbb{Q}}\xspace}
\newcommand{\Qp}{\Q_{+}}
\newcommand{\symlabel}[1]{\refstepcounter{symindex}\label{#1}}
\newcommand{\unit}{\ensuremath{\mathbf{1}}\xspace}
\newcommand{\vcentcolon}{:\!\!}
\newcommand{\Z}{\ensuremath{\mathbb{Z}}\xspace}
\let\autoref\cref
\let\setof\Set    

\begin{document}

\indexdef{rational numbers}%
\indexsee{number!rational}{rational numbers}%
We first construct the rational numbers \Q, as the reals can then be seen as a completion
of~\Q. An expert will point out that \Q could be replaced by any approximate field,
\indexdef{field!approximate}%
i.e., a subring of \Q in which arbitrarily precise approximate inverses
\index{inverse!approximate}%
exist. An example is the
ring of dyadic rationals,
\index{rational numbers!dyadic}%
which are those of the form $n/2^k$. 
If we were implementing constructive mathematics on a computer,
an approximate field would be more suitable, but we leave such finesse for those
who care about the digits of~$\pi$.

We constructed the integers \Z in \autoref{sec:set-quotients} as a quotient of $\N\times
\N$, and observed that this quotient is generated by an idempotent. In
\autoref{sec:free-algebras} we saw that \Z is the free group on \unit; we could similarly
show that it is the free commutative ring\index{ring} on \emptyt. The field of rationals \Q is
constructed along the same lines as well, namely as the quotient
%
\[ \Q \defeq (\Z \times \N)/{\approx} \]
%
where
\[ (u,a) \approx (v,b) \defeq (u (b + 1) = v (a + 1)). \]
%
In other words, a pair $(u, a)$ represents the rational number $u / (1 + a)$. There can be
no division by zero because we cunningly added one to the denominator~$a$. Here too we
have a canonical choice of representatives, namely fractions in lowest terms. Thus we may
apply \autoref{lem:quotient-when-canonical-representatives} to obtain a set \Q, which
again has a decidable equality.
\index{decidable!equality}%

We do not bother to write down the arithmetical operations on \Q as we trust our readers
know how to compute with fractions even in the case when one is added to the denominator.
Let us just record the conclusion that there is an entirely unproblematic construction of
the ordered field of rational numbers \Q, with a decidable equality and decidable order.
It can also be characterized as the initial ordered field.
\index{initial!ordered field}%

\symlabel{positive-rationals}
\indexdef{rational numbers!positive}%
\indexdef{positive!rational numbers}%
Let $\Qp = \setof{ q : \Q | q > 0 }$ be the type of positive rational numbers.


\end{document}
