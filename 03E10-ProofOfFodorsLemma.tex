\documentclass[12pt]{article}
\usepackage{pmmeta}
\pmcanonicalname{ProofOfFodorsLemma}
\pmcreated{2013-03-22 12:53:19}
\pmmodified{2013-03-22 12:53:19}
\pmowner{Henry}{455}
\pmmodifier{Henry}{455}
\pmtitle{proof of Fodor's lemma}
\pmrecord{4}{33234}
\pmprivacy{1}
\pmauthor{Henry}{455}
\pmtype{Proof}
\pmcomment{trigger rebuild}
\pmclassification{msc}{03E10}

% this is the default PlanetMath preamble.  as your knowledge
% of TeX increases, you will probably want to edit this, but
% it should be fine as is for beginners.

% almost certainly you want these
\usepackage{amssymb}
\usepackage{amsmath}
\usepackage{amsfonts}

% used for TeXing text within eps files
%\usepackage{psfrag}
% need this for including graphics (\includegraphics)
%\usepackage{graphicx}
% for neatly defining theorems and propositions
%\usepackage{amsthm}
% making logically defined graphics
%%%\usepackage{xypic}

% there are many more packages, add them here as you need them

% define commands here
%\PMlinkescapeword{theory}
\begin{document}
If we let $f^{-1}:\kappa\rightarrow P(S)$ be the inverse of $f$ restricted to $S$ then Fodor's lemma is equivalent to the claim that for any function such that $\alpha\in f(\kappa)\rightarrow \alpha>\kappa$ there is some $\alpha\in S$ such that $f^{-1}(\alpha)$ is stationary.

Then if Fodor's lemma is false, for every $\alpha\in S$ there is some club set $C_\alpha$ such that $C_\alpha\cap f^{-1}(\alpha)=\emptyset$.  Let $C=\Delta_{\alpha<\kappa} C_\alpha$.  The club sets are closed under diagonal intersection, so $C$ is also club and therefore there is some $\alpha\in S\cap C$.  Then $\alpha\in C_\beta$ for each $\beta<\alpha$, and so there can be no $\beta<\alpha$ such that $\alpha\in f^{-1}(\beta)$, so $f(\alpha)\geq\alpha$, a contradiction.
%%%%%
%%%%%
\end{document}
