\documentclass[12pt]{article}
\usepackage{pmmeta}
\pmcanonicalname{OrdinalExponentiation}
\pmcreated{2013-03-22 17:51:09}
\pmmodified{2013-03-22 17:51:09}
\pmowner{CWoo}{3771}
\pmmodifier{CWoo}{3771}
\pmtitle{ordinal exponentiation}
\pmrecord{6}{40327}
\pmprivacy{1}
\pmauthor{CWoo}{3771}
\pmtype{Definition}
\pmcomment{trigger rebuild}
\pmclassification{msc}{03E10}
\pmrelated{PropertiesOfOrdinalArithmetic}

\endmetadata

\usepackage{amssymb,amscd}
\usepackage{amsmath}
\usepackage{amsfonts}
\usepackage{mathrsfs}

% used for TeXing text within eps files
%\usepackage{psfrag}
% need this for including graphics (\includegraphics)
%\usepackage{graphicx}
% for neatly defining theorems and propositions
\usepackage{amsthm}
% making logically defined graphics
%%\usepackage{xypic}
\usepackage{pst-plot}

% define commands here
\newcommand*{\abs}[1]{\left\lvert #1\right\rvert}
\newtheorem{prop}{Proposition}
\newtheorem{thm}{Theorem}
\newtheorem{ex}{Example}
\newcommand{\real}{\mathbb{R}}
\newcommand{\pdiff}[2]{\frac{\partial #1}{\partial #2}}
\newcommand{\mpdiff}[3]{\frac{\partial^#1 #2}{\partial #3^#1}}
\begin{document}
Let $\alpha,\beta$ be ordinals.  We define $\alpha^\beta$ as follows:
\begin{displaymath}
\alpha^\beta:= \left\{
\begin{array}{ll}
1 & \textrm{if }\beta=0,\\
\alpha^\gamma\cdot \alpha & \textrm{if $\beta$ is a successor ordinal and }\beta=S\gamma, \\
\sup\lbrace \alpha^\gamma \mid \gamma<\beta \rbrace & \textrm{if $\beta$ is a limit ordinal and }\beta=\sup\lbrace \gamma\mid \gamma<\beta\rbrace.
\end{array}
\right.
\end{displaymath}

Some properties of exponentiation:
\begin{enumerate}
\item $0^\alpha=0$ if $\alpha>0$
\item $1^\alpha=1$
\item $\alpha^1=\alpha$
\item $\alpha^\beta\cdot \alpha^\gamma=\alpha^{\beta+\gamma}$
\item $(\alpha^\beta)^\gamma=\alpha^{\beta\cdot\gamma}$
\item For any ordinals $\alpha,\beta$ with $\alpha>0$ and $\beta>1$, there exists a unique triple $(\gamma,\delta,\epsilon)$ of ordinals such that
$$\alpha=\beta^\gamma\cdot \delta+\epsilon$$
where $0<\delta<\beta$ and $\epsilon<\beta^\delta$.
\end{enumerate}

All of these properties can be proved using transfinite induction.
%%%%%
%%%%%
\end{document}
