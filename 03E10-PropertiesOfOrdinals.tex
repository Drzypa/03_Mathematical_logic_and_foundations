\documentclass[12pt]{article}
\usepackage{pmmeta}
\pmcanonicalname{PropertiesOfOrdinals}
\pmcreated{2013-03-22 18:48:58}
\pmmodified{2013-03-22 18:48:58}
\pmowner{CWoo}{3771}
\pmmodifier{CWoo}{3771}
\pmtitle{properties of ordinals}
\pmrecord{15}{41618}
\pmprivacy{1}
\pmauthor{CWoo}{3771}
\pmtype{Result}
\pmcomment{trigger rebuild}
\pmclassification{msc}{03E10}

\usepackage{amssymb,amscd}
\usepackage{amsmath}
\usepackage{amsfonts}
\usepackage{mathrsfs}

% used for TeXing text within eps files
%\usepackage{psfrag}
% need this for including graphics (\includegraphics)
%\usepackage{graphicx}
% for neatly defining theorems and propositions
\usepackage{amsthm}
% making logically defined graphics
%%\usepackage{xypic}
\usepackage{pst-plot}

% define commands here
\newcommand*{\abs}[1]{\left\lvert #1\right\rvert}
\newtheorem{prop}{Proposition}
\newtheorem{thm}{Theorem}
\newtheorem{ex}{Example}
\newcommand{\real}{\mathbb{R}}
\newcommand{\pdiff}[2]{\frac{\partial #1}{\partial #2}}
\newcommand{\mpdiff}[3]{\frac{\partial^#1 #2}{\partial #3^#1}}
\begin{document}
Recall that an ordinal is a transitive set well-ordered by the membership relation $\in$.  Below we list and prove some basic properties of ordinals.

\begin{enumerate}
\item $\varnothing$ is an ordinal.  This is vacuously true.
\item If $\alpha$ is an ordinal, then $\alpha\notin \alpha$, for otherwise $\alpha\in \alpha\in \alpha$, contradicting the fact that $\in$ is irreflexive on $\alpha$.  Notice that by the axiom of foundation, this statement is true even if $\alpha$ is not an ordinal.  But in this case, there is no need to resort to the axiom.
\item If $\alpha$ is an ordinal, and $\beta  \in \alpha$, then $\beta$ is an ordinal.
\begin{proof} Since $\alpha$ is transitive, $\beta\in \alpha$ implies that $\beta \subseteq \alpha$, and hence $\beta$ is well-ordered by $\in$ (inherited from $\alpha$).  Now, suppose $\delta \in \gamma\in \beta$.  Then $\gamma\in \alpha$ and $\delta\in \alpha$.  Since $\in$ is a linear ordering in $\alpha$, $\delta \in \beta$, as desired.
\end{proof}
\item If $\alpha$ is an ordinal, so is $\alpha^+:=\alpha \cup \lbrace \alpha\rbrace$ ($\alpha^+$ is the successor of $\alpha$).
\begin{proof}  If $\gamma\in \alpha^+$, then either $\gamma=\alpha$ or $\gamma \in \alpha$, and in the latter case, $\gamma\subseteq \alpha$, since $\alpha$ is transitive.  In either case, $\gamma \subseteq \alpha^+$ so that $\alpha^+$ is transitive.  Now, suppose $\delta\subseteq \alpha^+$.  Then $\delta - \lbrace \alpha\rbrace \subseteq\alpha$, so has a least element $\eta$.  Since $\eta\in \alpha$, $\eta$ is also the least element of $\delta$.  This shows that $\alpha^+$ is well-ordered by $\in$, and therefore $\alpha^+$ is an ordinal.
\end{proof}
\item If $\alpha$ is an ordinal and $\beta \subset \alpha$ is transitive, then $\beta \in \alpha$.  As a result, if $\alpha \ne \beta$ are ordinals such that $\beta\subset \alpha$, then $\beta\in \alpha$.
\begin{proof}  $\beta$ is well-ordered since $\alpha$ is, and since $\beta$ is transitive by assumption, $\beta$ is an ordinal.  Let $\gamma$ be the least element of $\alpha-\beta$, and suppose $\delta\in \beta$.  Since $\in$ is a linear ordering on $\alpha$, either $\delta\in \gamma$ or $\gamma\in \delta$.  If the latter is true, then $\gamma\in \delta\in \beta$, and therefore $\gamma\in \beta$ since $\beta$ is transitive, contradicting $\gamma \in \alpha - \beta$.  So $\delta\in \gamma$, and therefore $\beta \subseteq \gamma$.  Moreover, if $\delta\in \gamma$, then $\delta \notin \alpha-\beta$ since $\gamma$ is the least element in $\alpha-\beta$.  Therefore $\delta\in \beta$, so $\gamma\subseteq \beta$, and hence $\beta=\gamma\in \alpha$.
\end{proof}
\item (Law of Trichotomy) For any ordinals $\alpha,\beta$, exactly one of the following is true: $$\alpha=\beta,\qquad \alpha \in \beta,\qquad \mbox{and}\qquad \beta \in \alpha.$$
\begin{proof}  If $\theta\in \delta\in \alpha\cap \beta$, then $\theta\in \alpha\cap \beta$ since both $\alpha$ and $\beta$ are transitive.  So $\alpha\cap\beta$ is transitive, and hence an ordinal.  If $\alpha\cap \beta \subset \alpha$ and $\alpha\cap \beta \subset \beta$, then $\alpha\cap \beta \in \alpha\cap \beta$ by 5, which is impossible by 2.  This means that either $\alpha\cap \beta=\alpha$ or $\alpha\cap \beta=\beta$.  If $\alpha\ne \beta$, then $\alpha \subset \beta$, and hence by 5, $\alpha\in\beta$ in the former case, and $\beta \in \alpha$ in the latter.
\end{proof}
\item If $A$ is a non-empty class of ordinals, then $\bigcap A$ is the least element of $A$.  Furthermore, $A$ is well-ordered.
\begin{proof}  Set $\alpha:=\bigcap A$.  There are three steps: 
\begin{itemize}
\item $\alpha$ is an ordinal:  First, notice that $\alpha$ is well-ordered, as the ordering is inherited from members of $A$.  Next, if $\beta\in \alpha$, then $\beta\in \gamma$ for every $\gamma \in A$.  Since each $\gamma$ is transitive, $\beta\subseteq \gamma$, and $\beta\subseteq \alpha$ as a result, showing that $\alpha$ is transitive.  Finally, $\alpha$ is a set, since it is a subclass of each element (which is a set) of $A$.  This shows that $\alpha$ is an ordinal.
\item $\alpha \in A$:  By definition, $\alpha\subseteq \gamma$ for each $\gamma\in A$.  Suppose in addition that $\alpha\subset \gamma$.  Since $\alpha$ is an ordinal, then $\alpha\in \gamma$ by 5.  This shows that $\alpha \in \bigcap\lbrace \gamma\mid \gamma\in A\rbrace = \alpha$, which is impossible by 2.  Therefore $\alpha=\gamma$ for some $\gamma \in A$, as desired.
\item $\alpha$ is the least element in $A$:  If there is some $\delta \in A$ such that $\delta \in \alpha$.  But $\alpha = \bigcap \lbrace \gamma \mid \gamma \in A\rbrace \subseteq \delta$, implying that $\delta\in \delta$, a contradiction.
\end{itemize}
If $B$ is any non-empty subclass of $A$, then $\bigcap B$ is the least element of $B$ by the result above.  Hence $A$ is well-ordered.
\end{proof}
\item \textbf{On}, the class of all ordinals is transitive, well-ordered (by $\in$), and proper (not a set).
\begin{proof}  If $\alpha \in \beta\in $ \textbf{On}, then $\alpha \in $ \textbf{On} by 3, so \textbf{On} is transitive.  It is well-ordered by 7.  Finally, if \textbf{On} were a set, then \textbf{On} is an ordinal since it is transitive and well-ordered.  This means that \textbf{On} $\in$ \textbf{On}, contradicts 2 above.
\end{proof}
\item If $A$ is a non-empty set of ordinals, then $\bigcup A$ is an ordinal, and $\bigcup A = \sup A$.
\begin{proof}  Set $\alpha:=\bigcup A$.  First, $\alpha$ is a set because $A$ is.  If $\gamma \in \beta\in \alpha$, then $\gamma \in \beta \in \delta$ for some $\delta \in A$, and therefore $\gamma\in \delta \subseteq \alpha$, showing that $\alpha$ is transitive.  If $x\subseteq \alpha$, then every element of $x$ is an element of some $\delta\in A$.  Hence $x$ is a set of ordinals, and therefore well-ordered by 7.  This shows that $\alpha$ is an ordinal.

Now, pick any $\delta\in A$.  Then $\delta\subseteq \bigcup A=\alpha$, so either $\delta=\alpha$ or $\delta\subset \alpha$.  In the latter case, we get $\delta\in \alpha$ by 5.  This shows that $\alpha$ is an upper bound of $A$.  If $\theta$ is an ordinal such that $\delta\in \theta$ for every $\delta \in A$, then $\delta \subset \theta$, which implies $\alpha \subset \theta$, so that $\alpha \in \theta$ by 5, showing that $\alpha$ is the least upper bound of $A$.
\end{proof}
\item A transitive set of ordinals is an ordinal, true since the set is well-ordered by 7.  This is a partial converse of 3.  In particular, this provides an alternative proof of 4.  Another corollary is that an initial segment of an ordinal is an ordinal.
\item Two ordinals are isomorphic iff they are the same.
\begin{proof}  One direction is obvious.  Suppose now that $\phi: \alpha\to \beta$ is an order isomorphism between ordinals $\alpha$ and $\beta$.  Suppose $\alpha\ne \beta$.  Then either $\alpha\in \beta$ or $\beta\in \alpha$, which means that $\alpha\subset \beta$ or $\beta\subset \alpha$.  In either case, we get a contradiction since $\phi$ is a bijection.
\end{proof}
\item Every well-ordered set is order isomorphic to exactly one ordinal.
\begin{proof}  Uniqueness is provided by 12.  Now, let us prove the existence of such an ordinal.  First, notice that if there is an order preserving injection $W$ into an ordinal $\alpha$, then the image of $W$ is an initial segment of $\alpha$, which is an ordinal, and we are done.  Now, assume the contrary.  Since ordinals are well-ordered, the assumption implies that every $\alpha \in $ \textbf{On} has a strict injection into $W$ (this is an important property on well-ordered sets, see a \PMlinkname{proof here}{PropertiesOfWellOrderedSets}).  We shall derive a contradiction by showing that \textbf{On} is a set.  First, for each ordinal, there is a unique order-preserving injection into $W$.  In addition, if two ordinals are such that their corresponding injections into $W$ agree, then they are isomorphic and hence identical.  Thus, elements of \textbf{On} are completely determined by subsets (in fact, initial segments) of $W$.  Since $W$ is a set, so is \textbf{On}, a contradiction!
\end{proof}
\item An ordinal has exactly one of the following forms: $\varnothing$, $\bigcup \alpha$, or $\beta^+$ for some ordinals $\alpha,\beta$.
\begin{proof}  Suppose an ordinal $\alpha\ne \varnothing$.  Being a transitive set of ordinals (by 3), $\beta:=\bigcup \alpha$ is also an ordinal by 9. So $\beta\subseteq \alpha$.  Then either $\beta=\alpha$ or $\beta\subset \alpha$.  In the former case, $\alpha=\bigcup \alpha$.  In the latter case, $\beta\in \alpha$.  On the one hand, since $\beta = \sup \alpha$, any $\gamma \in \alpha$, either $\gamma=\beta$ or $\gamma \in \beta$, so that $\gamma\in \beta^+$ in any case.  This means that $\alpha\subseteq \beta^+$.  On the other hand, $\beta\in \alpha$ combined with $\beta \subset \alpha$ give us the other inclusion $\beta^+\subseteq \alpha$.  As a result, $\alpha=\beta^+$.
\end{proof}
\item A set is an ordinal iff it is a von Neumann ordinal.  Every von Neumann ordinal is clearly an ordinal by 1, 4, and 9.  The converse is true by 13.
\end{enumerate}

\textbf{Remark}.  Property 8 above also resolves one of the set theory paradoxes called the \emph{Burali-Forti paradox}: the collection of all ordinals is an ordinal, which is not true because the collection is not a set.
%%%%%
%%%%%
\end{document}
