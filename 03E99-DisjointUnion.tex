\documentclass[12pt]{article}
\usepackage{pmmeta}
\pmcanonicalname{DisjointUnion}
\pmcreated{2013-03-22 14:13:24}
\pmmodified{2013-03-22 14:13:24}
\pmowner{yark}{2760}
\pmmodifier{yark}{2760}
\pmtitle{disjoint union}
\pmrecord{9}{35659}
\pmprivacy{1}
\pmauthor{yark}{2760}
\pmtype{Definition}
\pmcomment{trigger rebuild}
\pmclassification{msc}{03E99}
\pmrelated{CardinalityOfDisjointUnionOfFiniteSets}

\usepackage{amssymb}
\usepackage{amsmath}
\usepackage{amsfonts}

\begin{document}
\section*{Disjoint union of two sets}

Let $A$ and $B$ be sets.  Then their \emph{disjoint union} is the union
\[
A\coprod B := A\cup(B\times\{\star\}),
\]
where $\star$ is an object chosen so that
$A$ and $B\times\{\star\}$ are disjoint.
Normally $B\times\{\star\}$ is identified with $B$ in the obvious way.
The element $\star$ is almost never mentioned;
it serves only as a ``tag'' to make the two sets disjoint.
If $A$ and $B$ are already disjoint,
then $A\coprod B$ is isomorphic to $A\cup B$;
this is the most common situation in practice.

\section*{Disjoint union of many sets}
If we have a collection of sets $\{A_i\}$ indexed by some set $I$,
then the disjoint union
\[
\coprod_{i\in I} A_i := \bigcup_{i\in I} A_i\times\{i\}.
\]
Observe that we have a natural isomorphism $A_i \to A_i\times\{i\}$,
and that the images of any pair of these isomorphisms have empty intersection.
This is also often called being pairwise disjoint
and is a much stronger condition than that
the intersection of all the images is empty.
As before, if the $A_i$ are already pairwise disjoint, then 
\[
\coprod_{i\in I} A_i \cong \bigcup_{i\in I} A_i.
\]

\section*{Explanation}
If one is working in some category,
the term ``disjoint union'' often means ``coproduct''.

For example, as sets, $\mathbb{R}\coprod\mathbb{R}$
is two copies of the real line.
As topological spaces, $\mathbb{R}\coprod\mathbb{R}$
is again two copies of the real line
with a topology whose open sets are pairs of real open sets,
one for each copy of $\mathbb{R}$.
This is the coproduct in the category of topological spaces.

Of course, there are many categories where this usage is unnatural.
For example, in the category of pointed sets,
the coproduct is the disjoint union
with the distinguished points identified.
In the category of abelian groups, the coproduct is the direct sum.

Another closely related usage should be mentioned.
Occasionally an author will write
``\ldots and $A\cup B$ is a disjoint union\ldots''.
What this means is that $A\cup B$ is isomorphic to $A\coprod B$,
which is to say that $A$ and $B$ are already disjoint.
%%%%%
%%%%%
\end{document}
