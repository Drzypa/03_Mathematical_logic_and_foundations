\documentclass[12pt]{article}
\usepackage{pmmeta}
\pmcanonicalname{EveryPropositionIsEquivalentToAPropositionInDNF}
\pmcreated{2013-03-22 15:06:58}
\pmmodified{2013-03-22 15:06:58}
\pmowner{rspuzio}{6075}
\pmmodifier{rspuzio}{6075}
\pmtitle{every proposition is equivalent to a proposition in DNF}
\pmrecord{10}{36853}
\pmprivacy{1}
\pmauthor{rspuzio}{6075}
\pmtype{Theorem}
\pmcomment{trigger rebuild}
\pmclassification{msc}{03B05}

% this is the default PlanetMath preamble.  as your knowledge
% of TeX increases, you will probably want to edit this, but
% it should be fine as is for beginners.

% almost certainly you want these
\usepackage{amssymb}
\usepackage{amsmath}
\usepackage{amsfonts}

% used for TeXing text within eps files
%\usepackage{psfrag}
% need this for including graphics (\includegraphics)
%\usepackage{graphicx}
% for neatly defining theorems and propositions
\usepackage{amsthm}
% making logically defined graphics
%%%\usepackage{xypic}

% there are many more packages, add them here as you need them

% define commands here
\theoremstyle{theorem}
\newtheorem*{thm}{Theorem}
\begin{document}
\begin{thm}
Given any proposition, there exists a proposition in disjunctive normal form which is equivalent to that proposition.
\end{thm}

\begin{proof}
Any two propositions are equivalent if and only if they determine the same truth function.  Therefore, if one can exhibit a mapping which assigns to a given truth function $f$ a proposition in disjunctive normal form such that the truth function of this proposition is $f$, the theorem follows immediately.  

Let $n$ denote the number of arguments $f$ takes.  Define
\[ V(f) = \{ X \in \{T,F\}^n | f(X) = T \} \]
For every $X \in \{T,F\}^n$, define $L_i (X) \colon \{T,F\}^n \to  \{T,F\}$ as follows:
\[ L_i (X)(Y) = \left\{ \begin{matrix} Y_i & X_i = T \cr \neg Y_i & X_i = F \end{matrix} \right. \]

Then, we claim that
 \[ f(Y) = \bigwedge_{X \in V(f)} \bigvee_{i=1}^n L_i (X)(Y) \]

On the one hand, suppose that $f(Y) = T$ for a certain $Y \in \{T,F\}^n$. By definition of $V(f)$, we have $Y \in V(f)$.  By definition of $L_i$, we have 
 \[L_i (Y) (Y) = \left\{ \begin{matrix} Y_i & Y_i = T \cr \neg Y_i & Y_i = F \end{matrix} \right. \]
In either case, $L_i (Y) (Y) = T$.  Since a conjunction equals $T$ if and only if each term of the conjunction equals $T$, it follows that  $\bigvee_{i=1}^n L_i (Y) (Y) = T$.  Finally, since a disjunction equals $T$ if and only if there exists a term which equals $T$, it follows the right hand side equals equals $T$ when the left-hand side equals $T$.

On the one hand, suppose that $f(Y) = F$ for a certain $Y \in \{T,F\}^n$.  Let $X$ be any element of $V(f)$.  Since $Y \notin V(f)$, there must exist an index $i$ such that $X_i \neq Y_i$.  For this choice of $i$, $Y_i = \neg X_i$ Then we have
  \[L_i (X) (Y) = \left\{ \begin{matrix} \neg X_i & X_i = T \cr \neg \neg X_i & X_i = F \end{matrix} \right. \]
In either case, $L_i (X) (Y) = F$.  Since a conjunction equals $F$ if and only if there exists a term which evaluates to $F$, it follows that $\bigvee_{i=1}^n L_i (X) (Y) = F$ for all $X \in V(f)$.  Since a disjunction equals $F$ if and only if each term of the conjunction equals $F$, it follows that the right hand side equals equals $F$ when the left-hand side equals $F$.

\end{proof}
%%%%%
%%%%%
\end{document}
