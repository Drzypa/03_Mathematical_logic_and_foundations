\documentclass[12pt]{article}
\usepackage{pmmeta}
\pmcanonicalname{BuraliFortiParadox}
\pmcreated{2013-03-22 13:04:28}
\pmmodified{2013-03-22 13:04:28}
\pmowner{Henry}{455}
\pmmodifier{Henry}{455}
\pmtitle{Burali-Forti paradox}
\pmrecord{8}{33484}
\pmprivacy{1}
\pmauthor{Henry}{455}
\pmtype{Definition}
\pmcomment{trigger rebuild}
\pmclassification{msc}{03-00}

% this is the default PlanetMath preamble.  as your knowledge
% of TeX increases, you will probably want to edit this, but
% it should be fine as is for beginners.

% almost certainly you want these
\usepackage{amssymb}
\usepackage{amsmath}
\usepackage{amsfonts}

% used for TeXing text within eps files
%\usepackage{psfrag}
% need this for including graphics (\includegraphics)
%\usepackage{graphicx}
% for neatly defining theorems and propositions
%\usepackage{amsthm}
% making logically defined graphics
%%%\usepackage{xypic}

% there are many more packages, add them here as you need them

% define commands here
%\PMlinkescapeword{theory}
\begin{document}
The \emph{Burali-Forti} paradox demonstrates that the class of all ordinals is not a set.  If there were a set of all ordinals, $Ord$, then it would follow that $Ord$ was itself an ordinal, and therefore that $Ord\in Ord$.  \PMlinkescapetext{Even} if sets in general are allowed to contain themselves, ordinals cannot since they are defined so that $\in$ is well founded over them.

This paradox is similar to both Russell's paradox and Cantor's paradox, although it predates both.  All of these paradoxes prove that a certain object is ``too large'' to be a set.
%%%%%
%%%%%
\end{document}
