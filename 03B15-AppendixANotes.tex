\documentclass[12pt]{article}
\usepackage{pmmeta}
\pmcanonicalname{AppendixANotes}
\pmcreated{2013-11-09 6:02:54}
\pmmodified{2013-11-09 6:02:54}
\pmowner{PMBookProject}{1000683}
\pmmodifier{PMBookProject}{1000683}
\pmtitle{Appendix A Notes}
\pmrecord{1}{}
\pmprivacy{1}
\pmauthor{PMBookProject}{1000683}
\pmtype{Feature}
\pmclassification{msc}{03B15}

\endmetadata

\usepackage{amssyb}
\usepackage{amsmath}
\usepackage{amsfonts}
\usepackage{amsthm}
\makeatletter
\newcommand{\sectionNotes}{\phantomsection\section*{Notes}\addcontentsline{toc}{section}{Notes}\markright{\textsc{\@chapapp{} \thechapter{} Notes}}}
\let\autoref\cref
\makeatother

\begin{document}
% This presentation is strongly inspired by two  Martin-L\"of 1972 and 1973.

The system of rules with introduction (primitive constants) and elimination
and computation rules (defined constant) is inspired by Gentzen natural
deduction. The possibility of strengthening the elimination rule for
existential quantification was indicated in \cite{howard:pat}. The
strengthening of the axioms for disjunction appears in \cite{Martin-Lof-1972},
and for absurdity elimination and identity type in \cite{Martin-Lof-1973}. The
$W$-types were introduced in \cite{Martin-Lof-1979}. They generalize a notion
of trees introduced by \cite{Tait-1968}.
\index{Martin-L\"of}%

%inspired from unpublished work of Spector.

The generalized form of primitive recursion for natural numbers and ordinals
appear in \cite{Hilbert-1925}. This motivated G\"odel's system $T$,
\cite{Goedel-T-1958}, which was analyzed by \cite{Tait-1966}, who used,
following \cite{Goedel-T-1958}, the terminology ``definitional equality'' for
conversion: two terms are \emph{judgmentally equal} if they reduce to a
common term by means of a sequence of applications of the reduction
rules. This terminology was also used by de Bruijn \cite{deBruijn-1973} in his
presentation of \emph{AUTOMATH}.\index{AUTOMATH}

Streicher \cite[Theorem 4.13]{Streicher-1991}, explains how to give the
semantics in a contextual category of terms in normal form using a simple syntax
similar to the one we have presented.

Our second presentation comprises fairly standard presentation of
intensional Martin-L\"{o}f type theory, with some additional features needed in
homotopy type theory. Compared to a reference presentation of
\cite{hofmann:syntax-and-semantics}, the type theory of this book has a few
non-critical differences:
%
\begin{itemize}
\item universes \`{a} la Russell, in the sense of
\cite{martin-lof:bibliopolis}; and
\item judgmental $\eta$ and function extensionality for $\Pi$ types;
\end{itemize}
and a few features essential for homotopy type theory:
\begin{itemize}
\item the univalence axiom; and
\item higher inductive types.
\end{itemize}
%
As a matter of convenience, the book primarily defines functions by induction
using definition by \emph{pattern matching}.
\index{pattern matching}%
\index{definition!by pattern matching}%
It is possible to formalize the
notion of pattern matching, as done in \autoref{sec:syntax-informally}. However, the
standard type-theoretic presentation, adopted in \autoref{sec:syntax-more-formally}, is to introduce a single \emph{dependent
eliminator} for each type former, from which functions out of that type must be
defined. This approach is easier to formalize both syntactically and
semantically, as it amounts to the universal property of the type former.
The two approaches are equivalent; see \PMlinkname{\S 1.10}{110patternmatchingandrecursion} for a
longer discussion.

\index{type theory!formal|)}%
\index{formal!type theory|)}%


%%% Local Variables: 
%%% mode: latex
%%% TeX-master: "hott-online"
%%% End: 

\end{document}
