\documentclass[12pt]{article}
\usepackage{pmmeta}
\pmcanonicalname{1134CauchyRealsAreCauchyComplete}
\pmcreated{2013-11-06 18:23:05}
\pmmodified{2013-11-06 18:23:05}
\pmowner{PMBookProject}{1000683}
\pmmodifier{PMBookProject}{1000683}
\pmtitle{11.3.4 Cauchy reals are Cauchy complete}
\pmrecord{1}{}
\pmprivacy{1}
\pmauthor{PMBookProject}{1000683}
\pmtype{Feature}
\pmclassification{msc}{03B15}

\endmetadata

\usepackage{xspace}
\usepackage{amssyb}
\usepackage{amsmath}
\usepackage{amsfonts}
\usepackage{amsthm}
\newcommand{\bsim}{\frown}
\newcommand{\closesym}{\mathord\sim}
\newcommand{\defeq}{\vcentcolon\equiv}  
\newcommand{\define}[1]{\textbf{#1}}
\newcommand{\indexdef}[1]{\index{#1|defstyle}}   
\newcommand{\Q}{\ensuremath{\mathbb{Q}}\xspace}
\newcommand{\Qp}{\Q_{+}}
\newcommand{\RC}{\ensuremath{\mathbb{R}_\mathsf{c}}\xspace} 
\newcommand{\rclim}{\mathsf{lim}} 
\newcommand{\rcrat}{\mathsf{rat}} 
\newcommand{\vcentcolon}{:\!\!}
\newcounter{mathcount}
\setcounter{mathcount}{1}
\newtheorem{prethm}{Theorem}
\newenvironment{thm}{\begin{prethm}}{\end{prethm}\addtocounter{mathcount}{1}}
\renewcommand{\theprethm}{11.3.\arabic{mathcount}}
\let\autoref\cref

\begin{document}

We constructed $\RC$ by closing $\Q$ under limits of Cauchy approximations, so it better
be the case that $\RC$ is Cauchy complete. Thanks to \autoref{RC-sim-eqv-le} there is no
difference between a Cauchy approximation $x : \Qp \to \RC$ as defined in the construction
of $\RC$, and a Cauchy approximation in the sense of \autoref{defn:cauchy-approximation}
(adapted to $\RC$).

Thus, given a Cauchy approximation $x : \Qp \to \RC$ it is quite natural to expect that
$\rclim(x)$ is its limit, where the notion of limit is defined as in
\autoref{defn:cauchy-approximation}. But this is so by \autoref{RC-sim-eqv-le} and
\autoref{thm:RC-sim-lim-term}. We have proved:

\begin{thm}
  Every Cauchy approximation in $\RC$ has a limit.
\end{thm}

An archimedean ordered field in which every Cauchy approximation has a limit is called
\define{Cauchy complete}.
\indexdef{Cauchy!completeness}%
\indexdef{complete!ordered field, Cauchy}%
\index{ordered field}%
The Cauchy reals are the least such field.

\begin{thm} \label{RC-initial-Cauchy-complete}
  The Cauchy reals embed into every Cauchy complete archimedean ordered field.
\end{thm}

\begin{proof}
  \index{limit!of a Cauchy approximation}%
  Suppose $F$ is a Cauchy complete archimedean ordered field. Because limits are unique,
  there is an operator $\lim$ which takes Cauchy approximations in $F$ to their limits. We
  define the embedding $e : \RC \to F$ by $(\RC, {\closesym})$-recursion as
  %
  \begin{equation*}
    e(\rcrat(q)) \defeq q
    \qquad\text{and}\qquad
    e(\rclim(x)) \defeq \lim (e \circ x).
  \end{equation*}
  %
  A suitable $\bsim$ on $F$ is
  %
  \begin{equation*}
    (a \bsim_\epsilon b) \defeq |a - b| < \epsilon.
  \end{equation*}
  %
  This is a separated relation because $F$ is archimedean. The rest of the clauses for
  $(\RC, {\closesym})$-recursion are easily checked. One would also have to check that $e$ is
  an embedding of ordered fields which fixes the rationals.
\end{proof}

\index{real numbers!Cauchy|)}%


\end{document}
