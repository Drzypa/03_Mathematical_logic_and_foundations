\documentclass[12pt]{article}
\usepackage{pmmeta}
\pmcanonicalname{75Connectedness}
\pmcreated{2013-11-21 15:40:15}
\pmmodified{2013-11-21 15:40:15}
\pmowner{PMBookProject}{1000683}
\pmmodifier{rspuzio}{6075}
\pmtitle{7.5 Connectedness}
\pmrecord{3}{87702}
\pmprivacy{1}
\pmauthor{PMBookProject}{6075}
\pmtype{Feature}
\pmclassification{msc}{03B15}

\usepackage{xspace}
\usepackage{amssyb}
\usepackage{amsmath}
\usepackage{amsfonts}
\usepackage{amsthm}
\makeatletter
\newcommand{\bfalse}{{0_{\bool}}}
\newcommand{\blank}{\mathord{\hspace{1pt}\text{--}\hspace{1pt}}}
\newcommand{\bool}{\ensuremath{\mathbf{2}}\xspace}
\newcommand{\brck}[1]{\trunc{}{#1}}
\newcommand{\defeq}{\vcentcolon\equiv}  
\newcommand{\define}[1]{\textbf{#1}}
\def\@dprd#1{\prod_{(#1)}\,}
\def\@dprd@noparens#1{\prod_{#1}\,}
\def\@dsm#1{\sum_{(#1)}\,}
\def\@dsm@noparens#1{\sum_{#1}\,}
\def\@eatprd\prd{\prd@parens}
\def\@eatsm\sm{\sm@parens}
\newcommand{\eqv}[2]{\ensuremath{#1 \simeq #2}\xspace}
\newcommand{\eqvsym}{\simeq}    
\newcommand{\hfib}[2]{{\mathsf{fib}}_{#1}(#2)}
\newcommand{\indexdef}[1]{\index{#1|defstyle}}   
\newcommand{\indexsee}[2]{\index{#1|see{#2}}}    
\newcommand{\iscontr}{\ensuremath{\mathsf{isContr}}}
\def\lam#1{{\lambda}\@lamarg#1:\@endlamarg\@ifnextchar\bgroup{.\,\lam}{.\,}}
\def\@lamarg#1:#2\@endlamarg{\if\relax\detokenize{#2}\relax #1\else\@lamvar{\@lameatcolon#2},#1\@endlamvar\fi}
\def\@lameatcolon#1:{#1}
\def\@lamvar#1,#2\@endlamvar{(#2\,{:}\,#1)}
\newcommand{\map}[2]{\ensuremath{{#1}\mathopen{}\left({#2}\right)\mathclose{}}\xspace}
\newcommand{\opp}[1]{\mathord{{#1}^{-1}}}
\newcommand{\pairr}[1]{{\mathopen{}(#1)\mathclose{}}}
\newcommand{\Parens}[1]{\Bigl(#1\Bigr)}
\def\prd#1{\@ifnextchar\bgroup{\prd@parens{#1}}{\@ifnextchar\sm{\prd@parens{#1}\@eatsm}{\prd@noparens{#1}}}}
\def\prd@noparens#1{\mathchoice{\@dprd@noparens{#1}}{\@tprd{#1}}{\@tprd{#1}}{\@tprd{#1}}}
\def\prd@parens#1{\@ifnextchar\bgroup  {\mathchoice{\@dprd{#1}}{\@tprd{#1}}{\@tprd{#1}}{\@tprd{#1}}\prd@parens}  {\@ifnextchar\sm    {\mathchoice{\@dprd{#1}}{\@tprd{#1}}{\@tprd{#1}}{\@tprd{#1}}\@eatsm}    {\mathchoice{\@dprd{#1}}{\@tprd{#1}}{\@tprd{#1}}{\@tprd{#1}}}}}
\newcommand{\proj}[1]{\ensuremath{\mathsf{pr}_{#1}}\xspace}
\newcommand{\refl}[1]{\ensuremath{\mathsf{refl}_{#1}}\xspace}
\def\sm#1{\@ifnextchar\bgroup{\sm@parens{#1}}{\@ifnextchar\prd{\sm@parens{#1}\@eatprd}{\sm@noparens{#1}}}}
\def\sm@noparens#1{\mathchoice{\@dsm@noparens{#1}}{\@tsm{#1}}{\@tsm{#1}}{\@tsm{#1}}}
\def\sm@parens#1{\@ifnextchar\bgroup  {\mathchoice{\@dsm{#1}}{\@tsm{#1}}{\@tsm{#1}}{\@tsm{#1}}\sm@parens}  {\@ifnextchar\prd    {\mathchoice{\@dsm{#1}}{\@tsm{#1}}{\@tsm{#1}}{\@tsm{#1}}\@eatprd}    {\mathchoice{\@dsm{#1}}{\@tsm{#1}}{\@tsm{#1}}{\@tsm{#1}}}}}
\newcommand{\total}[1]{\ensuremath{\mathsf{total}(#1)}}
\def\@tprd#1{\mathchoice{{\textstyle\prod_{(#1)}}}{\prod_{(#1)}}{\prod_{(#1)}}{\prod_{(#1)}}}
\newcommand{\tproj}[3][]{\mathopen{}\left|#3\right|_{#2}^{#1}\mathclose{}}
\newcommand{\tprojf}[2][]{|\blank|_{#2}^{#1}}
\newcommand{\trans}[2]{\ensuremath{{#1}_{*}\mathopen{}\left({#2}\right)\mathclose{}}\xspace}
\newcommand{\trunc}[2]{\mathopen{}\left\Vert #2\right\Vert_{#1}\mathclose{}}
\newcommand{\Trunc}[2]{\Bigl\Vert #2\Bigr\Vert_{#1}}
\def\@tsm#1{\mathchoice{{\textstyle\sum_{(#1)}}}{\sum_{(#1)}}{\sum_{(#1)}}{\sum_{(#1)}}}
\newcommand{\typele}[1]{\ensuremath{{#1}\text-\mathsf{Type}}\xspace}
\newcommand{\unit}{\ensuremath{\mathbf{1}}\xspace}
\newcommand{\UU}{\ensuremath{\mathcal{U}}\xspace}
\newcommand{\vcentcolon}{:\!\!}
\newcounter{mathcount}
\setcounter{mathcount}{1}
\newtheorem{precor}{Corollary}
\newenvironment{cor}{\begin{precor}}{\end{precor}\addtocounter{mathcount}{1}}
\renewcommand{\theprecor}{7.5.\arabic{mathcount}}
\newtheorem{predefn}{Definition}
\newenvironment{defn}{\begin{predefn}}{\end{predefn}\addtocounter{mathcount}{1}}
\renewcommand{\thepredefn}{7.5.\arabic{mathcount}}
\newtheorem{prelem}{Lemma}
\newenvironment{lem}{\begin{prelem}}{\end{prelem}\addtocounter{mathcount}{1}}
\renewcommand{\theprelem}{7.5.\arabic{mathcount}}
\newtheorem{prermk}{Remark}
\newenvironment{rmk}{\begin{prermk}}{\end{prermk}\addtocounter{mathcount}{1}}
\renewcommand{\theprermk}{7.5.\arabic{mathcount}}
\let\ap\map
\let\autoref\cref
\let\hfiber\hfib
\let\ntype\typele
\let\type\UU
\makeatother

\begin{document}
An $n$-type is one that has no interesting information above dimension $n$.
By contrast, an \emph{$n$-connected type} is one that has no interesting information \emph{below} dimension $n$.
It turns out to be natural to study a more general notion for functions as well.

\begin{defn}
A function $f:A\to B$ is said to be \define{$n$-connected}
\indexdef{function!n-connected@$n$-connected}%
\indexsee{n-connected@$n$-connected!function}{function, $n$-connected}%
if for all $b:B$, the type $\trunc n{\hfiber f b}$ is contractible:
\begin{equation*}
  \mathsf{conn}_n(f)\defeq \prd{b:B}\iscontr(\trunc n{\hfiber{f}b}). 
\end{equation*}
A type $A$ is said to be \define{$n$-connected}
\indexsee{n-connected@$n$-connected!type}{type, $n$-connected}%
\indexdef{type!n-connected@$n$-connected}%
 if the unique function $A\to\unit$ is $n$-connected, i.e.\ if $\trunc nA$ is contractible.
\end{defn}
\indexsee{connected!function}{function, $n$-connected}

Thus, a function $f:A\to B$ is $n$-connected if and only if $\hfib{f}b$ is $n$-connected for every $b:B$.
Of course, every function is $(-2)$-connected.
At the next level, we have:

\begin{lem}\label{thm:minusoneconn-surjective}
  \index{function!surjective}%
  A function $f$ is $(-1)$-connected if and only if it is surjective in the sense of \PMlinkname{\S 4.6}{46surjectionsandembeddings}.
\end{lem}
\begin{proof}
  We defined $f$ to be surjective if $\trunc{-1}{\hfiber f b}$ is inhabited for all $b$.
  But since it is a mere proposition, inhabitation is equivalent to contractibility.
\end{proof}

Thus, $n$-connectedness of a function for $n\ge 0$ can be thought of as a strong form of surjectivity.
Category-theoretically, $(-1)$-connectedness corresponds to essential surjectivity on objects, while $n$-connectedness corresponds to essential surjectivity on $k$-morphisms for $k\le n+1$.

\PMlinkname{Lemma 7.5.2}{75connectedness#Thmprelem1} also implies that a type $A$ is $(-1)$-connected if and only if it is merely inhabited.
When a type is $0$-connected we may simply say that it is \define{connected},
\indexdef{connected!type}%
\indexdef{type!connected}%
and when it is $1$-connected we say it is \define{simply connected}.
\indexdef{simply connected type}%
\indexdef{type!simply connected}%

\begin{rmk}\label{rmk:connectedness-indexing}
  While our notion of $n$-connectedness for types agrees with the standard notion in homotopy theory, our notion of $n$-connectedness for \emph{functions} is off by one from a common indexing in classical homotopy theory.
  Whereas we say a function $f$ is $n$-connected if all its fibers are $n$-connected, some classical homotopy theorists would call such a function $(n+1)$-connected.
  (This is due to a historical focus on \emph{cofibers} rather than fibers.)
\end{rmk}

We now observe a few closure properties of connected maps.
\index{function!n-connected@$n$-connected}

\begin{lem}
\index{retract!of a function}%
Suppose that $g$ is a retract of a $n$-connected function $f$.  Then $g$ is
$n$-connected.
\end{lem}
\begin{proof}
This is a direct consequence of \PMlinkname{Lemma 4.7.3}{47closurepropertiesofequivalences#Thmprelem1}.
\end{proof}

\begin{cor}
If $g$ is homotopic to a $n$-connected function $f$, then $g$ is $n$-connected.
\end{cor}

\begin{lem}\label{lem:nconnected_postcomp}
Suppose that $f:A\to B$ is $n$-connected. Then $g:B\to C$ is $n$-connected if and only if $g\circ f$ is
$n$-connected.
\end{lem}

\begin{proof}
For any $c:C$, we have
\begin{align*}
  \trunc n{\hfib{g\circ f}c}
  & \eqvsym \Trunc n{ \sm{w:\hfib{g}c}\hfib{f}{\proj1 w}}
  \tag{by \PMlinkexternal{Exercise 4.4}{http://planetmath.org/node/87774}}\\
  & \eqvsym \Trunc n{\sm{w:\hfib{g}c} \trunc n{\hfib{f}{\proj1 w}}}
  \tag{by \PMlinkname{Theorem 7.3.9}{73truncations#Thmprethm3}}\\
  & \eqvsym \trunc n{\hfib{g}c}.
  \tag{since $\trunc n{\hfib{f}{\proj1 w}}$ is contractible}
\end{align*}
It follows that $\trunc n{\hfib{g}c}$ is contractible if and only if $\trunc n{\hfib{g\circ f}c}$ is
contractible.
\end{proof}

Importantly, $n$-connected functions can be equivalently characterized as those which satisfy an ``induction principle'' with respect to $n$-types.\index{induction principle!for connected maps} 
This idea will lead directly into our proof of the Freudenthal suspension theorem in \PMlinkname{\S 8.6}{86thefreudenthalsuspensiontheorem}.

\begin{lem}\label{prop:nconnected_tested_by_lv_n_dependent types}
For $f:A\to B$ and $P:B\to\type$, consider the following function:
\begin{equation*}
\lam{s} s\circ f :\Parens{\prd{b:B} P(b)}\to\Parens{\prd{a:A}P(f(a))}.
\end{equation*}
For a fixed $f$ and $n\ge -2$, the following are equivalent.
\begin{enumerate}
\item $f$ is $n$-connected.\label{item:conntest1}
\item For every $P:B\to\ntype{n}$, the map $\lam{s} s\circ f$ is an equivalence.\label{item:conntest2}
\item For every $P:B\to\ntype{n}$, the map $\lam{s} s\circ f$ has a section.\label{item:conntest3}
\end{enumerate}
\end{lem}

\begin{proof}
Suppose that $f$ is $n$-connected and let $P:B\to\ntype{n}$. Then we have the equivalences
\begin{align}
  \prd{b:B} P(b) & \eqvsym \prd{b:B} \Parens{\trunc n{\hfib{f}b} \to P(b)}
  \tag{since $\trunc n{\hfib{f}b}$ is contractible}\\
  & \eqvsym \prd{b:B} \Parens{\hfib{f}b\to P(b)}
  \tag{since $P(b)$ is an $n$-type}\\
  & \eqvsym \prd{b:B}{a:A}{p:f(a)= b} P(b)
  \tag{by the left universal property of $\Sigma$-types}\\
  & \eqvsym \prd{a:A} P(f(a)).
  \tag{by the left universal property of path types}
\end{align}
We omit the proof that this equivalence is indeed given by $\lam{s} s\circ f$.
Thus,~\ref{item:conntest1}$\Rightarrow$\ref{item:conntest2}, and clearly~\ref{item:conntest2}$\Rightarrow$\ref{item:conntest3}.
To show~\ref{item:conntest3}$\Rightarrow$\ref{item:conntest1}, consider the type family
\begin{equation*}
P(b)\defeq \trunc n{\hfib{f}b}.
\end{equation*}
Then~\ref{item:conntest3} yields a map $c:\prd{b:B} \trunc n{\hfib{f}b}$ with
$c(f(a))=\tproj n{\pairr{a,\refl{f(a)}}}$. To show that each $\trunc n{\hfib{f}b}$ is contractible,
we will find a function of type
\begin{equation*}
\prd{b:B}{w:\trunc n{\hfib{f}b}} w= c(b).
\end{equation*}
By \PMlinkname{Theorem 7.3.2}{73truncations#Thmprethm1}, for this it suffices to find a function of type
\begin{equation*}
\prd{b:B}{a:A}{p:f(a)= b} \tproj n{\pairr{a,p}}= c(b).
\end{equation*}
But by rearranging variables and path induction, this is equivalent to the type
\begin{equation*}
\prd{a:A} \tproj n{\pairr{a,\refl{f(a)}}}= c(f(a)).
\end{equation*}
This property holds by our choice of $c(f(a))$. 
\end{proof}

\begin{cor}\label{cor:totrunc-is-connected}
For any $A$, the canonical function $\tprojf n:A\to\trunc n A$ is $n$-connected.
\end{cor}
\begin{proof}
By \PMlinkname{Theorem 7.3.2}{73truncations#Thmprethm1} and the associated uniqueness principle, the condition of \PMlinkname{Lemma 7.5.7}{75connectedness#Thmprelem4} holds.
\end{proof}

For instance, when $n=-1$, \PMlinkname{Corollary 7.5.8}{75connectedness#Thmprecor2} says that the map $A\to \brck A$ from a type to its propositional truncation is surjective.

\begin{cor}\label{thm:nconn-to-ntype-const}\label{connectedtotruncated}
A type $A$ is $n$-connected if and only if the map
\begin{equation*}
  \lam{b}{a} b: B \to (A\to B)
\end{equation*}
is an equivalence for every $n$-type $B$.
In other words, ``every map from $A$ to an $n$-type is constant''.
\end{cor}
\begin{proof}
  By \PMlinkname{Lemma 7.5.7}{75connectedness#Thmprelem4} applied to a function with codomain $\unit$.
\end{proof}

\begin{lem}\label{lem:nconnected_to_leveln_to_equiv}
Let $B$ be an $n$-type and let $f:A\to B$ be a function. Then the induced function $g:\trunc n A\to B$ is an
equivalence if and only if $f$ is $n$-connected.
\end{lem}

\begin{proof}
By \PMlinkname{Corollary 7.5.8}{75connectedness#Thmprecor2}, $\tprojf n$ is $n$-connected.
Thus, since $f = g\circ \tprojf n$, by
\PMlinkname{Lemma 7.5.6}{75connectedness#Thmprelem3} $f$ is $n$-connected if and only if $g$ is $n$-connected.
But since $g$ is a function between $n$-types, its fibers are also $n$-types.
Thus, $g$ is $n$-connected if and only if it is an equivalence.
\end{proof}

We can also characterize connected pointed types in terms of connectivity of the inclusion of their basepoint.

\begin{lem}\label{thm:connected-pointed}
  \index{basepoint}%
  Let $A$ be a type and $a_0:\unit\to A$ a basepoint, with $n\ge -1$.
  Then $A$ is $n$-connected if and only if the map $a_0$ is $(n-1)$-connected.
\end{lem}
\begin{proof}
  First suppose $a_0:\unit\to A$ is $(n-1)$-connected and let $B$ be an $n$-type; we will use \PMlinkname{Corollary 7.5.9}{75connectedness#Thmprecor3}.
  The map $\lam{b}{a} b: B \to (A\to B)$ has a retraction given by $f\mapsto f(a_0)$, so it suffices to show it also has a section, i.e.\ that for any $f:A\to B$ there is $b:B$ such that $f = \lam{a}b$.
  We choose $b\defeq f(a_0)$.
  Define $P:A\to\type$ by $P(a) \defeq (f(a)=f(a_0))$.
  Then $P$ is a family of $(n-1)$-types and we have $P(a_0)$; hence we have $\prd{a:A} P(a)$ since $a_0:\unit\to A$ is $(n-1)$-connected.
  Thus, $f = \lam{a} f(a_0)$ as desired.

  Now suppose $A$ is $n$-connected, and let $P:A\to\ntype{(n-1)}$ and $u:P(a_0)$ be given.
  By \PMlinkname{Lemma 7.5.7}{75connectedness#Thmprelem4}, it will suffice to construct $f:\prd{a:A} P(a)$ such that $f(a_0)=u$.
  Now $\ntype{(n-1)}$ is an $n$-type and $A$ is $n$-connected, so by \PMlinkname{Corollary 7.5.9}{75connectedness#Thmprecor3}, there is an $n$-type $B$ such that $P = \lam{a} B$.
  Hence, we have a family of equivalences $g:\prd{a:A} (\eqv{P(a)}{B})$.
  Define $f(a) \defeq \opp{g_a}(g_{a_0}(u))$; then $f:\prd{a:A} P(a)$ and $f(a_0) = u$ as desired.
\end{proof}

In particular, a pointed type $(A,a_0)$ is 0-connected if and only if $a_0:\unit\to A$ is surjective, which is to say $\prd{x:A} \brck{x=a_0}$.

A useful variation on \PMlinkname{Lemma 7.5.6}{75connectedness#Thmprelem3} is:

\begin{lem}\label{lem:nconnected_postcomp_variation}
Let $f:A\to B$ be a function and $P:A\to\type$ and $Q:B\to\type$ be type families. Suppose that $g:\prd{a:A} P(a)\to Q(f(a))$
is a fiberwise $n$-connected%
\index{fiberwise!n-connected family of functions@$n$-connected family of functions}
family of functions, i.e.\ each function $g_a : P(a) \to Q(f(a))$ is $n$-connected. Then the function
\begin{align*}
\varphi &:\Parens{\sm{a:A} P(a)}\to\Parens{\sm{b:B} Q(b)}\\
\varphi(a,u) &\defeq \pairr{f(a),g_a(u)}
\end{align*}
is $n$-connected if and only if $f$ is $n$-connected.
\end{lem}

\begin{proof}
For $b:B$ and $v:Q(b)$ we have
{\allowdisplaybreaks
\begin{align*}
\trunc n{\hfib{\varphi}{\pairr{b,v}}} & \eqvsym \Trunc n{\sm{a:A}{u:P(a)}{p:f(a)= b} \trans{\ap f p }{g_a(u)}= v}\\
& \eqvsym \Trunc n{\sm{w:\hfib{f}b}{u:P(\proj1(w))} g_{\proj 1 w}(u)= \trans{\opp{\ap f {\proj2 w}}}{v}}\\
& \eqvsym \Trunc n{\sm{w:\hfib{f}b} \hfib{g(\proj1 w)}{\trans{\opp{\ap f {\proj 2 w}}}{v}}}\\
& \eqvsym \Trunc n{\sm{w:\hfib{f}b} \trunc n{\hfib{g(\proj1 w)}{\trans{\opp{\ap f {\proj 2 w}}}{v}}}}\\
& \eqvsym \trunc n{\hfib{f}b}
\end{align*}}
where the transportations along $f(p)$ and $f(p)^{-1}$ are with respect to $Q$.
Therefore, if either is contractible, so is the other.
\end{proof}

In the other direction, we have

\begin{lem}\label{prop:nconn_fiber_to_total}
Let $P,Q:A\to\type$ be type families and consider a fiberwise transformation\index{fiberwise!transformation}
\begin{equation*}
f:\prd{a:A} \Parens{P(a)\to Q(a)}
\end{equation*}
from $P$ to $Q$. Then the induced map $\total f: \sm{a:A}P(a) \to \sm{a:A} Q(a)$ is $n$-connected if and only if each $f(a)$ is $n$-connected. 
\end{lem}

\begin{proof}
By \PMlinkname{Theorem 4.7.6}{47closurepropertiesofequivalences#Thmprethm3}, we have
$\hfib{\total f}{\pairr{x,v}}\eqvsym\hfib{f(x)}v$
for each $x:A$ and $v:Q(x)$. Hence $\trunc n{\hfib{\total f}{\pairr{x,v}}}$ is contractible if and only if
$\trunc n{\hfib{f(x)}v}$ is contractible.
\end{proof}

Another useful fact about connected maps is that they induce an
equivalence on $n$-truncations:

\begin{lem} \label{lem:connected-map-equiv-truncation}
If $f : A \to B$ is $n$-connected, then it induces an equivalence
$\eqv{\trunc{n}{A}}{\trunc{n}{B}}$.
\end{lem}
\begin{proof}
Let $c$ be the proof that $f$ is $n$-connected.  From left to right, we
use the map $\trunc{n}{f} : \trunc{n}{A} \to \trunc{n}{B}$.
To define the map from right to left, by the universal property of
truncations, it suffices to give a map $\mathsf{back} : B \to {\trunc{n}{A}}$.  We can
define this map as follows:
\[
\mathsf{back}(y) \defeq \trunc{n}{\proj{1}}{(\proj{1}{(c(y))})}
\]
By definition, $c(y)$ has type $\iscontr(\trunc n {\hfiber{f}y})$, so its
first component has type $\trunc n{\hfiber{f}y}$, and we can obtain an
element of $\trunc n A$ from this by projection.

Next, we show that the composites are the identity.  In both directions,
because the goal is a path in an $n$-truncated type, it suffices to
cover the case of the constructor $\tprojf{n}$.

In one direction, we must show that for all $x:A$, 
\[
\trunc{n}{\proj{1}}{(\proj{1}{(c(f(x)))})} = \tproj{n}{x}
\]
But $\tproj{n}{(x, \refl{})} : \trunc n{\hfiber{f}y}$, and
$c(y)$ says that this type is contractible, so 
\[
\proj{1}{(c(f(x)))} = \tproj{n}{(x, \refl{})}
\]
Applying $\trunc{n}{\proj{1}}$ to both sides of this equation gives the
result.  

In the other direction, we must show that for all $y:B$, 
\[
\trunc{n}{f}(\trunc{n}{\proj{1}} (\proj{1}{(c(y))})) = \tproj{n}{y}
\]
$\proj{1}{(c(y))}$ has type $\trunc n {\hfiber{f}y}$, and the path we
want is essentially the second component of the $\hfiber{f}y$, but we
need to make sure the truncations work out.  

In general, suppose we are given $p:\trunc{n}{\sm{x:A} B(x)}$ and wish to prove
$P(\trunc{n}{\proj{1}{}}(p))$. By truncation induction, it suffices to
prove $P(\tproj{n}{a})$ for all $a:A$ and $b:B(a)$.  Applying this
principle in this case, it suffices to prove
\[
\trunc{n}{f}(\tproj{n}{a}) = \tproj{n}{y}
\]
given $a:A$ and $b:f (a) = y$.  But the left-hand side equals $\tproj{n}{f (a)}$,
so applying $\tprojf{n}$ to both sides of $b$ gives the result.
\end{proof}

One might guess that this fact characterizes the $n$-connected maps, but in fact being $n$-connected is a bit stronger than this.
For instance, the inclusion $\bfalse:\unit \to\bool$ induces an equivalence on $(-1)$-truncations, but is not surjective (i.e.\ $(-1)$-connected).
In \PMlinkname{\S 8.4}{84fibersequencesandthelongexactsequence} we will see that the difference in general is an analogous extra bit of surjectivity.



\end{document}
