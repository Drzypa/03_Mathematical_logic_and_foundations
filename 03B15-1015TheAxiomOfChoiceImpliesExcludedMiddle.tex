\documentclass[12pt]{article}
\usepackage{pmmeta}
\pmcanonicalname{1015TheAxiomOfChoiceImpliesExcludedMiddle}
\pmcreated{2013-11-06 17:12:29}
\pmmodified{2013-11-06 17:12:29}
\pmowner{PMBookProject}{1000683}
\pmmodifier{PMBookProject}{1000683}
\pmtitle{10.1.5 The axiom of choice implies excluded middle}
\pmrecord{1}{}
\pmprivacy{1}
\pmauthor{PMBookProject}{1000683}
\pmtype{Feature}
\pmclassification{msc}{03B15}

\endmetadata

\usepackage{xspace}
\usepackage{amssyb}
\usepackage{amsmath}
\usepackage{amsfonts}
\usepackage{amsthm}
\makeatletter
\newcommand{\bbrck}[1]{\ttrunc{}{#1}}
\newcommand{\bfalse}{{0_{\bool}}}
\newcommand{\bool}{\ensuremath{\mathbf{2}}\xspace}
\newcommand{\brck}[1]{\trunc{}{#1}}
\newcommand{\btrue}{{1_{\bool}}}
\newcommand{\choice}[1]{\ensuremath{\mathsf{AC}_{#1}}\xspace}
\newcommand{\ct}{  \mathchoice{\mathbin{\raisebox{0.5ex}{$\displaystyle\centerdot$}}}             {\mathbin{\raisebox{0.5ex}{$\centerdot$}}}             {\mathbin{\raisebox{0.25ex}{$\scriptstyle\,\centerdot\,$}}}             {\mathbin{\raisebox{0.1ex}{$\scriptscriptstyle\,\centerdot\,$}}}}
\newcommand{\defeq}{\vcentcolon\equiv}  
\def\@dprd#1{\prod_{(#1)}\,}
\def\@dprd@noparens#1{\prod_{#1}\,}
\def\@dsm#1{\sum_{(#1)}\,}
\def\@dsm@noparens#1{\sum_{#1}\,}
\def\@eatprd\prd{\prd@parens}
\def\@eatsm\sm{\sm@parens}
\newcommand{\epi}{\ensuremath{\twoheadrightarrow}}
\newcommand{\eqv}[2]{\ensuremath{#1 \simeq #2}\xspace}
\newcommand{\eqvsym}{\simeq}    
\newcommand{\hfib}[2]{{\mathsf{fib}}_{#1}(#2)}
\newcommand{\id}[3][]{\ensuremath{#2 =_{#1} #3}\xspace}
\newcommand{\idtypevar}[1]{\ensuremath{\mathsf{Id}_{#1}}\xspace}
\newcommand{\jdeq}{\equiv}      
\newcommand{\LEM}[1]{\ensuremath{\mathsf{LEM}_{#1}}\xspace}
\newcommand{\merid}{\mathsf{merid}}
\newcommand{\narrowequation}[1]{$#1$}
\newcommand{\north}{\mathsf{N}}
\newcommand{\opp}[1]{\mathord{{#1}^{-1}}}
\newcommand{\pairr}[1]{{\mathopen{}(#1)\mathclose{}}}
\def\prd#1{\@ifnextchar\bgroup{\prd@parens{#1}}{\@ifnextchar\sm{\prd@parens{#1}\@eatsm}{\prd@noparens{#1}}}}
\def\prd@noparens#1{\mathchoice{\@dprd@noparens{#1}}{\@tprd{#1}}{\@tprd{#1}}{\@tprd{#1}}}
\def\prd@parens#1{\@ifnextchar\bgroup  {\mathchoice{\@dprd{#1}}{\@tprd{#1}}{\@tprd{#1}}{\@tprd{#1}}\prd@parens}  {\@ifnextchar\sm    {\mathchoice{\@dprd{#1}}{\@tprd{#1}}{\@tprd{#1}}{\@tprd{#1}}\@eatsm}    {\mathchoice{\@dprd{#1}}{\@tprd{#1}}{\@tprd{#1}}{\@tprd{#1}}}}}
\newcommand{\refl}[1]{\ensuremath{\mathsf{refl}_{#1}}\xspace}
\def\sm#1{\@ifnextchar\bgroup{\sm@parens{#1}}{\@ifnextchar\prd{\sm@parens{#1}\@eatprd}{\sm@noparens{#1}}}}
\def\sm@noparens#1{\mathchoice{\@dsm@noparens{#1}}{\@tsm{#1}}{\@tsm{#1}}{\@tsm{#1}}}
\def\sm@parens#1{\@ifnextchar\bgroup  {\mathchoice{\@dsm{#1}}{\@tsm{#1}}{\@tsm{#1}}{\@tsm{#1}}\sm@parens}  {\@ifnextchar\prd    {\mathchoice{\@dsm{#1}}{\@tsm{#1}}{\@tsm{#1}}{\@tsm{#1}}\@eatprd}    {\mathchoice{\@dsm{#1}}{\@tsm{#1}}{\@tsm{#1}}{\@tsm{#1}}}}}
\newcommand{\south}{\mathsf{S}}
\newcommand{\susp}{\Sigma}
\def\@tprd#1{\mathchoice{{\textstyle\prod_{(#1)}}}{\prod_{(#1)}}{\prod_{(#1)}}{\prod_{(#1)}}}
\newcommand{\trans}[2]{\ensuremath{{#1}_{*}\mathopen{}\left({#2}\right)\mathclose{}}\xspace}
\newcommand{\trunc}[2]{\mathopen{}\left\Vert #2\right\Vert_{#1}\mathclose{}}
\def\@tsm#1{\mathchoice{{\textstyle\sum_{(#1)}}}{\sum_{(#1)}}{\sum_{(#1)}}{\sum_{(#1)}}}
\newcommand{\ttrunc}[2]{\bigl\Vert #2\bigr\Vert_{#1}}
\newcommand{\unit}{\ensuremath{\mathbf{1}}\xspace}
\newcommand{\uset}{\ensuremath{\mathcal{S}et}\xspace}
\newcommand{\UU}{\ensuremath{\mathcal{U}}\xspace}
\newcommand{\vcentcolon}{:\!\!}
\newcounter{mathcount}
\setcounter{mathcount}{1}
\newtheorem{precor}{Corollary}
\newenvironment{cor}{\begin{precor}}{\end{precor}\addtocounter{mathcount}{1}}
\renewcommand{\theprecor}{10.1.\arabic{mathcount}}
\newtheorem{prelem}{Lemma}
\newenvironment{lem}{\begin{prelem}}{\end{prelem}\addtocounter{mathcount}{1}}
\renewcommand{\theprelem}{10.1.\arabic{mathcount}}
\newtheorem{prermk}{Remark}
\newenvironment{rmk}{\begin{prermk}}{\end{prermk}\addtocounter{mathcount}{1}}
\renewcommand{\theprermk}{10.1.\arabic{mathcount}}
\newtheorem{prethm}{Theorem}
\newenvironment{thm}{\begin{prethm}}{\end{prethm}\addtocounter{mathcount}{1}}
\renewcommand{\theprethm}{10.1.\arabic{mathcount}}
\let\autoref\cref
\let\hfiber\hfib
\let\type\UU
\makeatother

\begin{document}

% In this section we prove a classic result that the axiom of choice implies excluded
% middle.

We begin with the following lemma.

\begin{lem}\label{prop:trunc_of_prop_is_set}
If $A$ is a mere proposition then its suspension $\susp(A)$ is a set,
and $A$ is equivalent to $\id[\susp(A)]{\north}{\south}$.
\end{lem}

\begin{proof}
To show that $\susp(A)$ is a set, we define a
family $P:\susp(A)\to\susp(A)\to\type$ with the 
property that $P(x,y)$ is a mere proposition for each $x,y:\susp(A)$,
and which is equivalent to its identity type $\idtypevar{\susp(A)}$.
%
We make the following definitions:
\begin{align*}
P(\north,\north) & \defeq \unit &
P(\south,\north) & \defeq A\\
P(\north,\south) & \defeq A &
P(\south,\south) & \defeq \unit.
\end{align*}
We have to check that the definition preserves paths.
Given any $a : A$, there is a meridian $\merid(a) : \north = \south$,
so we should also have
%
\begin{equation*}
  P(\north, \north) = P(\north, \south) = P(\south, \north) = P(\south, \south).
\end{equation*}
%
But since $A$ is inhabited by $a$, it is equivalent to $\unit$, so we have
%
\begin{equation*}
  P(\north, \north) \eqvsym P(\north, \south) \eqvsym P(\south, \north) \eqvsym P(\south, \south).
\end{equation*}
%
The univalence axiom turns these into the desired equalities. Also, $P(x,y)$ is a mere
proposition for all $x, y : \susp(A)$, which is proved by induction on $x$ and $y$, and
using the fact that being a mere proposition is a mere proposition.

Note that $P$ is a reflexive relation.
Therefore we may apply \autoref{thm:h-set-refrel-in-paths-sets}, so it suffices to
construct $\tau : \prd{x,y:\susp(A)}P(x,y)\to(x=y)$. We do this by a double induction.
When $x$ is $\north$, we define $\tau(\north)$ by
%
\begin{equation*}
  \tau(\north,\north,u) \defeq \refl{\north}
  \qquad\text{and}\qquad
  \tau(\north,\south,a) \defeq \merid(a).
\end{equation*}
%
If $A$ is inhabited by $a$ then $\merid(a) : \north = \south$ so we also need
\narrowequation{
  \trans{\merid(a)}{\tau(\north, \north)} = \tau(\north, \south).
}
This we get by function extensionality using the fact that, for all $x : A$,
%
\begin{multline*}
  \trans{\merid(a)}{\tau(\north,\north,x)} =
  \tau(\north,\north,x) \ct \opp{\merid(a)} \jdeq \\
  \refl{\north} \ct \merid(a) =
  \merid(a) =
  \merid(x) \jdeq
  \tau(\north, \south, x).
\end{multline*}
In a symmetric fashion we may define $\tau(\south)$ by
%
\begin{equation*}
  \tau(\south,\north, a) \defeq \opp{\merid(a)}
  \qquad\text{and}\qquad
  \tau(\south,\south, u) \defeq \refl{\south}.
\end{equation*}
%
To complete the construction of $\tau$, we need to check $\trans{\merid(a)}{\tau(\north)} = \tau(\south)$,
given any $a : A$. The verification proceeds much along the same lines by induction on the
second argument of $\tau$.

Thus, by \autoref{thm:h-set-refrel-in-paths-sets} we have that $\susp(A)$ is a set and that $P(x,y) \eqvsym (\id{x}{y})$ for all $x,y:\susp(A)$.
Taking $x\defeq \north$ and $y\defeq \south$ yields $\eqv{A}{(\id[\susp(A)]\north\south)}$ as desired.
\end{proof}

\begin{thm}[Diaconescu]\label{thm:1surj_to_surj_to_pem}
  \index{axiom!of choice}%
  \index{excluded middle}%
  \index{Diaconescu's theorem}\index{theorem!Diaconescu's}%
  The axiom of choice implies the law of excluded middle.
\end{thm}

\begin{proof}
We use the equivalent form of choice given in \autoref{thm:ac-epis-split}.
Consider a mere proposition $A$.
The function $f:\bool\to\susp(A)$ defined by
$f(\bfalse) \defeq \north$ and $f(\btrue) \defeq \south$
is surjective.
Indeed, we have
$\pairr{\bfalse,\refl{\north}} : \hfiber{f}{\north}$
and $\pairr{\btrue,\refl{\south}} :\hfiber{f}{\south}$.
Since $\bbrck{\hfiber{f}{x}}$ is a mere proposition, by induction the claimed surjectivity follows.

By \autoref{prop:trunc_of_prop_is_set} the suspension $\susp(A)$
is a set, so by the axiom of choice there merely exists a
section $g: \susp(A) \to \bool$ of $f$.
As equality on $\bool$ is decidable we get
\begin{equation*}
 (g(f(\bfalse))= g(f(\btrue))) +
 \lnot (g(f(\bfalse))= g(f(\btrue))),
\end{equation*}
and, since $g$ is a section of $f$, hence injective,
\begin{equation*}
(f(\bfalse) = f(\btrue)) +
\lnot (f(\bfalse) = f(\btrue)).
\end{equation*}
Finally, since $(f(\bfalse)=f(\btrue)) = (\north=\south) = A$ by \autoref{prop:trunc_of_prop_is_set}, we have $A+\neg A$.
\end{proof}

% This conclusion needs only \LEM{}, see \autoref{ex:lemnm}.

% \begin{cor}\label{cor:ACtoLEM0}
%   If the axiom of choice \choice{} holds then $\brck{A + \neg A}$ for every set $A$.
% \end{cor}

% \begin{proof}
%   There is a surjection
%   \[
%   A + \neg A \epi \brck{A} + \brck{\neg A} \epi
%   \brck{(\brck{A} + \brck{\neg A})} = \brck{A} \vee \brck{\neg A} = \brck{A} \vee \neg \brck{A} = \unit,
%   \]
%   %
%   where in the last step excluded middle is available as a consequence of the axiom of choice.
%   Again by the axiom of choice there merely exists a section of the surjection, but this
%   is none other than an inhabitant of $A + \neg A$. Therefore $\brck{A+\neg A}$.
% \end{proof}

\index{denial}
\begin{thm}\label{thm:ETCS}
  \index{Elementary Theory of the Category of Sets}%
  \index{category!well-pointed}%
  If the axiom of choice holds then the category $\uset$ is a well-pointed boolean\index{topos!boolean}\index{boolean!topos} elementary topos\index{topos} with choice.  
\end{thm}

\begin{proof}
  Since \choice{} implies \LEM{}, we have a boolean elementary topos with choice by \autoref{thm:settopos} and the remark following it.  We leave the proof of well-pointedness as
an exercise for the reader (\autoref{ex:well-pointed}).
\end{proof}

\begin{rmk}
  The conditions on a category mentioned in the theorem are known as Lawvere's\index{Lawvere}
  axioms for the \emph{Elementary Theory of the Category of Sets}~\cite{lawvere:etcs-long}.
\end{rmk}


\end{document}
