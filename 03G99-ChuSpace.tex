\documentclass[12pt]{article}
\usepackage{pmmeta}
\pmcanonicalname{ChuSpace}
\pmcreated{2013-03-22 13:04:51}
\pmmodified{2013-03-22 13:04:51}
\pmowner{Henry}{455}
\pmmodifier{Henry}{455}
\pmtitle{Chu space}
\pmrecord{6}{33495}
\pmprivacy{1}
\pmauthor{Henry}{455}
\pmtype{Definition}
\pmcomment{trigger rebuild}
\pmclassification{msc}{03G99}
\pmdefines{perp}
\pmdefines{carrier}
\pmdefines{cocarrier}
\pmdefines{normal}
\pmdefines{normal Chu space}
\pmdefines{separable}
\pmdefines{extensional}
\pmdefines{biextensional}
\pmdefines{row}
\pmdefines{column}

% this is the default PlanetMath preamble.  as your knowledge
% of TeX increases, you will probably want to edit this, but
% it should be fine as is for beginners.

% almost certainly you want these
\usepackage{amssymb}
\usepackage{amsmath}
\usepackage{amsfonts}

% used for TeXing text within eps files
%\usepackage{psfrag}
% need this for including graphics (\includegraphics)
%\usepackage{graphicx}
% for neatly defining theorems and propositions
%\usepackage{amsthm}
% making logically defined graphics
%%%\usepackage{xypic}

% there are many more packages, add them here as you need them

% define commands here
%\PMlinkescapeword{theory}
\begin{document}
A \emph{Chu space} over a set $\Sigma$ is a triple $(\mathcal{A},r,\mathcal{X})$ with $r:\mathcal{A}\times\mathcal{X}\rightarrow\Sigma$.  $\mathcal{A}$ is called the \emph{carrier} and $\mathcal{X}$ the \emph{cocarrier}.

Although the definition is symmetrical, in practice asymmetric uses are common.  In particular, often $\mathcal{X}$ is just taken to be a set of function from $\mathcal{A}$ to $\Sigma$, with $r(a,x)=x(a)$ (such a Chu space is called \emph{normal} and is abbreviated $(\mathcal{A},\mathcal{X})$).

We define the \emph{perp} of a Chu space $\mathcal{C}=(\mathcal{A},r,\mathcal{X})$ to be $\mathcal{C}^\perp=(\mathcal{X},r^\smallsmile,\mathcal{A})$ where $r^\smallsmile(x,a)=r(a,x)$.

Define $\hat{r}$ and $\check{r}$ to be functions defining the \emph{rows} and \emph{columns} of $\mathcal{C}$ respectively, so that $\hat{r}(a):\mathcal{X}\rightarrow\Sigma$ and $\check{r}(x):\mathcal{A}\rightarrow\Sigma$ are given by $\hat{r}(a)(x)=\check{r}(x)(a)=r(a,x)$.  Clearly the rows of $\mathcal{C}$ are the columns of $\mathcal{C}^\perp$.

Using these definitions, a Chu space can be represented using a matrix.

If $\hat{r}$ is injective then we call $\mathcal{C}$ \emph{separable} and if $\check{r}$ is injective we call $\mathcal{C}$ \emph{extensional}.  A Chu space which is both separable and extensional is \emph{biextensional}.
%%%%%
%%%%%
\end{document}
