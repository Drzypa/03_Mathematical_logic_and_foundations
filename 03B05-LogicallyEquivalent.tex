\documentclass[12pt]{article}
\usepackage{pmmeta}
\pmcanonicalname{LogicallyEquivalent}
\pmcreated{2013-03-22 13:17:00}
\pmmodified{2013-03-22 13:17:00}
\pmowner{sleske}{997}
\pmmodifier{sleske}{997}
\pmtitle{logically equivalent}
\pmrecord{10}{33769}
\pmprivacy{1}
\pmauthor{sleske}{997}
\pmtype{Definition}
\pmcomment{trigger rebuild}
\pmclassification{msc}{03B05}
\pmsynonym{tautologically equivalent}{LogicallyEquivalent}
\pmsynonym{semantically equivalent}{LogicallyEquivalent}
\pmsynonym{tautological equivalence}{LogicallyEquivalent}
\pmsynonym{semantical equivalence}{LogicallyEquivalent}
\pmsynonym{tautological consequence}{LogicallyEquivalent}
\pmsynonym{semantical consequence}{LogicallyEquivalent}
%\pmkeywords{equivalent}
%\pmkeywords{equivalence}
%\pmkeywords{formal logic}
\pmrelated{Biconditional}
\pmdefines{logical equivalence}
\pmdefines{logical consequence}

\endmetadata

% this is the default PlanetMath preamble.  as your knowledge
% of TeX increases, you will probably want to edit this, but
% it should be fine as is for beginners.

% almost certainly you want these
\usepackage{amssymb}
\usepackage{amsmath}
\usepackage{amsfonts}

% used for TeXing text within eps files
%\usepackage{psfrag}
% need this for including graphics (\includegraphics)
%\usepackage{graphicx}
% for neatly defining theorems and propositions
%\usepackage{amsthm}
% making logically defined graphics
%%%\usepackage{xypic}

% there are many more packages, add them here as you need them

% define commands here
\begin{document}
Two formulas $A$ and $B$ are said to be \emph{logically equivalent} (typically shortened to \emph{equivalent}) when $A$ is true if and only if $B$ is true (that is, $A$ implies $B$ and $B$ implies $A$):
$$\models A\leftrightarrow B.$$
This is sometimes abbreviated as $A \Leftrightarrow B$.

For example, for any integer $z$, the statement ``$z$ is positive'' is equivalent to ``$z$ is not negative and $z\neq 0$''.

More generally, one says that a formula $A$ is a logical consequence of a set $\Gamma$ of formulas, written $$\Gamma \models A$$
if whenever every formula in $\Gamma$ is true, so is $A$.  If $\Gamma$ is a singleton consisting of formula $B$, we also write $$B\models A.$$
Using this, one sees that $$\models A\leftrightarrow B\qquad \mbox{iff} \qquad A\models B\mbox{ and } B\models A.$$

To see this: if $\models A\leftrightarrow B$, then $A\to B$ and $B\to A$ are both true, which means that if $A$ is true so is $B$ and that if $B$ is true so is $A$, or $A\models B$ and $B\models A$.  The argument can be reversed.

\textbf{Remark}.  Some authors call the above notion semantical equivalence or tautological equivalence, rather than logical equivalence.  In their view, logical equivalence is a syntactic notion: $A$ and $B$ are logically equivalent whenever $A$ is deducible from $B$ and $B$ is deducible from $A$ in some deductive system.
%%%%%
%%%%%
\end{document}
