\documentclass[12pt]{article}
\usepackage{pmmeta}
\pmcanonicalname{AdditivelyIndecomposable}
\pmcreated{2013-03-22 13:29:04}
\pmmodified{2013-03-22 13:29:04}
\pmowner{mathcam}{2727}
\pmmodifier{mathcam}{2727}
\pmtitle{additively indecomposable}
\pmrecord{11}{34056}
\pmprivacy{1}
\pmauthor{mathcam}{2727}
\pmtype{Definition}
\pmcomment{trigger rebuild}
\pmclassification{msc}{03F15}
\pmclassification{msc}{03E10}
\pmrelated{OrdinalArithmetic}
\pmdefines{epsilon number}
\pmdefines{epsilon zero}

\endmetadata

\usepackage{amssymb}
\usepackage{amsmath}
\usepackage{amsfonts}
\def\indecomp{\mathbb{H}}
\begin{document}
An ordinal $\alpha$ is called \emph{additively indecomposable} if it is not $0$ and for any $\beta,\gamma<\alpha$, we have $\beta+\gamma<\alpha$.
The set of additively indecomposable ordinals is denoted $\indecomp$.

Obviously $1\in\indecomp$, since $0+0<1$. 
No finite ordinal other than $1$ is in $\indecomp$.
Also, $\omega\in\indecomp$, since the sum of two finite ordinals is still finite.
More generally, every infinite cardinal is in $\indecomp$.

$\indecomp$ is closed and unbounded, so the enumerating function of $\indecomp$ is normal.
In fact, $f_\indecomp(\alpha)=\omega^\alpha$.

The derivative $f_\indecomp^\prime(\alpha)$ is written $\epsilon_\alpha$.
Ordinals of this form (that is, fixed points of $f_\indecomp$) are called \emph{epsilon numbers}.
The number $\epsilon_0=\omega^{\omega^{\omega^{\cdot^{\cdot^\cdot}}}}$ is therefore the first fixed point of the series 
$\omega,\omega^\omega\!,\omega^{\omega^\omega}\!\!,\ldots$
%%%%%
%%%%%
\end{document}
