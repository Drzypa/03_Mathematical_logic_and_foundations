\documentclass[12pt]{article}
\usepackage{pmmeta}
\pmcanonicalname{64CirclesAndSpheres}
\pmcreated{2013-11-18 14:56:53}
\pmmodified{2013-11-18 14:56:53}
\pmowner{PMBookProject}{1000683}
\pmmodifier{rspuzio}{6075}
\pmtitle{6.4 Circles and spheres}
\pmrecord{4}{87686}
\pmprivacy{1}
\pmauthor{PMBookProject}{6075}
\pmtype{Feature}
\pmclassification{msc}{03B15}

\endmetadata

\usepackage{xspace}
\usepackage{amssyb}
\usepackage{amsmath}
\usepackage{amsfonts}
\usepackage{amsthm}
\makeatletter
\newcommand{\apdtwo}[2]{\ensuremath{\apdtwofunc{#1}\mathopen{}\left(#2\right)\mathclose{}}\xspace}
\newcommand{\apdtwofunc}[1]{\ensuremath{\mathsf{apd}^2_{#1}}\xspace}
\newcommand{\aptwo}[2]{\ensuremath{\aptwofunc{#1}\mathopen{}\left({#2}\right)\mathclose{}}\xspace}
\newcommand{\aptwofunc}[1]{\ensuremath{\mathsf{ap}^2_{#1}}\xspace}
\newcommand{\base}{\ensuremath{\mathsf{base}}\xspace}
\newcommand{\ct}{  \mathchoice{\mathbin{\raisebox{0.5ex}{$\displaystyle\centerdot$}}}             {\mathbin{\raisebox{0.5ex}{$\centerdot$}}}             {\mathbin{\raisebox{0.25ex}{$\scriptstyle\,\centerdot\,$}}}             {\mathbin{\raisebox{0.1ex}{$\scriptscriptstyle\,\centerdot\,$}}}}
\newcommand{\defeq}{\vcentcolon\equiv}  
\newcommand{\defid}{\coloneqq}
\newcommand{\dpath}[4]{#3 =^{#1}_{#2} #4}
\def\@dprd#1{\prod_{(#1)}\,}
\def\@dprd@noparens#1{\prod_{#1}\,}
\def\@dsm#1{\sum_{(#1)}\,}
\def\@dsm@noparens#1{\sum_{#1}\,}
\def\@eatprd\prd{\prd@parens}
\def\@eatsm\sm{\sm@parens}
\newcommand{\eqv}[2]{\ensuremath{#1 \simeq #2}\xspace}
\newcommand{\id}[3][]{\ensuremath{#2 =_{#1} #3}\xspace}
\newcommand{\idfunc}[1][]{\ensuremath{\mathsf{id}_{#1}}\xspace}
\newcommand{\indexdef}[1]{\index{#1|defstyle}}   
\newcommand{\indexsee}[2]{\index{#1|see{#2}}}    
\newcommand{\jdeq}{\equiv}      
\newcommand{\lloop}{\ensuremath{\mathsf{loop}}\xspace}
\newcommand{\map}[2]{\ensuremath{{#1}\mathopen{}\left({#2}\right)\mathclose{}}\xspace}
\newcommand{\mapdep}[2]{\ensuremath{\mapdepfunc{#1}\mathopen{}\left(#2\right)\mathclose{}}\xspace}
\newcommand{\mapdepfunc}[1]{\ensuremath{\mathsf{apd}_{#1}}\xspace} 
\newcommand{\opp}[1]{\mathord{{#1}^{-1}}}
\def\prd#1{\@ifnextchar\bgroup{\prd@parens{#1}}{\@ifnextchar\sm{\prd@parens{#1}\@eatsm}{\prd@noparens{#1}}}}
\def\prd@noparens#1{\mathchoice{\@dprd@noparens{#1}}{\@tprd{#1}}{\@tprd{#1}}{\@tprd{#1}}}
\def\prd@parens#1{\@ifnextchar\bgroup  {\mathchoice{\@dprd{#1}}{\@tprd{#1}}{\@tprd{#1}}{\@tprd{#1}}\prd@parens}  {\@ifnextchar\sm    {\mathchoice{\@dprd{#1}}{\@tprd{#1}}{\@tprd{#1}}{\@tprd{#1}}\@eatsm}    {\mathchoice{\@dprd{#1}}{\@tprd{#1}}{\@tprd{#1}}{\@tprd{#1}}}}}
\newcommand{\refl}[1]{\ensuremath{\mathsf{refl}_{#1}}\xspace}
\def\sm#1{\@ifnextchar\bgroup{\sm@parens{#1}}{\@ifnextchar\prd{\sm@parens{#1}\@eatprd}{\sm@noparens{#1}}}}
\def\sm@noparens#1{\mathchoice{\@dsm@noparens{#1}}{\@tsm{#1}}{\@tsm{#1}}{\@tsm{#1}}}
\def\sm@parens#1{\@ifnextchar\bgroup  {\mathchoice{\@dsm{#1}}{\@tsm{#1}}{\@tsm{#1}}{\@tsm{#1}}\sm@parens}  {\@ifnextchar\prd    {\mathchoice{\@dsm{#1}}{\@tsm{#1}}{\@tsm{#1}}{\@tsm{#1}}\@eatprd}    {\mathchoice{\@dsm{#1}}{\@tsm{#1}}{\@tsm{#1}}{\@tsm{#1}}}}}
\newcommand{\Sn}{\mathbb{S}}
\newcommand{\surf}{\ensuremath{\mathsf{surf}}\xspace}
\newcommand{\symlabel}[1]{\refstepcounter{symindex}\label{#1}}
\def\@tprd#1{\mathchoice{{\textstyle\prod_{(#1)}}}{\prod_{(#1)}}{\prod_{(#1)}}{\prod_{(#1)}}}
\newcommand{\trans}[2]{\ensuremath{{#1}_{*}\mathopen{}\left({#2}\right)\mathclose{}}\xspace}
\newcommand{\transfib}[3]{\ensuremath{\mathsf{transport}^{#1}(#2,#3)\xspace}}
\newcommand{\transtwo}[2]{\ensuremath{\mathsf{transport}^2\mathopen{}\left({#1},{#2}\right)\mathclose{}}\xspace}
\def\@tsm#1{\mathchoice{{\textstyle\sum_{(#1)}}}{\sum_{(#1)}}{\sum_{(#1)}}{\sum_{(#1)}}}
\newcommand{\UU}{\ensuremath{\mathcal{U}}\xspace}
\newcommand{\vcentcolon}{:\!\!}
\newcounter{mathcount}
\setcounter{mathcount}{1}
\newtheorem{precor}{Corollary}
\newenvironment{cor}{\begin{precor}}{\end{precor}\addtocounter{mathcount}{1}}
\renewcommand{\theprecor}{6.4.\arabic{mathcount}}
\newtheorem{prelem}{Lemma}
\newenvironment{lem}{\begin{prelem}}{\end{prelem}\addtocounter{mathcount}{1}}
\renewcommand{\theprelem}{6.4.\arabic{mathcount}}
\let\ap\map
\let\apd\mapdep
\let\autoref\cref
\let\type\UU
\makeatother
\def\coloneqq{:=}
\begin{document}
\index{type!circle|(}%
We have already discussed the circle $\Sn^1$ as the higher inductive type generated by
\begin{itemize}
\item A point $\base:\Sn^1$, and
\item A path $\lloop : {\id[\Sn^1]\base\base}$.
\end{itemize}
\index{induction principle!for S1@for $\Sn^1$}%
Its induction principle says that given $P:\Sn^1\to\type$ along with $b:P(\base)$ and $\ell :\dpath P \lloop b b$, we have $f:\prd{x:\Sn^1} P(x)$ with $f(\base)\jdeq b$ and $\apd f \lloop = \ell$.
Its non-dependent recursion principle says that given $B$ with $b:B$ and $\ell:b=b$, we have $f:\Sn^1\to B$ with $f(\base)\jdeq b$ and $\ap f \lloop = \ell$.

We observe that the circle is nontrivial.

\begin{lem}\label{thm:loop-nontrivial}
  $\lloop\neq\refl{\base}$.
\end{lem}
\begin{proof}
  Suppose that $\lloop=\refl{\base}$.
  Then since for any type $A$ with $x:A$ and $p:x=x$, there is a function $f:\Sn^1\to A$ defined by $f(\base)\defeq x$ and $\ap f \lloop \defid p$, we have
  \[p = f(\lloop) = f(\refl{\base}) = \refl{x}.\]
  But this implies that every type is a set, which as we have seen is not the case (see \PMlinkname{Example 3.1.9}{31setsandntypes#Thmpreeg6}).
\end{proof}

The circle also has the following interesting property, which is useful as a source of counterexamples.

\begin{lem}\label{thm:S1-autohtpy}
  There exists an element of $\prd{x:\Sn^1} (x=x)$ which is not equal to $x\mapsto \refl{x}$.
\end{lem}
\begin{proof}
  We define $f:\prd{x:\Sn^1} (x=x)$ by $\Sn^1$-induction.
  When $x$ is $\base$, we let $f(\base)\defeq \lloop$.
  Now when $x$ varies along $\lloop$ (see \PMlinkname{Remark 6.2.4}{62inductionprinciplesanddependentpaths#Thmprermk3}), we must show that $\transfib{x\mapsto x=x}{\lloop}{\lloop} = \lloop$.
  However, in \PMlinkname{\S 2.11}{211identitytype} we observed that $\transfib{x\mapsto x=x}{p}{q} = \opp{p} \ct q \ct p$, so what we have to show is that $\opp{\lloop} \ct \lloop \ct \lloop = \lloop$.
  But this is clear by canceling an inverse.

  To show that $f\neq (x\mapsto \refl{x})$, it suffices by function extensionality to show that $f(\base) \neq \refl{\base}$.
  But $f(\base)=\lloop$, so this is just the previous lemma.
\end{proof}

For instance, this enables us to extend \PMlinkname{Example 3.1.9}{31setsandntypes#Thmpreeg6} by showing that any universe which contains the circle cannot be a 1-type.

\begin{cor}
  If the type $\Sn^1$ belongs to some universe \type, then \type is not a 1-type.
\end{cor}
\begin{proof}
  The type $\Sn^1=\Sn^1$ in \type is, by univalence, equivalent to the type $\eqv{\Sn^1}{\Sn^1}$ of auto\-equivalences of $\Sn^1$, so it suffices to show that $\eqv{\Sn^1}{\Sn^1}$ is not a set.
  \index{automorphism!of S1@of $\Sn^1$}%
  For this, it suffices to show that its equality type $\id[(\eqv{\Sn^1}{\Sn^1})]{\idfunc[\Sn^1]}{\idfunc[\Sn^1]}$ is not a mere proposition.
  Since being an equivalence is a mere proposition, this type is equivalent to $\id[(\Sn^1\to\Sn^1)]{\idfunc[\Sn^1]}{\idfunc[\Sn^1]}$.
  But by function extensionality, this is equivalent to $\prd{x:\Sn^1} (x=x)$, which as we have seen in \PMlinkname{Lemma 6.4.2}{64circlesandspheres#Thmprelem2} contains two unequal elements.
\end{proof}

\index{type!circle|)}%

\index{type!2-sphere|(}%
\indexsee{sphere type}{type, sphere}%
We have also mentioned that the 2-sphere $\Sn^2$ should be the higher inductive type generated by
\symlabel{s2b}
\begin{itemize}
\item A point $\base:\Sn^2$, and
\item A 2-dimensional path $\surf:\refl{\base} = \refl{\base}$ in ${\base=\base}$.
\end{itemize}
\index{recursion principle!for S2@for $\Sn^2$}%
The recursion principle for $\Sn^2$ is not hard: it says that given $B$ with $b:B$ and $s:\refl b = \refl b$, we have $f:\Sn^2\to B$ with $f(\base)\jdeq b$ and $\aptwo f \surf = s$.
Here by ``$\aptwo f \surf$'' we mean an extension of the functorial action of $f$ to two-dimensional paths, which can be stated precisely as follows.

\begin{lem}\label{thm:ap2}
  Given $f:A\to B$ and $x,y:A$ and $p,q:x=y$, and $r:p=q$, we have a path $\aptwo f r : \ap f p = \ap f q$.
\end{lem}
\begin{proof}
  By path induction, we may assume $p\jdeq q$ and $r$ is reflexivity.
  But then we may define $\aptwo f {\refl p} \defeq \refl{\ap f p}$.
\end{proof}

In order to state the general induction principle, we need a version of this lemma for dependent functions, which in turn requires a notion of dependent two-dimensional paths.
As before, there are many ways to define such a thing; one is by way of a two-dimensional version of transport.

\begin{lem}\label{thm:transport2}
  Given $P:A\to\type$ and $x,y:A$ and $p,q:x=y$ and $r:p=q$, for any $u:P(x)$ we have $\transtwo r u : \trans p u = \trans q u$.
\end{lem}
\begin{proof}
  By path induction.
\end{proof}

Now suppose given $x,y:A$ and $p,q:x=y$ and $r:p=q$ and also points $u:P(x)$ and $v:P(y)$ and dependent paths $h:\dpath P p u v$ and $k:\dpath P q u v$.
By our definition of dependent paths, this means $h:\trans p u = v$ and $k:\trans q u = v$.
Thus, it is reasonable to define the type of dependent 2-paths over $r$ to be
\[ (\dpath P r h k )\defeq (h = \transtwo r u \ct k). \]
We can now state the dependent version of \PMlinkname{Lemma 6.4.4}{64circlesandspheres#Thmprelem3}.

\begin{lem}\label{thm:apd2}
  Given $P:A\to\type$ and $x,y:A$ and $p,q:x=y$ and $r:p=q$ and a function $f:\prd{x:A} P(x)$, we have
  $\apdtwo f r : \dpath P r {\apd f p}{\apd f q}$.
\end{lem}
\begin{proof}
  Path induction.
\end{proof}

\index{induction principle!for S2@for $\Sn^2$}%
Now we can state the induction principle for $\Sn^2$: given $P:\Sn^2\to P$ with $b:P(\base)$ and $s:\dpath P \surf {\refl b}{\refl b}$, there is a function $f:\prd{x:\Sn^2} P(x)$ such that $f(\base)\jdeq b$ and $\apdtwo f \surf = s$.

\index{type!2-sphere|)}%

Of course, this explicit approach gets more and more complicated as we go up in dimension.
Thus, if we want to define $n$-spheres for all $n$, we need some more systematic idea.
One approach is to work with $n$-dimensional loops\index{loop!n-@$n$-} directly, rather than general $n$-dimensional paths.\index{path!n-@$n$-}

\index{type!pointed}%
Recall from \PMlinkname{\S 2.1}{21typesarehighergroupoids} the definitions of \emph{pointed types} $\type_*$, and the $n$-fold loop space\index{loop space!iterated} $\Omega^n : \type_* \to \type_*$
(\PMlinkname{Definition 2.1.7}{21typesarehighergroupoids#Thmpredefn1} and \PMlinkname{Definition 2.1.8}{21typesarehighergroupoids#Thmpredefn2}).  Now we can define the
$n$-sphere $\Sn^n$ to be the higher inductive type generated by
\index{type!n-sphere@$n$-sphere}%
\begin{itemize}
\item A point $\base:\Sn^n$, and
\item An $n$-loop $\lloop_n : \Omega^n(\Sn^n,\base)$.
\end{itemize}
In order to write down the induction principle for this presentation, we would need to define a notion of ``dependent $n$-loop\indexdef{loop!dependent n-@dependent $n$-}'', along with the action of dependent functions on $n$-loops.
We leave this to the reader (see \PMlinkexternal{Exercise 6.4}{http://planetmath.org/node/87769}); in the next section we will discuss a different way to define the spheres that is sometimes more tractable.



\end{document}
