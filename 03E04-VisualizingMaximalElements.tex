\documentclass[12pt]{article}
\usepackage{pmmeta}
\pmcanonicalname{VisualizingMaximalElements}
\pmcreated{2013-03-22 15:39:36}
\pmmodified{2013-03-22 15:39:36}
\pmowner{stevecheng}{10074}
\pmmodifier{stevecheng}{10074}
\pmtitle{visualizing maximal elements}
\pmrecord{6}{37593}
\pmprivacy{1}
\pmauthor{stevecheng}{10074}
\pmtype{Example}
\pmcomment{trigger rebuild}
\pmclassification{msc}{03E04}

\usepackage{amssymb}
\usepackage{amsmath}
\usepackage{amsfonts}

\newcommand{\complex}{\mathbb{C}}
\newcommand{\real}{\mathbb{R}}
\begin{document}
Although the mathematical definitions of the terms
``greatest element'' and ``maximal element'' are straightforward enough,
one may easily confuse these two terms when one first encounters them, 
especially since they are not distinguished in the natural language usage
of the words.  Here we give a graphical example that should 
help visualize the concept:

Consider the complex logarithm, which is a multi-valued function on 
$\complex \setminus \{ 0 \}$.  (Or equivalently, the angle function $\theta$ on $\real^2 \setminus \{0\}$.)
If $U$ is any simply-connected open set in $\complex \setminus \{ 0 \}$,
then it is possible to define a single-valued branch $f\colon U \to \complex$ of the complex logarithm.
For example, $U = \complex \setminus \{ x  \colon x \leq 0 \}$.
This is a \emph{maximal set} (with set inclusion as the partial ordering)
for the domain of $f$ : $f$  becomes discontinuous if we attempt to add more points to $U$.
But it is certainly not the \emph{largest} set on which we can define $f$:
the set $U = \complex \setminus \{ iy  \colon y \geq 0\}$ works also.  Indeed, it is clearly seen that there is no such thing
as \emph{the} largest set for the domain of $f$.
And there are obviously infinitely many maximal sets for the domain of $f$.

(Nevertheless, some people will loosely say that there is ``\emph{a} largest domain'' for the single-valued branch of the logarithm.)
%%%%%
%%%%%
\end{document}
