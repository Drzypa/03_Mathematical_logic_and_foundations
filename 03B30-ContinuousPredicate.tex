\documentclass[12pt]{article}
\usepackage{pmmeta}
\pmcanonicalname{ContinuousPredicate}
\pmcreated{2013-11-16 13:45:43}
\pmmodified{2013-11-16 13:45:43}
\pmowner{Jon Awbrey}{15246}
\pmmodifier{Jon Awbrey}{15246}
\pmtitle{continuous predicate}
\pmrecord{7}{40393}
\pmprivacy{1}
\pmauthor{Jon Awbrey}{15246}
\pmtype{Definition}
\pmcomment{trigger rebuild}
\pmclassification{msc}{03B30}
\pmclassification{msc}{03B22}
\pmclassification{msc}{03B15}
\pmclassification{msc}{03A05}
\pmclassification{msc}{00A30}
\pmclassification{msc}{03B42}
\pmsynonym{continuous relation}{ContinuousPredicate}
\pmrelated{HypostaticAbstraction}

% this is the default PlanetMath preamble.  as your knowledge
% of TeX increases, you will probably want to edit this, but
% it should be fine as is for beginners.

% almost certainly you want these
\usepackage{amssymb}
\usepackage{amsmath}
\usepackage{amsfonts}

% used for TeXing text within eps files
%\usepackage{psfrag}
% need this for including graphics (\includegraphics)
%\usepackage{graphicx}
% for neatly defining theorems and propositions
%\usepackage{amsthm}
% making logically defined graphics
%%%\usepackage{xypic}

% there are many more packages, add them here as you need them

% define commands here

\begin{document}
A \textbf{continuous predicate}, as described by Charles Sanders Peirce, is a special type of \PMlinkname{relational}{RelationTheory} predicate that arises as the limit of an iterated process of hypostatic abstraction.

The following is an extended exposition of the concept in Peirce's own words:

\begin{quote}
When we have analyzed a proposition so as to throw into the subject everything that can be removed from the predicate, all that it remains for the predicate to represent is the form of connection between the different subjects as expressed in the propositional \textit{form}.  What I mean by ``everything that can be removed from the predicate" is best explained by giving an example of something not so removable.\\

But first take something removable.  ``Cain kills Abel."  Here the predicate appears as ``--- kills ---."  But we can remove killing from the predicate and make the latter ``--- stands in the relation --- to ---."  Suppose we attempt to remove more from the predicate and put the last into the form ``--- exercises the function of relate of the relation --- to ---" and then putting ``the function of relate to the relation" into another subject leave as predicate ``--- exercises --- in respect to --- to ---."  But this ``exercises" expresses ``exercises the function".  Nay more, it expresses ``exercises the function of relate", so that we find that though we may put this into a separate subject, it continues in the predicate just the same.\\

Stating this in another form, to say that ``$A$ is in the relation $R$ to $B$" is to say that $A$ is in a certain relation to $R$.  Let us separate this out thus:  ``$A$ is in the relation $R^1$ (where $R^1$ is the relation of a relate to the relation of which it is the relate) to $R$ to $B$".  But $A$ is here said to be in a certain relation to the relation $R^1$.  So that we can express the same fact by saying, ``$A$ is in the relation $R^1$ to the relation $R^1$ to the relation $R$ to $B$", and so on \textit{ad infinitum}.\\

A predicate which can thus be analyzed into parts all homogeneous with the whole I call a \textit{continuous predicate}.  It is very important in logical analysis, because a continuous predicate obviously cannot be a \textit{compound} except of continuous predicates, and thus when we have carried analysis so far as to leave only a continuous predicate, we have carried it to its ultimate elements.  (Peirce 1908/1966, pp. 396--397).
\end{quote}

\section{References}

\begin{itemize}\item
Peirce, Charles Sanders (1908/1966), Letter of 14 December 1908, ``Letters to Lady Welby", pp. 380--432 in \textit{Charles S. Peirce : Selected Writings (Values in a Universe of Chance)}, Philip P. Wiener (ed.), Dover Publications, New York, NY, 1966.
\end{itemize}

%%%%%
%%%%%
\end{document}
