\documentclass[12pt]{article}
\usepackage{pmmeta}
\pmcanonicalname{211IdentityType}
\pmcreated{2013-11-14 14:58:34}
\pmmodified{2013-11-14 14:58:34}
\pmowner{PMBookProject}{1000683}
\pmmodifier{rspuzio}{6075}
\pmtitle{2.11 Identity type}
\pmrecord{6}{87617}
\pmprivacy{1}
\pmauthor{PMBookProject}{6075}
\pmtype{Feature}
\pmclassification{msc}{03B15}

\endmetadata

\usepackage{xspace}
\usepackage{amssyb}
\usepackage{amsmath}
\usepackage{amsfonts}
\usepackage{amsthm}
\makeatletter
\newcommand{\blank}{\mathord{\hspace{1pt}\text{--}\hspace{1pt}}}
\newcommand{\ct}{  \mathchoice{\mathbin{\raisebox{0.5ex}{$\displaystyle\centerdot$}}}             {\mathbin{\raisebox{0.5ex}{$\centerdot$}}}             {\mathbin{\raisebox{0.25ex}{$\scriptstyle\,\centerdot\,$}}}             {\mathbin{\raisebox{0.1ex}{$\scriptscriptstyle\,\centerdot\,$}}}}
\def\@dprd#1{\prod_{(#1)}\,}
\def\@dprd@noparens#1{\prod_{#1}\,}
\def\@dsm#1{\sum_{(#1)}\,}
\def\@dsm@noparens#1{\sum_{#1}\,}
\def\@eatprd\prd{\prd@parens}
\def\@eatsm\sm{\sm@parens}
\newcommand{\eqvspaced}[2]{\ensuremath{#1 \;\simeq\; #2}\xspace}
\newcommand{\happly}{\mathsf{happly}}
\newcommand{\id}[3][]{\ensuremath{#2 =_{#1} #3}\xspace}
\newcommand{\mapdepfunc}[1]{\ensuremath{\mathsf{apd}_{#1}}\xspace} 
\newcommand{\mapfunc}[1]{\ensuremath{\mathsf{ap}_{#1}}\xspace} 
\newcommand{\opp}[1]{\mathord{{#1}^{-1}}}
\def\prd#1{\@ifnextchar\bgroup{\prd@parens{#1}}{\@ifnextchar\sm{\prd@parens{#1}\@eatsm}{\prd@noparens{#1}}}}
\def\prd@noparens#1{\mathchoice{\@dprd@noparens{#1}}{\@tprd{#1}}{\@tprd{#1}}{\@tprd{#1}}}
\def\prd@parens#1{\@ifnextchar\bgroup  {\mathchoice{\@dprd{#1}}{\@tprd{#1}}{\@tprd{#1}}{\@tprd{#1}}\prd@parens}  {\@ifnextchar\sm    {\mathchoice{\@dprd{#1}}{\@tprd{#1}}{\@tprd{#1}}{\@tprd{#1}}\@eatsm}    {\mathchoice{\@dprd{#1}}{\@tprd{#1}}{\@tprd{#1}}{\@tprd{#1}}}}}
\newcommand{\proj}[1]{\ensuremath{\mathsf{pr}_{#1}}\xspace}
\newcommand{\projpath}[1]{\ensuremath{\apfunc{\proj{#1}}}\xspace}
\newcommand{\refl}[1]{\ensuremath{\mathsf{refl}_{#1}}\xspace}
\def\sm#1{\@ifnextchar\bgroup{\sm@parens{#1}}{\@ifnextchar\prd{\sm@parens{#1}\@eatprd}{\sm@noparens{#1}}}}
\def\sm@noparens#1{\mathchoice{\@dsm@noparens{#1}}{\@tsm{#1}}{\@tsm{#1}}{\@tsm{#1}}}
\def\sm@parens#1{\@ifnextchar\bgroup  {\mathchoice{\@dsm{#1}}{\@tsm{#1}}{\@tsm{#1}}{\@tsm{#1}}\sm@parens}  {\@ifnextchar\prd    {\mathchoice{\@dsm{#1}}{\@tsm{#1}}{\@tsm{#1}}{\@tsm{#1}}\@eatprd}    {\mathchoice{\@dsm{#1}}{\@tsm{#1}}{\@tsm{#1}}{\@tsm{#1}}}}}
\def\@tprd#1{\mathchoice{{\textstyle\prod_{(#1)}}}{\prod_{(#1)}}{\prod_{(#1)}}{\prod_{(#1)}}}
\newcommand{\transfib}[3]{\ensuremath{\mathsf{transport}^{#1}(#2,#3)\xspace}}
\newcommand{\transfibf}[1]{\ensuremath{\mathsf{transport}^{#1}\xspace}}
\def\@tsm#1{\mathchoice{{\textstyle\sum_{(#1)}}}{\sum_{(#1)}}{\sum_{(#1)}}{\sum_{(#1)}}}
\newcommand{\UU}{\ensuremath{\mathcal{U}}\xspace}
\newcounter{mathcount}
\setcounter{mathcount}{1}
\newtheorem{prelem}{Lemma}
\newenvironment{lem}{\begin{prelem}}{\end{prelem}\addtocounter{mathcount}{1}}
\renewcommand{\theprelem}{2.11.\arabic{mathcount}}
\newtheorem{prethm}{Theorem}
\newenvironment{thm}{\begin{prethm}}{\end{prethm}\addtocounter{mathcount}{1}}
\renewcommand{\theprethm}{2.11.\arabic{mathcount}}
\let\apdfunc\mapdepfunc
\let\apfunc\mapfunc
\let\autoref\cref
\let\type\UU
\makeatother

\begin{document}
\index{type!identity|(}%
Just as the type \id[A]{a}{a'} is characterized up to isomorphism, with
a separate ``definition'' for each $A$, there is no simple
characterization of the type \id[{\id[A]{a}{a'}}]{p}{q} of paths between
paths $p,q : \id[A]{a}{a'}$.
However, our other general classes of theorems do extend to identity types, such as the fact that they respect equivalence.

\begin{thm}\label{thm:paths-respects-equiv}
  If $f : A \to B$ is an equivalence, then for all $a,a':A$, so is
  \[\apfunc{f} : (\id[A]{a}{a'}) \to (\id[B]{f(a)}{f(a')}).\]
\end{thm}
\begin{proof}
  Let $\opp f$ be a quasi-inverse of $f$, with homotopies
  %
  \begin{equation*}
    \alpha:\prd{b:B} (f(\opp f(b))=b)
    \qquad\text{and}\qquad
    \beta:\prd{a:A} (\opp f(f(a)) = a).
  \end{equation*}
  %
  The quasi-inverse of $\apfunc{f}$ is, essentially,
  \[\apfunc{\opp f} : (\id{f(a)}{f(a')}) \to (\id{\opp f(f(a))}{\opp f(f(a'))}).\]
  However, in order to obtain an element of $\id[A]{a}{a'}$ from $\apfunc{\opp f}(q)$, we must concatenate with the paths $\opp{\beta_a}$ and $\beta_{a'}$ on either side.
  To show that this gives a quasi-inverse of $\apfunc{f}$, on one hand we must show that for any $p:\id[A]{a}{a'}$ we have
  \[ \opp{\beta_a} \ct \apfunc{\opp f}(\apfunc{f}(p)) \ct \beta_{a'} = p. \]
  This follows from the functoriality of $\apfunc{}$ and the naturality of homotopies, \PMlinkname{Lemma 2.2.4}{22functionsarefunctors#Thmprelem4},\PMlinkname{Lemma 2.4.3}{24homotopiesandequivalences#Thmprelem2}.
  On the other hand, we must show that for any $q:\id[B]{f(a)}{f(a')}$ we have
  \[ \apfunc{f}\big( \opp{\beta_a} \ct \apfunc{\opp f}(q) \ct \beta_{a'} \big) = q. \]
  The proof of this is a little more involved, but each step is again an application of \PMlinkname{Lemma 2.2.4}{22functionsarefunctors#Thmprelem4},\PMlinkname{Lemma 2.4.3}{24homotopiesandequivalences#Thmprelem2} (or simply canceling inverse paths):
  \begin{align*}
    \apfunc{f}\big( \opp{\beta_a} \ct \apfunc{\opp f}(q) \ct \beta_{a'} \big)
    &= \opp{\alpha_{f(a)}} \ct {\alpha_{f(a)}} \ct
    \apfunc{f}\big( \opp{\beta_a} \ct \apfunc{\opp f}(q) \ct \beta_{a'} \big)
    \ct \opp{\alpha_{f(a')}} \ct {\alpha_{f(a')}}\\
    &= \opp{\alpha_{f(a)}} \ct
    \apfunc f \big(\apfunc{\opp f}\big(\apfunc{f}\big( \opp{\beta_a} \ct \apfunc{\opp f}(q) \ct \beta_{a'} \big)\big)\big)
    \ct {\alpha_{f(a')}}\\
    &= \opp{\alpha_{f(a)}} \ct
    \apfunc f \big(\beta_a \ct \opp{\beta_a} \ct \apfunc{\opp f}(q) \ct \beta_{a'} \ct \opp{\beta_{a'}} \big)
    \ct {\alpha_{f(a')}}\\
    &= \opp{\alpha_{f(a)}} \ct
    \apfunc f (\apfunc{\opp f}(q))
    \ct {\alpha_{f(a')}}\\
    &= q.\qedhere
  \end{align*}
\end{proof}

Thus, if for some type $A$ we have a full characterization of $\id[A]{a}{a'}$, the type $\id[{\id[A]{a}{a'}}]{p}{q}$ is determined as well.  
For example:
\begin{itemize}
\item Paths $p = q$, where $p,q : \id[A \times B]{w}{w'}$, are equivalent to pairs of paths
  \[\id[{\id[A]{\proj{1} w}{\proj{1} w'}}]{\projpath{1}{p}}{\projpath{1}{q}}
  \quad\text{and}\quad
  \id[{\id[B]{\proj{2} w}{\proj{2} w'}}]{\projpath{2}{p}}{\projpath{2}{q}}.
  \]
\item Paths $p = q$, where $p,q : \id[\prd{x:A} B(x)]{f}{g}$, are equivalent to homotopies
  \[\prd{x:A} (\id[f(x)=g(x)] {\happly(p)(x)}{\happly(q)(x)}).\]
\end{itemize}

\index{transport!in identity types}%
Next we consider transport in families of paths, i.e.\ transport in $C:A\to\type$ where each $C(x)$ is an identity type.
The simplest case is when $C(x)$ is a type of paths in $A$ itself, perhaps with one endpoint fixed.

\begin{lem}\label{cor:transport-path-prepost}
  For any $A$ and $a:A$, with $p:x_1=x_2$, we have
  %
  \begin{align*}
    \transfib{x \mapsto (\id{a}{x})} {p} {q} &= q \ct p
    & &\text{for $q:a=x_1$,}\\
    \transfib{x \mapsto (\id{x}{a})} {p} {q} &= \opp {p} \ct q 
    & &\text{for $q:x_1=a$,}\\
    \transfib{x \mapsto (\id{x}{x})} {p} {q} &= \opp{p} \ct q \ct p
    & &\text{for $q:x_1=x_1$.}
  \end{align*}
\end{lem}
\begin{proof}
  Path induction on $p$, followed by the unit laws for composition.
\end{proof}

In other words, transporting with ${x \mapsto \id{c}{x}}$ is post-composition, and transporting with ${x \mapsto \id{x}{c}}$ is contravariant pre-composition.
These may be familiar as the functorial actions of the covariant and contravariant hom-functors $\hom(c, {\blank})$ and $\hom({\blank},c)$ in category theory.

Combining \PMlinkname{Lemma 2.11.2}{211identitytype#Thmprelem1},\PMlinkname{Lemma 2.3.8}{23typefamiliesarefibrations#Thmprelem7}, we obtain a more general form:

\begin{thm}\label{thm:transport-path}
  For $f,g:A\to B$, with $p : \id[A]{a}{a'}$ and $q : \id[B]{f(a)}{g(a)}$, we have
  \begin{equation*}
    \id[f(a') = g(a')]{\transfib{x \mapsto \id[B]{f(x)}{g(x)}}{p}{q}}
    {\opp{(\apfunc{f}{p})} \ct q \ct \apfunc{g}{p}}.
  \end{equation*}
\end{thm}

Because $\apfunc{(x \mapsto x)}$ is the identity function and $\apfunc{(x \mapsto c)}$ (where $c$ is a constant) is \refl{c}, \PMlinkname{Lemma 1}{211identitytype#Thmlem1} is a special case.
A yet more general version is when $B$ can be a family of types indexed on $A$:

\begin{thm}\label{thm:transport-path2}
  Let $B : A \to \type$ and $f,g : \prd{x:A} B(x)$, with $p : \id[A]{a}{a'}$ and $q : \id[B(a)]{f(a)}{g(a)}$.
  Then we have
  \begin{equation*}
    \transfib{x \mapsto \id[B(x)]{f(x)}{g(x)}}{p}{q} = 
    \opp{(\apdfunc{f}(p))} \ct \apfunc{(\transfibf{A}{p})}(q) \ct \apdfunc{g}(p).
  \end{equation*}
\end{thm}

Finally, as in \PMlinkname{\S 2.9}{29pitypesandthefunctionextensionalityaxiom}, for families of identity types there is another equivalent characterization of dependent paths.
\index{path!dependent!in identity types}

\begin{thm}\label{thm:dpath-path}
  For $p:\id[A]a{a'}$ with $q:a=a$ and $r:a'=a'$, we have
  \[ \eqvspaced{ \big(\transfib{x\mapsto (x=x)}{p}{q} = r \big) }{ \big( q \ct p = p \ct r \big). } \]
\end{thm}
\begin{proof}
  Path induction on $p$, followed by the fact that composing with the unit equalities $q\ct 1 = q$ and $r = 1\ct r$ is an equivalence.
\end{proof}

There are more general equivalences involving the application of functions, akin to \PMlinkname{Theorem 2.11.3}{211identitytype#S0.Thmprethm2},\PMlinkname{Theorem 2.11.4}{211identitytype#S0.Thmprethm3}.

\index{type!identity|)}%


\end{document}
