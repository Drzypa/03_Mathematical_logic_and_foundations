\documentclass[12pt]{article}
\usepackage{pmmeta}
\pmcanonicalname{SoleSufficientOperator}
\pmcreated{2013-03-22 17:51:52}
\pmmodified{2013-03-22 17:51:52}
\pmowner{Jon Awbrey}{15246}
\pmmodifier{Jon Awbrey}{15246}
\pmtitle{sole sufficient operator}
\pmrecord{6}{40342}
\pmprivacy{1}
\pmauthor{Jon Awbrey}{15246}
\pmtype{Definition}
\pmcomment{trigger rebuild}
\pmclassification{msc}{03B70}
\pmclassification{msc}{03B35}
\pmclassification{msc}{03B22}
\pmclassification{msc}{03B05}
\pmsynonym{sole sufficient connective}{SoleSufficientOperator}
\pmrelated{Ampheck}
\pmrelated{LogicalConnective}

% this is the default PlanetMath preamble.  as your knowledge
% of TeX increases, you will probably want to edit this, but
% it should be fine as is for beginners.

% almost certainly you want these
\usepackage{amssymb}
\usepackage{amsmath}
\usepackage{amsfonts}

% used for TeXing text within eps files
%\usepackage{psfrag}
% need this for including graphics (\includegraphics)
%\usepackage{graphicx}
% for neatly defining theorems and propositions
%\usepackage{amsthm}
% making logically defined graphics
%%%\usepackage{xypic}

% there are many more packages, add them here as you need them

% define commands here

\begin{document}
A \textbf{sole sufficient operator} or a \textbf{sole sufficient connective} is an operator that is sufficient by itself to define all of the operators in a specified set of operators.

In logical contexts this refers to a logical operator that suffices to define all of the Boolean-valued functions, $f : X \to \mathbb{B}$, where $X$ is an arbitrary set and where $\mathbb{B}$ is a generic 2-element set, typically $\mathbb{B} = \{ 0, 1 \} = \{ \mathrm{false}, \mathrm{true} \}$, in particular, to define all of the finitary Boolean functions, $f : \mathbb{B}^k \to \mathbb{B}$.

%%%%%
%%%%%
\end{document}
