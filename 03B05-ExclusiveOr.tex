\documentclass[12pt]{article}
\usepackage{pmmeta}
\pmcanonicalname{ExclusiveOr}
\pmcreated{2013-03-22 18:04:59}
\pmmodified{2013-03-22 18:04:59}
\pmowner{me_and}{17092}
\pmmodifier{me_and}{17092}
\pmtitle{exclusive or}
\pmrecord{6}{40620}
\pmprivacy{1}
\pmauthor{me_and}{17092}
\pmtype{Definition}
\pmcomment{trigger rebuild}
\pmclassification{msc}{03B05}
\pmsynonym{xor}{ExclusiveOr}
\pmsynonym{logical xor}{ExclusiveOr}
\pmsynonym{logical exclusive or}{ExclusiveOr}
\pmrelated{Disjunction}
\pmrelated{PropositionalLogic}

%\usepackage{amssymb}
%\usepackage{amsmath} %Needed for align & align*, and to correctly render proofs
%\usepackage{amsfonts}
%\usepackage{amsthm}

%Named sets
%\newcommand{\R}{\mathbb{R}} %Real numbers (amssymb or amsfonts)
%\newcommand{\C}{\mathbb{C}} %Complex numbers (amssymb or amsfonts)

%Functions
%\newcommand{\modulus}[1]{\left|{#1}\right|} %|z|
%\newcommand{\integral}[4]{\int_{#1}^{#2}\!{#3}\,\mathrm{d}{#4}}

%Numbers
%\newcommand{\I}{\mathrm{i}} %sqrt{-1}
%\newcommand{\e}{\mathrm{e}} %exponential

%Greek
%\newcommand{\ve}{\varepsilon} %nice epsilon
\begin{document}
\PMlinkescapeword{similarity}
\PMlinkescapeword{normal}

The logical operator xor, short for ``exclusive or'', is a binary operator that is true iff exactly one of its parameters is true, and thus corresponds to the normal use of ``or'' in English. Xor uses the symbol $\oplus$ due to its similarity to disjunction, which sometimes uses $+$. The truth table for xor is

\begin{center}
\begin{tabular}{ccc}
$a$ & $b$ & $a \oplus b$ \\
\hline
F & F & F \\
F & T & T \\
T & F & T \\
T & T & F
\end{tabular}
\end{center}
%%%%%
%%%%%
\end{document}
