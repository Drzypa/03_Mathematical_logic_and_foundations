\documentclass[12pt]{article}
\usepackage{pmmeta}
\pmcanonicalname{GalileosParadox}
\pmcreated{2013-03-22 19:15:45}
\pmmodified{2013-03-22 19:15:45}
\pmowner{pahio}{2872}
\pmmodifier{pahio}{2872}
\pmtitle{Galileo's paradox}
\pmrecord{6}{42192}
\pmprivacy{1}
\pmauthor{pahio}{2872}
\pmtype{Definition}
\pmcomment{trigger rebuild}
\pmclassification{msc}{03E10}
\pmrelated{Paradox}

% this is the default PlanetMath preamble.  as your knowledge
% of TeX increases, you will probably want to edit this, but
% it should be fine as is for beginners.

% almost certainly you want these
\usepackage{amssymb}
\usepackage{amsmath}
\usepackage{amsfonts}

% used for TeXing text within eps files
%\usepackage{psfrag}
% need this for including graphics (\includegraphics)
%\usepackage{graphicx}
% for neatly defining theorems and propositions
 \usepackage{amsthm}
% making logically defined graphics
%%%\usepackage{xypic}

% there are many more packages, add them here as you need them

% define commands here

\theoremstyle{definition}
\newtheorem*{thmplain}{Theorem}

\begin{document}
Galileo Galilei (1564---1642) has realised the ostensible contradiction in the situation, that although the set
$$1,\,2,\,3,\,4,\,5,\,\ldots$$
of the positive integers \PMlinkescapetext{contains} all the members of the set
$$1,\,4,\,9,\,16,\,\ldots$$
of the perfect squares and in \PMlinkescapetext{addition} many others, however both sets are equally great in the sense that any member of the former set has as its square a unique counterpart in the latter set and also any member of the latter set has as its square root a unique counterpart in the former set.\, Galileo explained this by the infinitude of the sets.

In modern mathematical \PMlinkescapetext{expressions}, we say that an infinite set and its proper subset set may have the same cardinality.

%%%%%
%%%%%
\end{document}
