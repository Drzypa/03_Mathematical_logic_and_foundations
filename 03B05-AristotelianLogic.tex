\documentclass[12pt]{article}
\usepackage{pmmeta}
\pmcanonicalname{AristotelianLogic}
\pmcreated{2013-03-22 14:21:32}
\pmmodified{2013-03-22 14:21:32}
\pmowner{Daume}{40}
\pmmodifier{Daume}{40}
\pmtitle{Aristotelian logic}
\pmrecord{9}{35839}
\pmprivacy{1}
\pmauthor{Daume}{40}
\pmtype{Topic}
\pmcomment{trigger rebuild}
\pmclassification{msc}{03B05}
\pmclassification{msc}{01A20}
\pmsynonym{term logic}{AristotelianLogic}

% this is the default PlanetMath preamble.  as your knowledge
% of TeX increases, you will probably want to edit this, but
% it should be fine as is for beginners.

% almost certainly you want these
\usepackage{amssymb}
\usepackage{amsmath}
\usepackage{amsfonts}

% used for TeXing text within eps files
%\usepackage{psfrag}
% need this for including graphics (\includegraphics)
\usepackage{graphicx}
% for neatly defining theorems and propositions
%\usepackage{amsthm}
% making logically defined graphics
%%%\usepackage{xypic} 

% there are many more packages, add them here as you need them

% define commands here

% The below lines should work as the command
% \renewcommand{\bibname}{References}
% without creating havoc when rendering an entry in
% the page-image mode.
%\makeatletter
%\@ifundefined{bibname}{}{\renewcommand{\bibname}{References}}
%\makeatother
\begin{document}
\PMlinkescapeword{analytics}
\PMlinkescapeword{order}
\PMlinkescapeword{mean}
\PMlinkescapeword{categories}
\PMlinkescapeword{interpretation}
\PMlinkescapeword{sections}
\PMlinkescapeword{section}
\PMlinkescapeword{sentence}
\PMlinkescapeword{universal}
\PMlinkescapeword{proposition}
\PMlinkescapeword{propositions}
\PMlinkescapeword{singular}
\PMlinkescapeword{term}
\PMlinkescapeword{terms}
\PMlinkescapeword{type}
\PMlinkescapeword{types}
\PMlinkescapeword{properties}
\PMlinkescapeword{code}
\PMlinkescapeword{relation}
\PMlinkescapeword{opposites}
\PMlinkescapeword{opposite}
\PMlinkescapeword{information}
\PMlinkescapeword{character}
\PMlinkescapeword{states}
\PMlinkescapeword{state}
\PMlinkescapeword{name}
\PMlinkescapeword{minor}
\PMlinkescapeword{contains}
\PMlinkescapeword{contain}
\PMlinkescapeword{divides}
\PMlinkescapeword{simple}
\PMlinkescapeword{sequence}
\PMlinkescapeword{decomposition}
\PMlinkescapeword{succeed}
\PMlinkescapeword{satisfy}
\PMlinkescapeword{sources}
\PMlinkescapeword{perfect}
\PMlinkescapeword{represents}
\PMlinkescapeword{represent}



\subsection*{Introduction}
What we call today Aristotelian logic, Aristotle himself would have labelled analytics.  The term logic he reserved to mean dialectics.  All that remains of his analytics are the Organon.\cite{IEP}  %%Not entirely true in context of history.%%The Organon are the only books of Aristotle that were never lost.  The rest of his work had been lost for over eleven centuries and later found.\cite{WAL}  
Most of Aristotle's work is probably not authentic, since it was most likely edited by students and later lecturers.\cite{BI}  The order in which the books were written is not certain.  But some probable order is given, this order is given on analysis of the texts and styles, and references by Aristotle.\cite{BI}  The logical works of Aristotle were compiled in about 1 B.C.E..  They were compiled in 6 books listed below.  \textit{(There is also one additional volume of Aristotle's concerning logic, it is the called the fourth book of Metaphysics.)}\cite{BI}
\begin{center}
\begin{enumerate}
\item Categories
\item On Interpretation
\item Prior Analytics
\item Posterior Analytics
\item Topics
\item On Sophistical Refutations
\end{enumerate}
\end{center} 
If we introduce Aristotelian logic by stating that the following is an example of a syllogism
\begin{quote}
\begin{center}
All men are mortal,\\
Socrates is a man,\\
therefore Socrates is mortal.
\end{center}
\end{quote}
then we are wrong.\cite{LJ}  The above example is not an example of an Aristotelian Syllogism,  although it is often mistaken for one.  In fact it is a Peripatetic syllogism, a form that was not considered by Aristotle.  A better example for illustrating an Aristotelian Syllogism would be the following 
\begin{quote}
\begin{center}
If all men are mortal\\
and all Greeks are men,\\
then all Greeks are mortal.\cite{LJ}
\end{center}
\end{quote}
The two above examples will be further explained in the following sections.\

\subsection*{Terms}
In Aristotelian logic one subject and one predicate are used in a sentence\textit{(\PMlinkescapetext{proposition})}.  A \emph{\PMlinkescapetext{term}} is either a subject or a predicate.  A subject has a quantity and the subject together with its quantity is known as the \emph{grammatical subject}.  The quantity of the subject is \emph{particular} if we refer to some subset of the set of all subjects.  A subject is \emph{\PMlinkescapetext{universal}} if we refer to the set of all subjects.  For example ``All planets'' is a universal grammatical subject whereas ``Some of the planets'' is a particular grammatical subject.  Predicates however, can only be universal, and we refer to the \emph{grammatical predicate} as the predicate, verb and modifiers.  For example
\begin{center}
$\overbrace{\text{Some}\;\underbrace{\text{Greeks}}_{\text{subject}}}^{\text{particular gram. subject}} \overbrace{\text{are strong}\; \underbrace{\text{men}}_{\text{predicate}}}^{\text{gram. predicate}}$
\end{center}

In Aristotelian logic subjects and predicates must have the possibility of being interchanged.  That is, a subject in one proposition could be a predicate in an other.  This explains why Aristotle excluded what are called individual subjects or singular terms from his logic.

\subsubsection*{Singular Terms}
Some examples of singular terms are `Socrates', `Plato', `Xenocrates'.  Singular terms cannot have a universal quantity, we can not for example say ``All Plato'' \textit{(here we consider `Plato' to be a unique individual)}.  Aristotle presented the following dichotomy ``Some things are universal, others individual.''(\cite{AI} 1:9).  Elaborating: ``By the term `universal' I mean that which is of such a nature as to be predicated of many subjects''(\cite{AI} 1:9) then defining `individual' as ``that which is not thus predicated''(\cite{AI} 1:9).  The inability to deal with individual subjects is considered to be one of the greatest flaws of Aristotelian logic.  Although many have argued that we could universally predicate an individual subject such as `Socrates', Aristotle himself disagreed: `` `Socrates' is not predicable of more than one subject, and therefore we do not say `every Socrates' as we say 'every man'.''(\cite{AM} 5:9).  The following example illustrates the awkwardness of propositions containing singular terms.
\begin{quote}
\begin{center}
All men are mortals,\\
all Socrates are men,\\
therefore all Socrates are mortals.\\
\end{center}
\end{quote}
Although one would be tempted to convert an ungrammatical sentence like ``All Socrates are men'' into the singular proposition ``Socrates is a man'' this type of sentence would never have been considered by Aristotle.  All singular terms and consequently singular propositions are ignored in Aristotelian logic.

\subsection*{Propositions}
An Aristotelian proposition is constructed from two terms; a grammatical subject and a grammatical predicate.  A proposition has two properties: quality and quantity.  The quality of a proposition is either \emph{affirmative} that is, the predicate is affirmed for the subject or \emph{negative} if the predicate is denied for the subject.  Since the quality of a proposition is either affirmative or negative, they are referred to as \emph{affirmative propositions} or \emph{negative propositions}.  The quantity of a proposition is either universal or particular and are consequently called \emph{universal propositions} or \emph{particular propositions}.  Thus there are four types of propositions in Aristotelian logic, they are listed below:
\begin{center}
\begin{tabular}{c l l l}
\textbf{code letter} & \textbf{quantity} & \textbf{quality} & \textbf{example}\\
\hline
$A$ & universal & affirmative & All men are mortal.\\
$E$ & universal & negative & No men are immortal.\\
$I$ & particular & affirmative & Some men are weak.\\
$O$ & particular & negative & Some men are not moral.\\
\end{tabular}
\end{center}
These are the only accepted types of proposition in Aristotelian logic.  They are denoted by the code letters $A$,$E$,$I$,$O$.  These code letters are derived from the vowels of the two Latin words \textit{\textbf{a}ff\textbf{i}rmo} and \textit{n\textbf{e}g\textbf{o}}.\cite{WTL}  They also have the following Latin verse has been used to remember the code letters
\begin{quote}
Asserit A, negat E, verum generaliter ambo;\\
Asserit I, negat O, sed particulariter ambo.\cite{CL}
\end{quote}
Below is a table that compares modern predicate logic propositions to Aristotelian propositions.
\begin{center}
\begin{tabular}{c l l l l}
\textbf{code letter} & \textbf{quantity} & \textbf{quality} & \textbf{proposition} & \textbf{modern notation}\\
\hline
$Asp$ & Universal & Affirmative & All S are P & $\forall x\; (S(x)\to P(x))$\\
$Esp$ & Universal & Negative & No S are P & $\forall x\; (S(x)\to \neg P(x))$\\
$Isp$ & Particular & Affirmative & Some S are P & $\exists x\; (S(x) \wedge P(x))$\\
$Osp$ & Particular & Negative & Some S are not P & $\exists x\; (S(x) \wedge \neg P(x)$
\end{tabular}
\end{center}
The code letters differ slightly from the code letters in the previous table.  Here we specify the subject and predicate represented by $s$ and $p$ respectively. 

\subsubsection*{The Square of Opposition}
The square of opposition is used to analyze the relation between propositions in Aristotelian logic.  Two propositions with the same subject and same predicate are said to be \emph{opposites} if one of or both their quality and quantity are different \textit{(i.e. All men are mortal and Some men are mortal, is an example of opposite propositions)}.  There are different types of opposite propositions: two propositions are said to be \emph{alterns} if they only differ in quantity \textit{(i.e. All S are P and Some S are P)}.  Two universal propositions are said to be \emph{contraries} if they only differ in quality \textit{(i.e. All S are P and No S are P)}.  Two particular propositions are said to be \emph{subcontraries} if they differ in quality \textit{(i.e. Some S are P and Some S are not P)}.  Two propositions are said to be \emph{contradictories} if they differ in quality and quantity \textit{(i.e. No S are P and Some S are P, also All S are P and Some S are not P)}.  The following diagram illustrates the relation between the opposites:
\begin{center}
\includegraphics[scale=1]{trad_squ_opp.eps}
\end{center}
The square itself was not thought up by Aristotle but the information to devise the square comes directly from the writings of Aristotle.  Below is Aristotle's definition of contraries:
\begin{quote}
... a man states a positive and a negative proposition of universal character with regard to a universal, these two propositions are `contrary'. (\cite{AI} 1:7)
\end{quote}
and his definition of contradictory:
\begin{quote}
An affirmation is opposed to a denial in the sense which I denote by the term `contradictory', when, while the subject remains the same, the affirmation is of universal character and the denial is not. (\cite{AI} 1:7)
\end{quote}
The diagram itself is at least as old as the second century C.E. where it is present in the writings of Boethius.\cite{PT}

\subsubsection*{Singular Propositions}
In Aristotelian logic, as noted earlier, there are no singular propositions.  Therefore propositions like `Socrates is a man' would never be present in his logic.  This is why the original example 
\begin{quote}
\begin{center}
All men are mortal,\\
Socrates is a man,\\
therefore Socrates is mortal.
\end{center}
\end{quote}
is not an Aristotelian syllogism.

\subsection*{Syllogisms}
An Aristotelian syllogism is composed of two premises and a conclusion which follows from the two premises in a particular way.  The two premises and conclusion are of the propositional form presented earlier \textit{(i.e. either $A$,$I$,$E$,$O$)}.  Every term was given a name by Aristotle.  The two premises must have a term in common which is called the \emph{middle term}.  The other terms of the premises are called \emph{extreme terms}.   In addition the predicate of the conclusion is called the \emph{major term} while the subject of the conclusion is called the \emph{minor term}.  The premise that contains the minor term is called the \emph{minor premise} and similarly the premise containing the major term is called the \emph{major premise}.\cite{SR}  Furthermore the order of the premises are not important to Aristotle although according to Jan Lukasiewicz ``some queer philosophical prejudices which cannot be explained rationally'' led the commentators of Aristotle to state that the major premise must be first, the minor premise must be second, and the conclusion last.\cite{LJ}  Some argue that the order is important simply to prevent the creation of new forms of arguments that are similar to other arguments.\cite{PH}
Here is an example illustrating the above definitions
\begin{center}
\begin{tabular}{l c}
\textbf{major premise} & All $\overbrace{\text{men}}^{\text{middle term}}$ are $\overbrace{\text{mortals}}^{\text{major term}}$\\
\textbf{minor premise} & and all $\underbrace{\text{Greeks}}_{\text{minor term}}$ are $\underbrace{\text{men}}_{\text{middle term}}$\\
\textbf{conclusion} & then all Greeks are mortals.\\
\end{tabular}
\end{center}

Aristotle divides syllogisms into two categories a \emph{perfect syllogism} ``which needs nothing other than what has been stated to make plain what necessarily follows''(\cite{AP} 1:1) and \emph{imperfect syllogism} which ``needs either one or more propositions, which are indeed the necessary consequences of the terms set down, but have not been expressly stated as premisses.''(\cite{AP} 1:1).  The \emph{perfect syllogisms} are not provable and are taken as axioms in Aristotelian logic.\cite{LJ}  Aristotle uses these axioms to prove the imperfect syllogisms with the use of conversion.

\subsubsection*{Conversion}
A \emph{conversion} is the process of changing a proposition by reversing the subject and predicate while maintaining the same quality.  There are two types of conversions: \emph{simple conversions} where the quantity of the proposition is kept unchanged and \emph{conversion per accidens} where the quantity of the proposition is changed from being universal to being particular.  It is important to note that not all conversions are valid and that some conversions do not exist.  Below is a table of all conversions of Aristotelian propositions which have a conversion.\cite{PH}
\begin{center}
\begin{tabular}{l l|l l|l l}
\multicolumn{2}{c}{\textbf{original proposition}} & \multicolumn{2}{c}{\textbf{simple conversion}} & \multicolumn{2}{c}{\textbf{per accidens conversion}}\\
\hline
All S are P & $Asp$ & not valid & & Some P are S & $Ips$\\
No S are P & $Esp$ & No P are S & $Eps$ & Some P are not S & $Ops$\\
Some S are P & $Isp$ & Some P are S & $Ips$ & does not exist & \\
Some S are not P & $Osp$ & not valid & & does not exist &
\end{tabular}
\end{center}
We see that Aristotle accepts the conversion per accidens of $Asp$ to $Ips$ but not the simple conversion of $Asp$ to $Aps$ stating ``the terms of the affirmative must be convertible, not however, universally, but in part''(\cite{AP} 1:2) and gives the following example of this accepted conversion: ``if every pleasure,is good, some good must be pleasure''(\cite{AP} 1:2).  He also accepts the conversion of $Esp$ to $Eps$ and also the per accidens conversion $Esp$ to $Osp$.  Below is Aristotle's proof of the simple conversion of $Esp$ to $Eps$:
\begin{quote}
If no B is A, neither can any A be B. For if some A (say C) were B, it would not be true that no B is A; for C is a B. But if every B is A then some A is B. For if no A were B, then no B could be A. But we assumed that every B is A. (\cite{AP} 1:2)
\end{quote}
He also proves the simple conversion $Isp$ to $Ips$ and also states various counter examples for non valid conversions.  He presents a counter example for the simple conversion of $Osp$ to $Ops$: ``but the particular negative need not convert, for if some animal is not man, it does not follow that some man is not animal.''(\cite{AP} 1:2)

It is important to note that for the conversion per accidens to be valid there must be a presupposition.  The presupposition being that there exists at least one subject in the universe of discourse.  This presupposition is the simplest one that keeps the Aristotelian logic valid.  Many controversies and attacks on Aristotelian logic have been based on this problem of having a possible empty universe of discourse. 

\subsubsection*{The Four Figures}
The position of the middle term \textit{(the term which appears in both the major and minor premise)} gives rise to what are called \emph{figures}.  The figures represent the possible placement of the middle term in a syllogism.  The \emph{first figure} is based on what Aristotle thought was the deduction which was closest to natural reasoning.  So the first figure will only generate perfect syllogisms.  The figures below are illustrated using $M$ as the middle term, $P$ as the major term, and as $S$ the minor term.  It is clear that there are only 4 such figures. 
\begin{center}
\begin{tabular}{l| c c | c c | c c | c c}
& \multicolumn{2}{c}{\textbf{first figure}} & \multicolumn{2}{c}{\textbf{second figure}} & \multicolumn{2}{c}{\textbf{third figure}} & \multicolumn{2}{c}{\textbf{fourth figure}}\\
\hline
\textbf{major premise} & $M$ & $P$ & $P$ & $M$ & $M$ & $P$  & $P$ & $M$\\
\textbf{minor premise} & $S$ & $M$ & $S$ & $M$ & $M$ & $S$  & $M$ & $S$\\
\textbf{conclusion} & $S$ & $P$ & $S$ & $P$ & $S$ & $P$  & $S$ & $P$\\
\end{tabular}
\end{center}
Note that the fourth figure was not explicitly stated in Aristotle's work.  Although he accepted the reasoning which can be derived from that figure.  The fourth figure is sometimes called the Galenian figure as it was possibly first used or discovered by Galen(131-201 C.E.) although some contest its discovery.\cite{LJ}  Carl Prantl rejects the fourth figure since it does not appear directly in Aristotle's work, however fails to recognize that Aristotle accepted its derived results.\cite{LJ}    

\subsubsection{Syllogistic Moods}
The \emph{mood} of a syllogism is a sequence of propositions and conclusions.  The figures associated with a mood make a syllogism.  Below is a table of the \textbf{valid} syllogisms with their associated moods and figures.
\begin{center}
\begin{tabular}{c c c c}
\textbf{first figure} & \textbf{second figure} & \textbf{third figure} &\textbf{fourth figure}\\
\hline
$AAA$ & $AEE$ & $AAI$ & $AAI$\\
$EAE$ & $EAE$ & $IAI$ & $AEE$\\
$AII$ & $EIO$ & $AII$ & $IAI$\\
$EIO$ & $AOO$ & $EAO$ & $EAO$\\
& & $OAO$ & $EIO$\\ 
& & $EIO$ &
\end{tabular}
\end{center}

The table below show the valid syllogistic moods derived from the first figure.  As stated above these syllogisms are perfect.
\begin{center}
\begin{tabular}{l l}
\textbf{form} & \textbf{mnemonic}\\
\hline
$Amp \;\&\; Asm\;\therefore Asp$ & Barbara\\
$Emp \;\&\; Asm\;\therefore Esp$ & Celarent\\
$Amp \;\&\; Ism\;\therefore Isp$ & Darii\\
$Emp \;\&\; Ism\;\therefore Osp$ & Ferio
\end{tabular}
\end{center}
Although all the above are perfect syllogisms, for Aristotle, the most clear syllogisms are Barbara and Celarent.  In his later work he deduced the Darii, and Ferio syllogisms as they were less natural to him.\cite{LJ} The mnemonics are used to remember the valid moods.  Take the classical perfect syllogism `Barbara', the important information in the mnemonic are the vowels.  `B\textbf{a}rb\textbf{a}r\textbf{a}' represents the mood $AAA$.  In addition to observing the vowels one must know ahead of time that `Barbara' is a mood of the first figure.  We now have all the information to deduce that the syllogism has the following form: $Amp \;\&\; Asm\;\therefore Asp$.  This syllogism can be written as
\begin{center}
All $\textbf{M}$ are $\textbf{P}$\\
and all $\textbf{S}$ are $\textbf{M}$\\
then all $\textbf{S}$ are $\textbf{P}$.
\end{center}
Where the placement of the $M$'s, $P$'s, and $S$'s correspond to the first figure \textit{(shown below)}.
\begin{center}
\begin{tabular}{l| c c}
major premise & $\textbf{M}$ & $\textbf{P}$\\
minor premise & $\textbf{S}$ & $\textbf{M}$\\
conclusion & $\textbf{S}$ & $\textbf{P}$
\end{tabular}
\end{center}

To remember in which figure each mood is derived from, medieval logicians invented verses to remember them.  For example this verse:
\begin{quote}
Barbara, Celarent, Darii, Ferio-que prioris.\\
Cesare, Camestres, Festino, Baroko, secundae.\\
Tertia Darapti, Disamis, Datisi, Felapton,\\
Bocardo, Ferison habet. Quarta insuper addit\\
Bramantip, Camenes, Dimaris, Fesapo, Fresison.
\end{quote}
The key words in the above verse are \textit{prioris} which refers to moods of the first figure, \textit{secundae} to those of second figure, \textit{tertia} to those of the third figure and finally \textit{quarta} to those of the fourth figure.

The moods of the second figure are listed below, they are all imperfect syllogisms.
\begin{center}
\begin{tabular}{l l}
\textbf{form} & \textbf{mnemonic}\\
\hline
$Epm \;\&\; Asm\;\therefore Esp$ & Cesare\\
$Apm \;\&\; Esm\;\therefore Esp$ & Camestres\\
$Epm \;\&\; Ism\;\therefore Osp$ & Festimo\\
$Apm \;\&\; Osm\;\therefore Osp$ & Baroco
\end{tabular}
\end{center}
Since the above syllogisms are imperfect, their  validity must be proven.  For example Aristotle proves the validity of the syllogism Cesare:
\begin{quote}
Let M be predicated of no N, but of all O. Since, then, the negative relation is convertible, N will belong to no M: but M was assumed to belong to all O: consequently N will belong to no O. This has already been proved.(\cite{AP} 1:5)
\end{quote}
Therefore Cesare is a valid mood, here is a decomposition of Aristotle's proof:\\
\begin{quote}
\begin{tabular}{l l l l l}
1. & Let M be predicated of no N, & No N are M & $Enm$ & premise\\
2. & but of all O. & All O are M & $Aom$ & premise\\
3. & N will belong to no M & No M are N & $Emn$ & since the simple conversion from $Esp$ to $Eps$ is valid\\
4. & consequently N will belong to no O. & No O are N & $Eon$ & using line 2 and 3 we reduce to the syllogism Celarent which is an axiom\\
\end{tabular}
\end{quote}
The process illustrated above is called \emph{reduction} where premises and conclusions of a certain argument are converted to a first figure syllogism to conclude that the argument is valid.

The moods of the third figure are listed below, noting that they are also all imperfect
\begin{center}
\begin{tabular}{l l}
\textbf{form} & \textbf{mnemonic}\\
\hline
$Amp \;\&\; Ams\;\therefore Isp$ & Darapti\\
$Emp \;\&\; Ams\;\therefore Osp$ & Felapton\\
$Imp \;\&\; Ams\;\therefore Isp$ & Disamis\\
$Amp \;\&\; Ims\;\therefore Isp$ & Datisi\\
$Omp \;\&\; Ams\;\therefore Osp$ & Bocardo\\
$Emp \;\&\; Ims\;\therefore Osp$ & Ferison
\end{tabular}
\end{center}
The mnemonics contain more information than simply the mood.  The mnemonic of the second, third and fourth figures have first letter: `B', `C', `D' or `F' this indicates to which first figure mood the syllogism is reduced to, to prove its validity \textit{(for example Felapton will be reduced to Ferio)}.  In addition  some of the letters that succeed vowels in the mnemonic have a particular meaning.  If a `c' follows an `o' it indicates that the proof must be done \emph{reductio ad impossibile}.\textit{(There are two such syllogisms, Bocardo and Baroco which are not correctly proven by Aristotle.  To see the reason why consult Aristotle's Syllogistic, From the Standpoint of Modern Formal Logic by Jan Lukasiewicz, section 17.)}  If an `s'\textit{(or `p')} follows the first or second vowel, this indicates that the proposition corresponding to this vowel is simply\textit{(or per accidens)} converted in the proof.  If an `s'\textit{(or `p')} follows the last vowel this implies that the conclusion  will be derived by making a simple\textit{(or per accidens)} conversion of the conclusion of the first figure syllogism that is used in the reduction.  Finally an `m' indicates a rearrangement of the premises to satisfy the order: major premise, minor premise and conclusion.\cite{PH}

The moods of the fourth figure are listed below, these are all imperfect
\begin{center}
\begin{tabular}{l l}
\textbf{form} & \textbf{mnemonic}\\
\hline
$Apm \;\&\; Ams\;\therefore Isp$ & Bramantip\\
$Apm \;\&\; Ems\;\therefore Esp$ & Camenes\\
$Ipm \;\&\; Ams\;\therefore Isp$ & Dimaris\\
$Epm \;\&\; Ams\;\therefore Osp$ & Fesapo\\
$Epm \;\&\; Ims\;\therefore Osp$ & Fresison
\end{tabular}
\end{center}
The proof of the validity of all imperfect syllogisms can now be given by using the rules associated to the mnemonics.  This will be illustrated using the syllogism associated to the mnemonic `Dimaris'.  It is known to begin with, that `Dimaris' is a fourth figure syllogism.  Its vowels are $IAI$ thus the following syllogism will be proven: $Ipm \;\&\; Ams\;\therefore Isp$.  Since `Dimaris' begins with the letter `D' a reduction to the first figure syllogism `Darii' will be used.  The `s' following the last vowel indicates that a simple conversion in the conclusion of the syllogism `Darii' will be necessary.  In addition the `m' indicates a reordering of the premises.  The details of the proof are given below
\begin{quote}
\begin{tabular}{l l l l}
1. & Some P are M & $Ipm$ & premise\\
2. & All M are S & $Ams$ & premise\\
3. & Some P are M & $Ipm$ & repetition of line 1 to reorder the premises\\
4. & Some P are S & $Ips$ & by line 2 and 3 and the axiom `Darii'\\
5. & Some S are P & $Isp$ & by simple conversion of line 4
\end{tabular}
\end{quote} 


\section*{Conclusion}
The above example brings together the various concepts presented in this discourse.  These include figures, moods, conversion and reduction.  Presented were the mnemonic techniques that have been developed to better master Aristotle's work.  His logic is riddled with subtleties and rules which today seem restricting.  We tried to present a strict account of Aristotelian logic as summarized by later contributors.  Although, accounts of his original work differ from various commentators.  What is repeated by all sources is the recognition of Aristotle's work whether for historic or scientific value.

\section*{See also}
\begin{itemize}
\item Stanford Encyclopedia of Philosophy: \PMlinkexternal{Traditional Square of Opposition}{http://plato.stanford.edu/entries/square/}
\item Stanford Encyclopedia of Philosophy: \PMlinkexternal{Aristotle's Logic}{http://plato.stanford.edu/entries/aristotle-logic/}
\item Wikipedia: \PMlinkexternal{Aristotelian Logic}{http://en.wikipedia.org/Aristotelian\_logic}
\item Wikipedia: \PMlinkexternal{Term Logic}{http://en.wikipedia.org/Term\_Logic}
\item The works of Aristotle can be obtained at \PMlinkexternal{http://classics.mit.edu}{http://classics.mit.edu}
\end{itemize}

\section*{Recommended reading}
\begin{itemize}
\item \L ukasiewicz, Jan: Aristotle's Syllogistic, From the Standpoint of Modern Formal Logic. Clarendon Press. Oxford, 1951.
\end{itemize}

%=======================================================[[Bibliography begins here]]
\begin{thebibliography}{99}
\bibitem[AI]{AI} Aristotle: On Interpretation. Translated by E. M. Edghill, 350 BCE.
\bibitem[AM]{AM} Aristotle: Metaphysics. Translated by W. D. Ross, 350 BCE.
\bibitem[AP]{AP} Aristotle: Prior Analytics. Translated by A.J. Jenkinson, 350 BCE.
\bibitem[BI]{BI} Boche\'nski, I. M.: Ancient Formal Logic. North-Holland Publishing Company, Amsterdam, 1951.
\bibitem[CL]{CL} Couturat, Louis: La Logique de Leibniz.  Georg Olms Verlagsbuchhandlung, Hildesheim, 1961.
\bibitem[IEP]{IEP} The Internet Encyclopedia of Philosophy: Aristotle. [online] \PMlinkexternal{http://www.utm.edu/research/iep/a/aristotl.htm}{http://www.utm.edu/research/iep/a/aristotl.htm} , 2004.
\bibitem[LJ]{LJ} Lukasiewicz, Jan: Aristotle's Syllogistic, From the Standpoint of Modern Formal Logic. Clarendon Press. Oxford, 1951.
\bibitem[PH]{PH} Parry, William T., Hacker, Edward A.: Aristotelian Logic.  State University of New York Press. Albany, 1991.
\bibitem[PT]{PT} Parsons, Terence: Stanford Encyclopedia of Philosophy: Traditional Square of Opposition. [online] \PMlinkexternal{http://plato.stanford.edu/entries/square/}{http://plato.stanford.edu/entries/square/}
\bibitem[RL]{RL} Rose, Lynn E.: Aristotle's Syllogistic.  Charles C Thomas Publisher, Springfield, 1968. 
\bibitem[SR]{SR} Smith, Robin: Stanford Encyclopedia of Philosophy:  Aristotle's Logic. [online] \PMlinkexternal{http://plato.stanford.edu/entries/aristotle-logic/}{http://plato.stanford.edu/entries/aristotle-logic/}
\bibitem[WA]{WA} Wikipedia: Aristotle. [online] \PMlinkexternal{http://en.wikipedia.org/Aristotle}{http://en.wikipedia.org/Aristotle} , 2004.
\bibitem[WAL]{WAL} Wikipedia: Aristotelian Logic. [online] \PMlinkexternal{http://en.wikipedia.org/Aristotelian_logic}{http://en.wikipedia.org/Aristotelian\_logic} , 2004.
\bibitem[WTL]{WTL} Wikipedia: Term Logic. [online] \PMlinkexternal{http://en.wikipedia.org/Term_Logic}{http://en.wikipedia.org/Term\_Logic} , 2004.
\end{thebibliography}
%%%%%
%%%%%
\end{document}
