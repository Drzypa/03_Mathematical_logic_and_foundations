\documentclass[12pt]{article}
\usepackage{pmmeta}
\pmcanonicalname{HerbrandStructure}
\pmcreated{2013-03-22 14:33:13}
\pmmodified{2013-03-22 14:33:13}
\pmowner{iwnbap}{1760}
\pmmodifier{iwnbap}{1760}
\pmtitle{Herbrand structure}
\pmrecord{4}{36105}
\pmprivacy{1}
\pmauthor{iwnbap}{1760}
\pmtype{Definition}
\pmcomment{trigger rebuild}
\pmclassification{msc}{03B10}
\pmsynonym{HerbrandModel HerbrandInterpretation}{HerbrandStructure}
\pmrelated{HerbrandsTheoremFirstOrderLogic}
\pmdefines{HerbrandModel HerbrandInterpretation HerbrandUniverse}

% this is the default PlanetMath preamble.  as your knowledge
% of TeX increases, you will probably want to edit this, but
% it should be fine as is for beginners.

% almost certainly you want these
\usepackage{amssymb}
\usepackage{amsmath}
\usepackage{amsfonts}

% used for TeXing text within eps files
%\usepackage{psfrag}
% need this for including graphics (\includegraphics)
%\usepackage{graphicx}
% for neatly defining theorems and propositions
%\usepackage{amsthm}
% making logically defined graphics
%%%\usepackage{xypic}

% there are many more packages, add them here as you need them

% define commands here
\begin{document}
For a language $\mathcal{L}$, define the \emph{Herbrand universe} to be the set of closed terms (alternatively ground terms) of $L$.  

A structure $\mathfrak{M}$ for $\mathcal{L}$ is a \emph{Herbrand structure} if the domain of $\mathfrak{M}$ is the Herbrand universe of $\mathcal{L}$. This fixes the domain of $\mathfrak{M}$, and so each Herbrand structure can be identified with its interpretation, leading to the alternative nomenclature of \emph{Herbrand interpretation}.

A \emph{Herbrand model} of a theory $T$ is a Herbrand structure which is a model of $T$.
%%%%%
%%%%%
\end{document}
