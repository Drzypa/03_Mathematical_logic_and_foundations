\documentclass[12pt]{article}
\usepackage{pmmeta}
\pmcanonicalname{ClassesOfOrdinalsAndEnumeratingFunctions}
\pmcreated{2013-03-22 13:28:55}
\pmmodified{2013-03-22 13:28:55}
\pmowner{mathcam}{2727}
\pmmodifier{mathcam}{2727}
\pmtitle{classes of ordinals and enumerating functions}
\pmrecord{14}{34053}
\pmprivacy{1}
\pmauthor{mathcam}{2727}
\pmtype{Definition}
\pmcomment{trigger rebuild}
\pmclassification{msc}{03F15}
\pmclassification{msc}{03E10}
\pmdefines{order type}
\pmdefines{enumerating function}
\pmdefines{closed}
\pmdefines{kappa-closed}
\pmdefines{continuous}
\pmdefines{kappa-continuous}
\pmdefines{continuous function}
\pmdefines{kappa-continuous function}
\pmdefines{closed class}
\pmdefines{kappa-closed class}
\pmdefines{normal function}
\pmdefines{kappa-normal function}
\pmdefines{normal}
\pmdefines{kappa-normal}
\pmdefines{unbounded}
\pmdefines{unbounded clas}

\usepackage{amssymb}
\usepackage{amsmath}
\usepackage{amsfonts}

\def\dom{\operatorname{dom}}
\begin{document}
\PMlinkescapeword{class}
\PMlinkescapeword{order}

A \emph{class of ordinals} is just a subclass of the \PMlinkname{class}{Class} $\mathbf{On}$ of all ordinals. For every class of ordinals $M$ there is an \emph{enumerating function} $f_M$ defined by transfinite recursion:
$$f_M(\alpha)=\min\{x\in M\mid f(\beta)<x\text{ for all }\beta<\alpha\},$$
and we define the \emph{order type} of $M$ by $\operatorname{otype}(M)=\dom(f)$.  The possible values for this value are either $\mathbf{On}$ or some ordinal $\alpha$. The above function simply lists the elements of $M$ in order. Note that it is not necessarily defined for all ordinals, although it is defined for a segment of the ordinals. If $\alpha<\beta$ then $f_M(\alpha)<f_M(\beta)$, so $f_M$ is an order isomorphism between $\operatorname{otype}(M)$ and $M$.

For an ordinal $\kappa$, we say $M$ is $\kappa$-\emph{closed} if for any $N\subseteq M$ such that $|N|<\kappa$, also $\sup N\in M$.

We say $M$ is \emph{$\kappa$-unbounded} if for any $\alpha<\kappa$ there is some $\beta\in M$ such that $\alpha<\beta$.

We say a function $f\colon M\rightarrow\mathbf{On}$ is $\kappa$-\emph{continuous} if $M$ is $\kappa$-closed and
$$f(\sup N)=\sup \{f(\alpha)\mid \alpha\in N\}$$

A function is \emph{$\kappa$-normal} if it is order preserving ($\alpha<\beta$ implies $f(\alpha)<f(\beta)$) and continuous. In particular, the enumerating function of a $\kappa$-closed class is always $\kappa$-normal.

All these definitions can be easily extended to all ordinals: a class is \emph{closed} (resp. \emph{unbounded}) if it is $\kappa$-closed (unbounded) for all $\kappa$. A function is \emph{continuous} (resp. \emph{normal}) if it is $\kappa$-continuous (normal) for all $\kappa$.
%%%%%
%%%%%
\end{document}
