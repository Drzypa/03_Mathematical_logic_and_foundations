\documentclass[12pt]{article}
\usepackage{pmmeta}
\pmcanonicalname{ExamplesOfCountableSets}
\pmcreated{2013-03-22 19:02:59}
\pmmodified{2013-03-22 19:02:59}
\pmowner{CWoo}{3771}
\pmmodifier{CWoo}{3771}
\pmtitle{examples of countable sets}
\pmrecord{10}{41928}
\pmprivacy{1}
\pmauthor{CWoo}{3771}
\pmtype{Example}
\pmcomment{trigger rebuild}
\pmclassification{msc}{03E10}
\pmrelated{AlgebraicNumbersAreCountable}

\endmetadata

\usepackage{amssymb,amscd}
\usepackage{amsmath}
\usepackage{amsfonts}
\usepackage{mathrsfs}

% used for TeXing text within eps files
%\usepackage{psfrag}
% need this for including graphics (\includegraphics)
%\usepackage{graphicx}
% for neatly defining theorems and propositions
\usepackage{amsthm}
% making logically defined graphics
%%\usepackage{xypic}
\usepackage{pst-plot}

% define commands here
\newcommand*{\abs}[1]{\left\lvert #1\right\rvert}
\newtheorem{prop}{Proposition}
\newtheorem{thm}{Theorem}
\newtheorem{ex}{Example}
\newcommand{\real}{\mathbb{R}}
\newcommand{\pdiff}[2]{\frac{\partial #1}{\partial #2}}
\newcommand{\mpdiff}[3]{\frac{\partial^#1 #2}{\partial #3^#1}}
\begin{document}
This entry lists some common examples of countable sets.

\textbf{Derived Examples}
\begin{enumerate}
\item any finite set, including the empty set $\varnothing$ (\PMlinkname{proof}{AlternativeDefinitionsOfCountable}).
\item any subset of a countable set (\PMlinkname{proof}{SubsetsOfCountableSets}).
\item any finite product of countable sets (\PMlinkname{proof}{UnionOfCountableSets}).
\item any countable union of countable sets (\PMlinkname{proof}{ProductOfCountableSets}).
\item the set of all finite subsets of a countable set.
\begin{proof}  Let $A$ be a countable set, and $F(A)$ the set of all finite subsets of $A$.  Let $A_n$ be the set of all subsets of $A$ of cardinality at most $n$.  Then $A_1$ is countable, since $A$ is.  Suppose now that $A_n$ is countable.  The function $f: A_n \times A_1 \to A_{n+1}$ where $f(X,Y)=X\cup Y$ is easily seen to be onto.  Since $A_n\times A_1$ is countable, so is $A_{n+1}$.  Now, $F(A)$ is just the union of all the countable sets $A_i$, and this union is a countable union, we see that $F(A)$ is countable too.
\end{proof}
\item the set of all cofinite subsets of a countable set.  This is true, because there is a one-to-one correspondence between the set $F(A)$ of finite sets and the set $\operatorname{co-}F(A)$ of cofinite sets: $X\mapsto A-X$.
\item the set of all finite sequences over a countable set.
\begin{proof}
Let $A$ be a countable set, and $A_F$ the set of all finite sequences over $A$.  An element of $A_F$ can be identified with an element of $A^n$, and vice versa (the bijection is clear).  Therefore, $A_F$ can be identified with the union of $A^i$, for $i=0,1,2,\ldots$.  Since each $A^i$ is countable (because $A$ is), and we are taking a countable union, $A_F$ is countable as a result.
\end{proof}
\item fix countable sets $A,B$.  The set $X$ of all functions from finite subsets of $B$ into $A$ is countable.
\begin{proof}
For each finite subset $C$ of $B$, the set of all functions from $C$ to $A$ is just $A^C$, which has cardinality $|A|^{|C|}$, and thus is countable since $A$ is. Since $X$ is just the union of all $A^C$, where $C$ ranges over the finite subsets of $B$, and there are countably many of them (as $B$ is countable), $X$ is also countable.
\end{proof}
\item fix countable sets $A,B$ and an element $a\in A$.  The set $Y$ of all functions from $B$ to $A$ such that $f(b)=a$ for all but a finite number of $b\in B$ is countable.
\begin{proof}  For any $f: B\to A$, call the \emph{support} of $f$ the set $\lbrace b \in B\mid f(b)\ne a\rbrace$, and denote it by $\operatorname{supp}(f)$.  Then every $f\in Y$ has finite support.  The map $G:Y\to X$ (where $X$ is defined in the last example) given by $G(f)=f|\operatorname{supp}(f)$ is an injection: if $G(f)=G(h)$, then $f(b)=h(b)$ for any $b\in \operatorname{supp}(f)=\operatorname{supp}(h)$, and $f(b)=a=h(b)$ otherwise, whence $f=g$.  But since $X$ is countable, so is $Y$.
\end{proof}
\end{enumerate}

\textbf{Concrete Examples}
\begin{enumerate}
\item the sets $\mathbb{N}$ (natural numbers), $\mathbb{Z}$ (integers), and $\mathbb{Q}$ (rational numbers)
\item the set of all algebraic numbers
\begin{proof}  Let $\mathbb{A}$ be the set of all algebraic numbers over $\mathbb{Q}$.  For each polynomial $p$ (in one variable $X$) over $\mathbb{Q}$, let $R_p$ be the set of roots of $p$ over $\mathbb{Q}$.  By definition, $\mathbb{A}$ is the union of all $R_p$, where $p$ ranges over the set $P$ of all polynomials over $\mathbb{Q}$.  For any $p \in P$ of degree $n$, we may associate a vector $v_p \in \mathbb{Q}^{n+1}$ : $$p=a_0 + a_1X + \cdots + a_n X^n \qquad \Longleftrightarrow \qquad v_p=(a_0,a_1,\ldots, a_n).$$  The association can be reversed.  So the set $P_n \subset P$ of all polynomials of degree $n$ is equinumerous to $\mathbb{Q}^{n+1}$, and therefore countable.  As $P$ is just the countable union of all $P_n$, $P$ is countable, which means $\mathbb{A} = \bigcup \lbrace R_p \mid p \in P\rbrace$ is countable also.
\end{proof}
\item the set of all algebraic integers, because every algebraic integer is an algebraic number.
\item the set of all words over an alphabet, because ever word can be thought of as a finite sequence over the alphabet, which is finite.
\end{enumerate}
%%%%%
%%%%%
\end{document}
