\documentclass[12pt]{article}
\usepackage{pmmeta}
\pmcanonicalname{Operator}
\pmcreated{2013-03-22 12:20:26}
\pmmodified{2013-03-22 12:20:26}
\pmowner{rmilson}{146}
\pmmodifier{rmilson}{146}
\pmtitle{operator}
\pmrecord{6}{31982}
\pmprivacy{1}
\pmauthor{rmilson}{146}
\pmtype{Definition}
\pmcomment{trigger rebuild}
\pmclassification{msc}{03-00}
\pmsynonym{mapping function}{Operator}

\endmetadata

\usepackage{amsmath}
\usepackage{amsfonts}
\usepackage{amssymb}


\newtheorem{proposition}{Proposition}
\begin{document}
Synonym of mapping and function. Often used to refer to mappings where the domain
and codomain are, in some sense a space of functions.

Examples: differential operator, convolution operator.

In the study of algebraic systems such as groups and rings, an operator often refers to a mapping from some cartesian power $A^{\lambda}$ of a set $A$ to the set $A$, where $\lambda$ is a cardinal.  For example, multiplication of integers can be thought of as an operator from $\mathbb{Z}^2$ to $\mathbb{Z}$.
%%%%%
%%%%%
\end{document}
