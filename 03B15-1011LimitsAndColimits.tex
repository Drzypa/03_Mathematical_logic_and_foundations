\documentclass[12pt]{article}
\usepackage{pmmeta}
\pmcanonicalname{1011LimitsAndColimits}
\pmcreated{2013-11-06 17:04:50}
\pmmodified{2013-11-06 17:04:50}
\pmowner{PMBookProject}{1000683}
\pmmodifier{PMBookProject}{1000683}
\pmtitle{10.1.1 Limits and colimits}
\pmrecord{1}{}
\pmprivacy{1}
\pmauthor{PMBookProject}{1000683}
\pmtype{Feature}
\pmclassification{msc}{03B15}

\endmetadata

\usepackage{xspace}
\usepackage{amssyb}
\usepackage{amsmath}
\usepackage{amsfonts}
\usepackage{amsthm}
\makeatletter
\def\@dprd#1{\prod_{(#1)}\,}
\def\@dprd@noparens#1{\prod_{#1}\,}
\def\@dsm#1{\sum_{(#1)}\,}
\def\@dsm@noparens#1{\sum_{#1}\,}
\def\@eatprd\prd{\prd@parens}
\def\@eatsm\sm{\sm@parens}
\newcommand{\emptyt}{\ensuremath{\mathbf{0}}\xspace}
\newcommand{\eqvsym}{\simeq}    
\newcommand{\Parens}[1]{\Bigl(#1\Bigr)}
\def\prd#1{\@ifnextchar\bgroup{\prd@parens{#1}}{\@ifnextchar\sm{\prd@parens{#1}\@eatsm}{\prd@noparens{#1}}}}
\def\prd@noparens#1{\mathchoice{\@dprd@noparens{#1}}{\@tprd{#1}}{\@tprd{#1}}{\@tprd{#1}}}
\def\prd@parens#1{\@ifnextchar\bgroup  {\mathchoice{\@dprd{#1}}{\@tprd{#1}}{\@tprd{#1}}{\@tprd{#1}}\prd@parens}  {\@ifnextchar\sm    {\mathchoice{\@dprd{#1}}{\@tprd{#1}}{\@tprd{#1}}{\@tprd{#1}}\@eatsm}    {\mathchoice{\@dprd{#1}}{\@tprd{#1}}{\@tprd{#1}}{\@tprd{#1}}}}}
\def\sm#1{\@ifnextchar\bgroup{\sm@parens{#1}}{\@ifnextchar\prd{\sm@parens{#1}\@eatprd}{\sm@noparens{#1}}}}
\def\sm@noparens#1{\mathchoice{\@dsm@noparens{#1}}{\@tsm{#1}}{\@tsm{#1}}{\@tsm{#1}}}
\def\sm@parens#1{\@ifnextchar\bgroup  {\mathchoice{\@dsm{#1}}{\@tsm{#1}}{\@tsm{#1}}{\@tsm{#1}}\sm@parens}  {\@ifnextchar\prd    {\mathchoice{\@dsm{#1}}{\@tsm{#1}}{\@tsm{#1}}{\@tsm{#1}}\@eatprd}    {\mathchoice{\@dsm{#1}}{\@tsm{#1}}{\@tsm{#1}}{\@tsm{#1}}}}}
\def\@tprd#1{\mathchoice{{\textstyle\prod_{(#1)}}}{\prod_{(#1)}}{\prod_{(#1)}}{\prod_{(#1)}}}
\def\@tsm#1{\mathchoice{{\textstyle\sum_{(#1)}}}{\sum_{(#1)}}{\sum_{(#1)}}{\sum_{(#1)}}}
\newcommand{\uset}{\ensuremath{\mathcal{S}et}\xspace}
\let\autoref\cref
\makeatother

\begin{document}

\index{limit!of sets}%
\index{colimit!of sets}%

Since sets are closed under products, the universal property of products in \autoref{thm:prod-ump} shows immediately that \uset has finite products.
In fact, infinite products follow just as easily from the equivalence
\[ \Parens{X\to \prd{a:A} B(a)} \eqvsym \Parens{\prd{a:A} (X\to B(a))}.\]
And we saw in \autoref{ex:pullback}\index{pullback} that the pullback of $f:A\to C$ and $g:B\to C$ can be defined as $\sm{a:A}{b:B} f(a)=g(b)$; this is a set if $A,B,C$ are and inherits the correct universal property.
Thus, \uset is a \emph{complete} category in the obvious sense.
\index{category!complete}%
\index{complete!category}%

Since sets are closed under $+$ and contain \emptyt, \uset has finite coproducts.
Similarly, since $\sm{a:A}B(a)$ is a set whenever $A$ and each $B(a)$ are, it yields a coproduct of the family $B$ in \uset.
Finally, we showed in \autoref{sec:pushouts} that pushouts exist in $n$-types, which includes \uset in particular.
Thus, \uset is also \emph{cocomplete}.
\index{category!cocomplete}%
\index{cocomplete category}%


\end{document}
