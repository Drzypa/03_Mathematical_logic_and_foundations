\documentclass[12pt]{article}
\usepackage{pmmeta}
\pmcanonicalname{Fibre}
\pmcreated{2013-03-22 12:55:23}
\pmmodified{2013-03-22 12:55:23}
\pmowner{mathcam}{2727}
\pmmodifier{mathcam}{2727}
\pmtitle{fibre}
\pmrecord{8}{33276}
\pmprivacy{1}
\pmauthor{mathcam}{2727}
\pmtype{Definition}
\pmcomment{trigger rebuild}
\pmclassification{msc}{03E20}
\pmsynonym{fiber}{Fibre}
\pmrelated{LevelSet}

% this is the default PlanetMath preamble.  as your knowledge
% of TeX increases, you will probably want to edit this, but
% it should be fine as is for beginners.

% almost certainly you want these
\usepackage{amssymb}
\usepackage{amsmath}
\usepackage{amsfonts}

% used for TeXing text within eps files
%\usepackage{psfrag}
% need this for including graphics (\includegraphics)
%\usepackage{graphicx}
% for neatly defining theorems and propositions
%\usepackage{amsthm}
% making logically defined graphics
%%%\usepackage{xypic} 

% there are many more packages, add them here as you need them

% define commands here
\begin{document}
Given a function $f\colon X \longrightarrow Y$, a \emph{fibre} is an inverse image of an element of $Y$. That is given $y \in Y$, $f^{-1}(\{y\}) = \{ x \in X \mid f(x) = y \}$ is a fibre. 

{\bf Example:}
Define $f\colon \mathbb{R}^2 \longrightarrow \mathbb{R}$ by $f(x,y) = x^2 + y^2$. Then the fibres of $f$ consist of concentric circles about the origin, the origin itself, and empty sets depending on whether we look at the inverse image of a positive number, zero, or a negative number respectively.

{\bf Example:}
Suppose $M$ is a manifold, and $\pi\colon TM\to M$ is the 
    canonical projection from the tangent bundle $TM$ to $M$. Then 
    fibres of $\pi$ are the tangent spaces $T_x(M)$ for $x\in M$.
%%%%%
%%%%%
\end{document}
