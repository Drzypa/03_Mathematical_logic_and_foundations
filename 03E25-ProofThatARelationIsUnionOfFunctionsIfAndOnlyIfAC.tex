\documentclass[12pt]{article}
\usepackage{pmmeta}
\pmcanonicalname{ProofThatARelationIsUnionOfFunctionsIfAndOnlyIfAC}
\pmcreated{2013-03-22 16:34:23}
\pmmodified{2013-03-22 16:34:23}
\pmowner{ratboy}{4018}
\pmmodifier{ratboy}{4018}
\pmtitle{proof that a relation is union of functions if and only if AC}
\pmrecord{4}{38763}
\pmprivacy{1}
\pmauthor{ratboy}{4018}
\pmtype{Proof}
\pmcomment{trigger rebuild}
\pmclassification{msc}{03E25}

% this is the default PlanetMath preamble.  as your knowledge
% of TeX increases, you will probably want to edit this, but
% it should be fine as is for beginners.

% almost certainly you want these
\usepackage{amssymb}
\usepackage{amsmath}
\usepackage{amsfonts}

% used for TeXing text within eps files
%\usepackage{psfrag}
% need this for including graphics (\includegraphics)
%\usepackage{graphicx}
% for neatly defining theorems and propositions
\usepackage{amsthm}
% making logically defined graphics
%%%\usepackage{xypic}

% there are many more packages, add them here as you need them

% define commands here
\DeclareMathOperator{\dom}{dom}
\DeclareMathOperator{\rng}{rng}

\begin{document}
\newtheorem*{theorem}{Theorem}

\begin{theorem}
A relation $R$ is the union of a set of functions each of which
has the same domain as $R$ if and only if for each set $A$ of
nonempty sets, there is a choice function on $A$.
\end{theorem}
\begin{proof}
Suppose that $R$ is a relation with $\dom(R) = A \mbox{ and }
\rng(R) \subseteq B$. Let $g \colon A \to \mathcal{P}(B)$ be given
by $a \mapsto R\left[\{a\}\right]$. There is be a choice function
$c$ on $g\left[A\right]$. Let $f = c \circ g$, and for each pair
$\langle a,b\rangle \in R$, let $f_{ab}$ send $a$ to $b$ and agree
with $f$ elsewhere. Let $F = \{f_{ab}\mid \langle a,b \rangle \in
R\}$. Clearly $\bigcup F \subseteq A \times B$, so suppose
$\langle u,v \rangle \in \bigcup F$; then there is a pair $\langle
a,b\rangle \in R$ such that $\langle u,v \rangle \in f_{ab} \in
F$. Either $\langle u,v \rangle = \langle a,b\rangle$, or $v =
f(u) = c \circ g(u) = c(R\left[\{u\}\right]) \in
R\left[\{u\}\right]$. In each case, $\langle u,v \rangle \in R$.
Thus, $\bigcup F \subseteq R.$ For each pair $\langle a,b\rangle
\in R$, $\langle a,b\rangle \in f_{ab} \in F$, so $R \subseteq
\bigcup F$. Therefore, $R = \bigcup F$.
\\
Suppose that $A$ is  set of nonempty sets. Let $R = \bigcup
\{\{a\} \times a \mid a \in A \}$. A set $x$ is an element of $
\dom(R)$ if and only if $x \in \{a\}$ for some $a \in A$. Thus,
$\dom(R) = A$. There is a set $F$ of functions, each of which has
domain $A$, such that $R = \bigcup F$. Let $f \in F$; then
$\dom(f) = A$, and for each pair $\langle a,f(a)\rangle \in f$,
$\langle a,f(a)\rangle \in \{a\} \times a$; i.e., $f(a) \in a$.
Each such $f$ is, thus, a choice function on $A$.
\end{proof}

%%%%%
%%%%%
\end{document}
