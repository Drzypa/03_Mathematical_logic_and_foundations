\documentclass[12pt]{article}
\usepackage{pmmeta}
\pmcanonicalname{ExampleOfPolyadicAlgebra}
\pmcreated{2013-03-22 17:53:20}
\pmmodified{2013-03-22 17:53:20}
\pmowner{CWoo}{3771}
\pmmodifier{CWoo}{3771}
\pmtitle{example of polyadic algebra}
\pmrecord{15}{40372}
\pmprivacy{1}
\pmauthor{CWoo}{3771}
\pmtype{Example}
\pmcomment{trigger rebuild}
\pmclassification{msc}{03G15}
\pmdefines{functional polyadic algebra}

\endmetadata

\usepackage{amssymb,amscd}
\usepackage{amsmath}
\usepackage{amsfonts}
\usepackage{mathrsfs}

% used for TeXing text within eps files
%\usepackage{psfrag}
% need this for including graphics (\includegraphics)
%\usepackage{graphicx}
% for neatly defining theorems and propositions
\usepackage{amsthm}
% making logically defined graphics
%%\usepackage{xypic}
\usepackage{pst-plot}

% define commands here
\newcommand*{\abs}[1]{\left\lvert #1\right\rvert}
\newtheorem{prop}{Proposition}
\newtheorem{thm}{Theorem}
\newtheorem{ex}{Example}
\newcommand{\real}{\mathbb{R}}
\newcommand{\pdiff}[2]{\frac{\partial #1}{\partial #2}}
\newcommand{\mpdiff}[3]{\frac{\partial^#1 #2}{\partial #3^#1}}
\begin{document}
Recall that the canonical example of a monadic algebra is that of a functional monadic algebra, which is a pair $(B,\exists)$ such that $B$ is the set of all functions from a non-empty set $X$ to a Boolean algebra $A$ such that, for each $f\in B$, the supremum and the infimum of $f(X)$ exist, and $\exists$ is a function on $B$ that maps each element $f$ to $f^{\exists}$, a constant element whose range is a singleton consisting of the supremum of $f(X)$.

The canonical example of a polyadic algebra is an extension (generalization) of a functional monadic algebra, known as the \emph{functional polyadic algebra}.  Instead of looking at functions from $X$ to $A$, we look at functions from $X^I$ (where $I$ is some set), the $I$-fold cartesian power of $X$, to $A$.  In this entry, an element $x\in X^I$ is written as a sequence of elements of $A$: $(x_i)_{i\in I}$ where $x_i\in A$, or $(x_i)$ for short.

Before constructing \emph{the} functional polyadic algebra based on the sets $X,I$ and the Boolean algebra $A$, we first introduce the following notations:
\begin{itemize}
\item for any $J\subseteq I$ and $x\in X^I$, define the subset (of $X^I$) $$[x]_J:=\lbrace y\in X^I \mid x_i=y_i\mbox{ for every }i\notin J\rbrace,$$
\item for any function $\tau:I\to I$ and any $f:X^I\to A$, define the function $f_{\tau}$ from $X^I$ to $A$, given by $$f_{\tau}(x_i):=f(x_{\tau(i)}).$$
\end{itemize}

Now, let $B$ be the set of all functions from $X^I$ to $A$ such that
\begin{enumerate}
\item for every $f\in B$, every $J\subseteq I$ and every $x\in X^I$, the arbitrary join $$\bigvee f\left([x]_J\right)$$ exists.

Before stating the next condition, we introduce, for each $f\in B$, a function $f^{\exists J}:X^I\to A$ as follows: $$f^{\exists J}(x):=\bigvee f\left([x]_J\right).$$  Now, we are ready for the next condition:
\item if $f\in B$, then $f^{\exists J}\in B$,
\item if $f\in B$, then $f_{\tau}\in B$ for $\tau:I\to I$.
\end{enumerate}
Note that if $A$ were a complete Boolean algebra, we can take $B$ to be $A^{X^I}$, the set of all functions from $X^I$ to $A$.

Next, define $\exists: P(I)\to B^B$ by $\exists(J)(f)=f^{\exists J}$, and let $S$ be the semigroup of functions on $I$ (with functional compositions as multiplications), then we call the quadruple $(B,I,\exists,S)$ the \emph{functional polyadic algebra} for the triple $(A,X,I)$.

\textbf{Remarks}.  Let $(B,I,\exists,S)$ be the functional polyadic algebra for $(A,X,I)$.
\begin{itemize}
\item $(B,I,\exists,S)$ is a polyadic algebra.  The proof of this is not difficult, but involved, and can be found in the reference below.
\item If $I$ is a singleton, then $(B,I,\exists,S)$ can be identified with the functional monadic algebra $(B,\exists)$ for $(A,X)$, for $S$ is just $I$, and $X^I$ is just $X$.
\item If $I$ is $\varnothing$, then $(B,I,\exists,S)$ can be identified with the Boolean algebra $A$, for $S=\varnothing$ and $X^I$ is a singleton, and hence the set of functions from $X^I$ to $A$ is identified with $A$.
\end{itemize}

\begin{thebibliography}{8}
\bibitem{ph} P. Halmos, \emph{Algebraic Logic}, Chelsea Publishing Co. New York (1962).
\end{thebibliography}

%%%%%
%%%%%
\end{document}
