\documentclass[12pt]{article}
\usepackage{pmmeta}
\pmcanonicalname{53Wtypes}
\pmcreated{2013-11-05 23:31:37}
\pmmodified{2013-11-05 23:31:37}
\pmowner{PMBookProject}{1000683}
\pmmodifier{rspuzio}{6075}
\pmtitle{5.3 W-types}
\pmrecord{1}{}
\pmprivacy{1}
\pmauthor{PMBookProject}{6075}
\pmtype{Feature}
\pmclassification{msc}{03B15}

\endmetadata

\usepackage{xspace}
\usepackage{amssyb}
\usepackage{amsmath}
\usepackage{amsfonts}
\usepackage{amsthm}
\makeatletter
\newcommand{\bfalse}{{0_{\bool}}}
\newcommand{\bool}{\ensuremath{\mathbf{2}}\xspace}
\newcommand{\btrue}{{1_{\bool}}}
\newcommand{\dbl}{\ensuremath{\mathsf{double}}}
\newcommand{\defeq}{\vcentcolon\equiv}  
\def\@dprd#1{\prod_{(#1)}\,}
\def\@dprd@noparens#1{\prod_{#1}\,}
\def\@dsm#1{\sum_{(#1)}\,}
\def\@dsm@noparens#1{\sum_{#1}\,}
\def\@eatprd\prd{\prd@parens}
\def\@eatsm\sm{\sm@parens}
\newcommand{\emptyt}{\ensuremath{\mathbf{0}}\xspace}
\newcommand{\id}[3][]{\ensuremath{#2 =_{#1} #3}\xspace}
\newcommand{\indexsee}[2]{\index{#1|see{#2}}}    
\newcommand{\jdeq}{\equiv}      
\def\lamu#1{{\lambda}\@lamuarg#1:\@endlamuarg\@ifnextchar\bgroup{.\,\lamu}{.\,}}
\def\@lamuarg#1:#2\@endlamuarg{#1}
\newcommand{\lst}[1]{\mathsf{List}(#1)}
\newcommand{\natw}{\ensuremath{\mathbf{N^w}}\xspace}
\newcommand{\Parens}[1]{\Bigl(#1\Bigr)}
\def\prd#1{\@ifnextchar\bgroup{\prd@parens{#1}}{\@ifnextchar\sm{\prd@parens{#1}\@eatsm}{\prd@noparens{#1}}}}
\def\prd@noparens#1{\mathchoice{\@dprd@noparens{#1}}{\@tprd{#1}}{\@tprd{#1}}{\@tprd{#1}}}
\def\prd@parens#1{\@ifnextchar\bgroup  {\mathchoice{\@dprd{#1}}{\@tprd{#1}}{\@tprd{#1}}{\@tprd{#1}}\prd@parens}  {\@ifnextchar\sm    {\mathchoice{\@dprd{#1}}{\@tprd{#1}}{\@tprd{#1}}{\@tprd{#1}}\@eatsm}    {\mathchoice{\@dprd{#1}}{\@tprd{#1}}{\@tprd{#1}}{\@tprd{#1}}}}}
\newcommand{\rec}[1]{\mathsf{rec}_{#1}}
\def\sm#1{\@ifnextchar\bgroup{\sm@parens{#1}}{\@ifnextchar\prd{\sm@parens{#1}\@eatprd}{\sm@noparens{#1}}}}
\def\sm@noparens#1{\mathchoice{\@dsm@noparens{#1}}{\@tsm{#1}}{\@tsm{#1}}{\@tsm{#1}}}
\def\sm@parens#1{\@ifnextchar\bgroup  {\mathchoice{\@dsm{#1}}{\@tsm{#1}}{\@tsm{#1}}{\@tsm{#1}}\sm@parens}  {\@ifnextchar\prd    {\mathchoice{\@dsm{#1}}{\@tsm{#1}}{\@tsm{#1}}{\@tsm{#1}}\@eatprd}    {\mathchoice{\@dsm{#1}}{\@tsm{#1}}{\@tsm{#1}}{\@tsm{#1}}}}}
\newcommand{\supp}{\ensuremath\suppsym\xspace}
\newcommand{\suppsym}{{\mathsf{sup}}}
\newcommand{\symlabel}[1]{\refstepcounter{symindex}\label{#1}}
\def\@tprd#1{\mathchoice{{\textstyle\prod_{(#1)}}}{\prod_{(#1)}}{\prod_{(#1)}}{\prod_{(#1)}}}
\def\@tsm#1{\mathchoice{{\textstyle\sum_{(#1)}}}{\sum_{(#1)}}{\sum_{(#1)}}{\sum_{(#1)}}}
\newcommand{\ttt}{\ensuremath{\star}\xspace}
\def\@twtype#1{\mathchoice{{\textstyle\mathsf{W}_{(#1)}}}{\mathsf{W}_{(#1)}}{\mathsf{W}_{(#1)}}{\mathsf{W}_{(#1)}}}
\newcommand{\unit}{\ensuremath{\mathbf{1}}\xspace}
\newcommand{\UU}{\ensuremath{\mathcal{U}}\xspace}
\newcommand{\vcentcolon}{:\!\!}
\newcommand{\w}{\mathsf{W}}
\def\wtype#1{\@ifnextchar\bgroup  {\mathchoice{\@twtype{#1}}{\@twtype{#1}}{\@twtype{#1}}{\@twtype{#1}}\wtype}  {\mathchoice{\@twtype{#1}}{\@twtype{#1}}{\@twtype{#1}}{\@twtype{#1}}}}
\newcommand{\zerow}{\ensuremath{0^\mathbf{w}}\xspace}
\newcounter{mathcount}
\setcounter{mathcount}{1}
\newtheorem{prethm}{Theorem}
\newenvironment{thm}{\begin{prethm}}{\end{prethm}\addtocounter{mathcount}{1}}
\renewcommand{\theprethm}{5.3.\arabic{mathcount}}
\let\bbU\UU
\let\type\UU
\makeatother

\begin{document}

Inductive types are very general, which is excellent for their usefulness and applicability, but makes them difficult to study as a whole.
Fortunately, they can all be formally reduced to a few special cases.
It is beyond the scope of this book to discuss this reduction --- which is anyway irrelevant to the mathematician using type theory in practice --- but we will take a little time to discuss the one of the basic special cases that we have not yet met.
These are Martin-L{\"o}f's \emph{$\w$-types}, also known as the types of \emph{well-founded trees}.
\index{tree, well-founded}%
$\w$-types are a generalization of such types as natural numbers, lists, and binary trees, which are sufficiently general to encapsulate the ``recursion'' aspect of \emph{any} inductive type.

A particular $\w$-type is specified by giving two parameters $A : \type$ and $B : A \to \type$, in which case the resulting $\w$-type is written $\wtype{a:A} B(a)$.
The type $A$ represents the type of \emph{labels} for $\wtype{a :A} B(a)$, which function as constructors (however, we reserve that word for the actual functions which arise in inductive definitions). For instance, when defining natural numbers as a $\w$-type,%
\index{natural numbers!encoded as a W-type@encoded as a $\w$-type}
the type $A$ would be the type $\bool$ inhabited by the two elements $\bfalse$ and $\btrue$, since there are precisely two ways to obtain a natural number --- either it will be zero or a successor of another natural number. 

The type family $B : A \to \type$ is used to record the arity\index{arity} of labels: a label $a : A$ will take a family of inductive arguments, indexed over $B(a)$. We can therefore think of the ``$B(a)$-many'' arguments of $a$. These arguments are represented by a function $f : B(a) \to \wtype{a :A} B(a)$, with the understanding that for any $b : B(a)$, $f(b)$ is the ``$b$-th'' argument to the label $a$. The $\w$-type $\wtype{a :A} B(a)$ can thus be thought of as the type of well-founded trees, where nodes are labeled by elements of $A$ and each node labeled by $a : A$ has $B(a)$-many branches.

In the case of natural numbers, the label $\bfalse$ has arity 0, since it constructs the constant zero\index{zero}; the label $\btrue$ has arity 1, since it constructs the successor\index{successor} of its argument. We can capture this by using simple elimination on $\bool$ to define a function $\rec\bool(\bbU,\emptyt,\unit)$ into a universe of types; this function returns the empty type $\emptyt$ for $\bfalse$ and the unit\index{type!unit} type $\unit$ for $\btrue$. We can thus define
\symlabel{natw}
\index{type!unit}%
\index{type!empty}%
\[ \natw \defeq \wtype{b:\bool} \rec\bool(\bbU,\emptyt,\unit) \]
where the superscript $\mathbf{w}$ serves to distinguish this version of natural numbers from the previously used one.
Similarly, we can define the type of lists\index{type!of lists} over $A$ as a $\w$-type with $\unit + A$ many labels: one nullary label for the empty list, plus one unary label for each $a : A$, corresponding to appending $a$ to the head of a list:
\[ \lst A \defeq \wtype{x: \unit + A} \rec{\unit + A}(\bbU, \; \emptyt, \; \lamu{a:A} \unit). \]
%
\index{W-type@$\w$-type}%
\indexsee{type!W-@$\w$-}{$\w$-type}%
In general, the $\w$-type $\wtype{x:A} B(x)$ specified by  $A : \type$ and $B : A \to \type$ is the inductive type generated by the following constructor:
\begin{itemize}
\item \label{defn:supp}
  $\supp : \prd{a:A} \Big(B(a) \to \wtype{x:A} B(x)\Big) \to \wtype{x:A} B(x)$.
\end{itemize}
%
The constructor $\supp$ (short for supremum\index{supremum!constructor of a W-type@constructor of a $\w$-type}) takes a label $a : A$ and a function $f : B(a) \to \wtype{x:A} B(x)$ representing the arguments to $a$, and constructs a new element of $\wtype{x:A} B(x)$. Using our previous encoding of natural numbers as $\w$-types, we can for instance define
\begin{equation*}
\zerow \defeq \supp(\bfalse, \; \lamu{x:\emptyt} \rec\emptyt(\natw,x)).
\end{equation*}
Put differently, we use the label $\bfalse$ to construct $\zerow$. Then, $\rec\bool(\bbU,\emptyt,\unit, \bfalse)$ evaluates to $\emptyt$, as it should since $\bfalse$ is a nullary label. Thus, we need to construct a function $f : \emptyt \to \natw$, which represents the (zero) arguments supplied to $\bfalse$. This is of course trivial, using simple elimination on $\emptyt$ as shown. Similarly, we can define
\begin{align*}
1^{\mathbf{w}} &\defeq \supp(\btrue, \; \lamu{x:\unit} 0^{\mathbf{w}}) \\
2^{\mathbf{w}} &\defeq \supp(\btrue, \; \lamu{x:\unit} 1^{\mathbf{w}})
\end{align*}
and so on.

\index{induction principle!for W-types@for $\w$-types}%
We have the following induction principle for $\w$-types:
\begin{itemize}
\item When proving a statement $E : \big(\wtype{x:A} B(x)\big) \to \type$ about \emph{all} elements of the $\w$-type $\wtype{x:A} B(x)$, it suffices to prove it for $\supp(a,f)$, assuming it holds for all $f(b)$ with $b : B(a)$. 
In other words, it suffices to give a proof 
\begin{equation*}
e : \prd{a:A}{f : B(a) \to \wtype{x:A} B(x)}{g : \prd{b : B(a)} E(f(b))} E(\supp(a,f))
\end{equation*}
\end{itemize}

\index{variable}%
The variable $g$ represents our inductive hypothesis, namely that all arguments of $a$ satisfy $E$. To state this, we quantify over all elements of type $B(a)$, since each $b : B(a)$ corresponds to one argument $f(b)$ of $a$.

How would we define the function $\dbl$ on natural numbers encoded as a $\w$-type? We would like to use the recursion principle of $\natw$ with a codomain of $\natw$ itself. We thus need to construct a suitable function
\[e : \prd{a : \bool}{f : B(a) \to \natw}{g : B(a) \to \natw} \natw\]
which will represent the recurrence for the $\dbl$ function; for simplicity we denote the type family $\rec\bool(\bbU,\emptyt,\unit)$ by $B$.

Clearly, $e$ will be a function taking $a : \bool$ as its first argument. The next step is to perform case analysis on $a$ and proceed based on whether it is $\bfalse$ or $\btrue$. This suggests the following form
\[ e \defeq \lamu{a:\bool} \rec\bool(C,e_0,e_1,a) \]
where \[C \defeq \prd{f : B(a) \to \natw}{g : B(a) \to \natw} \natw\]
If $a$ is $\bfalse$, the type $B(a)$ becomes $\emptyt$. Thus, given $f : \emptyt \to \natw$ and $g : \emptyt \to \natw$, we want to construct an element of $\natw$. Since the label $\bfalse$ represents $\emptyt$, it needs zero inductive arguments and the variables $f$ and $g$ are irrelevant. We return $\zerow$\index{zero} as a result:
\[ e_0 \defeq \lamu{f:\emptyt \to \natw}{g:\emptyt \to \natw} \zerow \]
Analogously, if $a$ is $\btrue$, the type $B(a)$ becomes $\unit$.
Since the label $\btrue$ represents the successor\index{successor} operator, it needs one inductive argument --- the predecessor\index{predecessor} --- which is represented by the variable $f : \unit \to \natw$.
The value of the recursive call on the predecessor is represented by the variable $g : \unit \to \natw$.
Thus, taking this value (namely $g(\ttt)$) and applying the successor operator twice thus yields the desired result:
\begin{equation*}
e_1 \defeq \; \lamu{f:\unit \to \natw}{g:\unit \to \natw}
  \supp(\btrue, (\lamu{x:\unit} \supp(\btrue, (\lamu{y : \unit} g(\ttt))))).
\end{equation*}
Putting this together, we thus have
\[ \dbl \defeq \rec\natw(\natw, e) \]
with $e$ as defined above.

\symlabel{defn:recursor-wtype}
The associated computation rule for the function $\rec{\wtype{x:A} B(x)}(E,e) : \prd{w : \wtype{x:A} B(x)} E(w)$ is as follows.
\index{computation rule!for W-types@for $\w$-types}%
\begin{itemize}
\item
  For any $a : A$ and $f : B(a) \to \wtype{x:A} B(x)$ we have 
  \begin{equation*}
    \rec{\wtype{x:A} B(x)}(E,e,\supp(a,f)) \jdeq
    e(a,f,\big(\lamu{b:B(a)} \rec{\wtype{x:A} B(x)}(E,f(b))\big)).
  \end{equation*}
\end{itemize}
In other words, the function $\rec{\wtype{x:A} B(x)}(E,e)$ satisfies the recurrence $e$.

By the above computation rule, the function $\dbl$ behaves as expected:
\begin{align*}
\dbl(\zerow) & \jdeq \rec\natw(\natw, e, \supp(\bfalse, \; \lamu{x:\emptyt} \rec\emptyt(\natw,x))) \\
& \jdeq e(\bfalse, \big(\lamu{x:\emptyt} \rec\emptyt(\natw,x)\big), 
   \big(\lamu{x:\emptyt} \dbl(\rec\emptyt(\natw,x))\big)) \\
 & \jdeq e_t(\big(\lamu{x:\emptyt} \rec\emptyt(\natw,x)\big), \big(\lamu{x:\emptyt} \dbl(\rec\emptyt(\natw,x))\big))\\
 & \jdeq \zerow \\
 \intertext{and}
\dbl(1^{\mathbf{w}}) & \jdeq \rec\natw(\natw, e, \supp(\btrue, \; \lamu{x:\unit} \zerow)) \\
& \jdeq e(\btrue, \big(\lamu{x:\unit} \zerow\big), \big(\lamu{x:\unit} \dbl(\zerow)\big)) \\
 & \jdeq e_f(\big(\lamu{x:\unit} \zerow\big), \big(\lamu{x:\unit} \dbl(\zerow)\big)) \\
 & \jdeq \supp(\btrue, (\lamu{x:\unit} \supp(\btrue,(\lamu{y:\unit} \dbl(\zerow))))) \\
 & \jdeq \supp(\btrue, (\lamu{x:\unit} \supp(\btrue,(\lamu{y:\unit} \zerow)))) \\
 & \jdeq 2^{\mathbf{w}}
\end{align*}
and so on.

Just as for natural numbers, we can prove a uniqueness theorem for 
$\w$-types:
\begin{thm}\label{thm:w-uniq}
  \index{uniqueness!principle, propositional!for functions on W-types@for functions on $\w$-types}%
Let $g,h : \prd{w:\wtype{x:A}B(x)} E(w)$ be two functions which satisfy the recurrence
%
\begin{equation*}
  e : \prd{a,f} \Parens{\prd{b : B(a)} E(f(b))} \to  E(\supp(a,f)),
\end{equation*}
%
i.e., such that
%
\begin{gather*}
 \prd{a,f} \id{g(\supp(a,f))} {e(a,f,\lamu{b:B(a)} g(f(b)))}, \\
 \prd{a,f} \id{h(\supp(a,f))}{e(a,f,\lamu{b:B(a)} h(f(b)))}.
\end{gather*}
Then $g$ and $h$ are equal. 
\end{thm}



\end{document}
