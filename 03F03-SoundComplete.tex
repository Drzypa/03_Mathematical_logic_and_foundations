\documentclass[12pt]{article}
\usepackage{pmmeta}
\pmcanonicalname{SoundComplete}
\pmcreated{2013-03-22 13:02:33}
\pmmodified{2013-03-22 13:02:33}
\pmowner{Henry}{455}
\pmmodifier{Henry}{455}
\pmtitle{sound,, complete}
\pmrecord{7}{33445}
\pmprivacy{1}
\pmauthor{Henry}{455}
\pmtype{Definition}
\pmcomment{trigger rebuild}
\pmclassification{msc}{03F03}
\pmdefines{sound}
\pmdefines{complete}

\endmetadata

% this is the default PlanetMath preamble.  as your knowledge
% of TeX increases, you will probably want to edit this, but
% it should be fine as is for beginners.

% almost certainly you want these
\usepackage{amssymb}
\usepackage{amsmath}
\usepackage{amsfonts}

% used for TeXing text within eps files
%\usepackage{psfrag}
% need this for including graphics (\includegraphics)
%\usepackage{graphicx}
% for neatly defining theorems and propositions
%\usepackage{amsthm}
% making logically defined graphics
%%%\usepackage{xypic}

% there are many more packages, add them here as you need them

% define commands here
%\PMlinkescapeword{theory}
\begin{document}
If $Th$ and $Pr$ are two sets of facts (in particular, a theory of some language and the set of things provable by some method) we say $Pr$ is \emph{sound} for $Th$ if $Pr\subseteq Th$.  Typically we have a theory and set of rules for constructing proofs, and we say the set of rules are sound (which theory is intended is usually clear from context) since everything they prove is true (in $Th$).

If $Th\subseteq Pr$ we say $Pr$ is \emph{complete} for $Th$.  Again, we usually have a theory and a set of rules for constructing proofs, and say that the set of rules is complete since everything true (in $Th$) can be proven.
%%%%%
%%%%%
\end{document}
