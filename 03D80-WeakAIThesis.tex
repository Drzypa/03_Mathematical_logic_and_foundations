\documentclass[12pt]{article}
\usepackage{pmmeta}
\pmcanonicalname{WeakAIThesis}
\pmcreated{2013-03-22 17:07:15}
\pmmodified{2013-03-22 17:07:15}
\pmowner{dankomed}{17058}
\pmmodifier{dankomed}{17058}
\pmtitle{weak AI thesis}
\pmrecord{9}{39423}
\pmprivacy{1}
\pmauthor{dankomed}{17058}
\pmtype{Topic}
\pmcomment{trigger rebuild}
\pmclassification{msc}{03D80}
\pmsynonym{weak artificial intelligence thesis}{WeakAIThesis}
\pmrelated{StrongAIThesis}

% this is the default PlanetMath preamble.  as your knowledge
% of TeX increases, you will probably want to edit this, but
% it should be fine as is for beginners.

% almost certainly you want these
\usepackage{amssymb}
\usepackage{amsmath}
\usepackage{amsfonts}

% used for TeXing text within eps files
%\usepackage{psfrag}
% need this for including graphics (\includegraphics)
%\usepackage{graphicx}
% for neatly defining theorems and propositions
%\usepackage{amsthm}
% making logically defined graphics
%%%\usepackage{xypic}

% there are many more packages, add them here as you need them

% define commands here

\begin{document}
\begin{verse}Weak AI (artificial intelligence) thesis: A digital computer is a powerful tool for
studying intelligence and developing useful technology, and it enables us to formulate and test hypotheses in a more rigorous and precise fashion. However a \PMlinkescapetext{running} AI program is not itself a cognitive process, at most it is a simulation of such process.
\end{verse}

References

1. Mooney RJ (1999) \PMlinkexternal{Philosophical Arguments Against AI.}{http://www.cs.utexas.edu/~mooney/cs343/slide-handouts/philosophy.4.pdf}

2. Searle JR. (1980) \PMlinkexternal{Minds, brains, and programs. Behavioral and Brain Sciences 3(3): 417-457.}{http://members.aol.com/NeoNoetics/MindsBrainsPrograms.html}
%%%%%
%%%%%
\end{document}
