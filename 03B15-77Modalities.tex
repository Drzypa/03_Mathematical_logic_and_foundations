\documentclass[12pt]{article}
\usepackage{pmmeta}
\pmcanonicalname{77Modalities}
\pmcreated{2013-11-06 2:00:52}
\pmmodified{2013-11-06 2:00:52}
\pmowner{PMBookProject}{1000683}
\pmmodifier{rspuzio}{6075}
\pmtitle{7.7 Modalities}
\pmrecord{1}{}
\pmprivacy{1}
\pmauthor{PMBookProject}{6075}
\pmtype{Feature}
\pmclassification{msc}{03B15}

\endmetadata

\usepackage{xspace}
\usepackage{amssyb}
\usepackage{amsmath}
\usepackage{amsfonts}
\usepackage{amsthm}
\makeatletter
\newcommand{\blank}{\mathord{\hspace{1pt}\text{--}\hspace{1pt}}}
\newcommand{\defeq}{\vcentcolon\equiv}  
\newcommand{\define}[1]{\textbf{#1}}
\def\@dprd#1{\prod_{(#1)}\,}
\def\@dprd@noparens#1{\prod_{#1}\,}
\def\@dsm#1{\sum_{(#1)}\,}
\def\@dsm@noparens#1{\sum_{#1}\,}
\def\@eatprd\prd{\prd@parens}
\def\@eatsm\sm{\sm@parens}
\newcommand{\eqvsym}{\simeq}    
\newcommand{\function}[4]{\left\{\begin{array}{rcl}#1 &      \longrightarrow & #2 \\ #3 & \longmapsto & #4 \end{array}\right.}
\newcommand{\hfib}[2]{{\mathsf{fib}}_{#1}(#2)}
\newcommand{\idfunc}[1][]{\ensuremath{\mathsf{id}_{#1}}\xspace}
\newcommand{\ind}[1]{\mathsf{ind}_{#1}}
\newcommand{\indexdef}[1]{\index{#1|defstyle}}   
\newcommand{\isequiv}{\ensuremath{\mathsf{isequiv}}}
\def\lam#1{{\lambda}\@lamarg#1:\@endlamarg\@ifnextchar\bgroup{.\,\lam}{.\,}}
\def\@lamarg#1:#2\@endlamarg{\if\relax\detokenize{#2}\relax #1\else\@lamvar{\@lameatcolon#2},#1\@endlamvar\fi}
\def\@lameatcolon#1:{#1}
\def\@lamvar#1,#2\@endlamvar{(#2\,{:}\,#1)}
\newcommand{\modal}{\ensuremath{\ocircle}}
\newcommand{\modaltype}{\ensuremath{\type_\modal}}
\newcommand{\mreturn}{\ensuremath{\eta}}
\newcommand{\narrowequation}[1]{$#1$}
\newcommand{\opp}[1]{\mathord{{#1}^{-1}}}
\newcommand{\Parens}[1]{\Bigl(#1\Bigr)}
\def\prd#1{\@ifnextchar\bgroup{\prd@parens{#1}}{\@ifnextchar\sm{\prd@parens{#1}\@eatsm}{\prd@noparens{#1}}}}
\def\prd@noparens#1{\mathchoice{\@dprd@noparens{#1}}{\@tprd{#1}}{\@tprd{#1}}{\@tprd{#1}}}
\def\prd@parens#1{\@ifnextchar\bgroup  {\mathchoice{\@dprd{#1}}{\@tprd{#1}}{\@tprd{#1}}{\@tprd{#1}}\prd@parens}  {\@ifnextchar\sm    {\mathchoice{\@dprd{#1}}{\@tprd{#1}}{\@tprd{#1}}{\@tprd{#1}}\@eatsm}    {\mathchoice{\@dprd{#1}}{\@tprd{#1}}{\@tprd{#1}}{\@tprd{#1}}}}}
\newcommand{\proj}[1]{\ensuremath{\mathsf{pr}_{#1}}\xspace}
\newcommand{\prop}{\ensuremath{\mathsf{Prop}}\xspace}
\newcommand{\rec}[1]{\mathsf{rec}_{#1}}
\def\sm#1{\@ifnextchar\bgroup{\sm@parens{#1}}{\@ifnextchar\prd{\sm@parens{#1}\@eatprd}{\sm@noparens{#1}}}}
\def\sm@noparens#1{\mathchoice{\@dsm@noparens{#1}}{\@tsm{#1}}{\@tsm{#1}}{\@tsm{#1}}}
\def\sm@parens#1{\@ifnextchar\bgroup  {\mathchoice{\@dsm{#1}}{\@tsm{#1}}{\@tsm{#1}}{\@tsm{#1}}\sm@parens}  {\@ifnextchar\prd    {\mathchoice{\@dsm{#1}}{\@tsm{#1}}{\@tsm{#1}}{\@tsm{#1}}\@eatprd}    {\mathchoice{\@dsm{#1}}{\@tsm{#1}}{\@tsm{#1}}{\@tsm{#1}}}}}
\def\@tprd#1{\mathchoice{{\textstyle\prod_{(#1)}}}{\prod_{(#1)}}{\prod_{(#1)}}{\prod_{(#1)}}}
\newcommand{\trans}[2]{\ensuremath{{#1}_{*}\mathopen{}\left({#2}\right)\mathclose{}}\xspace}
\def\@tsm#1{\mathchoice{{\textstyle\sum_{(#1)}}}{\sum_{(#1)}}{\sum_{(#1)}}{\sum_{(#1)}}}
\newcommand{\unit}{\ensuremath{\mathbf{1}}\xspace}
\newcommand{\UU}{\ensuremath{\mathcal{U}}\xspace}
\newcommand{\vcentcolon}{:\!\!}
\newcounter{mathcount}
\setcounter{mathcount}{1}
\newtheorem{precor}{Corollary}
\newenvironment{cor}{\begin{precor}}{\end{precor}\addtocounter{mathcount}{1}}
\renewcommand{\theprecor}{7.7.\arabic{mathcount}}
\newtheorem{predefn}{Definition}
\newenvironment{defn}{\begin{predefn}}{\end{predefn}\addtocounter{mathcount}{1}}
\renewcommand{\thepredefn}{7.7.\arabic{mathcount}}
\newenvironment{myeqn}{\begin{equation}}{\end{equation}\addtocounter{mathcount}{1}}
\renewcommand{\theequation}{7.7.\arabic{mathcount}}
\newtheorem{prethm}{Theorem}
\newenvironment{thm}{\begin{prethm}}{\end{prethm}\addtocounter{mathcount}{1}}
\renewcommand{\theprethm}{7.7.\arabic{mathcount}}
\let\autoref\cref
\let\project\mreturn
\let\reflect\modal
\let\setof\Set    
\let\type\UU
\makeatother

\begin{document}

\index{modality|(}

Nearly all of the theory of $n$-types and connectedness can be done in much greater generality.
This section will not be used in the rest of the book.

Our first thought regarding generalizing the theory of $n$-types might be to take \autoref{thm:trunc-reflective} as a definition.

\begin{defn}\label{defn:reflective-subuniverse}
  A \define{reflective subuniverse}
  \indexdef{reflective!subuniverse}%
  \indexdef{subuniverse, reflective}%
  is a predicate $P:\type\to\prop$ such that
  for every $A:\type$ we have a type $\reflect A$ such that $P(\reflect A)$ and a map
  $\project_A:A\to\reflect A$, with the property that for every $B:\type$ with $P(B)$, the following map is an equivalence:
  \[\function{(\reflect A\to{}B)}{(A\to{}B)}{f}{f\circ\project_A}.\]
\end{defn}

We write $\P \defeq \setof{A:\type | P(A)}$, so $A:\P$ means that $A:\type$ and we have $P(A)$.
We also write $\rec{\modal}$ for the quasi-inverse of the above map.
The notation $\reflect$ may seem slightly odd, but it will make more sense soon.

For any reflective subuniverse, we can prove all the familiar facts about reflective subcategories from category theory, in the usual way.
For instance, we have:
\begin{itemize}
\item A type $A$ lies in $\P$ if and only if $\project_A:A\to\reflect A$ is an equivalence.
\item $\P$ is closed under retracts.
  In particular, $A$ lies in $\P$ as soon as $\project_A$ admits a retraction.
\item The operation $\reflect$ is a functor in a suitable up-to-coherent-homotopy sense, which we can make precise at as high levels as necessary.
\item The types in $\P$ are closed under all limits such as products and pullbacks.
  In particular, for any $A:\P$ and $x,y:A$, the identity type $(x=_A y)$ is also in $\P$, since it is a pullback of two functions $\unit\to A$.
\item Colimits in $\P$ can be constructed by applying $\reflect$ to ordinary colimits of types.
\end{itemize}

Importantly, closure under products extends also to ``infinite products'', i.e.\ dependent function types.

\begin{thm}\label{thm:reflsubunv-forall}
  If $B:A\to\P$ is any family of types in a reflective subuniverse \P, then $\prd{x:A} B(x)$ is also in \P.
\end{thm}
\begin{proof}
  For any $x:A$, consider the function $\mathsf{ev}_x : (\prd{x:A} B(x)) \to B(x)$ defined by $\mathsf{ev}_x(f) \defeq f(x)$.
  Since $B(x)$ lies in $P$, this extends to a function
  \[ \rec{\modal}(\mathsf{ev}_x) : \reflect\Parens{\prd{x:A} B(x)} \to B(x). \]
  Thus we can define $h:\reflect(\prd{x:A} B(x)) \to \prd{x:A} B(x)$ by $h(z)(x) \defeq \rec{\modal}(\mathsf{ev}_x)(z)$.
  Then $h$ is a retraction of $\project_{\prd{x:A} B(x)}$, so that ${\prd{x:A} B(x)}$ is in $\P$.
\end{proof}

In particular, if $B:\P$ and $A$ is any type, then $(A\to B)$ is in \P.
In categorical language, this means that any reflective subuniverse is an \define{exponential ideal}.
\indexdef{exponential ideal}%
This, in turn, implies by a standard argument that the reflector preserves finite products.

\begin{cor}\label{cor:trunc_prod}
  For any types $A$ and $B$ and any reflective subuniverse, the induced map $\reflect(A\times B) \to \reflect(A) \times \reflect(B)$ is an equivalence.
\end{cor}
\begin{proof}
  It suffices to show that $\reflect(A) \times \reflect(B)$ has the same universal property as $\reflect(A\times B)$.
  Thus, let $C:\P$; we have
  \begin{align*}
    (\reflect(A) \times \reflect(B) \to C)
    &= (\reflect(A) \to (\reflect(B) \to C))\\
    &= (\reflect(A) \to (B \to C))\\
    &= (A \to (B \to C))\\
    &= (A \times B \to C)
  \end{align*}
  using the universal properties  of $\reflect(B)$ and $\reflect(A)$, along with the fact that $B\to C$ is in \P since $C$ is.
  It is straightforward to verify that this equivalence is given by composing with $\mreturn_A \times \mreturn_B$, as needed.
\end{proof}

It may seem odd that every reflective subcategory of types is automatically an exponential ideal, with a product-preserving reflector.
However, this is also the case classically in the category of \emph{sets}, for the same reasons.
It's just that this fact is not usually remarked on, since the classical category of sets---in contrast to the category of homotopy
types---does not have many interesting reflective subcategories.

Two basic properties of $n$-types are \emph{not} shared by general reflective subuniverses: \autoref{thm:ntypes-sigma} (closure under $\Sigma$-types) and \autoref{thm:truncn-ind} (truncation induction).
However, the analogues of these two properties are equivalent to each other.


\begin{thm}\label{thm:modal-char}
  For a reflective subuniverse \P, the following are logically equivalent.
  \begin{enumerate}
  \item If $A:\P$ and $B:A\to \P$, then $\sm{x:A} B(x)$ is in \P.\label{item:mchr1}
  \item for every $A:\type$, type family $B:\reflect A\to\P$, and map $g:\prd{a:A} B(\project(a))$, there exists $f:\prd{z:\reflect A} B(z)$ such that $f(\project(a)) = g(a)$ for all $a:A$.\label{item:mchr2}
  \end{enumerate}
\end{thm}
\begin{proof}
  Suppose~\ref{item:mchr1}.
  Then in the situation of~\ref{item:mchr2}, the type $\sm{z:\reflect A} B(z)$ lies in $\P$, and we have $g':A\to \sm{z:\reflect A} B(z)$ defined by $g'(a)\defeq (\project(a),g(a))$.
  Thus, we have $\rec{\modal}(g'):\reflect A \to \sm{z:\reflect A} B(z)$ such that $\rec{\modal}(g')(\project(a)) = (\project(a),g(a))$.

  Now consider the functions $\proj2 \circ \rec{\modal}(g') : \reflect A \to \reflect A$ and $\idfunc[\reflect A]$.
  By assumption, these become equal when precomposed with $\project$.
  Thus, by the universal property of $\reflect$, they are equal already, i.e.\ we have $p_z:\proj2(\rec{\modal}(g')(z)) = z$ for all $z$.
  Now we can define
  %
  \narrowequation{f(z) \defeq \trans{p_z}{\proj2(\rec{\modal}(g')(z))},}
  %
  and the second component of
  %
  \narrowequation{\rec{\modal}(g')(\project(a)) = (\project(a),g(a))}
  %
  yields $f(\project(a)) = g(a)$.

  Conversely, suppose~\ref{item:mchr2}, and that $A:\P$ and $B:A\to\P$.
  Let $h$ be the composite
  \[ \reflect\Parens{\sm{x:A} B(x)} \xrightarrow{\reflect(\proj1)} \reflect A \xrightarrow{\opp{(\project_A)}} A. \]
  Then for $z:\sm{x:A} B(x)$ we have
  \begin{align*}
    h(\project(z)) &= \opp\project(\reflect(\proj1)(\project(z)))\\
    &= \opp\project(\project(\proj1(z)))\\
    &= \proj1(z).
  \end{align*}
  Denote this path by $p_z$.
  Now if we define $C:\reflect(\sm{x:A} B(x)) \to \type$ by $C(w) \defeq B(h(w))$, we have
  \[ g \defeq \lam{z} \trans{p_z}{\proj2(z)} \;:\; \prd{z:\sm{x:A} B(x)} C(\project(z)). \]
  Thus, the assumption yields
  %
  \narrowequation{f:\prd{w:\reflect(\sm{x:A}B(x))} C(w)}
  %
  such that $f(\project(z)) = g(z)$.
  Together, $h$ and $f$ give a function
  %
  \narrowequation{k:\reflect(\sm{x:A}B(x)) \to \sm{x:A}B(x)}
  %
  defined by $k(w) \defeq (h(w),f(w))$, while $p_z$ and the equality $f(\project(z)) = g(z)$ show that $k$ is a retraction of $\project_{\sm{x:A}B(x)}$.
  Therefore, $\sm{x:A}B(x)$ is in \P.
\end{proof}

Note the similarity to the discussion in \autoref{sec:htpy-inductive}.
\index{recursion principle!for a modality}%
\index{induction principle!for a modality}%
\index{uniqueness!principle, propositional!for a modality}%
The universal property of the reflector of a reflective subuniverse is like a recursion principle with its uniqueness property, while \autoref{thm:modal-char}\ref{item:mchr2} is like the corresponding induction principle.
Unlike in \autoref{sec:htpy-inductive}, the two are not equivalent here, because of the restriction that we can only eliminate into types that lie in $\P$.
Condition~\ref{item:mchr1} of \autoref{thm:modal-char} is what fixes the disconnect.

Unsurprisingly, of course, if we have the induction principle, then we can derive the recursion principle.
We can also derive its uniqueness property, as long as we allow ourselves to eliminate into path types.
This suggests the following definition.
Note that any reflective subuniverse can be characterized by the operation $\reflect:\type\to\type$ and the functions $\project_A:A\to \reflect A$, since we have $P(A) = \isequiv(\project_A)$.

\begin{defn}\label{defn:modality}
A \define{modality}
\indexdef{modality}
is an operation $\modal:\type\to\type$ for which there are
\begin{enumerate}
\item functions $\mreturn^\modal_A:A\to\modal(A)$ for every type $A$.\label{item:modal1}
\item for every $A:\type$ and every type family $B:\modal(A)\to\type$, a function\label{item:modal2}
\begin{equation*}
\ind{\modal}:\Parens{\prd{a:A}\modal(B(\mreturn^\modal_A(a)))}\to\prd{z:\modal(A)}\modal(B(z)).
\end{equation*}
\item A path $\ind\modal(f)(\mreturn^\modal_A(a)) = f(a)$ for each $f:\prd{a:A}\modal(B(\mreturn^\modal_A(a)))$.\label{item:modal3}
\item For any $z,z':\modal(A)$, the function $\mreturn^\modal_{z=z'} : (z=z') \to \modal(z=z')$ is an equivalence.\label{item:modal4}
\end{enumerate}
We say that $A$ is \define{modal}
\indexdef{modal!type}%
\indexdef{type!modal}%
for $\modal$ if $\mreturn^\modal_A:A\to\modal(A)$ is an equivalence, and we write
\begin{myeqn}
  \modaltype\defeq\setof{X:\type | X \text{ is $\modal$-modal} }\label{eq:modaltype}
\end{myeqn}
for the type of modal types.
\end{defn}

Conditions~\ref{item:modal2} and~\ref{item:modal3} are very similar to \autoref{thm:modal-char}\ref{item:mchr2}, but phrased using $\modal B(z)$ rather than assuming $B$ to be valued in $\P$.
This allows us to state the condition purely in terms of the operation $\modal$, rather than requiring the predicate $P:\type\to\prop$ to be given in advance.
(It is not entirely satisfactory, since we still have to refer to $P$ not-so-subtly in clause~\ref{item:modal4}.
We do not know whether~\ref{item:modal4} follows from~\ref{item:modal1}--\ref{item:modal3}.)
However, the stronger-looking property of \autoref{thm:modal-char}\ref{item:mchr2} follows from \autoref{defn:modality}\ref{item:modal2} and~\ref{item:modal3}, since for any $C:\modal A \to \modaltype$ we have $C(z) \eqvsym \modal C(z)$, and we can pass back across this equivalence.

\index{universal!property!of a modality}%
As with other induction principles, this implies a universal property.

\begin{thm}\label{prop:lv_n_deptype_sec_equiv_by_precomp}
Let $A$ be a type and let $B:\modal(A)\to\modaltype$. Then the function
\begin{equation*}
(\blank\circ \mreturn^\modal_A) : \Parens{\prd{z:\modal(A)}B(z)} \to \Parens{\prd{a:A}B(\mreturn^\modal_A(a))}
\end{equation*}
is an equivalence.
\end{thm}
\begin{proof}
By definition, the operation $\ind{\modal}$ is a right inverse to $(\blank\circ \mreturn^\modal_A)$.
Thus, we only need to find a homotopy
\begin{equation*}
\prd{z:\modal(A)}s(z)= \ind{\modal}(s\circ \mreturn^\modal_A)(z)
\end{equation*}
for each $s:\prd{z:\modal(A)}B(z)$, exhibiting it as a left inverse as well.
By assumption, each $B(z)$ is modal, and hence each type $s(z)= R^\modal_X(s\circ \mreturn^\modal_A)(z)$
is also modal.
Thus, it suffices to find a function of type
\begin{equation*}
\prd{a:A}s(\mreturn^\modal_A(a))= \ind{\modal}(s\circ \mreturn^\modal_A)(\mreturn^\modal_A(a)).
\end{equation*}
which follows from \autoref{defn:modality}\ref{item:modal3}.
\end{proof}

In particular, for every type $A$ and every modal type $B$, we have an equivalence $(\modal A\to B)\eqvsym (A\to B)$.

\begin{cor}
  For any modality $\modal$, the $\modal$-modal types form a reflective subuniverse satisfying the equivalent conditions of \autoref{thm:modal-char}.
\end{cor}

Thus, modalities can be identified with reflective subuniverses closed under $\Sigma$-types.
The name \emph{modality} comes, of course, from \emph{modal logic}\index{modal!logic}, which studies logic where we can form statements such as ``possibly $A$'' (usually written $\diamond A$) or ``necessarily $A$'' (usually written $\Box A$).
The symbol $\modal$ is somewhat common for an arbitrary modal operator\index{modal!operator}. % (rather than a specific one such as $\diamond$ or $\Box$).
Under the propositions-as-types principle, a modality in the sense of modal logic corresponds to an operation on \emph{types}, and \autoref{defn:modality} seems a reasonable candidate for how such an operation should be defined.
(More precisely, we should perhaps call these \emph{idempotent, monadic} modalities; see the Notes.)
\index{idempotent!modality}%
As mentioned in \autoref{subsec:when-trunc}, we may in general use adverbs\index{adverb} to speak informally about such modalities, such as ``merely''\index{merely} for the propositional truncation and ``purely''\index{purely} for the identity modality
\index{identity!modality}%
\index{modality!identity}%
(i.e.\ the one defined by $\modal A \defeq A$).

For any modality $\modal$, we define a map $f:A\to B$ to be \define{$\modal$-connected}
\indexdef{function!.circle-connected@$\modal$-connected}%
\indexdef{.circle-connected function@$\modal$-connected function}%
if $\modal(\hfib f b)$ is contractible for all $b:B$, and to be \define{$\modal$-truncated}
\indexdef{function!.circle-truncated@$\modal$-truncated}%
\indexdef{.circle-truncated function@$\modal$-truncated function}%
if $\hfib f b$ is modal for all $b:B$.
All of the theory of \autoref{sec:connectivity},\autoref{sec:image-factorization} which doesn't involve relating $n$-types for different values of $n$ applies verbatim in this generality.
\index{orthogonal factorization system}%
\index{unique!factorization system}%
In particular, we have an orthogonal factorization system.

An important class of modalities which does \emph{not} include the $n$-trun\-ca\-tions is the \emph{left exact} modalities: those for which the functor $\modal$ preserves pullbacks as well as finite products.
\index{topology!Lawvere-Tierney}%
These are a categorification of ``Lawvere-Tierney\index{Lawvere}\index{Tierney} topologies'' in elementary topos\index{topos} theory,
and correspond in higher-categorical semantics to sub-$(\infty,1)$-toposes.
\index{.infinity1-topos@$(\infty,1)$-topos}%
However, this is beyond the scope of this book.

Some particular examples of modalities other than $n$-truncation can be found in the exercises.

\index{modality|)}


\end{document}
