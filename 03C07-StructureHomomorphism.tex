\documentclass[12pt]{article}
\usepackage{pmmeta}
\pmcanonicalname{StructureHomomorphism}
\pmcreated{2013-03-22 12:43:22}
\pmmodified{2013-03-22 12:43:22}
\pmowner{almann}{2526}
\pmmodifier{almann}{2526}
\pmtitle{structure homomorphism}
\pmrecord{14}{33021}
\pmprivacy{1}
\pmauthor{almann}{2526}
\pmtype{Definition}
\pmcomment{trigger rebuild}
\pmclassification{msc}{03C07}
\pmsynonym{homomorphism}{StructureHomomorphism}
\pmsynonym{morphism}{StructureHomomorphism}
\pmsynonym{monomorphism}{StructureHomomorphism}
\pmsynonym{epimorphism}{StructureHomomorphism}
\pmsynonym{bimorphism}{StructureHomomorphism}
\pmsynonym{embedding}{StructureHomomorphism}
\pmsynonym{isomorphism}{StructureHomomorphism}
\pmsynonym{endomorphism}{StructureHomomorphism}
\pmsynonym{automorphism}{StructureHomomorphism}
\pmrelated{AxiomaticTheoryOfSupercategories}
\pmdefines{structure morphism}
\pmdefines{structure monomorphism}
\pmdefines{structure epimorphism}
\pmdefines{structure bimorphism}
\pmdefines{structure embedding}
\pmdefines{structure isomorphism}
\pmdefines{structure endomorphism}
\pmdefines{structure automorphism}

% this is the default PlanetMath preamble.  as your knowledge
% of TeX increases, you will probably want to edit this, but
% it should be fine as is for beginners.

% almost certainly you want these
\usepackage{amssymb}
\usepackage{amsmath}
\usepackage{amsfonts}

% used for TeXing text within eps files
%\usepackage{psfrag}
% need this for including graphics (\includegraphics)
%\usepackage{graphicx}
% for neatly defining theorems and propositions
%\usepackage{amsthm}
% making logically defined graphics
\usepackage[arrow,curve,poly,arc,2cell,frame,web]{xypic}

% there are many more packages, add them here as you need them

% define commands here
\newcommand{\br}{[\![}
\newcommand{\rb}{]\!]}
\newcommand{\oq}{\text{``}}
\newcommand{\cq}{\text{''}}


\newcommand{\im}{\mathbf{Im}}
\newcommand{\dom}{\mathbf{Dom}}


\newcommand{\Or}{\vee}
\newcommand{\Implies}{\Rightarrow}
\newcommand{\Iff}{\Leftrightarrow}
\newcommand{\proves}{\vdash}
\renewcommand{\And}{\wedge}
\newcommand{\Sup}{\bigwedge}
\newcommand{\Inf}{\bigvee}
\newcommand{\Z}{\mathbb{Z}}
\newcommand{\F}{\mathbb{F}}
\newcommand{\Q}{\mathbb{Q}}
\newcommand{\R}{\mathbb{R}}
\newcommand{\C}{\mathbb{C}}
\newcommand{\Nat}{\mathbb{N}}
\newcommand{\M}{\mathfrak{M}}
\newcommand{\N}{\mathfrak{N}}
\newcommand{\A}{\mathfrak{A}}
\newcommand{\B}{\mathfrak{B}}
\newcommand{\K}{\mathfrak{K}}
\newcommand{\G}{\mathbb{G}}
\newcommand{\Def}{\overset{\operatorname{def}}{:=}}



\newcommand{\spec}{\text{{\bf Spec}}}
\newcommand{\stab}{\text{{\bf Stab}}}
\newcommand{\ann}{\text{{\bf Ann}}}
\newcommand{\irr}{\text{{\bf Irr}}}
\newcommand{\qt}{\text{{\bf Qt}}}
\newcommand{\st}{\mathcal{Qt}}
\newcommand{\ro}{\mathbf{r.o.}}


\newcommand{\Endo}{\text{{\bf End}}}
\newcommand{\mat}{\text{{\bf Mat}}}
\newcommand{\der}{\text{{\bf Der}}}
\newcommand{\rad}{\text{{\bf Rad}}}
\newcommand{\trd}{\text{{\bf tr.d.}}}
\newcommand{\cl}{\text{{\bf acl}}}
\newcommand{\Int}{\text{{\bf int}}}
\newcommand{\V}{\mathbb{V}}
\newcommand{\D}{\mathbf{D}}

\newcommand{\del}{\partial}
\renewcommand{\O}{\mathcal{O}}
\newcommand{\aut}{\mathbf{Aut}}
\newcommand{\height}{\text{\bf Height}}
\newcommand{\coheight}{\text{\bf Co-height}}

\newcommand{\lcm}{\operatorname{lcm}}

\newcommand{\Gal}{\operatorname{Gal}}
\newcommand{\x}{\mathbf{x}}
\newcommand{\y}{\mathbf{y}}
\newcommand{\inner}[2]{\langle #1|#2\rangle}
\renewcommand{\r}{{r}}
\renewcommand{\t}{{t}}

\newcommand{\restr}{\upharpoonright}
\newcommand{\Matrix}[4]{\left(\begin{array}{cc} #1 & #2 \\ #3 & #4 
\end{array}\right)}
\begin{document}
\PMlinkescapeword{embedding}

Let $\Sigma$ be a fixed signature, and $\A$ and $\B$ be two structures for $\Sigma$.  The interesting functions from $\A$ to $\B$ are the ones that preserve the structure.

A function $f\colon \A \to \B$ is said to be a \emph{homomorphism} (or simply \emph{morphism}) if and only if:
\begin{enumerate}
 \item For every constant symbol $c$ of $\Sigma$, $f(c^\A)=c^\B$.
 \item For every natural number $n$ and every $n$-ary function symbol $F$ of
 $\Sigma$,
\[
f(F^\A(a_1,...,a_n))=F^\B(f(a_1),...,f(a_n)).
\]
 \item For every natural number $n$ and every $n$-ary relation symbol $R$
 of $\Sigma$,
\[
R^\A(a_1, \ldots ,a_n) \Implies R^\B(f(a_1), \ldots,f(a_n)).
\]
\end{enumerate}

Homomorphisms with various additional properties have special names:
\begin{itemize}
  \item An \PMlinkname{injective}{Injective} homomorphism is called a \emph{monomorphism}.
  \item A surjective homomorphism is called an \emph{epimorphism}.
  \item A bijective homomorphism is called a \emph{bimorphism}.
  \item An injective homomorphism $f$ is called an \emph{embedding} if, for every natural number $n$ and every $n$-ary relation symbol $R$ of $\Sigma$,
\[
R^\B(f(a_1), \ldots,f(a_n)) \Implies R^\A(a_1, \ldots ,a_n),
\]
the converse of condition 3 above, holds.
  \item A surjective embedding is called an \emph{isomorphism}.
  \item A homomorphism from a structure to itself (\PMlinkname{e.g.}{Eg}, $f\colon \A \to \A$) is called an \emph{\PMlinkescapetext{endomorphism}}.
  \item An isomorphism from a structure to itself is called an \emph{automorphism}.
\end{itemize}
%%%%%
%%%%%
\end{document}
