\documentclass[12pt]{article}
\usepackage{pmmeta}
\pmcanonicalname{Alphabet}
\pmcreated{2013-03-22 12:15:58}
\pmmodified{2013-03-22 12:15:58}
\pmowner{mathcam}{2727}
\pmmodifier{mathcam}{2727}
\pmtitle{alphabet}
\pmrecord{7}{31681}
\pmprivacy{1}
\pmauthor{mathcam}{2727}
\pmtype{Definition}
\pmcomment{trigger rebuild}
\pmclassification{msc}{03C07}
\pmsynonym{powers of an alphabet}{Alphabet}
\pmrelated{KleeneStar}
\pmrelated{Substring}
\pmrelated{Language}
\pmrelated{HuffmanCoding}
\pmrelated{Word}

\usepackage{amssymb}
\usepackage{amsmath}
\usepackage{amsfonts}
\begin{document}
An \emph{alphabet} $\Sigma$ is a nonempty finite set such that every string formed by elements of $\Sigma$ can be decomposed uniquely into elements of $\Sigma$.

For example, $\{b,lo,g,bl,og\}$
is not a valid alphabet because the string $blog$ can be broken up in two ways: $\mbox{b lo g}$ and $\mbox{bl og}$.
$\{\mathbb{C}a,\ddot{n}a,{\rm d},a\}$ \emph{is} a valid alphabet, because there
is only one way to fully break up any given string formed from it.

If $\Sigma$ is our alphabet and $n \in \mathbb{Z}^+$,
we define the following as the \emph{powers of $\Sigma$}:
\begin{itemize}
\item $\Sigma^0 = {\lambda }$, where $\lambda$ stands for the empty string.
\item $\Sigma^n = \{xy|x \in \Sigma, y \in \Sigma^{n-1}\}$ ($xy$ is the juxtaposition of $x$ and $y$)
\end{itemize}

So, $\Sigma^n$ is the set of all strings formed from $\Sigma$ of length $n$.
%%%%%
%%%%%
%%%%%
\end{document}
