\documentclass[12pt]{article}
\usepackage{pmmeta}
\pmcanonicalname{EquivalentFormulationOfSubstitutability}
\pmcreated{2013-03-22 19:35:57}
\pmmodified{2013-03-22 19:35:57}
\pmowner{CWoo}{3771}
\pmmodifier{CWoo}{3771}
\pmtitle{equivalent formulation of substitutability}
\pmrecord{6}{42590}
\pmprivacy{1}
\pmauthor{CWoo}{3771}
\pmtype{Definition}
\pmcomment{trigger rebuild}
\pmclassification{msc}{03B05}
\pmclassification{msc}{03B10}

\usepackage{amssymb,amscd}
\usepackage{amsmath}
\usepackage{amsfonts}
\usepackage{mathrsfs}
\usepackage{proof}
\usepackage{bussproofs}

% used for TeXing text within eps files
%\usepackage{psfrag}
% need this for including graphics (\includegraphics)
%\usepackage{graphicx}
% for neatly defining theorems and propositions
\usepackage{amsthm}
% making logically defined graphics
%%\usepackage{xypic}
\usepackage{pst-plot}
\usepackage{multicol}
\usepackage{enumerate}
\usepackage{tabls}

% define commands here
\newcommand*{\abs}[1]{\left\lvert #1\right\rvert}
\newtheorem{prop}{Proposition}
\newtheorem{thm}{Theorem}
\newtheorem{lem}{Lemma}
\newtheorem{cor}{Corollary}
\newtheorem{ex}{Example}

\begin{document}
\begin{prop} Suppose a variable $x$ occurs free in a wff $A$.  A term $t$ is free for $x$ in $A$ iff no variables in $t$ are bound by a quantifier in $A[t/x]$. \end{prop}
\begin{proof} We do induction on the complexity of $A$.  
\begin{itemize}
\item If $A$ is atomic, then any $t$ is free for $x$ in $A$, and clearly $A[t/x]$ is just $A$, which has no bound variables.  
\item If $A$ is of the form $B\to C$, then $t$ is free for $x$ in $A$ iff $t$ is free for $x$ in both $B$ and $C$ iff no variables in $t$ are bound in either $B$ or $C$ iff no variables in $t$ are bound in $A$.  
\item Finally, suppose $A$ is of the form $\exists y B$.  Since $x$ is free in $A$, $x$ is not $y$, and $t$ is free for $x$ in $A$ iff $y$ is not in $t$ and $t$ is free for $x$ in $B$ iff, by induction, $y$ is not in $t$ and no variables of $t$ are bound in $B[t/x]$ iff no variables of $t$ are bound in $\exists y B[t/x]$, which is just $A[t/x]$ (since $x\ne y$).
\end{itemize}
\end{proof}
If $x$ does not occur free in $A$ (either $x$ occurs bound in $A$ or not at all in $A$), then $t$ is obviously free for $x$ in $A$, but $A[t/x]$ is just $A$, and there is no guarantee that variables in $t$ are bound in $A$ or not.

In the special case where $t$ is a variable $y$, we see that $y$ is free for $x$ in $A$ iff $y$ is not bound in $A[y/x]$, provided that $x$ occurs free in $A$.  In other words, $y$ is free for $x$ in $A$ iff no free occurrences of $x$ in $A$ are in the scope of $Qy$, where $Q$ is either $\exists$ or $\forall$.  So if $y$ is not bound in $A$, $y$ is free for $x$ in $A$, regardless of whether $x$ is free or bound in $A$.  Also, $x$ is always free for $x$ in $A$.

%%%%%
%%%%%
\end{document}
