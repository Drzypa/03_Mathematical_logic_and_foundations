\documentclass[12pt]{article}
\usepackage{pmmeta}
\pmcanonicalname{Pointwise}
\pmcreated{2013-03-22 15:25:00}
\pmmodified{2013-03-22 15:25:00}
\pmowner{lars_h}{9802}
\pmmodifier{lars_h}{9802}
\pmtitle{pointwise}
\pmrecord{4}{37260}
\pmprivacy{1}
\pmauthor{lars_h}{9802}
\pmtype{Definition}
\pmcomment{trigger rebuild}
\pmclassification{msc}{03-00}
\pmclassification{msc}{08-00}
\pmdefines{pointwise operation}
\pmdefines{pointwise addition}
\pmdefines{pointwise muliplication}

\endmetadata

% this is the default PlanetMath preamble.  as your knowledge
% of TeX increases, you will probably want to edit this, but
% it should be fine as is for beginners.

% almost certainly you want these
\usepackage{amssymb}
\usepackage{amsmath}
\usepackage{amsfonts}

% used for TeXing text within eps files
%\usepackage{psfrag}
% need this for including graphics (\includegraphics)
%\usepackage{graphicx}
% for neatly defining theorems and propositions
%\usepackage{amsthm}
% making logically defined graphics
%%%\usepackage{xypic}

% there are many more packages, add them here as you need them

% define commands here
\begin{document}
When concepts (properties, operations, etc.) on a set $Y$ 
are extended to functions $f\colon X \longrightarrow Y$ 
by treating each function value $f(x)$ in isolation, the 
extended concept is often qualified with the word 
\emph{pointwise}. One example is pointwise convergence 
of functions---a sequence $\{f_n\}_{n=1}^\infty$ of 
functions $X \longrightarrow Y$ converges pointwise to 
a function $f$ if \(\lim_{n \rightarrow \infty} f_n(x) = f(x)\) 
for all \(x \in X\).

An important \PMlinkescapetext{class} of pointwise concepts 
are the \emph{pointwise operations}---operations defined 
on functions by applying the operations to function values 
separately for each point in the domain of definition. These 
include
\begin{align*}
  (f+g)(x) ={}& f(x)+g(x) && \text{(pointwise addition)}\\
  (f \cdot g)(x) ={}& f(x) \cdot g(x) &&
    \text{(pointwise multiplication)}\\
  (\lambda f)(x) ={}& \lambda \cdot f(x) &&
    \text{(pointwise multiplication by scalar)}
\end{align*}
where the identities hold for all \(x \in X\). Pointwise 
operations inherit such properties as associativity, commutativity, 
and distributivity from corresponding operations on $Y$.

An example of an operation on functions which is \emph{not} 
pointwise is the \PMlinkname{convolution}{Convolution} product.
%%%%%
%%%%%
\end{document}
