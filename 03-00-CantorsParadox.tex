\documentclass[12pt]{article}
\usepackage{pmmeta}
\pmcanonicalname{CantorsParadox}
\pmcreated{2013-03-22 13:04:39}
\pmmodified{2013-03-22 13:04:39}
\pmowner{Henry}{455}
\pmmodifier{Henry}{455}
\pmtitle{Cantor's paradox}
\pmrecord{6}{33488}
\pmprivacy{1}
\pmauthor{Henry}{455}
\pmtype{Definition}
\pmcomment{trigger rebuild}
\pmclassification{msc}{03-00}

% this is the default PlanetMath preamble.  as your knowledge
% of TeX increases, you will probably want to edit this, but
% it should be fine as is for beginners.

% almost certainly you want these
\usepackage{amssymb}
\usepackage{amsmath}
\usepackage{amsfonts}

% used for TeXing text within eps files
%\usepackage{psfrag}
% need this for including graphics (\includegraphics)
%\usepackage{graphicx}
% for neatly defining theorems and propositions
%\usepackage{amsthm}
% making logically defined graphics
%%%\usepackage{xypic}

% there are many more packages, add them here as you need them

% define commands here
%\PMlinkescapeword{theory}
\begin{document}
\emph{Cantor's paradox} demonstrates that there can be no largest cardinality.  In particular, there must be an unlimited number of infinite cardinalities.  For suppose that $\alpha$ were the largest cardinal.  Then we would have $|\mathcal{P}(\alpha)|=|\alpha|$.  (Here $\mathcal{P}(\alpha)$ denotes the power set of $\alpha$.)  Suppose $f:\alpha\rightarrow\mathcal{P}(\alpha)$ is a bijection proving their equicardinality.  Then $X=\{\beta\in\alpha\mid \beta\not\in f(\beta)\}$ is a subset of $\alpha$, and so there is some $\gamma\in\alpha$ such that $f(\gamma)=X$.  But $\gamma\in X\leftrightarrow\gamma\notin X$, which is a paradox.

The key part of the argument strongly resembles Russell's paradox, which is in some sense a generalization of this paradox.

Besides allowing an unbounded number of cardinalities as ZF set theory does, this paradox could be avoided by a few other tricks, for instance by not allowing the construction of a power set or by adopting paraconsistent logic.
%%%%%
%%%%%
\end{document}
