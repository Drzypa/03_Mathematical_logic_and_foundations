\documentclass[12pt]{article}
\usepackage{pmmeta}
\pmcanonicalname{Subformula}
\pmcreated{2013-03-22 12:42:52}
\pmmodified{2013-03-22 12:42:52}
\pmowner{CWoo}{3771}
\pmmodifier{CWoo}{3771}
\pmtitle{subformula}
\pmrecord{13}{33001}
\pmprivacy{1}
\pmauthor{CWoo}{3771}
\pmtype{Definition}
\pmcomment{trigger rebuild}
\pmclassification{msc}{03C07}
\pmclassification{msc}{03B10}
\pmrelated{Substitutability}
\pmdefines{literal subformula}

% this is the default PlanetMath preamble.  as your knowledge
% of TeX increases, you will probably want to edit this, but
% it should be fine as is for beginners.

% almost certainly you want these
\usepackage{amssymb}
\usepackage{amsmath}
\usepackage{amsfonts}

% used for TeXing text within eps files
%\usepackage{psfrag}
% need this for including graphics (\includegraphics)
%\usepackage{graphicx}
% for neatly defining theorems and propositions
%\usepackage{amsthm}
% making logically defined graphics
\usepackage[arrow,curve,poly,arc,2cell,frame,web]{xypic}

% there are many more packages, add them here as you need them

% define commands here
\newcommand{\br}{[\![}
\newcommand{\rb}{]\!]}
\newcommand{\oq}{\text{``}}
\newcommand{\cq}{\text{''}}


\newcommand{\im}{\mathbf{Im}}
\newcommand{\dom}{\mathbf{Dom}}


\newcommand{\Or}{\vee}
\newcommand{\Implies}{\Rightarrow}
\newcommand{\Iff}{\Leftrightarrow}
\newcommand{\proves}{\vdash}
\renewcommand{\And}{\wedge}
\newcommand{\Sup}{\bigwedge}
\newcommand{\Inf}{\bigvee}
\newcommand{\Z}{\mathbb{Z}}
\newcommand{\F}{\mathbb{F}}
\newcommand{\Q}{\mathbb{Q}}
\newcommand{\R}{\mathbb{R}}
\newcommand{\C}{\mathbb{C}}
\newcommand{\Nat}{\mathbb{N}}
\newcommand{\M}{\mathfrak{M}}
\newcommand{\N}{\mathfrak{N}}
\newcommand{\A}{\mathfrak{A}}
\newcommand{\B}{\mathfrak{B}}
\newcommand{\K}{\mathfrak{K}}
\newcommand{\G}{\mathbb{G}}
\newcommand{\Def}{\overset{\operatorname{def}}{:=}}



\newcommand{\spec}{\text{{\bf Spec}}}
\newcommand{\stab}{\text{{\bf Stab}}}
\newcommand{\ann}{\text{{\bf Ann}}}
\newcommand{\irr}{\text{{\bf Irr}}}
\newcommand{\qt}{\text{{\bf Qt}}}
\newcommand{\st}{\mathcal{Qt}}
\newcommand{\ro}{\mathbf{r.o.}}


\newcommand{\Endo}{\text{{\bf End}}}
\newcommand{\mat}{\text{{\bf Mat}}}
\newcommand{\der}{\text{{\bf Der}}}
\newcommand{\rad}{\text{{\bf Rad}}}
\newcommand{\trd}{\text{{\bf tr.d.}}}
\newcommand{\cl}{\text{{\bf acl}}}
\newcommand{\Int}{\text{{\bf int}}}
\newcommand{\V}{\mathbb{V}}
\newcommand{\D}{\mathbf{D}}

\newcommand{\del}{\partial}
\renewcommand{\O}{\mathcal{O}}
\newcommand{\aut}{\mathbf{Aut}}
\newcommand{\height}{\text{\bf Height}}
\newcommand{\coheight}{\text{\bf Co-height}}

\newcommand{\lcm}{\operatorname{lcm}}

\newcommand{\Gal}{\operatorname{Gal}}
\newcommand{\x}{\mathbf{x}}
\newcommand{\y}{\mathbf{y}}
\newcommand{\inner}[2]{\langle #1|#2\rangle}
\renewcommand{\r}{{r}}
\renewcommand{\t}{{t}}

\newcommand{\restr}{\upharpoonright}
\newcommand{\Matrix}[4]{\left(\begin{array}{cc} #1 & #2 \\ #3 & #4 
\end{array}\right)}

\begin{document}
Let $L$ be a first order language and suppose $\varphi$ is a formula of $L$.  A \emph{subformula} of $\varphi$ is defined as any of the following:
\begin{enumerate}
\item $\varphi$ is a subformula of $\varphi$;
\item if $\neg \psi$ is a subformula of $\varphi$ for some $L$-formula $\psi$, then so is $\psi$;
\item if $\alpha\wedge \beta$ is a subformula of $\varphi$ for some $L$-formulas $\alpha,\beta$, then so are $\alpha$ and $\beta$;
\item if $\exists x (\psi)$ is a subformula of $\varphi$ for some $L$-formula $\psi$, then so is $\psi[t/x]$ for any $t$ free for $x$ in $\psi$. 
\end{enumerate}
\textbf{Remark}.  And if the language contains a modal connective, say $\square$, then we also have
\begin{enumerate}
\setcounter{enumi}{4}
\item if $\square \alpha$ is a subformula of $\varphi$ for some $L$-formula $\alpha$, then so is $\alpha$. 
\end{enumerate}

The phrase ``$t$ is free for $x$ in $\psi$'' means that after substituting the term $t$ for the variable $x$ in the formula $\psi$, no free variables in $t$ will become bound variables in $\psi[t/x]$.  

For example, if $\varphi=\alpha \vee \beta$, then $\alpha$ and $\beta$ are subformulas of $\varphi$.  This is so because  $\alpha\vee \beta = \neg(\neg \alpha \wedge \neg \beta)$, so that $\neg \alpha\wedge \neg\beta$ is a subformula of $\varphi$ by applications of 1 followed by 2 above.  By 3 above, $\neg \alpha$ and $\neg \beta$ are subformulas of $\varphi$.  Therefore, by 2 again, $\alpha$ and $\beta$ are subformulas of $\varphi$.

For another example, if $\varphi=\exists x (\exists y (x^2+y^2=1))$, then $\exists y (t^2+y^2=1)$ is a subformula of $\varphi$ as long as $t$ is a term that does not contain the variable $y$.  Therefore, if $t=y+2$, then $\exists y ((y+2)^2+y^2=1)$ is not a subformula of $\varphi$.  In fact, if $y\in \mathbb{R}$, the equation $(y+2)^2+y^2=1$ is never true.

Finally, it is easy to see (by induction) that if $\alpha$ is a subformula of $\psi$ and $\psi$ is a subformula of $\varphi$, then $\alpha$ is a subformula of $\varphi$.  ``Being a subformula of'' is a reflexive transitive relation on $L$-formulas.

\textbf{Remark}.  There is also the notion of a \emph{literal subformula} of a formula $\varphi$.  A formula $\psi$ is a literal subformula of $\varphi$ if it is a subformula of $\varphi$ obtained in any one of the first three ways above, or if $\exists x (\psi)$ is a literal subformula of $\varphi$.

Note that any literal subformula of $\varphi$ is a subformula of $\varphi$, for if $\varphi=\exists x (\psi)$, then $x$ occurs free in $\psi$ and $\psi=\psi[x/x]$.

In the second example above, $\exists y (x^2+y^2=1)$ and $x^2+y^2=1$ are both literal subformulas of $\varphi=\exists x (\exists y (x^2+y^2=1))$.

%%%%%
%%%%%
\end{document}
