\documentclass[12pt]{article}
\usepackage{pmmeta}
\pmcanonicalname{OnetooneFunctionFromOntoFunction}
\pmcreated{2013-03-22 16:26:55}
\pmmodified{2013-03-22 16:26:55}
\pmowner{mathcam}{2727}
\pmmodifier{mathcam}{2727}
\pmtitle{one-to-one function from onto function}
\pmrecord{8}{38604}
\pmprivacy{1}
\pmauthor{mathcam}{2727}
\pmtype{Definition}
\pmcomment{trigger rebuild}
\pmclassification{msc}{03E25}
%\pmkeywords{one-to-one}
%\pmkeywords{onto}
%\pmkeywords{choice function}
%\pmkeywords{axiom of choice}
\pmrelated{function}
\pmrelated{ChoiceFunction}
\pmrelated{AxiomOfChoice}
\pmrelated{set}
\pmrelated{onto}
\pmrelated{SchroederBernsteinTheorem}
\pmrelated{AnInjectionBetweenTwoFiniteSetsOfTheSameCardinalityIsBijective}
\pmrelated{ASurjectionBetweenTwoFiniteSetsOfTheSameCardinalityIsBijective}
\pmrelated{Set}
\pmrelated{Surjective}

% this is the default PlanetMath preamble.  as your knowledge
% of TeX increases, you will probably want to edit this, but
% it should be fine as is for beginners.

% almost certainly you want these
\usepackage{amssymb}
\usepackage{amsmath}
\usepackage{amsfonts}
\usepackage{amsthm}
\usepackage{mathrsfs}

% used for TeXing text within eps files
%\usepackage{psfrag}
% need this for including graphics (\includegraphics)
%\usepackage{graphicx}
% for neatly defining theorems and propositions
%\usepackage{amsthm}
% making logically defined graphics
%%%\usepackage{xypic}

% there are many more packages, add them here as you need them

% define commands here
\theoremstyle{plain}
\newtheorem*{thm}{Theorem}

\begin{document}
\begin{thm}
Given an onto function from a set $A$ to a set $B$, there exists a one-to-one function from $B$ to $A$.
\end{thm}
\begin{proof}
Suppose $f:A\rightarrow B$ is onto, and define $\mathcal{F}=\big\{f^{-1}(\{b\}):b\in B\big\}$; that is, $\mathcal{F}$ is the set containing the pre-image of each singleton subset of $B$. Since $f$ is onto, no element of $\mathcal{F}$ is empty, and since $f$ is a function, the elements of $\mathcal{F}$ are mutually disjoint, for if $a\in f^{-1}(\{b_1\})$ and $a\in f^{-1}(\{b_2\})$, we have $f(a)=b_1$ and $f(a)=b_2$, whence $b_1=b_2$. Let $\mathscr{C}:\mathcal{F}\rightarrow\bigcup\mathcal{F}$ be a choice function, noting that $\bigcup\mathcal{F}=A$, and define $g:B\rightarrow A$ by $g(b)=\mathscr{C}\big(f^{-1}(\{b\})\big)$. To see that $g$ is one-to-one, let $b_1,b_2\in B$, and suppose that $g(b_1)=g(b_2)$. This gives 
$\mathscr{C}\big(f^{-1}(\{b_1\})\big)=\mathscr{C}\big(f^{-1}(\{b_2\})\big)$, but since the elements of $\mathcal{F}$ are disjoint, this implies that $f^{-1}(\{b_1\})=f^{-1}(\{b_2\})$, and thus $b_1=b_2$. So $g$ is a one-to-one function from $B$ to $A$. 
\end{proof}
%%%%%
%%%%%
\end{document}
