\documentclass[12pt]{article}
\usepackage{pmmeta}
\pmcanonicalname{PigeonholePrinciple}
\pmcreated{2013-03-22 11:53:32}
\pmmodified{2013-03-22 11:53:32}
\pmowner{djao}{24}
\pmmodifier{djao}{24}
\pmtitle{pigeonhole principle}
\pmrecord{11}{30502}
\pmprivacy{1}
\pmauthor{djao}{24}
\pmtype{Theorem}
\pmcomment{trigger rebuild}
\pmclassification{msc}{03E05}
\pmclassification{msc}{03B22}
\pmclassification{msc}{03-01}
\pmclassification{msc}{03-00}
\pmsynonym{box principle}{PigeonholePrinciple}
\pmsynonym{Dirichlet principle}{PigeonholePrinciple}

\usepackage{amssymb}
\usepackage{amsmath}
\usepackage{amsfonts}
\usepackage{graphicx}
%%%%\usepackage{xypic}
\begin{document}
For any natural number $n$, there does not exist a bijection between $n$ and a proper subset of $n$.

The name of the theorem is based upon the observation that pigeons will not occupy a pigeonhole that already contains a pigeon, so there is no way to fit $n$ pigeons in fewer than $n$ pigeonholes.
%%%%%
%%%%%
%%%%%
%%%%%
\end{document}
