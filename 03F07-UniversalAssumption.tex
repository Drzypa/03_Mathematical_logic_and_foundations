\documentclass[12pt]{article}
\usepackage{pmmeta}
\pmcanonicalname{UniversalAssumption}
\pmcreated{2013-03-22 17:11:47}
\pmmodified{2013-03-22 17:11:47}
\pmowner{Wkbj79}{1863}
\pmmodifier{Wkbj79}{1863}
\pmtitle{universal assumption}
\pmrecord{5}{39515}
\pmprivacy{1}
\pmauthor{Wkbj79}{1863}
\pmtype{Definition}
\pmcomment{trigger rebuild}
\pmclassification{msc}{03F07}
\pmclassification{msc}{03B05}
\pmsynonym{universal hypothesis}{UniversalAssumption}
\pmrelated{NecessaryAndSufficient}

\endmetadata

\usepackage{amssymb}
\usepackage{amsmath}
\usepackage{amsfonts}
\usepackage{pstricks}
\usepackage{psfrag}
\usepackage{graphicx}
\usepackage{amsthm}
%%\usepackage{xypic}

\begin{document}
Given a biconditional statement, a \emph{universal assumption} is an assumption that can be used in a proof of both the necessity and sufficiency directions.

For an example of a universal assumption, please see the entry continuous functions on the extended real numbers.
%%%%%
%%%%%
\end{document}
