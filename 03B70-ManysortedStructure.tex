\documentclass[12pt]{article}
\usepackage{pmmeta}
\pmcanonicalname{ManysortedStructure}
\pmcreated{2013-03-22 17:45:17}
\pmmodified{2013-03-22 17:45:17}
\pmowner{CWoo}{3771}
\pmmodifier{CWoo}{3771}
\pmtitle{many-sorted structure}
\pmrecord{9}{40206}
\pmprivacy{1}
\pmauthor{CWoo}{3771}
\pmtype{Definition}
\pmcomment{trigger rebuild}
\pmclassification{msc}{03B70}
\pmclassification{msc}{03B10}
\pmclassification{msc}{03C07}
\pmsynonym{many sorted structure}{ManysortedStructure}
\pmsynonym{many sorted algebra}{ManysortedStructure}
\pmdefines{many-sorted interpretation}
\pmdefines{many-sorted algebra}

\usepackage{amssymb,amscd}
\usepackage{amsmath}
\usepackage{amsfonts}
\usepackage{mathrsfs}

% used for TeXing text within eps files
%\usepackage{psfrag}
% need this for including graphics (\includegraphics)
%\usepackage{graphicx}
% for neatly defining theorems and propositions
\usepackage{amsthm}
% making logically defined graphics
%%\usepackage{xypic}
\usepackage{pst-plot}

% define commands here
\newcommand*{\abs}[1]{\left\lvert #1\right\rvert}
\newtheorem{prop}{Proposition}
\newtheorem{thm}{Theorem}
\newtheorem{ex}{Example}
\newcommand{\real}{\mathbb{R}}
\newcommand{\pdiff}[2]{\frac{\partial #1}{\partial #2}}
\newcommand{\mpdiff}[3]{\frac{\partial^#1 #2}{\partial #3^#1}}
\begin{document}
Let $L$ be a many-sorted language and $S$ the set of sorts.  A \emph{many-sorted structure} $M$ for $L$, or simply an $L$-structure consists of the following:
\begin{enumerate}
\item for each sort $s\in S$, a non-empty set $A_s$,
\item for each function symbol $f$ of sort type $(s_1,\ldots, s_n)$:
\begin{itemize}
\item if $n>1$, a function $f_M: A_{s_1}\times \cdots \times A_{s_{n-1}} \to A_{s_n}$
\item if $n=1$ (constant symbol), an element $f_M\in A_{s_1}$
\end{itemize}
\item for each relation symbol $r$ of sort type $(s_1,\ldots,s_n)$, a relation (or subset) $$r_M\subseteq A_{s_1}\times \cdots \times A_{s_n}.$$
\end{enumerate}

A \emph{many-sorted algebra} is a many-sorted structure without any relations.

\textbf{Remark}.  A many-sorted structure is a special case of a more general concept called a \emph{many-sorted interpretation}, which consists all of items 1-3 above, as well as the following:
\begin{itemize}
\item[4.] an element $x_M\in A_s$ for each variable $x$ of sort $s$.
\end{itemize}

\textbf{Examples}. 
\begin{enumerate}
\item 
A left module over a ring can be thought of as a two-sorted algebra (say, with sorts $\lbrace s_1,s_2\rbrace$), for there are
\begin{itemize}
\item there are two non-empty sets $M$ (corresponding to sort $s_1$) and $R$ (corresponding to sort $s_2$), where
\item $M$ has the structure of an abelian group (equipped with three operations: $0,-,+$, corresponding to function symbols of sort types $(s_1), (s_1,s_1)$, and $(s_1,s_1,s_1)$)
\item $R$ has the structure of a ring (equipped with at least four operations: $0,-,+,\times$, corresponding to function symbols of sort types $(s_2), (s_2,s_2)$ and $(s_2,s_2,s_2)$ for $+$ and $\times$, and possibly a fifth operation $1$ of sort type $(s_2)$)
\item a function $\cdot: R\times M\to M$, which corresponds to a function symbol of sort type $(s_2,s_1,s_1)$.  Clearly, $\cdot$ is the scalar multiplication on the module $M$.
\end{itemize}
For a right module over a ring, one merely replaces the sort type of the last function symbol by the sort type $(s_1,s_2,s_1)$.
\item
A deterministic semiautomaton $A=(S,\Sigma,\delta)$ is a two-sorted algebra, where 
\begin{itemize}
\item $S$ and $\Sigma$ are non-empty sets, corresponding to sorts, say, $s_1$ and $s_2$,
\item $\delta: S\times \Sigma \to S$ is a function corresponding to a function symbol of sort type $(s_1,s_2,s_1)$.  
\end{itemize}
\item
A deterministic automaton $B=(S,\Sigma,\delta,\sigma,F)$ is a two-sorted structure, where 
\begin{itemize}
\item $(S,\Sigma,\delta)$ is a semiautomaton discussed earlier,
\item $\sigma$ is a constant corresponding to a nullary function symbol of sort type $(s_1)$,
\item $F$ is a unary relation corresponding to a relation symbol of sort type $(s_1)$.  
\end{itemize}
Because $F$ is a relation, $B$ is not an algebra.
\item
A complete sequential machine $M=(S,\Sigma,\Delta,\delta,\lambda)$ is a three-sorted algebra, where 
\begin{itemize}
\item $(S,\Sigma,\delta)$ is a semiautomaton discussed earlier,
\item $\Delta$ is a non-empty sets, corresponding to sort, say, $s_3$,
\item $\lambda: S\times \Sigma \to \Delta$ is a function corresponding to a function symbol of sort type $(s_1,s_2,s_3)$.
\end{itemize}
\end{enumerate}

\begin{thebibliography}{9}
\bibitem{dm} J. D. Monk, \emph{Mathematical Logic}, Springer, New York (1976).
\end{thebibliography}
%%%%%
%%%%%
\end{document}
