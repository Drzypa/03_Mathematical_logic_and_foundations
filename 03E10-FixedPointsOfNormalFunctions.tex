\documentclass[12pt]{article}
\usepackage{pmmeta}
\pmcanonicalname{FixedPointsOfNormalFunctions}
\pmcreated{2013-03-22 13:28:59}
\pmmodified{2013-03-22 13:28:59}
\pmowner{Henry}{455}
\pmmodifier{Henry}{455}
\pmtitle{fixed points of normal functions}
\pmrecord{6}{34054}
\pmprivacy{1}
\pmauthor{Henry}{455}
\pmtype{Definition}
\pmcomment{trigger rebuild}
\pmclassification{msc}{03E10}
\pmrelated{ProofOfPowerRule}
\pmrelated{LeibnizNotation}
\pmrelated{ProofOfProductRule}
\pmrelated{ProofOfSumRule}
\pmrelated{SumRule}
\pmrelated{DirectionalDerivative}
\pmrelated{NewtonsMethod}
\pmdefines{derivative}

\endmetadata

% this is the default PlanetMath preamble.  as your knowledge
% of TeX increases, you will probably want to edit this, but
% it should be fine as is for beginners.

% almost certainly you want these
\usepackage{amssymb}
\usepackage{amsmath}
\usepackage{amsfonts}

% used for TeXing text within eps files
%\usepackage{psfrag}
% need this for including graphics (\includegraphics)
%\usepackage{graphicx}
% for neatly defining theorems and propositions
%\usepackage{amsthm}
% making logically defined graphics
%%%\usepackage{xypic}

% there are many more packages, add them here as you need them

% define commands here
%\PMlinkescapeword{theory}
\begin{document}
If $f\colon M\rightarrow\mathbf{On}$ is a function from any set of ordinals to the class of ordinals then $\operatorname{Fix}(f)=\{x\in M\mid f(x)=x\}$ is the set of fixed points of $f$.  $f^\prime$, the \emph{derivative} of $f$, is the enumerating function of $\operatorname{Fix}(f)$.

If $f$ is \PMlinkname{$\kappa$-normal}{KappaNormal} then $\operatorname{Fix}(f)$ is $\kappa$-closed and $\kappa$-normal, and therefore $f^\prime$ is also $\kappa$-normal.

For example, the function which takes an ordinal $\alpha$ to the ordinal $1+\alpha$ has a fixed point at every ordinal $\geq\omega$, so $f^\prime(\alpha)=\omega+\alpha$.
%%%%%
%%%%%
\end{document}
