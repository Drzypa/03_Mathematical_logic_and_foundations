\documentclass[12pt]{article}
\usepackage{pmmeta}
\pmcanonicalname{ConstructionOfBanachLimitUsingLimitAlongAnUltrafilter}
\pmcreated{2013-03-22 15:32:29}
\pmmodified{2013-03-22 15:32:29}
\pmowner{kompik}{10588}
\pmmodifier{kompik}{10588}
\pmtitle{construction of Banach limit using limit along an ultrafilter}
\pmrecord{8}{37437}
\pmprivacy{1}
\pmauthor{kompik}{10588}
\pmtype{Application}
\pmcomment{trigger rebuild}
\pmclassification{msc}{03E99}
\pmclassification{msc}{40A05}
\pmrelated{BanachLimit}

\endmetadata

% this is the default PlanetMath preamble. as your knowledge
% of TeX increases, you will probably want to edit this, but
% it should be fine as is for beginners.

% almost certainly you want these
\usepackage{amssymb}
\usepackage{amsmath}
\usepackage{amsfonts}
\usepackage{amsthm}

% used for TeXing text within eps files
%\usepackage{psfrag}
% need this for including graphics (\includegraphics)
%\usepackage{graphicx}
% for neatly defining theorems and propositions
%
% making logically defined graphics
%%%\usepackage{xypic}

% there are many more packages, add them here as you need them

% define commands here

\newcommand{\sR}[0]{\mathbb{R}}
\newcommand{\sC}[0]{\mathbb{C}}
\newcommand{\sN}[0]{\mathbb{N}}
\newcommand{\sZ}[0]{\mathbb{Z}}
\newcommand{\N}[0]{\mathbb{N}}


\usepackage{bbm}
\newcommand{\Z}{\mathbbmss{Z}}
\newcommand{\C}{\mathbbmss{C}}
\newcommand{\R}{\mathbbmss{R}}
\newcommand{\Q}{\mathbbmss{Q}}



\newcommand*{\norm}[1]{\lVert #1 \rVert}
\newcommand*{\abs}[1]{| #1 |}

\newcommand{\Map}[3]{#1:#2\to#3}
\newcommand{\Emb}[3]{#1:#2\hookrightarrow#3}
\newcommand{\Mor}[3]{#2\overset{#1}\to#3}

\newcommand{\Cat}[1]{\mathcal{#1}}
\newcommand{\Kat}[1]{\mathbf{#1}}
\newcommand{\Func}[3]{\Map{#1}{\Cat{#2}}{\Cat{#3}}}
\newcommand{\Funk}[3]{\Map{#1}{\Kat{#2}}{\Kat{#3}}}

\newcommand{\intrv}[2]{\langle #1,#2 \rangle}

\newcommand{\vp}{\varphi}
\newcommand{\ve}{\varepsilon}

\newcommand{\Invimg}[2]{\inv{#1}(#2)}
\newcommand{\Img}[2]{#1[#2]}
\newcommand{\ol}[1]{\overline{#1}}
\newcommand{\ul}[1]{\underline{#1}}
\newcommand{\inv}[1]{#1^{-1}}
\newcommand{\limti}[1]{\lim\limits_{#1\to\infty}}

\newcommand{\Ra}{\Rightarrow}

%fonts
\newcommand{\mc}{\mathcal}

%shortcuts
\newcommand{\Ob}{\mathrm{Ob}}
\newcommand{\Hom}{\mathrm{hom}}
\newcommand{\homs}[2]{\mathrm{hom(}{#1},{#2}\mathrm )}
\newcommand{\Eq}{\mathrm{Eq}}
\newcommand{\Coeq}{\mathrm{Coeq}}

%theorems
\newtheorem{THM}{Theorem}
\newtheorem{DEF}{Definition}
\newtheorem{PROP}{Proposition}
\newtheorem{LM}{Lemma}
\newtheorem{COR}{Corollary}
\newtheorem{EXA}{Example}

%categories
\newcommand{\Top}{\Kat{Top}}
\newcommand{\Haus}{\Kat{Haus}}
\newcommand{\Set}{\Kat{Set}}

%diagrams
\newcommand{\UnimorCD}[6]{
\xymatrix{ {#1} \ar[r]^{#2} \ar[rd]_{#4}& {#3} \ar@{-->}[d]^{#5} \\
& {#6} } }

\newcommand{\RovnostrCD}[6]{
\xymatrix@C=10pt@R=17pt{
& {#1} \ar[ld]_{#2} \ar[rd]^{#3} \\
{#4} \ar[rr]_{#5} && {#6} } }

\newcommand{\RovnostrCDii}[6]{
\xymatrix@C=10pt@R=17pt{
{#1} \ar[rr]^{#2} \ar[rd]_{#4}&& {#3} \ar[ld]^{#5} \\
& {#6} } }

\newcommand{\RovnostrCDiiop}[6]{
\xymatrix@C=10pt@R=17pt{
{#1}  && {#3} \ar[ll]_{#2}  \\
& {#6} \ar[lu]^{#4} \ar[ru]_{#5} } }

\newcommand{\StvorecCD}[8]{
\xymatrix{
{#1} \ar[r]^{#2} \ar[d]_{#4} & {#3} \ar[d]^{#5} \\
{#6} \ar[r]_{#7} & {#8}
}
}

\newcommand{\TriangCD}[6]{
\xymatrix{ {#1} \ar[r]^{#2} \ar[rd]_{#4}&
{#3} \ar[d]^{#5} \\
& {#6} } }

\newcommand{\F}{\mc F}
\newcommand{\Flim}{\operatorname{\F\text{-}\lim}}
\begin{document}
\section*{Construction of Banach limit using limit along an ultrafilter}

The existence of Banach limit is proved in mathematical analysis
usually by Hahn-Banach theorem. (This proof can be found e.g.~in
\cite{swartz}, \cite{costarapopa} or \cite{morisson}.) Here we
will show another approach using limit along a filter. In fact we
define it as an $\F$-limit of $(y_n)$, where $(y_n)$ is the
Ces\`aro mean of the sequence $(x_n)$ and $\F$ is an arbitrary
ultrafilter on $\N$.

\begin{THM}
Let $\F$ be a free ultrafilter on $\N$. Let $(x_n)$ be a \PMlinkname{bounded}{Bounded} real
sequence. Then the functional $\Map\vp{\ell_\infty}{\R}$
$$\vp(x_n)=\Flim \frac{x_1+\ldots+x_n}n$$
is a Banach limit.
\end{THM}

\begin{proof}
We first observe that $\vp$ is defined. Let us denote
$y_n:=\frac{x_1+\ldots+x_n}n$. Since $(x_n)$ is bounded, the
sequence $(y_n)$ is bounded as well. Every bounded sequence has a
limit along any ultrafilter. This means, that $\vp(x_n)=\Flim y_n$
exists.

To prove that $\vp$ is a Banach limit, we should verify its
continuity, positivity, linearity, shift-invariance and to verify
that it extends limits.

We first show the shift-invariance. By $Sx$ we denote the sequence
$x_{n+1}$ and we want to show $\vp(Sx)=\vp(x)$. We observe that
$\frac{x_1+\ldots+x_n}n - \frac{(Sx)_1+\ldots+(Sx)_n}n =
\frac{x_1+\ldots+x_n}n - \frac{x_2+\ldots+x_{n+1}}n=
\frac{x_1-x_{n+1}}n$. As the sequence $(x_n)$ is bounded, the last
expression converges to 0. Thus $\vp(x)-\vp(Sx)=\Flim
\frac{x_1-x_{n+1}}n =0$ and $\vp(x)=\vp(Sx)$.

The rest of the proof is relatively easy, we only need to use the
basic properties of a limit along a filter and of Ces\`aro mean.

Continuity: $\norm x \leq 1$ $\Ra$ $\abs{x_n}\leq 1$ $\Ra$
$\abs{y_n} \leq 1$ $\Ra$ $\abs{\vp(x)}\leq 1$.

Positivity and linearity follow from positivity and linearity of
$\F$-limit.

Extends limit: If $(x_n)$ is a convergent sequence, then its
Ces\`aro mean $(y_n)$ is convergent to the same limit.
\end{proof}

\begin{thebibliography}{1}

\bibitem{balste}
B.~Balcar and P.~{\v{S}}t\v{e}p\'anek, \emph{Teorie mno\v{z}in},
Academia,
  Praha, 1986 (Czech).

\bibitem{costarapopa}
C.~Costara and D.~Popa, \emph{Exercises in functional analysis},
Kluwer,
  Dordrecht, 2003.

\bibitem{hrjech}
K.~Hrbacek and T.~Jech, \emph{{Introduction to set theory}},
{Marcel Dekker},
  New York, 1999.

\bibitem{morisson}
T.~J. Morisson, \emph{Functional analysis: An introduction to
{B}anach space
  theory}, Wiley, 2000.

\bibitem{swartz}
Ch. Swartz, \emph{An introduction to functional analysis}, Marcel
Dekker, New
  York, 1992.

\end{thebibliography}
%%%%%
%%%%%
\end{document}
