\documentclass[12pt]{article}
\usepackage{pmmeta}
\pmcanonicalname{ConsequenceOperatorIsDeterminedByItsFixedPoints}
\pmcreated{2013-03-22 16:29:48}
\pmmodified{2013-03-22 16:29:48}
\pmowner{rspuzio}{6075}
\pmmodifier{rspuzio}{6075}
\pmtitle{consequence operator is determined by its fixed points}
\pmrecord{6}{38672}
\pmprivacy{1}
\pmauthor{rspuzio}{6075}
\pmtype{Theorem}
\pmcomment{trigger rebuild}
\pmclassification{msc}{03G10}
\pmclassification{msc}{03B22}
\pmclassification{msc}{03G25}

\endmetadata

% this is the default PlanetMath preamble.  as your knowledge
% of TeX increases, you will probably want to edit this, but
% it should be fine as is for beginners.

% almost certainly you want these
\usepackage{amssymb}
\usepackage{amsmath}
\usepackage{amsfonts}

% used for TeXing text within eps files
%\usepackage{psfrag}
% need this for including graphics (\includegraphics)
%\usepackage{graphicx}
% for neatly defining theorems and propositions
%\usepackage{amsthm}
% making logically defined graphics
%%%\usepackage{xypic}

% there are many more packages, add them here as you need them

% define commands here

\newtheorem{theorem}{Theorem}
\begin{document}
\begin{theorem}
Suppose that $C_1$ and $C_2$ are consequence operators on a set $L$ and that,
for every $X \subseteq L$, it happens that $C_1 (X) = X$ if and only if $C_2 (X)
= X$.  Then $C_1 = C_2$.
\end{theorem}

\begin{theorem}
Suppose that $C$ is a consequence operators on a set $L$.  Define $K = \{ X 
\subseteq L \mid C(X) = X\}$.  Then, for every $X \in L$, there exists a $Y \in K$
such that $X \subseteq Y$ and, for every $Z \in K$ such that $X \subseteq Z$, 
one has $Y \subseteq Z$.
\end{theorem}

\begin{theorem}
Given a set $L$, suppose that $K$ is a subset of $L$ such that, for every $X \in L$, there exists a $Y \in K$ such that $X \subseteq Y$ and, for every $Z \in K$ such that $X \subseteq Z$, one has $Y \subseteq Z$.  Then there exists a 
consequence operator $C \colon \mathcal{P}(L) \to \mathcal{P}(L)$ such that $C(X) = X$ if and only if $X \in K$.
\end{theorem}
%%%%%
%%%%%
\end{document}
