\documentclass[12pt]{article}
\usepackage{pmmeta}
\pmcanonicalname{SetClosedUnderAnOperation}
\pmcreated{2013-03-22 13:49:49}
\pmmodified{2013-03-22 13:49:49}
\pmowner{archibal}{4430}
\pmmodifier{archibal}{4430}
\pmtitle{set closed under an operation}
\pmrecord{8}{34562}
\pmprivacy{1}
\pmauthor{archibal}{4430}
\pmtype{Definition}
\pmcomment{trigger rebuild}
\pmclassification{msc}{03-00}

\endmetadata

% this is the default PlanetMath preamble.  as your knowledge
% of TeX increases, you will probably want to edit this, but
% it should be fine as is for beginners.

% almost certainly you want these
\usepackage{amssymb}
\usepackage{amsmath}
\usepackage{amsfonts}

% used for TeXing text within eps files
%\usepackage{psfrag}
% need this for including graphics (\includegraphics)
%\usepackage{graphicx}
% for neatly defining theorems and propositions
%\usepackage{amsthm}
% making logically defined graphics
%%%\usepackage{xypic}

% there are many more packages, add them here as you need them

% define commands here

\newcommand{\sR}[0]{\mathbb{R}}
\newcommand{\sC}[0]{\mathbb{C}}
\newcommand{\sN}[0]{\mathbb{N}}
\newcommand{\sZ}[0]{\mathbb{Z}}
\begin{document}
\PMlinkescapeword{type}
\PMlinkescapeword{closed}
A set $X$ is said to be {\bf closed} under some map 
$L$, if $L$ maps elements in $X$ to elements in $X$, i.e., $L:X\to X$. 
More generally, suppose $Y$ is the $n$-fold Cartesian product
$Y=X\times \cdots \times X$. If $L$ is a map $L:Y\to X$, then we
also say that $X$ is closed under the map $L$. 

The above definition has no relation with the definition of a
closed set in topology. Instead, 
one should  think of $X$ and $L$ as a closed 
system. 
%In this setting, the map is usually called an operation. 

\subsubsection*{Examples}
\begin{enumerate}
\item The set of invertible matrices is closed under matrix inversion. 
This means that the inverse of an invertible matrix is again an
invertible matrix.
\item Let $C(X)$ be the set of complex valued continuous functions on 
some topological space $X$. 
Suppose $f,g$ are functions in $C(X)$. Then we define the
pointwise product of $f$ and $g$ as the function $fg: x\mapsto f(x) g(x)$. 
Since $fg$ is continuous, we have that 
$C(X)$ is closed under pointwise multiplication. 
\end{enumerate}

In the first example, the operation is of the type $X\to X$. In the latter, 
pointwise multiplication is a map $C(X)\times C(X)\to C(X)$.

The second example illustrated the somewhat odd definition of this term. When a function is defined, its domain and codomain are part of its definition, so it's a little odd to talk about whether the function is closed or not: if you know what the function is, then you should know whether or not it is closed.  So in practice, the way the term is used is this: We have a set $X$ and we have a function $f\colon X^n\to X$. We are given a subset $Y$, and asked whether $Y$ is closed under $f$. In other words, there is a natural way to make $f$ into a function $f\colon Y^n\to X$, by restriction; the question is, does the result always lie in $Y$?  This would mean that $f$ yields a function $Y^n\to Y$, which is usually what we want.

Occasionally the word is used in a potentially confusing way.  For example, left ideals in a ring are supposed to be closed under addition (which is an example of what we just discussed) and left multiplication by arbitrary ring elements.  What this last condition means is that for every $r$ in the ring, the left ideal should be closed under the function $x\mapsto rx$.
%%%%%
%%%%%
\end{document}
