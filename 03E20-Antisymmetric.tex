\documentclass[12pt]{article}
\usepackage{pmmeta}
\pmcanonicalname{Antisymmetric}
\pmcreated{2013-03-22 12:15:50}
\pmmodified{2013-03-22 12:15:50}
\pmowner{aoh45}{5079}
\pmmodifier{aoh45}{5079}
\pmtitle{antisymmetric}
\pmrecord{14}{31666}
\pmprivacy{1}
\pmauthor{aoh45}{5079}
\pmtype{Definition}
\pmcomment{trigger rebuild}
\pmclassification{msc}{03E20}
\pmsynonym{antisymmetry}{Antisymmetric}
\pmrelated{Reflexive}
\pmrelated{Symmetric}
\pmrelated{ExteriorAlgebra}
\pmrelated{SkewSymmetricMatrix}

\endmetadata

\usepackage{amssymb}
\usepackage{amsmath}
\usepackage{amsfonts}
\begin{document}
\PMlinkescapeword{relations}
A relation $\mathcal{R}$ on $A$ is \emph{antisymmetric} iff
$\forall x, y \in A$, $(x\mathcal{R}y \land y\mathcal{R}x)\rightarrow (x=y)$.
For a finite set $A$ with $n$ elements, the number of possible antisymmetric relations is $2^n 3^{\frac{n^2-n}{2}}$ out of the $2^{n^2}$ total possible
relations.

Antisymmetric is not the same thing as ``not symmetric'', as it is possible
to have both at the same time. However, a relation $\mathcal{R}$ that is both
antisymmetric and symmetric has the condition that $ x\mathcal{R}y \Rightarrow x=y $.
There are only $2^n$ such possible relations on $A$.

An example of an antisymmetric relation on $A = \{\circ, \times, \star\}$
would be $\mathcal{R} = \{(\star,\star),(\times,\circ),(\circ,\star),(\star,\times)\}$.
One relation that isn't antisymmetric is $\mathcal{R} = \{ (\times,\circ), (\star, \circ), (\circ,\star) \} $
because we have both $\star \mathcal{R} \circ$ and $\circ \mathcal{R} \star$, but $\circ \not = \star$
%%%%%
%%%%%
%%%%%
\end{document}
