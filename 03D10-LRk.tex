\documentclass[12pt]{article}
\usepackage{pmmeta}
\pmcanonicalname{LRk}
\pmcreated{2013-03-22 19:00:31}
\pmmodified{2013-03-22 19:00:31}
\pmowner{CWoo}{3771}
\pmmodifier{CWoo}{3771}
\pmtitle{LR(k)}
\pmrecord{11}{41878}
\pmprivacy{1}
\pmauthor{CWoo}{3771}
\pmtype{Definition}
\pmcomment{trigger rebuild}
\pmclassification{msc}{03D10}
\pmclassification{msc}{68Q42}
\pmclassification{msc}{68Q05}
\pmsynonym{$LR(k)$}{LRk}
\pmrelated{LLk}

\endmetadata

\usepackage{amssymb,amscd}
\usepackage{amsmath}
\usepackage{amsfonts}
\usepackage{mathrsfs}

% used for TeXing text within eps files
%\usepackage{psfrag}
% need this for including graphics (\includegraphics)
%\usepackage{graphicx}
% for neatly defining theorems and propositions
\usepackage{amsthm}
% making logically defined graphics
%%\usepackage{xypic}
\usepackage{pst-plot}

% define commands here
\newcommand*{\abs}[1]{\left\lvert #1\right\rvert}
\newtheorem{prop}{Proposition}
\newtheorem{thm}{Theorem}
\newtheorem{ex}{Example}
\newcommand{\real}{\mathbb{R}}
\newcommand{\pdiff}[2]{\frac{\partial #1}{\partial #2}}
\newcommand{\mpdiff}[3]{\frac{\partial^#1 #2}{\partial #3^#1}}
\begin{document}
Given a word $u$ and a context-free grammar $G$, how do we determine if $u\in L(G)$?  

One way is to look for any subword $v$ of $u$ such that there is a production $A\to v$.  If this is successful, we may replace $v$ with $A$ in $u$ to obtain a word $w$, so that $w \Rightarrow u$.  We may then repeat the process on $w$ to obtain another word $x$ such that $x\Rightarrow w$ (if successful).  In the end, if everything works successfully, we arrive at the starting non-terminal symbol $\sigma$, and get a derivation $\sigma\Rightarrow^* u$ as a result, so that $u\in L(G)$.  This procedure is known as the bottom-up parsing of the word $u$.

In general, unless one is very lucky, successfully finding a derivation $\sigma\Rightarrow^* u$ requires many trials and errors, since at each stage, for a given word $u$, there may be several words $w$ such that $w\Rightarrow u$.

Nevertheless, there is a particular family of context-free grammars, called the $LR(k)$ grammars, which make the bottom-up parsing described above straightforward in the sense that, given a word $u$, once a word $w$ is found such that $w\Rightarrow_R u$, any other word $w'$ such that $w'\Rightarrow_R u$ forces $w'=w$.  Here, $\Rightarrow_R$ is known as the rightmost derivation (meaning that $u$ is obtained from $w$ by replacing the rightmost non-terminal in $w$).  The $L$ in $LR(k)$ means scanning the symbols of $u$ from left to right, $R$ stands for finding a rightmost derivation for $u$, and $k$ means having the allowance to look at up to $k$ symbols ahead while scanning.

The details are as follows:

\textbf{Definition}.  Let $G=(\Sigma,N,P,\sigma)$ be a context-free grammar such that $\sigma\to \sigma$ is not a production of $G$, and $k\ge 0$ an integer.  Suppose $U$ is any sentential form over $\Sigma$ with the following setup: $U=U_1U_2U_3$ where
\begin{itemize}
\item $U_3$ is a terminal word, 
\item $X\to U_2$ a production, and 
\item $\sigma \Rightarrow_R^* U_1XU_3 \Rightarrow_R U$.
\end{itemize}
Let $n=|U_1U_2|+k$, and $Z$ the prefix of $U$ of length $n$ (if $|U|<n$, then set $Z=U$).

Then $G$ is said to be $LR(k)$ if $W$ is another sentential form having $Z$ as a prefix, with the following setup: $W=W_1W_2W_3$, where
\begin{itemize}
\item $W_3$ is a terminal, 
\item $Y\to W_2$ is a production, and 
\item $\sigma \Rightarrow_R^* W_1YW_3 \Rightarrow_R W$,
\end{itemize}
implies that $$W_1=U_1,\qquad Y=X, \qquad \mbox{and} \qquad W_2=U_2.$$

Simply put, if $D_U$ and $D_W$ are the rightmost derivations of $U$ and $W$ respectively, and if the prefix of $U$ obtained by including $k$ symbols beyond the last replacement in $D_U$ is also a prefix of $W$, then the prefix of $U'$ obtained by including $k$ symbols beyond the last replacement in $D_U$ is also a prefix of $W'$, where $U'$ and $W'$ are words at the next to the last step in $D_U$ and $D_W$ respectively.  In particular, if $U=W$, then $U'=W'$.  This implies that any derivable in an $LR(k)$ grammar has a unique rightmost derivation, hence
\begin{prop} Any $LR(k)$ grammar is unambiguous. \end{prop}

\textbf{Examples}.
\begin{itemize}
\item Let $G$ be the grammar consisting of one non-terminal symbol $\sigma$ (which is also the final non-terminal symbol), two terminal symbols $a,b$, with productions $$\sigma \to a\sigma b, \qquad \sigma\to \sigma b \qquad \mbox{and} \qquad \sigma \to b.$$  Then $G$ is not $LR(k)$ for any $k\ge 0$.
For instance, look at the following two derivations of $U=a^2\sigma b^3$:
$$\sigma \Rightarrow^* a \sigma b^2 \Rightarrow a^2 \sigma b^3\qquad \mbox{and}\qquad \sigma \Rightarrow^* a^2 \sigma b^2 \Rightarrow a^2 \sigma b^3$$
Here, $U_1=a$, $U_2=\sigma b$.  Let $k=1$.  Then the criteria in the definition are satisfied.  Yet, $W_1=a^2\ne U_1$.  Therefore, $G$ is not $LR(1)$.
\item Note that the grammar $G$ above generates the language $L=\lbrace a^m b^n \mid n > m\rbrace$, which can also be generated by the grammar with three non-terminal symbols $\sigma, X,Y$, with $\sigma$ the final non-terminal symbol, where the productions are given by $$\sigma \to XY,\quad X\to aXb, \quad X\to \lambda, \quad Y\to Yb, \quad \mbox{and} \quad Y\to b.$$  However, this grammar is $LR(1)$.
\end{itemize}

Determining whether a context-free grammar is $LR(k)$ is a non-trivial problem.  Nevertheless, an algorithm exists for determining, given a context-free grammar $G$ and a non-negative integer $k$, whether $G$ is $LR(k)$.  On the other hand, without specifying $k$ in advance, no algorithms exist that determine if $G$ is $LR(k)$ for \emph{some} $k$.

\textbf{Definition}.  A language is said to be $LR(k)$ if it can be generated by an $LR(k)$ grammar.

\begin{thm} Every $LR(k)$ language is deterministic context-free.  Every deterministic context-free language is $LR(1)$.  \end{thm}

Hence, deterministic context-free languages are the same as $LR(1)$ languages.

\begin{thebibliography}{9}
\bibitem{AS} A. Salomaa, {\em Formal Languages}, Academic Press, New York (1973).
\bibitem{hu} J.E. Hopcroft, J.D. Ullman, {\em Formal Languages and Their Relation to Automata}, Addison-Wesley, (1969).
\end{thebibliography}
%%%%%
%%%%%
\end{document}
