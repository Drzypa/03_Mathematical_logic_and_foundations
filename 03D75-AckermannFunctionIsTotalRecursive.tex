\documentclass[12pt]{article}
\usepackage{pmmeta}
\pmcanonicalname{AckermannFunctionIsTotalRecursive}
\pmcreated{2013-03-22 19:07:53}
\pmmodified{2013-03-22 19:07:53}
\pmowner{CWoo}{3771}
\pmmodifier{CWoo}{3771}
\pmtitle{Ackermann function is total recursive}
\pmrecord{15}{42027}
\pmprivacy{1}
\pmauthor{CWoo}{3771}
\pmtype{Theorem}
\pmcomment{trigger rebuild}
\pmclassification{msc}{03D75}
\pmrelated{AckermannFunctionIsNotPrimitiveRecursive}

\usepackage{amssymb,amscd}
\usepackage{amsmath}
\usepackage{amsfonts}
\usepackage{mathrsfs}

% used for TeXing text within eps files
%\usepackage{psfrag}
% need this for including graphics (\includegraphics)
%\usepackage{graphicx}
% for neatly defining theorems and propositions
\usepackage{amsthm}
% making logically defined graphics
%%\usepackage{xypic}
\usepackage{pst-plot}

% define commands here
\newcommand*{\abs}[1]{\left\lvert #1\right\rvert}
\newtheorem{prop}{Proposition}
\newtheorem{thm}{Theorem}
\newtheorem{ex}{Example}
\newcommand{\real}{\mathbb{R}}
\newcommand{\pdiff}[2]{\frac{\partial #1}{\partial #2}}
\newcommand{\mpdiff}[3]{\frac{\partial^#1 #2}{\partial #3^#1}}
\begin{document}
In this entry, we give a formal proof that the Ackermann function $A(x,y)$, given by $$A(0,y)=y+1,\qquad A(x+1,0)=A(x,1),\qquad A(x+1,y+1)=A(x,A(x+1,y))$$
is both a total function and a recursive function.  Actually, the fact that $A$ is total is proved in \PMlinkname{this entry}{PropertiesOfAckermannFunction}.  It remains to show that $A$ is recursive.

Recall that the computation of $A(x,y)$, given $x,y$, can be thought of as an iterated operation performed on finite sequences of integers, starting with $x,y$ and ending with $z=A(x,y)$ (see \PMlinkname{this entry}{ComputingTheAckermannFunction}).  It is this process we will utilize to prove that $A$ is recursive.

In the proof below, the following notations and definitions are used to simplify matters:
\begin{itemize}
\item if $s$ is the sequence $r_1,\ldots, r_m$, then $E(s)$ or $\langle r_1,\ldots, r_m\rangle$ denote the code number of $s$ given the encoding $E$;
\item $\operatorname{lh}(n)$ is the length of the sequence whose code number is $n$;
\item $(n)_i$ is the $i$-th number in the sequence whose code number is $n$;
\item $(n)_{-i}$ is the $i$-th to the last number in the sequence whose code number is $n$ (so that $(n)_{-1}$ is the last number in the sequence whose code number is $n$);
\item $\operatorname{red}(n)$ is the code number of the sequence obtained by deleting the last number of the sequence whose code number is $n$;
\item $\operatorname{ext}(n,a)$ is the code number of the sequence obtained by appending $a$ to the end of the sequence whose code number is $n$.
\end{itemize}
If $E$ is a primitive recursive encoding, then each of the above function is primitive recursive.  For example, $(n)_{-i}=(n)_{\operatorname{lh}(n)\dot{-}i+1}$.

\begin{thm} $A$ is recursive. \end{thm}
\begin{proof}  In this proof, the choice of encoding $E$ is the multiplicative encoding, for it is convenient and, more importantly, a primitive recursive encoding.  Briefly, $$E(r_1,\ldots, r_m)=p_1^{r_1+1}\cdots p_m^{r_m+1},$$ where $p_i$ is the $i$-th prime number (so that $p_1=2$).

We know that computing $A(x,y)=z$ is basically a sequence of computations on finite sequences:
$$x,y \longrightarrow \cdots \longrightarrow z \longrightarrow z \longrightarrow \cdots $$
Let $s(x,y,i)$ denote the sequence at step $i$, then the above sequence can be rewritten:
$$s(x,y,0) \longrightarrow s(x,y,1) \longrightarrow \cdots \longrightarrow s(x,y,k) \longrightarrow \cdots $$
Define $f(x,y,i)=E(s(x,y,i))$.  From this we see that 
$$g(x,y)=\mu i [f(x,y,i)=f(x,y,i+1)].$$
is the function that computes the smallest number of steps needed so that the code number becomes stationary.  When the code number is decoded, we get the resulting value of $A(x,y)$: $$A(x,y)=D(f(x,y,g(x,y))),$$
where $D(m):=(m)_{-1}$, decodes $m$, and returns the last number in the sequence $s$ whose code number $E(s)$ is $m$.

Now the remaining task to show that $f$ is primitive recursive.  First, note that $$f(x,y,0) = \langle x,y\rangle = 2^{x+1}3^{y+1}$$ is primitive recursive.  Next, we want to express $$f(x,y,n+1)=h(f(x,y,n)),$$ where $h$ is the function that changes the code number of the sequence $s(x,y,n)$ to the code number of the sequence $s(x,y,n+1)$.  Once we obtain $h$ and show that $h$ is primitive recursive, then $f$ is primitive recursive, as it is defined by primitive recursion via primitive recursive functions $\langle x,y\rangle$ and $h$.

To find out what $h$ is, recall the four rules of constructing the next sequence from the current one givne in this \PMlinkname{this entry}{ComputingTheAckermannFunction}.  Let $n_1=E(s(x,y,k))$ and $n_2=E(s(x,y,k+1))$.  We rewrite the four rules using the notations and definitions here:
\begin{enumerate}
\item if $\operatorname{lh}(n_1)=1$, then $n_2=n_1$;
\item if $\operatorname{lh}(n_1)>1$, and $(n_1)_{-2}=0$, then $n_2=h_1(n_1)$, where $$h_1(n):= \operatorname{ext}(\operatorname{red}^2(n),(n)_{-1}+1);$$
\item if $\operatorname{lh}(n_1)>1$, and $(n_1)_{-2}>0$ and $(n_1)_{-1}=0$, then $n_2=h_2(n_1)$, where $$h_2(n):=\operatorname{ext}(\operatorname{ext}(\operatorname{red}^2(n),(n)_{-2}-1),1);$$ or
\item if $\operatorname{lh}(n_1)>1$, and $(n_1)_{-2}>0$ and $(n_1)_{-1}>0$, then then $n_2=h_3(n_1)$, where $$h_3(n):=\operatorname{ext}(\operatorname{ext}(\operatorname{ext}(\operatorname{red}^2(n),(n)_{-2}-1),(n)_{-2}),(n)_{-1}-1).$$
\end{enumerate}
If we define predicates:
\begin{enumerate}
\item $\Phi_0(n):=\operatorname{lh}(n)\le 1$,
\item $\Phi_1(n):=\operatorname{lh}(n)>1\textrm{, and }(n)_{-2}=0$,
\item $\Phi_2(n):=\operatorname{lh}(n)>1\textrm{, and }(n)_{-2}>0\textrm{ and }(n)_{-1}=0$,
\item $\Phi_3(n):=\operatorname{lh}(n)>1\textrm{, and }(n)_{-2}>0\textrm{ and }(n)_{-1}>0$.
\end{enumerate}
Then each $\Phi_i$ is primitive recursive, pairwise exclusive, and $\Phi_0\equiv \neg \Phi_1 \wedge \neg \Phi_2 \wedge \neg \Phi_3$.  Now, define $h$ as follows:
\begin{displaymath}
h(n):= \left\{
\begin{array}{ll}
\operatorname{id}(n) & \textrm{if }\Phi_0(n),\\
h_1(n) & \textrm{if }\Phi_1(n),\\
h_2(n) & \textrm{if }\Phi_2(n),\\
h_3(n) & \textrm{if }\Phi_3(n).
\end{array}
\right.
\end{displaymath}
Since $h$ is defined by cases, and each $h_i$ is primitive recursive, $h$ is also primitive recursive.
\end{proof}
%%%%%
%%%%%
\end{document}
