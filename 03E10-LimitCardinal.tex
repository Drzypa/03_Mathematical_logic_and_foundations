\documentclass[12pt]{article}
\usepackage{pmmeta}
\pmcanonicalname{LimitCardinal}
\pmcreated{2013-03-22 14:04:40}
\pmmodified{2013-03-22 14:04:40}
\pmowner{yark}{2760}
\pmmodifier{yark}{2760}
\pmtitle{limit cardinal}
\pmrecord{15}{35438}
\pmprivacy{1}
\pmauthor{yark}{2760}
\pmtype{Definition}
\pmcomment{trigger rebuild}
\pmclassification{msc}{03E10}
\pmrelated{SuccessorCardinal}
\pmdefines{strong limit cardinal}

\endmetadata

\usepackage{amssymb}
\usepackage{amsmath}
\usepackage{amsfonts}
\begin{document}
\PMlinkescapeword{case}
\PMlinkescapeword{cases}
\PMlinkescapeword{even}

A {\em limit cardinal} is a cardinal $\kappa$ such that $\lambda^+<\kappa$ for every cardinal $\lambda<\kappa$. Here $\lambda^+$ denotes the cardinal successor of $\lambda$. If $2^\lambda<\kappa$ for every cardinal $\lambda<\kappa$, then $\kappa$ is called a {\em strong limit cardinal}. 

Every strong limit cardinal is a limit cardinal, because $\lambda^+\leq2^\lambda$ holds for every cardinal $\lambda$.
Under GCH, every limit cardinal is a strong limit cardinal because in this case $\lambda^+=2^\lambda$ for every infinite cardinal $\lambda$.

The three smallest limit cardinals are $0$, $\aleph_0$ and $\aleph_\omega$.
Note that some authors do not count $0$, or sometimes even $\aleph_0$, as a limit cardinal.
An infinite cardinal $\aleph_\alpha$ is a limit cardinal
if and only if $\alpha$ is either $0$ or a limit ordinal.
%%%%%
%%%%%
\end{document}
