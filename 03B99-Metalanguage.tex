\documentclass[12pt]{article}
\usepackage{pmmeta}
\pmcanonicalname{Metalanguage}
\pmcreated{2013-03-22 18:06:05}
\pmmodified{2013-03-22 18:06:05}
\pmowner{yesitis}{13730}
\pmmodifier{yesitis}{13730}
\pmtitle{metalanguage}
\pmrecord{6}{40644}
\pmprivacy{1}
\pmauthor{yesitis}{13730}
\pmtype{Definition}
\pmcomment{trigger rebuild}
\pmclassification{msc}{03B99}

\endmetadata

% this is the default PlanetMath preamble.  as your knowledge
% of TeX increases, you will probably want to edit this, but
% it should be fine as is for beginners.

% almost certainly you want these
\usepackage{amssymb}
\usepackage{amsmath}
\usepackage{amsfonts}

% used for TeXing text within eps files
%\usepackage{psfrag}
% need this for including graphics (\includegraphics)
%\usepackage{graphicx}
% for neatly defining theorems and propositions
%\usepackage{amsthm}
% making logically defined graphics
%%%\usepackage{xypic}

% there are many more packages, add them here as you need them

% define commands here

\begin{document}
A remedy for Berry's Paradox and related paradoxes is to
separate the language used to formulate a particular mathematical
theory from the language used for its discourse.

The language used to formulate a mathematical theory is called the
\emph{object language} to contrast it from the \emph{metalanguage}
used for the discourse.

The most widely used object language is the first-order logic. The
metalanguage could be English or other natural languages plus
mathematical symbols such as $\Rightarrow$. \\

\textsc{Examples}
\begin{enumerate}
    \item The object language speaks of $(\neg A_n)$, but we speak
    of $\langle (, \neg, A_n, ) \rangle$ in the metalanguage.
    [Recall that a formula is some finite sequence of the symbols.
    Cf. First Order Logic or Propositional Logic.]

    \item In induction proofs, one might encounter ``the first
    symbol in the formula $\varphi$ is $($;'' we know that the first
    symbol is indeed $($ and not $\langle$ because $\langle$ is a
    symbol in our metalanguage. Similarly, ``the third symbol is
    $A_n$'' and not $,$ because $,$ is a symbol in our metalanguage.

    \item $\vdash$ and $\models$ are members of the metalanguage,
    \emph{not} of object language.

    \item Parallel with the notion of metalanguage is metatheorem.
    ``$\Gamma\vdash(\varphi\rightarrow\psi)$ if
    $\Gamma\cup\{\varphi\}\vdash\psi, \Gamma\subseteq\mathcal{L}_0,
    \varphi, \psi\in\mathcal{L}_0$" is a metatheorem.

    \item \emph{Examples from Set Theory}. Let ``Con" denote
    consistency. Then Con(ZF) and Con(ZF+AC+GCH) are metamathematical
    statements; they are statements in the metalanguage.
\end{enumerate}

\begin{thebibliography}{1}
\bibitem{Sc1997}
Schechter, E., \emph{Handbook of Analysis and Its Foundations}, 1st ed., Academic Press, 1997.
\end{thebibliography}

%%%%%
%%%%%
\end{document}
