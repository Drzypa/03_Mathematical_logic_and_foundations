\documentclass[12pt]{article}
\usepackage{pmmeta}
\pmcanonicalname{NaturalNumber}
\pmcreated{2013-03-22 11:50:05}
\pmmodified{2013-03-22 11:50:05}
\pmowner{djao}{24}
\pmmodifier{djao}{24}
\pmtitle{natural number}
\pmrecord{16}{30387}
\pmprivacy{1}
\pmauthor{djao}{24}
\pmtype{Definition}
\pmcomment{trigger rebuild}
\pmclassification{msc}{03E10}
\pmclassification{msc}{74D99}
\pmsynonym{$\mathbb{N}$}{NaturalNumber}
\pmrelated{InductiveSet}
\pmrelated{Successor}
\pmrelated{PeanoArithmetic}
\pmrelated{VonNeumannInteger}

\endmetadata

\usepackage{amssymb}
\usepackage{amsmath}
\usepackage{amsfonts}
\usepackage{graphicx}
%%%%\usepackage{xypic}
\begin{document}
Given the Zermelo-Fraenkel axioms of set theory, one can prove that there exists an inductive set $X$ such that $\emptyset \in X$. The {\em natural numbers} $\mathbb{N}$ are then defined to be the intersection of all subsets of $X$ which are inductive sets and contain the empty set as an element.

The first few natural numbers are:
\begin{itemize}
\item $0 := \emptyset$
\item $1 := 0' = \{0\} = \{ \emptyset \}$
\item $2 := 1' = \{0,1\} = \{\emptyset, \{ \emptyset \} \}$
\item $3 := 2' = \{0,1,2\} = \{\emptyset, \{ \emptyset \}, \{ \emptyset, \{ \emptyset \} \} \}$
\end{itemize}

Note that the set $0$ has zero elements, the set $1$ has one element, the set $2$ has two elements, etc. Informally, the set $n$ is the set consisting of the $n$ elements $0, 1, \dots, n-1$, and $n$ is both a subset of $\mathbb{N}$ and an element of $\mathbb{N}$.

In some contexts (most notably, in number theory), it is more convenient to exclude $0$ from the set of natural numbers, so that $\mathbb{N} = \{1,2,3,\dots\}$. When it is not explicitly specified, one must determine from context whether $0$ is being considered a natural number or not.

Addition of natural numbers is defined inductively as follows:
\begin{itemize}
\item $a + 0 := a$ for all $a \in \mathbb{N}$
\item $a + b' := (a+b)'$ for all $a,b \in \mathbb{N}$
\end{itemize}

Multiplication of natural numbers is defined inductively as follows:
\begin{itemize}
\item $a \cdot 0 := 0$ for all $a \in \mathbb{N}$
\item $a \cdot b' := (a\cdot b) + a$ for all $a,b \in \mathbb{N}$
\end{itemize}
The natural numbers form a monoid under either addition or multiplication. There is an ordering relation on the natural numbers, defined by: $a \leq b$ if $a \subseteq b$.
%%%%%
%%%%%
%%%%%
%%%%%
\end{document}
