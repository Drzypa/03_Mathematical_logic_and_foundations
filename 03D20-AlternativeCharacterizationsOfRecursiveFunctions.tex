\documentclass[12pt]{article}
\usepackage{pmmeta}
\pmcanonicalname{AlternativeCharacterizationsOfRecursiveFunctions}
\pmcreated{2013-03-22 14:34:42}
\pmmodified{2013-03-22 14:34:42}
\pmowner{rspuzio}{6075}
\pmmodifier{rspuzio}{6075}
\pmtitle{alternative characterizations of recursive functions}
\pmrecord{12}{36139}
\pmprivacy{1}
\pmauthor{rspuzio}{6075}
\pmtype{Topic}
\pmcomment{trigger rebuild}
\pmclassification{msc}{03D20}
\pmdefines{minimized inversion}

% this is the default PlanetMath preamble.  as your knowledge
% of TeX increases, you will probably want to edit this, but
% it should be fine as is for beginners.

% almost certainly you want these
\usepackage{amssymb}
\usepackage{amsmath}
\usepackage{amsfonts}

% used for TeXing text within eps files
%\usepackage{psfrag}
% need this for including graphics (\includegraphics)
%\usepackage{graphicx}
% for neatly defining theorems and propositions
%\usepackage{amsthm}
% making logically defined graphics
%%%\usepackage{xypic}

% there are many more packages, add them here as you need them

% define commands here
\begin{document}
The class of recursive functions may be characterized by considerably weaker conditions than those given in the entry ``\PMlinkname{recursive function}{RecursiveFunction}'' of this encyclopaedia.  This entry will discuss several such characterizations.

Criteria 2 and 3 in the list may be replaced by the considerably weaker criterion:

2') The successor function $S \colon \mathbb{Z}_+ \to \mathbb{Z}_+$ defined as $S(x) = x+1$ is a recursive function.

By means of a pairing function, the definition may be simplified considerably.   Using such a function and its inverses, the set of recursive functions of $m$ variables may be put in one-to-one correspondence with recursive functions of $n$ variables for any pair of non-zero positive integers $m$ and $n$.  Hence one can focus attention on recursive functions of a small fixed number of variables.  One characterization of recursive functions of not more than two variables is the following:

The class of recursive functions is the smallest class of positive integer valued functions of not more than two positive integers which satisfies the following criteria:

\begin{enumerate}
\item[1'] The constant function $c: \mathbb{Z}_+ \to \mathbb{Z}_+$ defined by $c(x) = 1$ for all $x \in \mathbb{Z}_+$ is a recursive function.
\item[2']  The successor function $S \colon \mathbb{Z}_+ \to \mathbb{Z}_+$ defined as $S(x) = x+1$ is a recursive function.
\item[3'] The projection functions $I^1_1 \colon \mathbb{Z}_+ \to \mathbb{Z}_+$, $I^2_1 \colon \mathbb{Z}_+^2 \to \mathbb{Z}_+$, and $I^2_2 \colon \mathbb{Z}_+^2 \to \mathbb{Z}_+$ defined as
 $$I^1_1 (x) = x$$ 
 $$I^2_1 (x,y) = x$$ 
 $$I^2_2 (x,y) = y$$ 
are recursive functions.
\item[4'] If $a \colon \mathbb{Z}_+^2 \to \mathbb{Z}_+$, $b \colon \mathbb{Z}_+^2 \to \mathbb{Z}_+$, $c \colon \mathbb{Z}_+^2 \to \mathbb{Z}_+$ ,$d \colon \mathbb{Z}_+ \to \mathbb{Z}_+$, and $e \colon \mathbb{Z}_+ \to \mathbb{Z}_+$ are recursive functions, then $f \colon \mathbb{Z}_+^2 \to \mathbb{Z}_+$, $g \colon \mathbb{Z}_+ \to \mathbb{Z}_+$, and $h \colon \mathbb{Z}_+^2 \to \mathbb{Z}_+$, defined by
 $$f(x,y) = d(a(x,y))$$
 $$g(x) = a(d(x),e(y))$$
 $$h(x,y) = a(b(x,y),c(x,y))$$
are recursive functions.
\item[5'] If $f \colon \mathbb{Z}_+ \to \mathbb{Z}_+$, $g \colon \mathbb{Z}_+ \to \mathbb{Z}_+$ are recursive functions, then the function $h \colon \mathbb{Z}_+^2 \to \mathbb{Z}_+$ defined by the recursion
 $$h(n+1,x) = g(h(n,x))$$
with the initial condition
 $$h(0,x) = f(x)$$
is a recursive function.
\item[6'] If $f \colon \mathbb{Z}_+^2 \to \mathbb{Z}_+$ is a recursive function then $g \colon \mathbb{Z}_+ \to \mathbb{Z}_+$ is a recursive function, where $g(x)$ is defined to equal $y$ if there exists a $y \in \mathbb{Z}_+$ such that 
\begin{enumerate}
\item $f(0, x), f(1, x), \ldots f(y, x)$ are all defined, 
\item $f(z, x) \ne 0$ when $1 \le z <y$, and 
\item $f(y, x) = 0$.
\end{enumerate}
Otherwise, $g(x)$ is undefined.
\end{enumerate}

The criterion 5' may be shown to follow from the remaining criteria, and hence it may be dropped.

By further exploiting the marvelous properties of the pairing function, criterion 6' may be replaced by the following:

6'') If $f \colon \mathbb{Z}_+ \to \mathbb{Z}_+$ is a recursive function then $g \colon \mathbb{Z}_+ \to \mathbb{Z}_+$ is a recursive function, where $g(x)$ is defined to equal $y$  if there exists a $y \in \mathbb{Z}_+$ such that 
\begin{enumerate}
\item[a] $f(0), f(1), \ldots f(y)$ are all defined, 
\item[b] $f(z) \ne x$ when $1 \le z <y$, and 
\item[c] $f(y) = x$.
\end{enumerate}
Otherwise, $g(x)$ is undefined.

The operation introduced in this new criterion is called \emph{minimized inversion} and will be denoted as $g = f^{-1}$.  Note that there is no conflict with the usual notion of inverse of a function because, if $f$ is invertible, the minimized inverse of $f$ is the same as the inverse of $f$ in the usual sense; otherwise the notion of minimized inverse extends the definition of inverse to a larger class of functions.

Finally, by taking these ideas even further, Czirmaz showed that recursive functions of a single variable had the following simple characterization:

The class of recursive functions is the smallest class of positive integer valued functions of a positive integer which satisfies the following criteria:
\begin{enumerate}
\item[1''] The constant function $c: \mathbb{Z}_+ \to \mathbb{Z}_+$ defined by $c(x) = 1$ for all $x \in \mathbb{Z}_+$ is a recursive function.
\item[2'']  The successor function $S \colon \mathbb{Z}_+ \to \mathbb{Z}_+$ defined as $S(x) = x+1$ is a recursive function.
\item[3'']  The function $Q \colon \mathbb{Z}_+ \to \mathbb{Z}_+$ defined as $Q(x) = x - \lceil \sqrt{x} \rceil^2$ is a recursive function.  In words, $Q(x)$ is the difference between $x$ and the largest square number smaller than $x$.
\item[4'']  If $f \colon \mathbb{Z}_+ \to \mathbb{Z}_+$ and $g \colon \mathbb{Z}_+ \to \mathbb{Z}_+$ are recursive functions, then $h \colon \mathbb{Z}_+ \to \mathbb{Z}_+$ is a recursive function, where
 $$h(x) = f(g(x))$$
\item[6''] If $f \colon \mathbb{Z}_+ \to \mathbb{Z}_+$ is a recursive function then $g \colon \mathbb{Z}_+ \to \mathbb{Z}_+$ is a recursive function, where $g(x)$ is defined to equal $y$  if there exists a $y \in \mathbb{Z}_+$ such that \begin{enumerate}
\item[a] $f(0), f(1), \ldots f(y)$ are all defined, 
\item[b] $f(z) \ne x$ when $1 \le z <y$, and 
\item[c] $f(y) = x$.
\end{enumerate}
Otherwise, $g(x)$ is undefined.
\end{enumerate}
%%%%%
%%%%%
\end{document}
