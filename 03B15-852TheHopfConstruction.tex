\documentclass[12pt]{article}
\usepackage{pmmeta}
\pmcanonicalname{852TheHopfConstruction}
\pmcreated{2013-11-06 14:24:49}
\pmmodified{2013-11-06 14:24:49}
\pmowner{PMBookProject}{1000683}
\pmmodifier{rspuzio}{6075}
\pmtitle{8.5.2 The Hopf construction}
\pmrecord{1}{}
\pmprivacy{1}
\pmauthor{PMBookProject}{6075}
\pmtype{Feature}
\pmclassification{msc}{03B15}

\usepackage{xspace}
\usepackage{amssyb}
\usepackage{amsmath}
\usepackage{amsfonts}
\usepackage{amsthm}
\newcommand{\blank}{\mathord{\hspace{1pt}\text{--}\hspace{1pt}}}
\newcommand{\defeq}{\vcentcolon\equiv}  
\newcommand{\define}[1]{\textbf{#1}}
\newcommand{\eqv}[2]{\ensuremath{#1 \simeq #2}\xspace}
\newcommand{\idfunc}[1][]{\ensuremath{\mathsf{id}_{#1}}\xspace}
\newcommand{\indexdef}[1]{\index{#1|defstyle}}   
\newcommand{\opp}[1]{\mathord{{#1}^{-1}}}
\newcommand{\proj}[1]{\ensuremath{\mathsf{pr}_{#1}}\xspace}
\newcommand{\prop}{\ensuremath{\mathsf{Prop}}\xspace}
\newcommand{\susp}{\Sigma}
\newcommand{\tproj}[3][]{\mathopen{}\left|#3\right|_{#2}^{#1}\mathclose{}}
\newcommand{\trunc}[2]{\mathopen{}\left\Vert #2\right\Vert_{#1}\mathclose{}}
\newcommand{\unit}{\ensuremath{\mathbf{1}}\xspace}
\newcommand{\vcentcolon}{:\!\!}
\newcounter{mathcount}
\setcounter{mathcount}{1}
\newtheorem{predefn}{Definition}
\newenvironment{defn}{\begin{predefn}}{\end{predefn}\addtocounter{mathcount}{1}}
\renewcommand{\thepredefn}{8.5.\arabic{mathcount}}
\newtheorem{prelem}{Lemma}
\newenvironment{lem}{\begin{prelem}}{\end{prelem}\addtocounter{mathcount}{1}}
\renewcommand{\theprelem}{8.5.\arabic{mathcount}}
\let\autoref\cref

\begin{document}
\begin{defn}
  An \define{H-space}
  \indexdef{H-space}%
  consists of
  \begin{itemize}
  \item a type $A$,
  \item a base point $e:A$,
  \item a binary operation $\mu:A\times A\to A$, and
  \item for every $a:A$, equalities $\mu(e,a)=a$ and $\mu(a,e)=a$.
  \end{itemize}
\end{defn}

\begin{lem}
  Let $A$ be a connected H-space. Then for every $a:A$, the maps $\mu(a,\blank):A\to
  A$ and $\mu(\blank,a):A\to A$ are equivalences.
\end{lem}

\begin{proof}
  Let us prove that for every $a:A$ the map $\mu(a,\blank)$ is an equivalence. The
  other statement is symmetric.
  %
  The statement that $\mu(a,\blank)$ is an equivalence corresponds to a type family
  $P:A\to\prop$ and proving it corresponds to finding a section of this type
  family.

  The type $\prop$ is a set (\autoref{thm:hleveln-of-hlevelSn}) hence we can
  define a new type family $P':\trunc0A\to\prop$ by $P'(\tproj0a)\defeq
  P(a)$. But $A$ is connected by assumption, hence $\trunc0A$ is
  contractible. This implies that in order to find a section of $P'$, it is
  enough to find a point in the fiber of $P'$ over $\tproj0e$. But we have
  $P'(\tproj0e)=P(e)$ which is inhabited because $\mu(e,\blank)$ is equal to the
  identity map by definition of an H-space, hence is an equivalence.

  We have proved that for every $x:\trunc0A$ the proposition $P'(x)$ is true,
  hence in particular for every $a:A$ the proposition $P(a)$ is true because
  $P(a)$ is $P'(\tproj0a)$.
\end{proof}

\begin{defn}
  Let $A$ be a connected H-space. We define a fibration over $\susp A$ using
  \autoref{lem:fibration-over-pushout}.

  Given that $\susp A$ is the pushout $\unit\sqcup^A\unit$, we can define a
  fibration over $\susp A$ by specifying
  \begin{itemize}
  \item two fibrations over $\unit$ (i.e. two types $F_1$ and $F_2$), and
  \item a family $e:A\to(\eqv{F_1}{F_2})$ of equivalences between
    $F_1$ and $F_2$, one for every element of $A$.
  \end{itemize}
  %
  We take $A$ for $F_1$ and $F_2$, and for $a:A$ we take the equivalence
  $\mu(a,\blank)$ for $e(a)$.
\end{defn}

According to \autoref{lem:fibration-over-pushout}, we have the following
diagram:
%
\[\xymatrix{A \ar@{->>}[d] & A \times A \ar[l]_-{\proj2} \ar@{->>}_{\proj1}[d]
  \ar[r]^-{\mu} & A \ar@{->>}[d] \\
  1 & A \ar[r] \ar[l] & 1}\]
%
and the fibration we just constructed is a fibration over $\susp A$ whose total
space is the pushout of the top line.

Moreover, with $f(x,y)\defeq(\mu(x,y),y)$ we have the following diagram:
%
\[\xymatrix{A \ar_\idfunc[d] & A \times A \ar[l]_-{\proj2} \ar^f[d]
  \ar[r]^-{\mu} & A \ar^\idfunc[d] \\
  A & A\times A \ar^-{\proj2}[l] \ar_-{\proj1}[r] & A}\]
%
The diagram commutes and the three vertical maps are equivalences, the inverse
of $f$ being the function $g$ defined by
\[g(u,v)\defeq(\opp{\mu(\blank,v)}(u),v).\]
%
This shows that the two lines are equivalent (hence equal) spans, so the total
space of the fibration we constructed is equivalent to the pushout of the bottom
line.
And by definition, this latter pushout is the \emph{join} of $A$ with itself (see \autoref{sec:colimits}).
%
We have proven:

\begin{lem}\label{lem:hopf-construction}
  Given a connected H-space $A$, there is a fibration, called the
  \define{Hopf construction},
  \indexdef{Hopf!construction}%
  over $\susp A$ with fiber $A$ and total space $A*A$.
\end{lem}


\end{document}
