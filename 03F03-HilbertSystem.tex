\documentclass[12pt]{article}
\usepackage{pmmeta}
\pmcanonicalname{HilbertSystem}
\pmcreated{2013-03-22 19:13:14}
\pmmodified{2013-03-22 19:13:14}
\pmowner{CWoo}{3771}
\pmmodifier{CWoo}{3771}
\pmtitle{Hilbert system}
\pmrecord{15}{42141}
\pmprivacy{1}
\pmauthor{CWoo}{3771}
\pmtype{Definition}
\pmcomment{trigger rebuild}
\pmclassification{msc}{03F03}
\pmclassification{msc}{03B99}
\pmclassification{msc}{03B22}
\pmsynonym{axiom system}{HilbertSystem}
\pmrelated{GentzenSystem}
\pmdefines{generalization}
\pmdefines{necessitation}
\pmdefines{double negation}

\usepackage{amssymb,amscd}
\usepackage{amsmath}
\usepackage{amsfonts}
\usepackage{mathrsfs}

% used for TeXing text within eps files
%\usepackage{psfrag}
% need this for including graphics (\includegraphics)
%\usepackage{graphicx}
% for neatly defining theorems and propositions
\usepackage{amsthm}
% making logically defined graphics
%%\usepackage{xypic}
\usepackage{pst-plot}

% define commands here
\newcommand*{\abs}[1]{\left\lvert #1\right\rvert}
\newtheorem{prop}{Proposition}
\newtheorem{thm}{Theorem}
\newtheorem{ex}{Example}
\newcommand{\real}{\mathbb{R}}
\newcommand{\pdiff}[2]{\frac{\partial #1}{\partial #2}}
\newcommand{\mpdiff}[3]{\frac{\partial^#1 #2}{\partial #3^#1}}
\begin{document}
A \emph{Hilbert system} is a style (formulation) of deductive system that emphasizes the role played by the axioms in the system.  Typically, a Hilbert system has many axiom schemes, but only a few, sometimes one, rules of inference.  As such, a Hilbert system is also called an \emph{axiom system}.  Below we list three examples of axiom systems in mathematical logic:

\begin{itemize}
\item (intuitionistic propositional logic)
\begin{itemize}
\item axiom schemes:
\begin{enumerate}
\item $A\to (B\to A)$
\item $(A\to (B\to C))\to ((A\to B)\to (A\to C))$
\item $A\to A\vee B$ 
\item $B\to A\vee B$
\item $(A\to C)\to ((B\to C)\to (A\vee B\to C))$
\item $A\wedge B \to A$
\item $A\wedge B \to B$
\item $A \to (B\to (A\wedge B))$
\item $\perp \to A$
\end{enumerate}
\item rule of inference: (modus ponens): from $A\to B$ and $A$, we may infer $B$
\end{itemize}
\item (classical predicate logic without equality)
\begin{itemize}
\item axiom schemes:
\begin{enumerate}
\item all of the axiom schemes above, and 
\item law of double negation: $\neg (\neg A) \to A$
\item $\forall x A \to A[x/y]$
\item $\forall x (A\to B) \to (A \to \forall y B[x/y])$
\end{enumerate}
In the last two axiom schemes, we require that $y$ is free for $x$ in $A$, and in the last axiom scheme, we also require that $x$ does not occur free in $A$.
\item rules of inference: 
\begin{enumerate}
\item modus ponens, and
\item generalization: from $A$, we may infer $\forall y A[x/y]$, where $y$ is free for $x$ in $A$
\end{enumerate}
\end{itemize}
\item (S4 modal propositional logic)
\begin{itemize}
\item axiom schemes:
\begin{enumerate}
\item all of the axiom schemes in intuitionistic propositional logic, as well as the law of double negation, and
\item Axiom K, or the normality axiom: $\square (A\to B) \to (\square A \to \square B)$
\item Axiom T: $\square A \to A$
\item Axiom 4: $\square A \to \square (\square A)$
\end{enumerate}
\item rules of inference: 
\begin{enumerate}
\item modus ponens, and
\item necessitation: from $A$, we may infer $\square A$
\end{enumerate}
\end{itemize}
\end{itemize}
where $A,B,C$ above are well-formed formulas, $x,y$ are individual variables, and $\to,\vee,\wedge$ are binary, $\square$ unary, and $\perp$ nullary logical connectives in the respective logical systems.  The connective $\neg$ may be defined as $\neg A:=A\to \perp$ for any formula $A$.

\textbf{Remarks}
\begin{itemize}
\item
Hilbert systems need not be unique for a given logical system.  For example, see \PMlinkname{this link}{LogicalAxiom}.
\item
For a given logical system, every Hilbert system is deductively equivalent to a Gentzen system: for any axiom $A$ in a Hilbert system $H$, convert it to the sequent $\Rightarrow A$, and for any rule: from $A_1,\ldots,A_n$ we may deduce $B$, convert it to the rule: from $\Delta\Rightarrow A_1,\ldots,A_n$, we may infer $\Delta \Rightarrow B$.
\item
Since axioms are semantically valid statements, the use of Hilbert systems is more about deriving other semantically valid statements, or theorems, and less about the syntactical analysis of deductions themselves.  Outside of structural proof theory, deductive systems a la Hilbert style are used almost exclusively everywhere in mathematics.
\end{itemize}

\begin{thebibliography}{7}
\bibitem{he} H. Enderton: {\em A Mathematical Introduction to Logic}, Academic Press, San Diego (1972).
\bibitem{TS} A. S. Troelstra, H. Schwichtenberg, {\it Basic Proof Theory}, 2nd Edition, Cambridge University Press (2000)
\bibitem{BS} B. F. Chellas, {\it Modal Logic, An Introduction}, Cambridge University Press (1980)
\end{thebibliography}
%%%%%
%%%%%
\end{document}
