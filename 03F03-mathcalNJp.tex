\documentclass[12pt]{article}
\usepackage{pmmeta}
\pmcanonicalname{mathcalNJp}
\pmcreated{2013-03-22 13:05:17}
\pmmodified{2013-03-22 13:05:17}
\pmowner{Henry}{455}
\pmmodifier{Henry}{455}
\pmtitle{$\mathcal{NJ}p$}
\pmrecord{7}{33504}
\pmprivacy{1}
\pmauthor{Henry}{455}
\pmtype{Definition}
\pmcomment{trigger rebuild}
\pmclassification{msc}{03F03}
\pmsynonym{NJp}{mathcalNJp}

\endmetadata

% this is the default PlanetMath preamble.  as your knowledge
% of TeX increases, you will probably want to edit this, but
% it should be fine as is for beginners.

% almost certainly you want these
\usepackage{amssymb}
\usepackage{amsmath}
\usepackage{amsfonts}

% used for TeXing text within eps files
%\usepackage{psfrag}
% need this for including graphics (\includegraphics)
%\usepackage{graphicx}
% for neatly defining theorems and propositions
%\usepackage{amsthm}
% making logically defined graphics
%%%\usepackage{xypic}

% there are many more packages, add them here as you need them

% define commands here
%\PMlinkescapeword{theory}
\begin{document}
$\mathcal{NJ}p$ is a natural deduction proof system for intuitionistic propositional logic.  Its only axiom is $\alpha\Rightarrow\alpha$ for any atomic $\alpha$.  Its rules are:

$$\begin{array}{cc}
\frac{\begin{array}{c}\Gamma\Rightarrow\alpha\end{array}}{
\begin{array}{cc}
\Gamma\Rightarrow\alpha\vee\beta&
\Gamma\Rightarrow\beta\vee\alpha
\end{array}{cc}}(\vee I)
&
\frac{\begin{array}{ccc}
\Gamma\Rightarrow\alpha&
\Sigma,\alpha^0\Rightarrow\phi&
\Pi,\beta^0\Rightarrow\phi
\end{array}}
{\begin{array}{c}[\Gamma,\Sigma,\Pi]\Rightarrow\phi\end{array}}(\vee E)
\end{array}$$

The syntax $\alpha^0$ indicates that the rule also holds if that formula is omitted.


$$\begin{array}{cc}
\frac{\begin{array}{cc}
\Gamma\Rightarrow\alpha&
\Sigma\Rightarrow\beta
\end{array}}{\begin{array}{c}
[\Gamma,\Sigma]\Rightarrow\alpha\wedge\beta\end{array}}(\wedge I)
&
\frac{\begin{array}{c}\Gamma\Rightarrow\alpha\wedge\beta\end{array}}
{\begin{array}{cc}
\Gamma\Rightarrow\alpha&
\Gamma\Rightarrow\beta
\end{array}}(\wedge E)
\end{array}$$

$$\begin{array}{cc}
\frac{\begin{array}{c}\Gamma,\alpha\Rightarrow\beta\end{array}}{
\begin{array}{c}\Gamma\Rightarrow\alpha\rightarrow\beta\end{array}}(\rightarrow I)
&
\frac{\begin{array}{cc}
\Gamma\Rightarrow\alpha\rightarrow\beta&
\Sigma\Rightarrow\alpha
\end{array}}
{\begin{array}{c}[\Gamma,\Sigma]\Rightarrow\beta\end{array}}(\rightarrow E)
\end{array}$$

$$\frac{\Gamma\Rightarrow\bot}{\Gamma\Rightarrow\alpha}(\bot_i),\quad \text{ where }\alpha\text{ is atomic}$$
%%%%%
%%%%%
\end{document}
