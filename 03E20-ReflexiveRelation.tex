\documentclass[12pt]{article}
\usepackage{pmmeta}
\pmcanonicalname{ReflexiveRelation}
\pmcreated{2013-03-22 12:15:36}
\pmmodified{2013-03-22 12:15:36}
\pmowner{yark}{2760}
\pmmodifier{yark}{2760}
\pmtitle{reflexive relation}
\pmrecord{17}{31644}
\pmprivacy{1}
\pmauthor{yark}{2760}
\pmtype{Definition}
\pmcomment{trigger rebuild}
\pmclassification{msc}{03E20}
\pmrelated{Symmetric}
\pmrelated{Transitive3}
\pmrelated{Antisymmetric}
\pmrelated{Irreflexive}
\pmdefines{reflexivity}
\pmdefines{reflexive}

\endmetadata

\usepackage{amssymb}
\usepackage{amsmath}
\usepackage{amsfonts}
\begin{document}
\PMlinkescapeword{contain}
\PMlinkescapeword{contains}
\PMlinkescapeword{reflexive}
\PMlinkescapeword{relations}

A relation $\mathcal{R}$ on a set $A$
is \emph{reflexive} if and only if $a\mathcal{R}a$ for all $a\in A$.

For example, let $A = \{1,2,3\}$.
Then $\{(1,1), (2,2), (3,3), (1,3), (3,2)\}$ is a reflexive relation on $A$,
because it contains $(a,a)$ for all $a \in A$.
However, $\{(1,1), (2,2), (2,3), (3,1)\}$ is not reflexive
because it does not contain $(3,3)$.

On a finite set with $n$ elements there are $2^{n^2}$ relations,
of which $2^{n^2-n}$ are reflexive.

%%%%%
%%%%%
%%%%%
\end{document}
