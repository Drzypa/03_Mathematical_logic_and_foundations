\documentclass[12pt]{article}
\usepackage{pmmeta}
\pmcanonicalname{LimitOfSequenceOfSets}
\pmcreated{2013-03-22 15:00:34}
\pmmodified{2013-03-22 15:00:34}
\pmowner{CWoo}{3771}
\pmmodifier{CWoo}{3771}
\pmtitle{limit of sequence of sets}
\pmrecord{8}{36715}
\pmprivacy{1}
\pmauthor{CWoo}{3771}
\pmtype{Theorem}
\pmcomment{trigger rebuild}
\pmclassification{msc}{03E20}
\pmclassification{msc}{28A05}
\pmclassification{msc}{60A99}
\pmrelated{LimitSuperior}

\endmetadata

% this is the default PlanetMath preamble.  as your knowledge
% of TeX increases, you will probably want to edit this, but
% it should be fine as is for beginners.

% almost certainly you want these
\usepackage{amssymb,amscd}
\usepackage{amsmath}
\usepackage{amsfonts}

% used for TeXing text within eps files
%\usepackage{psfrag}
% need this for including graphics (\includegraphics)
%\usepackage{graphicx}
% for neatly defining theorems and propositions
%\usepackage{amsthm}
% making logically defined graphics
%%%\usepackage{xypic}

% there are many more packages, add them here as you need them

% define commands here
\begin{document}
Recall that $\limsup$ and $\liminf$ of a sequence of sets $\lbrace A_i \rbrace$ denote the \PMlinkescapetext{limit superior} and the \PMlinkescapetext{limit inferior} of $\lbrace A_i \rbrace$, respectively.  Please click \PMlinkname{here}{LimitSuperiorOfSets} to see the definitions and \PMlinkname{here}{LimitSuperior} to see the specialized definitions when they are applied to the real numbers.

\textbf{Theorem}.  Let $\lbrace A_i \rbrace$ be a sequence of sets with $i\in\mathbb{Z}^{+}=\lbrace 1,2,\ldots \rbrace$.  Then
\begin{enumerate}
\item for $I$ ranging over all infinite subsets of $\mathbb{Z}^{+}$, $$\limsup A_i=\bigcup_{I}\bigcap_{i\in I}A_i,$$
\item for $I$ ranging over all subsets of $\mathbb{Z}^{+}$ with finite compliment, $$\liminf A_i=\bigcup_{I}\bigcap_{i\in I}A_i,$$
\item $\liminf A_i\subseteq\limsup A_i$.
\end{enumerate}

\textbf{Proof}. \begin{enumerate}
\item We need to show, for $I$ ranging over all infinite subsets of
$\mathbb{Z}^{+}$,
\begin{eqnarray}
\bigcup_{I}\bigcap_{i\in I} A_i=\bigcap_{n=1}^\infty
\bigcup_{i=n}^\infty A_k.\end{eqnarray} Let $x$ be an element of the
LHS, the left hand side of Equation (1). Then $x\in\bigcap_{i\in I}
A_i$ for some infinite subset $I\subseteq\mathbb{Z}^{+}$. Certainly,
$x\in\bigcup_{i=1}^{\infty} A_i$.  Now, suppose
$x\in\bigcup_{i=k}^{\infty} A_i$.  Since $I$ is infinite, we can
find an $l\in I$ such that $l>k$.  Being a member of $I$, we have
that $x\in A_l\subseteq\bigcup_{i=k+1}^{\infty} A_i$.  By induction,
we have $x\in\bigcup_{i=n}^{\infty} A_i$ for all
$n\in\mathbb{Z}^{+}$.  Thus $x$ is an element of the RHS. This
proves one side of the inclusion ($\subseteq$) in (1).

To show the other inclusion, let $x$ be an element of the RHS.  So
$x\in\bigcup_{i=n}^{\infty} A_i$ for all $n\in\mathbb{Z}^{+}$  In
$\bigcup_{i=1}^{\infty} A_i$, pick the least element $n_0$ such that
$x\in A_{n_0}$.  Next, in $\bigcup_{i=n_0+1}^{\infty} A_i$, pick the
least $n_1$ such that $x\in A_{n_1}$.  Then the set $I=\lbrace
n_0,n_1,\ldots \rbrace$ fulfills the requirement $x\in\bigcap_{i\in
I} A_i$, showing the other inclusion ($\supseteq$).
\item  Here we have to show, for $I$ ranging over all subsets of
$\mathbb{Z}^{+}$ with $\mathbb{Z}^{+}-I$ finite,
\begin{eqnarray}
\bigcup_{I}\bigcap_{i\in I} A_i=\bigcup_{n=1}^\infty
\bigcap_{i=n}^\infty A_k.\end{eqnarray}  Suppose first that $x$ is
an element of the LHS so that $x\in\bigcap_{i\in I} A_i$ for some
$I$ with $\mathbb{Z}^{+}-I$ finite.  Let $n_0$ be a upper bound of
the finite set $\mathbb{Z}^{+}-I$ such that for any
$n\in\mathbb{Z}^{+}-I$, $n<n_0$.  This means that any $m\geq n_0$,
we have $m\in I$. Therefore, $x\in\bigcap_{i=n_0}^{\infty} A_i$ and
$x$ is an element of the RHS.

Next, suppose $x$ is an element of the RHS so that
$x\in\bigcap_{k=n}^\infty A_k$ for some $n$.  Then the set
$I=\lbrace n_0,n_0+1,\ldots\rbrace$ is a subset of $\mathbb{Z}^{+}$
with finite complement that does the job for the LHS.
\item  The set of all subsets (of $\mathbb{Z}^{+}$) with finite
complement is a subset of the set of all infinite subsets.  The
third assertion is now clear from the previous two propositions. QED
\end{enumerate}

\textbf{Corollary}.  If $\lbrace A_i \rbrace$ is a decreasing
sequence of sets, then $$\liminf A_i=\limsup A_i=\lim A_i=\bigcap
A_i.$$  Similarly, if $\lbrace A_i \rbrace$ is an increasing
sequence of sets, then $$\liminf A_i=\limsup A_i=\lim A_i=\bigcup
A_i.$$

\textbf{Proof}.  We shall only show the case when we have a
descending chain of sets, since the other case is completely
analogous.  Let $A_1\supseteq A_2\supseteq\ldots$ be a descending
chain of sets.  Set $A=\bigcap_{i=1}^\infty A_i$.  We shall show
that $$\limsup A_i=\liminf A_i=\lim A_i=A.$$  First, by the
definition of \PMlinkescapetext{limit superior} of a sequence of sets:
$$\limsup A_i=\bigcap_{n=1}^\infty \bigcup_{i=n}^\infty
A_k=\bigcap_{n=1}^\infty A_n=A.$$  Now, by Assertion 3 of the above
Theorem, $\liminf A_i\subseteq\limsup A_i=A$, so we only need to
show that $A\subseteq\liminf A_i$.  But this is immediate from the
definition of $A$, being the intersection of all $A_i$ with subscripts $i$ taking on all values of $\mathbb{Z}^{+}$.  Its complement is the empty
set, clearly finite.  Having shown both the existence and equality
of the \PMlinkescapetext{limit superior} and \PMlinkescapetext{limit inferior} of the $A_i$'s, we conclude
that the limit of $A_i$'s exist as well and it is equal to $A$.
QED
%%%%%
%%%%%
\end{document}
