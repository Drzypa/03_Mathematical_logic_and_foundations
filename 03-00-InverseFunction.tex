\documentclass[12pt]{article}
\usepackage{pmmeta}
\pmcanonicalname{InverseFunction}
\pmcreated{2013-03-22 13:53:52}
\pmmodified{2013-03-22 13:53:52}
\pmowner{matte}{1858}
\pmmodifier{matte}{1858}
\pmtitle{inverse function}
\pmrecord{14}{34645}
\pmprivacy{1}
\pmauthor{matte}{1858}
\pmtype{Definition}
\pmcomment{trigger rebuild}
\pmclassification{msc}{03-00}
\pmclassification{msc}{03E20}
\pmsynonym{non-singular function}{InverseFunction}
\pmsynonym{nonsingular function}{InverseFunction}
\pmsynonym{non-singular}{InverseFunction}
\pmsynonym{nonsingular}{InverseFunction}
\pmsynonym{inverse}{InverseFunction}
\pmrelated{Function}
\pmdefines{invertible function}
\pmdefines{invertible}

\endmetadata

% this is the default PlanetMath preamble.  as your knowledge
% of TeX increases, you will probably want to edit this, but
% it should be fine as is for beginners.

% almost certainly you want these
\usepackage{amssymb}
\usepackage{amsmath}
\usepackage{amsfonts}

% used for TeXing text within eps files
%\usepackage{psfrag}
% need this for including graphics (\includegraphics)
%\usepackage{graphicx}
% for neatly defining theorems and propositions
%\usepackage{amsthm}
% making logically defined graphics
%%%\usepackage{xypic}

% there are many more packages, add them here as you need them

% define commands here

\newcommand{\sR}[0]{\mathbb{R}}
\newcommand{\sC}[0]{\mathbb{C}}
\newcommand{\sN}[0]{\mathbb{N}}
\newcommand{\sZ}[0]{\mathbb{Z}}
\begin{document}
\PMlinkescapeword{term}
\PMlinkescapeword{inverse}
\PMlinkescapeword{satisfies}
{\bf Definition}
Suppose $f:X\to Y$ is a function between sets $X$ and $Y$,
and suppose $f^{-1}:Y\to X$ is a mapping that satisfies 
\begin{eqnarray*}
f^{-1}\circ f &=& \operatorname{id}_X, \\
f\circ f^{-1} &=& \operatorname{id}_Y,
\end{eqnarray*}
where $\operatorname{id}_A$ denotes the identity function on the set $A$.
Then $f^{-1}$ is called the \emph{inverse of} $f$,
or the \emph{inverse function of} $f$. 
If $f$ has an inverse near a point $x\in X$, then $f$ is 
\emph{invertible near $x$}. (That is, if there is a set $U$ containing $x$
such that the restriction of $f$ to $U$ is invertible, then $f$ is invertible
near $x$.) If $f$ is invertible near all $x\in X$, then
$f$ is \emph{invertible}. 

\subsubsection*{Properties}
\begin{enumerate}
\item When an inverse function exists, it is unique.
\item The inverse function and the inverse image of a set coincide 
in the following sense. 
Suppose $f^{-1}(A)$ is the inverse image of a set $A\subset Y$
under a function $f:X\to Y$. 
If $f$ is a bijection, then $f^{-1}(y)=f^{-1}(\{y\})$. 
\item The inverse function of a function $f:X\to Y$ exists if and only
if $f$ is a bijection, that is, $f$ is an injection and a surjection. 
\item A linear mapping between vector spaces is invertible if and only if 
the determinant of the mapping is nonzero. 
\item For differentiable functions between Euclidean spaces, the inverse function 
theorem gives a necessary and sufficient condition for the inverse to exist. 
This can be generalized to maps between Banach spaces which are differentiable
in the sense of Frechet.
\end{enumerate}

\subsubsection*{Remarks}
When $f$ is a linear mapping (for instance, a matrix), the term \emph{non-singular} is 
also used as a synonym for invertible.
%%%%%
%%%%%
\end{document}
