\documentclass[12pt]{article}
\usepackage{pmmeta}
\pmcanonicalname{ComprehensionAxiom}
\pmcreated{2013-03-22 12:56:54}
\pmmodified{2013-03-22 12:56:54}
\pmowner{Henry}{455}
\pmmodifier{Henry}{455}
\pmtitle{comprehension axiom}
\pmrecord{10}{33307}
\pmprivacy{1}
\pmauthor{Henry}{455}
\pmtype{Definition}
\pmcomment{trigger rebuild}
\pmclassification{msc}{03F35}
\pmsynonym{CA}{ComprehensionAxiom}
\pmsynonym{-CA}{ComprehensionAxiom}
\pmsynonym{comprehension}{ComprehensionAxiom}
\pmsynonym{comprehension axiom}{ComprehensionAxiom}
\pmsynonym{axiom of comprehension}{ComprehensionAxiom}
\pmsynonym{separation}{ComprehensionAxiom}
\pmsynonym{separation axiom}{ComprehensionAxiom}
\pmsynonym{axiom of separation}{ComprehensionAxiom}
\pmsynonym{specification}{ComprehensionAxiom}
\pmsynonym{specification axiom}{ComprehensionAxiom}
\pmsynonym{axiom of specification}{ComprehensionAxiom}
\pmsynonym{Aussonderungsaxiom}{ComprehensionAxiom}

% this is the default PlanetMath preamble.  as your knowledge
% of TeX increases, you will probably want to edit this, but
% it should be fine as is for beginners.

% almost certainly you want these
\usepackage{amssymb}
\usepackage{amsmath}
\usepackage{amsfonts}

% used for TeXing text within eps files
%\usepackage{psfrag}
% need this for including graphics (\includegraphics)
%\usepackage{graphicx}
% for neatly defining theorems and propositions
%\usepackage{amsthm}
% making logically defined graphics
%%%\usepackage{xypic}

% there are many more packages, add them here as you need them

% define commands here
%\PMlinkescapeword{theory}
\begin{document}
\PMlinkescapeword{state}\PMlinkescapeword{between}
\PMlinkescapeword{place}


The \emph{axiom of comprehension} (CA) states that every formula defines a set.  That is,
$$\exists X\forall x(x\in X\leftrightarrow\phi(x))\text{for any formula}\phi\text{where}X\text{does not occur free in}\phi$$

The \PMlinkescapetext{names} specification and separation are sometimes used in place of comprehension, particularly for weakened forms of the axiom (see below).

In theories which make no distinction between objects and sets (such as ZF), this formulation leads to Russell's paradox, however in stratified theories this is not a problem (for example second order arithmetic includes the axiom of comprehension).

This axiom can be restricted in various ways.  One possibility is to restrict it to forming subsets of sets:
$$\forall Y\exists X\forall x(x\in X\leftrightarrow x\in Y\wedge\phi(x))\text{ for any formula }\phi\text{ where }X\text{ does not occur free in }\phi$$

This formulation (used in ZF set theory) is sometimes called the Aussonderungsaxiom.

Another way is to restrict $\phi$ to some family $F$, giving the axiom F-CA.  For instance the axiom $\Sigma^0_1$ -CA is:
$$\exists X\forall x(x\in X\leftrightarrow\phi(x))\text{ where }\phi\text{ is }\Sigma^0_1\text{ and }X\text{ does not occur free in }\phi$$

A third form (usually called separation) uses two formulas, and guarantees only that those satisfying one are included while those satisfying the other are excluded.  The unrestricted form is the same as unrestricted collection, but, for instance, $\Sigma^0_1$ separation:
$$\forall x\neg(\phi(x)\wedge\psi(x))\rightarrow\exists X\forall x((\phi(x)\rightarrow x\in X)\wedge(\psi(x)\rightarrow x\notin X))$$$$\text{ where }\phi\text{ and }\psi\text{ are }\Sigma^0_1\text{ and }X\text{ does not occur free in }\phi\text{ or }\psi$$
is weaker than $\Sigma^0_1$ -CA.
%%%%%
%%%%%
\end{document}
