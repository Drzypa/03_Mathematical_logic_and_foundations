\documentclass[12pt]{article}
\usepackage{pmmeta}
\pmcanonicalname{UnionOfCountableSets}
\pmcreated{2013-03-22 19:02:56}
\pmmodified{2013-03-22 19:02:56}
\pmowner{CWoo}{3771}
\pmmodifier{CWoo}{3771}
\pmtitle{union of countable sets}
\pmrecord{11}{41927}
\pmprivacy{1}
\pmauthor{CWoo}{3771}
\pmtype{Result}
\pmcomment{trigger rebuild}
\pmclassification{msc}{03E10}

\endmetadata

\usepackage{amssymb,amscd}
\usepackage{amsmath}
\usepackage{amsfonts}
\usepackage{mathrsfs}
\usepackage{tabls}

% used for TeXing text within eps files
%\usepackage{psfrag}
% need this for including graphics (\includegraphics)
%\usepackage{graphicx}
% for neatly defining theorems and propositions
\usepackage{amsthm}
% making logically defined graphics
%%\usepackage{xypic}
\usepackage{pst-plot}

% define commands here
\newcommand*{\abs}[1]{\left\lvert #1\right\rvert}
\newtheorem{prop}{Proposition}
\newtheorem{thm}{Theorem}
\newtheorem{cor}{Corollary}
\newtheorem{ex}{Example}
\newcommand{\real}{\mathbb{R}}
\newcommand{\pdiff}[2]{\frac{\partial #1}{\partial #2}}
\newcommand{\mpdiff}[3]{\frac{\partial^#1 #2}{\partial #3^#1}}
\begin{document}
In this entry, we prove a useful property of countability which will gives us many more examples of countable sets.

\begin{prop}  $A\cup B$ is countable iff $A$ and $B$ are countable. \end{prop}
\begin{proof}  Clearly, if $A\cup B$ is countable, then $A$ and $B$ are each countable, as they are subsets of a countable set.

Conversely, suppose $f: \mathbb{N}\to A$ and $g:\mathbb{N}\to B$ are two surjections.  Let $C=A\cup B$.  Define $h:\mathbb{N} \to C$ as follows: $h(2n+1)=f(n)$ for $n=0,1,\ldots$, and $h(2n)=g(n)$, for $n=1,2,\ldots$.  Then $h$ is a well-defined function, for each $i\in \mathbb{N}$ is either even or odd, so $h(i)$ is defined in either case.  Finally, $h$ is onto, for if $c\in C$, then $c\in A$ or $c\in B$.  If $c\in A$, then $h(2p+1)=c$ for some $p$, and if $c\in B$, then $h(2q)=c$ for some $q$.  Hence $C$ is countable.
\end{proof}
The idea behind the above proof is to realize that we can list elements of $C$ in the following manner:
\begin{center}
\begin{tabular}{ c c c c c c }
	$f(1)$ & $f(2)$ & $f(3)$ & $f(4)$ & $f(5)$ & $\cdots$ \\
	$g(1)$ & $g(2)$ & $g(3)$ & $g(4)$ & $g(5)$ & $\cdots$ 
\end{tabular}
\end{center}
Therefore, $h$ is defined so its first value is $f(1)$, its second value is $g(1)$, third is $f(2)$, fourth $g(2)$, etc... In the end, all of the elements of $C$ are exhausted by this way of counting.

As a corollary, we have 

\begin{cor} $A_1 \cup A_2 \cup \cdots \cup A_n$ is countable iff each $A_i$ is. \end{cor}
\begin{proof} This is true by induction. \end{proof}

The property can easily be extended to the countably infinite case, the proof of which is just a variant of the above methodology:

\begin{prop} $\bigcup \lbrace A_i \mid i \in \mathbb{N} \rbrace$ is countable iff each $A_i$ is. \end{prop}
\begin{proof}  Again, one direction is obvious, so we concentrate on the other direction.

Let $A=A_1\cup A_2\cup \cdots$.  Suppose we have surjections $f_i:\mathbb{N} \to A_i$ for $i=1,2,\ldots$.  Then listing elements of $A$ in the following manner (table below on the left)
\begin{center}
\begin{tabular}{ c|c c c c }
$i \backslash j$ & $1$ & $2$ & $3$ & $\cdots$ \\
\hline
$1$ &	$f_1(1)$ & $f_1(2)$ & $f_1(3)$ & $\cdots$ \\
$2$ &	$f_2(1)$ & $f_2(2)$ & $f_2(3)$ & $\cdots$ \\
$3$ &	$f_3(1)$ & $f_3(2)$ & $f_3(3)$ & $\cdots$ \\
$\vdots$ &	$\vdots$ & $\vdots$ & $\vdots$ & $\ddots$ 
\end{tabular}
\hspace{1.25cm}
\begin{tabular}{ c|c c c c }
$i \backslash j$ & $1$ & $2$ & $3$ & $\cdots$ \\
\hline
$1$ & $f(1)$ & $f(2)$ & $f(4)$ & $\cdots$ \\
$2$ & $f(3)$ & $f(5)$ & $f(8)$ & $\cdots$ \\
$3$ & $f(6)$ & $f(9)$ & $f(13)$ & $\cdots$ \\
$\vdots$ & $\vdots$ & $\vdots$ & $\vdots$ & $\ddots$ 
\end{tabular}
\end{center}
provides a surjection $f:\mathbb{N}\to A$ (table above on the right).  The first few values of $f$ are 
$$f(1)=f_1(1),\quad f(2)=f_1(2), \quad f(3)=f_2(1),\quad f(4)=f_1(3), \quad f(5)= f_2(2), \quad \ldots$$
\end{proof}
Notice the similarity between the function $f$ above, and the pairing function used in the proof that $\mathbb{N}^2$ is countable \PMlinkname{here}{ProductOfCountableSets}.

\textbf{Remark}. However, the property fails when there are uncountably many sets to deal with.  For example, the union of $\lbrace r\rbrace$ for each $r\in \mathbb{R}$ is uncountable.
%%%%%
%%%%%
\end{document}
