\documentclass[12pt]{article}
\usepackage{pmmeta}
\pmcanonicalname{AxiomSystemForIntuitionisticPropositionalLogic}
\pmcreated{2013-03-22 19:30:58}
\pmmodified{2013-03-22 19:30:58}
\pmowner{CWoo}{3771}
\pmmodifier{CWoo}{3771}
\pmtitle{axiom system for intuitionistic propositional logic}
\pmrecord{38}{42491}
\pmprivacy{1}
\pmauthor{CWoo}{3771}
\pmtype{Axiom}
\pmcomment{trigger rebuild}
\pmclassification{msc}{03F55}
\pmclassification{msc}{03B20}
\pmclassification{msc}{03B55}
\pmrelated{TruthValueSemanticsForIntuitionisticPropositionalLogic}
\pmrelated{SomeTheoremSchemasOfIntuitionisticPropositionalLogic}
\pmrelated{KripkeSemanticsForIntuitionisticPropositionalLogic}
\pmrelated{DeductionTheoremForIntuitionisticPropositionalLogic}
\pmrelated{DisjunctionProperty}

\usepackage{amssymb,amscd}
\usepackage{amsmath}
\usepackage{amsfonts}
\usepackage{mathrsfs}

% used for TeXing text within eps files
%\usepackage{psfrag}
% need this for including graphics (\includegraphics)
%\usepackage{graphicx}
% for neatly defining theorems and propositions
\usepackage{amsthm}
% making logically defined graphics
%%\usepackage{xypic}
\usepackage{pst-plot}
\usepackage{multicol}

% define commands here
\newcommand*{\abs}[1]{\left\lvert #1\right\rvert}
\newtheorem{prop}{Proposition}
\newtheorem{thm}{Theorem}
\newtheorem{ex}{Example}
\newcommand{\real}{\mathbb{R}}
\newcommand{\pdiff}[2]{\frac{\partial #1}{\partial #2}}
\newcommand{\mpdiff}[3]{\frac{\partial^#1 #2}{\partial #3^#1}}

\begin{document}
There are several Hilbert-style axiom systems for intuitionistic propositional logic, or PL$_i$ for short.  One such a system is by Heyting, and is presented in \PMlinkname{this entry}{IntuitionisticLogic}.  Here, we describe another, based on the one by Kleene.  The language of the logic consists of a countably infinite set of propositional letters $p,q,r,\ldots$, and symbols for logical connectives $\to$, $\land$, $\lor$.  Well-formed formulas (wff) are defined recursively as follows:
\begin{itemize}
\item propositional letters are wff
\item if $A,B$ are wff, so are $A\to B$, $A\land B$, $A\lor B$, and $\neg A$.
\end{itemize}
In addition, $A\leftrightarrow B$ (biconditional) is the abbreviation for $(A\to B) \land (B\to A)$, like PL$_c$ (classical propositional logic), 

We also use parentheses to avoid ambiguity.  The axiom schemas for PL$_i$ are
\begin{enumerate}
\item $A \to (B \to A)$.
\item $A \to (B \to A \land B)$.
\item $A \land B \to A$.
\item $A \land B \to B$.
\item $A \to A \lor B$.
\item $B \to A \lor B$.
\item $(A \to C) \to ((B \to C) \to (A \lor B \to C))$.
\item $(A \to B) \to ((A \to (B \to C)) \to (A \to C))$.
\item $(A \to B) \to ((A \to \neg B) \to \neg A)$.
\item $\neg A \to  (A \to B)$.
\end{enumerate}
where $A$, $B$, and $C$ are wff's.  In addition, modus ponens is the only rule of inference for PL$_i$.

As usual, given a set $\Sigma$ of wff's, a deduction of a wff $A$ from $\Sigma$ is a finite sequence of wff's $A_1,\ldots, A_n$ such that $A_n$ is $A$, and $A_i$ is either an axiom, a wff in $\Sigma$, or is obtained by an application of modus ponens on $A_j$ to $A_k$ where $j,k < i$.  In other words, $A_k$ is the wff $A_j \to A_i$.  We write $$\Sigma \vdash_i A$$ to mean that $A$ is a deduction from $\Sigma$.  When $\Sigma$ is the empty set, we say that $A$ is a theorem  (of PL$_i$), and simply write $\vdash_i A$ to mean that $A$ is a theorem.

As with PL$_c$, the deduction theorem holds for PL$_i$.  Using the deduction theorem, one can derive the well-known theorem schemas listed below:
\begin{enumerate}
\item $A\land B\to B\land A$.
\item $(A \to (B\to C)) \to ((A\to B)\to (A\to C))$
\item $A \land \neg A \to B$
\item $A \to \neg \neg A$
\item $\neg \neg \neg A \to \neg A$
\item $(A\to B) \to (\neg B \to \neg A)$
\item $\neg A \land \neg A$
\item $\neg \neg (A \lor \neg A)$
\end{enumerate}
For example, the first schema can be proved as follows:
\begin{proof} From the deduction, 
\begin{multicols}{3}
\begin{enumerate}
\item $A\land B\to A$, 
\item $A\land B\to B$, 
\item $A\land B$, 
\item $A$, 
\item $B$, 
\item $B \to (A \to B\land A)$, 
\item $A \to B\land A$, 
\item $B\land A$,
\end{enumerate}
\end{multicols}
we have $A \land B \vdash_i B\land A$, and therefore $\vdash_i (A\land B)\to (B\land A)$ by the deduction theorem.
\end{proof}
Deductions of the other theorem schemas can be found \PMlinkname{here}{SomeTheoremSchemasOfIntuitionisticPropositionalLogic}.  In fact, it is not hard to see that $\vdash_i X$ implies $\vdash_c X$ (that $X$ is a theorem of PL$_c$).  The converse is false.  The following are theorems of PL$_c$, not PL$_i$:
\begin{multicols}{2}
\begin{enumerate}
\item $A\lor \neg A$.
\item $\neg \neg A \to A$
\item $(\neg A \to \neg B) \to (B\to A)$
\item $((A\to B)\to A) \to A$
\item $(\neg A \to B) \to ((\neg A \to \neg B) \to A)$
\end{enumerate}
\end{multicols}
\textbf{Remark}.  It is interesting to note, however, if any one of the above schemas were added to the list of axioms for PL$_i$, then the resulting system is an axiom system for PL$_c$.  In notation, 
\begin{center}
PL$_i + X = $ PL$_c$,
\end{center}
where $X$ is one of the schemas above.  When this equation holds for some $X$, it is necessary that $\vdash_c X$ and $\not \vdash_i X$.  However, this condition is not sufficient to imply the equation, even if PL$_i +X$ is consistent (that is, the schema $C\land \neg C$ of wff's are not theorems).  One such schema is $\neg \neg A \lor \neg A$.  A logical system PL such that PL$_i \le $ PL $\le$ PL$_c$ is called an \emph{intermediate logic}.  It can be shown that there are infinitely many such intermediate logics.

\textbf{Remark}.  Yet another popular axiom system for PL$_i$ uses the symbol $\perp$ (for falsity) instead of $\neg$.  The wff's in this language consists of all propositional letters, the symbol $\perp$, and $A\land B$, $A\lor B$, and $A \to B$, whenever $A$ and $B$ are wff's.  The axiom schemas consist of the first seven axiom schemas in the first system above, as well as
\begin{enumerate}
\item $(A \to (B\to C)) \to ((A\to B)\to (A\to C))$ (the second theorem schema above)
\item $\perp \to A$.
\end{enumerate}
$\neg A$ is the abbreviation for $A \to \perp$.  The only rule of inference is modus ponens.  Deductions and theorems are defined in the exact same way as above.  Let us write $\vdash_{i1} A$ to mean wff $A$ is a theorem in this axiom system.  As mentioned, both axiom systems are equivalent, in that $\vdash_{i1} A$ implies $\vdash_i A$, and $\vdash_i A$ implies $\vdash_{i1} A^*$, where $A^*$ is the wff obtained from $A$ by replacing every occurrence of $\perp$ by the wff $(p\land \neg p)$, where $p$ is a propositional letter not occurring in $A$.

%%%%%
%%%%%
\end{document}
