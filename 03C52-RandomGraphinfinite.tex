\documentclass[12pt]{article}
\usepackage{pmmeta}
\pmcanonicalname{RandomGraphinfinite}
\pmcreated{2013-03-22 13:32:18}
\pmmodified{2013-03-22 13:32:18}
\pmowner{bbukh}{348}
\pmmodifier{bbukh}{348}
\pmtitle{random graph (infinite)}
\pmrecord{8}{34137}
\pmprivacy{1}
\pmauthor{bbukh}{348}
\pmtype{Definition}
\pmcomment{trigger rebuild}
\pmclassification{msc}{03C52}
\pmclassification{msc}{03C30}
\pmclassification{msc}{03C15}
\pmclassification{msc}{05C30}
\pmrelated{ExampleOfUniversalStructure}
\pmrelated{Homogeneous4}
\pmdefines{random graph}

\endmetadata

% this is the default PlanetMath preamble.  as your knowledge
% of TeX increases, you will probably want to edit this, but
% it should be fine as is for beginners.

% almost certainly you want these
\usepackage{amssymb}
\usepackage{amsmath}
\usepackage{amsfonts}

% used for TeXing text within eps files
%\usepackage{psfrag}
% need this for including graphics (\includegraphics)
%\usepackage{graphicx}
% for neatly defining theorems and propositions
%\usepackage{amsthm}
% making logically defined graphics
%%%\usepackage{xypic}

% there are many more packages, add them here as you need them

% define commands here

\newtheorem{pp}{Proposition}
\begin{document}
Suppose we have some method $M$ of generating sequences of letters from $\{p,q\}$ so that at each generation the probability of obtaining $p$ is $x$, a real number strictly between $0$ and $1$. 

Let $\{a_{i}:i<\omega\}$ be a set of vertices. For each $i<\omega$ , $i\geq 1$  we construct a graph $G_{i}$ on the vertices $a_{1},\ldots,a_{i}$ recursively.

\begin{itemize} \item $G_{1}$ is the unique graph on one vertex.

\item For $i>1$ we must describe for any $j<k\leq i$ when  $a_{j}$ and $a_{k}$ are joined.
\begin{itemize} \item If $k<i$ then join $a_{j}$ and $a_{k}$ in $G_{i}$ iff $a_{j}$ and $a_{k}$ are joined in $G_{i-1}$ 
\item If $k=i$ then generate a letter $l(j,k)$ with $M$. Join $a_{j}$ to $a_{k}$ iff $l(j,k)=p$.
\end{itemize}
\end{itemize}

Now let $\Gamma$ be the graph on $\{a_{i}:i<\omega\}$ so that for any $n,m <\omega$, $a_{n}$ is joined to $a_{m}$ in $\Gamma$ iff it is in some $G_{i}$.

Then we call $\Gamma$ a \emph{random graph}. Consider the following property which we shall call \emph{f-saturation}:

Given any finite disjoint $U$ and $V$, subsets of $\{a_{i}:i<\omega\}$ there is some $a_{n} \in \{a_{i}:i<\omega\} \setminus (U \cup V)$ so that $a_{n}$ is joined to every point of $U$ and no points in $V$.

\begin{pp} A random graph has f-saturation with probability $1$.
\end{pp}
Proof: Let $b_{1},b_{2},\ldots,b_{n},\ldots$ be an enumeration of $\{a_{i}:i<\omega\} \setminus (U \cup V)$. 
We say that $b_{i}$ is {\em correctly joined}
 to $(U,V)$ iff it is joined to all the members of $U$ and non of the members of $V$. Then the probability that $b_{i}$ is not correctly joined is $(1-x^{|U|}(1-x)^{|V|})$ which is some real number $y$ strictly between $0$ and $1$. 
The probability that none of the first $m$ are correctly joined is $y^{m}$ and the probability that none of the $b_{i}$s are correctly joined is $\textrm{lim}_{n \to \infty} y^{n}=0$. Thus one of the $b_{i}$s is correctly joined.

\begin{pp} Any two countable graphs with f-saturation are isomorphic.\end{pp}
Proof: This is via a back and forth argument. The property of f-saturation is exactly what is needed.

\medskip

Thus although the system of generation of a random graph looked as though it could deliver many potentially different graphs, this is not the case.  Thus we talk about \emph{the} random graph.


\medskip

The random graph can also be constructed as a Fraisse limit of all finite graphs, and in many other ways.  It is homogeneous and universal for the class of all countable graphs.

The theorem that almost every two infinite random graphs are isomorphic was first proved in \cite{cite:erdos_renyi_randgr}.

\begin{thebibliography}{1}

\bibitem{cite:erdos_renyi_randgr}
Paul Erd{\H{o}}s and Alfr{\'e}d R{\'e}nyi.
\newblock Asymmetric graphs.
\newblock {\em Acta Math. Acad. Sci. Hung.}, 14:295--315, 1963.

\end{thebibliography}
%
%@ARTICLE{cite:erdos_renyi_randgr,
%  author  = {Paul Erd{\H{o}}s and Alfr{\'e}d R{\'e}nyi},
%  title   = {Asymmetric Graphs},
%  journal = {Acta Math. Acad. Sci. Hung.},
%  volume  = 14,
%  year    = 1963,
%  pages   = {295--315}
%}
%
%%%%%
%%%%%
\end{document}
