\documentclass[12pt]{article}
\usepackage{pmmeta}
\pmcanonicalname{Ultrauniversal}
\pmcreated{2013-03-22 19:36:18}
\pmmodified{2013-03-22 19:36:18}
\pmowner{Naturman}{26369}
\pmmodifier{Naturman}{26369}
\pmtitle{ultra-universal}
\pmrecord{29}{42597}
\pmprivacy{1}
\pmauthor{Naturman}{26369}
\pmtype{Definition}
\pmcomment{trigger rebuild}
\pmclassification{msc}{03C20}
\pmclassification{msc}{03C52}
\pmclassification{msc}{03C50}
%\pmkeywords{ultra-universal}
\pmrelated{universal}
\pmrelated{jointembeddingproperty}
\pmrelated{JointEmbeddingProperty}
\pmdefines{ultra-universal model}
\pmdefines{ultra-universal theory}
\pmdefines{ultra-universal class}

\endmetadata

% this is the default PlanetMath preamble.  as your knowledge
% of TeX increases, you will probably want to edit this, but
% it should be fine as is for beginners.

% almost certainly you want these
\usepackage{amssymb}
\usepackage{amsmath}
\usepackage{amsfonts}

% used for TeXing text within eps files
%\usepackage{psfrag}
% need this for including graphics (\includegraphics)
%\usepackage{graphicx}
% for neatly defining theorems and propositions
%\usepackage{amsthm}
% making logically defined graphics
%%%\usepackage{xypic}

% there are many more packages, add them here as you need them

% define commands here

\begin{document}
Let $T$ be a first order theory. A model $M$ of $T$ is said to be an \emph{ultra-universal model} of $T$ iff for every model $A$ of $T$ there exists and ultra-power of $M$ into which $A$ can be embedded. \cite{VC, UM}

If $T$ has an ultra-universal model it is referred to as an \emph{ultra-universal theory}. The class of models of an ultra-universal theory is called an \emph{ultra-universal class}. If $T$ is an ultra-universal theory with elementary class $K$ and ultra-universal model $M$ then $M$ is said to be \emph{ultra-universal in} $K$. \cite{UM}

\subsubsection{Characterizations}

Ultra-universal classes are precisely the non-empty elementary classes having the joint embedding property. \cite{UM}

Ultra-universal models can be characterized in terms of universal or existential sentences:

Let $T$ be theory and let $M$ be a model of $T$. The following are equivalent: \cite{UM}

\begin{enumerate}
\item $M$ is an ultra-universal model of $T$
\item Every universal sentence holding in $M$ holds in all models of $T$
\item Every existential sentence holding in some model of $T$ holds in $M$
\end{enumerate}

A theory $T$ is ultra-universal iff it is consistent and for all universal sentences $\phi$ and $\psi$, $T\vdash\phi\vee\psi$ implies $T\vdash\phi$ or $T\vdash\psi$. \cite{UM}

A complete consistent theory is always ultra-universal. More generally the set of universal sentences $\Sigma$ of a complete consistent theory $T$ is always an ultra-universal theory - a model of $T$ is an ultra-universal model of $\Sigma$. Ultra-universal theories are precisely those theories $T$ which are consistent and can be extended to a complete consistent theory without introducing any universal sentences that are not deducible from $T$. \cite{UM}

In terms of the Lindenbaum-Tarski algebra for a first order language $L$, a theory $T$ in $L$ is ultra-universal iff the filter $F$ that it generates in the Lindenbaum-Tarski algebra is proper and can be extended to an ultrafilter $U$ such that $F \cap A = U \cap A$ where $A$ is the sub-lattice of universal sentences. Moreover $T$ is ultra-universal iff $F \cap A$ is a prime proper filter in $A$. Thus ultra-universal theories correspond to prime proper filters in the bounded distributive lattice of universal sentences. \cite{UM}

\subsubsection{Examples}

\begin{itemize}
\item Any infinite partition lattice is ultra-universal in the variety of lattices \cite{VC}
\item Any infinite symmetric group is ultra-universal in the variety of groups \cite{UM}
\item The monoid of functions defined on an infinite set is ultra-universal in the variety of monoids \cite{UM}
\item The semigroup reduct of the monoid of functions defined on an infinite set is ultra-universal in the varierty of semigroups \cite{UM}
\item The power set interior (or closure) algebra on Cantor's discontinuum or on a denumerable co-finite topological space is ultra-universal in the variety of interior (or closure) algebras \cite{IA}
\item The product of all fintely generated substructures (up to isomorphism) of members of a factor embeddable universal Horn class (in particular a factor embeddable variety of algebraic structures) is ultra-universal in the class. \cite{UM}
\item More generally, any model in an elementary class having the property that all fintely generated substructures of the class are embeddable in it, is ultra-universal in the class. \cite{UM}
\end{itemize}

\begin{thebibliography}{1}
\bibitem{VC} Peter Bruyns, Henry Rose: \emph{Varieties with cofinal sets: examples and amalgamation}, Proc. Amer. Math. Soc. 111 (1991), 833-840
\bibitem{IA} Colin Naturman, Henry Rose: \emph{Interior algebras: some universal algebraic aspects}, J. Korean Math. Soc. 30 (1993), No. 1, pp. 1-23
\bibitem{UM} Colin Naturman, Henry Rose: \emph{Ultra-universal models}, Quaestiones Mathematicae, 15(2), 1992, 189-195
\end{thebibliography}

%%%%%
%%%%%
\end{document}
