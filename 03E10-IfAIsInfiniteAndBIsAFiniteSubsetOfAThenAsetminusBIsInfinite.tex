\documentclass[12pt]{article}
\usepackage{pmmeta}
\pmcanonicalname{IfAIsInfiniteAndBIsAFiniteSubsetOfAThenAsetminusBIsInfinite}
\pmcreated{2013-03-22 13:34:42}
\pmmodified{2013-03-22 13:34:42}
\pmowner{mathcam}{2727}
\pmmodifier{mathcam}{2727}
\pmtitle{if $A$ is infinite and $B$ is a finite subset of $A\,\!,$ then $A\setminus B$ is infinite}
\pmrecord{7}{34199}
\pmprivacy{1}
\pmauthor{mathcam}{2727}
\pmtype{Theorem}
\pmcomment{trigger rebuild}
\pmclassification{msc}{03E10}

% this is the default PlanetMath preamble.  as your knowledge
% of TeX increases, you will probably want to edit this, but
% it should be fine as is for beginners.

% almost certainly you want these
\usepackage{amssymb}
\usepackage{amsmath}
\usepackage{amsfonts}

% used for TeXing text within eps files
%\usepackage{psfrag}
% need this for including graphics (\includegraphics)
%\usepackage{graphicx}
% for neatly defining theorems and propositions
%\usepackage{amsthm}
% making logically defined graphics
%%%\usepackage{xypic}

% there are many more packages, add them here as you need them

% define commands here
\newcommand{\sN}[0]{\mathbb{N}}
\begin{document}


{\bf Theorem.} If $A$ is an infinite set and $B$ is a 
finite subset of $A$, then $A\setminus B$ is infinite. 

\emph{Proof.} The proof is by contradiction. If $A\setminus B$ 
would be finite, there would exist a $k\in \sN$ and a bijection 
$f:\{1,\ldots, k\}\to A\setminus B$. Since $B$ is finite, there
also exists a bijection $g:\{1,\ldots, l\}\to B$. We can then define 
a mapping $h:\{1,\ldots, k+l\} \to A$ by
\begin{eqnarray*}
h(i)&=& \left\{ \begin {array}{ll} 
  f(i) & \mbox{when} \, i\in \{1,\ldots, k\}, \\
  g(i-k) & \mbox{when}\, i\in\{k+1,\ldots, k+l\}. \\
  \end{array} \right.
\end{eqnarray*}
Since $f$ and $g$ are bijections, $h$ is a bijection between
a finite subset of $\sN$ and $A$. This is a contradiction
since $A$ is infinite. $\Box$
%%%%%
%%%%%
\end{document}
