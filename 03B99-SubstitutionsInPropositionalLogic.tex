\documentclass[12pt]{article}
\usepackage{pmmeta}
\pmcanonicalname{SubstitutionsInPropositionalLogic}
\pmcreated{2013-03-22 19:32:41}
\pmmodified{2013-03-22 19:32:41}
\pmowner{CWoo}{3771}
\pmmodifier{CWoo}{3771}
\pmtitle{substitutions in propositional logic}
\pmrecord{22}{42525}
\pmprivacy{1}
\pmauthor{CWoo}{3771}
\pmtype{Definition}
\pmcomment{trigger rebuild}
\pmclassification{msc}{03B99}
\pmsynonym{instance}{SubstitutionsInPropositionalLogic}
\pmsynonym{US}{SubstitutionsInPropositionalLogic}
\pmrelated{Substitutability}
\pmdefines{simultaneous substitution}
\pmdefines{uniform substitution}
\pmdefines{schema}
\pmdefines{substitution instance}

\endmetadata

\usepackage{amssymb,amscd}
\usepackage{amsmath}
\usepackage{amsfonts}
\usepackage{mathrsfs}

% used for TeXing text within eps files
%\usepackage{psfrag}
% need this for including graphics (\includegraphics)
%\usepackage{graphicx}
% for neatly defining theorems and propositions
\usepackage{amsthm}
% making logically defined graphics
%%\usepackage{xypic}
\usepackage{pst-plot}

% define commands here
\newcommand*{\abs}[1]{\left\lvert #1\right\rvert}
\newtheorem{prop}{Proposition}
\newtheorem{thm}{Theorem}
\newtheorem{ex}{Example}
\newcommand{\real}{\mathbb{R}}
\newcommand{\pdiff}[2]{\frac{\partial #1}{\partial #2}}
\newcommand{\mpdiff}[3]{\frac{\partial^#1 #2}{\partial #3^#1}}

\begin{document}
\subsubsection*{Uniform Substitution}
One area of mathematics where substitution plays a prominent role is mathematical logic.  In this entry, we are mainly interested in propositional logic.  Recall that a substitution is a function $s:\Sigma_1^* \to P(\Sigma_2^*)$ preserving the empty word and concatenation.  In a propositional logic, $s$ has the following additional characteristics:
\begin{itemize}
\item only one alphabet $\Sigma$, often infinite, consisting of all propositional variables, logical connectives, and parentheses, is involved;
\item $s$ maps to singletons, so we might as well think of $s$ as mapping into $\Sigma^*$;
\item $s$ fixes the logical connectives and parentheses;
\item $s(\Sigma^*)$ is a set of well-formed formulas (rather than a set of words over $\Sigma$).
\end{itemize}
$s$ is also called a \emph{uniform substitution} because for any propositional variable $p$ that occurs in $A$, $s$ replaces each and every occurrence of $p$ in $A$ by $s(p)$.  In practice, we write $$A[B/p]$$ to mean change all occurrences of $p$ in $A$ by $B$, and leave all other variables fixed.  This includes the case when $p$ does not occur in $A$, in which case $A[B/p]=A$.  $A[B/p]$ corresponds to a substitution that sends $p$ to $B$ and fixes all variables in $A$ not equal to $p$.  This is called an \emph{individual substitution}.  More generally,
$$A[B_1/p_1,\ldots,B_n/p_n]$$
means: change all occurrences of $p_i$ in $A$ to $B_i$, for each $i=1,\ldots,n$, and leave all other variables fixed.  This is called a \emph{simultaneous substitution}, and corresponds to a substitution that sends $p_i$ to $B_i$ and fixes all other variables in $A$.  Simultaneous substitutions are not the same as iterated individual substitutions:
$$A[B_1/p_1,\ldots, B_n/p_n] \ne A[B_1/p_1]\cdots [B_n/p_n].$$
For example, if $A=p\lor q$, then $A[q/p,p/q]= q\lor p \ne p\lor p =A[q/p][p/q]$.

\subsubsection*{Recursive Definition of Substitution}
Substitutions can also be defined inducitvely, starting from propositional variables.  For the sake of simplicity, we only define uniform substitutions on one variable.

\textbf{Definition}.  Suppose $A$ and $B$ are wff's, and $p$ a propositional variable.  Then
\begin{enumerate}
\item $A$ is a propositional variable.  If $A$ is $p$, then $A[B/p]:=B$.  Otherwise, $A[B/p]:=A$.
\item $\perp[B/p]:=\perp$
\item If $A$ is $C\to D$, then $A[B/p]:=C[B/p]\to D[B/p]$.
\item If $A$ is $C\land D$, then $A[B/p]:=C[B/p]\land D[B/p]$.
\item If $A$ is $C\lor D$, then $A[B/p]:=C[B/p]\lor D[B/p]$.
\end{enumerate}
Since $\neg A$ is $A\to \perp$, we see that $(\neg A)[B/p]$ is $\neg (A[B/p])$.  In addition, if the language of the logic contains a modal connective, say $\square$, we have
\begin{enumerate}
\setcounter{enumi}{5}
\item If $A$ is $\square C$, then $A[B/p]:=\square C[B/p]$.
\end{enumerate}

\subsubsection*{Sets Closed under Uniform Substitution}
A set $\Lambda$ of wff's is said to be closed under uniform substitution if for any $A\in \Lambda$, $s(A)\in \Lambda$ for any (uniform) substitution $s$.  We also say the set is closed under US (for uniform substitution), or obeys the rule of US.  The smallest set containing a wff $A$ that is closed under US is called a \emph{schema} based on $A$, and is denoted by $\mathbf{A}$, the bold face version of $A$.  An element of $\mathbf{A}$ is called a \emph{substitution instance}, or simply an \emph{instance} of $A$.  For example, if $A$ is $p\to (q\to p)$, where $p$ and $q$ are propositional variables, then $$(D\to B)\to (((D\to B)\to C) \to (D\to B))$$ is a substitution instance of $A$, where $p$ is replaced by $D\to B$ and $q$ by $(D\to B)\to C$.

It is easy to see that a set is closed under US iff it is closed under individual substitution (IS).  Obviously, one direction is clear, as IS is just special case of US.  Conversely, suppose $A\in \Lambda$, which is closed under IS.  Let $p_1,\ldots,p_n$ be all the propositional variables in $A$, and $X_1,\ldots,X_n$ are arbitrary wff's.  Let $q_1,\ldots, q_n$ be propositional variables, none of which occur in any of $A, X_1,\ldots, X_n$.  Then
$$A[X_1/p_1,\ldots,X_n/p_n]=A[q_1/p_1]\cdots[q_n/p_n][X_1/q_1]\cdots [X_n/q_n] \in \Lambda.$$

There are in general two ways to specify a given axiom system for a propositional logic:
\begin{itemize}
\item list wff's $A_1,A_2,\ldots$ as axioms, and $R_1,\ldots$ as inference rules, including US, or
\item list schemas $\mathbf{A_1},\mathbf{A_2},\ldots$ as axiom schemas, and $R_1,\ldots$ as inference rules, excluding US
\end{itemize}
Both specifications are equivalent, in that they produce the same set of theorems.

\subsubsection*{Non-Uniform Substitution}
It is also possible to consider substitutions that only replace some, but not all, occurrences of a propositional variable in a formula $A$, or replace a variable at different locations in $A$ by different formulas.  For example, if $A$ is $(p\to q)\lor (q\to p)$, then 
\begin{itemize}
\item
$(B\to q)\lor (B\to p)$ is obtained by replacing the first occurrences of $p$ and $q$ by $B$;
\item
$(B\to q)\lor (q\to C)$ is obtained by replacing the first and second occurrences of $p$ by $B$ and $C$ respectively.
\end{itemize}
Replacements such as these are called \emph{non-uniform substitutions}.  Technically, these are no longer substitutions, for they are no longer functions on $\Sigma^*$, as individual variables may be mapped to different things depending on their location in the formula.  In the first example above, $p$ is mapped to both $B$ and $p$, depending on whether it is the first or second occurrence in $A$.

To present a non-uniform substitution, let $\overline{p}$ be the list all the propositional variables $p_1, \ldots, p_n$ in $A$ in order.  Note that since a propositional variable $p_i$ may occur multiple times in $A$, $p$ may occur multiple times in $\overline{p}$.  Suppose each $p_i$ is replaced by $B_i$.  Let $\overline{B}$ be the list $B_1,\ldots, B_n$.  Then we denote $$A[\overline{B}/\overline{p}]$$ by this non-uniform substitution.  In the two examples above, $A[(B,q,B,p)/(p,q,q,p)]$ is the first wff, while $A[(B,q,q,C)/(p,q,q,p)]$ is the second.

Nevertheless, non-uniform substitution is useful in one respect: preservation of theoremhood.  This fact is the famous substitution theorem, which says, if $p_1,\ldots, p_n$ are all the propositional variables (not necessarily distinct) in a wff $A$ that are listed in order of appearance in $A$, then replacing each variable by deductively equivalent formulas produces equivalent result.  In short, if $\vdash B_i \leftrightarrow C_i$ for $i=1,\ldots, n$, then $$\vdash A[\overline{B}/\overline{p}] \leftrightarrow A[\overline{C}/\overline{p}].$$

A set closed under non-uniform substitution (NUS) is defined in the same way as that of uniform substitution.  It is easy to see that the smallest set closed under NUS containing the formula $A$ is the schema $\mathbf{A[\overline{q}/\overline{p}]}$, where $\overline{q}$ is a list of pairwise distinct propositional variables.  For example, the smallest set closed under NUS containing $(p\to q)\lor (q\to p)$ is $\mathbf{(p\to q)\lor(r\to s)}$.  It is not hard to see that if the NUS closure of a formula is used as an axiom schema for a logic, with modus ponens as a rule of inference, then the logic is inconsistent.

\subsubsection*{First Order Logic}
In a first order logic, substitutions are more complicated.  Given a wff $A$, $A[B/p]$ does not necessarily mean replacing all occurrences of $p$ by $B$.  Here, again, a substitution is no longer a substitution in the sense above.  In fact, replacements of symbols, like non-uniform substitutions, are conditional on the locations of the symbols in the wff $A$.  These conditions are collectively known as the \emph{substitutability} of a term $B$ for the variable $p$, and is discussed in more detail \PMlinkname{here}{Substitutability}.

%%%%%
%%%%%
\end{document}
