\documentclass[12pt]{article}
\usepackage{pmmeta}
\pmcanonicalname{Signature}
\pmcreated{2013-03-22 13:51:48}
\pmmodified{2013-03-22 13:51:48}
\pmowner{CWoo}{3771}
\pmmodifier{CWoo}{3771}
\pmtitle{signature}
\pmrecord{16}{34603}
\pmprivacy{1}
\pmauthor{CWoo}{3771}
\pmtype{Definition}
\pmcomment{trigger rebuild}
\pmclassification{msc}{03C07}
\pmsynonym{language}{Signature}
\pmsynonym{non-logical symbols}{Signature}
\pmdefines{constant symbol}
\pmdefines{function symbol}
\pmdefines{relation symbol}

% this is the default PlanetMath preamble.  as your knowledge
% of TeX increases, you will probably want to edit this, but
% it should be fine as is for beginners.

% almost certainly you want these
\usepackage{amssymb}
\usepackage{amsmath}
\usepackage{amsfonts}

% used for TeXing text within eps files
%\usepackage{psfrag}
% need this for including graphics (\includegraphics)
%\usepackage{graphicx}
% for neatly defining theorems and propositions
%\usepackage{amsthm}
% making logically defined graphics
%%%\usepackage{xypic}

% there are many more packages, add them here as you need them

% define commands here
\let	\proves	=	\vdash
\let	\implies	=	\rightarrow
\let	\Implies	=	\Rightarrow
\let	\iff		=	\leftrightarrow
\let	\Iff		=	\Leftrightarrow
\let	\iso		=	\cong
\let	\embeds	=	\righthookarrow
\newcommand{\eqclass}[1]{[\![#1]\!]}
\newcommand{\set}[1]{\{#1\}}
\newcommand{\tuple}[1]{\langle#1\rangle}
\newcommand{\gen}[1]{\langle\!\langle #1 \rangle\!\rangle}
\newcommand{\powerset}{\mathcal P}
\newcommand{\A}{\mathfrak{A}}
\newcommand{\B}{\mathfrak{B}}
\DeclareMathOperator{\dom}{dom}
\DeclareMathOperator{\range}{range}
\DeclareMathOperator{\im}{im}
\begin{document}
A \emph{signature} $\Sigma$ is a set 
\begin{displaymath}
\Sigma:=\left(\bigcup_{n\in\omega}\mathcal{R}_n\right)\cup\left(\bigcup_{n\in\omega}\mathcal{F}_n
\right)\cup\mathcal{C}
\end{displaymath}
where for each natural number $n>0$, 
\begin{description}
\item[$\bullet$] $\mathcal{R}_n$ is a (usually countable) set of $n$-ary relation symbols.
\item[$\bullet$] $\mathcal{F}_n$ is a (usually countable) set of $n$-ary function symbols.
\item[$\bullet$] $\mathcal{C}$ is a (usually countable) set of constant symbols.
\end{description}

We require that all these sets be pairwise disjoint.

Given a signature $\Sigma$, a $\Sigma$-structure is then a structure $\mathcal{A}$, whose underlying set is some set $A$, with elements $\mathcal{A}_c \in A$ for each constant symbol $c\in \Sigma$, $n$-ary operations $\mathcal{A}_f$ on $A$ for each $n$-ary function symbol $f\in \Sigma$, for each $n$, and $m$-ary relations $\mathcal{A}_R$ on $A$ for each $m$-ary relation symbol $R\in \Sigma$.

On the other hand, every structure is associated with a signature.  For every structure, it has an underlying set, together with a collection of ``designated'' objects that ``define'' the structure.  These objects may be elements of the underlying set, operations on the set, or relations on the set.  For each such ``designated'' object, pick a symbol for it.  Make sure all symbols used are distinct from one another.  Then the collection of all such symbols is a signature for the structure.

For most structures that we encounter, the set $\Sigma$ is finite, but we allow it to be infinite, even uncountable, as this sometimes makes things easier, and just about everything still works when the signature is uncountable.

\textbf{Examples}:
\begin{itemize}
\item A signature of sets is the empty set.
\item A signature of pointed sets is a singleton consisting of a constant symbol.
\item A signature of groups is a set $\lbrace e, ^{-1}, \cdot \rbrace$, where 
\begin{enumerate}
\item $e$ (group identity symbol) is a constant symbol, 
\item $^{-1}$ (group inverse symbol) is a unary function symbol, and 
\item $\cdot$ (group multiplication symbol) is a binary function symbol.
\end{enumerate}
\item A signature of fields is a set $\lbrace 0,1, -, ^{-1}, +, \cdot \rbrace$, where 
\begin{enumerate}
\item $0$ (additive identity symbol) and $1$ (multiplicative identity symbol) are constant symbols, 
\item $-$ (the additive inverse symbol) and $^{-1}$ (the multiplicative inverse symbol) are the unary function symbols, and 
\item $+$ (addition symbol) $\cdot$ (multiplication symbol) are binary function symbols.
\end{enumerate}
\item A signature of posets is a singleton $\lbrace \le \rbrace$, where $\le$ (partial ordering symbol) is a binary relation symbol.
\item A signature of vector spaces over a fixed field $k$ consists of the following
\begin{enumerate}
\item $0$ (additive identity symbol) is the constant symbol,
\item $+$ (vector addition symbol) is the binary function symbol, and
\item $\cdot_r$ (multiplication by scalar $r$ symbol) is the unary function symbol, for \emph{each} $r\in k$.
\end{enumerate}
\end{itemize}

\textbf{Remark}.  Given a signature $\Sigma$, the set $L$ of logical symbols from first order logic, and a countably infinite set $V$ of variables, we can form a first order language, consisting of all formulas built from these symbols (in $\Sigma\cup L\cup V$).  The language so-created is uniquely determined by $\Sigma$.  In the literature, it is a common practice to identify $\Sigma$ both as a signature and the unique language it generates.

\begin{thebibliography}{99}
\bibitem{H}
W.~Hodges, {\it A Shorter Model Theory}, Cambridge University Press, (1997).
\bibitem{M}
D.~Marker, {\it Model Theory, An Introduction}, Springer, (2002).
\end{thebibliography}
%%%%%
%%%%%
\end{document}
