\documentclass[12pt]{article}
\usepackage{pmmeta}
\pmcanonicalname{SomeTheoremSchemasOfIntuitionisticPropositionalLogic}
\pmcreated{2013-03-22 19:31:21}
\pmmodified{2013-03-22 19:31:21}
\pmowner{CWoo}{3771}
\pmmodifier{CWoo}{3771}
\pmtitle{some theorem schemas of intuitionistic propositional logic}
\pmrecord{22}{42498}
\pmprivacy{1}
\pmauthor{CWoo}{3771}
\pmtype{Definition}
\pmcomment{trigger rebuild}
\pmclassification{msc}{03B20}
\pmclassification{msc}{03F55}
\pmrelated{AxiomSystemForIntuitionisticLogic}

\usepackage{amssymb,amscd}
\usepackage{amsmath}
\usepackage{amsfonts}
\usepackage{mathrsfs}

% used for TeXing text within eps files
%\usepackage{psfrag}
% need this for including graphics (\includegraphics)
%\usepackage{graphicx}
% for neatly defining theorems and propositions
\usepackage{amsthm}
% making logically defined graphics
%%\usepackage{xypic}
\usepackage{pst-plot}
\usepackage{multicol}

% define commands here
\newcommand*{\abs}[1]{\left\lvert #1\right\rvert}
\newtheorem{prop}{Proposition}
\newtheorem{thm}{Theorem}
\newtheorem{ex}{Example}
\newcommand{\real}{\mathbb{R}}
\newcommand{\pdiff}[2]{\frac{\partial #1}{\partial #2}}
\newcommand{\mpdiff}[3]{\frac{\partial^#1 #2}{\partial #3^#1}}

\begin{document}
We present some theorem schemas of intuitionistic propositional logic and their deductions, based on the axiom system given in \PMlinkname{this entry}{AxiomSystemForIntuitionisticLogic}.

1. $A\lor B\to B\lor A$
\begin{proof} From the deduction
\begin{enumerate}
\item $A \to B\lor A$, 
\item $B \to B\lor A$,
\item $(A\to B\lor A)\to ((B\to B\lor A)\to (A\lor B\to B\lor A))$,
\item $(B\to B\lor A)\to (A\lor B\to B\lor A)$,
\item $A\lor B\to B\lor A$,
\end{enumerate}
so we get $A \lor B \vdash_i B\lor A$, and therefore $\vdash_i A\lor B\to B\lor A$ by the deduction theorem.
\end{proof}

2. $(A\land B)\land C \to A \land (B\land C)$
\begin{proof} From the deduction
\begin{multicols}{2}
\begin{enumerate}
\item $(A\land B)\land C \to A\to B$, 
\item $(A\land B)\land C \to C$, 
\item $(A\land B)\land C$, 
\item $A\land B$, 
\item $C$, 
\item $A\land B \to A$,
\item $A\land B \to B$,
\item $A$,
\item $B$,
\item $B\to (C\to B\land C)$,
\item $C\to B\land C$,
\item $B\land C$,
\item $A\to (B\land C \to A\land (B\land C))$,
\item $B\land C\to A\land (B\land C)$,
\item $A\land (B\land C)$,
\end{enumerate}
\end{multicols}
so $(A \land B)\land C \vdash_i A \land (B\land C)$, and therefore $\vdash_i (A\land B)\land C \to A \land (B\land C)$ by the deduction theorem.
\end{proof}

3. $(A \to (B\to C)) \to ((A\to B)\to (A\to C))$
\begin{proof} From the deduction
\begin{enumerate}
\item $(A \to B) \to ((A \to (B \to C)) \to (A \to C))$,
\item $A\to B$,
\item $(A \to (B \to C)) \to (A \to C)$,
\item $A \to (B\to C)$, 
\item $A \to C$,
\end{enumerate}
so $A \to (B\to C), A\to B \vdash_i A\to C$.  By two applications of the deduction theorem, $\vdash_i (A \to (B\to C)) \to ((A\to B)\to (A\to C))$.
\end{proof}

4. $A\land \neg A\to B$
\begin{proof} From the deduction
\begin{multicols}{3}
\begin{enumerate}
\item $A\land \neg A\to A$,
\item $A\land \neg A\to \neg A$,
\item $A\land \neg A$,
\item $\neg A$,
\item $A$,
\item $\neg A \to (A\to B)$,
\item $A \to B$, 
\item $B$
\end{enumerate}
\end{multicols}
so $A\land \neg A \vdash_i B$.  By the deduction theorem, $\vdash_i A\land \neg A\to B$.
\end{proof}

5. $A \to \neg \neg A$
\begin{proof} From the deduction
\begin{multicols}{2}
\begin{enumerate}
\item $A\to (\neg A \to A)$,
\item $A$,
\item $\neg A \to A$,
\item $(\neg A \to A) \to ((\neg A \to \neg A) \to \neg \neg A)$,
\item $(\neg A \to \neg A) \to \neg \neg A$,
\item $\neg A \to \neg A$,
\item $\neg \neg A$,
\end{enumerate}
\end{multicols}
so $A \vdash_i \neg \neg A$.  By the deduction theorem, $\vdash_i A \to \neg A \neg A$.
\end{proof}
In the proof above, we use the schema $B\to B$ in step 6 of the deduction, because $\vdash_i B\to B$, as a result of applying the deduction theorem to $B\vdash_i B$.

6. $\neg \neg \neg A \to \neg A$
\begin{proof} From the deduction
\begin{multicols}{2}
\begin{enumerate}
\item $(A \to  \neg \neg A)\to ((A \to \neg \neg \neg A) \to \neg A)$,
\item $A \to \neg \neg A$,
\item $(A \to \neg \neg \neg A) \to \neg A$,
\item $\neg \neg \neg A \to (A \to \neg \neg \neg A)$,
\item $\neg \neg \neg A$,
\item $A \to \neg \neg \neg A$,
\item $\neg A$, 
\end{enumerate}
\end{multicols} 
so $A \to \neg \neg A, \neg \neg \neg A \vdash_i \neg A$.  By the deduction theorem, $A\to \neg \neg A, \vdash_i \neg \neg \neg A \to \neg A$.  Since $\vdash_i A \to \neg A\neg A$, $\vdash_i \neg \neg \neg A \to \neg A$ as a result.
\end{proof}
In the above proof, we use the fact that if $\vdash_i C$ and $C\vdash_i D$, then $\vdash_i D$.  This is the result of the following fact: if $\vdash_i C$ and $\vdash_i C\to D$, then $\vdash_i D$.

7. $(A\to B) \to (\neg B \to \neg A)$
\begin{proof} From the deduction
\begin{multicols}{2}
\begin{enumerate}
\item $(A\to B) \to ((A\to \neg B) \to \neg A)$,
\item $A \to B$,
\item $(A\to \neg B) \to \neg A$,
\item $\neg B \to (A\to \neg B)$,
\item $\neg B$,
\item $A\to \neg B$,
\item $\neg A$,
\end{enumerate}
\end{multicols}
so $A\to B, \neg B \vdash_i \neg A$.  Applying the deduction theorem twice gives us $\vdash_i (A \to B)\to (
\neg B \to \neg A)$.
\end{proof}

8. $\neg (A \land \neg A)$
\begin{proof} From the deduction
\begin{enumerate}
\item $(A\land \neg A\to B) \to ((A\land \neg A\to \neg B)\to \neg (A\land \neg A))$,
\item $A\land \neg A\to B$,
\item $(A\land \neg A\to \neg B)\to \neg (A\land \neg A)$,
\item $A\land \neg A\to \neg B$,
\item $\neg (A\land \neg A)$.
\end{enumerate}
Since $\vdash_i A\land \neg A\to B$ and $\vdash_i A\land \neg A\to \neg B$ are instances of theorem schema 4 above, $\vdash_i \neg (A\land \neg A)$ as a result.
\end{proof}
This also shows that $\vdash_i B \to \neg (A\land \neg A)$, which is the result of applying modus ponens to $\neg (A \land \neg A)$ to $\neg (A \land \neg A) \to (B \to \neg (A\land \neg A))$.

9. $(B \to A\land \neg A)\to \neg B$
\begin{proof} From the deduction
\begin{enumerate}
\item $B \to A\land \neg A$,
\item $(B \to A\land \neg A)\to ((B \to \neg (A\land \neg A))\to \neg B)$,
\item $(B \to \neg (A\land \neg A))\to \neg B$,
\item $B \to \neg (A\land \neg A)$,
\item $\neg B$,
\end{enumerate}
so $B \to A\land \neg A \vdash_i \neg B$.  Applying the deduction theorem gives us $\vdash_i (B \to A\land \neg A)
 \to \neg B$.
\end{proof}

10. $\neg \neg (A \lor \neg A)$
\begin{proof} From the deduction
\begin{enumerate}
\item $A \to (A\lor \neg A)$,
\item $(A \to (A\lor \neg A)) \to (\neg (A \lor \neg A) \to \neg A)$,
\item $\neg (A \lor \neg A) \to \neg A$,
\item $\neg A \to (A \lor \neg A)$,
\item $(\neg A \to (A\lor \neg A)) \to (\neg (A \lor \neg A) \to \neg \neg A)$,
\item $\neg (A \lor \neg A) \to \neg \neg A$,
\item $(\neg (A \lor \neg A) \to \neg A) \to ((\neg (A \lor \neg A) \to \neg \neg A) \to \neg \neg (A \lor \neg A))$,
\item $(\neg (A \lor \neg A) \to \neg \neg A) \to \neg \neg (A \lor \neg A)$,
\item $\neg \neg (A \lor \neg A)$,
\end{enumerate}
\end{proof}

\textbf{Remark}.  Again from \PMlinkname{this entry}{AxiomSystemForIntuitionisticLogic}, if we use the second axiom system instead, keeping in mind that $\neg A$ means $A\to \perp$, the following are theorem schemas:

1. $(A \to B) \to ((A \to (B \to C)) \to (A \to C))$.  The proof of this is essentially the same as the proof of the third theorem schema above.

2. $(A \to B) \to ((A \to \neg B) \to \neg A)$.  This is just $(A \to B) \to ((A \to (B \to \perp)) \to (A \to \perp))$, an instance of the theorem schema above.

3. $\neg A \to  (A \to B)$.
\begin{proof} From the deduction
\begin{multicols}{3}
\begin{enumerate}
\item $\neg A$ (which is $A\to \perp$),
\item $A$,
\item $\perp$,
\item $\perp \to B$,
\item $B$
\end{enumerate}
\end{multicols}
so $\neg A, A \vdash_i B$.  By two applications of the deduction theorem, we get $\vdash_i \neg A \to (A \to B)$.
\end{proof}

%%%%%
%%%%%
\end{document}
