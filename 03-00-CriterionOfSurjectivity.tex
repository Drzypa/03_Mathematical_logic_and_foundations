\documentclass[12pt]{article}
\usepackage{pmmeta}
\pmcanonicalname{CriterionOfSurjectivity}
\pmcreated{2013-03-22 18:04:56}
\pmmodified{2013-03-22 18:04:56}
\pmowner{pahio}{2872}
\pmmodifier{pahio}{2872}
\pmtitle{criterion of surjectivity}
\pmrecord{4}{40619}
\pmprivacy{1}
\pmauthor{pahio}{2872}
\pmtype{Theorem}
\pmcomment{trigger rebuild}
\pmclassification{msc}{03-00}
\pmsynonym{surjectivity criterion}{CriterionOfSurjectivity}
%\pmkeywords{surjective}
%\pmkeywords{surjection}
\pmrelated{Function}
\pmrelated{Image}
\pmrelated{Subset}

% this is the default PlanetMath preamble.  as your knowledge
% of TeX increases, you will probably want to edit this, but
% it should be fine as is for beginners.

% almost certainly you want these
\usepackage{amssymb}
\usepackage{amsmath}
\usepackage{amsfonts}

% used for TeXing text within eps files
%\usepackage{psfrag}
% need this for including graphics (\includegraphics)
%\usepackage{graphicx}
% for neatly defining theorems and propositions
 \usepackage{amsthm}
% making logically defined graphics
%%%\usepackage{xypic}

% there are many more packages, add them here as you need them

% define commands here

\theoremstyle{definition}
\newtheorem*{thmplain}{Theorem}

\begin{document}
\textbf{Theorem.}\, For surjectivity of a mapping \, $f\!:\,A \to B$,\, it's necessary and sufficient that
\begin{align}
B\!\smallsetminus\!f(X) \,\subseteq\, f(A\!\smallsetminus\!X) \quad \forall\, X \subseteq A.
\end{align}


{\em Proof.}\; $1^{\underline{o}}$.\, Suppose that\, $f\!:\,A \to B$\, is surjective.\, Let $X$ be an arbitrary subset of $A$ and $y$ any element of the set $B\!\smallsetminus\!f(X)$.\, By the surjectivity, there is an $x$ in $A$ such that\, $f(x) = y$, and since\, $y \notin f(X)$,\, the element $x$ is not in $X$, i.e.\, $x \in A\!\smallsetminus\!X$\, and thus\, $y = f(x) \in f(A\!\smallsetminus\!X)$.\, One can conclude that\, $B\!\smallsetminus\!f(X) \,\subseteq\, f(A\!\smallsetminus\!X)$\, for all\, $X \subseteq A$.

$2^{\underline{o}}$.\, Conversely, suppose the condition (1).\, Let again $X$ be an arbitrary subset of $A$ and $y$ any element of $B$.\, We have two possibilities:\\
a) $y \notin f(X)$; then\, $y \in B\!\smallsetminus\!f(X)$, and by (1), $y \in f(A\!\smallsetminus\!X)$.\, This means that there exists an element $x$ of\, $A\!\smallsetminus\!X \subseteq A$\, such that\, $f(x) = y$.\\
b) $y \in f(X)$; then there exists an $x \in X \subseteq A$\, such that\, $f(x) = y$.\\
The both cases show the surjectivity of $f$.
%%%%%
%%%%%
\end{document}
