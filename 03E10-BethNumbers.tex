\documentclass[12pt]{article}
\usepackage{pmmeta}
\pmcanonicalname{BethNumbers}
\pmcreated{2013-03-22 14:17:09}
\pmmodified{2013-03-22 14:17:09}
\pmowner{yark}{2760}
\pmmodifier{yark}{2760}
\pmtitle{beth numbers}
\pmrecord{9}{35740}
\pmprivacy{1}
\pmauthor{yark}{2760}
\pmtype{Definition}
\pmcomment{trigger rebuild}
\pmclassification{msc}{03E10}
\pmrelated{AlephNumbers}
\pmrelated{GeneralizedContinuumHypothesis}

\endmetadata

\usepackage{amssymb}
\usepackage{amsmath}
\usepackage{amsfonts}

%\usepackage{psfrag}
%\usepackage{graphicx}
%\usepackage{amsthm}
%%%\usepackage{xypic}

\renewcommand{\le}{\leqslant}
\renewcommand{\ge}{\geqslant}
\renewcommand{\leq}{\leqslant}
\renewcommand{\geq}{\geqslant}
\begin{document}
\PMlinkescapeword{alphabet}
\PMlinkescapeword{equivalent}
\PMlinkescapeword{similar}

The \emph{beth numbers} are infinite cardinal numbers 
defined in a similar manner to the aleph numbers, as described below.
They are written $\beth_\alpha$, where $\beth$ is beth,
the second letter of the Hebrew alphabet,
and $\alpha$ is an ordinal number.

We define $\beth_0$ to be the first infinite cardinal (that is, $\aleph_0$).
For each ordinal $\alpha$,
we define $\beth_{\alpha+1}=2^{\beth_\alpha}$.
For each limit ordinal $\delta$, 
we define $\beth_\delta=\bigcup_{\alpha\in\delta}\beth_\alpha$.

Note that $\beth_1$ is the cardinality of the continuum.

For any ordinal $\alpha$ the inequality $\aleph_\alpha\leq\beth_\alpha$ holds.
The Generalized Continuum Hypothesis is equivalent to the assertion that
$\aleph_\alpha=\beth_\alpha$ for every ordinal $\alpha$.

For every limit ordinal $\delta$, 
the cardinal $\beth_\delta$ is a strong limit cardinal.
Every uncountable strong limit cardinal arises in this way.
%%%%%
%%%%%
\end{document}
