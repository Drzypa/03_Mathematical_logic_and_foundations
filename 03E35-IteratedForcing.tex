\documentclass[12pt]{article}
\usepackage{pmmeta}
\pmcanonicalname{IteratedForcing}
\pmcreated{2013-03-22 12:54:47}
\pmmodified{2013-03-22 12:54:47}
\pmowner{Henry}{455}
\pmmodifier{Henry}{455}
\pmtitle{iterated forcing}
\pmrecord{5}{33264}
\pmprivacy{1}
\pmauthor{Henry}{455}
\pmtype{Definition}
\pmcomment{trigger rebuild}
\pmclassification{msc}{03E35}
\pmclassification{msc}{03E40}
\pmdefines{FS}
\pmdefines{CS}
\pmdefines{finite support}
\pmdefines{finite support iterated forcing}
\pmdefines{countable support}
\pmdefines{countable support iterated forcing}
\pmdefines{support iterated forcing}

\endmetadata

% this is the default PlanetMath preamble.  as your knowledge
% of TeX increases, you will probably want to edit this, but
% it should be fine as is for beginners.

% almost certainly you want these
\usepackage{amssymb}
\usepackage{amsmath}
\usepackage{amsfonts}

% used for TeXing text within eps files
%\usepackage{psfrag}
% need this for including graphics (\includegraphics)
%\usepackage{graphicx}
% for neatly defining theorems and propositions
%\usepackage{amsthm}
% making logically defined graphics
%%%\usepackage{xypic}

% there are many more packages, add them here as you need them

% define commands here
%\PMlinkescapeword{theory}
\begin{document}
We can define an \emph{iterated forcing} of length $\alpha$ by induction as follows:

Let $P_0=\emptyset$.

Let $\hat{Q}_0$ be a forcing notion.

For $\beta\leq\alpha$, $P_\beta$ is the set of all functions $f$ such that $\operatorname{dom}(f)\subseteq\beta$ and for any $i\in\operatorname{dom}(f)$, $f(i)$ is a $P_i$-name for a member of $\hat{Q}_i$.  Order $P_\beta$ by the rule $f\leq g$ iff $\operatorname{dom}(g)\subseteq\operatorname{dom}(f)$ and for any $i\in\operatorname{dom}(f)$, $g\upharpoonright i\Vdash f(i)\leq_{\hat{Q}_i}g(i)$.  (Translated, this means that any generic subset including $g$ restricted to $i$ forces that $f(i)$, an element of $\hat{Q}_i$, be less than $g(i)$.)

For $\beta<\alpha$, $\hat{Q}_\beta$ is a forcing notion in $P_\beta$ (so $\Vdash_{P_\beta} \hat{Q}_\beta$\texttt{ is a forcing notion}).

Then the sequence $\langle \hat{Q}_\beta\rangle_{\beta<\alpha}$ is an iterated forcing.

If $P_\beta$ is restricted to finite functions that it is called a \emph{finite support iterated forcing} (FS), if $P_\beta$ is restricted to countable functions, it is called a \emph{countable support iterated function} (CS), and in general if each function in each $P_\beta$ has size less than $\kappa$ then it is a \emph{$<\kappa$-support iterated forcing}.

Typically we construct the sequence of $\hat{Q}_\beta$'s by induction, using a function $F$ such that $F(\langle \hat{Q}_\beta\rangle_{\beta<\gamma})=\hat{Q}_\gamma$.
%%%%%
%%%%%
\end{document}
