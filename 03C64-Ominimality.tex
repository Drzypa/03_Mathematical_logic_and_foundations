\documentclass[12pt]{article}
\usepackage{pmmeta}
\pmcanonicalname{Ominimality}
\pmcreated{2013-03-22 13:23:01}
\pmmodified{2013-03-22 13:23:01}
\pmowner{Timmy}{1414}
\pmmodifier{Timmy}{1414}
\pmtitle{o-minimality}
\pmrecord{7}{33919}
\pmprivacy{1}
\pmauthor{Timmy}{1414}
\pmtype{Definition}
\pmcomment{trigger rebuild}
\pmclassification{msc}{03C64}
\pmclassification{msc}{14P10}
%\pmkeywords{definable set}
%\pmkeywords{semialgebraic}
%\pmkeywords{real algebraic}
%\pmkeywords{tame topology}
\pmrelated{StronglyMinimal}
\pmdefines{o-minimal}

% this is the default PlanetMath preamble.  as your knowledge
% of TeX increases, you will probably want to edit this, but
% it should be fine as is for beginners.

% almost certainly you want these
\usepackage{amssymb}
\usepackage{amsmath}
\usepackage{amsfonts}

% used for TeXing text within eps files
%\usepackage{psfrag}
% need this for including graphics (\includegraphics)
%\usepackage{graphicx}
% for neatly defining theorems and propositions
%\usepackage{amsthm}
% making logically defined graphics
%%%\usepackage{xypic}

% there are many more packages, add them here as you need them

% define commands here
\begin{document}
Let $M$ be an ordered structure. An interval in $M$ is any subset of $M$ that can be expressed in one of the following forms:
 \begin{itemize} \item $\{x:a<x<b\}$ for some $a,b$ from $M$
\item $\{x:x>a\}$ for some $a$ from $M$
\item $\{x:x<a\}$ for some $a$ from $M$
\end{itemize}

Then we define $M$ to be {\em o-minimal} iff every definable subset of $M$ is a finite union of intervals and points. This is a property of the theory of $M$ i.e. if $M \equiv N$ and $M$ is o-minimal, then $N$ is o-minimal. 
Note that $M$ being o-minimal is equivalent to every definable subset of $M$ being quantifier free definable in the language with just the ordering. Compare this with strong minimality.

\medskip

The model theory of o-minimal structures is well understood, for an excellent account see Lou van den Dries, Tame topology and o-minimal structures, CUP 1998.
In particular, although this condition is merely on definable subsets of $M$ it gives very good information about definable subsets of $M^{n}$ for $n \in \omega$.
%%%%%
%%%%%
\end{document}
