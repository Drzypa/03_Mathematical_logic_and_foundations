\documentclass[12pt]{article}
\usepackage{pmmeta}
\pmcanonicalname{AxiomOfExtensionality}
\pmcreated{2013-03-22 13:42:40}
\pmmodified{2013-03-22 13:42:40}
\pmowner{Sabean}{2546}
\pmmodifier{Sabean}{2546}
\pmtitle{axiom of extensionality}
\pmrecord{5}{34391}
\pmprivacy{1}
\pmauthor{Sabean}{2546}
\pmtype{Axiom}
\pmcomment{trigger rebuild}
\pmclassification{msc}{03E30}
\pmsynonym{extensionality}{AxiomOfExtensionality}

\endmetadata

% this is the default PlanetMath preamble.  as your knowledge
% of TeX increases, you will probably want to edit this, but
% it should be fine as is for beginners.

% almost certainly you want these
\usepackage{amssymb}
\usepackage{amsmath}
\usepackage{amsfonts}

% used for TeXing text within eps files
%\usepackage{psfrag}
% need this for including graphics (\includegraphics)
%\usepackage{graphicx}
% for neatly defining theorems and propositions
%\usepackage{amsthm}
% making logically defined graphics
%%%\usepackage{xypic}

% there are many more packages, add them here as you need them

% define commands here
\begin{document}
If $X$ and $Y$ have the same elements, then $X = Y$.

The Axiom of Extensionality is one of the axioms of Zermelo-Fraenkel set theory.
In symbols, it reads:
\[
\forall u(u \in X \leftrightarrow u \in Y) \rightarrow X = Y.
\]
Note that the converse,
\[
X = Y \rightarrow \forall u(u \in X \leftrightarrow u \in Y)
\]
is an axiom of the predicate calculus.  Hence we have,
\[
X = Y \leftrightarrow \forall u(u \in X \leftrightarrow u \in Y).
\]
Therefore the Axiom of Extensionality expresses the most fundamental notion of a set: a set is determined by its elements.
%%%%%
%%%%%
\end{document}
