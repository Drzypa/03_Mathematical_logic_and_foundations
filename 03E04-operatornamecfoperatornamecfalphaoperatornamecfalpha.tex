\documentclass[12pt]{article}
\usepackage{pmmeta}
\pmcanonicalname{operatornamecfoperatornamecfalphaoperatornamecfalpha}
\pmcreated{2013-03-22 18:11:22}
\pmmodified{2013-03-22 18:11:22}
\pmowner{yesitis}{13730}
\pmmodifier{yesitis}{13730}
\pmtitle{$\operatorname{cf}(\operatorname{cf} \alpha) = \operatorname{cf} \alpha$}
\pmrecord{8}{40764}
\pmprivacy{1}
\pmauthor{yesitis}{13730}
\pmtype{Proof}
\pmcomment{trigger rebuild}
\pmclassification{msc}{03E04}

\endmetadata

% this is the default PlanetMath preamble.  as your knowledge
% of TeX increases, you will probably want to edit this, but
% it should be fine as is for beginners.

% almost certainly you want these
\usepackage{amssymb}
\usepackage{amsmath}
\usepackage{amsfonts}

% used for TeXing text within eps files
%\usepackage{psfrag}
% need this for including graphics (\includegraphics)
%\usepackage{graphicx}
% for neatly defining theorems and propositions
%\usepackage{amsthm}
% making logically defined graphics
%%%\usepackage{xypic}

% there are many more packages, add them here as you need them

% define commands here

\begin{document}
Let \\

$\operatorname{cf} \alpha=\beta$ and $\langle\alpha_\xi:\xi<\beta\rangle$ be cofinal in $\alpha$, and \\

$\operatorname{cf} \beta=\gamma$ and $\langle\xi(\nu):\nu<\gamma\rangle$ be cofinal in $\beta$. \\

The claim of the theorem $\operatorname{cf}(\operatorname{cf} \alpha) = \operatorname{cf} \alpha$ means that $\gamma=\beta$; we prove this fact.

Suppose $\gamma\neq\beta$. Then $\gamma<\beta$ by $\operatorname{cf}\delta\leq\delta$.

Now, $\langle\alpha_{\xi(\nu)}:\nu<\gamma\rangle$ is seen to be confinal in $\alpha$, which means that $\operatorname{cf}\alpha=\gamma<\beta$, a contradiction. Therefore, $\gamma=\beta$.
%%%%%
%%%%%
\end{document}
