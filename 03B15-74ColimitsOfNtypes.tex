\documentclass[12pt]{article}
\usepackage{pmmeta}
\pmcanonicalname{74ColimitsOfNtypes}
\pmcreated{2013-11-20 21:53:39}
\pmmodified{2013-11-20 21:53:39}
\pmowner{PMBookProject}{1000683}
\pmmodifier{rspuzio}{6075}
\pmtitle{7.4 Colimits of n-types}
\pmrecord{8}{87701}
\pmprivacy{1}
\pmauthor{PMBookProject}{6075}
\pmtype{Feature}
\pmclassification{msc}{03B15}

\endmetadata

\usepackage{xspace}
\usepackage{amssyb}
\usepackage{amsmath}
\usepackage{amsfonts}
\usepackage{amsthm}
\makeatletter
\newcommand{\blank}{\mathord{\hspace{1pt}\text{--}\hspace{1pt}}}
\newcommand{\cocone}[2]{\mathsf{cocone}_{#1}(#2)}
\newcommand{\composecocone}[2]{#1\circ#2}
\newcommand{\Ddiag}{\mathscr{D}}
\newcommand{\defeq}{\vcentcolon\equiv}  
\newcommand{\define}[1]{\textbf{#1}}
\def\@dprd#1{\prod_{(#1)}\,}
\def\@dprd@noparens#1{\prod_{#1}\,}
\def\@dsm#1{\sum_{(#1)}\,}
\def\@dsm@noparens#1{\sum_{#1}\,}
\def\@eatprd\prd{\prd@parens}
\def\@eatsm\sm{\sm@parens}
\newcommand{\function}[4]{\left\{\begin{array}{rcl}#1 &      \longrightarrow & #2 \\ #3 & \longmapsto & #4 \end{array}\right.}
\newcommand{\funext}{\mathsf{funext}}
\newcommand{\happly}{\mathsf{happly}}
\newcommand{\htpy}{\sim}
\newcommand{\idfunc}[1][]{\ensuremath{\mathsf{id}_{#1}}\xspace}
\newcommand{\indexdef}[1]{\index{#1|defstyle}}   
\def\lam#1{{\lambda}\@lamarg#1:\@endlamarg\@ifnextchar\bgroup{.\,\lam}{.\,}}
\def\@lamarg#1:#2\@endlamarg{\if\relax\detokenize{#2}\relax #1\else\@lamvar{\@lameatcolon#2},#1\@endlamvar\fi}
\def\@lameatcolon#1:{#1}
\def\@lamvar#1,#2\@endlamvar{(#2\,{:}\,#1)}
\newcommand{\mapfunc}[1]{\ensuremath{\mathsf{ap}_{#1}}\xspace} 
\newcommand{\modal}{\ensuremath{\ocircle}}
\def\prd#1{\@ifnextchar\bgroup{\prd@parens{#1}}{\@ifnextchar\sm{\prd@parens{#1}\@eatsm}{\prd@noparens{#1}}}}
\def\prd@noparens#1{\mathchoice{\@dprd@noparens{#1}}{\@tprd{#1}}{\@tprd{#1}}{\@tprd{#1}}}
\def\prd@parens#1{\@ifnextchar\bgroup  {\mathchoice{\@dprd{#1}}{\@tprd{#1}}{\@tprd{#1}}{\@tprd{#1}}\prd@parens}  {\@ifnextchar\sm    {\mathchoice{\@dprd{#1}}{\@tprd{#1}}{\@tprd{#1}}{\@tprd{#1}}\@eatsm}    {\mathchoice{\@dprd{#1}}{\@tprd{#1}}{\@tprd{#1}}{\@tprd{#1}}}}}
\def\sm#1{\@ifnextchar\bgroup{\sm@parens{#1}}{\@ifnextchar\prd{\sm@parens{#1}\@eatprd}{\sm@noparens{#1}}}}
\def\sm@noparens#1{\mathchoice{\@dsm@noparens{#1}}{\@tsm{#1}}{\@tsm{#1}}{\@tsm{#1}}}
\def\sm@parens#1{\@ifnextchar\bgroup  {\mathchoice{\@dsm{#1}}{\@tsm{#1}}{\@tsm{#1}}{\@tsm{#1}}\sm@parens}  {\@ifnextchar\prd    {\mathchoice{\@dsm{#1}}{\@tsm{#1}}{\@tsm{#1}}{\@tsm{#1}}\@eatprd}    {\mathchoice{\@dsm{#1}}{\@tsm{#1}}{\@tsm{#1}}{\@tsm{#1}}}}}
\def\@tprd#1{\mathchoice{{\textstyle\prod_{(#1)}}}{\prod_{(#1)}}{\prod_{(#1)}}{\prod_{(#1)}}}
\newcommand{\tproj}[3][]{\mathopen{}\left|#3\right|_{#2}^{#1}\mathclose{}}
\newcommand{\tprojf}[2][]{|\blank|_{#2}^{#1}}
\newcommand{\trunc}[2]{\mathopen{}\left\Vert #2\right\Vert_{#1}\mathclose{}}
\def\@tsm#1{\mathchoice{{\textstyle\sum_{(#1)}}}{\sum_{(#1)}}{\sum_{(#1)}}{\sum_{(#1)}}}
\newcommand{\UU}{\ensuremath{\mathcal{U}}\xspace}
\newcommand{\vcentcolon}{:\!\!}
\newcounter{mathcount}
\setcounter{mathcount}{1}
\newtheorem{precor}{Corollary}
\newenvironment{cor}{\begin{precor}}{\end{precor}\addtocounter{mathcount}{1}}
\renewcommand{\theprecor}{7.4.\arabic{mathcount}}
\newtheorem{predefn}{Definition}
\newenvironment{defn}{\begin{predefn}}{\end{predefn}\addtocounter{mathcount}{1}}
\renewcommand{\thepredefn}{7.4.\arabic{mathcount}}
\newenvironment{myeqn}{\begin{equation}}{\end{equation}\addtocounter{mathcount}{1}}
\renewcommand{\theequation}{7.4.\arabic{mathcount}}
\newtheorem{prelem}{Lemma}
\newenvironment{lem}{\begin{prelem}}{\end{prelem}\addtocounter{mathcount}{1}}
\renewcommand{\theprelem}{7.4.\arabic{mathcount}}
\newtheorem{prethm}{Theorem}
\newenvironment{thm}{\begin{prethm}}{\end{prethm}\addtocounter{mathcount}{1}}
\renewcommand{\theprethm}{7.4.\arabic{mathcount}}
\let\apfunc\mapfunc
\let\autoref\cref
\let\reflect\modal
\let\type\UU
\makeatother

\begin{document}
Recall that in \PMlinkname{\S 6.8}{68pushouts}, we used higher inductive types to define pushouts of types, and proved their universal property.
In general, a (homotopy) colimit of $n$-types may no longer be an $n$-type (for an extreme counterexample, see \PMlinkexternal{Exercise 7.2}{http://planetmath.org/node/87811}).
However, if we $n$-truncate it, we obtain an $n$-type which satisfies the correct universal property with respect to other $n$-types.

In this section we prove this for pushouts, which are the most important and nontrivial case of colimits.
Recall the following definitions from \PMlinkname{\S 6.8}{68pushouts}.

\begin{defn}
  A \define{span} % in $\P$
  \indexdef{span} %
  is a 5-tuple $\Ddiag=(A,B,C,f,g)$ with % $A,B,C:\P$ and
  $f:C\to{}A$ and $g:C\to{}B$.
 \begin{figure}
 \centering
 \includegraphics{HoTT_fig_7.4.1.png}
\end{figure}
% \[\Ddiag=\quad\vcenter{\xymatrix{C \ar^g[r] \ar_f[d] & B \\ A & }}\]
\end{defn}

\begin{defn}
  Given a span $\Ddiag=(A,B,C,f,g)$ and a type $D$, a %$D:\P$, a
  \define{cocone under $\Ddiag$ with base $D$} is a triple $(i, j, h)$
  \index{cocone} %
  with $i:A\to{}D$, $j:B\to{}D$ and $h : \prd{c:C}i(f(c))=j(g(c))$:
\begin{figure}
 \centering
 \includegraphics{HoTT_fig_7.4.2.png}
\end{figure}
%  \[\uppercurveobject{{ }}\lowercurveobject{{ }}\twocellhead{{ }}
%  \xymatrix{C \ar^g[r] \ar_f[d] \drtwocell{^h} & B \ar^j[d] \\ A \ar_i[r] & D
%  }\]
  We denote by $\cocone{\Ddiag}{D}$ the type of all such cocones.
\end{defn}

The type of cocones is (covariantly) functorial.
For instance, given $D,E$ % $D,E:\P$
and a map $t:D\to{}E$, there is a map
  \[\function{\cocone{\Ddiag}{D}}{\cocone{\Ddiag}{E}}{c}{\composecocone{t}c}\]
  defined by:
  \[\composecocone{t}(i,j,h)=(t\circ{}i,t\circ{}j,\mapfunc{t}\circ{}h)\]
And given $D,E,F$, %$:\P$,
functions $t:D\to{}E$, $u:E\to{}F$ and $c:\cocone{\Ddiag}{D}$, we have
\begin{align}
  \composecocone{\idfunc[D]}c &= c \label{eq:composeconeid}\\
  \addtocounter{mathcount}{1}
  \composecocone{(u\circ{}t)}c&=\composecocone{u}(\composecocone{t}c). \label{eq:composeconefunc}
\end{align}
\addtocounter{mathcount}{1}

\begin{defn}
  Given a span $\Ddiag$ of $n$-types, an $n$-type $D$, and a cocone
  $c:\cocone{\Ddiag}{D}$, the pair $(D,c)$ is said to be a \define{pushout
  of $\Ddiag$ in $n$-types}
  \indexdef{pushout!in ntypes@in $n$-types}%
  if for every $n$-type $E$, the map
  \[\function{(D\to{}E)}{\cocone{\Ddiag}{E}}{t}{\composecocone{t}c}\]
  is an equivalence.
\end{defn}

%\begin{comment}
%We showed in \PMlinkname{Lemma 6.8.2}{68pushouts#Thmprelem1} that pushouts exist when $\P$ is \type itself, by giving a direct construction in terms of higher
%inductive types.
%For a general \P, pushouts may or may not exist, but if they do, then they are unique.
%
%\begin{lem}
%  If $(D,c)$ and $(D',c')$ are two pushouts of $\Ddiag$ in $\P$, then
%  $(D,c)=(D',c')$.
%\end{lem}
%\begin{proof}
%  We first prove that the two types $D$ and $D'$ are equivalent.
%
%  Using the universal property of $D$ with $D'$, we see that the following map is an
%  equivalence
  %
%  \[
%    \function{(D\to{}D')}{\cocone{\Ddiag}{D'}}{t}{\composecocone{t}c}
%  \]
  %
%  In particular, there is a function $f:D\to{}D'$ satisfying $\composecocone{f}c=c'$. In the
%  same way there is a function $g:D'\to{}D$ such that $\composecocone{g}c'=c$.
%
%  In order to prove that $g\circ{}f=\idfunc[D]$ we use the universal property of
%  $D$ for $D$, which says that the following map is an equivalence:
  %
%  \[
%  \function{(D\to{}D)}{\cocone{\Ddiag}{D}}{t}{\composecocone{t}c}
%  \]
  %
%  Using the functoriality of $t\mapsto{}\composecocone{t}c$ we see that
%  \begin{align*}
%    \composecocone{(g\circ{}f)}c &= \composecocone{g}(\composecocone{f}c) \\
%    &= \composecocone{g}c' \\
%    &= c \\
%    &= \composecocone{\idfunc[D]}c
%  \end{align*}
%  hence
%  $g\circ{}f=\idfunc[D]$, because equivalences are injective. The same argument
%  with $D'$ instead of $D$ shows that $f\circ{}g=\idfunc[D']$.
%
%  Hence $D$ and $D'$ are equal, and the fact that $(D,c)=(D',c')$ follows from
%  the fact that the equivalence between $D$ and $D'$ we just defined sends $c$
%  to $c'$.
%\end{proof}
%
%\begin{cor}
%  The type of pushouts of $\Ddiag$ in $\P$ is a mere proposition. In particular if
%  pushouts merely exist then they actually exist.
%\end{cor}
%
%As in the case of pullbacks, if \P is reflective, then pushouts in \P always exist.
%However, unlike the case of pullbacks, pushouts in \P are not the same as the pushouts in \type: they are obtained by applying the
%reflector.
%\end{comment}

In order to construct pushouts of $n$-types, we need to explain how to reflect spans and cocones.

\bgroup
\def\reflect(#1){\trunc n{#1}}

\begin{defn}
  Let
  \begin{figure}
 \centering
 \includegraphics{HoTT_fig_7.4.3.png}
\end{figure}
%  \[\Ddiag=\quad\vcenter{\xymatrix{C \ar^g[r] \ar_f[d] & B \\ A & }}\]
  be a span. We denote by $\reflect(\Ddiag)$ the following
  span of $n$-types:
\begin{figure}
 \centering
 \includegraphics{HoTT_fig_7.4.4.png}
\end{figure}
%\[\reflect(\Ddiag)\defeq\quad \vcenter{\xymatrix{\reflect(C) \ar^{\reflect(g)}[r]
%      \ar_{\reflect(f)}[d] & \reflect(B) \\ \reflect(A) & }}\]
\end{defn}

\begin{defn}
  Let $D:\type$ and $c=(i,j,h):\cocone{\Ddiag}{D}$.
  We define
  \[\reflect(c)=(\reflect(i),\reflect(j),\reflect(h)):
  \cocone{\reflect(\Ddiag)}{\reflect(D)}\]
  where $\reflect(h): \reflect(i) \circ \reflect(f) \htpy \reflect(j) \circ \reflect(g)$ is defined as in \PMlinkname{Lemma 7.3.5}{73truncations#Thmprelem3}.
  % \[\reflect(h):\prd{c:\reflect(C)}\reflect(i)(\reflect(f)(c))=\reflect(j)(\reflect(g)(c))\]
  % is defined in the following way:
\end{defn}

\egroup

We now observe that the maps from each type to its $n$-truncation assemble into a map of spans, in the following sense.

\begin{defn}
  Let 
  \[
  \includegraphics{HoTT_fig_7.4.5.png}
  %\Ddiag=\quad\vcenter{\xymatrix{C \ar^g[r] \ar_f[d] & B \\ A & }}
  \qquad\text{and}\qquad
  \includegraphics{HoTT_fig_7.4.6.png}
  %\Ddiag'=\quad\vcenter{\xymatrix{C' \ar^{g'}[r] \ar_{f'}[d] & B' \\ A' & }}
  \]
  be spans.
  A \define{map of spans}
  \indexdef{map!of spans}%
  $\Ddiag \to \Ddiag'$ consists of functions $\alpha:A\to A'$, $\beta:B\to B'$, and $\gamma:C\to C'$ and homotopies $\phi: \alpha\circ f \htpy f'\circ \gamma$ and $\psi:\beta\circ g \htpy g' \circ \gamma$.
\end{defn}

Thus, for any span $\Ddiag$, we have a map of spans $\tprojf[\Ddiag] n : \Ddiag \to \trunc n\Ddiag$ consisting of $\tprojf[A]n$, $\tprojf[B]n$, $\tprojf[C]n$, and the naturality homotopies $\mathsf{nat}^f_n$ and $\mathsf{nat}^g_n$ from~\PMlinkname{(7.3.4)}{73truncations#S0.E1}.

We also need to know that maps of spans behave functorially.
Namely, if $(\alpha,\beta,\gamma,\phi,\psi):\Ddiag \to \Ddiag'$ is a map of spans and $D$ any type, then we have
\[ \function{\cocone{\Ddiag'}{D}}{\cocone{\Ddiag}{D}}{(i,j,h)}{(i\circ \alpha,j\circ\beta, k)} \]
where $k: \prd{z:C} i(\alpha(f(z))) = j(\beta(g(z)))$ is the composite
\begin{myeqn}\label{eq:mapofspans-htpy}
\includegraphics{HoTT_fig_7.4.7.png}
%\xymatrix{
%  i(\alpha(f(z))) \ar@{=}[r]^{\apfunc{i}(\phi)} &
%  i(f'(\gamma(z))) \ar@{=}[r]^{h(\gamma(z))} &
%  j(g'(\gamma(z))) \ar@{=}[r]^{\apfunc{j}(\psi)} &
%  j(\beta(g(z))). }
\end{myeqn}
We denote this cocone by $(i,j,h) \circ (\alpha,\beta,\gamma,\phi,\psi)$.
Moreover, this functorial action commutes with the other functoriality of cocones:

\begin{lem}\label{thm:conemap-funct}
  Given $(\alpha,\beta,\gamma,\phi,\psi):\Ddiag \to \Ddiag'$ and $t:D\to E$, the following diagram commutes:
\begin{figure}
 \centering
 \includegraphics{HoTT_fig_7.4.8.png}
\end{figure}
%  \begin{equation*}
%    \vcenter{\xymatrix{
%        \cocone{\Ddiag'}{D}\ar[r]^{t \circ {\blank}}\ar[d] &
%        \cocone{\Ddiag'}{E}\ar[d]\\
%        \cocone{\Ddiag}{D}\ar[r]_{t \circ {\blank}} &
%        \cocone{\Ddiag}{E}
%      }}
%  \end{equation*}
\end{lem}
\begin{proof}
  Given $(i,j,h):\cocone{\Ddiag'}{D}$, note that both composites yield a cocone whose first two components are $t\circ i\circ \alpha$ and $t\circ j\circ\beta$.
  Thus, it remains to verify that the homotopies agree.
  For the top-right composite, the homotopy is~\eqref{eq:mapofspans-htpy} with $(i,j,h)$ replaced by $(t\circ i, t\circ j, \apfunc{t}\circ h)$:
\begin{figure}
 \centering
 \includegraphics{HoTT_fig_7.4.9.png}
\end{figure}
%\begin{equation*}
%    \xymatrix@+2.8em{
%      {t \, i \, \alpha \, f \, z} \ar@{=}[r]^{\apfunc{t\circ i}(\phi)} &
%      {t \, i \, f' \, \gamma \, z} \ar@{=}[r]^{\apfunc{t}(h(\gamma(z)))} &
%      {t \, j \, g' \, \gamma \, z} \ar@{=}[r]^{\apfunc{t\circ j}(\psi)} &
%      {t \, j \, \beta \, g \, z}
%    }
%  \end{equation*}
  (For brevity, we are omitting the parentheses around the arguments of functions.)
  On the other hand, for the left-bottom composite, the homotopy is $\apfunc{t}$ applied to~\eqref{eq:mapofspans-htpy}.
  Since $\apfunc{}$ respects path-concatenation, this is equal to
\begin{figure}
 \centering
 \includegraphics{HoTT_fig_7.4.10.png}
\end{figure}
%\begin{equation*}
%    \xymatrix@+2.8em{
%      {t \, i \, \alpha \, f \, z} \ar@{=}[r]^{\apfunc{t}(\apfunc{i}(\phi))} &
%      {t \, i \, f' \, \gamma \, z} \ar@{=}[r]^{\apfunc{t}(h(\gamma(z)))} &
%      {t \, j \, g' \, \gamma \, z} \ar@{=}[r]^{\apfunc{t}(\apfunc{j}(\psi))} &
%      {t \, j \, \beta \, g \, z}. }
%  \end{equation*}
  But $\apfunc{t}\circ \apfunc{i} = \apfunc{t\circ i}$ and similarly for $j$, so these two homotopies are equal.
\end{proof}

Finally, note that since we defined $\trunc nc : \cocone{\trunc n \Ddiag}{\trunc n D}$ using \PMlinkname{Lemma 7.3.5}{73truncations#Thmprelem3}, the additional condition~\PMlinkname{(7.3.6)}{73truncations#S0.E2} implies
\begin{myeqn}
  \tprojf[D] n \circ c = \trunc n c \circ \tprojf[\Ddiag]n. \label{eq:conetrunc}
\end{myeqn}
for any $c:\cocone{\Ddiag}{D}$.
Now we can prove our desired theorem.

\begin{thm}
  \label{reflectcommutespushout}
  \index{universal!property!of pushout}%
  Let $\Ddiag$ be a span and $(D,c)$ its pushout.
  Then $(\trunc nD,\trunc n c)$ is a pushout of $\trunc n\Ddiag$ in $n$-types.
\end{thm}
\begin{proof}
  Let $E$ be an $n$-type, and consider the following diagram:
\bgroup
\def\reflect(#1){\trunc n{#1}}
\begin{figure}
 \centering
 \includegraphics{HoTT_fig_7.4.11.png}
\end{figure}
%  \begin{equation*}
%  \vcenter{\xymatrix{
%      (\trunc nD \to E)\ar[r]^-{\blank\circ \tprojf[D] n}\ar[d]_{\blank\circ \trunc nc} &
%      (D\to E)\ar[d]^{\blank\circ c}\\
%      \cocone{\trunc n \Ddiag}{E}\ar[r]^-{\blank\circ \tprojf[\Ddiag]n}\ar@{<-}[d]_{\ell_1} &
%      \cocone{\Ddiag}{E}\ar@{<-}[d]^{\ell_2}\\
%      (\reflect(A)\to{}E)\times_{(\reflect(C)\to{}E)}(\reflect(B)\to{}E)\ar[r] &
%      (A\to{}E)\times_{(C\to{}E)}(B\to{}E)
%      }}
%  \end{equation*}
\egroup
  The upper horizontal arrow is an equivalence since $E$ is an $n$-type, while $\blank\circ c$ is an equivalence since $c$ is a pushout cocone.
  Thus, by the 2-out-of-3 property, to show that $\blank\circ \trunc nc$ is an equivalence, it will suffice to show that the upper square commutes and that the middle horizontal arrow is an equivalence.
  To see that the upper square commutes, let $t:\trunc nD \to E$; then
  \begin{align}
    \big(t \circ \trunc n c\big) \circ \tprojf[\Ddiag] n
    &= t \circ \big(\trunc n c \circ \tprojf[\Ddiag] n\big)
    \tag{by \PMlinkname{Lemma 7.4.10}{74colimitsofntypes#Thmprelem2}}\\
    &= t\circ \big(\tprojf[D]n \circ c\big)
    \tag{by~\eqref{eq:conetrunc}}\\
    &= \big(t\circ \tprojf[D]n\big) \circ c
    \tag{by~\eqref{eq:composeconefunc}}.
  \end{align}
  To show that the middle horizontal arrow is an equivalence, consider the lower square.
  The two lower vertical arrows are simply applications of $\happly$:
  \begin{align*}
    \ell_1(i,j,p) &\defeq (i,j,\happly(p))\\
    \ell_2(i,j,p) &\defeq (i,j,\happly(p))
  \end{align*}
  and hence are equivalences by function extensionality.
  The lowest horizontal arrow is defined by
  \[ (i,j,p) \mapsto \big( i\circ \tprojf[A]n,\;\; j \circ \tprojf[B] n,\;\; q\big) \]
  where $q$ is the composite
  \begin{align}
    i\circ \tprojf[A]n \circ f
    &= i\circ \trunc nf \circ \tprojf[C]n
    \tag{by $\funext(\lam{z} \apfunc{i}(\mathsf{nat}^f_n(z)))$}\\
    &= j\circ \trunc ng \circ \tprojf[C]n
    \tag{by $\apfunc{\blank\circ \tprojf[C] n}(p)$}\\
    &= j\circ \tprojf[B]n \circ g.
    \tag{by $\funext(\lam{z} \apfunc{j}(\mathsf{nat}^g_n(z)))$}
  \end{align}
  This is an equivalence, because it is induced by an equivalence of cospans.
  Thus, by 2-out-of-3, it will suffice to show that the lower square commutes.
  But the two composites around the lower square agree definitionally on the first two components, so it suffices to show that for $(i,j,p)$ in the lower left corner and $z:C$, the path
  \[ \happly(q,z) : i(\tproj n{f(z)}) = j(\tproj n{g(z)}) \]
  (with $q$ as above)
  is equal to the composite
  \begin{align}
    i(\tproj n{f(z)})
    &= i(\trunc nf(\tproj nz))
    \tag{by $\apfunc{i}(\mathsf{nat}^f_n(z))$}\\
    &= j(\trunc ng(\tproj nz))
    \tag{by $\happly(p,\tproj nz)$}\\
    &= j(\tproj n{g(z)}).
    \tag{by $\apfunc{j}(\mathsf{nat}^g_n(z))$}
  \end{align}
  However, since $\happly$ is functorial, it suffices to check equality for the three component paths:
  \begin{align*}
    \happly({\funext(\lam{z} \apfunc{i}(\mathsf{nat}^f_n(z)))},z)
    &= {\apfunc{i}(\mathsf{nat}^f_n(z))}\\
    \happly(\apfunc{\blank\circ \tprojf[C] n}(p), z)
    &= {\happly(p,\tproj nz)}\\
    \happly({\funext(\lam{z} \apfunc{j}(\mathsf{nat}^g_n(z)))},z)
    &= {\apfunc{j}(\mathsf{nat}^g_n(z))}.
  \end{align*}
  The first and third of these are just the fact that $\happly$ is quasi-inverse to $\funext$, while
  the second is an easy general lemma about $\happly$ and precomposition.
\end{proof}



\end{document}
