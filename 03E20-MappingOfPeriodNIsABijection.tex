\documentclass[12pt]{article}
\usepackage{pmmeta}
\pmcanonicalname{MappingOfPeriodNIsABijection}
\pmcreated{2013-03-22 13:48:57}
\pmmodified{2013-03-22 13:48:57}
\pmowner{Koro}{127}
\pmmodifier{Koro}{127}
\pmtitle{mapping of period $n$ is a bijection}
\pmrecord{7}{34540}
\pmprivacy{1}
\pmauthor{Koro}{127}
\pmtype{Proof}
\pmcomment{trigger rebuild}
\pmclassification{msc}{03E20}

\endmetadata

% this is the default PlanetMath preamble.  as your knowledge
% of TeX increases, you will probably want to edit this, but
% it should be fine as is for beginners.

% almost certainly you want these
\usepackage{amssymb}
\usepackage{amsmath}
\usepackage{amsfonts}

% used for TeXing text within eps files
%\usepackage{psfrag}
% need this for including graphics (\includegraphics)
%\usepackage{graphicx}
% for neatly defining theorems and propositions
%\usepackage{amsthm}
% making logically defined graphics
%%%\usepackage{xypic}

% there are many more packages, add them here as you need them

% define commands here

\newcommand{\sR}[0]{\mathbb{R}}
\newcommand{\sC}[0]{\mathbb{C}}
\newcommand{\sN}[0]{\mathbb{N}}
\newcommand{\sZ}[0]{\mathbb{Z}}
\begin{document}
 {\bf Theorem} Suppose $X$ is a set. 
 Then a mapping $f:X\to X$ \PMlinkname{of period}{PeriodOfMapping} $n$ is 
a bijection. 
 
 {\bf Proof.} If $n=1$, the claim is trivial;
 $f$ is the identity mapping.
Suppose $n=2,3,\ldots$.  
Then for any $x\in X$, we have $x=f\big(f^{n-1}(x)\big)$,
so $f$ is an surjection. To see that $f$ is a injection,
suppose $f(x)=f(y)$ for some $x,y$ in $X$. Since $f^n$
is the identity, it follows that $x=y$.  $\Box$
%%%%%
%%%%%
\end{document}
