\documentclass[12pt]{article}
\usepackage{pmmeta}
\pmcanonicalname{CartesianProduct}
\pmcreated{2013-03-22 11:48:56}
\pmmodified{2013-03-22 11:48:56}
\pmowner{djao}{24}
\pmmodifier{djao}{24}
\pmtitle{Cartesian product}
\pmrecord{10}{30359}
\pmprivacy{1}
\pmauthor{djao}{24}
\pmtype{Definition}
\pmcomment{trigger rebuild}
\pmclassification{msc}{03-00}
\pmclassification{msc}{81P10}
\pmclassification{msc}{81P05}
\pmrelated{GeneralizedCartesianProduct}

\usepackage{amssymb}
\usepackage{amsmath}
\usepackage{amsfonts}
\usepackage{graphicx}
%%%%\usepackage{xypic}
\begin{document}
For any sets $A$ and $B$, the {\em Cartesian product} $A \times B$ is the set consisting of all ordered pairs $(a,b)$ where $a \in A$ and $b \in B$.

The Cartesian product satisfies the following properties, for all sets $A$, $B$, $C$, and $D$:
\begin{itemize}
\item $A\times \emptyset = \emptyset$
\item $(A \times B) \cap (C \times D) = (A\cap C) \times (B\cap D)$
\item $(A \times B)^\complement = (A^\complement \times B^\complement)
 \cup (A^\complement \times B)
 \cup (A \times B^\complement)$
\end{itemize}

Here $\emptyset$ denotes the empty set, $\cap$ denotes intersection, $\cup$ denotes union, and ${}^\complement$ denotes complement with respect to some universal set $U$ containing $A$ and $B$.
%%%%%
%%%%%
%%%%%
%%%%%
\end{document}
