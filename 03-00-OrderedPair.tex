\documentclass[12pt]{article}
\usepackage{pmmeta}
\pmcanonicalname{OrderedPair}
\pmcreated{2013-03-22 11:48:53}
\pmmodified{2013-03-22 11:48:53}
\pmowner{djao}{24}
\pmmodifier{djao}{24}
\pmtitle{ordered pair}
\pmrecord{9}{30358}
\pmprivacy{1}
\pmauthor{djao}{24}
\pmtype{Definition}
\pmcomment{trigger rebuild}
\pmclassification{msc}{03-00}
\pmclassification{msc}{70A05}
\pmclassification{msc}{70G99}

\endmetadata

\usepackage{amssymb}
\usepackage{amsmath}
\usepackage{amsfonts}
\usepackage{graphicx}
%%%%\usepackage{xypic}
\begin{document}
For any sets $a$ and $b$, the {\em ordered pair} $(a,b)$ is the set $\{\{a\}, \{a,b\}\}$.

The characterizing property of an ordered pair is:
$$
(a,b) = (c,d) \iff a=b \text{\ \ and\ \ } c=d,
$$
and the above construction of ordered pair, as weird as it seems, is actually the simplest possible formulation which achieves this property.
%%%%%
%%%%%
%%%%%
%%%%%
\end{document}
