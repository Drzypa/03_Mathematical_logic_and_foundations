\documentclass[12pt]{article}
\usepackage{pmmeta}
\pmcanonicalname{SymmetricDifference}
\pmcreated{2013-03-22 11:59:41}
\pmmodified{2013-03-22 11:59:41}
\pmowner{CWoo}{3771}
\pmmodifier{CWoo}{3771}
\pmtitle{symmetric difference}
\pmrecord{20}{30916}
\pmprivacy{1}
\pmauthor{CWoo}{3771}
\pmtype{Definition}
\pmcomment{trigger rebuild}
\pmclassification{msc}{03E20}
\pmsynonym{set symmetric difference}{SymmetricDifference}
\pmsynonym{symmetric set difference}{SymmetricDifference}
\pmsynonym{symmetric difference between sets}{SymmetricDifference}
%\pmkeywords{set}
%\pmkeywords{union}
%\pmkeywords{intersection}
\pmrelated{SetDifference}
\pmrelated{ProofOfTheAssociativityOfTheSymmetricDifferenceOperator}
\pmdefines{symmetric difference operator}

\usepackage{amssymb}
\usepackage{amsmath}
\usepackage{amsfonts}
%\usepackage{graphicx}
%%%%\usepackage{xypic}
\usepackage{pstricks}
\newcommand{\symd}{\triangle}
\begin{document}
The \emph{symmetric difference} between two sets $A$ and $B$, written $A \symd B$, is the set of all $x$ such that either $x \in A$ or $x \in B$ but not both.  In other words, 
$$A\symd B:= (A\cup B)\setminus (A\cap B).$$

The Venn diagram for the symmetric difference of two sets $A,B$, represented by the two discs, is illustrated below, in light red:

\begin{center}
\begin{pspicture}(0,0)(8,4)
\begin{psclip}
{\pscircle[fillstyle=vlines,hatchcolor=red,hatchwidth=0.1\pslinewidth,hatchsep=1\pslinewidth](3,2){2}}
\pscircle[fillstyle=solid,hatchcolor=white,hatchwidth=0.1\pslinewidth,hatchsep=1\pslinewidth](5,2){2}
\end{psclip}
\begin{psclip}
{\pscircle[fillstyle=vlines,hatchcolor=red,hatchwidth=0.1\pslinewidth,hatchsep=1\pslinewidth](5,2){2}}
\pscircle[fillstyle=solid,hatchcolor=white,hatchwidth=0.1\pslinewidth,hatchsep=1\pslinewidth](3,2){2}
\end{psclip}
\rput(1.25,3.75){$A$}
\rput(6.75,3.75){$B$}
\rput(0,0){$.$}
\rput(8,4){$.$}
\end{pspicture}
\end{center}

\subsubsection*{Properties}
Suppose that $A$, $B$, and $C$ are sets.
\begin{itemize}
\item $A \symd B=(A\setminus B) \cup (B\setminus A)$.
\item $A \symd B=A^c \symd B^c$, where the superscript $c$ denotes taking complements.
\item Note that for any set $A$, the symmetric difference satisfies $A \symd A=\emptyset$ and $A \symd \emptyset=A$.
\item The symmetric difference operator is commutative since $A \symd B=(A\setminus B) \cup (B\setminus A) = (B\setminus A) \cup (A\setminus B) = B \symd A$.
\item The symmetric difference operation is associative:  $(A \symd B) \symd C = A \symd (B \symd C)$.  This means that we may drop the parentheses without any ambiguity, and we can talk about the symmetric difference of multiple sets.
\item Let $A_1,\ldots, A_n$ be sets.  The symmetric difference of these sets is written $$\substack{n \\ \displaystyle{\symd} \\ i=1} A_i.$$  In general, an element will be in the symmetric difference of several sets iff it is in an odd number of the sets.
\end{itemize}

It is worth noting that these properties show that the symmetric difference operation can be used as a group law to define an abelian group on the power set of some \PMlinkescapetext{fixed} set.  

Finally, we note that intersection distributes over the symmetric difference operator: $$A\cap(B\symd C)=(A\cap B)\symd(A\cap C),$$ giving us that the power set of a given fixed set can be made into a Boolean ring using symmetric difference as addition, and intersection as multiplication.
%%%%%
%%%%%
%%%%%
\end{document}
