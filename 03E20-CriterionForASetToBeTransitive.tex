\documentclass[12pt]{article}
\usepackage{pmmeta}
\pmcanonicalname{CriterionForASetToBeTransitive}
\pmcreated{2013-03-22 16:18:23}
\pmmodified{2013-03-22 16:18:23}
\pmowner{Wkbj79}{1863}
\pmmodifier{Wkbj79}{1863}
\pmtitle{criterion for a set to be transitive}
\pmrecord{6}{38429}
\pmprivacy{1}
\pmauthor{Wkbj79}{1863}
\pmtype{Theorem}
\pmcomment{trigger rebuild}
\pmclassification{msc}{03E20}
\pmrelated{CumulativeHierarchy}

\usepackage{amssymb}
\usepackage{amsmath}
\usepackage{amsfonts}

\usepackage{psfrag}
\usepackage{graphicx}
\usepackage{amsthm}
%%\usepackage{xypic}

\newtheorem*{thm*}{Theorem}
\begin{document}
\begin{thm*}
A set $X$ is transitive if and only if its power set $\mathcal{P}(X)$ is transitive.
\end{thm*}

\begin{proof}
First assume $X$ is transitive.  Let $A \in B \in \mathcal{P}(X)$.  Since $B \in \mathcal{P}(X)$, $B \subseteq X$.  Thus, $A \in X$.  Since $X$ is transitive, $A \subseteq X$.  Hence, $A \in \mathcal{P}(X)$.  It follows that $\mathcal{P}(X)$ is transitive.

Conversely, assume $\mathcal{P}(X)$ is transitive.  Let $a \in X$.  Then $\{a\} \in \mathcal{P}(X)$.  Since $\mathcal{P}(X)$ is transitive, $\{a\} \subseteq \mathcal{P}(X)$.  Thus, $a \in \mathcal{P}(X)$.  Hence, $a \subseteq X$.  It follows that $X$ is transitive.
\end{proof}
%%%%%
%%%%%
\end{document}
