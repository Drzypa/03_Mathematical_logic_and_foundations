\documentclass[12pt]{article}
\usepackage{pmmeta}
\pmcanonicalname{81pi1S1}
\pmcreated{2013-11-06 2:14:41}
\pmmodified{2013-11-06 2:14:41}
\pmowner{PMBookProject}{1000683}
\pmmodifier{rspuzio}{6075}
\pmtitle{8.1 $\pi_1 (S_1)$}
\pmrecord{1}{}
\pmprivacy{1}
\pmauthor{PMBookProject}{6075}
\pmtype{Feature}
\pmclassification{msc}{03B15}

\endmetadata

\usepackage{xspace}
\usepackage{amssyb}
\usepackage{amsmath}
\usepackage{amsfonts}
\usepackage{amsthm}
\newcommand{\id}[3][]{\ensuremath{#2 =_{#1} #3}\xspace}
\newcommand{\Sn}{\mathbb{S}}
\newcommand{\trunc}[2]{\mathopen{}\left\Vert #2\right\Vert_{#1}\mathclose{}}
\newcommand{\Z}{\ensuremath{\mathbb{Z}}\xspace}
\let\autoref\cref
\begin{document}

In this section, our goal is to show that $\id {\pi_1(\Sn ^1)} {\Z}$.
In fact, we will show that the loop space\index{loop space} ${\Omega(\Sn ^1)}$ is equivalent to $\Z$.
This is a stronger statement, because $\id{\pi_1(\Sn ^1)} {\trunc 0 {\Omega(\Sn ^1)}}$ by
definition; so if $\id {\Omega(\Sn ^1)} {\Z}$, then $\id {\trunc
  0 {\Omega(\Sn ^1)}} {\trunc 0 {\Z}}$ by congruence, and
$\Z$ is a set by definition (being a set-quotient; see \autoref{defn-Z},\autoref{Z-quotient-by-canonical-representatives}), so $\id {\trunc 0 {\Z}} {\Z}$.
Moreover, knowing that ${\Omega(\Sn ^1)}$ is a set will imply that $\pi_n(\Sn^1)$ is trivial for $n>1$, so we will actually have calculated \emph{all} the homotopy groups of $\Sn^1$.


\end{document}
