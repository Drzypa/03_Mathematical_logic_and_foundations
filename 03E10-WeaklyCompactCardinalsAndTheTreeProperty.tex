\documentclass[12pt]{article}
\usepackage{pmmeta}
\pmcanonicalname{WeaklyCompactCardinalsAndTheTreeProperty}
\pmcreated{2013-03-22 12:52:51}
\pmmodified{2013-03-22 12:52:51}
\pmowner{Henry}{455}
\pmmodifier{Henry}{455}
\pmtitle{weakly compact cardinals and the tree property}
\pmrecord{7}{33223}
\pmprivacy{1}
\pmauthor{Henry}{455}
\pmtype{Result}
\pmcomment{trigger rebuild}
\pmclassification{msc}{03E10}
\pmrelated{TreeProperty}
\pmrelated{Aronszajn}

% this is the default PlanetMath preamble.  as your knowledge
% of TeX increases, you will probably want to edit this, but
% it should be fine as is for beginners.

% almost certainly you want these
\usepackage{amssymb}
\usepackage{amsmath}
\usepackage{amsfonts}

% used for TeXing text within eps files
%\usepackage{psfrag}
% need this for including graphics (\includegraphics)
%\usepackage{graphicx}
% for neatly defining theorems and propositions
%\usepackage{amsthm}
% making logically defined graphics
%%%\usepackage{xypic}

% there are many more packages, add them here as you need them

% define commands here
%\PMlinkescapeword{theory}
\begin{document}
\PMlinkescapeword{group}
\PMlinkescapeword{class}

A cardinal is weakly compact if and only if it is inaccessible and has the tree property.

\emph{Weak compactness implies tree property}

Let $\kappa$ be a weakly compact cardinal and let $(T,<_T)$ be a $\kappa$ tree with all levels smaller than $\kappa$.  We define a theory in $L_{\kappa,\kappa}$ with for each $x\in T$, a constant $c_x$, and a single unary relation $B$.  Then our theory $\Delta$ consists of the sentences:
\begin{itemize}

\item $\neg \left[B(c_x) \wedge B(c_y)\right]$ for every incompatible $x,y\in T$

\item $\bigvee_{x\in T(\alpha)} B(c_x)$ for each $\alpha<\kappa$
\end{itemize}

It should be clear that $B$ represents membership in a cofinal branch, since the first class of sentences asserts that no incompatible elements are both in $B$ while the second class states that the branch intersects every level.

Clearly $|\Delta|=\kappa$, since there are $\kappa$ elements in $T$, and hence fewer than $\kappa\cdot\kappa=\kappa$ sentences in the first group, and of course there are $\kappa$ levels and therefore $\kappa$ sentences in the second group.

Now consider any $\Sigma\subseteq\Delta$ with $|\Sigma|<\kappa$.  Fewer than $\kappa$ sentences of the second group are included, so the set of $x$ for which the corresponding $c_x$ must all appear in $T(\alpha)$ for some $\alpha<\kappa$.  But since $T$ has branches of arbitrary height, $T(\alpha)\models\Sigma$.

Since $\kappa$ is weakly compact, it follows that $\Delta$ also has a model, and that model obviously has a set of $c_x$ such that $B(c_x)$ whose corresponding elements of $T$ intersect every level and are compatible, therefore forming a cofinal branch of $T$, proving that $T$ is not Aronszajn.
%%%%%
%%%%%
\end{document}
