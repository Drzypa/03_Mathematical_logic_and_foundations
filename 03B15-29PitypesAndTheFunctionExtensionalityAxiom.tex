\documentclass[12pt]{article}
\usepackage{pmmeta}
\pmcanonicalname{29PitypesAndTheFunctionExtensionalityAxiom}
\pmcreated{2013-11-18 16:30:41}
\pmmodified{2013-11-18 16:30:41}
\pmowner{PMBookProject}{1000683}
\pmmodifier{rspuzio}{6075}
\pmtitle{2.9 $\Pi$-types and the function extensionality axiom}
\pmrecord{4}{87615}
\pmprivacy{1}
\pmauthor{PMBookProject}{6075}
\pmtype{Feature}
\pmclassification{msc}{03B15}

\endmetadata

\usepackage{xspace}
\usepackage{amssyb}
\usepackage{amsmath}
\usepackage{amsfonts}
\usepackage{amsthm}
\makeatletter
\newcommand{\ct}{  \mathchoice{\mathbin{\raisebox{0.5ex}{$\displaystyle\centerdot$}}}             {\mathbin{\raisebox{0.5ex}{$\centerdot$}}}             {\mathbin{\raisebox{0.25ex}{$\scriptstyle\,\centerdot\,$}}}             {\mathbin{\raisebox{0.1ex}{$\scriptscriptstyle\,\centerdot\,$}}}}
\newcommand{\defeq}{\vcentcolon\equiv}  
\def\@dprd#1{\prod_{(#1)}\,}
\def\@dprd@noparens#1{\prod_{#1}\,}
\def\@dsm#1{\sum_{(#1)}\,}
\def\@dsm@noparens#1{\sum_{#1}\,}
\def\@eatprd\prd{\prd@parens}
\def\@eatsm\sm{\sm@parens}
\newcommand{\eqvspaced}[2]{\ensuremath{#1 \;\simeq\; #2}\xspace}
\newcommand{\funext}{\mathsf{funext}}
\newcommand{\happly}{\mathsf{happly}}
\newcommand{\id}[3][]{\ensuremath{#2 =_{#1} #3}\xspace}
\newcommand{\indexdef}[1]{\index{#1|defstyle}}   
\newcommand{\indexsee}[2]{\index{#1|see{#2}}}    
\newcommand{\narrowbreak}{}
\newcommand{\opp}[1]{\mathord{{#1}^{-1}}}
\newcommand{\pairpath}{\ensuremath{\mathsf{pair}^{\mathord{=}}}\xspace}
\newcommand{\Parens}[1]{\Bigl(#1\Bigr)}
\def\prd#1{\@ifnextchar\bgroup{\prd@parens{#1}}{\@ifnextchar\sm{\prd@parens{#1}\@eatsm}{\prd@noparens{#1}}}}
\def\prd@noparens#1{\mathchoice{\@dprd@noparens{#1}}{\@tprd{#1}}{\@tprd{#1}}{\@tprd{#1}}}
\def\prd@parens#1{\@ifnextchar\bgroup  {\mathchoice{\@dprd{#1}}{\@tprd{#1}}{\@tprd{#1}}{\@tprd{#1}}\prd@parens}  {\@ifnextchar\sm    {\mathchoice{\@dprd{#1}}{\@tprd{#1}}{\@tprd{#1}}{\@tprd{#1}}\@eatsm}    {\mathchoice{\@dprd{#1}}{\@tprd{#1}}{\@tprd{#1}}{\@tprd{#1}}}}}
\newcommand{\proj}[1]{\ensuremath{\mathsf{pr}_{#1}}\xspace}
\newcommand{\refl}[1]{\ensuremath{\mathsf{refl}_{#1}}\xspace}
\def\sm#1{\@ifnextchar\bgroup{\sm@parens{#1}}{\@ifnextchar\prd{\sm@parens{#1}\@eatprd}{\sm@noparens{#1}}}}
\def\sm@noparens#1{\mathchoice{\@dsm@noparens{#1}}{\@tsm{#1}}{\@tsm{#1}}{\@tsm{#1}}}
\def\sm@parens#1{\@ifnextchar\bgroup  {\mathchoice{\@dsm{#1}}{\@tsm{#1}}{\@tsm{#1}}{\@tsm{#1}}\sm@parens}  {\@ifnextchar\prd    {\mathchoice{\@dsm{#1}}{\@tsm{#1}}{\@tsm{#1}}{\@tsm{#1}}\@eatprd}    {\mathchoice{\@dsm{#1}}{\@tsm{#1}}{\@tsm{#1}}{\@tsm{#1}}}}}
\def\@tprd#1{\mathchoice{{\textstyle\prod_{(#1)}}}{\prod_{(#1)}}{\prod_{(#1)}}{\prod_{(#1)}}}
\newcommand{\trans}[2]{\ensuremath{{#1}_{*}\mathopen{}\left({#2}\right)\mathclose{}}\xspace}
\newcommand{\transfib}[3]{\ensuremath{\mathsf{transport}^{#1}(#2,#3)\xspace}}
\newcommand{\Transfib}[3]{\ensuremath{\mathsf{transport}^{#1}\Big(#2,\, #3\Big)\xspace}}
\def\@tsm#1{\mathchoice{{\textstyle\sum_{(#1)}}}{\sum_{(#1)}}{\sum_{(#1)}}{\sum_{(#1)}}}
\newcommand{\UU}{\ensuremath{\mathcal{U}}\xspace}
\newcommand{\vcentcolon}{:\!\!}
\newcounter{mathcount}
\setcounter{mathcount}{1}
\newtheorem{preaxiom}{Axiom}
\newenvironment{axiom}{\begin{preaxiom}}{\end{preaxiom}\addtocounter{mathcount}{1}}
\renewcommand{\thepreaxiom}{2.9.\arabic{mathcount}}
\newenvironment{myeqn}{\begin{equation}}{\end{equation}\addtocounter{mathcount}{1}}
\renewcommand{\theequation}{2.9.\arabic{mathcount}}
\newtheorem{prelem}{Lemma}
\newenvironment{lem}{\begin{prelem}}{\end{prelem}\addtocounter{mathcount}{1}}
\renewcommand{\theprelem}{2.9.\arabic{mathcount}}
\newenvironment{narrowmultline*}{\csname equation*\endcsname}{\csname endequation*\endcsname}
\let\autoref\cref
\let\type\UU
\makeatother

\begin{document}
\index{type!dependent function|(}%
\index{type!function|(}%
\index{homotopy|(}%
Given a type $A$ and a type family $B : A \to \type$, consider the dependent function type $\prd{x:A}B(x)$.
We expect the type $f=g$ of paths from $f$ to $g$ in $\prd{x:A} B(x)$ to be equivalent to 
the type of pointwise paths:\index{pointwise!equality of functions}
\begin{myeqn}
  \eqvspaced{(\id{f}{g})}{\Parens{\prd{x:A} (\id[B(x)]{f(x)}{g(x)})}}.\label{eq:path-forall}
\end{myeqn}
From a traditional perspective, this would say that two functions which are equal at each point are equal as functions.
\index{continuity of functions in type theory@``continuity'' of functions in type theory}%
From a topological perspective, it would say that a path in a function space is the same as a continuous homotopy.
\index{functoriality of functions in type theory@``functoriality'' of functions in type theory}%
And from a categorical perspective, it would say that an isomorphism in a functor category is a natural family of isomorphisms.

Unlike the case in the previous sections, however, the basic type theory presented in \PMlinkexternal{Chapter 1}{http://planetmath.org/node/87533} is insufficient to prove~\eqref{eq:path-forall}.
All we can say is that there is a certain function
\begin{myeqn}\label{eq:happly}
  \happly : (\id{f}{g}) \to \prd{x:A} (\id[B(x)]{f(x)}{g(x)})
\end{myeqn}
which is easily defined by path induction.
For the moment, therefore, we will assume:

\begin{axiom}[Function extensionality]\label{axiom:funext}
  \indexsee{axiom!function extensionality}{function extensionality}%
  \indexdef{function extensionality}%
  For any $A$, $B$, $f$, and $g$, the function~\eqref{eq:happly} is an equivalence.
\end{axiom}

We will see in later chapters that this axiom follows both from univalence (see \PMlinkname{\S 2.10}{210universesandtheunivalenceaxiom},\PMlinkname{\S 4.9}{49univalenceimpliesfunctionextensionality}) and from an interval type (see \PMlinkname{\S 6.3}{63theinterval}).

In particular, \PMlinkname{Axiom 2.9.3}{29pitypesandthefunctionextensionalityaxiom#Thmaxiom1} implies that~\eqref{eq:happly} has a quasi-inverse
\[
\funext : \Parens{\prd{x:A} (\id{f(x)}{g(x)})} \to {(\id{f}{g})}.
\]
This function is also referred to as ``function extensionality''.
As we did with $\pairpath$ in \PMlinkname{\S 2.6}{26cartesianproducttypes}, we can regard $\funext$ as an \emph{introduction rule} for the type $\id f g$.
From this point of view, $\happly$ is the \emph{elimination rule}, while the homotopies witnessing $\funext$ as quasi-inverse to $\happly$ become a propositional computation rule\index{computation rule!propositional!for identities between functions}
\[
\id{\happly({\funext{(h)}},x)}{h(x)} \qquad\text{for }h:\prd{x:A} (\id{f(x)}{g(x)})
\]
and a propositional uniqueness principle\index{uniqueness!principle!for identities between functions}:
\[
\id{p}{\funext (x \mapsto \happly(p,{x}))} \qquad\text{for } p: (\id f g).
\]

We can also compute the identity, inverses, and composition in $\Pi$-types; they are simply given by pointwise operations:\index{pointwise!operations on functions}.
\begin{align*}
\refl{f} &= \funext(x \mapsto \refl{f(x)}) \\
\opp{\alpha} &= \funext (x \mapsto \opp{\happly (\alpha,x)})  \\
{\alpha} \ct \beta &= \funext (x \mapsto {\happly({\alpha},x) \ct \happly({\beta},x)}).
\end{align*}
The first of these equalities follows from the definition of $\happly$, while the second and third are easy path inductions.

Since the non-dependent function type $A\to B$ is a special case of the dependent function type $\prd{x:A} B(x)$ when $B$ is independent of $x$, everything we have said above applies in non-dependent cases as well.
\index{transport!in function types}%
The rules for transport, however, are somewhat simpler in the non-dependent case.
Given a type $X$, a path $p:\id[X]{x_1}{x_2}$, type families $A,B:X\to \type$, and a function $f : A(x_1) \to B(x_1)$,  we have
\begin{align}\label{eq:transport-arrow}
  \transfib{A\to B}{p}{f} &=
  \Big(x \mapsto \transfib{B}{p}{f(\transfib{A}{\opp p}{x})}\Big)
\end{align}
\addtocounter{mathcount}{1}
where $A\to B$ denotes abusively the type family $X\to \type$ defined by
\[(A\to B)(x) \defeq (A(x)\to B(x)).\]
In other words, when we transport a function $f:A(x_1)\to B(x_1)$ along a path $p:x_1=x_2$, we obtain the function $A(x_2)\to B(x_2)$ which transports its argument backwards along $p$ (in the type family $A$), applies $f$, and then transports the result forwards along $p$ (in the type family $B$).
This can be proven easily by path induction.

\index{transport!in dependent function types}%
Transporting dependent functions is similar, but more complicated.
Suppose given $X$ and $p$ as before, type families $A:X\to \type$ and $B:\prd{x:X} (A(x)\to\type)$, and also a dependent function $f : \prd{a:A(x_1)} B(x_1,a)$.
Then for $a:A(x_2)$, we have
\begin{narrowmultline*}
  \transfib{\Pi_A(B)}{p}{f}(a) = \narrowbreak
  \Transfib{\widehat{B}}{\opp{(\pairpath(\opp{p},\refl{ \trans{\opp p}{a} }))}}{f(\transfib{A}{\opp p}{a})}
\end{narrowmultline*}
where $\Pi_A(B)$ and $\widehat{B}$ denote respectively the type families
\begin{myeqn}\label{eq:transport-arrow-families}
\begin{array}{rclcl}
\Pi_A(B) &\defeq& \big(x\mapsto \prd{a:A(x)} B(x,a) \big) &:& X\to \type\\
\widehat{B} &\defeq& \big(w \mapsto B(\proj1w,\proj2w) \big) &:& \big(\sm{x:X} A(x)\big) \to \type.
\end{array}
\end{myeqn}
If these formulas look a bit intimidating, don't worry about the details.
The basic idea is just the same as for the non-dependent function type: we transport the argument backwards, apply the function, and then transport the result forwards again.

Now recall that for a general type family $P:X\to\type$, in \PMlinkname{\S 2.2}{22functionsarefunctors} we defined the type of \emph{dependent paths} over $p:\id[X]xy$ from $u:P(x)$ to $v:P(y)$ to be $\id[P(y)]{\trans{p}{u}}{v}$.
When $P$ is a family of function types, there is an equivalent way to represent this which is often more convenient.
\index{path!dependent!in function types}

\begin{lem}\label{thm:dpath-arrow}
  Given type families $A,B:X\to\type$ and $p:\id[X]xy$, and also $f:A(x)\to B(x)$ and $g:A(y)\to B(y)$, we have an equivalence
  \[ \eqvspaced{ \big(\trans{p}{f} = {g}\big) } { \prd{a:A(x)}  (\trans{p}{f(a)} = g(\trans{p}{a})) }. \]
  Moreover, if $q:\trans{p}{f} = {g}$ corresponds under this equivalence to $\widehat q$, then for $a:A(x)$, the path
  \[ \happly(q,\trans p a) : (\trans p f)(\trans p a) = g(\trans p a)\]
  is equal to the composite
  \begin{align*}
    (\trans p f)(\trans p a)
    &= \trans p {f (\trans {\opp p}{\trans p a})}
    \tag{by~\eqref{eq:transport-arrow}}\\
    &= \trans p {f(a)}\\
    &= g(\trans p a).
    \tag{by $\widehat{q}$}
  \end{align*}
\end{lem}
\begin{proof}
  By path induction, we may assume $p$ is reflexivity, in which case the desired equivalence reduces to function extensionality.
  The second statement then follows by the computation rule for function extensionality.
\end{proof}

As usual, the case of dependent functions is similar, but more complicated.
\index{path!dependent!in dependent function types}

\begin{lem}\label{thm:dpath-forall}
  Given type families $A:X\to\type$ and $B:\prd{x:X} A(x)\to\type$ and $p:\id[X]xy$, and also $f:\prd{a:A(x)} B(x,a)$ and $g:\prd{a:A(y)} B(y,a)$, we have an equivalence
  \[ \eqvspaced{ \big(\trans{p}{f} = {g}\big) } { \Parens{\prd{a:A(x)}  \transfib{\widehat{B}}{\pairpath(p,\refl{\trans pa})}{f(a)} = g(\trans{p}{a}) } } \]
  with $\widehat{B}$ as in~\eqref{eq:transport-arrow-families}.
\end{lem}

We leave it to the reader to prove this and to formulate a suitable computation rule.

\index{homotopy|)}%
\index{type!dependent function|)}%
\index{type!function|)}%

\end{document}
