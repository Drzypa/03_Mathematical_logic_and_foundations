\documentclass[12pt]{article}
\usepackage{pmmeta}
\pmcanonicalname{810AdditionalResults}
\pmcreated{2013-11-06 15:38:44}
\pmmodified{2013-11-06 15:38:44}
\pmowner{PMBookProject}{1000683}
\pmmodifier{rspuzio}{6075}
\pmtitle{8.10 Additional Results}
\pmrecord{1}{}
\pmprivacy{1}
\pmauthor{PMBookProject}{6075}
\pmtype{Feature}
\pmclassification{msc}{03B15}

\endmetadata

\usepackage{xspace}
\usepackage{amssyb}
\usepackage{amsmath}
\usepackage{amsfonts}
\usepackage{amsthm}
\newcommand{\choice}[1]{\ensuremath{\mathsf{AC}_{#1}}\xspace}
\newcommand{\id}[3][]{\ensuremath{#2 =_{#1} #3}\xspace}
\newcommand{\indexdef}[1]{\index{#1|defstyle}}   
\newcommand{\indexsee}[2]{\index{#1|see{#2}}}    
\newcommand{\inl}{\ensuremath\inlsym\xspace}
\newcommand{\inlsym}{{\mathsf{inl}}}
\newcommand{\inr}{\ensuremath\inrsym\xspace}
\newcommand{\inrsym}{{\mathsf{inr}}}
\newcommand{\LEM}[1]{\ensuremath{\mathsf{LEM}_{#1}}\xspace}
\newcommand{\Sn}{\mathbb{S}}
\newcommand{\Z}{\ensuremath{\mathbb{Z}}\xspace}
\newcounter{mathcount}
\setcounter{mathcount}{1}
\newtheorem{prethm}{Theorem}
\newenvironment{thm}{\begin{prethm}}{\end{prethm}\addtocounter{mathcount}{1}}
\renewcommand{\theprethm}{8.10.\arabic{mathcount}}
\let\autoref\cref

\begin{document}

Though we do not present the proofs in this chapter, following results have also been established in homotopy type theory.  

\begin{thm}
\index{homotopy!group!of sphere}
There exists a $k$ such that for all $n \ge 3$, $\pi_{n+1}(\Sn ^n) =
\Z_k$.  
\end{thm}

\begin{proof}[Notes on the proof.]
The proof consists of a calculation of $\pi_4(\Sn ^3)$, together with an
appeal to stability (\cref{cor:stability-spheres}).  In the classical
statement of this result, $k$ is $2$.  While we have not yet checked that
$k$ is in fact $2$, our calcluation of $\pi_4(\Sn ^3)$ is constructive,\index{mathematics!constructive}
like all the rest of the proofs in this chapter.
(More precisely, it doesn't use any additional axioms such as \LEM{} or \choice{}, making it as constructive as
univalence and higher inductive types are.)  Thus, given a
computational interpretation of homotopy type theory, we could run the
proof on a computer to verify that $k$ is $2$.  This example is quite
intriguing, because it is the first calculation of a homotopy group
for which we have not needed to know the answer in advance.
\end{proof}

% Recall from \autoref{sec:colimits} that $X \sqcup^C Y$ denotes the
% (homotopy) pushout of $X$ and $Y$ along $C$.
\index{pushout}%

\begin{thm}[Blakers--Massey theorem]\label{Blakers-Massey}
  \indexdef{theorem!Blakers--Massey}%
  \indexsee{Blakers--Massey theorem}{theorem, Blakers--Massey}%
  Suppose we are given maps $f : C  \rightarrow X$, and $g : C \rightarrow Y$. Taking first the pushout $X \sqcup^C Y $ of $f$ and $g$ and then the pullback of its inclusions $\inl : X \rightarrow X \sqcup^C Y \leftarrow Y : \inr$, we have an induced map $C \to X \times_{(X \sqcup^C Y)} Y$.

  If $f$ is $i$-connected and $g$ is $j$-connected, then this induced map is $(i+j)$-connected. In other words, for any points $x:X$, $y:Y$, the corresponding fiber $C_{x,y}$ of $(f,g) : C \to X \times Y $ gives an approximation to the path space $\id[X \sqcup^C Y]{\inl(x)}{\inr(y)}$ in the pushout.
\end{thm}

It should be noted that in classical algebraic topology, the Blakers--Massey theorem is often stated in a somewhat different form, where the maps $f$ and $g$ are replaced by inclusions of subcomplexes of CW complexes, and the homotopy pushout and homotopy pullback by a union and intersection, respectively.
In order to express the theorem in homotopy type theory, we have to replace notions of this sort with ones that are homotopy-invariant.
We have seen another example of this in the van Kampen theorem (\autoref{sec:van-kampen}), where we had to replace a union of open subsets by a homotopy pushout.

\begin{thm}[Eilenberg--Mac Lane Spaces]\label{Eilenberg-Mac-Lane-Spaces}
\index{Eilenberg--Mac Lane space}
For any abelian\index{group!abelian} group $G$ and positive integer $n$, there is an $n$-type
$K(G,n)$ such that $\pi_n(K(G,n)) = G$, and  $\pi_k(K(G,n)) = 0$
for $k\neq n$.
\end{thm}

\begin{thm}[Covering spaces]\label{thm:covering-spaces}
  \index{covering space}%
  For a connected space $A$, there is an equivalence between covering spaces over $A$ and sets with an action of $\pi_1(A)$.
\end{thm}



\end{document}
