\documentclass[12pt]{article}
\usepackage{pmmeta}
\pmcanonicalname{SpecialElementsInARelationAlgebra}
\pmcreated{2013-03-22 17:48:43}
\pmmodified{2013-03-22 17:48:43}
\pmowner{CWoo}{3771}
\pmmodifier{CWoo}{3771}
\pmtitle{special elements in a relation algebra}
\pmrecord{9}{40275}
\pmprivacy{1}
\pmauthor{CWoo}{3771}
\pmtype{Definition}
\pmcomment{trigger rebuild}
\pmclassification{msc}{03G15}
\pmdefines{function element}
\pmdefines{injective element}
\pmdefines{surjective element}
\pmdefines{reflexive element}
\pmdefines{symmetric element}
\pmdefines{transitive element}
\pmdefines{equivalence element}
\pmdefines{domain element}
\pmdefines{range element}
\pmdefines{ideal element}
\pmdefines{rectangle}
\pmdefines{square}
\pmdefines{antisymmetric element}
\pmdefines{subidentity}

\usepackage{amssymb,amscd}
\usepackage{amsmath}
\usepackage{amsfonts}
\usepackage{mathrsfs}
\usepackage{tabls}

% used for TeXing text within eps files
%\usepackage{psfrag}
% need this for including graphics (\includegraphics)
%\usepackage{graphicx}
% for neatly defining theorems and propositions
\usepackage{amsthm}
% making logically defined graphics
%%\usepackage{xypic}
\usepackage{pst-plot}

% define commands here
\newcommand*{\abs}[1]{\left\lvert #1\right\rvert}
\newtheorem{prop}{Proposition}
\newtheorem{thm}{Theorem}
\newtheorem{ex}{Example}
\newcommand{\real}{\mathbb{R}}
\newcommand{\pdiff}[2]{\frac{\partial #1}{\partial #2}}
\newcommand{\mpdiff}[3]{\frac{\partial^#1 #2}{\partial #3^#1}}
\begin{document}
Let $A$ be a relation algebra with operators $(\vee,\wedge,\ ;,',^-,0,1,i)$ of type $(2,2,2,1,1,0,0,0)$.  Then $a\in A$ is called a
\begin{itemize}
\item \emph{function element} if $e^-\ ; e\le i$,
\item \emph{injective element} if it is a function element such that $e\ ; e^-\le i$,
\item \emph{surjective element} if $e^-\ ;e=i$,
\item \emph{reflexive element} if $i\le a$,
\item \emph{symmetric element} if $a^-\le a$, 
\item \emph{transitive element} if $a\ ; a\le a$,
\item \emph{subidentity} if $a\le i$,
\item \emph{antisymmetric element} if $a\wedge a^-$ is a subidentity,
\item \emph{equivalence element} if it is symmetric and transitive (not necessarily reflexive!),
\item \emph{domain element} if $a\ ; 1 = a$,
\item \emph{range element} if $1\ ; a=a$,
\item \emph{ideal element} if $1\ ; a\ ; 1=a$,
\item \emph{rectangle} if $a=b\ ; 1\ ; c$ for some $b,c\in A$, and
\item \emph{square} if it is a rectangle where $b=c$ (using the notations above).
\end{itemize}

These special elements are so named because they are the names of the corresponding binary relations on a set.  The following table shows the correspondence.

\begin{center}
\begin{tabular}{|c|c|}
\hline
element in relation algebra $A$ & binary relation on set $S$ \\
\hline\hline
function element & function (on $S$) \\
\hline
injective element & injection \\
\hline
surjective element & surjection \\
\hline
reflexive element & reflexive relation \\
\hline
symmetric element & symmetric relation \\
\hline
transitive element & transitive relation \\
\hline
subidentity & $I_T:=\lbrace (x,x)\mid x\in T\rbrace$ where $T\subseteq S$ \\
\hline
antisymmetric element & antisymmetric relation \\
\hline
equivalence element & symmetric reflexive relation (not an equivalence relation!) \\
\hline
domain element & $\operatorname{dom}(R)\times S$ where $R\subseteq S^2$ \\
\hline
range element & $S\times \operatorname{ran}(R)$ where $R\subseteq S^2$ \\
\hline
ideal element &  \\
\hline
rectangle & $U\times V\subseteq S^2$ \\
\hline
square & $U^2$, where $U\subseteq S$ \\
\hline
\end{tabular}
\end{center}

\begin{thebibliography}{8}
\bibitem{sg} S. R. Givant, \emph{The Structure of Relation Algebras Generated by Relativizations}, American Mathematical Society (1994).
\end{thebibliography}
%%%%%
%%%%%
\end{document}
