\documentclass[12pt]{article}
\usepackage{pmmeta}
\pmcanonicalname{ProofEquivalenceOfFormulationOfFoundation}
\pmcreated{2013-03-22 13:04:37}
\pmmodified{2013-03-22 13:04:37}
\pmowner{Henry}{455}
\pmmodifier{Henry}{455}
\pmtitle{proof equivalence of formulation of foundation}
\pmrecord{6}{33487}
\pmprivacy{1}
\pmauthor{Henry}{455}
\pmtype{Proof}
\pmcomment{trigger rebuild}
\pmclassification{msc}{03C99}

% this is the default PlanetMath preamble.  as your knowledge
% of TeX increases, you will probably want to edit this, but
% it should be fine as is for beginners.

% almost certainly you want these
\usepackage{amssymb}
\usepackage{amsmath}
\usepackage{amsfonts}

% used for TeXing text within eps files
%\usepackage{psfrag}
% need this for including graphics (\includegraphics)
%\usepackage{graphicx}
% for neatly defining theorems and propositions
%\usepackage{amsthm}
% making logically defined graphics
%%%\usepackage{xypic}

% there are many more packages, add them here as you need them

% define commands here
%\PMlinkescapeword{theory}
\begin{document}
We show that each of the three formulations of the axiom of foundation given are equivalent.

\section*{$1\Rightarrow 2$}

Let $X$ be a set and consider any function $f:\omega\rightarrow \operatorname{tc}(X)$.  Consider $Y=\{f(n)\mid n<\omega\}$.  By assumption, there is some $f(n)\in Y$ such that $f(n)\cap Y=\emptyset$, hence $f(n+1)\notin f(n)$.

\section*{$2\Rightarrow 3$}

Let $\phi$ be some formula such that $\phi(x)$ is true and for every $X$ such that $\phi(X)$, there is some $y\in X$ such that $\phi(y)$.  Then define $f(0)=x$ and $f(n+1)$ is some $y\in f(n)$ such that $\phi(y)$.  This would construct a function violating the assumption, so there is no such $\phi$.

\section*{$3\Rightarrow1$}

Let $X$ be a nonempty set and define $\phi(x)\equiv x\in X$.  Then $\phi$ is true for some $X$, and by assumption, there is some $y$ such that $\phi(y)$ but there is no $z\in y$ such that $\phi(z)$.  Hence $y\in X$ but $y\cap X=\emptyset$.
%%%%%
%%%%%
\end{document}
