\documentclass[12pt]{article}
\usepackage{pmmeta}
\pmcanonicalname{PropertiesOfAFunction}
\pmcreated{2013-03-22 16:21:38}
\pmmodified{2013-03-22 16:21:38}
\pmowner{CWoo}{3771}
\pmmodifier{CWoo}{3771}
\pmtitle{properties of a function}
\pmrecord{24}{38497}
\pmprivacy{1}
\pmauthor{CWoo}{3771}
\pmtype{Definition}
\pmcomment{trigger rebuild}
\pmclassification{msc}{03E20}
\pmrelated{PropertiesOfFunctions}

\usepackage{amssymb,amscd}
\usepackage{amsmath}
\usepackage{amsfonts}

% used for TeXing text within eps files
%\usepackage{psfrag}
% need this for including graphics (\includegraphics)
%\usepackage{graphicx}
% for neatly defining theorems and propositions
%\usepackage{amsthm}
% making logically defined graphics
%%\usepackage{xypic}
\usepackage{pst-plot}
\usepackage{psfrag}

% define commands here

\begin{document}
Let $X,Y$ be sets and $f:X\to Y$ be a function.  For any $A\subseteq X$, define $$f(A):=\lbrace f(x)\in Y \mid x\in A\rbrace$$ and any $B\subseteq Y$, define $$f^{-1}(B):=\lbrace x\in X\mid f(x)\in B\rbrace.$$
So $f(A)$ is a subset of $Y$ and $f^{-1}(B)$ is a subset of $X$.

Let $A,A_1,A_2,A_i$ be arbitrary subsets of $X$ and $B,B_1,B_2,B_j$ be arbitrary subsets of $Y$, where $i$ belongs to the index set $I$ and $j$ to the index set $J$.\, We have the following properties:

\begin{enumerate}
\item If $A_1\subset A_2$, then $f(A_1)\subseteq f(A_2)$.  In particular, $f(A)\subseteq f(X)$.
\item $f(A_1\cup A_2)=f(A_1)\cup f(A_2)$.  More generally, $f(\bigcup_i A_i)=\bigcup_i f(A_i)$.
\item \label{intersection_property} $f(A_1\cap A_2)\subseteq f(A_1)\cap f(A_2)$.  The equality fails in the example where $f$ is a real function defined by $f(x)=x^2$ and $A_1=\lbrace 1\rbrace$, $A_2=\lbrace -1\rbrace$.  Equality occurs iff $f$ is one-to-one: \begin{quote} Suppose $f(x)=f(y)=z$. Pick $A_1=\lbrace x\rbrace$ and $A_2=
\lbrace y\rbrace$.   Then $f(A_1\cap A_2) = f(A_1)\cap f(A_2)= \lbrace z\rbrace\ne\varnothing $.  This means that $A_1\cap A_2\ne \varnothing$.  Since both $A_1$ and $A_2$ are singletons, $A_1=A_2$, or $x=y$.  \end{quote}
\begin{quote} Conversely, let's show that $f$ is one-to-one then $f(A_1\cap A_2) = f(A_1)\cap f(A_2)$.  To do this, we only need to show the right hand side is included in the left, and this follows since if $x \in f(A_1)\cap f(A_2)$ then for some $a_1 \in A_1$ and $a_2 \in A_2$ we have $x = f(a_1) = f(a_2)$.  As $f$ is one-to-one, $a_1 = a_2$ and so $a_1$ lies in $A_1 \cap A_2$ and $x$ is in $f(A_1 \cap A_2)$. \end{quote}

More generally, $f(\bigcap_i A_i)\subseteq \bigcap_i f(A_i)$.
\item $f(A_1)-f(A_2)\subseteq f(A_1-A_2)$: If $y\in f(A_1)-f(A_2)$, then $y=f(x)$ for some $x\in A_1$.  If $x\in A_2$, then $y=f(x)\in f(A_2)$ as well, a contradiction.  So $x\in A_1-A_2$, and $y=f(x)\in f(A_1-A_2)$.  The inequality is strict in the case when $f:\mathbb{Z}\to \mathbb{Z}$ given by $f(x)=1$, and $A_1=\mathbb{Z}$ and $A_2=\lbrace 2\rbrace$.
\item \label{inv_image_property} $A\subseteq f^{-1}f(A)$.  Again, one finds that equality fails for the real function $f(x)=x^2$ by selecting $A=\lbrace 1\rbrace$.  Equality again holds iff $f$ is injective:
\begin{quote}
Suppose $x \in f^{-1}f(A)$.  By definition this means that $f(x) = f(a)$ for some $x \in A$, and since $f$ is injective we have $x = a \in A$. It follows that $f^{-1}f(A) \subseteq A$.  Convserly, if $f(x)=f(y)=z$, then $\lbrace x,y\rbrace=f^{-1}f(\lbrace x,y\rbrace)=f^{-1}(\lbrace z\rbrace)$.  On the other hand $\lbrace x\rbrace =f^{-1}f(\lbrace x\rbrace)=f^{-1}(\lbrace z\rbrace)$.  So $\lbrace x,y\rbrace =\lbrace x\rbrace$, $x=y$.
\end{quote}
\item If $B_1\subseteq B_2$, then $f^{-1}(B_1)\subseteq f^{-1}(B_2)$.  In particular, $f^{-1}(B)\subseteq f^{-1}(Y)$.
\item $f^{-1}(B_1\cup B_2)=f^{-1}(B_1)\cup f^{-1}(B_2)$.  More generally, $f^{-1}(\bigcup_j B_j)=\bigcup_j f^{-1}(B_j)$.
\item $f^{-1}(B_1\cap B_2)=f^{-1}(B_1)\cap f^{-1}(B_2)$.  More generally, $f^{-1}(\bigcap_j B_j)=\bigcap_j f^{-1}(B_j)$.
\item $f^{-1}(Y-B)=X-f^{-1}(B)$.  As a result, $f^{-1}(B_1-B_2)=f^{-1}(B_1)-f^{-1}(B_2)$.
\item \label{inv_image_property_2} $ff^{-1}(B) \subseteq B$.  Yet again, one finds that equality fails for the real function $f(x)=x^2$ by selecting $B=[-1,1]$.  Equality holds iff $f$ is surjective:
\begin{quote}
Suppose $f$ is onto.  Pick any $y\in B\subset Y$.  Then $y=f(x)$ for some $x\in X$.  In other words, $x\in f^{-1}(B)$ and hence $y=f(x)\in ff^{-1}(B)$.  Now suppose the convserse, then pick $B=Y$, and we have $Y=ff^{-1}(Y)=f(X)$.
\end{quote}
\item \label{quasi_inverse_property} Combining \ref{inv_image_property_2} and \ref{inv_image_property}, we have that $ff^{-1}f(A)=f(A)$ and $f^{-1}ff^{-1}(B)=f^{-1}(B)$.  Let's show the first equality:
\begin{quote}
From \ref{inv_image_property}, $A\subseteq f^{-1}f(A)$, so that $f(A)\subseteq ff^{-1}f(A)$ (by 1).  Set $B=f(A)$.  Then by \ref{inv_image_property_2}, $ff^{-1}f(A)=ff^{-1}(B)\subseteq B=f(A)$.
\end{quote}
\end{enumerate}

\textbf{Remarks.}  
\begin{itemize}
\item $f^{-1}f$ and $ff^{-1}$ \PMlinkescapetext{mean} the compositions of the function and its inverse as defined at the beginning of the entry, so that $f^{-1}f(A)=f^{-1}(f(A))$ and $ff^{-1}(B)=f(f^{-1}(B))$.
\item From the definition above, we see that a function $f:X\to Y$ induces two functions $[f]$ and $[f^{-1}]$ defined by
$$[f]:2^X\to 2^Y\mbox{ such that }[f](A):=f(A)\mbox{ and}$$
$$[f^{-1}]:2^Y\to 2^X\mbox{ such that }[f^{-1}](B):=f^{-1}(B).$$
The last property \ref{quasi_inverse_property} says that $[f]$ and $[f^{-1}]$ are quasi-inverses of each other.
\item $f$ is a bijection iff $[f]$ and $[f^{-1}]$ are inverses of one another.
\end{itemize}
%%%%%
%%%%%
\end{document}
