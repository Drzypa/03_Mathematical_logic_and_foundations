\documentclass[12pt]{article}
\usepackage{pmmeta}
\pmcanonicalname{WellfoundedRelation}
\pmcreated{2013-03-22 17:24:31}
\pmmodified{2013-03-22 17:24:31}
\pmowner{ratboy}{4018}
\pmmodifier{ratboy}{4018}
\pmtitle{well-founded relation}
\pmrecord{9}{39781}
\pmprivacy{1}
\pmauthor{ratboy}{4018}
\pmtype{Definition}
\pmcomment{trigger rebuild}
\pmclassification{msc}{03E20}
\pmrelated{Relation}
\pmrelated{RMinimalElement}
\pmdefines{well-founded}

\endmetadata

% this is the default PlanetMath preamble.  as your knowledge
% of TeX increases, you will probably want to edit this, but
% it should be fine as is for beginners.

% almost certainly you want these
\usepackage{amssymb}
\usepackage{amsmath}
\usepackage{amsfonts}

% used for TeXing text within eps files
%\usepackage{psfrag}
% need this for including graphics (\includegraphics)
%\usepackage{graphicx}
% for neatly defining theorems and propositions
%\usepackage{amsthm}
% making logically defined graphics
%%%\usepackage{xypic}

% there are many more packages, add them here as you need them

% define commands here

\begin{document}
\PMlinkescapeword{class}

A binary relation $R$ on a \PMlinkname{class}{Class} $X$ is \emph{well-founded} if and only if 

\begin{itemize}
\item each nonempty subclass of $X$ contains an $R$-minimal element and, 
\item for each $x \in X$, $\lbrace y \mid y\,R\,x \rbrace$ is a set.
\end{itemize}

 The notion of a well-founded relation is a generalization of that of a well-ordering relation: proof by induction and definition by recursion may be carried out over well-founded relations. 
%%%%%
%%%%%
\end{document}
