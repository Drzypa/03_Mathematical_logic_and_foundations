\documentclass[12pt]{article}
\usepackage{pmmeta}
\pmcanonicalname{PropertiesOfSymmetricDifference}
\pmcreated{2013-03-22 14:36:56}
\pmmodified{2013-03-22 14:36:56}
\pmowner{CWoo}{3771}
\pmmodifier{CWoo}{3771}
\pmtitle{properties of symmetric difference}
\pmrecord{14}{36192}
\pmprivacy{1}
\pmauthor{CWoo}{3771}
\pmtype{Derivation}
\pmcomment{trigger rebuild}
\pmclassification{msc}{03E20}

\endmetadata

% this is the default PlanetMath preamble.  as your knowledge
% of TeX increases, you will probably want to edit this, but
% it should be fine as is for beginners.

% almost certainly you want these
\usepackage{amssymb}
\usepackage{amsmath}
\usepackage{amsfonts}
\usepackage{amsthm}

% used for TeXing text within eps files
%\usepackage{psfrag}
% need this for including graphics (\includegraphics)
%\usepackage{graphicx}
% for neatly defining theorems and propositions
%\usepackage{amsthm}
% making logically defined graphics
%%%\usepackage{xypic}

% there are many more packages, add them here as you need them

% define commands here

\newcommand{\mc}{\mathcal}
\newcommand{\mb}{\mathbb}
\newcommand{\mf}{\mathfrak}
\newcommand{\ol}{\overline}
\newcommand{\ra}{\rightarrow}
\newcommand{\la}{\leftarrow}
\newcommand{\La}{\Leftarrow}
\newcommand{\Ra}{\Rightarrow}
\newcommand{\nor}{\vartriangleleft}
\newcommand{\Gal}{\text{Gal}}
\newcommand{\GL}{\text{GL}}
\newcommand{\Z}{\mb{Z}}
\newcommand{\R}{\mb{R}}
\newcommand{\Q}{\mb{Q}}
\newcommand{\C}{\mb{C}}
\newcommand{\<}{\langle}
\renewcommand{\>}{\rangle}
\newcommand{\symd}{\triangle}
\begin{document}
Recall that the symmetric difference of two sets $A,B$ is the set $A\cup B-(A\cap B)$.  In this entry, we list and prove some of the basic properties of $\symd$.

\begin{enumerate}
\item (commutativity of $\symd$) $A\symd B=B\symd A$, because $\cup$ and $\cap$ are commutative.
\item If $A\subseteq B$, then $A\symd B=B-A$, because $A\cup B= B$ and $A\cap B =A$.
\item $A\symd \varnothing = A$, because $\varnothing \subseteq A$, and $A-\varnothing=A$.
\item $A\symd A=\varnothing$, because $A\subseteq A$ and $A-A=\varnothing$.
\item $A\symd B=(A-B)\cup (B-A)$ (hence the name symmetric difference).
\begin{proof}
$A\symd B= (A\cup B)-(A\cap B)= (A\cup B)\cap (A\cap B)'= (A\cup B)\cap (A'\cup B')= ((A\cup B)\cap A')\cup ((A\cup B)\cap B')= (B\cap A')\cup (A\cap B')= (B-A)\cup (A-B)$.
\end{proof}
\item $A'\symd B'=A\symd B$, because $A'\symd B'=(A'-B')\cup (B'-A')=(A'\cap B)\cup (B'\cap A)=(B-A)\cap (A-B)=A\symd B$.
\item (distributivity of $\cap$ over $\symd$) $A\cap (B\symd C)=(A\cap B)\symd (A\cap C)$.
\begin{proof}
$A\cap (B\symd C) = A\cap ((B\cup C)-(B\cap C))$, which is $(A\cap (B\cup C))-(A\cap (B\cap C))$, one of the properties of set difference (see proof \PMlinkname{here}{PropertiesOfSetDifference}).  This in turns is equal to $((A\cap B)\cup (A\cap C))-((A\cap B)\cap (A\cap C)) = (A\cap B)\symd (A\cap C)$.
\end{proof}
\item (associativity of $\symd$) $(A \symd B) \symd C = A \symd (B \symd C)$.
\begin{proof}
Let $U$ be a set containing $A,B,C$ as subsets (take $U=A\cup B\cup C$ if necessary).  For a given $B$, let $f:P(U)\times P(U)\to P(U)$ be a function defined by $f(A,C)= (A\symd B)\symd C$.  Associativity of $\symd$ is then then same as showing that $f(A,C)=f(C,A)$, since $A\symd (B\symd C)= (B\symd C)\symd A = (C\symd B)\symd A$.

By expanding $f(A,C)$, we have
\begin{eqnarray*} (A\symd B)\symd C &=& ((A\symd B)-C)\cup (C-(A\symd B)) \\ 
&=& (((A-B)\cup (B-A))\cap C')\cup (C- ((A\cup B)-(A\cap B))) \\
&=& (((A\cap B')\cup (B\cap A'))\cap C')\cup ((C\cap A\cap B)\cup (C-(A\cup B)) \\
&=& ((A\cap B'\cap C')\cup (B\cap A'\cap C'))\cup ((C\cap A\cap B)\cup (C\cap A'\cap B')) \\
&=& (B\cap A'\cap C')\cup (B\cap A\cap C)\cup (B'\cap A\cap C')\cup (B'\cap A'\cap C).
\end{eqnarray*}
It is now easy to see that the last expression does not change if one exchanges $A$ and $C$.  Hence, $f(A,C)=f(C,A)$ and this shows that $\symd$ is associative.
\end{proof}
\end{enumerate}

\textbf{Remark}.  All of the properties of $\symd$ on sets can be generalized to \PMlinkname{$\symd$}{DerivedBooleanOperations} on Boolean algebras.
%%%%%
%%%%%
\end{document}
