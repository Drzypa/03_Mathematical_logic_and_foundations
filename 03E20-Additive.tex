\documentclass[12pt]{article}
\usepackage{pmmeta}
\pmcanonicalname{Additive}
\pmcreated{2013-03-22 13:00:58}
\pmmodified{2013-03-22 13:00:58}
\pmowner{Andrea Ambrosio}{7332}
\pmmodifier{Andrea Ambrosio}{7332}
\pmtitle{additive}
\pmrecord{10}{33400}
\pmprivacy{1}
\pmauthor{Andrea Ambrosio}{7332}
\pmtype{Definition}
\pmcomment{trigger rebuild}
\pmclassification{msc}{03E20}
\pmsynonym{additivity}{Additive}
\pmdefines{countable additivity}
\pmdefines{countably additive}
\pmdefines{$\sigma$-additive}
\pmdefines{sigma-additive}

% this is the default PlanetMath preamble.  as your knowledge
% of TeX increases, you will probably want to edit this, but
% it should be fine as is for beginners.

% almost certainly you want these
\usepackage{amssymb}
\usepackage{amsmath}
\usepackage{amsfonts}

% used for TeXing text within eps files
%\usepackage{psfrag}
% need this for including graphics (\includegraphics)
%\usepackage{graphicx}
% for neatly defining theorems and propositions
%\usepackage{amsthm}
% making logically defined graphics
%%%\usepackage{xypic} 

% there are many more packages, add them here as you need them

% define commands here
\begin{document}
Let $\phi$ be some positive-valued set function defined on an algebra of sets $\mathcal{A}$.  We say that $\phi$ is \emph{additive} if, whenever $A$ and $B$ are disjoint sets in $\mathcal{A}$, we have
$$\phi(A \cup B) = \phi(A) + \phi(B) .$$

Given any sequence $\langle A_i \rangle$ of disjoint sets in A and whose union is also in A, if we have
$$\phi\left( \bigcup A_i \right) = \sum \phi(A_i)$$
we say that $\phi$ is \emph{countably additive} or \emph{$\sigma$-additive}.

Useful properties of an additive set function $\phi$ include the following:
\begin{enumerate}
\item $\phi(\emptyset) = 0$.
\item If $A \subseteq B$, then $\phi(A) \leq \phi(B)$.
\item If $A \subseteq B$, then $\phi(B \setminus A) = \phi(B) - \phi(A)$.
\item Given $A$ and $B$, $\phi(A \cup B) + \phi(A \cap B) = \phi(A) + \phi(B)$.  
\end{enumerate}
%%%%%
%%%%%
\end{document}
