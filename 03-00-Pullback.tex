\documentclass[12pt]{article}
\usepackage{pmmeta}
\pmcanonicalname{Pullback}
\pmcreated{2013-03-22 13:50:04}
\pmmodified{2013-03-22 13:50:04}
\pmowner{matte}{1858}
\pmmodifier{matte}{1858}
\pmtitle{pullback}
\pmrecord{14}{34569}
\pmprivacy{1}
\pmauthor{matte}{1858}
\pmtype{Definition}
\pmcomment{trigger rebuild}
\pmclassification{msc}{03-00}
\pmrelated{InclusionMapping}
\pmrelated{RestrictionOfAFunction}
\pmrelated{PullbackOfAKForm}

% this is the default PlanetMath preamble.  as your knowledge
% of TeX increases, you will probably want to edit this, but
% it should be fine as is for beginners.

% almost certainly you want these
\usepackage{amssymb}
\usepackage{amsmath}
\usepackage{amsfonts}

% used for TeXing text within eps files
%\usepackage{psfrag}
% need this for including graphics (\includegraphics)
%\usepackage{graphicx}
% for neatly defining theorems and propositions
%\usepackage{amsthm}
% making logically defined graphics
%%%\usepackage{xypic}
%%\usepackage{xypic}
% there are many more packages, add them here as you need them

% define commands here

\newcommand{\sR}[0]{\mathbb{R}}
\newcommand{\sC}[0]{\mathbb{C}}
\newcommand{\sN}[0]{\mathbb{N}}
\newcommand{\sZ}[0]{\mathbb{Z}}

% The below lines should work as the command
% \renewcommand{\bibname}{References}
% without creating havoc when rendering an entry in 
% the page-image mode.
\makeatletter
\@ifundefined{bibname}{}{\renewcommand{\bibname}{References}}
\makeatother

\newcommand*{\norm}[1]{\lVert #1 \rVert}
\newcommand*{\abs}[1]{| #1 |}
\begin{document}
{\bf Definition}
Suppose $X,Y,Z$ are sets, and we have maps
\begin{eqnarray*}
 f\colon Y&\to& Z, \\
 \Phi\colon X&\to& Y.
\end{eqnarray*}
Then the {\bf pullback} of $f$ under $\Phi$ is the mapping
\begin{eqnarray*}
\Phi^\ast f\colon  X &\to& Z, \\
             x&\mapsto& (f\circ\Phi)(x).
\end{eqnarray*}

Let us denote by $M(X,Y)$ the set of all mappings $f\colon X\to Y$.
We then see that $\Phi^\ast$ is a mapping $M(Y,Z)\to M(X,Z)$.
In other words, $\Phi^\ast$ pulls back the set where $f$ is 
defined on from $Y$ to $X$. This is illustrated in the below diagram.
$$
\xymatrix{
 X \ar[r]^\Phi\ar[dr]_{\Phi^\ast f} & Y \ar[d]_{f} \\
  &  Z 
 }
$$

\subsubsection{Properties}
\begin{enumerate}
\item For any set $X$,
 $(\operatorname{id}_X)^\ast = \operatorname{id}_{M(X,X)}$.
\item Suppose we have maps
\begin{eqnarray*}
 \Phi\colon X&\to& Y, \\
 \Psi\colon Y&\to& Z
\end{eqnarray*}
between sets $X,Y,Z$. Then
$$ (\Psi\circ \Phi)^\ast = \Phi^\ast \circ \Psi^\ast.$$
\item If $\Phi\colon X\to Y$ is a bijection, then
$\Phi^\ast$ is a bijection and
$$
  \big(\Phi^\ast\big)^{-1} = \big(\Phi^{-1}\big)^\ast.
$$
\item Suppose $X,Y$ are sets with $X\subset Y$. 
Then we have the inclusion map $\iota:X\hookrightarrow Y$, and 
for any $f\colon Y\to Z$, we have 
$$ 
  \iota^\ast f = f|_X,
$$
where $f|_X$ is the \PMlinkname{restriction}{RestrictionOfAFunction} of $f$ to $X$. 
\end{enumerate}
%%%%%
%%%%%
\end{document}
