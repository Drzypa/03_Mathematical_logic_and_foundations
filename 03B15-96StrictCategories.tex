\documentclass[12pt]{article}
\usepackage{pmmeta}
\pmcanonicalname{96StrictCategories}
\pmcreated{2013-11-06 16:39:10}
\pmmodified{2013-11-06 16:39:10}
\pmowner{PMBookProject}{1000683}
\pmmodifier{PMBookProject}{1000683}
\pmtitle{9.6 Strict categories}
\pmrecord{1}{}
\pmprivacy{1}
\pmauthor{PMBookProject}{1000683}
\pmtype{Feature}
\pmclassification{msc}{03B15}

\usepackage{xspace}
\usepackage{amssyb}
\usepackage{amsmath}
\usepackage{amsfonts}
\usepackage{amsthm}
\newcommand{\define}[1]{\textbf{#1}}
\newcommand{\idtoiso}{\ensuremath{\mathsf{idtoiso}}\xspace}
\newcommand{\indexdef}[1]{\index{#1|defstyle}}   
\newcommand{\indexsee}[2]{\index{#1|see{#2}}}    
\newcommand{\trans}[2]{\ensuremath{{#1}_{*}\mathopen{}\left({#2}\right)\mathclose{}}\xspace}
\newcounter{mathcount}
\setcounter{mathcount}{1}
\newtheorem{predefn}{Definition}
\newenvironment{defn}{\begin{predefn}}{\end{predefn}\addtocounter{mathcount}{1}}
\renewcommand{\thepredefn}{9.6.\arabic{mathcount}}
\newtheorem{preeg}{Example}
\newenvironment{eg}{\begin{preeg}}{\end{preeg}\addtocounter{mathcount}{1}}
\renewcommand{\thepreeg}{9.6.\arabic{mathcount}}
\let\autoref\cref

\begin{document}

\index{bargaining|(}%

\begin{defn}
  A \define{strict category}
  \indexdef{category!strict}%
  \indexdef{strict!category}%
  is a precategory whose type of objects is a set.
\end{defn}

In accordance with the mathematical red herring principle,\index{red herring principle} a strict category is not necessarily a category.
In fact, a category is a strict category precisely when it is gaunt (\autoref{ct:gaunt}).
\index{gaunt category}%
\index{category!gaunt}%
Most of the time, category theory is about categories, not strict ones, but sometimes one wants to consider strict categories.
The main advantage of this is that strict categories have a stricter notion of ``sameness'' than equivalence, namely isomorphism (or equivalently, by \autoref{ct:cat-eq-iso}, equality).

Here is one origin of strict categories.

\begin{eg}
  Let $A$ be a precategory and $x:A$ an object.
  Then there is a precategory $\mathsf{mono}(A,x)$ as follows:
  \index{monomorphism}
  \indexsee{mono}{monomorphism}
  \indexsee{monic}{monomorphism}
  \begin{itemize}
  \item Its objects consist of an object $y:A$ and a monomorphism $m:\hom_A(y,x)$.
    (As usual, $m:\hom_A(y,x)$ is a \define{monomorphism} (or is \define{monic}) if $(m\circ f = m\circ g) \Rightarrow (f=g)$.)
  \item Its morphisms from $(y,m)$ to $(z,n)$ are arbitrary morphisms from $y$ to $z$ in $A$ (not necessarily respecting $m$ and $n$).
  \end{itemize}
  An equality $(y,m)=(z,n)$ of objects in $\mathsf{mono}(A,x)$ consists of an equality $p:y=z$ and an equality $\trans{p}{m}=n$, which by \autoref{ct:idtoiso-trans} is equivalently an equality $m=n\circ \idtoiso(p)$.
  Since hom-sets are sets, the type of such equalities is a mere proposition.
  But since $m$ and $n$ are monomorphisms, the type of morphisms $f$ such that $m = n\circ f$ is also a mere proposition.
  Thus, if $A$ is a category, then $(y,m)=(z,n)$ is a mere proposition, and hence $\mathsf{mono}(A,x)$ is a strict category.
\end{eg}

This example can be dualized, and generalized in various ways.
Here is an interesting application of strict categories.

\begin{eg}\label{ct:galois}
  Let $E/F$ be a finite Galois extension
  \index{Galois!extension}%
  of fields, and $G$ its Galois group.
  \index{Galois!group}%
  Then there is a strict category whose objects are intermediate fields $F\subseteq K\subseteq E$, and whose morphisms are field homomorphisms\index{homomorphism!field} which fix $F$ pointwise (but need not commute with the inclusions into $E$).
  There is another strict category whose objects are subgroups $H\subseteq G$, and whose morphisms are morphisms of $G$-sets $G/H \to G/K$.
  The fundamental theorem of Galois theory
  \index{fundamental!theorem of Galois theory}%
  says that these two precategories are isomorphic (not merely equivalent).
\end{eg}

\index{bargaining|)}%


\end{document}
