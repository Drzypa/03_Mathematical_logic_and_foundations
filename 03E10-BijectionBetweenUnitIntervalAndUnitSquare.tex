\documentclass[12pt]{article}
\usepackage{pmmeta}
\pmcanonicalname{BijectionBetweenUnitIntervalAndUnitSquare}
\pmcreated{2015-02-03 21:45:39}
\pmmodified{2015-02-03 21:45:39}
\pmowner{pahio}{2872}
\pmmodifier{pahio}{2872}
\pmtitle{bijection between unit interval and unit square}
\pmrecord{18}{42575}
\pmprivacy{1}
\pmauthor{pahio}{2872}
\pmtype{Result}
\pmcomment{trigger rebuild}
\pmclassification{msc}{03E10}
\pmrelated{JuliusKonig}
\pmrelated{BijectionBetweenClosedAndOpenInterval}

\endmetadata

% this is the default PlanetMath preamble.  as your knowledge
% of TeX increases, you will probably want to edit this, but
% it should be fine as is for beginners.

% almost certainly you want these
\usepackage{amssymb}
\usepackage{amsmath}
\usepackage{amsfonts}

% used for TeXing text within eps files
%\usepackage{psfrag}
% need this for including graphics (\includegraphics)
%\usepackage{graphicx}
% for neatly defining theorems and propositions
 \usepackage{amsthm}
% making logically defined graphics
%%%\usepackage{xypic}

% there are many more packages, add them here as you need them

% define commands here

\theoremstyle{definition}
\newtheorem*{thmplain}{Theorem}

\begin{document}
The real numbers in the open unit interval\, $I \,=\,(0,\,1)$\, can be uniquely represented by their decimal expansions, when these must not end in an infinite string of 9's.\, Correspondingly, the elements of the open unit square $I\!\times\!I$ are represented by the pairs of such decimal expansions.

Let 
$$P \;:=\; (0.x_1x_2x_3\ldots,\,0.y_1y_2y_3\ldots)$$ 
be such a pair representing an arbitrary point in $I\!\times\!I$ and let\, 
$$p \;:=\; 0.x_1y_1x_2y_2x_3y_3\ldots$$ 
Then it's apparent that
\begin{align}
P \mapsto p
\end{align}
is an injective mapping from $I\!\times\!I$ to $I$.\, Thus
$$|I\!\times\!I| \;\le\; |I|.$$
But since $I\!\times\!I$ contains more than one horizontal open segment equally long as $I$ (and accordingly there is a natural injection from $I$ to $I\!\times\!I$), we must have also
$$|I\!\times\!I| \;\ge\; |I|.$$
The conclusion is that
$$|I\!\times\!I| \;=\; |I|,$$
i.e. that the sets $I\!\times\!I$ and $I$ have equal cardinalities, 
and the Schr\"oder$-$Bernstein theorem even garantees a bijection between the sets.\\


\textbf{Remark 1.}\, Georg Cantor utilised continued fractions for constructing such a bijection between the unit interval and the unit square; cf. e.g. \PMlinkexternal{this}{http://www.maa.org/pubs/AMM-March11_Cantor.pdf} MAA article.\\

\textbf{Remark 2.}\, Since the mapping\, $g\!:\,I \to \mathbb{R}$\, defined by
$$g(x) \;=\; \tan\left(\pi{x}-\frac{\pi}{2}\right)$$
is bijective, we can conclude that the sets $\mathbb{R}$ and $\mathbb{R}\!\times\!\mathbb{R}$, i.e. the set of the points of a line and the set of the points of a plane, have the same cardinalities.\, This common cardinality is $2^{\aleph_0}$.



%%%%%
%%%%%
\end{document}
