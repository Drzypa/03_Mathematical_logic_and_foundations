\documentclass[12pt]{article}
\usepackage{pmmeta}
\pmcanonicalname{Diamond}
\pmcreated{2013-03-22 12:53:49}
\pmmodified{2013-03-22 12:53:49}
\pmowner{Henry}{455}
\pmmodifier{Henry}{455}
\pmtitle{$\Diamond$}
\pmrecord{8}{33245}
\pmprivacy{1}
\pmauthor{Henry}{455}
\pmtype{Definition}
\pmcomment{trigger rebuild}
\pmclassification{msc}{03E65}
\pmsynonym{diamond}{Diamond}
\pmrelated{Clubsuit}
\pmrelated{DiamondIsEquivalentToClubsuitAndContinuumHypothesis}
\pmrelated{ProofOfDiamondIsEquivalentToClubsuitAndContinuumHypothesis}
\pmrelated{CombinatorialPrinciple}
\pmdefines{$\Diamond_S$}

\endmetadata

% this is the default PlanetMath preamble.  as your knowledge
% of TeX increases, you will probably want to edit this, but
% it should be fine as is for beginners.

% almost certainly you want these
\usepackage{amssymb}
\usepackage{amsmath}
\usepackage{amsfonts}

% used for TeXing text within eps files
%\usepackage{psfrag}
% need this for including graphics (\includegraphics)
%\usepackage{graphicx}
% for neatly defining theorems and propositions
\usepackage{amsthm}
% making logically defined graphics
%%%\usepackage{xypic}

% there are many more packages, add them here as you need them

% define commands here
%\PMlinkescapeword{theory}
\newtheorem{defn}{Definition}
\begin{document}
\begin{defn}
Let $S\subseteq\kappa$ be a stationary set.  Then the combinatorial principle $\Diamond_S$ holds if and only if there is a sequence $\langle A_{\alpha}\rangle_{\alpha\in S}$ such that each $A_{\alpha}\subseteq\alpha$ and for any $A\subseteq\kappa$, $\{\alpha\in S\mid A\cap\alpha=A_{\alpha}\}$ is stationary.
\end{defn}

%$\Diamond_S$ is a combinatoric principle regarding a stationary set %$S\subseteq\kappa$.  It holds when there is a sequence $\langle %A_\alpha\rangle_{\alpha\in S}$ such that each $A_\alpha\subseteq\alpha$ and for %any $A\subseteq\kappa$, $\{\alpha\in S\mid A\cap\alpha=A_\alpha\}$ is stationary.

To get some sense of what this means, observe that for any $\lambda<\kappa$, $\{\lambda\}\subseteq\kappa$, so the set of $A_\alpha=\{\lambda\}$ is stationary (in $\kappa$).  More strongly, suppose $\kappa>\lambda$.  Then any subset of $T\subset\lambda$ is bounded in $\kappa$ so $A_\alpha=T$ on a stationary set.  Since $|S|=\kappa$, it follows that $2^\lambda\leq\kappa$.  Hence $\Diamond_{\aleph_1}$, the most common form (often written as just $\Diamond$), implies CH.

C. Akemann and N. Weaver used $\Diamond$ to construct a $C^*$-algebra serving as a counterexample to Naimark's problem.

\begin{thebibliography}{9}
\bibitem{AW}
Akemann, C., and N. Weaver, {\it Consistency of a counterexample to Naimark's problem}.  Preprint available on the arXiv at \PMlinkexternal{http://arxiv.org/abs/math.OA/0312135}{http://arxiv.org/abs/math.OA/0312135}.
\end{thebibliography}
%%%%%
%%%%%
\end{document}
