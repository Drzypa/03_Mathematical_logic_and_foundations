\documentclass[12pt]{article}
\usepackage{pmmeta}
\pmcanonicalname{Structure}
\pmcreated{2013-05-20 18:26:21}
\pmmodified{2013-05-20 18:26:21}
\pmowner{CWoo}{3771}
\pmmodifier{unlord}{1}
\pmtitle{structure}
\pmrecord{23}{33017}
\pmprivacy{1}
\pmauthor{CWoo}{1}
\pmtype{Definition}
\pmcomment{trigger rebuild}
\pmclassification{msc}{03C07}
\pmrelated{Substructure}
\pmrelated{AlgebraicStructure}
\pmrelated{Model}
\pmrelated{RelationalSystem}
\pmdefines{structure}
\pmdefines{interpretation}

% this is the default PlanetMath preamble.  as your knowledge
% of TeX increases, you will probably want to edit this, but
% it should be fine as is for beginners.

% almost certainly you want these
\usepackage{amssymb}
\usepackage{amsmath}
\usepackage{amsfonts}

% used for TeXing text within eps files
%\usepackage{psfrag}
% need this for including graphics (\includegraphics)
%\usepackage{graphicx}
% for neatly defining theorems and propositions
%\usepackage{amsthm}
% making logically defined graphics
%%\usepackage{xypic}

% there are many more packages, add them here as you need them

% define commands here
\newcommand{\br}{[\![}
\newcommand{\rb}{]\!]}
\newcommand{\oq}{\text{``}}
\newcommand{\cq}{\text{''}}


\newcommand{\im}{\mathbf{Im}}
\newcommand{\dom}{\mathbf{Dom}}


\newcommand{\Or}{\vee}
\newcommand{\Implies}{\Rightarrow}
\newcommand{\Iff}{\Leftrightarrow}
\newcommand{\proves}{\vdash}
\renewcommand{\And}{\wedge}
\newcommand{\Sup}{\bigwedge}
\newcommand{\Inf}{\bigvee}
\newcommand{\Z}{\mathbb{Z}}
\newcommand{\F}{\mathbb{F}}
\newcommand{\Q}{\mathbb{Q}}
\newcommand{\R}{\mathbb{R}}
\newcommand{\C}{\mathbb{C}}
\newcommand{\Nat}{\mathbb{N}}
\newcommand{\M}{\mathfrak{M}}
\newcommand{\N}{\mathfrak{N}}
\newcommand{\A}{\mathfrak{A}}
\newcommand{\B}{\mathfrak{B}}
\newcommand{\K}{\mathfrak{K}}
\newcommand{\G}{\mathbb{G}}
\newcommand{\Def}{\overset{\operatorname{def}}{:=}}



\newcommand{\spec}{\text{{\bf Spec}}}
\newcommand{\stab}{\text{{\bf Stab}}}
\newcommand{\ann}{\text{{\bf Ann}}}
\newcommand{\irr}{\text{{\bf Irr}}}
\newcommand{\qt}{\text{{\bf Qt}}}
\newcommand{\st}{\mathcal{Qt}}
\newcommand{\ro}{\mathbf{r.o.}}


\newcommand{\Endo}{\text{{\bf End}}}
\newcommand{\mat}{\text{{\bf Mat}}}
\newcommand{\der}{\text{{\bf Der}}}
\newcommand{\rad}{\text{{\bf Rad}}}
\newcommand{\trd}{\text{{\bf tr.d.}}}
\newcommand{\cl}{\text{{\bf acl}}}
\newcommand{\Int}{\text{{\bf int}}}
\newcommand{\V}{\mathbb{V}}
\newcommand{\D}{\mathbf{D}}

\newcommand{\del}{\partial}
\renewcommand{\O}{\mathcal{O}}
\newcommand{\aut}{\mathbf{Aut}}
\newcommand{\height}{\text{\bf Height}}
\newcommand{\coheight}{\text{\bf Co-height}}

\newcommand{\lcm}{\operatorname{lcm}}

\newcommand{\Gal}{\operatorname{Gal}}
\newcommand{\x}{\mathbf{x}}
\newcommand{\y}{\mathbf{y}}
\newcommand{\inner}[2]{\langle #1|#2\rangle}
\renewcommand{\r}{{r}}
\renewcommand{\t}{{t}}
\newcommand{\val}{\operatorname{Val}}
\newcommand{\J}{\mathcal{J}}
\newcommand{\restr}{\upharpoonright}
\newcommand{\Matrix}[4]{\left(\begin{array}{cc} #1 & #2 \\ #3 & #4 
\end{array}\right)}
\begin{document}
Let $\tau$ be a signature.
A \emph{$\tau$-structure} $\mathcal{A}$ comprises of a set $A$, called the \emph{\PMlinkescapetext{universe}} (or \emph{underlying set} or \emph{\PMlinkescapetext{domain}}) of $\mathcal{A}$, and an \emph{interpretation} of the symbols of $\tau$ as follows:

\begin{itemize}
\item for each constant symbol $c\in\tau$, an
element $c^A\in A$;
\item for each $n$-ary function symbol $f\in\tau$,
a function (or operation) $f^A:A^n\rightarrow A$;
\item for each $n$-ary relation symbol $R\in\tau$, 
a $n$-ary relation $R^A$ on $A$.
\end{itemize}

Some authors require that $A$ be non-empty.

If $\mathcal{A}$ is a structure, then the \emph{cardinality} (or \emph{power}) of $\mathcal{A}$, $|\mathcal{A}|$,  is the cardinality of its \PMlinkescapetext{universe} $A$.

Examples of structures abound in mathematics.  Here are some of them:
\begin{enumerate}
\item A set is a structure, with no constants, no functions, and no relations on it.
\item A partially ordered set is a structure, with one binary relation call partial order defined on the underlying set.
\item A group is a structure, with one binary operation called multiplication, one unary operation called inverse, and one constant called the multiplicative identity.
\item A vector space is a structure, with one binary operation called addition, unary operations called scalar multiplications, one for each element of the underlying set, and one constant $0$, the additive identity.
\item A partially ordered group is a structure like a group, but with the addition of a partial order on the underlying set.
\end{enumerate}

If $\tau$ contains only relation symbols, then a $\tau$-structure is called a relational structure.  If $\tau$ contains only function symbols, then a $\tau$-structure is called an algebraic structure.  In the examples above, $2$ is a relation structure, while $3,4$ are algebraic structures.
%%%%%
%%%%%.
\end{document}
