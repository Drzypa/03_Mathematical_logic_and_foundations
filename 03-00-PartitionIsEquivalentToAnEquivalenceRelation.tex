\documentclass[12pt]{article}
\usepackage{pmmeta}
\pmcanonicalname{PartitionIsEquivalentToAnEquivalenceRelation}
\pmcreated{2013-03-22 16:44:55}
\pmmodified{2013-03-22 16:44:55}
\pmowner{CWoo}{3771}
\pmmodifier{CWoo}{3771}
\pmtitle{partition is equivalent to an equivalence relation}
\pmrecord{7}{38973}
\pmprivacy{1}
\pmauthor{CWoo}{3771}
\pmtype{Derivation}
\pmcomment{trigger rebuild}
\pmclassification{msc}{03-00}

\endmetadata

\usepackage{amssymb,amscd}
\usepackage{amsmath}
\usepackage{amsfonts}

% used for TeXing text within eps files
%\usepackage{psfrag}
% need this for including graphics (\includegraphics)
%\usepackage{graphicx}
% for neatly defining theorems and propositions
\usepackage{amsthm}
% making logically defined graphics
%%\usepackage{xypic}
\usepackage{pst-plot}
\usepackage{psfrag}

% define commands here
\newtheorem{prop}{Proposition}
\newtheorem{cor}{Corollary}
\newtheorem{thm}{Theorem}
\newtheorem{ex}{Example}
\newcommand{\real}{\mathbb{R}}
\begin{document}
\begin{prop}  There is a one-to-one correspondence between the set $\operatorname{Part}(S)$ of partitions of $S$ and the set $\operatorname{Equiv}(S)$ of equivalence relations on $S$.
\end{prop}

\begin{proof}
Let $S$ be a set.

Suppose $P=\lbrace P_i\mid i\in I\rbrace$ is a partition of $S$.  Form $E=\bigcup \lbrace P_i\times P_i\mid i\in I\rbrace$.  Given any $a\in S$, $a\in P_i$ for some $i\in I$.  Then $(a,a)\in P_i\times P_i\in E$.  If $(a,b)\in E$, then $(a,b)\in P_i\times P_i$ for some $i\in I$, so $a,b\in P_i$, whence $(b,a)\in P_i\times P_i\subseteq E$.  Finally, suppose $(a,b),(b,c)\in E$.  Then $(a,b)\in P_i\times P_i$ and $(b,c)\in P_j\times P_j$, or $a,b\in P_i$ and $b,c\in P_j$.  Since $b\in P_i\cap P_j$, $P_i=P_j$ since $P$ is a partition.  So $a,c\in P_i=P_j$, or $(a,c)\in P_i\times P_i\subseteq E$.

Conversely, suppose $E$ is an equivalence relation on $S$.  For each $a\in S$, define $$[a]=\lbrace b\in S\mid (a,b)\in E\rbrace.$$  Then each $a\in [a]$ since $E$ is reflexive.  If $b\in [a]$, then $(a,b)\in E$ or $(b,a)\in E$ as $E$ is symmetric.  So $a\in [b]$ as well.  Next, pick any $c\in [b]$.  Then $(b,c)\in E$.  But $(a,b)\in E$.  So $(a,c)\in E$ since $E$ is transitive.  Therefore $c\in [a]$, which implies $[b]\subseteq [a]$.  Collect all the information we have so far, this implies $[a]=[b]$.  Therefore $P=\lbrace [a]\mid a\in S\rbrace$ forms a partitition of $S$ ($P$ is being interpreted as a set, not a multiset).  In fact, $E=\bigcup \lbrace [a]\times [a]\mid a\in S\rbrace$.
\end{proof}

From the above proof, we also have

\begin{cor}  For every partition $P=\lbrace P_i\mid i\in I\rbrace$ of $S$, $E= \bigcup \lbrace P_i^2\mid i\in I\rbrace$ is an equivalence relation.  Conversely, every equivalence relation on $S$ can be formed this way.
\end{cor}
%%%%%
%%%%%
\end{document}
