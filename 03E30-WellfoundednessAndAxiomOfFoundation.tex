\documentclass[12pt]{article}
\usepackage{pmmeta}
\pmcanonicalname{WellfoundednessAndAxiomOfFoundation}
\pmcreated{2013-03-22 17:25:34}
\pmmodified{2013-03-22 17:25:34}
\pmowner{CWoo}{3771}
\pmmodifier{CWoo}{3771}
\pmtitle{well-foundedness and axiom of foundation}
\pmrecord{6}{39800}
\pmprivacy{1}
\pmauthor{CWoo}{3771}
\pmtype{Theorem}
\pmcomment{trigger rebuild}
\pmclassification{msc}{03E30}

\endmetadata

\usepackage{amssymb,amscd}
\usepackage{amsmath}
\usepackage{amsfonts}
\usepackage{mathrsfs}

% used for TeXing text within eps files
%\usepackage{psfrag}
% need this for including graphics (\includegraphics)
%\usepackage{graphicx}
% for neatly defining theorems and propositions
\usepackage{amsthm}
% making logically defined graphics
%%\usepackage{xypic}
\usepackage{pst-plot}
\usepackage{psfrag}

% define commands here
\newtheorem{prop}{Proposition}
\newtheorem{thm}{Theorem}
\newtheorem{ex}{Example}
\newcommand{\real}{\mathbb{R}}
\newcommand{\pdiff}[2]{\frac{\partial #1}{\partial #2}}
\newcommand{\mpdiff}[3]{\frac{\partial^#1 #2}{\partial #3^#1}}
\begin{document}
Recall that a relation $R$ on a class $C$ is well-founded if 
\begin{enumerate}
\item
For any $x\in C$, the collection $\lbrace y\in C\mid yRx\rbrace$ is a set, and
\item
for any non-empty $B\subseteq C$, there is an element $z\in B$ such that if $yRz$, then $y\notin B$.  
\end{enumerate}
$z$ is called an $R$-minimal element of $B$.  It is clear that the membership relation $\in$ in the class of all sets satisfies the first condition above.

\begin{thm} Given ZF, $\in$ is a well-founded relation iff the Axiom of Foundation (AF) is true. \end{thm}

We will prove this using one of the equivalent versions of AF: for every non-empty set $A$, there is an $x\in A$ such that $x\cap A=\varnothing$.

\begin{proof}
Suppose $\in$ is well-founded and $A$ a non-empty set.  We want to find $x\in A$ such that $x\cap A=\varnothing$.  Since $\in$ is well-founded, there is a $\in$-minimal set $x$ such that $x\in A$.  Since no set $y$ such that $y\in x$ and $y\in A$ (otherwise $x$ would not be $\in$-minimal), we have that $x\cap A=\varnothing$.

Conversely, suppose that AF is true.  Let $A$ be any non-empty set.  We want to find a $\in$-minimal element in $A$.  Let $x\in A$ such that $x\cap A=\varnothing$.  Then $x$ is $\in$-minimal in $A$, for otherwise there is $y\in A$ such that $y\in x$, which implies $y\in x\cap A=\varnothing$, a contradiction.
\end{proof}
%%%%%
%%%%%
\end{document}
