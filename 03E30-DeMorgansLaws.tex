\documentclass[12pt]{article}
\usepackage{pmmeta}
\pmcanonicalname{DeMorgansLaws}
\pmcreated{2013-03-22 12:24:49}
\pmmodified{2013-03-22 12:24:49}
\pmowner{mathcam}{2727}
\pmmodifier{mathcam}{2727}
\pmtitle{de Morgan's laws}
\pmrecord{15}{32308}
\pmprivacy{1}
\pmauthor{mathcam}{2727}
\pmtype{Theorem}
\pmcomment{trigger rebuild}
\pmclassification{msc}{03E30}
\pmrelated{Set}
\pmrelated{Complement}

% this is the default PlanetMath preamble.  as your knowledge
% of TeX increases, you will probably want to edit this, but
% it should be fine as is for beginners.

% almost certainly you want these
\usepackage{amssymb}
\usepackage{amsmath}
\usepackage{amsfonts}

% used for TeXing text within eps files
%\usepackage{psfrag}
% need this for including graphics (\includegraphics)
%\usepackage{graphicx}
% for neatly defining theorems and propositions
%\usepackage{amsthm}
% making logically defined graphics
%%%\usepackage{xypic} 

% there are many more packages, add them here as you need them

% define commands here
\begin{document}
In set theory, \emph{de Morgan's laws}
relate the three basic set operations to each other; 
the union, the intersection, and the complement. 
de Morgan's laws are named after the 
Indian-born British mathematician and logician
Augustus De Morgan (1806-1871) \cite{wikidemorgan}.


If $A$ and $B$  are subsets of a set $X$, de Morgan's laws state that 
\begin{eqnarray*}
(A \cup B)^\complement &=& A^\complement \cap B^\complement, \\ 
(A \cap B)^\complement &=& A^\complement \cup B^\complement.
\end{eqnarray*}
Here, $\cup$ denotes the union, $\cap$ denotes the intersection, 
and $A^\complement$ denotes the set complement of $A$ in $X$, i.e., 
$A^\complement= X\setminus A$.

Above, de Morgan's laws are written for two sets. 
In this form, they are intuitively quite clear. 
For instance, the first claim states  that an element
that is not in $A\cup B$ is not in $A$
and not in $B$. It also states that an elements not in $A$
and not in $B$ is not in  $A\cup B$. 

For an arbitrary collection of subsets, de Morgan's laws are
as follows:

{\bf Theorem.}
 Let $X$ be a set with subsets $A_i \subset X$ for $i\in I$, where
 $I$ is an arbitrary index-set. In other words, $I$ can be finite,
 countable, or uncountable. Then
 \begin{eqnarray*}
 \Big( \bigcup_{i\in I} A_i \Big)^\complement &=& \bigcap_{i\in I} A_i^\complement, \\
 \Big( \bigcap_{i\in I} A_i \Big)^\complement &=& \bigcup_{i\in I} A_i^\complement.
 \end{eqnarray*}
(\PMlinkname{proof}{DeMorgansLawsProof})

{\bf de Morgan's laws in a \PMlinkescapetext{Boolean algebra} }\\
For Boolean variables $x$ and $y$ in a Boolean algebra, 
de Morgan's laws state that
\begin{eqnarray*}
(x \land y)' &=& x' \lor y', \\
(x \lor y)' &=& x' \land y'.
\end{eqnarray*}
Not surprisingly, de Morgan's laws form an indispensable tool
when simplifying digital circuits involving and, or, and not
gates \cite{mano}. 

\begin{thebibliography}{9}
\bibitem{wikidemorgan}
Wikipedia's \PMlinkexternal{entry on de Morgan}{http://www.wikipedia.org/wiki/Augustus_De_Morgan}, 4/2003.
\bibitem{mano}
M.M. Mano, 
\emph{Computer Engineering: Hardware Design},
Prentice Hall, 1988.
\end{thebibliography}
%%%%%
%%%%%
\end{document}
