\documentclass[12pt]{article}
\usepackage{pmmeta}
\pmcanonicalname{NormalModalLogic}
\pmcreated{2013-03-22 19:33:38}
\pmmodified{2013-03-22 19:33:38}
\pmowner{CWoo}{3771}
\pmmodifier{CWoo}{3771}
\pmtitle{normal modal logic}
\pmrecord{16}{42546}
\pmprivacy{1}
\pmauthor{CWoo}{3771}
\pmtype{Definition}
\pmcomment{trigger rebuild}
\pmclassification{msc}{03B45}
\pmrelated{DisjunctionProperty}
\pmdefines{law of distribution}
\pmdefines{necessitation}
\pmdefines{K}
\pmdefines{logic}

\usepackage{amssymb,amscd}
\usepackage{amsmath}
\usepackage{amsfonts}
\usepackage{mathrsfs}

% used for TeXing text within eps files
%\usepackage{psfrag}
% need this for including graphics (\includegraphics)
%\usepackage{graphicx}
% for neatly defining theorems and propositions
\usepackage{amsthm}
% making logically defined graphics
%%\usepackage{xypic}
\usepackage{pst-plot}
\usepackage{tabls}
\usepackage{multicol}

% define commands here
\newcommand*{\abs}[1]{\left\lvert #1\right\rvert}
\newtheorem{prop}{Proposition}
\newtheorem{thm}{Theorem}
\newtheorem{ex}{Example}
\newcommand{\real}{\mathbb{R}}
\newcommand{\pdiff}[2]{\frac{\partial #1}{\partial #2}}
\newcommand{\mpdiff}[3]{\frac{\partial^#1 #2}{\partial #3^#1}}

\begin{document}
The study of modal logic is based on the concept of a \emph{logic}, which is a set $\Lambda$ of wff's satisfying the following:
\begin{itemize}
\item contains all tautologies, and 
\item is closed under modus ponens.
\end{itemize}
The last condition means: if $A$ and $A\to B$ are in $\Lambda$, so is $B$ in $\Lambda$.

A \emph{normal modal logic} is a modal logic $\Lambda$ that includes the \emph{law of distribution} K (after Kripke):
$$\square (A\to B)\to (\square A \to \square B)$$
as an axiom schema, and obeying the \emph{rule of necessitation} $RN$:
\begin{center}
from $\vdash A$, we may infer $\vdash \square A$: if $A\in \Lambda$, then $\square A \in \Lambda$.
\end{center}

Normal modal logics are the most widely studied modal logics.  The smallest normal modal logic is called \textbf{K}.  Other normal modal logics are built from \textbf{K} by attaching wff's as axiom schemas.  Below is a list of schemas used to form some of the most common normal modal logics:
\begin{multicols}{2}{
\begin{itemize}
\item 4: $\square A\to \square \square A$
\item 5: $\Diamond A \to \square \Diamond A$
\item D: $\square A \to \Diamond A$
\item T: $\square A \to A$
\item B: $A\to \square \Diamond A$
\item C: $\square (A\wedge \square B)\to \square (A\wedge B)$
\item M: $\square (A\wedge B)\to \square A\wedge \square B$
\item G: $\Diamond \square A \to \square \Diamond A$
\item L: $\square (A\wedge \square A \to B) \vee \square (B\wedge \square B \to A)$
\item W: $\square (\square A \to A) \to \square A$
\end{itemize}
}\end{multicols}
For example, the normal modal logic \textbf{D} is the smallest normal modal logic containing $D$ as its axiom schema.

\textbf{Notation}.  The smallest normal modal logic containing schemas $\Sigma_1,\ldots, \Sigma_n$ is typically denoted 
\begin{center}
\textbf{K}$\mathbf{\Sigma_1 \cdots \Sigma_n}$.  
\end{center}
It is easy to see that \textbf{K}$\mathbf{\Sigma_1 \cdots \Sigma_n}$ can be built from the ``bottom up'': call a finite sequence of wff's a deduction if each wff is either a tautology, an instance of $\Sigma_i$ for some $i$, or as a result of an application of modus ponens or necessitation on earlier wff's in the sequence.  A wff is deducible from if it is the last member of some deduction.  Let $\Lambda_k$ be the set of all wff's deducible from deductions of lengths at most $k$.  Then 
\begin{center}
\textbf{K}$\mathbf{\Sigma_1 \cdots \Sigma_n} = \bigcup_{i=1}^{\infty} \Lambda_i$
\end{center}

Below are some of the most common normal modal logics:
\begin{center}
\begin{tabular}{|c||c|c|c|c|c|c|c|c|}
\hline
name & \textbf{D} & \textbf{T} & \textbf{B} & \textbf{S4} & \textbf{S5} & \textbf{GL} & \textbf{K4.3} & \textbf{S4.3} \\
\hline
notation & \textbf{KD} & \textbf{KT} & \textbf{KTB} & \textbf{KT4} & \textbf{KT5} & \textbf{KW} & \textbf{K4L} & \textbf{KT4L} \\
\hline
\end{tabular}
\end{center}
\textbf{Remarks}
\begin{itemize}
\item \textbf{D} is commonly used in the study of deontic logic (logic of obligation).  Extensions of \textbf{D} such as \textbf{KD4} and \textbf{KD45} are used in the study of doxastic logic (logic of belief).
\item \textbf{GL} is known as provability logic, where $\square A$ means $A$ is provable in Peano arithmetic.
\item \textbf{S4} and \textbf{S5} are two of the Lewis' 5 modal logical systems.  They are commonly used in the study of epistemic logic (logic of knowledge).  The modal logics \textbf{S1}, \textbf{S2}, and \textbf{S3} are non-normal.
\end{itemize}

\subsubsection*{Semantics}

The dominant semantics for normal modal logic is the Kripke semantics, or relational semantics.  More on this can be found \PMlinkid{here}{12541}.  A logic is \emph{sound} in a class of frames if every theorem is valid in every frame in the class, and \emph{complete} if any formula valid in every frame in the class is a theorem.  When a logic $\Lambda$ is both sound and complete in a class $\mathcal{C}$ of frames, we say that $\mathcal{C}$ \emph{describes} $\Lambda$.


The following table lists the logics \textbf{K}$\mathbf{\Sigma}$ and the corresponding sound and complete classes of (Kripke) frames:
\begin{center}
\begin{tabular}{|c|c|c|}
\hline
$\Sigma$ in  \textbf{K}$\mathbf{\Sigma}$ & frame \textbf{K}$\mathbf{\Sigma}$ is sound in & frame \textbf{K}$\mathbf{\Sigma}$ is complete in \\
\hline\hline
4 & transitive & transitive \\
\hline
5 & Euclidean & Euclidean \\
\hline
D & serial & serial \\
\hline
T & reflexive & reflexive \\
\hline
B & symmetric & symmetric \\
\hline
G & weakly directed & weakly directed \\
\hline
L & weakly connected & weakly connected \\
\hline
W & transitive and converse well-founded & finite transitive and irreflexive \\
\hline
\end{tabular}
\end{center}

%%%%%
%%%%%
\end{document}
