\documentclass[12pt]{article}
\usepackage{pmmeta}
\pmcanonicalname{PartitionsLessThanCofinality}
\pmcreated{2013-03-22 12:55:56}
\pmmodified{2013-03-22 12:55:56}
\pmowner{Henry}{455}
\pmmodifier{Henry}{455}
\pmtitle{partitions less than cofinality}
\pmrecord{6}{33287}
\pmprivacy{1}
\pmauthor{Henry}{455}
\pmtype{Result}
\pmcomment{trigger rebuild}
\pmclassification{msc}{03E04}
\pmrelated{Arrowsrelation}
\pmrelated{ArrowsRelation}

\endmetadata

% this is the default PlanetMath preamble.  as your knowledge
% of TeX increases, you will probably want to edit this, but
% it should be fine as is for beginners.

% almost certainly you want these
\usepackage{amssymb}
\usepackage{amsmath}
\usepackage{amsfonts}

% used for TeXing text within eps files
%\usepackage{psfrag}
% need this for including graphics (\includegraphics)
%\usepackage{graphicx}
% for neatly defining theorems and propositions
%\usepackage{amsthm}
% making logically defined graphics
%%%\usepackage{xypic}

% there are many more packages, add them here as you need them

% define commands here
%\PMlinkescapeword{theory}
\begin{document}
If $\lambda<\operatorname{cf}(\kappa)$ then $\kappa\rightarrow(\kappa)^1_\lambda$.

This follows easily from the definition of cofinality.  For any coloring $f:\kappa\rightarrow\lambda$ then define $g:\lambda\rightarrow\kappa+1$ by $g(\alpha)=|f^{-1}(\alpha)|$.  Then $\kappa=\sum_{\alpha<\lambda} g(\alpha)$, and by the normal rules of cardinal arithmetic $\operatorname{sup}_{\alpha<\lambda} g(\alpha)=\kappa$.  Since $\lambda<\operatorname{cf}(\kappa)$, there must be some $\alpha<\lambda$ such that $g(\alpha)=\kappa$.
%%%%%
%%%%%
\end{document}
