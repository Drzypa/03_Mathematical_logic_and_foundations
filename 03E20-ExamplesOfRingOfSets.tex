\documentclass[12pt]{article}
\usepackage{pmmeta}
\pmcanonicalname{ExamplesOfRingOfSets}
\pmcreated{2013-03-22 15:47:52}
\pmmodified{2013-03-22 15:47:52}
\pmowner{rspuzio}{6075}
\pmmodifier{rspuzio}{6075}
\pmtitle{examples of ring of sets}
\pmrecord{8}{37759}
\pmprivacy{1}
\pmauthor{rspuzio}{6075}
\pmtype{Example}
\pmcomment{trigger rebuild}
\pmclassification{msc}{03E20}
\pmclassification{msc}{28A05}

\endmetadata

% this is the default PlanetMath preamble.  as your knowledge
% of TeX increases, you will probably want to edit this, but
% it should be fine as is for beginners.

% almost certainly you want these
\usepackage{amssymb}
\usepackage{amsmath}
\usepackage{amsfonts}

% used for TeXing text within eps files
%\usepackage{psfrag}
% need this for including graphics (\includegraphics)
%\usepackage{graphicx}
% for neatly defining theorems and propositions
%\usepackage{amsthm}
% making logically defined graphics
%%%\usepackage{xypic}

% there are many more packages, add them here as you need them

% define commands here
\begin{document}
Every field of sets is a ring of sets.  Below are some examples of rings of sets that are not fields of sets.

\begin{enumerate}
\item Let $A$ be a non-empty set containing an element $a$.  Let $\mathcal{R}$ be the family of subsets of $A$ containing $a$.  Then $\mathcal{R}$ is a ring of sets, but not a field of sets, since $\lbrace a\rbrace \in \mathcal{R}$, but $A-\lbrace a\rbrace \notin \mathcal{R}$.
\item The collection of all open sets of a topological space is a ring of sets, which is in general not a field of sets, unless every open set is also closed.  Likewise, the collection of all closed sets of a topological space is also a ring of sets.
\item
A simple example of a ring of sets is the subset $\{ \{a\}, \{a,b\} \}$
of $2^{\{a,b\}}$.  That this is a ring of sets follows from the
observations that $\{a\} \cap \{a,b\} = \{a\}$ and $\{a\} \cup \{a,b\}
= \{a,b\}$.  Note that it is not a field of sets because the
complement of $\{a\}$, which is $\{b\}$, does not belong to the ring.
\item
Another example involves an infinite set.  Let $A$ be an infinite set.  Let $\mathcal{R}$ be the collection of finite subsets of $A$.  Since the union and the intersection of two finite set are finite sets, $\mathcal{R}$ is a ring of sets.  However, it is not a field of sets, because the complement of a finite subset of $A$ is infinite, and thus not a member of $\mathcal{R}$.
\end{enumerate}
%%%%%
%%%%%
\end{document}
