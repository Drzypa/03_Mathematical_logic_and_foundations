\documentclass[12pt]{article}
\usepackage{pmmeta}
\pmcanonicalname{PenrosesFirstGodelianArgument}
\pmcreated{2013-03-22 17:05:45}
\pmmodified{2013-03-22 17:05:45}
\pmowner{dankomed}{17058}
\pmmodifier{dankomed}{17058}
\pmtitle{Penrose's first G\"odelian argument}
\pmrecord{23}{39391}
\pmprivacy{1}
\pmauthor{dankomed}{17058}
\pmtype{Topic}
\pmcomment{trigger rebuild}
\pmclassification{msc}{03D80}
%\pmkeywords{Roger Penrose}
%\pmkeywords{nonalgorithmic mind}
%\pmkeywords{G\"odel's first theorem}
%\pmkeywords{artificial intelligence}

% this is the default PlanetMath preamble.  as your knowledge
% of TeX increases, you will probably want to edit this, but
% it should be fine as is for beginners.

% almost certainly you want these
\usepackage{amssymb}
\usepackage{amsmath}
\usepackage{amsfonts}

% used for TeXing text within eps files
%\usepackage{psfrag}
% need this for including graphics (\includegraphics)
%\usepackage{graphicx}
% for neatly defining theorems and propositions
%\usepackage{amsthm}
% making logically defined graphics
%%%\usepackage{xypic}

% there are many more packages, add them here as you need them

% define commands here

\begin{document}
The application of G\"odel's theorems to \PMlinkescapetext{fields} outside metamathematics,
notably the philosophy of mind, was initiated by G\"odel himself. He
had a strong philosophical bent towards Platonism, which also
motivated his mathematical discoveries (Saint-Andre, 1998). G\"odel first thought that his
theorems established the superiority of mind over machine (Wang, 1996a; 1996b). Later,
he came to a less decisive, conditional view: if machine can equal
mind, the fact that it does cannot be proved (Bojadziev, 1997).

\begin{verse} "either...the human mind (\PMlinkescapetext{even} within the realm of pure mathematics) infinitely surpasses the powers of any finite machine, or else there exist absolutely unsolvable diophantine problems." -- Godel, 1951*, p. 310
\end{verse}

The importance of G\"odel's theorem for understanding of human mind was
revived by Lucas (1961) and was brought in the scope of brain scientists
recently by Roger Penrose (1989, 1994).

In 1989 Roger Penrose suggested that human mind is nonalgorithmic (noncomputable) and more powerful than any formal system and this follows from G\"odel's first theorem. Indeed his argument aimed to show that we (humans) can unmistakenly find a true statement $G(\check{g})$ for given formal system $F$, which the formal system $F$ cannot prove, but which we (humans) know is true. Thus stated the statement for mind noncomputability follows from statement which is stronger than G\"odel's first theorem, and due to numerous errors and ambiguities in the exposition it is not clear whether Roger Penrose in his book \emph{"The Emperor's New Mind"} claims to have proved mind noncomputability from result stronger than G\"odel's first theorem, or proves it directly from G\"odel's first theorem. That is why below we provide concise but not equivalent formulations of Penrose's main thesis. Indeed it seems that Penrose himself makes no difference between these two.

\begin{verse}Penrose's first G\"odelian argument (formulation 1, mind noncomputability is derived from statement stronger than G\"odel's first theorem): Human mathematician sees the truthness of the G\"odelian sentence $G(\check{g})$ of given formal system $F$, therefore human mind is nonalgorithmic (noncomputable).
\end{verse}

\begin{verse}Penrose's first G\"odelian argument (formulation 2, mind noncomputability is derived directly from G\"odel's first theorem): Human mathematician understands (proves) G\"odel's first theorem, therefore human mind is nonalgorithmic (noncomputable).
\end{verse}

Both of these arguments are provably erroneous, and as stated are false. The correct full text of G\"odel's first theorem that can be proven by every sufficiently strong formal system is the following.

\begin{verse}
G\"odel's first theorem: for every sufficiently strong to capture basic
arithmetic formal system $F$ which is recursively axiomatizable (i.e. there is an algorithm for determining for each string of the underlying language whether it is an axiom or not) in
which all valid sentences are representable by finite strings of symbols
there exists true sentence $G(\check{g})$, which is undecidable within
$F$ if $F$ is consistent. (Note: $\check{g}$ is the numeral of the G\"odel number g of the formula $G(\check{g})$)
\end{verse}

Here we should note that if the formal system is inconsistent it will be able to prove any statement including the G\"odel's first theorem. If the formal system is consistent and sufficiently strong it will also be able to prove G\"odel's first theorem, and this was shown by G\"odel's himself in 1930.

G\"odel's first theorem is both human understandable and
it is a theorem in any sufficiently strong finite size formal system
$F$ that captures basic arithmetic. Thus computers also prove G\"odel's
first theorem. 

Penrose's argument (formulation 1) insisting that we
directly see the truthness of $G(\check{g})$ in $F$ is false because we
have to be able to directly decide the consistency of arbitrary finite
size formal system $F$, which is clearly not the case. 
Therefore what Penrose has overlooked is that human when given a formal system $F$ must be able to decide nonalgoritmically and unmistakenly whether $F$ is consistent or not, in order to determine whether $G(\check{g})$ is true or not. Clearly $G(\check{g})$ is false for inconsistent $F$! Since no human can decide nonalgoritmically and unmistakenly whether arbitrary $F$ is consistent or not then Penrose's first G\"odelian argument (formulation 1) is false, and the human mathematician cannot see directly that $G(\check{g})$ is true, as $G(\check{g})$ might be false for inconsistent $F$. 

Penrose's argument (formulation 2) is modification that derives mind noncomputability directly from G\"odel's first theorem in a fashion \emph{"the human mathematician can "see" that $G(\check{g})$ is true for consistent $F$ however the consistent $F$ cannot prove $G(\check{g})$"}. It should be noted however that this latter claim is not strong enough for one to establish superiority of mind over formal systems (algorithms) and does not prove noncomputability of mind, because the human does not "see" the truthness of $G(\check{g})$, the human derives (proves) it given the premise for consistent $F$. Thus Penrose puts the human and the formal system in not equivalent positions: the human knows explicitly that $F$ is consistent, while the $F$ is given no such access on grounds that consistent formal system cannot know its own consistency as implied by G\"odel's second theorem. This inequivalence of the initial "given" (not "seen"!) knowledge with which the human and the formal system start leads to erroneous final conclusions in the book \emph{"The Emperor's New Mind"}, and makes it impossible for the reader to understand whether Penrose makes distinction between the presented above formulation 1 and formulation 2 of his argument. The text of G\"odel's first theorem is provable within the formal system $F$ and it says exactly that \emph{"$G(\check{g})$ is true for consistent $F$"}. Therefore no matter what the ontology of our mind is, there is nothing novel about the G\"odel's first theorem and about the G\"odel's statement $G(\check{g})$ that the human knows but the artificial intellect does not prove.

References

1. Bojadziev D (1997) \PMlinkexternal{Mind versus G\"odel. In: M. Gams, M. Paprzycki and X. Wu (eds.), \emph{Mind Versus Computer}, IOS Press, pp. 202-210.}{http://nl.ijs.si/~damjan/g-m-c.html}

2. Crossley JN, Brickhill C, Ash C, Stillwell J, Williams
N (1972) \emph{What is mathematical logic?} Oxford University Press.

3. Detlovs V, Podnieks K (2006) \PMlinkexternal{\emph{Introduction to
Mathematical Logic}. Hypertextbook for students in mathematical logic}{http://www.ltn.lv/~podnieks/mlog/ml.htm} 

4. G\"odel K (1931) \"Uber formal unentscheidbare S\"atze der
Principia \PMlinkescapetext{Mathematica} und verwandter Systeme I. \emph{Monatshefte
f\"ur Mathematik und Physik} \textbf{38}: 173-198.

5. G\"odel K [1951*] \emph{Collected Works, Vol. III. Unpublished Essays and Lectures}. Oxford University Press, 1995.

6. Lucas JR (1961) Minds, Machines and G\"odel. \emph{Philosophy}
\textbf{36}: 112-127.

7. Mendelson E (1997) \emph{Introduction to Mathematical
Logic}, 4th ed. London: Chapman \& Hall.

8. Penrose R (1989) \emph{The Emperor's New Mind: Concerning
Computers, Minds, and The Laws of Physics}. Oxford University Press.

9. Penrose R (1994) \emph{Shadows of the Mind: A Search
for the Missing Science of Consciousness}. Oxford University Press.

10. Podnieks K (2006) \PMlinkexternal{\emph{What is Mathematics? G\"odel's
Theorem and Around}. Hypertextbook for students in mathematical logic}{http://www.ltn.lv/~podnieks/gt.html}

11. Saint-Andre P (1998) \PMlinkexternal{Objectivism Without Platonism: Hao Wang on Kurt G\"odel}{http://www.saint-andre.com/thoughts/wang-godel.html}

12. Wang H (1996a) A Logical Journey: From G\"odel to Philosophy. Cambridge, Massachusetts, The MIT Press.

13. Wang H (1996b) \PMlinkexternal{Can Minds do more than Brains? In \emph{How Things Are: A Science Tool-Kit for the Mind}, John Brockman and Katinka Matson (eds.)}{http://www.fortunecity.com/emachines/e11/86/moremind.html}

%%%%%
%%%%%
\end{document}
