\documentclass[12pt]{article}
\usepackage{pmmeta}
\pmcanonicalname{CNF}
\pmcreated{2013-03-22 14:02:35}
\pmmodified{2013-03-22 14:02:35}
\pmowner{rspuzio}{6075}
\pmmodifier{rspuzio}{6075}
\pmtitle{CNF}
\pmrecord{7}{35392}
\pmprivacy{1}
\pmauthor{rspuzio}{6075}
\pmtype{Definition}
\pmcomment{trigger rebuild}
\pmclassification{msc}{03B05}
\pmsynonym{conjunctive normal form}{CNF}
\pmrelated{DNF}
\pmrelated{AtomicFormula}

\endmetadata

% this is the default PlanetMath preamble.  as your knowledge
% of TeX increases, you will probably want to edit this, but
% it should be fine as is for beginners.

% almost certainly you want these
\usepackage{amssymb}
\usepackage{amsmath}
\usepackage{amsfonts}

% used for TeXing text within eps files
%\usepackage{psfrag}
% need this for including graphics (\includegraphics)
%\usepackage{graphicx}
% for neatly defining theorems and propositions
%\usepackage{amsthm}
% making logically defined graphics
%%%\usepackage{xypic}

% there are many more packages, add them here as you need them

% define commands here
\begin{document}
A propositional formula is a CNF formula, meaning Conjunctive Normal Form, if it is a conjunction of disjunction of literals (a literal is a propositional variable or its negation). Hence, a CNF is a formula of the form: $K_1 \wedge K_2 \wedge \ldots \wedge K_n$, where each $K_i$ is of the form $l_{i1} \vee l_{i2} \vee \ldots \vee l_{im}$ for literals $l_{ij}$ and some $m$ (which can vary for each $K_i$).

Example: $(x\vee y \vee \neg z) \wedge (y\vee  \neg w \vee \neg u) \wedge (x\vee v)$.
%%%%%
%%%%%
\end{document}
