\documentclass[12pt]{article}
\usepackage{pmmeta}
\pmcanonicalname{SuperfluityOfTheThirdDefiningPropertyForFiniteConsequenceOperator}
\pmcreated{2013-03-22 16:30:13}
\pmmodified{2013-03-22 16:30:13}
\pmowner{rspuzio}{6075}
\pmmodifier{rspuzio}{6075}
\pmtitle{superfluity of the third defining property for finite consequence operator}
\pmrecord{5}{38678}
\pmprivacy{1}
\pmauthor{rspuzio}{6075}
\pmtype{Theorem}
\pmcomment{trigger rebuild}
\pmclassification{msc}{03G25}
\pmclassification{msc}{03G10}
\pmclassification{msc}{03B22}

% this is the default PlanetMath preamble.  as your knowledge
% of TeX increases, you will probably want to edit this, but
% it should be fine as is for beginners.

% almost certainly you want these
\usepackage{amssymb}
\usepackage{amsmath}
\usepackage{amsfonts}

% used for TeXing text within eps files
%\usepackage{psfrag}
% need this for including graphics (\includegraphics)
%\usepackage{graphicx}
% for neatly defining theorems and propositions
\usepackage{amsthm}
% making logically defined graphics
%%%\usepackage{xypic}

% there are many more packages, add them here as you need them

% define commands here

\newtheorem*{theorem}{Theorem}
\begin{document}
In this entry, we demonstrate the claim made in section 1 of the 
\PMlinkid{parent entry}{8646} that the defining conditions for 
finitary consequence operator given there are redundant because
one of them may be derived from the other two.

\begin{theorem}
Let $L$ be a set.  Suppose that a mapping $C \colon \mathcal{P}(L) 
\to \mathcal{P}(L)$ satisfies the following three properties:
\begin{enumerate}
\item For all $X \subseteq L$, it happens that $X \subseteq C(X)$.
\item $C \circ C = C$
\item For all $X \in L$, it happens that $C(X) = \bigcup\limits_{Y \in 
\mathcal{F} (X)} C(Y)$.
\end{enumerate}
Then $ C$ also satisfies the following property:  For all $X, Y 
\subseteq L$, if $X \subseteq Y$, then $C(X) \subseteq C(Y)$.
\end{theorem}
%%%%%
%%%%%
\end{document}
