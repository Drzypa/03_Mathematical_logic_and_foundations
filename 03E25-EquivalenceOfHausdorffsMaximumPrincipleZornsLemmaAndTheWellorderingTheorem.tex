\documentclass[12pt]{article}
\usepackage{pmmeta}
\pmcanonicalname{EquivalenceOfHausdorffsMaximumPrincipleZornsLemmaAndTheWellorderingTheorem}
\pmcreated{2013-03-22 13:04:45}
\pmmodified{2013-03-22 13:04:45}
\pmowner{mathcam}{2727}
\pmmodifier{mathcam}{2727}
\pmtitle{equivalence of Hausdorff's maximum principle, Zorn's lemma and the well-ordering theorem}
\pmrecord{9}{33493}
\pmprivacy{1}
\pmauthor{mathcam}{2727}
\pmtype{Proof}
\pmcomment{trigger rebuild}
\pmclassification{msc}{03E25}
\pmsynonym{proof ofZorn's lemma}{EquivalenceOfHausdorffsMaximumPrincipleZornsLemmaAndTheWellorderingTheorem}
\pmsynonym{proof of Hausdorff's maximum principle}{EquivalenceOfHausdorffsMaximumPrincipleZornsLemmaAndTheWellorderingTheorem}
\pmsynonym{proof of the maximum principle}{EquivalenceOfHausdorffsMaximumPrincipleZornsLemmaAndTheWellorderingTheorem}
\pmrelated{ZornsLemma}
\pmrelated{AxiomOfChoice}
\pmrelated{ZermelosWellOrderingTheorem}
\pmrelated{HaudorffsMaximumPrinciple}

\endmetadata

% this is the default PlanetMath preamble.  as your knowledge
% of TeX increases, you will probably want to edit this, but
% it should be fine as is for beginners.

% almost certainly you want these
\usepackage{amssymb}
\usepackage{amsmath}
\usepackage{amsfonts}

% used for TeXing text within eps files
%\usepackage{psfrag}
% need this for including graphics (\includegraphics)
%\usepackage{graphicx}
% for neatly defining theorems and propositions
%\usepackage{amsthm}
% making logically defined graphics
%%%\usepackage{xypic}

% there are many more packages, add them here as you need them

% define commands here
\newcommand{\A}{\mathcal{A}}
\newcommand{\B}{\mathcal{B}}
\begin{document}
\paragraph{Hausdorff's maximum principle implies Zorn's lemma.}
Consider
a partially ordered set $X$, where every chain has an upper bound. According to the maximum principle
there exists a maximal totally ordered subset $Y\subseteq X$. This then has an upper bound, $x$. If
$x$ is not the largest element in $Y$ then $\{x\}\cup Y$ would be a totally ordered set in which $Y$
would be properly contained, contradicting the definition. Thus $x$ is a maximal element in $X$.

\paragraph{Zorn's lemma implies the well-ordering theorem.}
Let $X$ be any non-empty set, and let $\A$ be the collection of pairs $(A,\leq)$, where $A\subseteq X$
and $\leq$ is a well-ordering on $A$. Define a relation $\preceq$, on $\A$ so that for all $x,y\in\A: x\preceq y$
iff $x$ equals an initial of $y$. It is easy to see that this defines a partial order relation on $\A$
(it inherits reflexibility, anti symmetry and transitivity from one set being an initial and thus a subset of
the other).

For each chain $C\subseteq\A$, define $C'=(R,\leq')$ where R is the union of all the sets $A$
for all $(A,\leq)\in C$, and $\leq'$ is the union of all the relations $\leq$ for all $(A,\leq)\in C$.
It follows that $C'$ is an upper bound for $C$ in $\A$.

According to Zorn's lemma, $\A$ now has a maximal element, $(M,\leq_M)$. We postulate that $M$ contains all
members of $X$, for if this were not true we could for any $a\in X-M$ construct $(M_*,\leq_*)$ where
$M_*=M\cup\{a\}$ and $\leq_*$ is extended so $S_a(M_*)=M$. Clearly $\leq_*$ then defines a well-order on
$M_*$, and $(M_*,\leq_*)$ would be larger than $(M,\leq_M)$ contrary to the definition.

Since $M$ contains all the members of $X$ and $\leq_M$ is a well-ordering of $M$, it is also a
well-ordering on $X$ as required.

\paragraph{The well-ordering theorem implies Hausdorff's maximum principle.}
Let $(X,\preceq)$ be a partially ordered set, and let $\leq$ be a well-ordering on $X$. We define
the function $\phi$ by transfinite recursion over $(X,\leq)$ so that
$$\phi(a) = \begin{cases}
\{a\}         &\text{if }\{a\}\cup\bigcup_{b<a}\phi(b)\text{ is totally ordered under }\preceq.\\
\emptyset &\text{otherwise}.
\end{cases}.$$
It follows that $\bigcup_{x\in X}\phi(x)$ is a maximal totally ordered subset of $X$ as required.
%%%%%
%%%%%
\end{document}
