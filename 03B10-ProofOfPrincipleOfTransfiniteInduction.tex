\documentclass[12pt]{article}
\usepackage{pmmeta}
\pmcanonicalname{ProofOfPrincipleOfTransfiniteInduction}
\pmcreated{2013-03-22 12:29:06}
\pmmodified{2013-03-22 12:29:06}
\pmowner{jihemme}{316}
\pmmodifier{jihemme}{316}
\pmtitle{proof of principle of transfinite induction}
\pmrecord{11}{32704}
\pmprivacy{1}
\pmauthor{jihemme}{316}
\pmtype{Proof}
\pmcomment{trigger rebuild}
\pmclassification{msc}{03B10}
%\pmkeywords{well ordered set}

\endmetadata

% this is the default PlanetMath preamble.  as your knowledge
% of TeX increases, you will probably want to edit this, but
% it should be fine as is for beginners.

% almost certainly you want these
\usepackage{amssymb}
\usepackage{amsmath}
\usepackage{amsfonts}

% used for TeXing text within eps files
%\usepackage{psfrag}
% need this for including graphics (\includegraphics)
%\usepackage{graphicx}
% for neatly defining theorems and propositions
%\usepackage{amsthm}
% making logically defined graphics
%%%\usepackage{xypic} 

% there are many more packages, add them here as you need them

% define commands here
\newcommand{\Implies}{\Rightarrow}
\begin{document}
To prove the transfinite induction theorem, we note that the class of ordinals is well-ordered by $\in$.  So suppose for some $\Phi$, there are ordinals $\alpha$ such that $\Phi(\alpha)$ is not true.  Suppose further that $\Phi$ satisfies the hypothesis, i.e. 
$\forall\alpha(\forall\beta<\alpha(\Phi(\beta))\Rightarrow\Phi(\alpha))$.  We will reach a contradiction.  

The class $C=\{\alpha:\neg\Phi(\alpha)\}$ is not empty.  Note that it may be a proper class, but this is not important.  Let $\gamma=\min(C)$ be the $\in$-minimal element of $C$.  Then by assumption, for every $\lambda<\gamma$, $\Phi(\lambda)$ is true.  Thus, by hypothesis, $\Phi(\gamma)$ is true, contradiction.
%%%%%
%%%%%
\end{document}
