\documentclass[12pt]{article}
\usepackage{pmmeta}
\pmcanonicalname{AxiomatizableClass}
\pmcreated{2013-03-22 17:34:38}
\pmmodified{2013-03-22 17:34:38}
\pmowner{CWoo}{3771}
\pmmodifier{CWoo}{3771}
\pmtitle{axiomatizable class}
\pmrecord{7}{39989}
\pmprivacy{1}
\pmauthor{CWoo}{3771}
\pmtype{Definition}
\pmcomment{trigger rebuild}
\pmclassification{msc}{03C52}
\pmsynonym{axiomatisable class}{AxiomatizableClass}
\pmsynonym{finitely axiomatizable}{AxiomatizableClass}
\pmsynonym{finitely axiomatisable}{AxiomatizableClass}
\pmsynonym{EC}{AxiomatizableClass}
\pmsynonym{EC$_{\Delta}$}{AxiomatizableClass}
\pmrelated{Supercategories3}
\pmrelated{AxiomaticAndCategoricalFoundationsOfMathematicsII2}
\pmdefines{elementary class}

\usepackage{amssymb,amscd}
\usepackage{amsmath}
\usepackage{amsfonts}
\usepackage{mathrsfs}

% used for TeXing text within eps files
%\usepackage{psfrag}
% need this for including graphics (\includegraphics)
%\usepackage{graphicx}
% for neatly defining theorems and propositions
\usepackage{amsthm}
% making logically defined graphics
%%\usepackage{xypic}
\usepackage{pst-plot}
\usepackage{psfrag}

% define commands here
\newtheorem{prop}{Proposition}
\newtheorem{thm}{Theorem}
\newtheorem{ex}{Example}
\newcommand{\real}{\mathbb{R}}
\newcommand{\pdiff}[2]{\frac{\partial #1}{\partial #2}}
\newcommand{\mpdiff}[3]{\frac{\partial^#1 #2}{\partial #3^#1}}
\begin{document}
Let $L$ be a first order language and $T$ a theory in $L$.  Recall that a model $M$ is an $L$-structure such that $M$ satisfies every sentence in $T$.  We say that the structure $M$ is a model of $T$.  Let us write $\operatorname{Mod}(T)$ the class of all $L$-structures that are models of $T$.

\textbf{Definition}.  A class $K$ of $L$-structures is said to be \emph{axiomatizable} if there is a theory $T$ such that $K=\operatorname{Mod}(T)$.  Furthermore, $K$ is a \emph{finitely axiomatizable} or \emph{elemenary class} if $T$ is finite.

For example, the class of groups is elementary (and hence axiomatizable), because the set of group axioms is finite.  However, the class of infinite groups is axiomatizable but not elementary.  Similarly, the class of $R$-modules is elementary iff $R$ is finite.  The class of locally finite groups is an example of a non-axiomatizable class.

\textbf{Remarks}. 
\begin{itemize}
\item 
$K$ is an elementary class iff there is a sentence $\varphi$ such that $K=\operatorname{Mod}(\lbrace \varphi \rbrace)$, for sentences $\varphi_1,\ldots,\varphi_n$ can be combined to form $\varphi_1\wedge \cdots \wedge \varphi_n$, which is also a sentence since it has no free variables.
\item
A class is axiomatizable iff it is an intersection of elementary classes.  As such elementary class is sometimes abbreviated EC, and axiomatizable class EC$_{\Delta}$, where $\Delta$ means is another symbol for intersection.
\item
A caution to the reader: some authors call an elementary class an axiomatizable class that is defined here.
\end{itemize}
%%%%%
%%%%%
\end{document}
