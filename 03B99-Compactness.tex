\documentclass[12pt]{article}
\usepackage{pmmeta}
\pmcanonicalname{Compactness}
\pmcreated{2013-03-22 13:49:34}
\pmmodified{2013-03-22 13:49:34}
\pmowner{Aatu}{2569}
\pmmodifier{Aatu}{2569}
\pmtitle{compactness}
\pmrecord{5}{34557}
\pmprivacy{1}
\pmauthor{Aatu}{2569}
\pmtype{Definition}
\pmcomment{trigger rebuild}
\pmclassification{msc}{03B99}
\pmdefines{compactness}

\endmetadata

% this is the default PlanetMath preamble.  as your knowledge
% of TeX increases, you will probably want to edit this, but
% it should be fine as is for beginners.

% almost certainly you want these
\usepackage{amssymb}
\usepackage{amsmath}
\usepackage{amsfonts}

% used for TeXing text within eps files
%\usepackage{psfrag}
% need this for including graphics (\includegraphics)
%\usepackage{graphicx}
% for neatly defining theorems and propositions
%\usepackage{amsthm}
% making logically defined graphics
%%%\usepackage{xypic}

% there are many more packages, add them here as you need them

% define commands here
\begin{document}
A logic is said to be $(\kappa,\lambda)$-compact, if the following holds

\begin{quote}
If $\Phi$ is a set of sentences of cardinality less than or equal to $\kappa$ and all subsets of $\Phi$ of cardinality less than $\lambda$ are consistent, then $\Phi$ is consistent.
\end{quote}

For example, first order logic is $(\omega,\omega)$-compact, for if all finite subsets of some class of sentences are consistent, so is the class itself.
%%%%%
%%%%%
\end{document}
