\documentclass[12pt]{article}
\usepackage{pmmeta}
\pmcanonicalname{Filter}
\pmcreated{2013-03-22 12:09:06}
\pmmodified{2013-03-22 12:09:06}
\pmowner{Koro}{127}
\pmmodifier{Koro}{127}
\pmtitle{filter}
\pmrecord{19}{31342}
\pmprivacy{1}
\pmauthor{Koro}{127}
\pmtype{Definition}
\pmcomment{trigger rebuild}
\pmclassification{msc}{03E99}
\pmclassification{msc}{54A99}
%\pmkeywords{topology}
%\pmkeywords{set theory}
\pmrelated{Ultrafilter}
\pmrelated{KappaComplete}
\pmrelated{KappaComplete2}
\pmrelated{Net}
\pmrelated{LimitAlongAFilter}
\pmrelated{UpperSet}
\pmrelated{OrderIdeal}
\pmdefines{principal filter}
\pmdefines{nonprincipal filter}
\pmdefines{non-principal filter}
\pmdefines{free filter}
\pmdefines{fixed filter}
\pmdefines{neighbourhood filter}
\pmdefines{principal element}
\pmdefines{convergent filter}

\usepackage{amssymb}
\usepackage{amsmath}
\usepackage{amsfonts}
\begin{document}
\PMlinkescapeword{fixed}
\PMlinkescapeword{non-principal}
\PMlinkescapeword{principal}
\PMlinkescapeword{free}
\newcommand{\F}{\mathbb{F}}
Let $X$ be a set. A filter on $X$ is a set $\F$ of subsets of $X$
such that

\begin{itemize}
\item $X\in\F$
\item The intersection of any two elements of $\F$ is an element of $\F$.
\item $\emptyset\notin\F$ (some authors do not include this axiom in the definition of filter)
\item If $F\in\F$ and $F\subset G\subset X$ then $G\in\F$.
\end{itemize}
The first two axioms can be replaced by one:
\begin{itemize}
\item
Any finite intersection of elements of $\F$ is an element of $\F$.
\end{itemize}
with the usual understanding that the intersection of an empty family
of subsets of $X$ is the whole set $X$.

A filter $\F$ is said to be \emph{fixed}
or \emph{principal} if there is $F\in \F$ such that no proper subset of $F$ belongs to $\F$. In this case, $\F$ consists of all subsets of $X$ containing $F$, and $F$ is called a \emph{principal element} of $\F$. If $\F$ is not principal, it is said to be \emph{non-principal} or \emph{free}.

If $x$ is any point (or any subset) of any topological space $X$,
the set $\mathcal{N}_x$ of neighbourhoods of $x$ in $X$ is a filter,
called the \emph{neighbourhood filter} of $x$.
If $\F$ is any filter on the space $X$, $\F$
is said to \emph{converge} to $x$, and we write $\F\to x$,
if $\mathcal{N}_x\subset\F$.
If every neighbourhood of $x$ meets every set of $\F$, then
$x$ is called an \emph{accumulation point}
or \emph{cluster point} of $\F$.

\textbf{Remarks: }
The notion of filter (due to H. Cartan) has a simplifying effect on
various proofs in analysis and topology.
Tychonoff's theorem would be one example.
Also, the two kinds of limit that one sees in elementary real
analysis -- the limit of a sequence at infinity, and the limit
of a function at a point -- are both special cases of the limit
of a filter: the Fr\'echet filter and the neighbourhood filter
respectively.
The notion of a Cauchy sequence can be extended with no difficulty
to any uniform space (but not just a topological space),
getting what is called a Cauchy filter; any convergent filter on a
uniform space is a Cauchy filter, and if the converse holds then
we say that the uniform space is \emph{complete}.
%%%%%
%%%%%
%%%%%
\end{document}
