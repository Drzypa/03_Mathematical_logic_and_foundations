\documentclass[12pt]{article}
\usepackage{pmmeta}
\pmcanonicalname{ClubFilter}
\pmcreated{2013-03-22 12:53:11}
\pmmodified{2013-03-22 12:53:11}
\pmowner{Henry}{455}
\pmmodifier{Henry}{455}
\pmtitle{club filter}
\pmrecord{5}{33231}
\pmprivacy{1}
\pmauthor{Henry}{455}
\pmtype{Definition}
\pmcomment{trigger rebuild}
\pmclassification{msc}{03E10}
\pmdefines{club filter}

\endmetadata

% this is the default PlanetMath preamble.  as your knowledge
% of TeX increases, you will probably want to edit this, but
% it should be fine as is for beginners.

% almost certainly you want these
\usepackage{amssymb}
\usepackage{amsmath}
\usepackage{amsfonts}

% used for TeXing text within eps files
%\usepackage{psfrag}
% need this for including graphics (\includegraphics)
%\usepackage{graphicx}
% for neatly defining theorems and propositions
%\usepackage{amsthm}
% making logically defined graphics
%%%\usepackage{xypic}

% there are many more packages, add them here as you need them

% define commands here
%\PMlinkescapeword{theory}
\begin{document}
If $\kappa$ is a regular uncountable cardinal then $\operatorname{club}(\kappa)$, the filter of all sets containing a club subset of $\kappa$, is a $\kappa$-complete filter closed under diagonal intersection called the \emph{club filter}.

To see that this is a filter, note that $\kappa\in\operatorname{club}(\kappa)$ since it is obviously both closed and unbounded.  If $x\in\operatorname{club}(\kappa)$ then any subset of $\kappa$ containing $x$ is also in $\operatorname{club}(\kappa)$, since $x$, and therefore anything containing it, contains a club set.

It is a $\kappa$ complete filter because the intersection of fewer than $\kappa$ club sets is a club set.  To see this, suppose $\langle C_i\rangle_{i<\alpha}$ is a sequence of club sets where $\alpha<\kappa$.  Obviously $C=\bigcap C_i$ is closed, since any sequence which appears in $C$ appears in every $C_i$, and therefore its limit is also in every $C_i$.  To show that it is unbounded, take some $\beta<\kappa$.  Let $\langle \beta_{1,i}\rangle$ be an increasing sequence with $\beta_{1,1}>\beta$ and $\beta_{1,i}\in C_i$ for every $i<\alpha$.  Such a sequence can be constructed, since every $C_i$ is unbounded.  Since $\alpha<\kappa$ and $\kappa$ is regular, the limit of this sequence is less than $\kappa$.  We call it $\beta_2$, and define a new sequence $\langle\beta_{2,i}\rangle$ similar to the previous sequence.  We can repeat this process, getting a sequence of sequences $\langle\beta_{j,i}\rangle$ where each element of a sequence is greater than every member of the previous sequences.  Then for each $i<\alpha$, $\langle\beta_{j,i}\rangle$ is an increasing sequence contained in $C_i$, and all these sequences have the same limit (the limit of $\langle\beta_{j,i}\rangle$).  This limit is then contained in every $C_i$, and therefore $C$, and is greater than $\beta$.

To see that $\operatorname{club}(\kappa)$ is closed under diagonal intersection, let $\langle C_i\rangle$, $i<\kappa$ be a sequence, and let $C=\Delta_{i<\kappa} C_i$.  Since the diagonal intersection contains the intersection, obviously $C$ is unbounded.  Then suppose $S\subseteq C$ and $\sup(S\cap\alpha)=\alpha$.  Then $S\subseteq C_\beta$ for every $\beta\geq\alpha$, and since each $C_\beta$ is closed, $\alpha\in C_\beta$, so $\alpha\in C$.
%%%%%
%%%%%
\end{document}
