\documentclass[12pt]{article}
\usepackage{pmmeta}
\pmcanonicalname{VonNeumannInteger}
\pmcreated{2013-03-22 12:32:34}
\pmmodified{2013-03-22 12:32:34}
\pmowner{mathcam}{2727}
\pmmodifier{mathcam}{2727}
\pmtitle{von Neumann integer}
\pmrecord{7}{32786}
\pmprivacy{1}
\pmauthor{mathcam}{2727}
\pmtype{Definition}
\pmcomment{trigger rebuild}
\pmclassification{msc}{03E10}
\pmrelated{NaturalNumber}
\pmrelated{VonNeumannOrdinal}

\usepackage{amssymb}
\usepackage{amsmath}
\usepackage{amsfonts}
\begin{document}
A \emph{von Neumann \PMlinkescapetext{integer}} is not an integer, but instead a construction of a natural number using some basic set notation.  The von Neumann integers are defined inductively.  The von Neumann integer zero is defined to be the empty set, $\emptyset$, and there are no smaller von Neumann integers.
The von Neumann integer $N$ is then the set of all von Neumann integers less than $N$.  The set of von Neumann integers is the set of all finite \PMlinkname{von Neumann ordinals}{VonNeumannOrdinal}.

This form of construction from very basic notions of sets is applicable to various forms of set theory (for instance, Zermelo-Fraenkel set theory).  While this construction suffices to define the set of natural numbers, a little more work must be done to define the set of all \PMlinkname{integers}{Integer}.

\subsubsection*{Examples}

\begin{eqnarray*}
0 & = & \emptyset \\
1 & = & \left\{ 0 \right\} = \left\{ \emptyset \right\} \\
2 & = & \left\{ 0, 1 \right\} = \left\{ \emptyset, \left\{ \emptyset \right\} \right\} \\
3 & = & \left\{ 0, 1, 2 \right\} = \left\{ \emptyset, \left\{ \emptyset \right\}, \left\{ \left\{ \emptyset, \left\{ \emptyset \right\} \right\} \right\}\right\} \\
  & \vdots & \\
N & = & \left\{ 0, 1, \dots, N-1 \right\}
\end{eqnarray*}
%%%%%
%%%%%
\end{document}
