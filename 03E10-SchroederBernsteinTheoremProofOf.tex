\documentclass[12pt]{article}
\usepackage{pmmeta}
\pmcanonicalname{SchroederBernsteinTheoremProofOf}
\pmcreated{2013-03-22 12:49:56}
\pmmodified{2013-03-22 12:49:56}
\pmowner{mps}{409}
\pmmodifier{mps}{409}
\pmtitle{Schroeder-Bernstein theorem, proof of}
\pmrecord{19}{33156}
\pmprivacy{1}
\pmauthor{mps}{409}
\pmtype{Proof}
\pmcomment{trigger rebuild}
\pmclassification{msc}{03E10}
\pmsynonym{proof of Cantor-Bernstein theorem}{SchroederBernsteinTheoremProofOf}
\pmsynonym{proof of Cantor-Schroeder-Bernstein theorem}{SchroederBernsteinTheoremProofOf}
%\pmkeywords{cardinality}
%\pmkeywords{bijection}
%\pmkeywords{injection}
\pmrelated{AnInjectionBetweenTwoFiniteSetsOfTheSameCardinalityIsBijective}

\endmetadata

% this is the default PlanetMath preamble.  as your knowledge
% of TeX increases, you will probably want to edit this, but
% it should be fine as is for beginners.

% almost certainly you want these
\usepackage{amssymb}
\usepackage{amsmath}
\usepackage{amsfonts}

% used for TeXing text within eps files
%\usepackage{psfrag}
% need this for including graphics (\includegraphics)
%\usepackage{graphicx}
% for neatly defining theorems and propositions
%\usepackage{amsthm}

% making logically defined graphics
%%%\usepackage{xypic}

% there are many more packages, add them here as you need them

% define commands here
\begin{document}
We first prove as a lemma that for any $B\subset A$, if there is an
injection $f:A\to B$, then there is also a bijection
$h:A\to B$.

Inductively define a sequence $(C_n)$ of subsets of $A$ by $C_0=A\setminus B$
and $C_{n+1}=f(C_n)$.  
Now let $C=\bigcup_{k=0}^\infty C_k$, and define $h:A\rightarrow B$ by
\[h(z)=\begin{cases}
f(z), & z\in C \\
z,    & z\notin C
\end{cases}.\]
If $z\in C$, then $h(z)=f(z)\in B$.  But if $z\notin C$, then $z\in B$, and so $h(z)\in B$.  Hence $h$ is well-defined; $h$ is
injective by construction.  Let $b\in B$.  If $b\notin C$, then
$h(b)=b$.  Otherwise, $b\in C_k=f(C_{k-1})$ for some $k>0$, and
so there is some $a\in C_{k-1}$ such that $h(a)=f(a)=b$.  Thus $h$
is bijective; in particular, if $B=A$, then $h$ is simply the identity
map on $A$.

To prove the theorem, suppose $f:S\to T$ and $g:T\to S$
are injective.  Then the composition $gf:S\to g(T)$ is also
injective.  By the lemma, there is a bijection $h':S\to g(T)$.
The injectivity of $g$ implies that $g^{-1}:g(T)\to T$ exists
and is bijective.  Define $h:S\to T$ by $h(z)=g^{-1}h'(z)$; this
map is a bijection, and so $S$ and $T$ have the same cardinality.
%%%%%
%%%%%
\end{document}
