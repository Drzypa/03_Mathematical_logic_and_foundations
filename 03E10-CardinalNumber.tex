\documentclass[12pt]{article}
\usepackage{pmmeta}
\pmcanonicalname{CardinalNumber}
\pmcreated{2013-03-22 12:07:58}
\pmmodified{2013-03-22 12:07:58}
\pmowner{djao}{24}
\pmmodifier{djao}{24}
\pmtitle{cardinal number}
\pmrecord{8}{31302}
\pmprivacy{1}
\pmauthor{djao}{24}
\pmtype{Definition}
\pmcomment{trigger rebuild}
\pmclassification{msc}{03E10}
\pmsynonym{cardinal}{CardinalNumber}
\pmrelated{Cardinality}
\pmrelated{CardinalArithmetic}

\endmetadata

\usepackage{amssymb}
\usepackage{amsmath}
\usepackage{amsfonts}
\usepackage{graphicx}
%%%\usepackage{xypic}
\begin{document}
A cardinal number is an ordinal number $S$ with the property that $S \leq X$ for every ordinal number $X$ which has the same cardinality as $S$. Cardinal numbers have the property that for every set $A$, there exists a unique cardinal number having the same number of elements as $A$.

%%%%%
%%%%%
%%%%%
\end{document}
