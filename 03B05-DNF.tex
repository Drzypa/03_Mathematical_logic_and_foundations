\documentclass[12pt]{article}
\usepackage{pmmeta}
\pmcanonicalname{DNF}
\pmcreated{2013-03-22 14:14:08}
\pmmodified{2013-03-22 14:14:08}
\pmowner{rspuzio}{6075}
\pmmodifier{rspuzio}{6075}
\pmtitle{DNF}
\pmrecord{4}{35677}
\pmprivacy{1}
\pmauthor{rspuzio}{6075}
\pmtype{Definition}
\pmcomment{trigger rebuild}
\pmclassification{msc}{03B05}
\pmsynonym{disjunctive normal form}{DNF}
\pmrelated{CNF}
\pmrelated{AtomicFormula}

% this is the default PlanetMath preamble.  as your knowledge
% of TeX increases, you will probably want to edit this, but
% it should be fine as is for beginners.

% almost certainly you want these
\usepackage{amssymb}
\usepackage{amsmath}
\usepackage{amsfonts}

% used for TeXing text within eps files
%\usepackage{psfrag}
% need this for including graphics (\includegraphics)
%\usepackage{graphicx}
% for neatly defining theorems and propositions
%\usepackage{amsthm}
% making logically defined graphics
%%%\usepackage{xypic}

% there are many more packages, add them here as you need them

% define commands here
\begin{document}
A propositional formula is a DNF formula, meaning Disjunctive Normal Form, if it is a disjunction of conjunctions of literals (a literal is a propositional variable or its negation). Hence, a DNF is a formula of the form: $K_1 \vee K_2 \vee \ldots \vee K_n$, where each $K_i$ is of the form $l_{i1} \wedge l_{i2} \wedge \ldots \wedge l_{im}$ for literals $l_{ij}$ and some $m$ which can vary for each $K_i$.

Example: $(x\wedge  y \wedge \neg z) \vee (y\wedge \neg w \wedge \neg u) \vee (x \wedge v)$.
%%%%%
%%%%%
\end{document}
