\documentclass[12pt]{article}
\usepackage{pmmeta}
\pmcanonicalname{CompletenessTheoremForPropositionalLogic1}
\pmcreated{2013-03-22 19:32:47}
\pmmodified{2013-03-22 19:32:47}
\pmowner{CWoo}{3771}
\pmmodifier{CWoo}{3771}
\pmtitle{completeness theorem for propositional logic}
\pmrecord{21}{42527}
\pmprivacy{1}
\pmauthor{CWoo}{3771}
\pmtype{Theorem}
\pmcomment{trigger rebuild}
\pmclassification{msc}{03B05}

\usepackage{amssymb,amscd}
\usepackage{amsmath}
\usepackage{amsfonts}
\usepackage{mathrsfs}

% used for TeXing text within eps files
%\usepackage{psfrag}
% need this for including graphics (\includegraphics)
%\usepackage{graphicx}
% for neatly defining theorems and propositions
\usepackage{amsthm}
% making logically defined graphics
%%\usepackage{xypic}
\usepackage{pst-plot}

% define commands here
\newcommand*{\abs}[1]{\left\lvert #1\right\rvert}
\newtheorem{prop}{Proposition}
\newtheorem{thm}{Theorem}
\newtheorem{ex}{Example}
\newtheorem{lem}{Lemma}
\newcommand{\real}{\mathbb{R}}
\newcommand{\pdiff}[2]{\frac{\partial #1}{\partial #2}}
\newcommand{\mpdiff}[3]{\frac{\partial^#1 #2}{\partial #3^#1}}

\begin{document}
The completeness theorem of propositional logic is the statement that a wff is tautology iff it is a theorem.  In symbol, we have $$\models A \quad \mbox{iff} \quad \vdash A$$
for any wff $A$.  The ``if'' part of the statement is the soundness theorem, and is proved \PMlinkname{here}{TruthValueSemanticsForPropositionalLogicIsSound}.  We will prove the ``only if'' part, which is also known as the completeness portion of the theorem.  We will give a constructive proof of this.  Before proving this, we state and prove some preliminary facts:

\begin{enumerate}
\item $A,B \vdash A\to B$
\item $A,\neg B \vdash \neg (A \to B)$
\item $\neg A,B \vdash A \to B$
\item $\neg A, \neg B \vdash A \to B$
\item Let $v$ be a valuation.  For any wff $A$, let $v[A]$ be defined as follows: 
\begin{displaymath}
v[A] \textrm{ is }  \left\{
\begin{array}{ll}
A & \textrm{if } v(A)=1,\\
\neg A & \textrm{if } v(A)=0.
\end{array}
\right.
\end{displaymath}
Suppose $p_1,\ldots,p_n$ are the propositional variables in $A$.  Then $$v[p_1],\ldots, v[p_n]\vdash v[A].$$
\item if $\Delta, A \vdash B$ and $\Delta, \neg A \vdash B$, then $\vdash B$.
\end{enumerate}
\begin{proof}
Facts 1 and 3 come from the axiom schema $B\to (A\to B)$.  From $\vdash B\to (A\to B)$, we have $C\vdash B\to (A\to B)$, so $C,B \vdash A\to B$.  If $C$ is $A$, we have fact 1, and if $C$ is $\neg A$, we have fact 3.

Fact 2: this is proved \PMlinkname{here}{SubstitutionTheoremForPropositionalLogic}.

Fact 4: By ex falso quodlibet, $\vdash \perp \to B$, so $A\vdash \perp\to B$, and therefore $\vdash A\to (\perp\to B)$ by the deduction theorem.  Now, $(A \to (\perp \to B))\to ((A\to \perp)\to (A\to B))$ is an axiom instance, so $\vdash (A\to \perp) \to (A\to B)$, or $\vdash \neg A \to (A\to B)$, or $\neg A \vdash A\to B$, and $$ \neg A, \neg B \vdash A\to B$$ all the more so.

Fact 5: by induction on the number $n$ of occurrences of $\to$ in $A$.  If $n=0$, then $A$ is either $\perp$ or a propositional variable $p$.  In the first case, $v[A]$ is $\neg \perp$, and from $\perp \vdash \perp$, we get $\vdash \perp \to \perp$, or $\vdash \neg \perp$.  In the second case, $v[p]\vdash v[p]$.  Now, suppose there are $n+1$ occurrences of $\to$ in $A$.  Let $p_1,\ldots, p_m$ be the propositional variables in $A$.  By unique readability, $A$ is $B\to C$ for some unique wff's $B$ and $C$.  Since each $B$ and $C$ has no more than $n$ occurrences of $\to$, by induction, we have 
$$v[p_{i(1)}],\ldots, v[p_{i(s)}] \vdash v[B] \qquad \mbox{and} \qquad v[p_{j(1)}],\ldots,  v[p_{j(t)}] \vdash v[C],$$
where the propositional variables in $B$ are $p_{i(1)},\ldots,p_{i(s)}$, and in $C$ are $p_{j(1)},\ldots,p_{j(t)}$.  So
$$v[p_1],\ldots, v[p_m] \vdash v[B] \qquad \mbox{and} \qquad v[p_1],\ldots,  v[p_m] \vdash v[C].$$

Next, we want to show that $v[B],v[C] \vdash v[B\to C]$.  We break this into four cases:
\begin{itemize}
\item if $v[B]$ is $B$ and $v[C]$ is $C$: then $v[B\to C]$ is $B\to C$, and use Fact 1
\item if $v[B]$ is $B$ and $v[C]$ is $\neg C$: then $v[B\to C]$ is $\neg (B\to C)$, and use Fact 2
\item if $v[B]$ is $\neg B$ and $v[C]$ is $C$: then $v[B\to C]$ is $B\to C$, and use Fact 3
\item if $v[B]$ is $\neg B$ and $v[C]$ is $\neg C$: then $v[B\to C]$ is $B\to C$, and use Fact 4.
\end{itemize}
In all cases, we have by applying modus ponens,
$$v[p_1],\ldots, v[p_m] \vdash v[B \to C].$$

Fact 6 is proved \PMlinkname{here}{SubstitutionTheoremForPropositionalLogic}.
\end{proof}

\begin{thm} Propositional logic is complete with respect to truth-value semantics. \end{thm}
\begin{proof}  Suppose $A$ is a tautology.  Let $p_1,\ldots,p_n$ be the propositional variables in $A$.  Then $$ v[p_1],\ldots, v[p_n] \vdash v[A]$$
for any valuation $v$.  Since $v[A]$ is $A$.  We have
$$v[p_1],\ldots, v[p_n] \vdash A.$$
If $n=0$, then we are done.  So suppose $n>0$.  Pick a valuation $v$ such that $v(p_n)=1$, and a valuation $v'$ such that $v'(p_i)=v(p_i)$ and $v'(p_n)=0$  Then
$$v[p_1],\ldots, v[p_{n-1}],  p_n \vdash A \qquad \mbox{and} \qquad v[p_1],\ldots,  v[p_{n-1}], \neg p_n \vdash A,$$
where the first deducibility relation comes from $v$ and the second comes from $v'$.  By Fact 6 above,
$$v[p_1],\ldots, v[p_{n-1}] \vdash A.$$
So we have eliminated $v[p_n]$ from the left of $v[p_1],\ldots, v[p_n] \vdash A$.  Now, repeat this process until all of the $v[p_i]$ have been eliminated, and we have $\vdash A$.
\end{proof}

The completeness theorem can be used to show that certain complicated wff's are theorems.  For example, one of the  distributive laws
$$\vdash (A\land B)\lor C \leftrightarrow (A\lor C)\land (B\lor C)$$
To see that this is indeed a theorem, by the completeness theorem, all we need to show is that it is true using the truth table:
\begin{center}
\begin{tabular}{ccccccccccccc}
$(A$ & $\land$ & $B)$ & $\lor$ & $C$ & $\leftrightarrow$ & $(A$ & $\lor$ & $C)$ & $\land$ & $(B$ & $\lor$ & $C)$ \\
\hline 
T & T & T & T & T & T & T & T & T & T & T & T & T \\
T & T & T & T & F & T & T & T & F & T & T & T & F \\
T & F & F & T & T & T & T & T & T & T & F & T & T \\
T & F & F & F & F & T & T & T & F & F & F & F & F \\
F & F & T & T & T & T & F & T & T & T & T & T & T \\
F & F & T & F & F & T & F & F & F & F & T & T & F \\
F & F & F & T & T & T & F & T & T & T & F & T & T \\
F & F & F & F & F & T & F & F & F & F & F & F & F 
\end{tabular}
\end{center}
Similarly, one can show $\vdash (A\lor B)\land C \leftrightarrow (A\land C)\lor (B\land C)$.

%%%%%
%%%%%
\end{document}
