\documentclass[12pt]{article}
\usepackage{pmmeta}
\pmcanonicalname{IntuitionisticPropositionalLogicIsASubsystemOfClassicalPropositionalLogic}
\pmcreated{2013-03-22 19:33:12}
\pmmodified{2013-03-22 19:33:12}
\pmowner{CWoo}{3771}
\pmmodifier{CWoo}{3771}
\pmtitle{intuitionistic propositional logic is a subsystem of classical propositional logic}
\pmrecord{31}{42535}
\pmprivacy{1}
\pmauthor{CWoo}{3771}
\pmtype{Result}
\pmcomment{trigger rebuild}
\pmclassification{msc}{03B20}
\pmclassification{msc}{03F55}

\endmetadata

\usepackage{amssymb,amscd}
\usepackage{amsmath}
\usepackage{amsfonts}
\usepackage{mathrsfs}

% used for TeXing text within eps files
%\usepackage{psfrag}
% need this for including graphics (\includegraphics)
%\usepackage{graphicx}
% for neatly defining theorems and propositions
\usepackage{amsthm}
% making logically defined graphics
%%\usepackage{xypic}
\usepackage{pst-plot}
\usepackage{multicol}

% define commands here
\newcommand*{\abs}[1]{\left\lvert #1\right\rvert}
\newtheorem{prop}{Proposition}
\newtheorem{thm}{Theorem}
\newtheorem{ex}{Example}
\newcommand{\real}{\mathbb{R}}
\newcommand{\pdiff}[2]{\frac{\partial #1}{\partial #2}}
\newcommand{\mpdiff}[3]{\frac{\partial^#1 #2}{\partial #3^#1}}

\begin{document}
Suppose logical systems $\mathcal{L}_1$ and $\mathcal{L}_2$ share the same set $L$ of well-formed formulas (wff's).  Let $F_1$ be the set of all theorems of $\mathcal{L}_1$, and $F_2$ the set of all theorems of $\mathcal{L}_2$.  We say that $\mathcal{L}_1$ is a subsystem of $\mathcal{L}_2$ if $F_1 \subseteq F_2$, and a proper subsystem if $F_1 \subset F_2$.

In the subsequent discussion, PL$_c$ stands for classical propositional logic, and PL$_i$ for intuitionistic propositional logic.  We write PL$_i \le $ PL$_c$ to mean that PL$_i$ is a subsystem of PL$_c$, or $<$ if it is proper.

\begin{prop} PL$_i < $ PL$_c$.  \end{prop}

Unless otherwise specified, $\vdash A$ means the wff $A$ is a theorem of PL$_c$.

Before proving this, we need the following facts about PL$_c$:
\begin{multicols}{2}{
\begin{enumerate}
\item $\vdash A$ and $\vdash B$ iff $\vdash A\land B$.
\item $\vdash A$ implies $\vdash B$ iff $\vdash A\to B$.
\item $A\to B,  B\to C \vdash A\to C$
\item $A\to B,  \neg A \to B \vdash B$
\item if $\Delta \vdash A$ and $\Gamma, A \vdash B$, then $\Delta, \Gamma \vdash B$.
\end{enumerate}
\end{multicols}
\begin{proof}  The first two facts are proved \PMlinkid{here}{12533}, and the third is \PMlinkid{here}{12561}.

Fact 4 is equivalent to $\vdash (A\to B)\to ((\neg A\to B)\to B)$ by the deduction theorem twice.  Use the theorem schemas $\vdash (A\to B)\leftrightarrow (\neg B\to \neg A)$ and $\vdash A\leftrightarrow \neg \neg A$ (law of double negation), and the substitution theorem twice, it is enough to show $$\vdash (\neg B\to \neg A) \to ((\neg B\to A) \to \neg \neg B).$$  By the deduction theorem three times, it is enough to show $\neg B\to \neg A, \neg B\to A, \neg B \vdash \perp$, which is provided by the deduction $\neg B\to \neg A, \neg B\to A, \neg B, \neg A, A, \perp$.

For the last fact, given deduction $\mathcal{E}_1$ of $A$ from $\Delta$, and deduction $\mathcal{E}_2$ of $A\to B$ from $\Gamma$, $\mathcal{E}_1, \mathcal{E}_2, B$ is a deduction of $B$ from $\Delta \cup \Gamma$.
\end{proof}

\begin{proof}  Since in both systems, the only inference rule is modus ponens, and theoremhood of wff's is preserved by the inference rule (that is, if $\vdash A$ and $\vdash A\to B$, then $\vdash B$), all we need to show is that every axiom of PL$_i$ is a theorem of PL$_c$.
\begin{enumerate}
\item $A \to (B \to A)$.  This is just an axiom schema for PL$_c$.
\item $A \to (B \to A \land B)$.

$A,B,A\to \neg B, \neg B, \perp$ leads to $A,B,A\to \neg B \vdash \perp$.  Applying the deduction theorem three times, we get $\vdash A\to (B\to ((A\to \neg B)\to \perp))$, or $\vdash A \to (B \to A\land B)$.
\item $A \land B \to A$ and $A \land B \to B$.  See \PMlinkid{here}{12533}.

\item $A \to A \lor B$.

$A,\neg A, \perp, \perp\to B, B$ results in $A,\neg A \vdash B$, since $\perp \to B$ is a theorem.  By applying the deduction theorem twice, we get $\vdash A \to (\neg A \to B)$, or $\vdash A\to A\lor B$.
\item $B \to A \lor B$.

Clearly, $B, \neg A \vdash B$.  By the deduction theorem, $B\vdash \neg A \to B$, or $B \vdash A \lor B$.  By the deduction theorem again, $\vdash B \to A\lor B$.
\item $(A \to C) \to ((B \to C) \to (A \lor B \to C))$.

By fact 3, $A\to C, B\to C, \neg A\to B \vdash \neg A \to C$.  With fact 4: $A\to C, \neg A \to C \vdash C$, so $A\to C, B\to C, \neg A\to B \vdash C$ by fact 5.  Now apply the deduction theorem 3 times.
\item $(A \to B) \to ((A \to (B \to C)) \to (A \to C))$.

From $A \to B, A\to (B\to C), A, B\to C, B, C$, we get $A\to B, A\to (B\to C), A \vdash C$.  Applying the deduction theorem three times, we have the result.
\item $(A \to B) \to ((A \to \neg B) \to \neg A)$.  This is just 7, where $C$ is $\perp$.
\item $\neg A \to  (A \to B)$.

From $\neg A, A\vdash \perp$ and $\perp \vdash B$, we get $\neg A,A\vdash B$ by fact 5, and the result follows with two applications of the deduction theorem.
\end{enumerate}
Since $A\lor \neg A$ is a theorem of PL$_c$ and not of PL$_i$ (see \PMlinkid{here}{12493}), we conclude that PL$_i$ is a proper subsystem of PL$_c$.
\end{proof}

\textbf{Remarks}.  
\begin{enumerate}
\item
Strictly speaking, the language $L_i$ of PL$_i$ under \PMlinkid{this system}{12491} is different from the language $L_c$ of PL$_c$ under \PMlinkid{this system}{12507}.  In $L_i$, the logical connectives consist of $\to$, $\neg$, $\land$, $\lor$, whereas in $L_c$, only $\to$ is used.  The other connectives are introduced as abbreviational devices: $\neg A$ is $A\to \perp$, $A\lor B$ is $\neg A \to B$, and $A\land B$ is $\neg (A\to \neg B)$.  So it doesn't make much sense to say that PL$_i < $ PL$_c$.  
\item One way to get around this issue is to re-define what it means for one logical system to be a subsystem of another.   For example, one can define a mapping $\tau: L_i \to L_c$ such that 
\begin{itemize}
\item takes propositional variables to propositional variables
\item $\tau$ preserves the formation rules of wff's
\end{itemize}
A mapping satisfying these conditions is called a \emph{translation}.  Then, a system PL$_i$ is a subsystem of PL$_c$ if there is a translation $\tau: L_i \to L_c$ such that 
\begin{itemize}
\item for any wff $A$ in $L_i$, we have $\vdash_i A$ implies $\vdash_c \tau(A)$.
\end{itemize}
To further require that PL$_i$ be a proper subsystem of PL$_c$, we also need
\begin{itemize}
\item there is an wff $B$ in $L_i$ such that $\not \vdash_i B$ and $\vdash_c \tau(B)$.
\end{itemize}
\item Using this definition, define $\tau: L_i \to L_c$ by
\begin{displaymath}
\tau(A):= \left\{
\begin{array}{ll}
A & \textrm{if $A$ is a propositional variable}\\
\tau(B)\to \tau(C) & \textrm{if $A$ is $B\to C$}\\
\neg \tau(B) & \textrm{if $A$ is $\neg B$}\\
\tau(B) \land \tau(C) & \textrm{if $A$ is $B\land C$}\\
\tau(B)\lor \tau(C)  & \textrm{if $A$ is $B\lor C$.}
\end{array}
\right.
\end{displaymath}
where the symbols $\neg$, $\land$, and $\lor$ in $L_c$ are used as abbreviational tools (see the first remark).  Then we see that the proposition makes sense.
\item 
Another way of getting around this issue is to come up with another axiom system for PL$_c$ that uses $\to$, $\neg$, $\land$, and $\lor$ as primitive logical connectives.  Such an axiom system exists.  In fact, add $\neg \neg A \to A$ to the list of axiom schemas for PL$_i$, and we get an axiom system for PL$_c$.
\item
Using the idea in 2 above further, it can be shown that PL$_c$ is a subsystem of PL$_i$ (under the right translation)!  The existence of such a translation $\nu$ is known as the Glivenko's theorem.  This translation $\nu$ has the further property that for any wff $A$ in $L_c$, $$\vdash_c A \qquad\mbox{iff} \qquad \vdash_i \nu(A).$$
\end{enumerate}

%%%%%
%%%%%
\end{document}
