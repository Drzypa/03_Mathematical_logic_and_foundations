\documentclass[12pt]{article}
\usepackage{pmmeta}
\pmcanonicalname{112DedekindReals}
\pmcreated{2013-11-06 17:45:35}
\pmmodified{2013-11-06 17:45:35}
\pmowner{PMBookProject}{1000683}
\pmmodifier{PMBookProject}{1000683}
\pmtitle{11.2 Dedekind reals}
\pmrecord{1}{}
\pmprivacy{1}
\pmauthor{PMBookProject}{1000683}
\pmtype{Feature}
\pmclassification{msc}{03B15}

\usepackage{xspace}
\usepackage{amssyb}
\usepackage{amsmath}
\usepackage{amsfonts}
\usepackage{amsthm}
\makeatletter
\newcommand{\bool}{\ensuremath{\mathbf{2}}\xspace}
\newcommand{\dcut}{\mathsf{isCut}}  
\newcommand{\defeq}{\vcentcolon\equiv}  
\newcommand{\define}[1]{\textbf{#1}}
\def\exis#1{\exists (#1)\@ifnextchar\bgroup{.\,\exis}{.\,}}
\newcommand{\indexdef}[1]{\index{#1|defstyle}}   
\newcommand{\indexsee}[2]{\index{#1|see{#2}}}    
\newcommand{\jdeq}{\equiv}      
\newcommand{\prop}{\ensuremath{\mathsf{Prop}}\xspace}
\newcommand{\Q}{\ensuremath{\mathbb{Q}}\xspace}
\newcommand{\RD}{\ensuremath{\mathbb{R}_\mathsf{d}}\xspace} 
\newcommand{\UU}{\ensuremath{\mathcal{U}}\xspace}
\newcommand{\vcentcolon}{:\!\!}
\newcounter{mathcount}
\setcounter{mathcount}{1}
\newtheorem{predefn}{Definition}
\newenvironment{defn}{\begin{predefn}}{\end{predefn}\addtocounter{mathcount}{1}}
\renewcommand{\thepredefn}{11.2.\arabic{mathcount}}
\let\autoref\cref
\let\setof\Set    
\makeatother

\begin{document}

\index{real numbers!Dedekind|(}%
Let us first recall the basic idea of Dedekind's construction. We use two-sided Dedekind
cuts, as opposed to an often used one-sided version, because the symmetry makes
constructions more elegant, and it works constructively as well as classically.
\index{mathematics!constructive}%
A \emph{Dedekind cut}\index{cut!Dedekind} consists of a pair $(L, U)$ of subsets $L, U \subseteq \Q$, called the
\emph{lower} and \emph{upper cut} respectively, which are:
% 
\begin{enumerate}
\item \emph{inhabited:} there are $q \in L$ and $r \in U$,
\item \emph{rounded:} $q \in L \Leftrightarrow \exis {r \in \Q} q < r \land r \in L$
  and $r \in U \Leftrightarrow \exis {q \in \Q} q \in U \land q < r$,
  \index{rounded!Dedekind cut}
\item \emph{disjoint:} $\lnot (q \in L \land q \in U)$, and
\item \emph{located:} $q < r \Rightarrow q \in L \lor r \in U$.
  \index{locatedness}%
\end{enumerate}
%
Reading the roundedness condition from left to right tells us that cuts are \emph{open},
\index{open!cut}%
and from right to left that they are \emph{lower}, respectively \emph{upper}, sets. The
locatedness condition states that there is no large gap between $L$ and $U$. Because cuts
are always open, they never include the ``point in between'', even when it is rational. A
typical Dedekind cut looks like this:
%
\begin{center}
  \begin{tikzpicture}[x=\textwidth]
    \draw[<-),line width=0.75pt] (0,0) -- (0.297,0) node[anchor=south east]{$L\ $};
    \draw[(->,line width=0.75pt] (0.300, 0) node[anchor=south west]{$\ U$} -- (0.9, 0) ;
  \end{tikzpicture}
\end{center}
%
We might naively translate the informal definition into type theory by saying that a cut
is a pair of maps $L, U : \Q \to \prop$. But we saw in \autoref{subsec:prop-subsets} that
$\prop$ is an ambiguous\index{typical ambiguity} notation for $\prop_{\UU_i}$ where~$\UU_i$ is a universe. Once we
use a particular $\UU_i$ to define cuts, the type of reals will reside in the next
universe $\UU_{i+1}$, a property of reals two levels higher in $\UU_{i+2}$, a property of
subsets of reals in $\UU_{i+3}$, etc. In principle we should be able to keep track of the
universe levels\index{universe level}, especially with the help of a proof assistant, but doing so here would
just burden us with bureaucracy that we prefer to avoid. We shall therefore make a
simplifying assumption that a single type of propositions $\Omega$ is sufficient for all
our purposes.

In fact, the construction of the Dedekind reals is quite resilient to logical
manipulations. There are several ways in which we can make sense of using a single type
$\Omega$:
%
\begin{enumerate}

\item We could identify $\Omega$ with the ambiguous $\prop$ and track all the universes
  that appear in definitions and constructions.

\item We could assume the propositional resizing axiom,
  \index{propositional!resizing}%
  as in \autoref{subsec:prop-subsets}, which essentially collapses the $\prop_{\UU_i}$'s to the
  lowest level\index{universe level}, which we call $\Omega$.

\item A classical mathematician who is not interested in the intricacies of type-theoretic
  universes or computation may simply assume the law of excluded middle~\eqref{eq:lem} for
  mere propositions so that $\Omega \jdeq \bool$.
  \index{excluded middle}
  This not only eradicates questions about
  levels\index{universe level} of $\prop$, but also turns everything we do into the standard classical\index{mathematics!classical}
  construction of real numbers.

\item On the other end of the spectrum one might ask for a minimal requirement that makes
  the constructions work. The condition that a mere predicate be a Dedekind cut is
  expressible using only conjunctions, disjunctions, and existential quantifiers\index{quantifier!existential} over~\Q, which
  is a countable set. Thus we could take $\Omega$ to be the initial \emph{$\sigma$-frame},
  \index{initial!sigma-frame@$\sigma$-frame}%
  \index{sigma-frame@$\sigma$-frame!initial|defstyle}%
  i.e., a lattice\index{lattice} with countable joins\index{join!in a lattice} in which binary meets distribute over countable
  joins. (The initial $\sigma$-frame cannot be the two-point lattice $\bool$ because
  $\bool$ is not closed under countable joins, unless we assume excluded middle.) This
  would lead to a construction of~$\Omega$ as a higher inductive-inductive type, but one
  experiment of this kind in \autoref{sec:cauchy-reals} is enough.
\end{enumerate}

In all of the above cases $\Omega$ is a set.
%
Without further ado, we translate the informal definition into type theory.
Throughout this chapter, we use the
logical notation from \autoref{defn:logical-notation}.

\begin{defn} \label{defn:dedekind-reals}
  A \define{Dedekind cut}
  \indexsee{Dedekind!cut}{cut, Dedekind}%
  \indexdef{cut!Dedekind}%
  is a pair $(L, U)$ of mere predicates $L : \Q \to \Omega$ and $U
  : \Q \to \Omega$ which is:
  %
  \begin{enumerate}
  \item \emph{inhabited:} $\exis{q : \Q} L(q)$ and $\exis{r : Q} U(r)$,
  \item \emph{rounded:} for all $q, r : \Q$,
    \index{rounded!Dedekind cut}
    %
    \begin{align*}
      L(q) &\Leftrightarrow \exis{r : \Q} (q < r) \land L(r)
      \qquad\text{and}\\
      U(r) &\Leftrightarrow \exis{q : \Q} (q < r) \land U(q),
    \end{align*}
  \item \emph{disjoint:} $\lnot (L(q) \land U(q))$ for all $q : \Q$,
  \item \emph{located:} $(q < r) \Rightarrow L(q) \lor U(r)$ for all $q, r : \Q$.
  \index{locatedness}%
  \end{enumerate}
  %
  We let $\dcut(L, U)$ denote the conjunction of these conditions. The type of
  \define{Dedekind reals} is
  \indexsee{Dedekind!real numbers}{real numbers, De\-de\-kind}%
  \indexdef{real numbers!Dedekind}%
  %
  \begin{equation*}
    \RD \defeq \setof{ (L, U) : (\Q \to \Omega) \times (\Q \to \Omega) | \dcut(L,U)}.
  \end{equation*}
\end{defn}

It is apparent that $\dcut(L, U)$ is a mere proposition, and since $\Q \to \Omega$ is a
set the Dedekind reals form a set too. See
\autoref{ex:RD-extended-reals,ex:RD-lower-cuts,ex:RD-interval-arithmetic} for variants of
Dedekind cuts which lead to extended reals, lower and upper reals, and the interval
domain.

There is an embedding $\Q \to \RD$ which associates with each rational $q : \Q$ the cut
$(L_q, U_q)$ where
%
\begin{equation*}
  L_q(r) \defeq (r < q)
  \qquad\text{and}\qquad
  U_q(r) \defeq (q < r).
\end{equation*}
%
We shall simply write $q$ for the cut $(L_q, U_q)$ associated with a rational number.


\end{document}
