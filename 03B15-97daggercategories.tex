\documentclass[12pt]{article}
\usepackage{pmmeta}
\pmcanonicalname{97daggercategories}
\pmcreated{2013-11-06 16:41:16}
\pmmodified{2013-11-06 16:41:16}
\pmowner{PMBookProject}{1000683}
\pmmodifier{PMBookProject}{1000683}
\pmtitle{9.7 $\dagger$-categories}
\pmrecord{1}{}
\pmprivacy{1}
\pmauthor{PMBookProject}{1000683}
\pmtype{Feature}
\pmclassification{msc}{03B15}

\usepackage{xspace}
\usepackage{amssyb}
\usepackage{amsmath}
\usepackage{amsfonts}
\usepackage{amsthm}
\newcommand{\blank}{\mathord{\hspace{1pt}\text{--}\hspace{1pt}}}
\newcommand{\defeq}{\vcentcolon\equiv}  
\newcommand{\define}[1]{\textbf{#1}}
\newcommand{\dgr}[1]{{#1}^{\dagger}}
\newcommand{\idtoiso}{\ensuremath{\mathsf{idtoiso}}\xspace}
\newcommand{\indexdef}[1]{\index{#1|defstyle}}   
\newcommand{\indexsee}[2]{\index{#1|see{#2}}}    
\newcommand{\inv}[1]{{#1}^{-1}}
\newcommand{\refl}[1]{\ensuremath{\mathsf{refl}_{#1}}\xspace}
\newcommand{\uhilb}{\ensuremath{\mathcal{H}ilb}\xspace}
\newcommand{\unitaryiso}{\mathrel{\cong^\dagger}}
\newcommand{\urel}{\ensuremath{\mathcal{R}el}\xspace}
\newcommand{\vcentcolon}{:\!\!}
\newcounter{mathcount}
\setcounter{mathcount}{1}
\newtheorem{predefn}{Definition}
\newenvironment{defn}{\begin{predefn}}{\end{predefn}\addtocounter{mathcount}{1}}
\renewcommand{\thepredefn}{9.7.\arabic{mathcount}}
\newtheorem{preeg}{Example}
\newenvironment{eg}{\begin{preeg}}{\end{preeg}\addtocounter{mathcount}{1}}
\renewcommand{\thepreeg}{9.7.\arabic{mathcount}}
\newtheorem{prelem}{Lemma}
\newenvironment{lem}{\begin{prelem}}{\end{prelem}\addtocounter{mathcount}{1}}
\renewcommand{\theprelem}{9.7.\arabic{mathcount}}
\let\autoref\cref

\begin{document}
It is also worth mentioning a useful kind of precategory whose type of objects is not a set, but which is not a category either.

\begin{defn}
  A \define{$\dagger$-precategory}
  \indexdef{.dagger-precategory@$\dagger$-precategory}%
  \indexdef{precategory!.dagger-@$\dagger$-}%
  is a precategory $A$ together with the following.
  \begin{enumerate}
  \item For each $x,y:A$, a function $\dgr{(-)}:\hom_A(x,y) \to \hom_A(y,x)$.
  \item For all $x:A$, we have $\dgr{(1_x)} = 1_x$.
  \item For all $f,g$ we have $\dgr{(g\circ f)} = \dgr f \circ \dgr g$.
  \item For all $f$ we have $\dgr{(\dgr f)} = f$.
  \end{enumerate}
\end{defn}

\begin{defn}\label{ct:unitary}
  A morphism $f:\hom_A(x,y)$ in a $\dagger$-precategory is \define{unitary}
  \indexdef{.dagger-precategory@$\dagger$-precategory!unitary morphism in}%
  \indexdef{unitary morphism}%
  \indexdef{morphism!unitary}%
  \indexdef{isomorphism!unitary}%
  if $\dgr f \circ f = 1_x$ and $f\circ \dgr f = 1_y$.
\end{defn}

Of course, every unitary morphism is an isomorphism, and being unitary is a mere proposition.
Thus for each $x,y:A$ we have a set of unitary isomorphisms from $x$ to $y$, which we denote $(x\unitaryiso y)$.

\begin{lem}\label{ct:idtounitary}
  If $p:(x=y)$, then $\idtoiso(p)$ is unitary.
\end{lem}
\begin{proof}
  By induction, we may assume $p$ is $\refl x$.
  But then $\dgr{(1_x)} \circ 1_x = 1_x\circ 1_x = 1_x$ and similarly.
\end{proof}

\begin{defn}
  A \define{$\dagger$-category}
  \indexdef{.dagger-category@$\dagger$-category}%
  is a $\dagger$-precategory such that for all $x,y:A$, the function
  \[ (x=y) \to (x \unitaryiso y) \]
  from \autoref{ct:idtounitary} is an equivalence.
\end{defn}

\begin{eg}
  The category \urel from \autoref{ct:rel} becomes a $\dagger$-pre\-cat\-e\-go\-ry if we define $(\dgr R)(y,x) \defeq R(x,y)$.
  The proof that \urel is a category actually shows that every isomorphism is unitary; hence \urel is also a $\dagger$-category.
\end{eg}

\begin{eg}
  Any groupoid becomes a $\dagger$-category if we define $\dgr f \defeq \inv{f}$.
\end{eg}

\begin{eg}\label{ct:hilb}
  Let \uhilb be the following precategory.
  \begin{itemize}
  \item Its objects are finite-dimensional \index{finite!-dimensional vector space} vector spaces\index{vector!space} equipped with an inner product $\langle \blank,\blank\rangle$.
  \item Its morphisms are arbitrary linear maps.
    \index{function!linear}%
    \indexsee{linear map}{function, linear}%
  \end{itemize}
  By standard linear algebra, any linear map $f:V\to W$ between finite
  dimensional inner product spaces has a uniquely defined adjoint\index{adjoint!linear map} $\dgr f:W\to V$, characterized by $\langle f v,w\rangle = \langle v,\dgr f w\rangle$.
  In this way, \uhilb becomes a $\dagger$-precategory.
  Moreover, a linear isomorphism is unitary precisely when it is an \define{isometry},
  \indexdef{isometry}%
  i.e.\ $\langle fv,fw\rangle = \langle v,w\rangle$.
  It follows from this that \uhilb is a $\dagger$-category, though it is not a category (not every linear isomorphism is unitary).
\end{eg}

There has been a good deal of general theory developed for $\dagger$-cat\-e\-gor\-ies under classical\index{mathematics!classical}\index{classical!category theory} foundations.
It was observed early on that the unitary isomorphisms, not arbitrary isomorphisms, are the correct notion of ``sameness'' for objects of a $\dagger$-category, which has caused some consternation among category theorists.
Homotopy type theory resolves this issue by identifying $\dagger$-categories, like strict categories, as simply a different kind of precategory.



\end{document}
