\documentclass[12pt]{article}
\usepackage{pmmeta}
\pmcanonicalname{SubsetsOfCountableSetsAreCountable}
\pmcreated{2013-03-22 15:45:56}
\pmmodified{2013-03-22 15:45:56}
\pmowner{beke}{12826}
\pmmodifier{beke}{12826}
\pmtitle{subsets of countable sets are countable}
\pmrecord{7}{37721}
\pmprivacy{1}
\pmauthor{beke}{12826}
\pmtype{Corollary}
\pmcomment{trigger rebuild}
\pmclassification{msc}{03E10}
%\pmkeywords{countable}

% this is the default PlanetMath preamble.  as your knowledge
% of TeX increases, you will probably want to edit this, but
% it should be fine as is for beginners.

% almost certainly you want these
\usepackage{amssymb}
\usepackage{amsmath}
\usepackage{amsfonts}

% used for TeXing text within eps files
%\usepackage{psfrag}
% need this for including graphics (\includegraphics)
%\usepackage{graphicx}
% for neatly defining theorems and propositions
\usepackage{amsthm}
% making logically defined graphics
%%%\usepackage{xypic}

% there are many more packages, add them here as you need them

% define commands here
\newtheorem{thm}{Theorem}
\begin{document}
The definition of countable sets would not serve us well if it did not conform with our intuition about countable sets. So let us prove that countability is in a sense hereditary.
\begin{thm}  Every subset of a countable set is itself countable. \end{thm}
\begin{proof}
Let $B\subseteq A$ and $A$ countable with $f:A\rightarrow K$, $K\subseteq \mathbb{N}$ a bijective function as in the definition of countable sets.

Let us consider $f|_B$, the function  $f$ restricted to $B$, i.e. $f|_B: B \rightarrow f(B)$. Then $f|_B$ is trivially onto, but also one-to-one ($f$ was one-to-one!). So we have a bijective function from $B$ onto $f(B)\subseteq K \subseteq \mathbb{N}$, which \PMlinkescapetext{completes} the proof.
\end{proof}
%%%%%
%%%%%
\end{document}
