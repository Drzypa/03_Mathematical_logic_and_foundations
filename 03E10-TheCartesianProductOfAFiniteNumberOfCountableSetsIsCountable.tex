\documentclass[12pt]{article}
\usepackage{pmmeta}
\pmcanonicalname{TheCartesianProductOfAFiniteNumberOfCountableSetsIsCountable}
\pmcreated{2013-03-22 15:19:45}
\pmmodified{2013-03-22 15:19:45}
\pmowner{BenB}{9643}
\pmmodifier{BenB}{9643}
\pmtitle{the Cartesian product of a finite number of countable sets is countable}
\pmrecord{13}{37142}
\pmprivacy{1}
\pmauthor{BenB}{9643}
\pmtype{Theorem}
\pmcomment{trigger rebuild}
\pmclassification{msc}{03E10}
\pmsynonym{The product of a finite number of countable sets is countable}{TheCartesianProductOfAFiniteNumberOfCountableSetsIsCountable}
%\pmkeywords{Cardinality}
%\pmkeywords{countable}
%\pmkeywords{cartesian product}
%\pmkeywords{cross product}
\pmrelated{CardinalityOfACountableUnion}
\pmrelated{AlgebraicNumbersAreCountable}
\pmrelated{CardinalityOfTheRationals}

\endmetadata

% this is the default PlanetMath preamble.  as your knowledge
% of TeX increases, you will probably want to edit this, but
% it should be fine as is for beginners.

% almost certainly you want these
\usepackage{amssymb}
\usepackage{amsmath}
\usepackage{amsfonts}

% used for TeXing text within eps files
%\usepackage{psfrag}
% need this for including graphics (\includegraphics)
%\usepackage{graphicx}
% for neatly defining theorems and propositions
%\usepackage{amsthm}
% making logically defined graphics
%%%\usepackage{xypic}

% there are many more packages, add them here as you need them

% define commands here

\begin{document}
\newtheorem{thm}{Theorem}
\begin{thm}
The Cartesian product of a finite number of countable sets is countable.
\end{thm}

\emph{Proof:}
Let $A_1, A_2, \ldots, A_n$ be countable sets and let
$S = A_1 \times A_2 \times \cdots \times A_n$.
Since each $A_i$ is countable, there exists an injective function
$f_i\colon A_i \to \mathbb{N}$.
The function $h\colon S \to \mathbb{N}$ defined by
\[
  h(a_1, a_2, \ldots a_n) = \prod_{i=1}^n p_i^{f_i(a_i)}
\]
where $p_i$ is the $i$th prime
is, by the fundamental theorem of arithmetic, a bijection between
$S$ and a subset of $\mathbb{N}$ and therefore $S$ is also countable.

Note that this result does \emph{not} (in general) extend to the
Cartesian product of a countably infinite collection of countable
sets.  If such a collection contains more than a finite number of sets
with at least two elements, then Cantor's diagonal argument can be
used to show that the product is not countable.

For example, given $B = \{0, 1\}$, the set $F = B \times B \times
\cdots$ consists of all countably infinite sequences of zeros and ones.
Each element of $F$ can be viewed as a binary fraction and can
therefore be mapped to a unique
real number in $[0, 1)$ and $[0, 1)$ is, of course, not countable.

%%%%%
%%%%%
\end{document}
