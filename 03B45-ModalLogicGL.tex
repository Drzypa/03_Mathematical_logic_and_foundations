\documentclass[12pt]{article}
\usepackage{pmmeta}
\pmcanonicalname{ModalLogicGL}
\pmcreated{2013-03-22 19:35:33}
\pmmodified{2013-03-22 19:35:33}
\pmowner{CWoo}{3771}
\pmmodifier{CWoo}{3771}
\pmtitle{modal logic GL}
\pmrecord{19}{42582}
\pmprivacy{1}
\pmauthor{CWoo}{3771}
\pmtype{Definition}
\pmcomment{trigger rebuild}
\pmclassification{msc}{03B45}
\pmclassification{msc}{03F45}

\usepackage{amssymb,amscd}
\usepackage{amsmath}
\usepackage{amsfonts}
\usepackage{mathrsfs}
\usepackage{proof}
\usepackage{bussproofs}

% used for TeXing text within eps files
%\usepackage{psfrag}
% need this for including graphics (\includegraphics)
%\usepackage{graphicx}
% for neatly defining theorems and propositions
\usepackage{amsthm}
% making logically defined graphics
%%\usepackage{xypic}
\usepackage{pst-plot}
\usepackage{multicol}
\usepackage{enumerate}
\usepackage{tabls}

% define commands here
\newcommand*{\abs}[1]{\left\lvert #1\right\rvert}
\newtheorem{prop}{Proposition}
\newtheorem{thm}{Theorem}
\newtheorem{lem}{Lemma}
\newtheorem{cor}{Corollary}
\newtheorem{ex}{Example}

\begin{document}
The modal logic \textbf{GL} (after G\"odel and L\"ob) is the smallest normal modal logic containing the following schema:
\begin{itemize}
\item W: $\square (\square A \to A ) \to \square A$.
\end{itemize}
\textbf{GL} is also known as \emph{provability logic}, because it is used to study the provability and consistency of first order Peano arithmetic.

Recall that 4 is the schema $\square A \to \square \square A$.
\begin{prop} In any normal modal logic, $\vdash W$ implies $\vdash 4$. \end{prop}
The proof of this requires some \PMlinkname{theorems}{SomeTheoremSchemasOfNormalModalLogic} and \PMlinkname{meta-theorems}{SyntacticPropertiesOfANormalModalLogic} of a normal modal logic.
\begin{proof}  We start with the tautology $A \to ((\square \square A \land \square A) \to (\square A \land A))$, which an instance of the schema $X \to ((Y\land Z)\to (Z \land X))$.  Since $\square (\square A \land A) \leftrightarrow \square \square A \land \square A$ is a theorem in any normal modal logic, $A \to (\square (\square A \land A) \to (\square A\land A))$ is a theorem by the substitution theorem.  By the syntactic property RM, $\square A \to \square (\square (\square A \land A) \to (\square A\land A))$ is a theorem.  Since $\square (\square (\square A \land A) \to (\square A\land A)) \to \square (\square A \land A)$ is an instance of W, by law of syllogism, $\square A \to \square (\square A \land A)$ is a theorem.

Next, from the tautology $\square A \land A \to \square A$, we have the theorem $\square (\square A\land A) \to \square \square A$ by RM.  Combining this with the last theorem in the previous paragraph, we see that, by law of syllogism, $\square A \to \square \square A$, or 4, is a theorem.
\end{proof}

\begin{cor} 4 is a theorem of \textbf{GL}. \end{cor}

A binary relation is said to be \emph{converse well-founded} iff its inverse is well-founded.
\begin{prop} W is valid in a frame $\mathcal{F}$ iff $\mathcal{F}$ is transitive and converse well-founded. \end{prop}
\begin{proof}  
Suppose first that the schema W is valid in $\mathcal{F}=(U,R)$, then any theorem of \textbf{GL} is valid in $\mathcal{F}$, so in particular 4 is valid in $\mathcal{F}$, and hence $\mathcal{F}$ is transitive (see \PMlinkname{here}{ModalLogicS4}).  We next show that $R$ is converse well-founded.  Suppose not.  Then there is a non-empty subset $S\subseteq U$ such that $S$ has no $R$-maximal element.  We want to find a model $(U,R,V)$ such that, for some propositional variable $p$ and some world $u$ in $U$, $\not \models_u \square (\square p \to p) \to \square p$, or equivalently, $\models_u \square (\square p\to p)$ and $\not \models_u \square p$.  Let $V$ be the valuation such that $V(p):=\lbrace w \in U \mid w \notin S \rbrace$.  Pick any $u\in S$.  Suppose $u R v$.  To show that $\models_u \square (\square p\to p)$, we want to show that $\models_v \square p \to p$.  There are two cases: 
\begin{itemize}
\item
If $v\in S$, then $\not \models_v p$.  Furthermore, since $S$ does not contain an $R$-maximal element, there is a $w\in S$ such that $v R w$.  Since $w\in S$, $\not\models_w p$.  Since $v R w$, $\not \models_v \square p$.  As a result, $\models_v \square p \to p$.
\item
If $v\notin S$, then $\models_v p$, so that $\models_v \square p \to p$.
\end{itemize}
Next, we want to show that $\not \models_u \square p$.  Since $u\in S$, and $S$ does not have an $R$-maximal element, there is a $w\in S$ such that $u R w$.  Since $w\in S$, $\not \models_w p$.  But since $u R w$, $\not \models_u \square p$.

Conversely, let $\mathcal{F}$ be a transitive and converse well-founded frame, $M$ a model based on $\mathcal{F}$, and $u$ a world in $M$.  We want to show that $\models_u \square (\square p\to p) \to \square p$.  So suppose $\not \models_u \square p$.  Then the set $S:=\lbrace v\mid u R v \mbox{ and } \not \models_v p \rbrace$ is not empty.  Since $R$ is converse well-founded, $S$ has a $R$-maximal element, say $w$.  So $u R w$ and $\not \models_w p$.  Now, if $\models_w \square p \to p$, then $\not \models_w \square p$, which means there is a $v$ such that $w R v$ and $\not \models_v p$.  But since $R$ is transitive and $u R w$, we get $u R v$, implying $v \in S$, contradicting the $R$-maximality of $w$.  Therefore, $\not \models_w \square p \to p$, or $\not \models_u \square (\square p \to p)$.  As a result, $\models_u \square (\square p \to p) \to \square p$.
\end{proof}
Proposition 2 immediately implies
\begin{cor} \textbf{GL} is sound in the class of transitive and converse well-founded frames. \end{cor}

\textbf{Remark}.  However, unlike many other modal logics, \textbf{GL} \emph{is not} complete in the class of transitive and converse well-founded frames.  While its canonical model (hence the corresponding canonical frame) is transitive (because 4 is valid in it), it is not converse well-founded.

Instead, it can be shown that \textbf{GL} \emph{is} complete in the restricted class of finite transitive and converse well-founded frames, or equivalently, finite transitive and irreflexive frames.

%%%%%
%%%%%
\end{document}
