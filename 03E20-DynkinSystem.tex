\documentclass[12pt]{article}
\usepackage{pmmeta}
\pmcanonicalname{DynkinSystem}
\pmcreated{2013-03-22 12:21:19}
\pmmodified{2013-03-22 12:21:19}
\pmowner{mathwizard}{128}
\pmmodifier{mathwizard}{128}
\pmtitle{Dynkin system}
\pmrecord{9}{32024}
\pmprivacy{1}
\pmauthor{mathwizard}{128}
\pmtype{Definition}
\pmcomment{trigger rebuild}
\pmclassification{msc}{03E20}
\pmclassification{msc}{28A60}
\pmrelated{DynkinsLemma}

\usepackage{amssymb}
\usepackage{amsmath}
\usepackage{amsfonts}

% used for TeXing text within eps files
%\usepackage{psfrag}
% need this for including graphics (\includegraphics)
%\usepackage{graphicx}
% for neatly defining theorems and propositions
%\usepackage{amsthm}
% making logically defined graphics
%%%\usepackage{xypic} 

% there are many more packages, add them here as you need them

% define commands here
\newcommand{\md}{d}
\newcommand{\mv}[1]{\mathbf{#1}}	% matrix or vector
\newcommand{\mvt}[1]{\mv{#1}^{\mathrm{T}}}
\newcommand{\mvi}[1]{\mv{#1}^{-1}}
\newcommand{\mderiv}[1]{\frac{\md}{\md {#1}}} %d/dx
\newcommand{\mnthderiv}[2]{\frac{\md^{#2}}{\md {#1}^{#2}}} %d^n/dx
\newcommand{\mpderiv}[1]{\frac{\partial}{\partial {#1}}} %partial d^n/dx
\newcommand{\mnthpderiv}[2]{\frac{\partial^{#2}}{\partial {#1}^{#2}}} %partial d^n/dx
\newcommand{\borel}{\mathfrak{B}}
\newcommand{\integers}{\mathbb{Z}}
\newcommand{\rationals}{\mathbb{Q}}
\newcommand{\reals}{\mathbb{R}}
\newcommand{\complexes}{\mathbb{C}}
\newcommand{\naturals}{\mathbb{N}}
\newcommand{\defined}{:=}
\newcommand{\var}{\mathrm{var}}
\newcommand{\cov}{\mathrm{cov}}
\newcommand{\corr}{\mathrm{corr}}
\newcommand{\set}[1]{\{#1\}}
\newcommand{\powerset}[1]{\mathcal{P}(#1)}
\newcommand{\bra}[1]{\langle#1 \vert}
\newcommand{\ket}[1]{\vert \hspace{1pt}#1\rangle}
\newcommand{\braket}[2]{\langle #1 \ket{#2}}
\begin{document}
Let $\Omega$ be a set, and $\powerset{\Omega}$ be the power set of $\Omega$.  A \emph{Dynkin system} on $\Omega$ is a set $\mathcal{D} \subset \powerset{\Omega}$ such that

\begin{enumerate}
\item{$\Omega \in \mathcal{D}$}
\item{$A,B \in \mathcal{D} \text{ and } A \subset B \Rightarrow B \setminus A \in \mathcal{D}$}
\item{$A_n \in \mathcal{D},\ A_n \subset A_{n+1},\ n \ge 1 \Rightarrow \bigcup_{k=1}^{\infty}A_k \in \mathcal{D}$}.
\end{enumerate}

Let $F \subset \powerset{\Omega}$, and consider
\begin{equation}
\Gamma = \set{X : X\subset\powerset{\Omega} \text{ is a Dynkin system and } F \subset X}.
\end{equation}
We define the intersection of all the Dynkin systems containing $F$ as

\begin{equation}
\mathcal{D}(F) \defined \bigcap_{X \in \Gamma} X
\end{equation}

One can easily verify that $\mathcal{D}(F)$ is itself a Dynkin system and that it contains $F$.  We call $\mathcal{D}(F)$ the \emph{Dynkin system generated by $F$}.  It is the ``smallest'' Dynkin system containing $F$.

A Dynkin system which is also \PMlinkname{$\pi$-system}{PiSystem} is a \PMlinkname{$\sigma$-algebra}{SigmaAlgebra}.
%%%%%
%%%%%
\end{document}
