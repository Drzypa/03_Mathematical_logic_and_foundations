\documentclass[12pt]{article}
\usepackage{pmmeta}
\pmcanonicalname{ModelTheory}
\pmcreated{2013-03-22 16:27:30}
\pmmodified{2013-03-22 16:27:30}
\pmowner{mps}{409}
\pmmodifier{mps}{409}
\pmtitle{model theory}
\pmrecord{7}{38616}
\pmprivacy{1}
\pmauthor{mps}{409}
\pmtype{Topic}
\pmcomment{trigger rebuild}
\pmclassification{msc}{03C99}
\pmrelated{AxiomaticAndCategoricalFoundationsOfMathematicsII2}

\endmetadata

% this is the default PlanetMath preamble.  as your knowledge
% of TeX increases, you will probably want to edit this, but
% it should be fine as is for beginners.

% almost certainly you want these
\usepackage{amssymb}
\usepackage{amsmath}
\usepackage{amsfonts}

% used for TeXing text within eps files
%\usepackage{psfrag}
% need this for including graphics (\includegraphics)
%\usepackage{graphicx}
% for neatly defining theorems and propositions
%\usepackage{amsthm}
% making logically defined graphics
%%%\usepackage{xypic}

% there are many more packages, add them here as you need them

% define commands here

\begin{document}
\PMlinkescapeword{structure}
\PMlinkescapeword{structures}
\PMlinkescapeword{branch}
\PMlinkescapeword{rank}

\emph{Model theory} is a branch of mathematical logic which deals with the general problem of classifying mathematical structures.  While every branch of mathematics has its own classifying problem (such as the classification of \PMlinkname{finite simple groups}{ExamplesOfFiniteSimpleGroups} or the classification of smooth manifolds), the approach of model theory is unique in attempting to classify structures in general by the sentences which are true of those structures.  Like most classification problems, this problem is almost certainly unsolvable when stated in full generality.  However, many special cases are worthy of study.

The basic theoretical notions in model theory are the structure and the theory.  A \emph{\PMlinkname{structure}{StructuresAndSatisfaction}}, sometimes called an $L$-structure, is a set with associated \PMlinkname{symbols}{Signature} representing constants, relations, and functions.  A \emph{theory} is a collection of sentences in a formal language.  We say that a structure $\mathfrak{K}$ \emph{satisfies} a sentence $\varphi$ provided that the sentence is true no matter how its variables are interpreted in $\mathfrak{K}$.  If $\mathfrak{K}$ satisfies all sentences of a theory $T$, then we say that $\mathfrak{K}$ is a \emph{model} or $T$, or $\mathfrak{K}$ \emph{models} $T$, and write $\mathfrak{K}\models T$.  This allows us to talk about the class of models of a particular theory or the theory of a particular structure.

Basic constructions in model theory include elementary extensions, ultrapowers and ultraproducts, and the elimination of imaginaries.  Some of the basic results of model theory include the \PMlinkname{L\"owenheim-Skolem theorem}{DownwardLowenheimSkolemTheorem} (in downward and \PMlinkname{upward}{UpwardLowenheimSkolemTheorem} forms), the \PMlinkname{compactness theorem for first-order logic}{CompactnessTheoremForFirstOrderLogic}, and \L o\'s's theorem.  Important concepts include Morley rank, o-minimality, and quantifier elimination.

Early pioneers of model theory include L\"owenheim, Skolem, G\"odel, Tarski, and Maltsev, and the work of Henkin, Robinson, and (again) Tarski helped distinguish model theory from the rest of mathematical logic.  Ax and Kochen used model theory to prove a result on Diophantine problems over function fields, and Hrushovski used model theory to prove the Mordell--Lang conjecture for function fields.

\begin{thebibliography}{99}
\bibitem{CK}
  C.~C.~Chang and H.~J.~Keisler, {\it Model theory}, North-Holland Publishing Company, Amsterdam, 1973.

\bibitem{H}
  W.~Hodges, {\it A shorter model theory}, Cambridge University Press, 1997 (2000).

\bibitem{M}
  M.~Manzano, {\it Model theory}, Oxford Logic Guides 37, Clarendon Press, Oxford, 1999.

\bibitem{P}
  B.~Poizat, {\it Course in model theory: an introduction in contemporary mathematical logic}, Springer Verlag, 1999.
\end{thebibliography}

%%%%%
%%%%%
\end{document}
