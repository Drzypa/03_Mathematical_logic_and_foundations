\documentclass[12pt]{article}
\usepackage{pmmeta}
\pmcanonicalname{Logicism}
\pmcreated{2013-03-22 18:21:54}
\pmmodified{2013-03-22 18:21:54}
\pmowner{gribskoff}{21395}
\pmmodifier{gribskoff}{21395}
\pmtitle{logicism}
\pmrecord{15}{41004}
\pmprivacy{1}
\pmauthor{gribskoff}{21395}
\pmtype{Topic}
\pmcomment{trigger rebuild}
\pmclassification{msc}{03A05}
\pmclassification{msc}{03-01}
\pmsynonym{analytics}{Logicism}
\pmsynonym{ontology}{Logicism}
\pmsynonym{formal logics}{Logicism}
\pmsynonym{algebraic category}{Logicism}
\pmsynonym{logic algebras}{Logicism}
%\pmkeywords{analytic}
%\pmkeywords{theory of descriptions}
%\pmkeywords{cardinal numbers}
%\pmkeywords{theory of types}
\pmrelated{FormalLogicsAndMetaMathematics}
\pmrelated{AlgebraicCategoryOfLMnLogicAlgebras}
\pmrelated{IntuitionisticLogic}
\pmrelated{CategoricalOntology}
\pmrelated{InclusionOfClassicalIntoIntuitionisticLogic}
\pmrelated{InterpretationOfIntuitionisticLogicByMeansOfFunctionals}
\pmrelated{Predicativism}
\pmrelated{MathematicalPlatonism}
\pmrelated{Gene}
\pmdefines{logical foundations}

% this is the default PlanetMath preamble.  as your knowledge
% of TeX increases, you will probably want to edit this, but
% it should be fine as is for beginners.

% almost certainly you want these
\usepackage{amssymb}
\usepackage{amsmath}
\usepackage{amsfonts}

% used for TeXing text within eps files
%\usepackage{psfrag}
% need this for including graphics (\includegraphics)
%\usepackage{graphicx}
% for neatly defining theorems and propositions
%\usepackage{amsthm}
% making logically defined graphics
%%%\usepackage{xypic}

% there are many more packages, add them here as you need them

% define commands here

\begin{document}
In the literature on the foundations of mathematics logicism aims at proving that the propositions of pure mathematics can be derived from the propositions of logic.

Although the theory is mainly associated with the names of Frege and Russell as their first proponents, the conception of a reduction of Mathematics to Logic can be traced back to Leibniz.

Leibniz's idea can be delineated as follows: starting with a distinction between 'truth from reason' and 'truth from fact',  he argues that mathematical and logical truths are both forms of 'truth from reason' and thereby rest on what he calls the principle of non-contradiction. This principle had for Leibniz a form of evidence beyond reasonable doubt and he coined the term 'identical proposition' to refer to it. 

In particular a truth from reason is a predicative proposition in the normal form: 

\begin{center}
'$S$ is included in $S$ or $P$'
\end{center}

in which $S$ is the subject, $S$ or $P$ the predicate and the copula is 'is included in'. For such  propositions we have a rule of substitution \emph{salva veritate} for the terms which occur in predicate position, in such a way that we are then led to recognize the inclusion of the subject in the predicate and indeed with the already mentioned degree of evidence. 

It is this process of reiterated substitution that Leibniz calls a reduction. Thus if a mathematical proposition P is given, and its logical content is not evident, it is possible, after a finite number of substitutions salva veritate,to reduce it to an equivalent proposition P* whose logical content is now evident.

Leibniz conceived his identical propositions as being what we call today the tautologies of propositional calculus, and his examples of propositions of this kind are the law of double negation and the principle of contradiction. They belong to the class of logical  propositions true by reason alone and it was to this class that the propositions of mathematics could be proved to be reducible.

In the logicist program of Frege and Russell two aspects of Leibniz conception are preserved but the space is somewhat changed. Frege replaced Leibniz idea of an identical proposition, the one in which the inclusion of the subject in the predicate can be proved to be evident in a finite number of steps, by his notion of an 'analytical proposition'. For Frege a proposition $P$ is analytical if having as starting formulas laws of logic or definitions we can construct a derivation of $P$ in  which $P$ is the derivation's $s$ last step.

The second aspect of Leibniz conception  redefined by Frege relates to the role played by evidence. For Frege a proof establishes a proposition as analytical if the premises of the proof are laws of logic and the rules of inference are explicitly given. Thus his doctrine of the analytical character of the propositions of arithmetic depends on the fact  that the laws of logic and the rules of inference are not given as evidence but have to be  previously specified.

Frege soon realized that he had to create a symbolic system in which not only the concepts of mathematics but also the machinery of deductive reasoning in general had to be represented. In such a system a step  in a proof can be seen as a transformation of one or more expressions of the system, provided the transformation is performed according to the previously formulated rules.

Thus for example a proof of the analytical character of a proposition

\begin{center}
$P$ = '$1 + 1 = 2$'
\end{center}

would start with formulas in which only  logical symbols occur, with propositional variables and propositional functions, and then  various actions on these formulas performed according to the rules would then  produce in the end  '$1 + 1 = 2$'. Only the proof as a whole  can  secure the  logical character of the last formula.

To justify the transition of the obvious logical character of a step to a less obvious logical character of  another step(in the same proof), the logicist program uses the resources of definition, by means of which the less obvious logical symbols can be introduced.
 
In the \emph{Principia Mathematica} definition is conceived as notational tactics. It merely asserts that a symbol or a set of symbols has the same meaning as another set of symbols whose meaning is already known. It functions as an eliminability clause of the definiendum term and the value of the definiens  consists in the fact that it alone makes possible an analysis of the concept to be defined.
 
This kind of definition, known as contextual definition, does not assume the existence of the object to be defined and a fortiori does not create it. The situation is analogous to that of pronominal reference in natural languages, where words like 'nobody' in contexts like:

\begin{center}
'Nobody rides faster than Siegfried'
\end{center}
 
can be eliminated in favour of:

\begin{center}
'Siegfried is the fastest rider',
\end{center}

where the pronoun 'nobody' no longer occurs and has  vanishing reference.

More difficult than pronominal reference seemed to Russell to be the question of the denotation of terms like 

\begin{center}
'the even integer which is prime'
\end{center}

and

\begin{center}
'the class of the positive integers'
\end{center}
 
which prima facie seem to imply the existence of the objects to which a certain predicate is ascribed.
Since according to the program's key strategical option, one has to deduce the propositions of arithmetic from the propositions of logic, and these last ones for Russell have no content,one also necessarily  has to prove that those terms, which seem to denote objects, when they occur in the derivation of arithmetic from logic, can also be eliminated.

This is the purpose of another famous creation of Russell which he called the "theory of descriptions". A description is an expression in the general form:

\begin{center} 
'the $x$ such that $F (x)$'
\end{center}

in which the definite article appears to imply the existence of the denoted object. The theory provides a method for the elimination of such expressions. In essence the theory aims at showing that the meaning of such descriptive terms is perfectly captured by expressions of the predicate calculus in which the descriptive terms no longer occur. A term like:

\begin{center}
'the $x$ such that $F (x)$'
\end{center}

known as a definite description, has a logical content which is independent of the fact that it denotes an object. The theory secures the truth value True for a proposition like:

\begin{center}
'The author of the \emph{Organon} was Greek'
\end{center}

if and only if the conjunction of the propositions of the predicate calculus in which it can be decomposed is itself true. In gerenal the decomposition takes the form:

$$(\exists x) F (x) \wedge (\forall x) (\forall y) [ (F (x) \wedge F (y)) \rightarrow  x = y].$$

When Russell turned to the existence of classes, as it seems to be implied by expressions of the general form:

\begin{center} 
'the class of all $x$ such that $F (x)$',
\end{center}

he decided to avoid the problem by using the expedient of contextual definition. Thus classes were not  admitted as real objects and for that reason  his position became to be known as the 'no class theory'.

In his \emph{Grundlagen der Arithmetik} Frege went another way and rejected both the conception and the practice of contextual definition. He favoured instead the so called real definition which purports to define an object which already has an autonomous existence. Frege's example of such an object is the number concept, of which he also says that it is a  logical object. To define an object is not to create it but to reveal it as an autonomous entity, and such a task is beyond the scope of contextual definition.

For this reason Frege also rejects both the theory and practice of the formalists of his day, as we can see in the closing sections of \emph{Grundlagen}. For the formalists the existence of mathematical objects, like the imaginary numbers, is simply postulated,with no implication that they really exist. Frege argues that if imaginary or complex numbers really exist their existence is obviously  independent of their being postulated,and if they don't exist they will not be created by being postulated. In the end the purpose of a definition is to reveal to us a class of objects with sharply drawn borderlines so that membership to the class is always a decidable predicate. 

To show that (cardinal) numbers are objects in their own right Frege defines number as a second order predicate,namely a number is for him a predicate of a concept. For example a concept $C$ expressed by a formula like:

\begin{center}
'$x$ is integer and prime'
\end{center}

has the property that it is satisfied by one object only, and can therefore be called a unary-concept. We can then form a 2nd Order expression like $U (C)$ which means that the concept $C$ has the property $U$, of being unary. In general a concept is $n$-ary if it can be satisfied by $n$ objects. Two concepts $M$ and $N$ have the same $n$-arity if there is a function from $M$ onto $N$ with an inverse, i.e, a bijection from $M$ to $N$.

For example, if both $M$ and $N$ have 3-arity then each has 3 elements and therefore they are in the same $n$-arity class with all other concepts satisfied by 3 objects.

It is clear that the relation:

\begin{center}
'$M$ has the same $n$-arity as $N$'
\end{center}

is a partition of the class of concepts and thus defines an equivalence relation,since it is reflexive, symmetric and transitive. The equivalence classes generated by the relation 'having the same $n$-arity' are precisely the (finite) cardinal numbers.

The differences between Frege and Russell regarding the nature of definition and the existence of abstract objects only show that both nominalism and realism are part and parcel of the logicist program. But they had a common goal, which was to prove that mathematics is about concepts that can be entirely defined in terms of a small number of purely logical principles. In this context we must notice that  Frege had an enlarged conception of logic, which included not only the propostional calculus and predicate calculus of first and higher orders and the theory of identity , but also  the theory of classes.

There are two main kinds of difficulties which caused the proof of the analytical character of mathematical propositions not to meet all expectations.

The first was a contradiction created by Frege's attempt to formalize his theory in the system presented in his later work, \emph{Die Grundgesetze der Arithmetik}. As we saw in his definition of number by means of a partition of the class of all concepts, Frege did not question  the size of his classes. In  Axiom V of this system he proposes an axiom to the effect that to every predicate there is always the class of objects that fall under it, called  the range of values of the predicate. 

Russell was able to prove that this Axiom can be used to derive a contradiction, today known as Russell's paradox. Russell's ingenious idea was that if every predicate has an extension then the predicate:
 
\begin{center}
'is not a member of itself'
\end{center}

also must have an extension. If we make $R$ = 'the class of all classes that are not members of themselves', and assuming the decidability requirement already mentioned above, we can proceed  to decide the statement:

\begin{center}
'$R$ belongs to $R$'.
\end{center}

We soon realize that $R$ belongs to $R$ if and only if $R$ does not belong to $R$.

Several strategies have been proposed to block the derivation of Russell's paradox. The two best known are Russell's own theory of types, in which formulas like '$R$ belongs to $R$' are not well-formed and axiomatic set theory, in particular  Zermelo's axiom of subsets (Aussonderung), according to which the existence of a set defined by a property is secured only for subsets of a previously given set.

The second difficulty  has to do wih the analysis of concepts and has therefore of a certain structural significance. In \emph{Grundlagen} Frege tries to prove that every natural number has a successor (Section 76)  and eventually (Section 82) that there is no last element in the natural number sequence. Although the  proof is only a sketch it is obvious that the variables occurring in it have to be interpreted as ranging over an infinite domain . Thus our knowledge of the infinity of the natural number system is analytic only if we previously assume that there is an infinite set,and this assumption is certainly not analytic according to  the definition used above.

Gödel argues that it is the definition of the term 'analytic' which we have been using (in his terminology one should rather say  'tautological') that makes it impossible to prove the analytical character of the axioms of Principa. This definition implies the existence of a decision procedure for all arithmetic problems and Turing proved that there is no such a procedure.

There is however another definition of the term 'analytic' which could be more appealing to  logicist foundations. In this new definition a proposition  is to be called 'analytic' if it is  true only due to the meaning of the concepts occurring in it. In such a definition 'meaning' would have to be considered itself a primitive concept, unable to be further reduced to yet  another more fundamental one. Gödel maintains that if we except the Axiom of Infinity, all other axioms of \emph{Principia} are analytic, at least for some interpretations of the primitive terms.

\begin{thebibliography} {100}

\bibitem{AC} Church, A. \emph{Introduction to Mathematical Logic}, Princeton, 1956.
\bibitem{GF} Frege, G., \emph{Grundlagen der Arithmetik}, Breslau, 1884.
\bibitem{KG} G\"{o}del, K., "Russell's Mathematical Logic", in \emph{The Philosophy of Bertrand Russell}, ed. P. Schilpp, The Library of Living Philosophers, 1944.
\bibitem{WQ} Quine, W., \emph{Mathematical Logic}, Cambridge, M.A., 1955.
\bibitem{BR1} Russell, B., Whitehead, A., \emph{Principia Mathematica}, Cambridge University Press, 1962.
\bibitem{BR2} Russell, B., \emph{Introduction to Mathematical Philosophy}, London, 1993.
\end{thebibliography}

\end{document}

%%%%%
%%%%%
\end{document}
