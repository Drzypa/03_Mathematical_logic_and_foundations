\documentclass[12pt]{article}
\usepackage{pmmeta}
\pmcanonicalname{UniversalRelation}
\pmcreated{2013-03-22 12:58:18}
\pmmodified{2013-03-22 12:58:18}
\pmowner{Henry}{455}
\pmmodifier{Henry}{455}
\pmtitle{universal relation}
\pmrecord{6}{33342}
\pmprivacy{1}
\pmauthor{Henry}{455}
\pmtype{Definition}
\pmcomment{trigger rebuild}
\pmclassification{msc}{03B10}
\pmsynonym{universal}{UniversalRelation}
\pmdefines{universal function}

\endmetadata

% this is the default PlanetMath preamble.  as your knowledge
% of TeX increases, you will probably want to edit this, but
% it should be fine as is for beginners.

% almost certainly you want these
\usepackage{amssymb}
\usepackage{amsmath}
\usepackage{amsfonts}

% used for TeXing text within eps files
%\usepackage{psfrag}
% need this for including graphics (\includegraphics)
%\usepackage{graphicx}
% for neatly defining theorems and propositions
%\usepackage{amsthm}
% making logically defined graphics
%%%\usepackage{xypic}

% there are many more packages, add them here as you need them

% define commands here
%\PMlinkescapeword{theory}
\begin{document}
If $\Phi$ is a class of $n$-ary relations with $\vec{x}$ as the only free variables, an $n+1$-ary formula $\psi$ is \emph{universal} for $\Phi$ if for any $\phi\in\Phi$ there is some $e$ such that $\psi(e,\vec{x})\leftrightarrow\phi(\vec{x})$. In other words, $\psi$ can simulate any element of $\Phi$.

Similarly, if $\Phi$ is a class of function of $\vec{x}$, a formula $\psi$ is universal for $\Phi$ if for any $\phi\in\Phi$ there is some $e$ such that $\psi(e,\vec{x})=\phi(\vec{x})$.
%%%%%
%%%%%
\end{document}
