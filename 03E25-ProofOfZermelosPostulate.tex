\documentclass[12pt]{article}
\usepackage{pmmeta}
\pmcanonicalname{ProofOfZermelosPostulate}
\pmcreated{2013-03-22 16:14:25}
\pmmodified{2013-03-22 16:14:25}
\pmowner{Wkbj79}{1863}
\pmmodifier{Wkbj79}{1863}
\pmtitle{proof of Zermelo's postulate}
\pmrecord{9}{38342}
\pmprivacy{1}
\pmauthor{Wkbj79}{1863}
\pmtype{Proof}
\pmcomment{trigger rebuild}
\pmclassification{msc}{03E25}

\usepackage{amssymb}
\usepackage{amsmath}
\usepackage{amsfonts}

\usepackage{psfrag}
\usepackage{graphicx}
\usepackage{amsthm}
%%\usepackage{xypic}

\begin{document}
The following is a proof that the axiom of choice implies Zermelo's postulate.

\begin{proof}
Let $\mathcal{F}$ be a disjoint family of nonempty sets.  Let $\displaystyle f \colon \mathcal{F} \to \bigcup \mathcal{F}$ be a choice function.  Let $A,B \in \mathcal{F}$ with $A \neq B$.  Since $\mathcal{F}$ is a disjoint family of sets, $\displaystyle A \cap B = \emptyset$.  Since $f$ is a choice function, $f(A) \in A$ and $f(B) \in B$.  Thus, $f(A) \notin B$.  Hence, $f(A) \neq f(B)$.  It follows that $f$ is injective.

Let $\displaystyle C=\left\{f(B) \in \bigcup \mathcal{F} : B \in \mathcal{F} \right\}$.  Then $C$ is a set.

Let $A \in \mathcal{F}$.  Since $f$ is injective, $\displaystyle A \cap C=\{f(A)\}$.
\end{proof}
%%%%%
%%%%%
\end{document}
