\documentclass[12pt]{article}
\usepackage{pmmeta}
\pmcanonicalname{CourseofvaluesRecursion}
\pmcreated{2013-03-22 19:06:13}
\pmmodified{2013-03-22 19:06:13}
\pmowner{CWoo}{3771}
\pmmodifier{CWoo}{3771}
\pmtitle{course-of-values recursion}
\pmrecord{11}{41996}
\pmprivacy{1}
\pmauthor{CWoo}{3771}
\pmtype{Definition}
\pmcomment{trigger rebuild}
\pmclassification{msc}{03D20}
\pmsynonym{course-of-value}{CourseofvaluesRecursion}
\pmsynonym{course-of-values}{CourseofvaluesRecursion}

\usepackage{amssymb,amscd}
\usepackage{amsmath}
\usepackage{amsfonts}
\usepackage{mathrsfs}

% used for TeXing text within eps files
%\usepackage{psfrag}
% need this for including graphics (\includegraphics)
%\usepackage{graphicx}
% for neatly defining theorems and propositions
\usepackage{amsthm}
% making logically defined graphics
%%\usepackage{xypic}
\usepackage{pst-plot}

% define commands here
\newcommand*{\abs}[1]{\left\lvert #1\right\rvert}
\newtheorem{prop}{Proposition}
\newtheorem{thm}{Theorem}
\newtheorem{ex}{Example}
\newcommand{\real}{\mathbb{R}}
\newcommand{\pdiff}[2]{\frac{\partial #1}{\partial #2}}
\newcommand{\mpdiff}[3]{\frac{\partial^#1 #2}{\partial #3^#1}}
\begin{document}
In defining a function by primitive recursion, the value of the next argument $f(n+1)$ depends only on the value of the current argument $f(n)$.  Definition of functions by course-of-values recursion, $f(n+1)$ depends on value(s) of some or all of the preceding arguments $f(n),\ldots, f(0)$.  Two very basic examples of definition by course-of-values recursion are
\begin{enumerate}
\item (Fibonacci numbers).  $F(0)=F(1)=1$, and $F(n+1)=F(n)+F(n-1)$.
\item $f(0)=1$, and $f(n+1)=f(n)+f(n-1)+\cdots + f(0)$.
\end{enumerate}
In the first example, we may write $F(n+1)= g(F(n),F(n-1))$, where $g$ is the familiar addition function, a function of two arguments.  In other words, value of the current argument of $F$ depends on the values of a fixed number of preceding arguments (in this case, $2$).  In the second example, value of the current argument of $f$ depends on the values of all of the values of the preceding arguments.  Suppose we write $f(n+1)=h(f(n),\ldots, f(0))$, where $h$ is the $(n+1)$-ary addition function.  This $h$ is different from $g$ in the sense that $g$ has a fixed arity, whereas the arity of $h$ depends on the argument of $f$.  In other words, as $n$ varies, a ``different'' $h$ is required!  Is it possible to define $f$ via a fixed function as in the first example?  Moreover, what is the relation between course-of-values recursion and primitive recursion?

The answer to the first question is yes.  To get around the difficulty of ``varying arity'', we employ the technique of encoding the entire sequence of values of the preceding arguments of $f$ by a ``code number'', and come up with a function $h$ that depends on the this code number.  $h$ then can be used to define $f$.

Pick an encoding of finite sequences of non-negative integers, preferably a primitive recursive encoding.  As usual, we set $$\langle a_1, \ldots, a_m \rangle $$ as the code, or sequence number of the sequence $a_1, \ldots, a_m$, and if $x= \langle a_1, \ldots, a_m \rangle$, we set $(x)_i:=a_i$ if $i\le m$, and $0$ otherwise.

For this entry, we choose the multiplicative encoding of G\"{o}del (see \PMlinkname{this entry}{ExamplesOfPrimitiveRecursiveEncoding} for more detail) because of its convenience.  Then $$\langle a_1, \ldots, a_m \rangle := p_1^{s(a_1)} \cdots p_m^{s(a_m)},$$ where $p_i$ is the $i$-th prime number, and $\langle \rangle:=1$.

\textbf{Definition}.  For any function $f:\mathbb{N}^{k+1}\to \mathbb{N}$, define the \emph{course-of-values function} $\overline{f}$ of $f$ as follows: $\overline{f}$ has the same arity as $f$, and for any $\boldsymbol{x}\in \mathbb{N}^k$,
\begin{enumerate}
\item $\overline{f}(\boldsymbol{x},0):=\langle \rangle $, and
\item $\overline{f}(\boldsymbol{x},y+1):= \langle f(\boldsymbol{x},0), \ldots, f(\boldsymbol{x},y)\rangle$.
\end{enumerate}

For example, if $F(n)$ is the $n$-th Fibonacci number, then $\overline{F}(0)=1$, $\overline{F}(1)=\langle 1 \rangle = 4$, and $\overline{F}(2)=\langle 4,1 \rangle = 2^5\cdot 3^2 = 288$, etc...  It is evident that $\overline{F}$ grows very rapidly.

\textbf{Definition}.  A function $f:\mathbb{N}^{k+1}\to \mathbb{N}$ is said to be defined by \emph{course-of-values recursion} via function $h:\mathbb{N}^{k+2} \to \mathbb{N}$ if, for any $\boldsymbol{x} \in \mathbb{N}^k$:
$$f(\boldsymbol{x},y) = h(\boldsymbol{x},y, \overline{f}(\boldsymbol{x},y) ).$$

The two examples above are defined by course-of-values recursion:
\begin{itemize}
\item $F(n)=h(n,\overline{F}(n))$, where $h(x,y): = (y)_x + (y)_{x \dot{-} 1}$.
\item $f(n)=h(n,\overline{f}(n))$, where $ h(x,y)= \sum_{i=0}^{x \dot{-}1} (y)_i$.
\end{itemize}

The second question posed at the beginning can now be answered:

\begin{prop} $f$ is primitive recursive iff $\overline{f}$ is.  \end{prop} 
\begin{proof}  By definition, and using multiplicative encoding, we have $\overline{f}(\boldsymbol{x},0):=\langle \rangle=1 $, and $$\overline{f}(\boldsymbol{x},y+1):= p_1^{f(\boldsymbol{x},0)} \cdots p_{y+1}^{f(\boldsymbol{x},y)} = \overline{f}(\boldsymbol{x},y) p_{y+1}^{f(\boldsymbol{x},y)} $$
Thus, if $f$ is primitive recursive, so is $\overline{f}$ by primitive recursion.  Conversely, if $\overline{f}$ is primitive recursive, so is $f(\boldsymbol{x},y)=(\overline{f}(\boldsymbol{x},y+1))_{y+1}$.
\end{proof}

\begin{prop} If $h: \mathbb{N}^{k+1} \to \mathbb{N}$ is primitive recursive, so is $f$ defined by course-of-values recursion via $h$. \end{prop}
\begin{proof} From the proof of the proposition above, we see that $$\overline{f}(\boldsymbol{x},y+1)= \overline{f}(\boldsymbol{x},y) p_{y+1}^{f(\boldsymbol{x},y)} = \overline{f}(\boldsymbol{x},y) p_{y+1}^{h(\boldsymbol{x},y, \overline{f}(\boldsymbol{x},y))}.$$  Thus, as $h$ is primitive recursive, so is $\overline{f}$ by primitive recursion, and hence $f$ is primitive recursive by the previous proposition.
\end{proof}

\textbf{Remark}.  If the value of the next argument of $f$ depends on values of a fixed set of prior arguments, then the primitive recursiveness of $f$ can be proved via mutual recursion.  For example, suppose $f(k+1)=h(f(k-1),f(k-3),f(k-4))$.  By setting $f_i(n):=f(n+i)$ for $i=0,1,2,3$, we see that $$f_i(n+1)=f(n+i+1)=f_{i+1}(n) \quad \mbox{for }i=0,1,2.$$  Furthermore, 
\begin{eqnarray*}
f_3(n+1) &=&  f_0(n+4)=f(n+4)=h(f(n+2),f(n+1),f(n)) \\ &=& h(f_0(n+2),f_0(n+1),f_0(n))=h(f_2(n),f_1(n),f_0(n)).
\end{eqnarray*}
So the $f_i$ are defined by mutual recursion, and if $h$ is primitive recursive, so is each $f_i$.
%%%%%
%%%%%
\end{document}
