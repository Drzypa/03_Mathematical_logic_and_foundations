\documentclass[12pt]{article}
\usepackage{pmmeta}
\pmcanonicalname{SumAndProductAndQuotientOfFunctions}
\pmcreated{2013-03-22 17:44:24}
\pmmodified{2013-03-22 17:44:24}
\pmowner{pahio}{2872}
\pmmodifier{pahio}{2872}
\pmtitle{sum and product and quotient of functions}
\pmrecord{12}{40190}
\pmprivacy{1}
\pmauthor{pahio}{2872}
\pmtype{Definition}
\pmcomment{trigger rebuild}
\pmclassification{msc}{03E20}
\pmrelated{DirectSumOfEvenoddFunctionsExample}
\pmrelated{LimitRulesOfFunctions}
\pmrelated{PolynomialFunction}
\pmrelated{ProofOfLimitRuleOfProduct}
\pmrelated{ContinuousDerivativeImpliesBoundedVariation}
\pmrelated{PropertiesOfRiemannStieltjesIntegral}
\pmrelated{InfimumAndSupremumOfSumAndProduct}
\pmrelated{PropertiesOfVectorValuedFunctio}
\pmdefines{sum of functions}
\pmdefines{product of functions}
\pmdefines{quotient of functions}
\pmdefines{scalar-multiplied function}

% this is the default PlanetMath preamble.  as your knowledge
% of TeX increases, you will probably want to edit this, but
% it should be fine as is for beginners.

% almost certainly you want these
\usepackage{amssymb}
\usepackage{amsmath}
\usepackage{amsfonts}

% used for TeXing text within eps files
%\usepackage{psfrag}
% need this for including graphics (\includegraphics)
%\usepackage{graphicx}
% for neatly defining theorems and propositions
 \usepackage{amsthm}
% making logically defined graphics
%%%\usepackage{xypic}

% there are many more packages, add them here as you need them

% define commands here

\theoremstyle{definition}
\newtheorem*{thmplain}{Theorem}

\begin{document}
Let $A$ be a set and $M$ a left $R$-module.\, If\, $f\!: A \to M$\, and\, $g\!: A \to M$,\, then one may define the {\em sum of functions} $f$ and $g$ as the following function \;$f\!+\!g\!: A \to M$:
$$(f\!+\!g)(x) \;:=\; f(x)\!+\!g(x) \quad \forall x \in A$$
If $r$ is any element of the ring $R$, then the {\em scalar-multiplied function}\; $rf\!: A \to M$\, is defined as
$$(rf)(x) \;:=\; r\!\cdot\!f(x) \quad \forall x \in A.$$


Let $A$ again be a set and $K$ a field or a skew field.\, If\, $f\!: A \to K$\, and\, $g\!: A \to K$,\, then one can define the {\em product of functions} $f$ and $g$ as the function\; $fg\!: A \to K$ as follows:
$$(fg)(x) \;:=\; f(x)\!\cdot\!g(x) \quad \forall x \in A$$
The {\em quotient of functions} $f$ and $g$ is the function\; 
$\displaystyle\frac{f}{g}\!: \{a\in A\,\vdots\;\; g(a) \neq 0\} \to K$\; defined as
$$\frac{f}{g}(x) \;:=\; \frac{f(x)}{g(x)} \quad \forall x \in A\!\smallsetminus\!\{a\in A\,\vdots\;\; g(a) = 0\}.$$

In particular, the incremental quotient of functions $\frac{f(y)-f(x)}{y-x}$, as $y$ tends to $x$, gave rise to the important concept of derivative. As another example, we can with a \PMlinkescapetext{clear} conscience say that the \PMlinkname{tangent}{TrigonometricFunction} function is the quotient of the \PMlinkname{sine}{TrigonometricFunction} and the cosine functions.
%%%%%
%%%%%
\end{document}
