\documentclass[12pt]{article}
\usepackage{pmmeta}
\pmcanonicalname{Forcing}
\pmcreated{2013-03-22 12:44:17}
\pmmodified{2013-03-22 12:44:17}
\pmowner{ratboy}{4018}
\pmmodifier{ratboy}{4018}
\pmtitle{forcing}
\pmrecord{12}{33039}
\pmprivacy{1}
\pmauthor{ratboy}{4018}
\pmtype{Definition}
\pmcomment{trigger rebuild}
\pmclassification{msc}{03E50}
\pmclassification{msc}{03E35}
\pmclassification{msc}{03E40}
\pmrelated{ForcingRelation}
\pmrelated{CompositionOfForcingNotions}
\pmrelated{EquivalenceOfForcingNotions}
\pmrelated{FieldAdjunction}
\pmdefines{forcing}

\endmetadata

% this is the default PlanetMath preamble.  as your knowledge
% of TeX increases, you will probably want to edit this, but
% it should be fine as is for beginners.

% almost certainly you want these
\usepackage{amssymb}
\usepackage{amsmath}
\usepackage{amsfonts}

% used for TeXing text within eps files
%\usepackage{psfrag}
% need this for including graphics (\includegraphics)
%\usepackage{graphicx}
% for neatly defining theorems and propositions
%\usepackage{amsthm}
% making logically defined graphics
\usepackage[arrow,curve,poly,arc,2cell,frame,web]{xypic}

% there are many more packages, add them here as you need them

% define commands here
\newcommand{\br}{[\![}
\newcommand{\rb}{]\!]}
\newcommand{\oq}{\text{``}}
\newcommand{\cq}{\text{''}}


\newcommand{\im}{\mathbf{Im}}
\newcommand{\dom}{\mathbf{Dom}}


\newcommand{\Or}{\vee}
\newcommand{\Implies}{\Rightarrow}
\newcommand{\Iff}{\Leftrightarrow}
\newcommand{\proves}{\vdash}
\renewcommand{\And}{\wedge}
\newcommand{\Sup}{\bigwedge}
\newcommand{\Inf}{\bigvee}
\newcommand{\Z}{\mathbb{Z}}
\newcommand{\F}{\mathbb{F}}
\newcommand{\Q}{\mathbb{Q}}
\newcommand{\R}{\mathbb{R}}
\newcommand{\C}{\mathbb{C}}
\newcommand{\Nat}{\mathbb{N}}
\newcommand{\M}{\mathfrak{M}}
\newcommand{\N}{\mathfrak{N}}
\newcommand{\A}{\mathfrak{A}}
\newcommand{\B}{\mathfrak{B}}
\newcommand{\K}{\mathfrak{K}}
\newcommand{\G}{\mathbb{G}}
\newcommand{\Def}{\overset{\operatorname{def}}{:=}}



\newcommand{\spec}{\text{{\bf Spec}}}
\newcommand{\stab}{\text{{\bf Stab}}}
\newcommand{\ann}{\text{{\bf Ann}}}
\newcommand{\irr}{\text{{\bf Irr}}}
\newcommand{\qt}{\text{{\bf Qt}}}
\newcommand{\st}{\mathcal{Qt}}
\newcommand{\ro}{\mathbf{r.o.}}


\newcommand{\Endo}{\text{{\bf End}}}
\newcommand{\mat}{\text{{\bf Mat}}}
\newcommand{\der}{\text{{\bf Der}}}
\newcommand{\rad}{\text{{\bf Rad}}}
\newcommand{\trd}{\text{{\bf tr.d.}}}
\newcommand{\cl}{\text{{\bf acl}}}
\newcommand{\Int}{\text{{\bf int}}}
\newcommand{\V}{\mathbb{V}}
\newcommand{\D}{\mathbf{D}}

\newcommand{\del}{\partial}
\renewcommand{\O}{\mathcal{O}}
\newcommand{\aut}{\mathbf{Aut}}
\newcommand{\height}{\text{\bf Height}}
\newcommand{\coheight}{\text{\bf Co-height}}

\newcommand{\lcm}{\operatorname{lcm}}

\newcommand{\Gal}{\operatorname{Gal}}
\newcommand{\x}{\mathbf{x}}
\newcommand{\y}{\mathbf{y}}
\newcommand{\inner}[2]{\langle #1|#2\rangle}
\renewcommand{\r}{{r}}
\renewcommand{\t}{{t}}

\newcommand{\restr}{\upharpoonright}
\newcommand{\Matrix}[4]{\left(\begin{array}{cc} #1 & #2 \\ #3 & #4 
\end{array}\right)}

\newenvironment{definition}{{\bf Definition.}}{}
\newenvironment{theorem}{{\bf Theorem.}}{}
\newenvironment{proof}{{\bf Proof:}}{\hfill$\diamondsuit$}
\newenvironment{corollary}{{\bf Corollary}}{}
\newenvironment{proposition}{{\bf Proposition.}}{}
\newenvironment{example}{{\bf Example.}}{}
\begin{document}
Forcing is the method used by Paul Cohen to prove the independence of
the continuum hypothesis (CH).  In fact, the method was used by Cohen to
prove that CH could be violated.  

Adding a set to a model of set theory via forcing is similar to adjoining a new element to a field. Suppose we have a field $k$, and we want to add to this field an
element $\alpha$ such $\alpha^2=-1$.  We see
that we cannot simply drop a new $\alpha$ in $k$, since then we are
not guaranteed that we still have a field.  Neither can we simply
assume that $k$ already has such an element.  The standard way of
doing this is to start by adjoining a generic indeterminate $X$, and
impose a constraint on $X$, saying that $X^2+1=0$.  What we do is take
the quotient $k[X]/(X^2+1)$, and make a field out of it by taking the
quotient field.  We then obtain $k(\alpha)$, where $\alpha$ is the
equivalence class of $X$ in the quotient.
The general case of this is the theorem of algebra saying that every
polynomial $p$ over a field $k$ has a root in some extension field.

We can rephrase this and say that ``it is consistent with standard
field theory that $-1$ have a square root''.

When the theory we consider is ZFC, we run in exactly the same
problem : we can't just add a ``new'' set and pretend it has the
required properties, because then we may violate something else, like
foundation.  Let $\M$ be a transitive model of set theory, which we
call the {\bf ground model}.  We want to ``add a new set'' $S$ to $\M$ in
such a way that the extension $\M'$ has $\M$ as a subclass, and the
properties of $\M$ are preserved, and $S\in\M'$.

The first step is to ``approximate'' the new set using elements of
$\M$.  This is the analogue of finding the irreducible polynomial in
the algebraic example.  The set $P$ of such ``approximations'' can be
ordered by how much information the approximations give : let $p,q\in
P$, then $p\leq q$ if and only if $p$ ``is stronger than'' $q$.  We
call this set a set of {\bf forcing conditions}.  Furthermore, it is required that the set $P$ itself and the order relation be elements of $\M$.

Since $P$ is a partial order, some of its subsets have interesting
properties.  Consider $P$ as a topological space with the order
topology.  A subset $D\subseteq P$ is {\bf dense} in $P$ if and only
if for every $p\in P$, there is $d\in D$ such that $d\leq p$.  A
filter in $P$ is said to be {\bf $\M$-generic} if and only if it intersects
every one of the dense subsets of $P$ which are in $\M$.  An $\M$-generic
filter in $P$ is also referred to as a {\bf generic set of conditions}
in the literature.  In general, even though $P$ is a set in $\M$, generic filters are not elements of $\M$.

If $P$ is a set of forcing conditions, and $G$ is a generic set of
conditions in $P$, all in the ground model $\M$, then we define
$\M[G]$ to be the least model of ZFC that contains $G$. The
big theorem is this :

\begin{theorem}
$\M[G]$ is a model of ZFC, and has the same ordinals as $\M$, and
$\M\subseteq \M[G]$.
\end{theorem}

The way to prove that we can violate CH using a generic extension is
to add many new ``subsets of $\omega$'' in the following way : let
$\M$ be a transitive model of ZFC, and let $(P,\leq)$ be the set (in
$\M$) of all functions $f$ whose domain is a finite subset of
$\aleph_2\times\aleph_0$, and whose range is the set $\{0,1\}$.  The
ordering here is $p\leq q$ if and only if $p\supset q$.  Let
$G$ be a generic set of conditions in $P$.  Then $\bigcup G$ is a
total function whose domain is $\aleph_2\times\aleph_0$, and range is
$\{0,1\}$.  We can see this $f$ as coding $\aleph_2$ new functions
$f_\alpha:\aleph_0\to\{0,1\}$, $\alpha<\aleph_2$, 
 which are subsets of omega.  These
functions are all distinct. $(P,\leq)$ \PMlinkid{doesn't collapse cardinals}{3242} since it satisfies the countable chain condition. Thus $\aleph_2^{\M[G]} = \aleph_2^{\M}$ and CH is false in $\M[G]$.  

All this relies on a proper definition of the satisfaction relation in
$\M[G]$, and the forcing relation.  Details can be found in Thomas Jech's book {\em Set Theory}.
%%%%%
%%%%%
\end{document}
