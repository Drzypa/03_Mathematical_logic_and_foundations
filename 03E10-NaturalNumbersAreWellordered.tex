\documentclass[12pt]{article}
\usepackage{pmmeta}
\pmcanonicalname{NaturalNumbersAreWellordered}
\pmcreated{2013-03-22 19:02:36}
\pmmodified{2013-03-22 19:02:36}
\pmowner{pahio}{2872}
\pmmodifier{pahio}{2872}
\pmtitle{natural numbers are well-ordered}
\pmrecord{6}{41921}
\pmprivacy{1}
\pmauthor{pahio}{2872}
\pmtype{Theorem}
\pmcomment{trigger rebuild}
\pmclassification{msc}{03E10}
\pmrelated{AVariantDerivationOfWellOrderedSet}
\pmrelated{WellOrderedSet}
\pmrelated{WellOrderingPrincipleForNaturalNumbersProvenFromThePrincipleOfFiniteInduction}

% this is the default PlanetMath preamble.  as your knowledge
% of TeX increases, you will probably want to edit this, but
% it should be fine as is for beginners.

% almost certainly you want these
\usepackage{amssymb}
\usepackage{amsmath}
\usepackage{amsfonts}

% used for TeXing text within eps files
%\usepackage{psfrag}
% need this for including graphics (\includegraphics)
%\usepackage{graphicx}
% for neatly defining theorems and propositions
 \usepackage{amsthm}
% making logically defined graphics
%%%\usepackage{xypic}

% there are many more packages, add them here as you need them

% define commands here

\theoremstyle{definition}
\newtheorem*{thmplain}{Theorem}

\begin{document}
In many proofs, one needs the following property of positive and nonnegative integers:\\

\textbf{Theorem.}\, Any non-empty set of natural numbers contains a least number.

\emph{Proof.}\, Let $A$ be an arbitrary non-empty subset of $\mathbb{N}$.\, Denote
$$C \;=\; \{x \in \mathbb{N}\,\vdots\;\; x \leq a\; \forall a \in A\}.$$
Then of course,\, $0 \in C$.\, There exists surely an element $c$ of $C$ such that\, $c\!+\!1 \notin C$,\, since otherwise the induction property would imply that\, $C = \mathbb{N}$.\, Because\, $c\!+\!1 \notin C$,\, there is a number $a_0$ of the set $A$ such that\, $a_0 < c\!+\!1$.\, On the other \PMlinkescapetext{side}, we must have\, $c \leq a_0$.\, Consequently,\, $c = a_0$\, and therefore
$$a_0 \;=\; c \;\leq\; a\;\; \forall a \in A.$$
Hence, $A$ has the least number $a_0$.\, Q.E.D.

%%%%%
%%%%%
\end{document}
