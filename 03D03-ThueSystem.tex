\documentclass[12pt]{article}
\usepackage{pmmeta}
\pmcanonicalname{ThueSystem}
\pmcreated{2013-03-22 17:33:19}
\pmmodified{2013-03-22 17:33:19}
\pmowner{CWoo}{3771}
\pmmodifier{CWoo}{3771}
\pmtitle{Thue system}
\pmrecord{12}{39961}
\pmprivacy{1}
\pmauthor{CWoo}{3771}
\pmtype{Definition}
\pmcomment{trigger rebuild}
\pmclassification{msc}{03D03}
\pmclassification{msc}{03D40}
\pmclassification{msc}{68Q42}
\pmclassification{msc}{20M35}
\pmrelated{SemigroupWithInvolution}
\pmdefines{group system}

\endmetadata

\usepackage{amssymb,amscd}
\usepackage{amsmath}
\usepackage{amsfonts}
\usepackage{mathrsfs}

% used for TeXing text within eps files
%\usepackage{psfrag}
% need this for including graphics (\includegraphics)
%\usepackage{graphicx}
% for neatly defining theorems and propositions
\usepackage{amsthm}
% making logically defined graphics
%%\usepackage{xypic}
\usepackage{pst-plot}
\usepackage{psfrag}

% define commands here
\newtheorem{prop}{Proposition}
\newtheorem{thm}{Theorem}
\newtheorem{ex}{Example}
\newcommand{\real}{\mathbb{R}}
\newcommand{\pdiff}[2]{\frac{\partial #1}{\partial #2}}
\newcommand{\mpdiff}[3]{\frac{\partial^#1 #2}{\partial #3^#1}}
\newcommand{\derive}{\stackrel{*}{\Rightarrow}}
\begin{document}
A semi-Thue system $\mathfrak{S}=(\Sigma,R)$ is said to be a \emph{Thue system} if $R$ is a symmetric relation on $\Sigma^*$.  In other words, if $x\to y$ is a defining relation in $R$, then so is $y\to x$.

Like a semi-Thue system, we can define the concepts of immediately derivable and derivable pairs.  Let $R'$ and $R''$ be the respective collections of these pairs.  Since $R$ is symmetric, so is $R'$ and consequently $R''$.  Similarly, the notations are: for elements $(a,b)\in R'$, we write $a\Rightarrow b$, and for elements $(c,d)\in R''$, we write $a\derive b$.

If we regard $\Sigma^*$ as a free monoid with concatenation $\cdot$ as multiplication and the empty word $\lambda$ as the multiplicative identity, then $\derive$ is a congruence relation on $\Sigma^*$: it is an equivalence relation and respects concatenation, meaning that if $a\derive b$ and $c\derive d$, then $ac\derive bd$.  Therefore, we can take the quotent $\Sigma^*/\derive$ and the resulting set of equivalence classes is again a monoid with $[\lambda]$ as the multiplicative identity.  It is a monoid generated by $[a]$ whenever $a\in \Sigma$ with relations $[u]=[v]$ whenever $u\to v$ is a defining relation in $R$.  Thus, two elements are in the same equivalence class if one is derivable from another.  Let us denote this monoid by $[\Sigma]_{\mathfrak{S}}$.

Now let $\mathfrak{S}=(\Sigma,R)$ be a Thue system.  Then $\mathfrak{S}$ is called a \emph{group system} if there exists an involution $^{-1}$ on $\Sigma$ given by $a\mapsto a^{-1}$, and that for every $a\in \Sigma$, $aa^{-1}\to \lambda$ is a defining relation in $R$.  Since $^{-1}$ is an involution, if $b$ is the symbol in $\Sigma$ such that $b=a^{-1}$, then $b^{-1}=a$.  So $a^{-1}a = ba= bb^{-1} \to \lambda$ also.  In fact, it is not hard to see that for a group system $\mathfrak{S}$, $[\Sigma]_{\mathfrak{S}}$ is the group with generators $[a]$ whenever $a\in \Sigma$ and with relators $[u][v]^{-1}$ whenever $u\to v$ is a defining relation in $R$.  Every non-trivial element in $[\Sigma]_{\mathfrak{S}}$ has an expression $[a_1]^{p_1}\cdots [a_n]^{p_n}$, where each $a_i$ is a letter in $\Sigma$ such that it is distinct from its neighbors ($a_i\ne a_{i+1}$), and $p_i$ are non-zero integers.  This expression is unique in the sense that it is ``reduced''.  See reduced words for more detail.

\textbf{Remark}.  Like the word problem for semi-Thue systems, the word problem for Thue systems and group systems can be similarly posed.  It can be shown that the word problem for Thue systems and group systems are both unsolvable.  As a result, the corresponding word problems for semigroups and for groups are also unsolvable.

\begin{thebibliography}{9}
\bibitem{md} M. Davis, {\em Computability and Unsolvability}. Dover Publications, New York (1982).
\bibitem{hh} H. Hermes, {\em Enumerability, Decidability, Computability: An Introduction to the Theory of Recursive Functions}. Springer, New York, (1969).
\bibitem{hlcp} H.R. Lewis, C.H. Papadimitriou {\em Elements of the Theory of Computation}. Prentice-Hall, Englewood Cliffs, New Jersey (1981).
\bibitem{psn} P.S. Novikov, {\em On the algorithmic unsolvability of the word problem in group theory},
Trudy Mat. Inst. Steklov 44, 1-143 (1955).
\end{thebibliography}
%%%%%
%%%%%
\end{document}
