\documentclass[12pt]{article}
\usepackage{pmmeta}
\pmcanonicalname{85TheHopfFibration}
\pmcreated{2013-11-06 14:19:59}
\pmmodified{2013-11-06 14:19:59}
\pmowner{PMBookProject}{1000683}
\pmmodifier{rspuzio}{6075}
\pmtitle{8.5 The Hopf fibration}
\pmrecord{1}{}
\pmprivacy{1}
\pmauthor{PMBookProject}{6075}
\pmtype{Feature}
\pmclassification{msc}{03B15}

\usepackage{xspace}
\usepackage{amssyb}
\usepackage{amsmath}
\usepackage{amsfonts}
\usepackage{amsthm}
\newcommand{\define}[1]{\textbf{#1}}
\newcommand{\eqv}[2]{\ensuremath{#1 \simeq #2}\xspace}
\newcommand{\indexdef}[1]{\index{#1|defstyle}}   
\newcommand{\Sn}{\mathbb{S}}
\newcommand{\Z}{\ensuremath{\mathbb{Z}}\xspace}
\newcounter{mathcount}
\setcounter{mathcount}{1}
\newtheorem{precor}{Corollary}
\newenvironment{cor}{\begin{precor}}{\end{precor}\addtocounter{mathcount}{1}}
\renewcommand{\theprecor}{8.5.\arabic{mathcount}}
\newtheorem{prethm}{Theorem}
\newenvironment{thm}{\begin{prethm}}{\end{prethm}\addtocounter{mathcount}{1}}
\renewcommand{\theprethm}{8.5.\arabic{mathcount}}
\let\autoref\cref

\begin{document}

In this section we will define the \define{Hopf fibration}.
\indexdef{Hopf!fibration}%

\begin{thm}[Hopf Fibration]\label{thm:hopf-fibration}
There is a fibration $H$ over $\Sn ^2$ whose fiber over the basepoint is $\Sn ^1$ and
whose total space is $\Sn ^3$.
\end{thm}

The Hopf fibration will allow us to compute several homotopy groups of
spheres.
Indeed, it yields the following long exact sequence
\index{homotopy!group!of sphere}
\index{sequence!exact}
of homotopy groups
(see
\autoref{sec:long-exact-sequence-homotopy-groups}):
%
\[
\xymatrix@R=1.2pc{
  \pi_k(\Sn^1) \ar[r] & \pi_k(\Sn^3) \ar[r] & \pi_k(\Sn^2) \ar[lld] \\
  \vdots & \vdots & \vdots \ar[lld] \\
  \pi_2(\Sn^1) \ar[r] & \pi_2(\Sn^3) \ar[r] & \pi_2(\Sn^2) \ar[lld] \\
  \pi_1(\Sn^1) \ar[r] & \pi_1(\Sn^3) \ar[r] & \pi_1(\Sn^2)}
\]
%
We've already computed all $\pi_n(\Sn^1)$, and $\pi_k(\Sn^n)$ for $k<n$, so this
becomes the following:
%
\[
\xymatrix@R=1.2pc{
  0 \ar[r] & \pi_k(\Sn^3) \ar[r] & \pi_k(\Sn^2) \ar[lld] \\
  \vdots & \vdots & \vdots \ar[lld] \\
  0 \ar[r] & \pi_3(\Sn^3) \ar[r] & \pi_3(\Sn^2) \ar[lld] \\
  0 \ar[r] & 0 \ar[r] & \pi_2(\Sn^2) \ar[lld] \\
  \Z \ar[r] & 0 \ar[r] & 0}
\]
%
In particular we get the following result:

\begin{cor} \label{cor:pis2-hopf}
  We have $\eqv{\pi_2(\Sn^2)}{\Z}$ and $\eqv{\pi_k(\Sn^3)}{\pi_k(\Sn^2)}$ for
  every $k\ge3$ (where the map is induced by the Hopf fibration, seen as a map
  from the total space $\Sn^3$ to the base space $\Sn^2$).
\end{cor}

In fact, we can say more: the fiber sequence of the Hopf fibration will show that $\Omega^3(\Sn^3)$ is the fiber of a map from $\Omega^3(\Sn^2)$ to $\Omega^2(\Sn^1)$.
Since $\Omega^2(\Sn^1)$ is contractible, we have $\eqv{\Omega^3(\Sn^3)}{\Omega^3(\Sn^2)}$.
In classical homotopy theory, this fact would be a consequence of \autoref{cor:pis2-hopf} and Whitehead's theorem, but Whitehead's theorem is not necessarily valid in homotopy type theory (see \autoref{sec:whitehead}).
We will not use the more precise version here though.


\end{document}
