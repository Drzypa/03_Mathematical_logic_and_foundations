\documentclass[12pt]{article}
\usepackage{pmmeta}
\pmcanonicalname{CardinalExponentiationUnderGCH}
\pmcreated{2013-03-22 14:54:04}
\pmmodified{2013-03-22 14:54:04}
\pmowner{yark}{2760}
\pmmodifier{yark}{2760}
\pmtitle{cardinal exponentiation under GCH}
\pmrecord{9}{36584}
\pmprivacy{1}
\pmauthor{yark}{2760}
\pmtype{Theorem}
\pmcomment{trigger rebuild}
\pmclassification{msc}{03E10}
%\pmkeywords{GCH}
%\pmkeywords{exponentiation}
\pmrelated{CardinalArithmetic}
\pmrelated{GeneralizedContinuumHypothesis}

\usepackage{amssymb}
\usepackage{amsmath}
\usepackage{amsthm}

\DeclareMathOperator{\cf}{cf}
\newtheorem*{thm*}{Theorem}
\begin{document}
\PMlinkescapeword{addition}
\PMlinkescapeword{complete}
\PMlinkescapeword{nor}
\PMlinkescapeword{theorem}

Many results about cardinal exponentiation can neither be proved nor disproved in ZFC. If, however, we allow ourselves to use GCH in addition to ZFC, then we have the following theorem, which gives an essentially complete description of the way cardinal exponentiation involving infinite cardinals works.

\begin{thm*}
Assume the Generalized Continuum Hypothesis holds.
Let $\kappa$ and $\lambda$ be cardinals,
at least one of which is infinite,
and such that $\kappa>0$ and $\lambda>1$.
Then 
$$\lambda^\kappa=
\begin{cases}
\kappa^+,&\text{if}\quad\lambda\le\kappa^+;\\
\lambda^+,&\text{if}\quad\cf(\lambda)\le\kappa\le\lambda;\\
\lambda,&\text{if}\quad\kappa<\cf(\lambda).
\end{cases}$$
\end{thm*}

Here, $\cf(\lambda)$ is the cofinality of $\lambda$, and $\lambda^+$ is the cardinal successor of $\lambda$.
%%%%%
%%%%%
\end{document}
