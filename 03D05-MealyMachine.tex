\documentclass[12pt]{article}
\usepackage{pmmeta}
\pmcanonicalname{MealyMachine}
\pmcreated{2013-03-22 18:59:54}
\pmmodified{2013-03-22 18:59:54}
\pmowner{CWoo}{3771}
\pmmodifier{CWoo}{3771}
\pmtitle{Mealy machine}
\pmrecord{10}{41866}
\pmprivacy{1}
\pmauthor{CWoo}{3771}
\pmtype{Definition}
\pmcomment{trigger rebuild}
\pmclassification{msc}{03D05}
\pmclassification{msc}{68Q45}
\pmsynonym{complete sequential machine}{MealyMachine}
\pmrelated{MooreMachine}
\pmrelated{EquivalentMachines}
\pmdefines{flow table}
\pmdefines{combinational circuit}

\usepackage{amssymb,amscd}
\usepackage{amsmath}
\usepackage{amsfonts}
\usepackage{mathrsfs}
\usepackage{tabls}

% used for TeXing text within eps files
%\usepackage{psfrag}
% need this for including graphics (\includegraphics)
\usepackage{graphicx}
% for neatly defining theorems and propositions
\usepackage{amsthm}
% making logically defined graphics
%%\usepackage{xypic}
\usepackage{pst-plot}

% define commands here
\newcommand*{\abs}[1]{\left\lvert #1\right\rvert}
\newtheorem{prop}{Proposition}
\newtheorem{thm}{Theorem}
\newtheorem{ex}{Example}
\newcommand{\real}{\mathbb{R}}
\newcommand{\pdiff}[2]{\frac{\partial #1}{\partial #2}}
\newcommand{\mpdiff}[3]{\frac{\partial^#1 #2}{\partial #3^#1}}
\begin{document}
A \emph{Mealy machine} $M$ is a five-tuple $(S,\Sigma,\Delta,\sigma,\lambda)$, where
\begin{enumerate}
\item $S$ is an alphabet whose elements are called \emph{states},
\item $\Sigma$ is an alphabet whose elements are called \emph{input symbols},
\item $\Delta$ is an alphabet whose elements are called \emph{output symbols},
\item $\sigma: S\times \Sigma \to S$ is a function called the \emph{state transition function},
\item $\lambda: S\times \Sigma \to \Delta$ is a function called the \emph{output function}.
\end{enumerate}
Given an a pair $(s,a)$ of state $s$ and input symbol $a$, the value $\sigma(s,a)$ is the \emph{next state} of $(s,a)$ and $\lambda(s,a)$ is the \emph{output} of $(s,a)$.

Since all of the sets involved are finite, there are two ways to visualize a Mealy machine: via a table called the \emph{flow table} describing the transition and the output functions, or via a directed graph called the \emph{state diagram} (see details below).

For example, consider a Mealy machine $M$ where $S=\lbrace s,t\rbrace$, $\Sigma = \lbrace a,b\rbrace$, and $\Delta = \lbrace x,y\rbrace$.  The state transition function $\sigma$ and the output function $\lambda$ are defined by the following table:

\begin{center}
\begin{tabular}{|c||c|c||c|c|}
\hline
& $a$ & $b$ & $a$ & $b$ \\
\hline\hline
$s$ & $t$ & $s$ & $y$ & $y$ \\
\hline
$t$ & $s$ & $t$ & $x$ & $y$ \\
\hline
\end{tabular}
\end{center}
The first column represents all the states of $M$; the second and third columns are the next states corresponding to the input symbols specified on the top row; and the last two columns show the corresponding output symbols.

In addition, one can visualize a Mealy machine $M$ by way its state diagram $G_M$, which is a (labeled) directed graph.  $G_M$ is constructed as follows: the vertices of $G_M$ are states of $M$.  An edge from vertices $s$ to $t$ is constructed iff there is an input $a$ such that $\sigma(s,a)=t$.  Let $x=\lambda(s,a)$ be the corresponding output.  Then the edge from $s$ to $t$ is given a label $a|x$.  In the example above, the state diagram $G_M$ of $M$ is the following

\begin{figure}[htp]
\centering
\includegraphics[scale=0.80]{mealy.eps}
\end{figure}

\textbf{Remark}.  
\begin{itemize}
\item
A Mealy machine with only one state is called a \emph{combinational circuit}.  It is nothing more than a finite function $\lambda$ from the set of inputs $I$ to the set of outputs $O$.
\item
A Mealy machine can be thought of as a state-output machine, if one treats the values of $\sigma$ and $\lambda$ as singleton subsets of $S$ and $\Delta$ respectively.  In this regard, a Mealy machine is complete because both $\sigma(s,a)$ and $\lambda(s,a)$ contain at least one element each, for all pairs $(s,a)\in S\times \Sigma$, and sequential because both $\sigma(s,a)$ and $\lambda(s,a)$ contain at most one element each.  As such, a Mealy machine is also known as a \emph{complete sequential machine}.
\end{itemize}

Viewed as a state-output machine, the transition and output functions of a Mealy machine can be extended so input words may be applied to the machine.  For the transition function $\sigma$, the extension is defined inductively on the length of input words: 
$$\sigma(s,\epsilon):=s, \mbox{ and } \sigma(s,ua):=\sigma(\sigma(s,u),a), \mbox{ where }s\in S, u\in \Sigma^*, a\in \Sigma.$$  Like the transition function $\sigma$, the output function $\lambda$ is also extended inductively on the length of input words: $$\lambda(s,ua):=\lambda(\sigma(s,u),a), \mbox{ where }s\in S, u\in \Sigma^*, a\in \Sigma.$$  However, unlike $\sigma$, $\lambda$ is not defined for the empty input word $\epsilon$.

\begin{thebibliography}{8}
\bibitem{sg} S. Ginsburg, {\em An Introduction to Mathematical Machine Theory}, Addision-Wesley, (1962).
\bibitem{ma} M. Arbib, \emph{Algebraic Theory of Machines, Languages, and Semigroups}, Academic Press, (1968).
\bibitem{hs} J. Hartmanis, R.E. Stearns, \emph{Algebraic Structure Theory of Sequential Machines}, Prentice-Hall, (1966).
\end{thebibliography}
%%%%%
%%%%%
\end{document}
