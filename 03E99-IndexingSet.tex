\documentclass[12pt]{article}
\usepackage{pmmeta}
\pmcanonicalname{IndexingSet}
\pmcreated{2013-03-22 16:07:51}
\pmmodified{2013-03-22 16:07:51}
\pmowner{Wkbj79}{1863}
\pmmodifier{Wkbj79}{1863}
\pmtitle{indexing set}
\pmrecord{9}{38202}
\pmprivacy{1}
\pmauthor{Wkbj79}{1863}
\pmtype{Definition}
\pmcomment{trigger rebuild}
\pmclassification{msc}{03E99}
\pmsynonym{index set}{IndexingSet}
\pmdefines{subscript}
\pmdefines{index}
\pmdefines{indices}
\pmdefines{indexed by}
\pmdefines{double indices}
\pmdefines{multiple indices}

\endmetadata

\usepackage{amssymb}
\usepackage{amsmath}
\usepackage{amsfonts}

\usepackage{psfrag}
\usepackage{graphicx}
\usepackage{amsthm}
%%\usepackage{xypic}

\begin{document}
Let $\Lambda$ and $S$ be sets such that there exists a surjection $f \colon \Lambda \to S$.  Then $\Lambda$ is an \emph{indexing set} for $S$.  Also, $S$ is \emph{indexed by} $\Lambda$.

In such situations, the elements of $S$ could be referenced by using the indexing set $\Lambda$, such as $f(\lambda)$ for some $\lambda \in \Lambda$.  On the other hand, quite often, indexing sets are used without explicitly defining a surjective function.  When this occurs, the elements of $S$ are referenced by using \emph{subscripts} (also called \emph{indices}) which are elements of $\Lambda$, such as $s_{\lambda}$ for some $\lambda \in \Lambda$.  If, however, the surjection from $\Lambda$ to $S$ were called $s$, this notation would be quite \PMlinkescapetext{similar} to the function notation: $s(\lambda)=s_{\lambda}$.

Indexing sets are quite useful for describing sequences, nets, summations, products, unions, and intersections.

Multiple indices are possible.  For example, consider the set $X=\{x_{aa},x_{ab},x_{ac},x_{bb},x_{bc},x_{cc}\}$.  Some people would consider the indexing set for $X$ to be $\{aa,ab,ac,bb,bc,cc\}$.  Others would consider the indexing set to be $\{a,b,c\} \times \{a,b,c\}$.  (The double indices can be considered as ordered pairs.)  Thus, in the case of multiple indices, it need not be the case that the underlying function $f$ be a surjection.  On the other hand, $f$ must be a partial surjection.  For example, if a set $X$ is indexed by $A \times B$, the following must hold:

\begin{enumerate}
\item For every $x\in X$, there exist $i\in A$ and $j\in B$ such that $f(i,j)=x$;
\item For every $i\in A$, the map $f_i \colon B \to X$ defined by $f_i(j)=f(i,j)$ is a partial function;
\item For every $j\in B$, the map $f_j \colon A \to X$ defined by $f_j(i)=f(i,j)$ is a partial function.
\end{enumerate}
%%%%%
%%%%%
\end{document}
