\documentclass[12pt]{article}
\usepackage{pmmeta}
\pmcanonicalname{RecursiveSet}
\pmcreated{2013-03-22 17:34:52}
\pmmodified{2013-03-22 17:34:52}
\pmowner{CWoo}{3771}
\pmmodifier{CWoo}{3771}
\pmtitle{recursive set}
\pmrecord{16}{39993}
\pmprivacy{1}
\pmauthor{CWoo}{3771}
\pmtype{Definition}
\pmcomment{trigger rebuild}
\pmclassification{msc}{03B25}
\pmclassification{msc}{03D20}
\pmsynonym{decidable set}{RecursiveSet}
\pmsynonym{computable set}{RecursiveSet}
\pmsynonym{decidable predicate}{RecursiveSet}
\pmsynonym{computable predicate}{RecursiveSet}
%\pmkeywords{decidable}
\pmdefines{recursively enumerable set}
\pmdefines{recursive predicate}
\pmdefines{recursively enumerable predicate}
\pmdefines{recursive language}

\endmetadata

\usepackage{amssymb,amscd}
\usepackage{amsmath}
\usepackage{amsfonts}
\usepackage{mathrsfs}

% used for TeXing text within eps files
%\usepackage{psfrag}
% need this for including graphics (\includegraphics)
%\usepackage{graphicx}
% for neatly defining theorems and propositions
\usepackage{amsthm}
% making logically defined graphics
%%\usepackage{xypic}
\usepackage{pst-plot}
\usepackage{psfrag}

% define commands here
\newtheorem{prop}{Proposition}
\newtheorem{thm}{Theorem}
\newtheorem{ex}{Example}
\newcommand{\real}{\mathbb{R}}
\newcommand{\pdiff}[2]{\frac{\partial #1}{\partial #2}}
\newcommand{\mpdiff}[3]{\frac{\partial^#1 #2}{\partial #3^#1}}
\begin{document}
\PMlinkescapeword{recursive}

A subset $S$ of the natural numbers $\mathbb{N}$ is said to be \emph{recursive} if its characteristic function  
\begin{eqnarray*}
\chi_S(x) &:=& \left\{
\begin{array}{ll}
1 & \mbox{ if } x\in S \\
0 & \mbox{ if } x\in \mathbb{N}-S
\end{array}\right.
\end{eqnarray*}
is recursive (computable).  In other words, there is an algorithm (via Turing machine for example) that determines whether an element is in $S$ or not in $S$.

More generally, a subset $S\subseteq \mathbb{N}^n$ is \emph{recursive} if its characteristic function $f_S$ is recursive.

A recursive set is also known as a \emph{decidable set} or a \emph{computable set}.

Examples of recursive sets are finite subset of $\mathbb{N}$, the set $\mathbb{N}$ itself, the set of even integers, the set of Fibonacci numbers, the set of pairs $(a,b)$ where $a$ divides $b$, and the set of prime numbers.  In the last example, one may use the Sieve of Eratosthenes as an algorithm to determine the primality of an integer.

A set $S\subseteq \mathbb{N}$ is \emph{recursively enumerable} if the partial function 
\begin{eqnarray*}
f(x) &:=& \left\{
\begin{array}{ll}
1 & \mbox{ if } x\in S \\
\mbox{undefined} & \mbox{ if } x\in \mathbb{N}-S
\end{array}\right.
\end{eqnarray*}
is computable.  In other words, there is an algorithm that halts (and returns $1$) only when an element in $S$ is used as an input.

\textbf{Remarks}
\begin{itemize}
\item A special case of a recursive set is that of a \emph{primitive recursive set}.  A set is \emph{primitive recursive} if its characteristic function is \PMlinkname{primitive recursive}{PrimitiveRecursive}.  All of the examples cited above are primitive recursive.
\item On the other hand, one can broaden the scope of recursiveness to sets which are not necessarily subsets of $\mathbb{N}^n$.  Below are two examples:
\begin{itemize}
\item Since $\mathbb{Z}$ can be effectively embedded in $\mathbb{N}$, so the notion of recursive sets be extended to subsets of $\mathbb{Z}$.
\item Since every finite set $\Sigma$ can be encoded by the natural numbers, we can define a \emph{recursive language} over $\Sigma$ to be a subset $L\subseteq \Sigma^*$ such that $L$, when encoded by the natural numbers, is a recursive set.  Equivalently, $L$ is recursive iff there is a Turing machine that decides $L$ (accepts $L$ and rejects $\Sigma^*-L$).
\end{itemize}
\item Similarly, recursive enumerability can be defined on languages: a language $L$ over $\Sigma$ is \emph{recursively enumerable} if its encoding by the natural numbers is a recursively enumerable set.  This is equivalent to saying that $L$ is accepted by a Turing machine.
\item Using the above definition, one can define a \emph{recursive predicate} or a \emph{recursively enumerable predicate} $\varphi(x)$ according to whether $\lbrace x\mid \varphi(x)\rbrace$ is a recursive or recursively enumerable set respectively.
\end{itemize}
%%%%%
%%%%%
\end{document}
