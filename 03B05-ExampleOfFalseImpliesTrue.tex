\documentclass[12pt]{article}
\usepackage{pmmeta}
\pmcanonicalname{ExampleOfFalseImpliesTrue}
\pmcreated{2013-03-22 18:49:52}
\pmmodified{2013-03-22 18:49:52}
\pmowner{Algeboy}{12884}
\pmmodifier{Algeboy}{12884}
\pmtitle{example of false implies true}
\pmrecord{9}{41635}
\pmprivacy{1}
\pmauthor{Algeboy}{12884}
\pmtype{Example}
\pmcomment{trigger rebuild}
\pmclassification{msc}{03B05}

\endmetadata

\usepackage{latexsym}
\usepackage{amssymb}
\usepackage{amsmath}
\usepackage{amsfonts}
\usepackage{amsthm}

%%\usepackage{xypic}

%-----------------------------------------------------

%       Standard theoremlike environments.

%       Stolen directly from AMSLaTeX sample

%-----------------------------------------------------

%% \theoremstyle{plain} %% This is the default

\newtheorem{thm}{Theorem}

\newtheorem{coro}[thm]{Corollary}

\newtheorem{lem}[thm]{Lemma}

\newtheorem{lemma}[thm]{Lemma}

\newtheorem{prop}[thm]{Proposition}

\newtheorem{conjecture}[thm]{Conjecture}

\newtheorem{conj}[thm]{Conjecture}
\newtheorem{claim}{Claim}

\newtheorem{defn}[thm]{Definition}

\newtheorem{remark}[thm]{Remark}

\newtheorem{ex}[thm]{Example}



%\countstyle[equation]{thm}



%--------------------------------------------------

%       Item references.

%--------------------------------------------------


\newcommand{\exref}[1]{Example-\ref{#1}}

\newcommand{\thmref}[1]{Theorem-\ref{#1}}

\newcommand{\defref}[1]{Definition-\ref{#1}}

\newcommand{\eqnref}[1]{(\ref{#1})}

\newcommand{\secref}[1]{Section-\ref{#1}}

\newcommand{\lemref}[1]{Lemma-\ref{#1}}

\newcommand{\propref}[1]{Prop\-o\-si\-tion-\ref{#1}}

\newcommand{\corref}[1]{Cor\-ol\-lary-\ref{#1}}

\newcommand{\figref}[1]{Fig\-ure-\ref{#1}}

\newcommand{\conjref}[1]{Conjecture-\ref{#1}}


% Normal subgroup or equal.

\providecommand{\normaleq}{\unlhd}

% Normal subgroup.

\providecommand{\normal}{\lhd}

\providecommand{\rnormal}{\rhd}
% Divides, does not divide.

\providecommand{\divides}{\mid}

\providecommand{\ndivides}{\nmid}


\providecommand{\union}{\cup}

\providecommand{\bigunion}{\bigcup}

\providecommand{\intersect}{\cap}

\providecommand{\bigintersect}{\bigcap}










\begin{document}
A common mistake in writing proofs is to overlook that false can imply true.
Consider the following example:

\begin{claim}
$2=5$.
\end{claim}
\begin{proof}[False proof.]
$2=5 \Rightarrow 0\cdot 2=0\cdot 5 \Rightarrow 0=0$.  Clearly $0=0$ so $2=5$.
\end{proof}

Of course this is wrong and $2\neq 5$.  The mistake is that we start from a false premise,
that of $2=5$, and then use completely correct methods (multiply an equation on both sides by
the same number) to arrive a true statement, that $0=0$.
That process is logical: false can imply true.  However, the mistake in the ``proof'' is to 
think that only true can imply true, as then we are lead to believe that $0=0$ makes $2=5$.

This line of argument is absurd in the example just given, but it is often hard to spot and
many times a good looking proof is hiding such a mistake.  For example:

\begin{claim}
$\displaystyle \sum_{k=1}^n k = \frac{n(n+1)}{2}$ for all integers $n\geq 1$.
\end{claim}
\begin{proof}[Incorrect proof.]
We proceed by induction.  In the base case $n=1$:
\begin{align*}
   \sum_{k=1}^1 k & =\frac{1(1+1)}{2}\\
   1 & = 1.
\end{align*}
Therefore the base case is true.

Induction hypothesis: suppose that for some $n\geq 1$,
$\displaystyle \sum_{k=1}^n k =\frac{n(n+1)}{2}$.
Now consider the case $n+1$.
\begin{align*}
  \sum_{k=1}^{n+1} k & = \frac{(n+1)((n+1)+1)}{2}\\
  (n+1) + \sum_{k=1}^n k & = \frac{(n+1)(n+2)}{2}\\
  (n+1) + \frac{n(n+1)}{2} & = \frac{n^2+3n+2}{2}\\
  \frac{2n+2}{2} +\frac{n^2+n}{2} & = \frac{n^2+3n+2}{2}\\
  \frac{n^2+3n+2}{2} & = \frac{n^2+3n +2}{2}.
\end{align*}
So by induction the theorem is true for all $n\geq 1$.
\end{proof}

Notice both the base case and the inductive step assume what is to be proved and then
using only logical implications arrive at a place where something is obviously true,
such as $1=1$ or $\frac{n^2+3n+2}{2}=\frac{n^2+3n+2}{2}$.  What makes these both incorrect
is that:
\begin{quote}
  \textbf{We may arrive at truth by starting with either truth or fiction!}
\end{quote}
That is the way we define implication (at least in logic).

Some common signs of this mistake include:
\begin{enumerate}
\item The proof begins by assuming what is to be proved.  

The exception to this is a 
proof by contradiction.  Then the proof will end with something false.  
Implications do not allow this: true must imply only true.  Hence it is accurate
to then conclude that the assumption at the start was false.  

Because of the possible confusion, it is almost always necessary to start a proof by contradiction
with a phrase of the form "Suppose not" or "We proceed by contradiction".

\item The proof ends with something that is always true, such as $1=1$ or $0=0$ or $x=x$.

\item The proof cannot be read aloud.

This is an often overlooked tool in writing good proofs.  Try to insert words like ``implies''
or `` from this it follows that '' for symbols like $\rightarrow$  or $\Rightarrow$ or (even just new lines).
If it sounds silly when read aloud in this way the chances are high that something is wrong.
For example, the base case in the inductive proof above could be read as:
\begin{quote}
The sum from $k=1$ to $1$ of $k$ equals $\frac{1(1+1)}{2}$ implies that $1=1$.
\end{quote}
Think: isn't $1=1$ obvious?  Why would that need to be implied by some ugly formula?  Ah, it 
would \emph{not} need to be.  Perhaps the implication is not meaningful.
\end{enumerate}

\noindent\textbf{Correcting the mistake}

For many problems the mistaken proof is nevertheless a good way to begin thinking about a correct proof.
This is because by assuming what is to be proved one has something to begin with.  By simplifying the
problem down to something recognizably true, for example $1=1$, it then seems certain that the equation
is correct.  To prove the result it is often a matter of only reversing the scratch work.  To see this
we use the ingredients of the incorrect proof of $\sum_{k=1}^n k = \frac{n(n+1)}{2}$ in the following correct
proof.  Notice also that we have included many more words and punctuation to the proof which makes it
both readable and also helps us to understand the logic of the proof more easily.

\begin{proof}
We proceed by induction.  In the base case $n=1$ and so on the left hand side of our equation we find that
$\sum_{k=1}^n k=1$.  On the right hand side we have $\frac{1(1+1)}{2}=1$.  As the left hand side equals
the right hand side, the base case is settled.

Now suppose for induction that $\sum_{k=1}^n k = \frac{n(n+1)}{2}$ for some integer $n\geq 1$.
We then consider the case of the equation for $n+1$.  On the left hand side we have:
$\sum_{k=1}^{n+1} k= (n+1)+\sum_{k=1}^n k=n+1+\frac{n(n+1)}{2}$ with that last step applying
the induction hypothesis.  From here we use basic algebra to see that 
$n+1+\frac{n(n+1)}{2}=\frac{(n+1)(n+2)}{2}=\frac{(n+1)((n+1)+1)}{2}$.  Hence, by induction the
equation holds for all $n\geq 1$.
\end{proof}

%%%%%
%%%%%
\end{document}
