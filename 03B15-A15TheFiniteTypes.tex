\documentclass[12pt]{article}
\usepackage{pmmeta}
\pmcanonicalname{A15TheFiniteTypes}
\pmcreated{2013-11-09 4:53:25}
\pmmodified{2013-11-09 4:53:25}
\pmowner{PMBookProject}{1000683}
\pmmodifier{PMBookProject}{1000683}
\pmtitle{A.1.5 The finite types}
\pmrecord{1}{}
\pmprivacy{1}
\pmauthor{PMBookProject}{1000683}
\pmtype{Feature}
\pmclassification{msc}{03B15}

\endmetadata

\usepackage{xspace}
\usepackage{amssyb}
\usepackage{amsmath}
\usepackage{amsfonts}
\usepackage{amsthm}
\makeatletter
\newcommand{\defeq}{\vcentcolon\equiv}  
\newcommand{\emptyt}{\ensuremath{\mathbf{0}}\xspace}
\def\tprd#1{\@tprd{#1}\@ifnextchar\bgroup{\tprd}{}}
\def\@tprd#1{\mathchoice{{\textstyle\prod_{(#1)}}}{\prod_{(#1)}}{\prod_{(#1)}}{\prod_{(#1)}}}
\newcommand{\ttt}{\ensuremath{\star}\xspace}
\newcommand{\unit}{\ensuremath{\mathbf{1}}\xspace}
\newcommand{\UU}{\ensuremath{\mathcal{U}}\xspace}
\newcommand{\vcentcolon}{:\!\!}
\makeatother
\begin{document}
We introduce primitive constants $\ttt$, $\emptyt$, $\unit$, satisfying the following rules:
%
\begin{itemize}
\item $\emptyt : \UU_0$, $\unit : \UU_0$
\item $\ttt:\unit$
\end{itemize}

Given $C : \emptyt \rightarrow \UU_n$ we can introduce a defined constant $f:\tprd{x:\emptyt} C(x)$, with no defining equations.

Given $C : \unit \rightarrow \UU_n$ and $d : C(\ttt)$ we can introduce a defined constant $f:\tprd{x:\unit} C(x)$, with defining equation $f(\ttt) \defeq d$.

\end{document}
