\documentclass[12pt]{article}
\usepackage{pmmeta}
\pmcanonicalname{7HomotopyNtypes}
\pmcreated{2013-11-18 17:51:03}
\pmmodified{2013-11-18 17:51:03}
\pmowner{PMBookProject}{1000683}
\pmmodifier{rspuzio}{6075}
\pmtitle{7. Homotopy n-types}
\pmrecord{2}{87697}
\pmprivacy{1}
\pmauthor{PMBookProject}{6075}
\pmtype{Feature}
\pmclassification{msc}{03B15}

\usepackage{amssyb}
\usepackage{amsmath}
\usepackage{amsfonts}
\usepackage{amsthm}
\newcommand{\indexsee}[2]{\index{#1|see{#2}}}    
\let\autoref\cref
\begin{document}

\index{n-type@$n$-type|(}%
\indexsee{h-level}{$n$-type}

One of the basic notions of homotopy theory is that of a \emph{homotopy $n$-type}: a space containing no interesting homotopy above dimension $n$.
For instance, a homotopy $0$-type is essentially a set, containing no nontrivial paths, while a homotopy $1$-type may contain nontrivial paths, but no nontrivial paths between paths.
Homotopy $n$-types are also called \emph{$n$-truncated spaces}.
We have mentioned this notion already in \PMlinkname{\S 3.1}{31setsandntypes}; our first goal in this chapter is to give it a precise definition in homotopy type theory.

A dual notion to truncatedness is connectedness: a space is \emph{$n$-connected} if it has no interesting homotopy in dimensions $n$ and \emph{below}.
For instance, a space is $0$-connected (also called just ``connected'') if it has only one connected component, and $1$-connected (also called ``simply connected'') if it also has no nontrivial loops (though it may have nontrivial higher loops between loops\index{loop!n-@$n$-}).

The duality between truncatedness and connectedness is most easily seen by extending both notions to maps.
We call a map \emph{$n$-truncated} or \emph{$n$-connected} if all its fibers are so.
Then $n$-connected and $n$-truncated maps form the two classes of maps in an \emph{orthogonal factorization system},
\index{orthogonal factorization system}
\indexsee{factorization!system, orthogonal}{orthogonal factorization system}
i.e.\ every map factors uniquely as an $n$-connected map followed by an $n$-truncated one.

In the case $n={-1}$, the $n$-truncated maps are the embeddings and the $n$-connected maps are the surjections, as defined in \PMlinkname{\S 4.6}{46surjectionsandembeddings}.
Thus, the $n$-connected factorization system is a massive generalization of the standard image factorization of a function between sets into a surjection followed by an injection.
At the end of this chapter, we sketch briefly an even more general theory: any type-theoretic \emph{modality} gives rise to an analogous factorization system.


\end{document}
