\documentclass[12pt]{article}
\usepackage{pmmeta}
\pmcanonicalname{InverseStatement}
\pmcreated{2013-03-22 17:20:00}
\pmmodified{2013-03-22 17:20:00}
\pmowner{Wkbj79}{1863}
\pmmodifier{Wkbj79}{1863}
\pmtitle{inverse statement}
\pmrecord{10}{39686}
\pmprivacy{1}
\pmauthor{Wkbj79}{1863}
\pmtype{Definition}
\pmcomment{trigger rebuild}
\pmclassification{msc}{03B05}
\pmsynonym{inverse}{InverseStatement}
\pmrelated{Converse}
\pmrelated{SomethingRelatedToContrapositive}
\pmrelated{ConverseTheorem}

\usepackage{amssymb}
\usepackage{amsmath}
\usepackage{amsfonts}
\usepackage{pstricks}
\usepackage{psfrag}
\usepackage{graphicx}
\usepackage{amsthm}
%%\usepackage{xypic}

\begin{document}
\PMlinkescapeword{word}
\PMlinkescapeword{words}

Let a statement be of the form of an implication

\begin{center}
If $p$, then $q$
\end{center}

\PMlinkname{i.e.}{Ie}, it has a certain premise $p$ and a conclusion $q$. The statement in which one has negated the conclusion and the premise,

\begin{center}
If $\neg p$, then $\neg q$
\end{center}

is the \emph{inverse} (or \emph{inverse statement}) of the first.  Note that the following constructions yield the same statement:

\begin{itemize}
\item the inverse of the original statement;
\item the contrapositive of the converse of the original statement;
\item the converse of the contrapositive of the original statement.
\end{itemize}

Therefore, just as an implication and its contrapositive are logically equivalent (proven \PMlinkname{here}{SomethingRelatedToContrapositive}), the converse of the original statement and the inverse of the original statement are also logically equivalent.

The phrase ``inverse theorem'' is in \PMlinkescapetext{current} usage; however, it is nothing akin to the phrase ``\PMlinkname{converse theorem}{ConverseTheorem}''.  In the phrase ``inverse theorem'', the word ``inverse'' typically refers to a multiplicative inverse.  An example of this usage is the \PMlinkname{binomial inverse theorem}{BinomialInverseTheorem}.
%%%%%
%%%%%
\end{document}
