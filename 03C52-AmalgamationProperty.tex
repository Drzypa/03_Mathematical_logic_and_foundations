\documentclass[12pt]{article}
\usepackage{pmmeta}
\pmcanonicalname{AmalgamationProperty}
\pmcreated{2013-03-22 13:25:01}
\pmmodified{2013-03-22 13:25:01}
\pmowner{Mathprof}{13753}
\pmmodifier{Mathprof}{13753}
\pmtitle{amalgamation property}
\pmrecord{8}{33964}
\pmprivacy{1}
\pmauthor{Mathprof}{13753}
\pmtype{Definition}
\pmcomment{trigger rebuild}
\pmclassification{msc}{03C52}
\pmrelated{FreeProductWithAmalgamatedSubgroup}
\pmrelated{Confluence}
\pmrelated{JointEmbeddingProperty}
\pmdefines{amalgamation property}

\endmetadata

% this is the default PlanetMath preamble.  as your knowledge
% of TeX increases, you will probably want to edit this, but
% it should be fine as is for beginners.

% almost certainly you want these
\usepackage{amssymb}
\usepackage{amsmath}
\usepackage{amsfonts}

% used for TeXing text within eps files
%\usepackage{psfrag}
% need this for including graphics (\includegraphics)
%\usepackage{graphicx}
% for neatly defining theorems and propositions
%\usepackage{amsthm}
% making logically defined graphics
%%\usepackage{xypic}

% there are many more packages, add them here as you need them

% define commands here

\def\ra{\rightarrow}
\begin{document}
A class of $L$-structures $S$ has the {\em amalgamation property} if and only if
whenever $A,B_{1},B_{2} \in S$ and  $f_{i}:A \ra B_{i}$ are elementary embeddings 
for $i \in \{1,2\}$ then there is some $C \in S$ and some elementary embeddings 
$g_{i}:B_{i} \ra C$ for $i \in \{1,2\}$ so that $g_{1}(f_{1}(x))=g_{2}(f_{2}(x))$ 
for all $x \in A$. That is, the following diagram commutes.

$$\xymatrix{
& {A} \ar[dl]_{f_1} \ar[dr]^{f_2} & \\
{B_1} \ar[dr]_{g_1} & & {B_2} \ar[dl]^{g_2} \\
& {C} & 
}
$$


Compare this with the free product with amalgamated subgroup for groups and 
the definition of pushout \PMlinkescapetext{contained} there.
%%%%%
%%%%%
\end{document}
