\documentclass[12pt]{article}
\usepackage{pmmeta}
\pmcanonicalname{TheParadoxOfTristramShandy}
\pmcreated{2013-05-06 7:43:07}
\pmmodified{2013-05-06 7:43:07}
\pmowner{WM}{16977}
\pmmodifier{WM}{16977}
\pmtitle{The Paradox of Tristram Shandy}
\pmrecord{1}{}
\pmprivacy{1}
\pmauthor{WM}{16977}
\pmtype{Definition}
\pmclassification{msc}{03Exx}

\endmetadata

% this is the default PlanetMath preamble.  as your knowledge
% of TeX increases, you will probably want to edit this, but
% it should be fine as is for beginners.

% almost certainly you want these
\usepackage{amssymb}
\usepackage{amsmath}
\usepackage{amsfonts}

% need this for including graphics (\includegraphics)
\usepackage{graphicx}
% for neatly defining theorems and propositions
\usepackage{amsthm}

% making logically defined graphics
%\usepackage{xypic}
% used for TeXing text within eps files
%\usepackage{psfrag}

% there are many more packages, add them here as you need them

% define commands here

\begin{document}

Bijections between different infinite sets can lead to paradoxes like that of Tristram Shandy that Adolf Fraenkel tells us in order to explain this fundamental feature of set theory (A.A. Fraenkel: "Einleitung in die Mengenlehre" 3. Aufl., Springer, Berlin 1928, p. 24): "Well known is the story of Tristram Shandy who is going to write his biography so pedantically that he needs one year to complete a single day of his life. Of course his biography will never get ready, when continuing in this manner. But if he lived infinitely long (a countable infinity of years, say) then his biography would get 'ready', every day of his life, how late ever, finally would get a description." Tristram Shandy borrows time in order to catch up with time already spent. (Similarities with modern finance schemes are unmistakable.)

But we will show that this method leads to a contradiction. For that sake we use the less spectacular but better understandable ratio of 2 to 1. We fill successively pairs of consecutively enumerated marbles into an urn. Between every two steps of filling we remove the marble with the lowest number from the urn. If the numbers contained in the urn are separated by a point from those removed already, this yields the following sequence

21., 2.1, 432.1, 43.21, 6543.21, 654.321, ...

which however gets confusing when multi-digit numbers are involved. The picture gets clearer, when the even numbers are denoted by 0 and the odd numbers by 1.

01., 0.1, 010.1, 01.01, 0101.01, 010.101, ...

After infinitely many exchanges we have a contradiction between set theory and mathematics: Set theory yields as the limit the empty urn because for every marble the step can be determined, when it is removed.

But this very sequence can also be understood as a sequence of real numbers r. Mathematics shows that the (improper) limit is infinite, proved by the proper limit 0 of the sequence of the reciprocals 1/r.

Since the logarithm is a strictly increasing function of its argument and gives the number of digits of x by [log x] + 1, we have a contradiction in that, according to mathematics, there will be a set of infinitely many indexed digits on the left hand side of the point in the limit.

As one and the same sequence of sets of digits leads to different results in set theory and in mathematics, set theory can no longer be accepted as the basis of mathematics.  In particular the results of a scientific theory must not depend on the choice of symbols. But even that occurs in set theory. If we let the above thought-experiment run materially exactly as before, but remove always the largest number instead of the smallest, then the urn contains infinitely many marbles in the limit. The question what might happen when an accidental number is removed remains without answer.

How could such a contradiction appear? The true reason is the identification of all and each that is based upon and suggested by the use of the same symbol in logic. A bijection between natural numbers and a countable set enumerates each element like a street number is fixed on a house. But that means only for finite sets that all elements are enumerated. In infinite sets there is for each enumerated element an infinity of not enumerated elements - and this relation does never change. Unfortunately in set theory "never" is mistaken for a time that can be arrived at. Just like every infinity can be completed, so can eternity as an infinite set of seconds.


The basic error of set theory is that the correct meaning of potential infinity has been reversed to actual infinity.

Potential infinity: For every initial segment (1, 2, 3, ..., n) of natural numbers, there is a larger initial segment (1, 2, 3, ..., n, n+1).

Actual infinity: There is an initial segment (1, 2, 3, ...), namely the sequence of all natural numbers, that is larger than every other initial segment (1, 2, 3, ..., n).

That means, that a quantifier exchange has occured. The correct sentence "for every initial segment (1, 2, 3, ..., n) of the natural numbers there is a larger one" has been inverted to "there is an initial segment of the natural numbers (namely the complete set of all natural numbers) that is larger than every other initial segment". Actually mathematicians, strictly committed to logic, should have refused to be taken for a ride. Who would invert the sentence "every US citizen lives in America" to "every American lives in the US"? 


\end{document}
