\documentclass[12pt]{article}
\usepackage{pmmeta}
\pmcanonicalname{CircularReasoning}
\pmcreated{2013-03-22 16:06:32}
\pmmodified{2013-03-22 16:06:32}
\pmowner{Wkbj79}{1863}
\pmmodifier{Wkbj79}{1863}
\pmtitle{circular reasoning}
\pmrecord{15}{38174}
\pmprivacy{1}
\pmauthor{Wkbj79}{1863}
\pmtype{Definition}
\pmcomment{trigger rebuild}
\pmclassification{msc}{03F07}
\pmsynonym{circular argument}{CircularReasoning}

% this is the default PlanetMath preamble.  as your knowledge
% of TeX increases, you will probably want to edit this, but
% it should be fine as is for beginners.

% almost certainly you want these
\usepackage{amssymb}
\usepackage{amsmath}
\usepackage{amsfonts}

% used for TeXing text within eps files
%\usepackage{psfrag}
% need this for including graphics (\includegraphics)
%\usepackage{graphicx}
% for neatly defining theorems and propositions
%\usepackage{amsthm}
% making logically defined graphics
%%%\usepackage{xypic}

% there are many more packages, add them here as you need them

% define commands here

\begin{document}
{\sl Circular reasoning\/} is an attempted proof of a statement that uses at least one of the following two things:

\begin{itemize}
\item the statement that is to be proven
\item a fact that relies on the statement that is to be proven
\end{itemize}

Such proofs are not valid.

As an example, below is a faulty proof that the \PMlinkname{well-ordering principle implies the axiom of choice}{WellOrderingPrincipleImpliesAxiomOfChoice}.  The step where circular reasoning is used is surrounded by brackets [ ].

Let $C$ be a collection of nonempty sets.  By the well-ordering principle, each $S \in C$ is well-ordered.  [For each $S \in C$, let $<_S$ denote the well-ordering of $S$.]  Let $m_S$ denote the least member of each $S \in C$ with respect to $<_S$.  Then a choice function $\displaystyle f \colon C \to \bigcup_{S \in C} S$ can be defined by $f(S)=m_S$.

The step surrounded by brackets is faulty because it actually uses the axiom of choice, which is what is to be proven.  In the step, for each $S \in C$, an ordering is chosen.  This cannot be done in general without appealing to the axiom of choice.
%%%%%
%%%%%
\end{document}
