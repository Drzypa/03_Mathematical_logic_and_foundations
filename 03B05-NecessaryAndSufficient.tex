\documentclass[12pt]{article}
\usepackage{pmmeta}
\pmcanonicalname{NecessaryAndSufficient}
\pmcreated{2013-03-22 16:07:31}
\pmmodified{2013-03-22 16:07:31}
\pmowner{Wkbj79}{1863}
\pmmodifier{Wkbj79}{1863}
\pmtitle{necessary and sufficient}
\pmrecord{12}{38195}
\pmprivacy{1}
\pmauthor{Wkbj79}{1863}
\pmtype{Definition}
\pmcomment{trigger rebuild}
\pmclassification{msc}{03B05}
\pmclassification{msc}{03F07}
\pmrelated{UniversalAssumption}
\pmrelated{SufficientConditionOfPolynomialCongruence}
\pmdefines{necessary}
\pmdefines{necessity}
\pmdefines{sufficient}
\pmdefines{sufficiency}

% this is the default PlanetMath preamble.  as your knowledge
% of TeX increases, you will probably want to edit this, but
% it should be fine as is for beginners.

% almost certainly you want these
\usepackage{amssymb}
\usepackage{amsmath}
\usepackage{amsfonts}

% used for TeXing text within eps files
%\usepackage{psfrag}
% need this for including graphics (\includegraphics)
%\usepackage{graphicx}
% for neatly defining theorems and propositions
%\usepackage{amsthm}
% making logically defined graphics
%%%\usepackage{xypic}

% there are many more packages, add them here as you need them

% define commands here

\begin{document}
\PMlinkescapeword{terms}

The statement ``$p$ is {\sl necessary\/} for $q$'' \PMlinkescapetext{means} ``$q$ \PMlinkname{implies}{Implication} $p$''.

The statement ``$p$ is {\sl sufficient\/} for $q$'' \PMlinkescapetext{means} ``$p$ \PMlinkname{implies}{Implication} $q$''.

The statement ``$p$ is {\sl necessary and sufficent\/} for $q$'' \PMlinkescapetext{means} ``$p$ \PMlinkname{if and only if}{Iff} $q$''.

For an example of how these terms are used in mathematics, see the entry on complete ultrametric fields.

Biconditional statements are often proven by breaking them into two implications and proving them separately.  Often, the terms \emph{necessity} and \emph{sufficiency} are used to indicate which implication is being proven.  For an example of this usage, see the entry called relationship between totatives and divisors.
%%%%%
%%%%%
\end{document}
