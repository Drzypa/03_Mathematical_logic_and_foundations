\documentclass[12pt]{article}
\usepackage{pmmeta}
\pmcanonicalname{AxiomOfPairing}
\pmcreated{2013-03-22 13:42:43}
\pmmodified{2013-03-22 13:42:43}
\pmowner{Sabean}{2546}
\pmmodifier{Sabean}{2546}
\pmtitle{axiom of pairing}
\pmrecord{7}{34392}
\pmprivacy{1}
\pmauthor{Sabean}{2546}
\pmtype{Axiom}
\pmcomment{trigger rebuild}
\pmclassification{msc}{03E30}
\pmsynonym{pairing}{AxiomOfPairing}

\endmetadata

% this is the default PlanetMath preamble.  as your knowledge
% of TeX increases, you will probably want to edit this, but
% it should be fine as is for beginners.

% almost certainly you want these
\usepackage{amssymb}
\usepackage{amsmath}
\usepackage{amsfonts}

% used for TeXing text within eps files
%\usepackage{psfrag}
% need this for including graphics (\includegraphics)
%\usepackage{graphicx}
% for neatly defining theorems and propositions
%\usepackage{amsthm}
% making logically defined graphics
%%%\usepackage{xypic}

% there are many more packages, add them here as you need them

% define commands here
\begin{document}
For any $a$ and $b$ there exists a set $\{ a, b \}$ that contains exactly $a$ and $b$.

The Axiom of Pairing is one of the axioms of Zermelo-Fraenkel set theory.  In symbols, it reads:
\[
\forall a \forall b \exists c \forall x (x \in c \leftrightarrow x = a \lor x = b).
\]
Using the Axiom of Extensionality, we see that the set $c$ is unique, so it makes sense to define the pair
\[
\{ a, b \} = \mbox{ the unique } c \mbox{ such that } \forall x (x \in c \leftrightarrow x = a \lor x = b).
\]

Using the Axiom of Pairing, we may define, for any set $a$, the singleton
\[
\{ a \} = \{ a, a \}.
\]

We may also define, for any set $a$ and $b$, the ordered pair
\[
(a, b) = \{ \{ a \}, \{ a, b \} \}.
\]

Note that this definition satisfies the condition
\[
(a, b) = (c, d) \mbox{ iff } a = c \mbox{ and } b = d.
\]

We may define the ordered $n$-tuple recursively
\[
(a_1, \ldots, a_n) = ((a_1, \ldots, a_{n-1}), a_n).
\]
%%%%%
%%%%%
\end{document}
