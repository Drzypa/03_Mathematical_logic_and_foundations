\documentclass[12pt]{article}
\usepackage{pmmeta}
\pmcanonicalname{ShefferStroke}
\pmcreated{2013-03-22 18:51:55}
\pmmodified{2013-03-22 18:51:55}
\pmowner{CWoo}{3771}
\pmmodifier{CWoo}{3771}
\pmtitle{Sheffer stroke}
\pmrecord{4}{41695}
\pmprivacy{1}
\pmauthor{CWoo}{3771}
\pmtype{Definition}
\pmcomment{trigger rebuild}
\pmclassification{msc}{03B05}
\pmsynonym{alternative denial}{ShefferStroke}
\pmsynonym{NAND}{ShefferStroke}
\pmsynonym{joint denial}{ShefferStroke}
\pmsynonym{NOR}{ShefferStroke}
\pmdefines{Peirce arrow}

\endmetadata

\usepackage{amssymb,amscd}
\usepackage{amsmath}
\usepackage{amsfonts}
\usepackage{mathrsfs}

% used for TeXing text within eps files
%\usepackage{psfrag}
% need this for including graphics (\includegraphics)
%\usepackage{graphicx}
% for neatly defining theorems and propositions
\usepackage{amsthm}
% making logically defined graphics
%%\usepackage{xypic}
\usepackage{pst-plot}

% define commands here
\newcommand*{\abs}[1]{\left\lvert #1\right\rvert}
\newtheorem{prop}{Proposition}
\newtheorem{thm}{Theorem}
\newtheorem{ex}{Example}
\newcommand{\real}{\mathbb{R}}
\newcommand{\pdiff}[2]{\frac{\partial #1}{\partial #2}}
\newcommand{\mpdiff}[3]{\frac{\partial^#1 #2}{\partial #3^#1}}
\begin{document}
In the late 19th century and early 20th century, Charles Sanders Peirce and H.M. Sheffer independently discovered that a single binary logical connective suffices to define all logical connectives (they are each functionally complete).  Two such connectives are
\begin{itemize}
\item
$\uparrow$: the \emph{Sheffer stroke} (sometimes denoted by $|$) and

\item
$\downarrow$: the \emph{Peirce arrow} (sometimes denoted by $\bot$).
\end{itemize}

The Sheffer stroke is defined by the truth table
\begin{center}
\begin{tabular}{ccc}
$P$ & $Q$ & $P \uparrow Q$ \\
\hline
F & F & T \\
F & T & T \\
T & F & T \\
T & T & F
\end{tabular}
\end{center}
Observe that $P\uparrow Q$ is true if and only if either $P$ or $Q$ is false.  For this reason, the Sheffer stroke is sometimes called \emph{alternative denial} or \emph{NAND}.

The Peirce arrow is defined by the truth table
\begin{center}
\begin{tabular}{ccc}
$P$ & $Q$ & $P \downarrow Q$ \\
\hline
F & F & T \\
F & T & F \\
T & F & F \\
T & T & F
\end{tabular}
\end{center}
The proposition $P\downarrow Q$ is true if and only if both $P$ and $Q$ are false.  For this reason, the Peirce arrow is sometimes called \emph{joint denial} or \emph{NOR}.

To show the sufficiency of the Sheffer stroke, all we have to do is define both $\lnot$ and $\lor$ in terms of $\uparrow$.  The proposition $P\uparrow P$ asserts that either $P$ is false, or $P$ is false; thus we can define $\lnot$ by $\lnot P := P\uparrow P$.  We define $\lor$ by 
\[
P \lor Q := (P\uparrow P)\uparrow(Q\uparrow Q),
\]
since this asserts that either $P\uparrow P$ is false (that is, that $P$ is true) or that $Q\uparrow Q$ is false (that is, that $Q$ is true).

We can show the sufficiency of the Peirce arrow in a similar way.  Define
\[
\lnot P := P\downarrow P
\]
and
\[
P\lor Q := (P\downarrow Q)\downarrow(P\downarrow Q).
\]
This expression asserts that $P\downarrow Q$ is false, that is, that it is false that both $P$ and $Q$ are false.  By DeMorgan's law, this is equivalent to asserting that at least one of $P$ and $Q$ is true.

\textbf{Remark}.  It can be shown that no binary connective, other than Sheffer stroke and Peirce arrow, is functionally complete.
%%%%%
%%%%%
\end{document}
