\documentclass[12pt]{article}
\usepackage{pmmeta}
\pmcanonicalname{DefinitionOfWellOrderedSetAVariant}
\pmcreated{2013-03-22 18:04:47}
\pmmodified{2013-03-22 18:04:47}
\pmowner{yesitis}{13730}
\pmmodifier{yesitis}{13730}
\pmtitle{definition of well ordered set, a variant}
\pmrecord{9}{40616}
\pmprivacy{1}
\pmauthor{yesitis}{13730}
\pmtype{Derivation}
\pmcomment{trigger rebuild}
\pmclassification{msc}{03E25}
\pmclassification{msc}{06A05}
\pmrelated{SomethingRelatedToNaturalNumber}
\pmrelated{NaturalNumbersAreWellOrdered}

% this is the default PlanetMath preamble.  as your knowledge
% of TeX increases, you will probably want to edit this, but
% it should be fine as is for beginners.

% almost certainly you want these
\usepackage{amssymb}
\usepackage{amsmath}
\usepackage{amsfonts}

% used for TeXing text within eps files
%\usepackage{psfrag}
% need this for including graphics (\includegraphics)
%\usepackage{graphicx}
% for neatly defining theorems and propositions
%\usepackage{amsthm}
% making logically defined graphics
%%%\usepackage{xypic}

% there are many more packages, add them here as you need them

% define commands here

\begin{document}
A well-ordered set is normally defined as a \emph{totally} ordered set in which every nonempty subset has a least member, as the parent object does.

It is possible to define well-ordered sets as follows:

\emph{a well-ordered set $X$ is a partially ordered set in which every nonempty subset of $X$ has a least member.}

To justify the alternative, we prove that every partially ordered set $X$ in which every nonempty subset has a least member is total:

let $x\in X$ and $y\in X$, $x\neq y$. Now, $\{x, y\}$ has a least member, a fortiori, $x, y$ are comparable. Hence, $X$ is totally ordered.

The alternative has the benefit of being a stronger statement in the sense that

\begin{equation*}
(partial\;order) \Longrightarrow (total\;order)
\end{equation*}

given that every nonempty subset has a least member.

\begin{thebibliography}{1}
\bibitem{Sc1997}
Schechter, E., \emph{Handbook of Analysis and Its Foundations}, 1st ed., Academic Press, 1997.

\bibitem{Je2002}
Jech, T., \emph{Set Theory}, 3rd millennium ed., Springer, 2002.
\end{thebibliography}

%%%%%
%%%%%
\end{document}
