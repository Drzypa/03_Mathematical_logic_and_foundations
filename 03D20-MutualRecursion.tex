\documentclass[12pt]{article}
\usepackage{pmmeta}
\pmcanonicalname{MutualRecursion}
\pmcreated{2013-03-22 19:06:31}
\pmmodified{2013-03-22 19:06:31}
\pmowner{CWoo}{3771}
\pmmodifier{CWoo}{3771}
\pmtitle{mutual recursion}
\pmrecord{9}{42001}
\pmprivacy{1}
\pmauthor{CWoo}{3771}
\pmtype{Definition}
\pmcomment{trigger rebuild}
\pmclassification{msc}{03D20}
\pmsynonym{simultaneous recursion}{MutualRecursion}

\usepackage{amssymb,amscd}
\usepackage{amsmath}
\usepackage{amsfonts}
\usepackage{mathrsfs}
\usepackage{tabls}

% used for TeXing text within eps files
%\usepackage{psfrag}
% need this for including graphics (\includegraphics)
%\usepackage{graphicx}
% for neatly defining theorems and propositions
\usepackage{amsthm}
% making logically defined graphics
%%\usepackage{xypic}
\usepackage{pst-plot}

% define commands here
\newcommand*{\abs}[1]{\left\lvert #1\right\rvert}
\newtheorem{prop}{Proposition}
\newtheorem{thm}{Theorem}
\newtheorem{ex}{Example}
\newcommand{\real}{\mathbb{R}}
\newcommand{\pdiff}[2]{\frac{\partial #1}{\partial #2}}
\newcommand{\mpdiff}[3]{\frac{\partial^#1 #2}{\partial #3^#1}}
\begin{document}
Mutual recursion is a way of defining functions via recursion involving several functions simultaneously.  In mutual recursion, the value of the next argument of \emph{any} function involved depends on values of the current arguments of \emph{all} functions involved.  The following are two simple examples:
\begin{itemize}
\item the pair of functions $f,g$ such that $f(0)=0$ and $g(0)=1$, with $f(n+1)=f(n)g(n)$ and $g(n+1)=f(n)+g(n)$ is defined via mutual recursion.  It is easy to see that $f(n)=0$ and $g(n)=1$.
\item Fibonacci numbers can be interpreted via mutual recursion: $F(0)=1$ and $G(0)=1$, with $F(n+1)=F(n)+G(n)$ and $G(n+1)=F(n)$.
\end{itemize}

Formally,

\textbf{Definition}.  Functions $f_1,\ldots, f_m: \mathbb{N}^{k+1} \to \mathbb{N}$ are said to be defined by \emph{mutual recursion} via functions $g_1,\ldots g_m: \mathbb{N}^k \to \mathbb{N}$ and functions $h_1,\ldots, h_m: \mathbb{N}^{k+m+1} \to \mathbb{N}$, if 
\begin{eqnarray*}
f_i(\boldsymbol{x},0) &=& g_i(\boldsymbol{x}), \\
f_i(\boldsymbol{x},n+1) &=& h_i(\boldsymbol{x},n,f_1(\boldsymbol{x},n),\ldots, f_m(\boldsymbol{x},n)),
\end{eqnarray*}
for any $\boldsymbol{x}\in \mathbb{N}^k$, and $i=1,\ldots, m$.

Mutual recursion is an apparent generalization of primitive recursion, where the value of the next argument of a function depends on the value of the current argument of the same function, apparent because it is in fact equivalent to primitive recursion.

\begin{prop} As above, if all of $g_i$ and $h_i$ are primitive recursive, so are all of $f_i$. \end{prop}
\begin{proof}  We use the multiplicative encoding technique.  Define function $F:\mathbb{N}^{k+1} \to \mathbb{N}$ as follows:
$$F(\boldsymbol{x},n)=\operatorname{exp}(p_1, f_1(\boldsymbol{x},n))\cdots \operatorname{exp}(p_m, f_m(\boldsymbol{x},n)),$$
where $p_i$ is the $i$-th prime number ($p_1=2$, etc...).  Furthermore, $$(F(\boldsymbol{x},n))_i = f_i(\boldsymbol{x},n),$$
where $(z)_i$ denotes the exponent of prime $p_i$ in $z$.  Then 
\begin{eqnarray*}
F(\boldsymbol{x},0) &=& \prod_{i=1}^m \operatorname{exp}(p_i, g_i(\boldsymbol{x})) = G(\boldsymbol{x}), \mbox{ and} \\
F(\boldsymbol{x},n+1) &=& \prod_{i=1}^m \operatorname{exp}(p_1, f_1(\boldsymbol{x},n+1)) \\
&=& \prod_{i=1}^m \operatorname{exp}(p_i, h_i(\boldsymbol{x},n,f_1(\boldsymbol{x},n),\ldots, f_m(\boldsymbol{x},n))) \\
&=& \prod_{i=1}^m \operatorname{exp}(p_i, h_i(\boldsymbol{x},n, (F(\boldsymbol{x},n))_1,\ldots, (F(\boldsymbol{x},n))_m) \\
&=& H(\boldsymbol{x},n, F(\boldsymbol{x},n))
\end{eqnarray*}
where $$G(\boldsymbol{x}) = \prod_{i=1}^m \operatorname{exp}(p_i, g_i(\boldsymbol{x})),$$ and 
$$H(\boldsymbol{x},y,z) = \prod_{i=1}^m \operatorname{exp}(p_i, h_i(\boldsymbol{x},y, (z)_1,\ldots, (z)_m) $$
Since each of the $g_i$ is primitive recursive, $G$ is primitive recursive.  Since each of the $h_i$ is primitive recursive, $H$ is primitive recursive.  So $F$ is primitive recursive, as it is defined via primitive recursion by $G$ and $H$.  Therefore, each $f_i = (F)_i$ is primitive recursive.
\end{proof}

\textbf{Remark}.  The primitive recursiveness of mutual recursion can be derived from course-of-values recursion.  The idea is the following: list the $f_i$'s in order, and construct a single function $F$ so the its values correspond to the values of the $f_i$'s in the order given.  For example, if two unary functions $f_1$ and $f_2$ defined by mutual recursion, then 
\begin{center}
\begin{tabular}{c c c c c c c c c c c}
$f_1(0)$ & $f_2(0)$ & $f_1(1)$ & $f_2(1)$ & $f_1(2)$ & $\cdots$ & $f_1(k)$ & $f_2(k)$ & $f_1(k+1)$ & $f_2(k+1)$ & $\cdots$ \\
$F(0)$ & $F(1)$ & $F(2)$ & $F(3)$ & $F(4)$ & $\cdots$ & $F(2k)$ & $F(2k+1)$ & $F(2k+2)$ & $F(2k+3)$ &  $\cdots$ \\
\end{tabular}
\end{center}
Then it is not hard to see that $F(k+1)$ depends on $F(k),F(k-1)$, and $F(k-2)$, and an explicit formula can be derived expressing the dependency.  Furthermore, this formula is easily seen to be primitive recursive, and hence so is $F(k)$.
%%%%%
%%%%%
\end{document}
