\documentclass[12pt]{article}
\usepackage{pmmeta}
\pmcanonicalname{Paradox}
\pmcreated{2013-03-22 16:07:49}
\pmmodified{2013-03-22 16:07:49}
\pmowner{Wkbj79}{1863}
\pmmodifier{Wkbj79}{1863}
\pmtitle{paradox}
\pmrecord{14}{38201}
\pmprivacy{1}
\pmauthor{Wkbj79}{1863}
\pmtype{Definition}
\pmcomment{trigger rebuild}
\pmclassification{msc}{03A05}
\pmclassification{msc}{03B99}
\pmsynonym{paradoxical}{Paradox}
\pmsynonym{paradoxically}{Paradox}
\pmsynonym{dilemma}{Paradox}
\pmrelated{GalileosParadox}

\usepackage{amssymb}
\usepackage{amsmath}
\usepackage{amsfonts}

\usepackage{psfrag}
\usepackage{graphicx}
\usepackage{amsthm}
%%\usepackage{xypic}

\begin{document}
A {\sl paradox\/} is an assertion that is apparently self-contradictory, though based on a valid deduction from acceptable premises.

Paradoxes typically lead to a reevaluation of the axioms of mathematics.  \PMlinkescapetext{Even} after axioms are assumed so that the paradox is averted, the statement is still usually referred to as a paradox.

Occasionally, one may refer to a surprising result as a paradox.  Such is the case in the birthday paradox, which is not apparently self-contradictory.

Examples of paradoxes include:

\begin{itemize}
\item Banach-Tarski paradox
\item binary tree paradox
\item \PMlinkname{birthday paradox}{BirthdayParadox}
\item Burali-Forti paradox
\item Cantor's paradox
\item Hausdorff paradox
\item Russell's paradox
\item Simpson's paradox
\item Zeno's paradox
\end{itemize}


%%%%%
%%%%%
\end{document}
