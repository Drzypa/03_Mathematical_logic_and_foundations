\documentclass[12pt]{article}
\usepackage{pmmeta}
\pmcanonicalname{MeasurableAndRealvaluedMeasurableCardinals}
\pmcreated{2013-03-22 18:54:53}
\pmmodified{2013-03-22 18:54:53}
\pmowner{yesitis}{13730}
\pmmodifier{yesitis}{13730}
\pmtitle{measurable and real-valued measurable cardinals}
\pmrecord{4}{41764}
\pmprivacy{1}
\pmauthor{yesitis}{13730}
\pmtype{Definition}
\pmcomment{trigger rebuild}
\pmclassification{msc}{03E55}

% this is the default PlanetMath preamble.  as your knowledge
% of TeX increases, you will probably want to edit this, but
% it should be fine as is for beginners.

% almost certainly you want these
\usepackage{amssymb}
\usepackage{amsmath}
\usepackage{amsfonts}

% used for TeXing text within eps files
%\usepackage{psfrag}
% need this for including graphics (\includegraphics)
%\usepackage{graphicx}
% for neatly defining theorems and propositions
%\usepackage{amsthm}
% making logically defined graphics
%%%\usepackage{xypic}

% there are many more packages, add them here as you need them

% define commands here

\begin{document}
Let $\kappa$ be an uncountable cardinal. Then

\begin{enumerate}
\item $\kappa$ is \emph{measurable} if there exists a nonprincipal $\kappa$-complete ultrafilter $U$ on $\kappa$;
\item $\kappa$ is \emph{real-valued measurable} if there exists a nontrivial $\kappa$-additive measure $\mu$ on $\kappa$.
\end{enumerate}

If $\kappa$ is measurable, then it is real-valued measurable. This is so because the ultrafilter $U$ and its dual ideal $I$ induce a two-valued measure $\mu$ on $\kappa$ where every member of $U$ is mapped to 1 and every member of $I$ is mapped to 0. Since $U$ is $\kappa$-complete, $I$ is also $\kappa$-complete. It can then be proved that if $I_\mu$--the ideal of those sets whose measures are 0--is $\kappa$-complete, then $I_\mu$ is $\kappa$-additive.

On the converse side, if $\kappa$ is not real-valued measurable, then $\kappa\leq 2^{\aleph_0}$. It can be shown that if $\kappa$ is real-valued measurable, then it is regular; a further result is that $\kappa$ is weakly inaccessible. Inaccessible cardinals are in some sense "large."
%%%%%
%%%%%
\end{document}
