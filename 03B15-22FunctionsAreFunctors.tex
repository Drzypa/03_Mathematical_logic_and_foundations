\documentclass[12pt]{article}
\usepackage{pmmeta}
\pmcanonicalname{22FunctionsAreFunctors}
\pmcreated{2013-11-14 3:10:52}
\pmmodified{2013-11-14 3:10:52}
\pmowner{PMBookProject}{1000683}
\pmmodifier{rspuzio}{6075}
\pmtitle{2.2 Functions are functors}
\pmrecord{5}{87599}
\pmprivacy{1}
\pmauthor{PMBookProject}{6075}
\pmtype{Feature}
\pmclassification{msc}{03B15}

\endmetadata

\usepackage{xspace}
\usepackage{amssyb}
\usepackage{amsmath}
\usepackage{amsfonts}
\usepackage{amsthm}
\makeatletter
\newcommand{\ct}{  \mathchoice{\mathbin{\raisebox{0.5ex}{$\displaystyle\centerdot$}}}             {\mathbin{\raisebox{0.5ex}{$\centerdot$}}}             {\mathbin{\raisebox{0.25ex}{$\scriptstyle\,\centerdot\,$}}}             {\mathbin{\raisebox{0.1ex}{$\scriptscriptstyle\,\centerdot\,$}}}}
\newcommand{\defeq}{\vcentcolon\equiv}  
\def\@dprd#1{\prod_{(#1)}\,}
\def\@dprd@noparens#1{\prod_{#1}\,}
\def\@dsm#1{\sum_{(#1)}\,}
\def\@dsm@noparens#1{\sum_{#1}\,}
\def\@eatprd\prd{\prd@parens}
\def\@eatsm\sm{\sm@parens}
\newcommand{\id}[3][]{\ensuremath{#2 =_{#1} #3}\xspace}
\newcommand{\idfunc}[1][]{\ensuremath{\mathsf{id}_{#1}}\xspace}
\newcommand{\indexdef}[1]{\index{#1|defstyle}}   
\newcommand{\jdeq}{\equiv}      
\def\lam#1{{\lambda}\@lamarg#1:\@endlamarg\@ifnextchar\bgroup{.\,\lam}{.\,}}
\def\@lamarg#1:#2\@endlamarg{\if\relax\detokenize{#2}\relax #1\else\@lamvar{\@lameatcolon#2},#1\@endlamvar\fi}
\def\@lameatcolon#1:{#1}
\def\@lamvar#1,#2\@endlamvar{(#2\,{:}\,#1)}
\newcommand{\map}[2]{\ensuremath{{#1}\mathopen{}\left({#2}\right)\mathclose{}}\xspace}
\newcommand{\mapfunc}[1]{\ensuremath{\mathsf{ap}_{#1}}\xspace} 
\newcommand{\opp}[1]{\mathord{{#1}^{-1}}}
\def\prd#1{\@ifnextchar\bgroup{\prd@parens{#1}}{\@ifnextchar\sm{\prd@parens{#1}\@eatsm}{\prd@noparens{#1}}}}
\def\prd@noparens#1{\mathchoice{\@dprd@noparens{#1}}{\@tprd{#1}}{\@tprd{#1}}{\@tprd{#1}}}
\def\prd@parens#1{\@ifnextchar\bgroup  {\mathchoice{\@dprd{#1}}{\@tprd{#1}}{\@tprd{#1}}{\@tprd{#1}}\prd@parens}  {\@ifnextchar\sm    {\mathchoice{\@dprd{#1}}{\@tprd{#1}}{\@tprd{#1}}{\@tprd{#1}}\@eatsm}    {\mathchoice{\@dprd{#1}}{\@tprd{#1}}{\@tprd{#1}}{\@tprd{#1}}}}}
\newcommand{\refl}[1]{\ensuremath{\mathsf{refl}_{#1}}\xspace}
\def\sm#1{\@ifnextchar\bgroup{\sm@parens{#1}}{\@ifnextchar\prd{\sm@parens{#1}\@eatprd}{\sm@noparens{#1}}}}
\def\sm@noparens#1{\mathchoice{\@dsm@noparens{#1}}{\@tsm{#1}}{\@tsm{#1}}{\@tsm{#1}}}
\def\sm@parens#1{\@ifnextchar\bgroup  {\mathchoice{\@dsm{#1}}{\@tsm{#1}}{\@tsm{#1}}{\@tsm{#1}}\sm@parens}  {\@ifnextchar\prd    {\mathchoice{\@dsm{#1}}{\@tsm{#1}}{\@tsm{#1}}{\@tsm{#1}}\@eatprd}    {\mathchoice{\@dsm{#1}}{\@tsm{#1}}{\@tsm{#1}}{\@tsm{#1}}}}}
\def\@tprd#1{\mathchoice{{\textstyle\prod_{(#1)}}}{\prod_{(#1)}}{\prod_{(#1)}}{\prod_{(#1)}}}
\def\@tsm#1{\mathchoice{{\textstyle\sum_{(#1)}}}{\sum_{(#1)}}{\sum_{(#1)}}{\sum_{(#1)}}}
\newcommand{\UU}{\ensuremath{\mathcal{U}}\xspace}
\newcommand{\vcentcolon}{:\!\!}
\newcounter{mathcount}
\setcounter{mathcount}{1}
\newtheorem{prelem}{Lemma}
\newenvironment{lem}{\begin{prelem}}{\end{prelem}\addtocounter{mathcount}{1}}
\renewcommand{\theprelem}{2.2.\arabic{mathcount}}
\let\ap\map
\let\apfunc\mapfunc
\let\autoref\cref
\let\type\UU
\makeatother

\begin{document}
\index{function|(}%
\index{functoriality of functions in type theory@``functoriality'' of functions in type theory}%
Now we wish to establish that functions $f:A\to B$ behave functorially on paths.
In traditional type theory, this is equivalently the statement that functions respect equality.
\index{continuity of functions in type theory@``continuity'' of functions in type theory}%
Topologically, this corresponds to saying that every function is ``continuous'', i.e.\ preserves paths.

\begin{lem}\label{lem:map}
  Suppose that $f:A\to B$ is a function.
  Then for any $x,y:A$ there is an operation
  \begin{equation*}
    \apfunc f : (\id[A] x y) \to (\id[B] {f(x)} {f(y)}).
  \end{equation*}
  Moreover, for each $x:A$ we have $\apfunc{f}(\refl{x})\jdeq \refl{f(x)}$.
  \indexdef{application!of function to a path}%
  \indexdef{path!application of a function to}%
  \indexdef{function!application to a path of}%
  \indexdef{action!of a function on a path}%
\end{lem}

The notation $\apfunc f$ can be read either as the \underline{ap}plication of $f$ to a path, or as the \underline{a}ction on \underline{p}aths of $f$.

\begin{proof}[First proof]
  Let $D:\prd{x,y:A}{p:x=y}\type$ be the type family defined by
  \[D(x,y,p)\defeq (f(x)= f(y)).\]
  Then we have
  \begin{equation*}
    d\defeq\lam{x} \refl{f(x)}:\prd{x:A} D(x,x,\refl{x}).
  \end{equation*}
  By path induction, we obtain $\apfunc f : \prd{x,y:A}{p:x=y}(f(x)=g(x))$.
  The computation rule implies $\apfunc f({\refl{x}})\jdeq\refl{f(x)}$ for each $x:A$.
\end{proof}

\begin{proof}[Second proof]
  By induction, it suffices to assume $p$ is $\refl{x}$.
  In this case, we may define $\apfunc f(p) \defeq \refl{f(x)}:f(x)= f(x)$.
\end{proof}

We will often write $\apfunc f (p)$ as simply $\ap f p$.
This is strictly speaking ambiguous, but generally no confusion arises.
It matches the common convention in category theory of using the same symbol for the application of a functor to objects and to morphisms.

We note that $\apfunc{}$ behaves functorially, in all the ways that one might expect.

\begin{lem}\label{lem:ap-functor}
  For functions $f:A\to B$ and $g:B\to C$ and paths $p:\id[A]xy$ and $q:\id[A]yz$, we have:
  \begin{enumerate}
  \item $\apfunc f(p\ct q) = \apfunc f(p) \ct \apfunc f(q)$.\label{item:apfunctor-ct}
  \item $\apfunc f(\opp p) = \opp{\apfunc f (p)}$.\label{item:apfunctor-opp}
  \item $\apfunc g (\apfunc f(p)) = \apfunc{g\circ f} (p)$.\label{item:apfunctor-compose}
  \item $\apfunc {\idfunc[A]} (p) = p$.
  \end{enumerate}
\end{lem}
\begin{proof}
  Left to the reader.
\end{proof}
\index{function|)}%

\index{function|(}%
\index{functoriality of functions in type theory@``functoriality'' of functions in type theory}%
Now we wish to establish that functions $f:A\to B$ behave functorially on paths.
In traditional type theory, this is equivalently the statement that functions respect equality.
\index{continuity of functions in type theory@``continuity'' of functions in type theory}%
Topologically, this corresponds to saying that every function is ``continuous'', i.e.\ preserves paths.

\begin{lem}\label{lem:map}
  Suppose that $f:A\to B$ is a function.
  Then for any $x,y:A$ there is an operation
  \begin{equation*}
    \apfunc f : (\id[A] x y) \to (\id[B] {f(x)} {f(y)}).
  \end{equation*}
  Moreover, for each $x:A$ we have $\apfunc{f}(\refl{x})\jdeq \refl{f(x)}$.
  \indexdef{application!of function to a path}%
  \indexdef{path!application of a function to}%
  \indexdef{function!application to a path of}%
  \indexdef{action!of a function on a path}%
\end{lem}

The notation $\apfunc f$ can be read either as the \underline{ap}plication of $f$ to a path, or as the \underline{a}ction on \underline{p}aths of $f$.

\begin{proof}[First proof]
  Let $D:\prd{x,y:A}{p:x=y}\type$ be the type family defined by
  \[D(x,y,p)\defeq (f(x)= f(y)).\]
  Then we have
  \begin{equation*}
    d\defeq\lam{x} \refl{f(x)}:\prd{x:A} D(x,x,\refl{x}).
  \end{equation*}
  By path induction, we obtain $\apfunc f : \prd{x,y:A}{p:x=y}(f(x)=g(x))$.
  The computation rule implies $\apfunc f({\refl{x}})\jdeq\refl{f(x)}$ for each $x:A$.
\end{proof}

\begin{proof}[Second proof]
  By induction, it suffices to assume $p$ is $\refl{x}$.
  In this case, we may define $\apfunc f(p) \defeq \refl{f(x)}:f(x)= f(x)$.
\end{proof}

We will often write $\apfunc f (p)$ as simply $\ap f p$.
This is strictly speaking ambiguous, but generally no confusion arises.
It matches the common convention in category theory of using the same symbol for the application of a functor to objects and to morphisms.

We note that $\apfunc{}$ behaves functorially, in all the ways that one might expect.

\begin{lem}\label{lem:ap-functor}
  For functions $f:A\to B$ and $g:B\to C$ and paths $p:\id[A]xy$ and $q:\id[A]yz$, we have:
  \begin{enumerate}
  \item $\apfunc f(p\ct q) = \apfunc f(p) \ct \apfunc f(q)$.\label{item:apfunctor-ct}
  \item $\apfunc f(\opp p) = \opp{\apfunc f (p)}$.\label{item:apfunctor-opp}
  \item $\apfunc g (\apfunc f(p)) = \apfunc{g\circ f} (p)$.\label{item:apfunctor-compose}
  \item $\apfunc {\idfunc[A]} (p) = p$.
  \end{enumerate}
\end{lem}
\begin{proof}
  Left to the reader.
\end{proof}
\index{function|)}%

As was the case for the equalities in \PMlinkname{Lemma 2.1.4}{21typesarehighergroupoids#Thmlem3}, those in \PMlinkname{Lemma 2.2.2}{22functionsarefunctors#Thmlem2} are themselves paths, which satisfy their own coherence laws (which can be proved in the same way), and so on.



\end{document}
