\documentclass[12pt]{article}
\usepackage{pmmeta}
\pmcanonicalname{SchroderBernsteinTheoremProof}
\pmcreated{2013-03-22 16:40:36}
\pmmodified{2013-03-22 16:40:36}
\pmowner{sauravbhaumik}{15615}
\pmmodifier{sauravbhaumik}{15615}
\pmtitle{Schr\"{o}der Bernstein Theorem: Proof}
\pmrecord{10}{38884}
\pmprivacy{1}
\pmauthor{sauravbhaumik}{15615}
\pmtype{Proof}
\pmcomment{trigger rebuild}
\pmclassification{msc}{03E20}

\endmetadata

% this is the default PlanetMath preamble.  as your knowledge
% of TeX increases, you will probably want to edit this, but
% it should be fine as is for beginners.

% almost certainly you want these
\usepackage{amssymb}
\usepackage{amsmath}
\usepackage{amsfonts}

% used for TeXing text within eps files
%\usepackage{psfrag}
% need this for including graphics (\includegraphics)
%\usepackage{graphicx}
% for neatly defining theorems and propositions
%\usepackage{amsthm}
% making logically defined graphics
%%%\usepackage{xypic}

% there are many more packages, add them here as you need them

% define commands here

\begin{document}
Let $A$ and $B$ be two nonempty sets; and let there be, in addition, two one-one functions $f:A\rightarrowtail B$ and $g:B\rightarrowtail A$. We propose to show that $A$ and $B$ are equinumerous i.e., they are in one to one correspondence.

Consider the notation:
\begin{eqnarray*}
 g^{-1}(x),&\mbox{if}& x\in g(B)\\
 f^{-1}(g^{-1}(x)),&\mbox{if}& g^{-1}(x)\in f(A)\\
 g^{-1}(f^{-1}(g^{-1}(x))),&\mbox{if}& f^{-1}(g^{-1}(x))\in g(B)\\
 .....
\end{eqnarray*}
Define, for each $x\in A$, the order of it, denoted by $\circ(x)$, to be the number of such preimage(s) which exist. In a similer way, we'd be able to define the order of an element $y\in B$, i.e., by considering the sequence $f^{-1}(y),g^{-1}(f^{-1}(y)),...$. 

Now define, for each $x\in A$,
\begin{eqnarray*}
\phi(x):&=& f(x),\quad  \circ(x)=\infty \\
&=&f(x),\quad  \circ(x)=2n,\mbox{ for some } n\in\omega\\
&=&b,\quad \circ(x)=2n+1,i.e.,\exists b\in B : g(b)=x\\
\end{eqnarray*}

Notice that if the order is infinite, $\phi(x)=f(x)$ is also infinite. Because, otherwise $x$ would have to have a finite order. On the other hand, if $y\in B$ and $\circ(y)$ is infinite, then $f^{-1}(y)$ exists and has an infinite order; call the latter one $x$. This means, $\phi$ maps the infinite order elements of $A$ bijectively onto the infinite order elements of $B$.

Next, if $\circ(x)=2n$, then the order of $f(x)$ is sheer $2n+1$. Similer to the above para, if for $y\in B$, the order is $2n+1$, as the order is non-zero, $f^{-1}(y)$ exists and it must have order $2n$. To formally show it you need a tedious inductive reasoning!

Last, if $\circ(x)=2n+1$, then the order is $\ge 1$, and so, $g^{-1}(x)$ exists and the order of $g^{-1}(x)$ is sheer $2n$(looking upon $g^{-1}(x)$ as an element of $B$). Conversely, similer to above, if there is an element $y$ of order $2n$ in $B$, take $x=g(y)$ and the order of $x$ is indeed $2n+1$. All that you need to convince a sceptic is a long, tedious, involved induction!!

What we learn from what precedes is that the one-one function $\phi$ maps the infinite order elements onto infinite order elements, odd order onto odd order, and even order onto even order; a simple set theory reveals that $\phi$ is a one-one map from $A$ onto $B$. This completes the proof. 

%%%%%
%%%%%
\end{document}
