\documentclass[12pt]{article}
\usepackage{pmmeta}
\pmcanonicalname{PrenexForm}
\pmcreated{2013-03-22 15:06:55}
\pmmodified{2013-03-22 15:06:55}
\pmowner{rspuzio}{6075}
\pmmodifier{rspuzio}{6075}
\pmtitle{prenex form}
\pmrecord{7}{36852}
\pmprivacy{1}
\pmauthor{rspuzio}{6075}
\pmtype{Definition}
\pmcomment{trigger rebuild}
\pmclassification{msc}{03B10}
\pmclassification{msc}{03C07}

\endmetadata

% this is the default PlanetMath preamble.  as your knowledge
% of TeX increases, you will probably want to edit this, but
% it should be fine as is for beginners.

% almost certainly you want these
\usepackage{amssymb}
\usepackage{amsmath}
\usepackage{amsfonts}

% used for TeXing text within eps files
%\usepackage{psfrag}
% need this for including graphics (\includegraphics)
%\usepackage{graphicx}
% for neatly defining theorems and propositions
%\usepackage{amsthm}
% making logically defined graphics
%%%\usepackage{xypic}

% there are many more packages, add them here as you need them

% define commands here
\begin{document}
A formula in first order logic is said to be in \emph{prenex form} if all quantifiers occur in the front of the formula, before any occurrences of predicates and connectives.  Schematically, a proposition in prenex form will appear as follows

\[ (Q_1 x_1) (Q_1 x_1) \hdots \langle \hbox{matrix} \rangle \]

where each ``$Q$'' stands for either ``$\forall$'' or ``$\exists$'' and  ``$\langle \hbox{matrix} \rangle$'' is constructed from predicates and connectives.  For example, the proposition

\[ (\forall x) (\exists y) (\forall z) \, (x > z \rightarrow y > z) \]

is in prenex form whilst the statement

\[ (\forall x) \, (x > 0 \rightarrow (\exists y) (x = y^2)) \]

is not in prenex form, but is equivalent to the statement

\[ (\forall x) (\exists y) \, (x > 0 \rightarrow x = y^2) \]

which is in prenex form.

This requirement that the statement be expressed in prenex form puts no real restriction on the statements which we may consider because it is possible to systematically express any statement in prenex form by a systematic use of the several equivalences:
%%%%%
%%%%%
\end{document}
