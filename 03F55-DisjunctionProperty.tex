\documentclass[12pt]{article}
\usepackage{pmmeta}
\pmcanonicalname{DisjunctionProperty}
\pmcreated{2013-03-22 19:35:37}
\pmmodified{2013-03-22 19:35:37}
\pmowner{CWoo}{3771}
\pmmodifier{CWoo}{3771}
\pmtitle{disjunction property}
\pmrecord{11}{42583}
\pmprivacy{1}
\pmauthor{CWoo}{3771}
\pmtype{Definition}
\pmcomment{trigger rebuild}
\pmclassification{msc}{03F55}
\pmclassification{msc}{03B55}
\pmclassification{msc}{03B20}
\pmsynonym{DP}{DisjunctionProperty}
\pmrelated{AxiomSystemForIntuitionisticLogic}
\pmrelated{NaturalDeductionForIntuitionisticPropositionalLogic}
\pmrelated{NormalModalLogic}

\endmetadata

\usepackage{amssymb,amscd}
\usepackage{amsmath}
\usepackage{amsfonts}
\usepackage{mathrsfs}
\usepackage{proof}
\usepackage{bussproofs}

% used for TeXing text within eps files
%\usepackage{psfrag}
% need this for including graphics (\includegraphics)
%\usepackage{graphicx}
% for neatly defining theorems and propositions
\usepackage{amsthm}
% making logically defined graphics
%%\usepackage{xypic}
\usepackage{pst-plot}
\usepackage{multicol}
\usepackage{enumerate}
\usepackage{tabls}

% define commands here
\newcommand*{\abs}[1]{\left\lvert #1\right\rvert}
\newtheorem{prop}{Proposition}
\newtheorem{thm}{Theorem}
\newtheorem{lem}{Lemma}
\newtheorem{cor}{Corollary}
\newtheorem{ex}{Example}

\begin{document}
The \emph{disjunction property} (or \emph{DP} for short) is the meta-statement in logic, which says
\begin{center}
if $\vdash A \lor B$, then $\vdash A$ or $\vdash B$.
\end{center}
DP fails for classical propositional logic, but is true for intuitionisitc propositional logic.  In fact, there are infinitely many intermediate logics between classical and intuitionistic logics that satisfy DP.  Furthermore, there are no intermediate logics maximal with respect to satisfying DP.  With respect to predicate logic, DP is ture in first order intuitionistic logic without function symbols.

There is also a modal version of the disjunction property (or \emph{MDP} for short), which states:
\begin{center}
if $\vdash \square A \lor \square B$, then $\vdash A$ or $\vdash B$.
\end{center}
It is not hard to see that MDP holds in normal modal logics K, T, K4, S4, and GL, and fails in D, B, and S5.

\textbf{Remark}.  In predicate logic, there is also a sort of infinitary analog of DP called the existence property (EP), or the witness property, which states:
\begin{center}
if $\vdash \exists x P(x)$, then there is a closed term $t$ such that $\vdash P(t)$.
\end{center}
Like DP, EP fails in classical first-order logic, but true in first-order intuitionistic logic without function symbols and with at least one constant symbol.

%%%%%
%%%%%
\end{document}
