\documentclass[12pt]{article}
\usepackage{pmmeta}
\pmcanonicalname{PartialOrderWithChainConditionDoesNotCollapseCardinals}
\pmcreated{2013-03-22 12:53:40}
\pmmodified{2013-03-22 12:53:40}
\pmowner{mathcam}{2727}
\pmmodifier{mathcam}{2727}
\pmtitle{partial order with chain condition does not collapse cardinals}
\pmrecord{6}{33242}
\pmprivacy{1}
\pmauthor{mathcam}{2727}
\pmtype{Theorem}
\pmcomment{trigger rebuild}
\pmclassification{msc}{03E35}
\pmrelated{PartialOrder}
\pmrelated{ChainCondition}

% this is the default PlanetMath preamble.  as your knowledge
% of TeX increases, you will probably want to edit this, but
% it should be fine as is for beginners.

% almost certainly you want these
\usepackage{amssymb}
\usepackage{amsmath}
\usepackage{amsfonts}

% used for TeXing text within eps files
%\usepackage{psfrag}
% need this for including graphics (\includegraphics)
%\usepackage{graphicx}
% for neatly defining theorems and propositions
%\usepackage{amsthm}
% making logically defined graphics
%%%\usepackage{xypic}

% there are many more packages, add them here as you need them

% define commands here
%\PMlinkescapeword{theory}
\begin{document}
\PMlinkescapeword{limit}

If $P$ is a partial order which satisfies the $\kappa$ chain condition and $G$ is a generic subset of $P$ then for any $\kappa<\lambda\in\mathfrak{M}$, $\lambda$ is also a cardinal in $\mathfrak{M}[G]$, and if $\operatorname{cf}(\alpha)=\lambda$ in $\mathfrak{M}$ then also $\operatorname{cf}(\alpha)=\lambda$ in $\mathfrak{M}[G]$.

This theorem is the simplest way to control a notion of forcing, since it means that a notion of forcing does not have an effect above a certain point.  Given that any $P$ satisfies the $|P|^+$ chain condition, this means that most forcings leaves all of $\mathfrak{M}$ above a certain point alone.  (Although it is possible to get around this limit by forcing with a proper class.)
%%%%%
%%%%%
\end{document}
