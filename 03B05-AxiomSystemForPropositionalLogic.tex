\documentclass[12pt]{article}
\usepackage{pmmeta}
\pmcanonicalname{AxiomSystemForPropositionalLogic}
\pmcreated{2013-03-22 19:31:50}
\pmmodified{2013-03-22 19:31:50}
\pmowner{CWoo}{3771}
\pmmodifier{CWoo}{3771}
\pmtitle{axiom system for propositional logic}
\pmrecord{13}{42507}
\pmprivacy{1}
\pmauthor{CWoo}{3771}
\pmtype{Definition}
\pmcomment{trigger rebuild}
\pmclassification{msc}{03B05}
\pmsynonym{axiom system for classical propositional logic}{AxiomSystemForPropositionalLogic}
\pmrelated{DeductionTheoremHoldsForClassicalPropositionalLogic}
\pmrelated{SubstitutionTheoremForPropositionalLogic}

\usepackage{amssymb,amscd}
\usepackage{amsmath}
\usepackage{amsfonts}
\usepackage{mathrsfs}

% used for TeXing text within eps files
%\usepackage{psfrag}
% need this for including graphics (\includegraphics)
%\usepackage{graphicx}
% for neatly defining theorems and propositions
\usepackage{amsthm}
% making logically defined graphics
%%\usepackage{xypic}
\usepackage{pst-plot}

% define commands here
\newcommand*{\abs}[1]{\left\lvert #1\right\rvert}
\newtheorem{prop}{Proposition}
\newtheorem{thm}{Theorem}
\newtheorem{ex}{Example}
\newcommand{\real}{\mathbb{R}}
\newcommand{\pdiff}[2]{\frac{\partial #1}{\partial #2}}
\newcommand{\mpdiff}[3]{\frac{\partial^#1 #2}{\partial #3^#1}}

\begin{document}
The language of (classical) propositional logic PL$_c$ consists of a set of propositional letters or variables, the symbol $\perp$ (for falsity), together with two symbols for logical connectives $\neg$ and $\to$.  The well-formed formulas (wff's) of PL$_c$ are inductively defined as follows:
\begin{itemize}
\item each propositional letter is a wff
\item $\perp$ is a wff
\item if $A$ and $B$ are wff's, then $A\to B$ is a wff
\end{itemize}
We also use parentheses $($ and $)$ to remove ambiguities.  The other familiar logical connectives may be defined in terms of $\to$: $\neg A$ is $A\to \perp$, $A\lor B$ is the abbreviation for $\neg A \to B$, $A\land B$ is the abbreviation for $\neg (A\to \neg B)$, and $A \leftrightarrow B$ is the abbreviation for $(A\to B)\land (B\to A)$.

The axiom system for PL$_c$ consists of sets of wffs called \emph{axiom schemas} together with a rule of inference.  The axiom schemas are:
\begin{enumerate}
\item $A\to (B\to A)$,
\item $(A\to (B\to C)) \to ((A\to B)\to (A\to C))$,
\item $(\neg A\to \neg B)\to (B\to A)$,
\end{enumerate}
and the rule of inference is modus ponens (MP): from $A\to B$ and $A$, we may infer $B$.

A \emph{deduction} is a finite sequence of wff's $A_1, \ldots, A_n$ such that each $A_i$ is either an instance of one of the axiom schemas above, or as a result of applying rule MP to earlier wff's in the sequence.  In other words, there are $j,k<i$ such that $A_k$ is the wff $A_j \to A_i$.  The last wff $A_n$ in the deduction is called a \emph{theorem} of PL$_c$.  When $A$ is a theorem of PL$_c$, we write $$\vdash_c A \qquad\qquad \mbox{or simply} \qquad\qquad \vdash A.$$
For example, $\vdash A \to A$, whose deduction is
\begin{enumerate}
\item $(A \to ((B\to A) \to A)) \to ((A \to (B\to A)) \to (A \to A))$ by Axiom II,
\item $A \to ((B \to A) \to A)$ by Axiom I,
\item $(A \to (B\to A)) \to (A \to A)$ by modus ponens on $2$ to $1$,
\item $A\to (B\to A)$ by Axiom I,
\item $A\to A$ by modus ponens on $4$ to $3$.
\end{enumerate}

More generally, given a set $\Sigma$ of wff's, we write $$\Sigma \vdash A$$ if there is a finite sequence of wff's such that each wff is either an axiom, a member of $\Sigma$, or as a result of applying MP to earlier wff's in the sequence.  An important (meta-)theorem called the deduction theorem, states: if $\Sigma, A \vdash B$, then $\Sigma \vdash A\to B$.  The deduction theorem holds for PL$_c$ (proof \PMlinkname{here}{deductiontheoremholdsforclassicalpropositionallogic})

\textbf{Remark}.  The axiom system above was first introduced by Polish logician Jan \text{\L}ukasiewicz.  Two axiom systems are said to be \emph{deductively equivalent} if every theorem in one system is also a theorem in the other system.  There are many axiom systems for PL$_c$ that are deductively equivalent to \text{\L}ukasiewicz's system.  One such system consists of the first two axiom schemas above, but the third axiom schema is $\neg \neg A \to A$, with MP its sole inference rule.

%%%%%
%%%%%
\end{document}
