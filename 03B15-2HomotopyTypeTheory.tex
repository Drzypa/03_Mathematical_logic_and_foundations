\documentclass[12pt]{article}
\usepackage{pmmeta}
\pmcanonicalname{2HomotopyTypeTheory}
\pmcreated{2013-11-13 22:13:29}
\pmmodified{2013-11-13 22:13:29}
\pmowner{PMBookProject}{1000683}
\pmmodifier{rspuzio}{6075}
\pmtitle{2. Homotopy Type Theory}
\pmrecord{6}{87597}
\pmprivacy{1}
\pmauthor{PMBookProject}{6075}
\pmtype{Feature}
\pmclassification{msc}{03B15}

\endmetadata

\usepackage{xspace}
\usepackage{amssyb}
\usepackage{amsmath}
\usepackage{amsfonts}
\usepackage{amsthm}
\makeatletter
\newcommand{\blank}{\mathord{\hspace{1pt}\text{--}\hspace{1pt}}}
\newcommand{\ct}{  \mathchoice{\mathbin{\raisebox{0.5ex}{$\displaystyle\centerdot$}}}             {\mathbin{\raisebox{0.5ex}{$\centerdot$}}}             {\mathbin{\raisebox{0.25ex}{$\scriptstyle\,\centerdot\,$}}}             {\mathbin{\raisebox{0.1ex}{$\scriptscriptstyle\,\centerdot\,$}}}}
\newcommand{\define}[1]{\textbf{#1}}
\def\@dprd#1{\prod_{(#1)}\,}
\def\@dprd@noparens#1{\prod_{#1}\,}
\def\@dsm#1{\sum_{(#1)}\,}
\def\@dsm@noparens#1{\sum_{#1}\,}
\def\@eatprd\prd{\prd@parens}
\def\@eatsm\sm{\sm@parens}
\newcommand{\id}[3][]{\ensuremath{#2 =_{#1} #3}\xspace}
\newcommand{\idtype}[3][]{\ensuremath{\mathsf{Id}_{#1}(#2,#3)}\xspace}
\newcommand{\ind}[1]{\mathsf{ind}_{#1}}
\newcommand{\indexdef}[1]{\index{#1|defstyle}}   
\newcommand{\indexsee}[2]{\index{#1|see{#2}}}    
\newcommand{\indid}[1]{\ind{=_{#1}}} 
\newcommand{\jdeq}{\equiv}      
\newcommand{\mentalpause}{\medskip} 
\newcommand{\N}{\ensuremath{\mathbb{N}}\xspace}
\newcommand{\opp}[1]{\mathord{{#1}^{-1}}}
\def\prd#1{\@ifnextchar\bgroup{\prd@parens{#1}}{\@ifnextchar\sm{\prd@parens{#1}\@eatsm}{\prd@noparens{#1}}}}
\def\prd@noparens#1{\mathchoice{\@dprd@noparens{#1}}{\@tprd{#1}}{\@tprd{#1}}{\@tprd{#1}}}
\def\prd@parens#1{\@ifnextchar\bgroup  {\mathchoice{\@dprd{#1}}{\@tprd{#1}}{\@tprd{#1}}{\@tprd{#1}}\prd@parens}  {\@ifnextchar\sm    {\mathchoice{\@dprd{#1}}{\@tprd{#1}}{\@tprd{#1}}{\@tprd{#1}}\@eatsm}    {\mathchoice{\@dprd{#1}}{\@tprd{#1}}{\@tprd{#1}}{\@tprd{#1}}}}}
\newcommand{\R}{\ensuremath{\mathbb{R}}\xspace}           
\newcommand{\refl}[1]{\ensuremath{\mathsf{refl}_{#1}}\xspace}
\def\sm#1{\@ifnextchar\bgroup{\sm@parens{#1}}{\@ifnextchar\prd{\sm@parens{#1}\@eatprd}{\sm@noparens{#1}}}}
\def\sm@noparens#1{\mathchoice{\@dsm@noparens{#1}}{\@tsm{#1}}{\@tsm{#1}}{\@tsm{#1}}}
\def\sm@parens#1{\@ifnextchar\bgroup  {\mathchoice{\@dsm{#1}}{\@tsm{#1}}{\@tsm{#1}}{\@tsm{#1}}\sm@parens}  {\@ifnextchar\prd    {\mathchoice{\@dsm{#1}}{\@tsm{#1}}{\@tsm{#1}}{\@tsm{#1}}\@eatprd}    {\mathchoice{\@dsm{#1}}{\@tsm{#1}}{\@tsm{#1}}{\@tsm{#1}}}}}
\def\@tprd#1{\mathchoice{{\textstyle\prod_{(#1)}}}{\prod_{(#1)}}{\prod_{(#1)}}{\prod_{(#1)}}}
\def\@tsm#1{\mathchoice{{\textstyle\sum_{(#1)}}}{\sum_{(#1)}}{\sum_{(#1)}}{\sum_{(#1)}}}
\newcommand{\UU}{\ensuremath{\mathcal{U}}\xspace}
\newcounter{mathcount}
\setcounter{mathcount}{1}
\newenvironment{myeqn}{\begin{equation}}{\end{equation}\addtocounter{mathcount}{1}}
\renewcommand{\theequation}{2.0.\arabic{mathcount}}
\let\autoref\cref
\let\nat\N
\let\type\UU
\makeatother

\begin{document}
%
The central new idea in homotopy type theory is that types can be regarded as
spaces in homotopy theory, or higher-dimensional groupoids in category
theory.  

\index{classical!homotopy theory|(}
\index{higher category theory|(}
We begin with a brief summary of the connection between homotopy theory
and higher-dimensional category theory.  
In classical homotopy theory, a space $X$ is a set of points equipped
with a topology,
\indexsee{space!topological}{topological space}
\index{topological!space}
and a path between points $x$ and $y$ is represented by
a continuous map $p : [0,1] \to X$, where $p(0) = x$ and $p(1) = y$.
\index{path!topological}
\index{topological!path}
This function can be thought of as giving a point in $X$ at each
``moment in time''.  For many purposes, strict equality of paths
(meaning, pointwise equal functions) is too fine a notion.  For example,
one can define operations of path concatenation (if $p$ is a path from
$x$ to $y$ and $q$ is a path from $y$ to $z$, then the concatenation $p
\ct q$ is a path from $x$ to $z$) and inverses ($\opp p$ is a path
from $y$ to $x$).  However, there are natural equations between these
operations that do not hold for strict equality: for example, the path
$p \ct \opp p$ (which walks from $x$ to $y$, and then back along the
same route, as time goes from $0$ to $1$) is not strictly equal to the
identity path (which stays still at $x$ at all times).

The remedy is to consider a coarser notion of equality of paths called
\emph{homotopy}.
\index{homotopy!topological}
A homotopy between a pair of continuous maps $f :
X_1 \to X_2$ and $g : X_1\to X_2$ is a continuous map $H : X_1
\times [0, 1] \to X_2$ satisfying $H(x, 0) = f (x)$ and $H(x, 1) =
g(x)$.  In the specific case of paths $p$ and $q$ from $x$ to $y$, a homotopy is a
continuous map $H : [0,1] \times [0,1] \rightarrow X$
such that $H(s,0) = p(s)$ and $H(s,1) = q(s)$ for all $s\in [0,1]$.
In this case we require also that $H(0,t) = x$ and $H(1,t)=y$ for all $t\in [0,1]$,
so that for each $t$ the function $H(\blank,t)$ is again a path from $x$ to $y$;
a homotopy of this sort is said to be \emph{endpoint-preserving} or \emph{rel endpoints}.
Such a homotopy is the
image in $X$ of a square that fills in the space between $p$ and $q$,
which can be thought of as a ``continuous deformation'' between $p$ and
$q$, or a 2-dimensional \emph{path between paths}.\index{path!2-}

For example, because
$p \ct \opp p$ walks out and back along the same route, you know that
you can continuously shrink $p \ct \opp p$ down to the identity
path---it won't, for example, get snagged around a hole in the space.
Homotopy is an equivalence relation, and operations such as
concatenation, inverses, etc., respect it.  Moreover, the homotopy
equivalence classes of loops\index{loop} at some point $x_0$ (where two loops $p$
and $q$ are equated when there is a \emph{based} homotopy between them,
which is a homotopy $H$ as above that additionally satisfies $H(0,t) =
H(1,t) = x_0$ for all $t$) form a group called the \emph{fundamental
  group}.\index{fundamental!group}  This group is an \emph{algebraic invariant} of a space, which
can be used to investigate whether two spaces are \emph{homotopy
  equivalent} (there are continuous maps back and forth whose composites
are homotopic to the identity), because equivalent spaces have
isomorphic fundamental groups.

Because homotopies are themselves a kind of 2-dimensional path, there is
a natural notion of 3-dimensional \emph{homotopy between homotopies},\index{path!3-}
and then \emph{homotopy between homotopies between homotopies}, and so
on.  This infinite tower of points, path, homotopies, homotopies between
homotopies, \ldots, equipped with algebraic operations such as the
fundamental group, is an instance of an algebraic structure called a
(weak) \emph{$\infty$-groupoid}.  An $\infty$-groupoid\index{.infinity-groupoid@$\infty$-groupoid} consists of a
collection of objects, and then a collection of \emph{morphisms}\indexdef{morphism!in an .infinity-groupoid@in an $\infty$-groupoid} between
objects, and then \emph{morphisms between morphisms}, and so on,
equipped with some complex algebraic structure; a morphism at level $k$ is called a \define{$k$-morphism}\indexdef{k-morphism@$k$-morphism}.  Morphisms at each level
have identity, composition, and inverse operations, which are weak in
the sense that they satisfy the groupoid laws (associativity of
composition, identity is a unit for composition, inverses cancel) only
up to morphisms at the next level, and this weakness gives rise to
further structure. For example, because associativity of composition of
morphisms $p \ct (q \ct r) = (p \ct q) \ct r$ is itself a
higher-dimensional morphism, one needs an additional operation relating
various proofs of associativity: the various ways to reassociate $p \ct
(q \ct (r \ct s))$ into $((p \ct q) \ct r) \ct s$ give rise to Mac
Lane's pentagon\index{pentagon, Mac Lane}.  Weakness also creates non-trivial interactions between
levels.

Every topological space $X$ has a \emph{fundamental $\infty$-groupoid}
\index{.infinity-groupoid@$\infty$-groupoid!fundamental}
\index{fundamental!.infinity-groupoid@$\infty$-groupoid}
whose
$k$-mor\-ph\-isms are the $k$-dimen\-sional paths in $X$.  The weakness of the
$\infty$-group\-oid corresponds directly to the fact that paths form a
group only up to homotopy, with the $(k+1)$-paths serving as the
homotopies between the $k$-paths.  Moreover, the view of a space as an
$\infty$-groupoid preserves enough aspects of the space to do homotopy theory:
the fundamental $\infty$-groupoid construction is adjoint\index{adjoint!functor} to the
geometric\index{geometric realization} realization of an $\infty$-groupoid as a space, and this
adjunction preserves homotopy theory (this is called the \emph{homotopy
  hypothesis/theorem},
\index{hypothesis!homotopy}
\index{homotopy!hypothesis}
because whether it is a hypothesis or theorem
depends on how you define $\infty$-groupoid).  For example, you can
easily define the fundamental group of an $\infty$-groupoid, and if you
calculate the fundamental group of the fundamental $\infty$-groupoid of
a space, it will agree with the classical definition of fundamental
group of that space.  Because of this correspondence, homotopy theory
and higher-dimensional category theory are intimately related.

\index{classical!homotopy theory|)}%
\index{higher category theory|)}%

\mentalpause

Now, in homotopy type theory each type can be seen to have the structure
of an $\infty$-groupoid.  Recall that for any type $A$, and any $x,y:A$,
we have a identity type $\id[A]{x}{y}$, also written $\idtype[A]{x}{y}$
or just $x=y$.  Logically, we may think of elements of $x=y$ as evidence
that $x$ and $y$ are equal, or as identifications of $x$ with
$y$. Furthermore, type theory (unlike, say, first-order logic) allows us
to consider such elements of $\id[A]{x}{y}$ also as individuals which
may be the subjects of further propositions.  Therefore, we can
\emph{iterate} the identity type: we can form the type
$\id[{(\id[A]{x}{y})}]{p}{q}$ of identifications between
identifications $p,q$, and the type
$\id[{(\id[{(\id[A]{x}{y})}]{p}{q})}]{r}{s}$, and so on.  The structure
of this tower of identity types corresponds precisely to that of the
continuous paths and (higher) homotopies between them in a space, or an
$\infty$-groupoid.\index{.infinity-groupoid@$\infty$-groupoid}


Thus, we will frequently refer to an element $p : \id[A]{x}{y}$ as
a \define{path}
\index{path}
from $x$ to $y$; we call $x$ its \define{start point}
\indexdef{start point of a path}
\indexdef{path!start point of}
and $y$ its \define{end point}.
\indexdef{end point of a path}
\indexdef{path!end point of}
Two paths $p,q : \id[A]{x}{y}$ with the same start and end point are said to be \define{parallel},
\indexdef{parallel paths}
\indexdef{path!parallel}
in which case an element $r : \id[{(\id[A]{x}{y})}]{p}{q}$ can
be thought of as a homotopy, or a morphism between morphisms;
we will often refer to it as a \define{2-path}
\indexdef{path!2-}\indexsee{2-path}{path, 2-}%
or a \define{2-dimensional path}
\index{dimension!of paths}%
\indexsee{2-dimensional path}{path, 2-}\indexsee{path!2-dimensional}{path, 2-}%
Similarly, $\id[{(\id[{(\id[A]{x}{y})}]{p}{q})}]{r}{s}$ is the type of
\define{3-dimensional paths}
\indexdef{path!3-}\indexsee{3-path}{path, 3-}\indexsee{3-dimensional path}{path, 3-}\indexsee{path!3-dimensional}{path, 3-}%
between two parallel 2-dimensional paths, and so on.  If the
type $A$ is ``set-like'', such as \nat, these iterated identity types
will be uninteresting (see \PMlinkname{\S 3.1}{31setsandntypes}), but in the
general case they can model non-trivial homotopy types.

%% (Obviously, the
%% notation ``$\id[A]{x}{y}$'' has its limitations here.  The style
%% $\idtype[A]{x}{y}$ is only slightly better in iterations:
%% $\idtype[{\idtype[{\idtype[A]{x}{y}}]{p}{q}}]{r}{s}$.)

An important difference between homotopy type theory and classical homotopy theory is that homotopy type theory provides a \emph{synthetic}
\index{synthetic mathematics}%
\index{geometry, synthetic}%
\index{Euclid of Alexandria}%
description of spaces, in the following sense. Synthetic geometry is geometry in the style of Euclid~\cite{Euclid}: one starts from some basic notions (points and lines), constructions (a line connecting any two points), and axioms
(all right angles are equal), and deduces consequences logically.  This is in contrast with analytic
\index{analytic mathematics}%
geometry, where notions such as points and lines are represented concretely using cartesian coordinates in $\R^n$---lines are sets of points---and the basic constructions and axioms are derived from this representation.  While classical homotopy theory is analytic (spaces and paths are made of points), homotopy type theory is synthetic: points, paths, and paths between paths are basic, indivisible, primitive notions.

Moreover, one of the amazing things about homotopy type theory is that all of the basic constructions and axioms---all of the
higher groupoid structure----arises automatically from the induction
principle for identity types.
Recall from \PMlinkname{\S 1.12}{112identitytypes} that this says that if
\begin{itemize}
\item for every $x,y:A$ and every $p:\id[A]xy$ we have a type $D(x,y,p)$, and
\item for every $a:A$ we have an element $d(a):D(a,a,\refl a)$, 
\end{itemize}
then
\begin{itemize}
\item there exists an element $\indid{A}(D,d,x,y,p):D(x,y,p)$ for \emph{every} two elements $x,y:A$ and $p:\id[A]xy$, such that $\indid{A}(D,d,a,a,\refl a) \jdeq d(a)$.
\end{itemize}
In other words, given dependent functions
\begin{align*}
D & :\prd{x,y:A}{p:\id{x}{y}} \type\\
d & :\prd{a:A} D(a,a,\refl{a})
\end{align*}
there is a dependent function
\[\indid{A}(D,d):\prd{x,y:A}{p:\id{x}{y}} D(x,y,p)\]
such that 
\begin{myeqn}\label{eq:Jconv}
\indid{A}(D,d,a,a,\refl{a})\jdeq d(a)
\end{myeqn}
for every $a:A$.
Usually, every time we apply this induction rule we will either not care about the specific function being defined, or we will immediately give it a different name.

Informally, the induction principle for identity types says that if we want to construct an object (or prove a statement) which depends on an inhabitant $p:\id[A]xy$ of an identity type, then it suffices to perform the construction (or the proof) in the special case when $x$ and $y$ are the same (judgmentally) and $p$ is the reflexivity element $\refl{x}:x=x$ (judgmentally).
When writing informally, we may express this with a phrase such as ``by induction, it suffices to assume\dots''.
This reduction to the ``reflexivity case'' is analogous to the reduction to the ``base case'' and ``inductive step'' in an ordinary proof by induction on the natural numbers, and also to the ``left case'' and ``right case'' in a proof by case analysis on a disjoint union or disjunction.\index{induction principle!for identity type}%


The ``conversion rule''~\eqref{eq:Jconv} is less familiar in the context of proof by induction on natural numbers, but there is an analogous notion in the related concept of definition by recursion.
If a sequence\index{sequence} $(a_n)_{n\in \mathbb{N}}$ is defined by giving $a_0$ and specifying $a_{n+1}$ in terms of $a_n$, then in fact the $0^{\mathrm{th}}$ term of the resulting sequence \emph{is} the given one, and the given recurrence relation relating $a_{n+1}$ to $a_n$ holds for the resulting sequence.
(This may seem so obvious as to not be worth saying, but if we view a definition by recursion as an algorithm\index{algorithm} for calculating values of a sequence, then it is precisely the process of executing that algorithm.)
The rule~\eqref{eq:Jconv} is analogous: it says that if we define an object $f(p)$ for all $p:x=y$ by specifying what the value should be when $p$ is $\refl{x}:x=x$, then the value we specified is in fact the value of $f(\refl{x})$.

This induction principle endows each type with the structure of an $\infty$-groupoid\index{.infinity-groupoid@$\infty$-groupoid}, and each function between two types the structure of an $\infty$-functor\index{.infinity-functor@$\infty$-functor} between two such groupoids.  This is interesting from a mathematical point view, because it gives a new way to work with
$\infty$-groupoids.  It is interesting from a type-theoretic point view, because it reveals new operations that are associated with each type and function.  In the remainder of this chapter, we begin to explore this structure.

\begin{thebibliography}{99}

\bibitem{Euclid} {Euclid}, \emph{{Elements,{V}ols.\ 1--13}}  {Elsevier},{300 {BC}}

\end{thebibliography}

\end{document}
