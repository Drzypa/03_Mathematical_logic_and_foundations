\documentclass[12pt]{article}
\usepackage{pmmeta}
\pmcanonicalname{Biconditional}
\pmcreated{2013-03-22 11:53:06}
\pmmodified{2013-03-22 11:53:06}
\pmowner{Mathprof}{13753}
\pmmodifier{Mathprof}{13753}
\pmtitle{biconditional}
\pmrecord{17}{30484}
\pmprivacy{1}
\pmauthor{Mathprof}{13753}
\pmtype{Definition}
\pmcomment{trigger rebuild}
\pmclassification{msc}{03-00}
\pmsynonym{iff}{Biconditional}
\pmrelated{PropositionalLogic}
\pmrelated{Equivalent3}

\usepackage{amssymb}
\usepackage{amsmath}
\usepackage{amsfonts}
\usepackage{graphicx}
%%%%\usepackage{xypic}
\begin{document}
\section{Biconditional}

A \emph{biconditional} is a truth function that is true only in the case that both parameters are true or both are false.  

Symbolically, the biconditional is written as

$$ a \Leftrightarrow b$$
or 
$$ a \leftrightarrow b$$ 

with the latter being rare outside of formal logic.  The truth table for the biconditional is

\begin{center}
\begin{tabular}{ccc}
a & b & $a \Leftrightarrow b$ \\ 
\hline 
F & F & T \\
F & T & F \\
T & F & F \\ 
T & T & T 
\end{tabular}
\end{center}

The biconditional function is often written as ``iff,'' meaning ``if and only if.''  

It  gets its name from the fact that it is really two conditionals in conjunction, 

$$ (a \rightarrow b) \land (b \rightarrow a) $$

This fact is important to recognize when writing a mathematical proof, as both conditionals must be proven independently.

\section{Colloquial Usage}

The only unambiguous way of stating a biconditional in plain English is of the form ``$b$ if $a$ and $a$ if $b$.''  Slightly more formal, one would say ``$b$ implies $a$ and $a$ implies $b$.''  The plain English ``if'' may sometimes be used as a biconditional.  One must weigh context heavily.

For example, ``I'll buy you an ice cream if you pass the exam'' is meant as a biconditional, since the speaker doesn't intend a valid outcome to be buying the ice cream whether or not you pass the exam (as in a conditional).   However, ``it is cloudy if it is raining'' is \emph{not} meant as a biconditional, since it can obviously be cloudy while not raining.
%%%%%
%%%%%
%%%%%
%%%%%
\end{document}
