\documentclass[12pt]{article}
\usepackage{pmmeta}
\pmcanonicalname{OrdinalArithmetic}
\pmcreated{2013-03-22 13:28:52}
\pmmodified{2013-03-22 13:28:52}
\pmowner{Henry}{455}
\pmmodifier{Henry}{455}
\pmtitle{ordinal arithmetic}
\pmrecord{7}{34052}
\pmprivacy{1}
\pmauthor{Henry}{455}
\pmtype{Topic}
\pmcomment{trigger rebuild}
\pmclassification{msc}{03E10}
\pmrelated{AdditivelyIndecomposable}
\pmrelated{CardinalArithmetic}

\endmetadata

% this is the default PlanetMath preamble.  as your knowledge
% of TeX increases, you will probably want to edit this, but
% it should be fine as is for beginners.

% almost certainly you want these
\usepackage{amssymb}
\usepackage{amsmath}
\usepackage{amsfonts}

% used for TeXing text within eps files
%\usepackage{psfrag}
% need this for including graphics (\includegraphics)
%\usepackage{graphicx}
% for neatly defining theorems and propositions
%\usepackage{amsthm}
% making logically defined graphics
%%%\usepackage{xypic}

% there are many more packages, add them here as you need them

% define commands here
%\PMlinkescapeword{theory}
\begin{document}
Ordinal arithmetic is the extension of normal arithmetic to the transfinite ordinal numbers.  The successor operation $Sx$ (sometimes written $x+1$, although this notation risks confusion with the general definition of addition) is part of the definition of the ordinals, and addition is naturally defined by recursion over this:

\begin{itemize}
\item $x+0=x$
\item $x+Sy=S(x+y)$
\item $x+\alpha=\operatorname{sup}_{\gamma<\alpha} (x+\gamma)$ for limit ordinal $\alpha$
\end{itemize}

If $x$ and $y$ are finite then $x+y$ under this definition is just the usual sum, however when $x$ and $y$ become infinite, there are differences.  In particular, ordinal addition is not commutative.  For example, 
$$\omega+1=\omega+S0=S(\omega+0)=S\omega$$
but
$$1+\omega=\operatorname{sup}_{n<\omega} 1+n=\omega$$

Multiplication in turn is defined by iterated addition:

\begin{itemize}
\item $x\cdot 0=0$
\item $x\cdot Sy=x\cdot y+x$
\item $x\cdot \alpha=\operatorname{sup}_{\gamma<\alpha} (x\cdot \gamma)$ for limit ordinal $\alpha$
\end{itemize}

Once again this definition is equivalent to normal multiplication when $x$ and $y$ are finite, but is not commutative:
$$\omega\cdot 2=\omega\cdot 1+\omega=\omega+\omega$$
but
$$2\cdot\omega=\operatorname{sup}_{n<\omega} 2\cdot n=\omega$$

Both these functions are strongly increasing in the second argument and weakly increasing in the first argument.  That is, if $\alpha<\beta$ then 
\begin{itemize}
\item $\gamma+\alpha<\gamma+\beta$
\item $\gamma\cdot\alpha<\gamma\cdot\beta$
\item $\alpha+\gamma\leq\beta+\gamma$
\item $\alpha\cdot\gamma\leq\beta\cdot\gamma$
\end{itemize}
%%%%%
%%%%%
\end{document}
