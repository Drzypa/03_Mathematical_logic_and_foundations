\documentclass[12pt]{article}
\usepackage{pmmeta}
\pmcanonicalname{StronglyMinimal}
\pmcreated{2013-03-22 13:27:13}
\pmmodified{2013-03-22 13:27:13}
\pmowner{Timmy}{1414}
\pmmodifier{Timmy}{1414}
\pmtitle{strongly minimal}
\pmrecord{5}{34019}
\pmprivacy{1}
\pmauthor{Timmy}{1414}
\pmtype{Definition}
\pmcomment{trigger rebuild}
\pmclassification{msc}{03C07}
\pmclassification{msc}{03C10}
\pmclassification{msc}{03C45}
%\pmkeywords{minimal}
\pmrelated{OMinimality}
\pmdefines{strongly minimal}
\pmdefines{minimal}

\endmetadata

% this is the default PlanetMath preamble.  as your knowledge
% of TeX increases, you will probably want to edit this, but
% it should be fine as is for beginners.

% almost certainly you want these
\usepackage{amssymb}
\usepackage{amsmath}
\usepackage{amsfonts}

% used for TeXing text within eps files
%\usepackage{psfrag}
% need this for including graphics (\includegraphics)
%\usepackage{graphicx}
% for neatly defining theorems and propositions
%\usepackage{amsthm}
% making logically defined graphics
%%%\usepackage{xypic}

% there are many more packages, add them here as you need them

% define commands here
\begin{document}
Let $L$ be a first order language and let $M$ be an $L$-structure. Let $S$, a subset of the domain of $M$ be a definable infinite set. Then $S$ is {\em minimal} iff every definable $C \subseteq S$ we have either $C$ is finite or $S \setminus C$ is finite. We say that $M$ is {\em minimal} iff the domain of $M$ is a strongly minimal set.

\medskip

We say that $M$ is {\em strongly minimal} iff for every $N \equiv M$, we have that $N$ is minimal. Thus if $T$ is a complete $L$ theory then we say $T$ is {\em strongly minimal} if it has some model (equivalently all models) which is strongly minimal.

\medskip

Note that $M$ is strongly minimal iff every definable subset of $M$ is quantifier free definable in a language with just equality. Compare this to the notion of o-minimal structures.
%%%%%
%%%%%
\end{document}
