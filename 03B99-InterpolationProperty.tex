\documentclass[12pt]{article}
\usepackage{pmmeta}
\pmcanonicalname{InterpolationProperty}
\pmcreated{2013-03-22 13:49:36}
\pmmodified{2013-03-22 13:49:36}
\pmowner{CWoo}{3771}
\pmmodifier{CWoo}{3771}
\pmtitle{interpolation property}
\pmrecord{7}{34558}
\pmprivacy{1}
\pmauthor{CWoo}{3771}
\pmtype{Definition}
\pmcomment{trigger rebuild}
\pmclassification{msc}{03B99}
\pmdefines{interpolation property}

% this is the default PlanetMath preamble.  as your knowledge
% of TeX increases, you will probably want to edit this, but
% it should be fine as is for beginners.

% almost certainly you want these
\usepackage{amssymb}
\usepackage{amsmath}
\usepackage{amsfonts}

% used for TeXing text within eps files
%\usepackage{psfrag}
% need this for including graphics (\includegraphics)
%\usepackage{graphicx}
% for neatly defining theorems and propositions
%\usepackage{amsthm}
% making logically defined graphics
%%%\usepackage{xypic}

% there are many more packages, add them here as you need them

% define commands here
\begin{document}
A logic is said to have the \emph{interpolation property} if whenever $\phi(R,S) \rightarrow \psi(R,T)$ holds, then there is a sentence $\theta(R)$, so that both $\phi(R,S) \rightarrow \theta(R)$ and $\theta(R) \rightarrow \psi(R,T)$ hold, where $R,S$ and $T$ are some sets of symbols that occur in the formulas, $R$ being the set of symbols common to both $\phi$ and $\psi$.

The interpolation property holds for first order logic. The interpolation property is related to Beth definability property and Robinson's consistency property. Also, a natural generalisation is the concept $\Delta$-closed logic.
%%%%%
%%%%%
\end{document}
