\documentclass[12pt]{article}
\usepackage{pmmeta}
\pmcanonicalname{Transposition}
\pmcreated{2013-03-22 12:24:30}
\pmmodified{2013-03-22 12:24:30}
\pmowner{drini}{3}
\pmmodifier{drini}{3}
\pmtitle{transposition}
\pmrecord{6}{32274}
\pmprivacy{1}
\pmauthor{drini}{3}
\pmtype{Definition}
\pmcomment{trigger rebuild}
\pmclassification{msc}{03-00}
\pmclassification{msc}{05A05}
\pmclassification{msc}{20B99}
\pmrelated{Cycle2}
\pmrelated{SignatureOfAPermutation}

%\usepackage{graphicx}
%%%\usepackage{xypic} 
\usepackage{bbm}
\newcommand{\Z}{\mathbbmss{Z}}
\newcommand{\C}{\mathbbmss{C}}
\newcommand{\R}{\mathbbmss{R}}
\newcommand{\Q}{\mathbbmss{Q}}
\newcommand{\mathbb}[1]{\mathbbmss{#1}}
\begin{document}
Given a finite set $X=\{a_1,a_2,\ldots,a_n\}$, a transposition is a permutation (bijective function of $X$ onto itself) $f$ such that there exist indices $i,j$ such that
$f(a_i)=a_j$, $f(a_j)=a_i$ and $f(a_k)=a_k$ for all other indices $k$. This is often denoted (in the cycle notation) as $(a, b)$.

Example:
If $X=\{a,b,c,d,e\}$ the function $\sigma$ given by
\begin{eqnarray*}
\sigma(a)&=&a\\
\sigma(b)&=&e\\
\sigma(c)&=&c\\
\sigma(d)&=&d\\
\sigma(e)&=&b
\end{eqnarray*}
is a transposition.

One of the main results on symmetric groups states that any permutation can be expressed as composition (product) of transpositions, and for any two decompositions of a given permutation, the number of transpositions is always even or always odd.
%%%%%
%%%%%
\end{document}
