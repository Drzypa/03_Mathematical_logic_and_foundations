\documentclass[12pt]{article}
\usepackage{pmmeta}
\pmcanonicalname{ArithmeticalHierarchy}
\pmcreated{2013-03-22 12:55:11}
\pmmodified{2013-03-22 12:55:11}
\pmowner{CWoo}{3771}
\pmmodifier{CWoo}{3771}
\pmtitle{arithmetical hierarchy}
\pmrecord{19}{33272}
\pmprivacy{1}
\pmauthor{CWoo}{3771}
\pmtype{Definition}
\pmcomment{trigger rebuild}
\pmclassification{msc}{03B10}
\pmsynonym{arithmetic hierarchy}{ArithmeticalHierarchy}
\pmsynonym{arithmetic}{ArithmeticalHierarchy}
\pmsynonym{arithmetical}{ArithmeticalHierarchy}
\pmsynonym{arithmetic formula}{ArithmeticalHierarchy}
\pmsynonym{arithmetical formulas}{ArithmeticalHierarchy}
\pmrelated{AnalyticHierarchy}
\pmdefines{sigma n}
\pmdefines{sigma-n}
\pmdefines{pi n}
\pmdefines{pi-n}
\pmdefines{delta n}
\pmdefines{delta-n}
\pmdefines{recursive}
\pmdefines{recursively enumerable}
\pmdefines{delta-0}
\pmdefines{delta 0}
\pmdefines{delta-1}
\pmdefines{delta 1}
\pmdefines{arithmetical}

\endmetadata

% this is the default PlanetMath preamble.  as your knowledge
% of TeX increases, you will probably want to edit this, but
% it should be fine as is for beginners.

% almost certainly you want these
\usepackage{amssymb}
\usepackage{amsmath}
\usepackage{amsfonts}

% used for TeXing text within eps files
%\usepackage{psfrag}
% need this for including graphics (\includegraphics)
%\usepackage{graphicx}
% for neatly defining theorems and propositions
%\usepackage{amsthm}
% making logically defined graphics
%%%\usepackage{xypic}

% there are many more packages, add them here as you need them

% define commands here
%\PMlinkescapeword{theory}
\begin{document}
The \emph{arithmetical hierarchy} is a hierarchy of either (depending on the context) formulas or relations.  The relations of a particular level of the hierarchy are exactly the relations defined by the formulas of that level, so the two uses are essentially the same.

The first level consists of formulas with only bounded quantifiers, the corresponding relations are also called the Primitive Recursive relations (this definition is equivalent to the definition from computer science).  This level is called any of $\Delta^0_0$, $\Sigma^0_0$ and $\Pi^0_0$, depending on context.

A formula $\phi$ is $\Sigma^0_n$ if there is some $\Delta^0_0$ formula $\psi$ such that $\phi$ can be written:

$$\phi(\vec k)=\exists x_1\forall x_2\cdots {Q}x_n\psi(\vec k,\vec x)$$
$$\text{ where }Q\text{ is either }\forall\text{ or }\exists\text{, whichever maintains the pattern of alternating quantifiers}
$$

The $\Sigma^0_1$ relations are the same as the Recursively Enumerable relations.

Similarly, $\phi$ is a $\Pi^0_n$ relation if there is some $\Delta^0_0$ formula $\psi$ such that:
$$\phi(\vec k)=\forall x_1\exists x_2\cdots Q x_n\psi(\vec k,\vec x)$$
$$\text{ where }Q\text{ is either }\forall\text{ or }\exists\text{, whichever maintains the pattern of alternating quantifiers}
$$

A formula is $\Delta^0_n$ if it is both $\Sigma^0_n$ and $\Pi^0_n$.  Since each $\Sigma^0_n$ formula is just the negation of a $\Pi^0_n$ formula and vice-versa, the $\Sigma^0_n$ relations are the complements of the $\Pi^0_n$ relations.

The relations in $\Delta^0_1 = \Sigma^0_1 \cap \Pi^0_1$ are the Recursive relations.

Higher levels on the hierarchy correspond to broader and broader classes of relations.  A formula or relation which is $\Sigma^0_n$ (or, equivalently, $\Pi^0_n$) for some integer $n$ is called \emph{arithmetical}.

The superscript $0$ is often omitted when it is not necessary to distinguish from the analytic hierarchy.

Functions can be described as being in one of the levels of the hierarchy if the graph of the function is in that level.
%%%%%
%%%%%
\end{document}
