\documentclass[12pt]{article}
\usepackage{pmmeta}
\pmcanonicalname{WeaklyCompactCardinal}
\pmcreated{2013-03-22 12:50:53}
\pmmodified{2013-03-22 12:50:53}
\pmowner{Henry}{455}
\pmmodifier{Henry}{455}
\pmtitle{weakly compact cardinal}
\pmrecord{5}{33177}
\pmprivacy{1}
\pmauthor{Henry}{455}
\pmtype{Definition}
\pmcomment{trigger rebuild}
\pmclassification{msc}{03E10}
\pmsynonym{weakly compact}{WeaklyCompactCardinal}
%\pmkeywords{compactness}
%\pmkeywords{tree property}
%\pmkeywords{compact}
%\pmkeywords{inaccessible cardinal}
%\pmkeywords{infinite cardinal}
\pmrelated{CardinalNumber}
\pmdefines{weakly compact cardinal}
\pmdefines{weak compactness theorem}

\endmetadata

% this is the default PlanetMath preamble.  as your knowledge
% of TeX increases, you will probably want to edit this, but
% it should be fine as is for beginners.

% almost certainly you want these
\usepackage{amssymb}
\usepackage{amsmath}
\usepackage{amsfonts}

% used for TeXing text within eps files
%\usepackage{psfrag}
% need this for including graphics (\includegraphics)
%\usepackage{graphicx}
% for neatly defining theorems and propositions
%\usepackage{amsthm}
% making logically defined graphics
%%%\usepackage{xypic}

% there are many more packages, add them here as you need them

% define commands here
\begin{document}
Weakly compact cardinals are (large) infinite cardinals which have a property related to the syntactic compactness theorem for first order logic.  Specifically, for any infinite cardinal $\kappa$, consider the language $L_{\kappa,\kappa}$.  

This language is identical to \PMlinkescapeword{normal} first \PMlinkescapeword{order} logic except that:

\begin{itemize}
\item infinite conjunctions and disjunctions of fewer than $\kappa$ formulas are allowed
\item infinite strings of fewer than $\kappa$ quantifiers are allowed
\end{itemize}

The weak compactness theorem for $L_{\kappa,\kappa}$ states that if $\Delta$ is a set of sentences of $L_{\kappa,\kappa}$ such that $|\Delta|=\kappa$ and any $\theta\subset\Delta$ with $|\theta|<\kappa$ is consistent then $\Delta$ is consistent.

A cardinal is weakly compact if the weak compactness theorem holds for $L_{\kappa,\kappa}$.
%%%%%
%%%%%
\end{document}
