\documentclass[12pt]{article}
\usepackage{pmmeta}
\pmcanonicalname{Mapping}
\pmcreated{2013-03-22 12:20:16}
\pmmodified{2013-03-22 12:20:16}
\pmowner{rmilson}{146}
\pmmodifier{rmilson}{146}
\pmtitle{mapping}
\pmrecord{18}{31975}
\pmprivacy{1}
\pmauthor{rmilson}{146}
\pmtype{Definition}
\pmcomment{trigger rebuild}
\pmclassification{msc}{03E20}
\pmsynonym{map}{Mapping}
\pmrelated{DirectImage}
\pmrelated{InverseImage}
\pmrelated{Domain}
\pmrelated{Codomain}
\pmrelated{Set}
\pmrelated{Function}
\pmrelated{Transformation}

\endmetadata

\usepackage{amsmath}
\usepackage{amsfonts}
\usepackage{amssymb}


\newtheorem{proposition}{Proposition}
\begin{document}
\PMlinkescapeword{generic}
\PMlinkescapeword{term}
\PMlinkescapeword{field}
The term \emph{mapping} is a synonym of \PMlinkname{function}{Function}, although usage patterns suggest that  ``mapping'' is the more generic term.

In a geometric context, the term ``function'' often connotes a mapping whose purpose is to assign values to the elements of its domain. In other words, a function defines a field of values.  By contrast, ``mapping'' has a more geometric connotation, as in ``a mapping of one space to another''.
%%%%%
%%%%%
\end{document}
