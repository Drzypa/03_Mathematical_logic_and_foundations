\documentclass[12pt]{article}
\usepackage{pmmeta}
\pmcanonicalname{Theory}
\pmcreated{2013-03-22 13:00:12}
\pmmodified{2013-03-22 13:00:12}
\pmowner{CWoo}{3771}
\pmmodifier{CWoo}{3771}
\pmtitle{theory}
\pmrecord{8}{33383}
\pmprivacy{1}
\pmauthor{CWoo}{3771}
\pmtype{Definition}
\pmcomment{trigger rebuild}
\pmclassification{msc}{03B10}
\pmclassification{msc}{03B05}

\endmetadata

% this is the default PlanetMath preamble.  as your knowledge
% of TeX increases, you will probably want to edit this, but
% it should be fine as is for beginners.

% almost certainly you want these
\usepackage{amssymb}
\usepackage{amsmath}
\usepackage{amsfonts}

% used for TeXing text within eps files
%\usepackage{psfrag}
% need this for including graphics (\includegraphics)
%\usepackage{graphicx}
% for neatly defining theorems and propositions
%\usepackage{amsthm}
% making logically defined graphics
%%%\usepackage{xypic}

% there are many more packages, add them here as you need them

% define commands here
%\PMlinkescapeword{theory}
\begin{document}
If $L$ is a logical language for some logic $\mathcal{L}$, a set $T$ of formulas with no free variables is called a \emph{theory} (of $\mathcal{L}$).  If $\mathcal{L}$ is a first-order logic, then $T$ is called a \emph{first-order theory}.

We write $T\vDash \phi$ for any formula $\phi$ if every model $\mathcal{M}$ of $\mathcal{L}$ such that $M\vDash T$, $M\vDash\phi$.

We write $T\vdash\phi$ is for there is a proof of $\phi$ from $T$.

\textbf{Remark}.  Let $S$ be an $L$-structure for some signature $L$. The \emph{theory of $S$} is the set of formulas satisfied by $S$: $$\lbrace \varphi \mid S\models \varphi \rbrace,$$ and is denoted by $\operatorname{Th}(S)$.
%%%%%
%%%%%
\end{document}
