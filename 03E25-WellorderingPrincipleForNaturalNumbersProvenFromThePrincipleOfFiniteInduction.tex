\documentclass[12pt]{article}
\usepackage{pmmeta}
\pmcanonicalname{WellorderingPrincipleForNaturalNumbersProvenFromThePrincipleOfFiniteInduction}
\pmcreated{2013-03-22 16:38:02}
\pmmodified{2013-03-22 16:38:02}
\pmowner{CWoo}{3771}
\pmmodifier{CWoo}{3771}
\pmtitle{well-ordering principle for natural numbers proven from the principle of finite induction}
\pmrecord{5}{38835}
\pmprivacy{1}
\pmauthor{CWoo}{3771}
\pmtype{Proof}
\pmcomment{trigger rebuild}
\pmclassification{msc}{03E25}
\pmrelated{NaturalNumbersAreWellOrdered}

\endmetadata

% this is the default PlanetMath preamble.  as your knowledge
% of TeX increases, you will probably want to edit this, but
% it should be fine as is for beginners.

% almost certainly you want these
\usepackage{amssymb}
\usepackage{amsmath}
\usepackage{amsfonts}

% used for TeXing text within eps files
%\usepackage{psfrag}
% need this for including graphics (\includegraphics)
%\usepackage{graphicx}
% for neatly defining theorems and propositions
%\usepackage{amsthm}
% making logically defined graphics
%%%\usepackage{xypic}

% there are many more packages, add them here as you need them

% define commands here

\begin{document}
\PMlinkescapeword{fix}
Let $S$ be a nonempty set of natural numbers. We show that there is an $a\in S$ such that for all $b\in S$, $a\leq b$. Suppose not, then
$$(*)\ \ \ \ \ \forall a\in S,\exists b\in S\ \ b<a.$$
We will use the principle of finite induction (the strong form) to  show that $S$ is empty, a contradition.

Fix any natural number $n$, and suppose that for all natural numbers $m<n$, $m\in \mathbb{N}\setminus S$. If $n\in S$, then (*) implies that there is an element $b\in S$ such that $b< n$. This would be incompatible with the assumption that for all natural numbers $m<n$, $m\in \mathbb{N}\setminus S$.
Hence, we conclude that $n$ is not in $S$.

Therefore, by induction, no natural number is a member of $S$. The set is empty.
%%%%%
%%%%%
\end{document}
