\documentclass[12pt]{article}
\usepackage{pmmeta}
\pmcanonicalname{Function}
\pmcreated{2013-03-22 11:48:58}
\pmmodified{2013-03-22 11:48:58}
\pmowner{djao}{24}
\pmmodifier{djao}{24}
\pmtitle{function}
\pmrecord{23}{30360}
\pmprivacy{1}
\pmauthor{djao}{24}
\pmtype{Definition}
\pmcomment{trigger rebuild}
\pmclassification{msc}{03E20}
\pmclassification{msc}{44A20}
\pmclassification{msc}{33E20}
\pmclassification{msc}{30D15}
\pmsynonym{map}{Function}
\pmrelated{Mapping}
\pmrelated{InjectiveFunction}
\pmrelated{Surjective}
\pmrelated{Bijection}
\pmrelated{Relation}
\pmdefines{domain}
\pmdefines{codomain}
\pmdefines{composition}
\pmdefines{image}
\pmdefines{range}
\pmdefines{composite function}

\endmetadata

\usepackage{amssymb}
\usepackage{amsmath}
\usepackage{amsfonts}
\usepackage{graphicx}
%%%%\usepackage{xypic}
\begin{document}
A \emph{function} is a triplet $(f,A,B)$ where:
\begin{enumerate}
\item $A$ is a set (called the \emph{domain} of the function).
\item $B$ is a set (called the \emph{codomain} of the function).
\item $f$ is a binary relation between $A$ and $B$.
\item For every $a \in A$, there exists $b \in B$ such that $(a,b) \in f$.
\item If $a \in A$, $b_1,b_2 \in B$, and $(a,b_1) \in f$ and $(a,b_2) \in f$, then $b_1 = b_2$.
\end{enumerate}
The triplet $(f,A,B)$ is usually written with the specialized notation $f\colon A \to B$. This notation visually conveys the fact that $f$ maps elements of $A$ into elements of $B$.

Other standard notations for functions are as follows:
\begin{itemize}
\item For $a \in A$, one denotes by $f(a)$ the unique element $b \in B$ such that $(a,b) \in f$.
\item The \emph{image} of $(f,A,B)$, denoted $f(A)$, is the set
$$
\{b \in B \mid f(a) = b \text{ for some } a \in A\}
$$
consisting of all elements of $B$ which equal $f(a)$ for some element $a \in A$. Note that, by abuse of notation, the set $f(A)$ is almost always called the image of $f$, rather than the image of $(f,A,B)$.
\item In cases where the function $f$ is clear from context, the notation $a \mapsto b$ is equivalent to the statement $f(a) = b$.
\item Given two functions $f\colon A \to B$ and $g\colon B \to C$, there exists a unique function $g \circ f\colon A \to C$ satisfying the equation $g \circ f(a) = g(f(a))$. The function $g \circ f$ is called the \emph{composition} of $f$ and $g$, and a function constructed in this manner is called a \emph{composite function}. Composition is associative, meaning that $h \circ (g \circ f) = (h \circ g) \circ f$ provided that either expression is defined.
\item When a function $f\colon A \to A$ has its domain equal to its codomain, one often writes $f^n$ for the $n$-fold composition
$$
\underbrace{f \circ f \circ \cdots \circ f}_{n\text{ times}}
$$
where $n$ is any natural number.  Occasionally this can be confused with ordinary exponentiation (for example the function $x\mapsto (\sin x)(\sin x)$ is conventionally written as $\sin^2$); in such cases one usually writes $f^{[n]}$ to denote the $n$-fold composition.
\end{itemize}
There is no universal agreement as to the definition of the \emph{range} of a function. Some authors define the range of a function to be equal to the codomain, and others define the range of a function to be equal to the image.

\textbf{Remark}.  In set theory, a function is defined as a relation $f$, such that whenever $(a,b),(a,c)\in f$, then $b=c$.  Notice that the sets $A,B$ are not specified in advance, unlike the defintion given in the beginning of the article.  The \emph{domain} and \emph{range} of the function $f$ is the domain and range of $f$ as a relation.  Using this definition of a function, we may recapture the defintion at the top of the entry by saying that a function $f$ \emph{maps from a set $A$ into a set $B$}, if the domain of $f$ is $A$, and the range of $f$ is a subset of $B$.
%%%%%
%%%%%
%%%%%
%%%%%
\end{document}
