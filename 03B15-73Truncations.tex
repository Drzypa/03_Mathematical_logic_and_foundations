\documentclass[12pt]{article}
\usepackage{pmmeta}
\pmcanonicalname{73Truncations}
\pmcreated{2013-11-21 15:00:46}
\pmmodified{2013-11-21 15:00:46}
\pmowner{PMBookProject}{1000683}
\pmmodifier{rspuzio}{6075}
\pmtitle{7.3 Truncations}
\pmrecord{4}{87700}
\pmprivacy{1}
\pmauthor{PMBookProject}{6075}
\pmtype{Feature}
\pmclassification{msc}{03B15}

\usepackage{xspace}
\usepackage{amssyb}
\usepackage{amsmath}
\usepackage{amsfonts}
\usepackage{amsthm}
\makeatletter
\newcommand{\base}{\ensuremath{\mathsf{base}}\xspace}
\newcommand{\blank}{\mathord{\hspace{1pt}\text{--}\hspace{1pt}}}
\newcommand{\ct}{  \mathchoice{\mathbin{\raisebox{0.5ex}{$\displaystyle\centerdot$}}}             {\mathbin{\raisebox{0.5ex}{$\centerdot$}}}             {\mathbin{\raisebox{0.25ex}{$\scriptstyle\,\centerdot\,$}}}             {\mathbin{\raisebox{0.1ex}{$\scriptscriptstyle\,\centerdot\,$}}}}
\newcommand{\decode}{\ensuremath{\mathsf{decode}}\xspace}
\newcommand{\defeq}{\vcentcolon\equiv}  
\newcommand{\define}[1]{\textbf{#1}}
\newcommand{\dpath}[4]{#3 =^{#1}_{#2} #4}
\def\@dprd#1{\prod_{(#1)}\,}
\def\@dprd@noparens#1{\prod_{#1}\,}
\def\@dsm#1{\sum_{(#1)}\,}
\def\@dsm@noparens#1{\sum_{#1}\,}
\def\@eatprd\prd{\prd@parens}
\def\@eatsm\sm{\sm@parens}
\newcommand{\encode}{\ensuremath{\mathsf{encode}}\xspace}
\newcommand{\eqv}[2]{\ensuremath{#1 \simeq #2}\xspace}
\newcommand{\eqvsym}{\simeq}    
\newcommand{\ext}{\mathsf{ext}}
\newcommand{\extend}[1]{\extendsmb(#1)}
\newcommand{\function}[4]{\left\{\begin{array}{rcl}#1 &      \longrightarrow & #2 \\ #3 & \longmapsto & #4 \end{array}\right.}
\newcommand{\htpy}{\sim}
\newcommand{\idfunc}[1][]{\ensuremath{\mathsf{id}_{#1}}\xspace}
\newcommand{\indexsee}[2]{\index{#1|see{#2}}}    
\newcommand{\jdeq}{\equiv}      
\newcommand{\Map}{\mathsf{Map}}
\newcommand{\mapfunc}[1]{\ensuremath{\mathsf{ap}_{#1}}\xspace} 
\newcommand{\nameless}{\mathord{\hspace{1pt}\underline{\hspace{1ex}}\hspace{1pt}}}
\newcommand{\narrowequation}[1]{$#1$}
\newcommand{\opp}[1]{\mathord{{#1}^{-1}}}
\newcommand{\pairr}[1]{{\mathopen{}(#1)\mathclose{}}}
\def\prd#1{\@ifnextchar\bgroup{\prd@parens{#1}}{\@ifnextchar\sm{\prd@parens{#1}\@eatsm}{\prd@noparens{#1}}}}
\def\prd@noparens#1{\mathchoice{\@dprd@noparens{#1}}{\@tprd{#1}}{\@tprd{#1}}{\@tprd{#1}}}
\def\prd@parens#1{\@ifnextchar\bgroup  {\mathchoice{\@dprd{#1}}{\@tprd{#1}}{\@tprd{#1}}{\@tprd{#1}}\prd@parens}  {\@ifnextchar\sm    {\mathchoice{\@dprd{#1}}{\@tprd{#1}}{\@tprd{#1}}{\@tprd{#1}}\@eatsm}    {\mathchoice{\@dprd{#1}}{\@tprd{#1}}{\@tprd{#1}}{\@tprd{#1}}}}}
\newcommand{\refl}[1]{\ensuremath{\mathsf{refl}_{#1}}\xspace}
\def\sm#1{\@ifnextchar\bgroup{\sm@parens{#1}}{\@ifnextchar\prd{\sm@parens{#1}\@eatprd}{\sm@noparens{#1}}}}
\def\sm@noparens#1{\mathchoice{\@dsm@noparens{#1}}{\@tsm{#1}}{\@tsm{#1}}{\@tsm{#1}}}
\def\sm@parens#1{\@ifnextchar\bgroup  {\mathchoice{\@dsm{#1}}{\@tsm{#1}}{\@tsm{#1}}{\@tsm{#1}}\sm@parens}  {\@ifnextchar\prd    {\mathchoice{\@dsm{#1}}{\@tsm{#1}}{\@tsm{#1}}{\@tsm{#1}}\@eatprd}    {\mathchoice{\@dsm{#1}}{\@tsm{#1}}{\@tsm{#1}}{\@tsm{#1}}}}}
\newcommand{\Sn}{\mathbb{S}}
\newcommand{\symlabel}[1]{\refstepcounter{symindex}\label{#1}}
\def\@tprd#1{\mathchoice{{\textstyle\prod_{(#1)}}}{\prod_{(#1)}}{\prod_{(#1)}}{\prod_{(#1)}}}
\newcommand{\tproj}[3][]{\mathopen{}\left|#3\right|_{#2}^{#1}\mathclose{}}
\newcommand{\tprojf}[2][]{|\blank|_{#2}^{#1}}
\newcommand{\trans}[2]{\ensuremath{{#1}_{*}\mathopen{}\left({#2}\right)\mathclose{}}\xspace}
\newcommand{\transfib}[3]{\ensuremath{\mathsf{transport}^{#1}(#2,#3)\xspace}}
\newcommand{\trunc}[2]{\mathopen{}\left\Vert #2\right\Vert_{#1}\mathclose{}}
\newcommand{\Trunc}[2]{\Bigl\Vert #2\Bigr\Vert_{#1}}
\newcommand{\truncf}[1]{\Vert \blank \Vert_{#1}}
\def\@tsm#1{\mathchoice{{\textstyle\sum_{(#1)}}}{\sum_{(#1)}}{\sum_{(#1)}}{\sum_{(#1)}}}
\newcommand{\ttrunc}[2]{\bigl\Vert #2\bigr\Vert_{#1}}
\newcommand{\typele}[1]{\ensuremath{{#1}\text-\mathsf{Type}}\xspace}
\newcommand{\unit}{\ensuremath{\mathbf{1}}\xspace}
\newcommand{\UU}{\ensuremath{\mathcal{U}}\xspace}
\newcommand{\vcentcolon}{:\!\!}
\newcounter{mathcount}
\setcounter{mathcount}{1}
\newtheorem{precor}{Corollary}
\newenvironment{cor}{\begin{precor}}{\end{precor}\addtocounter{mathcount}{1}}
\renewcommand{\theprecor}{7.3.\arabic{mathcount}}
\newenvironment{myeqn}{\begin{equation}}{\end{equation}\addtocounter{mathcount}{1}}
\renewcommand{\theequation}{7.3.\arabic{mathcount}}
\newtheorem{prelem}{Lemma}
\newenvironment{lem}{\begin{prelem}}{\end{prelem}\addtocounter{mathcount}{1}}
\renewcommand{\theprelem}{7.3.\arabic{mathcount}}
\newtheorem{prethm}{Theorem}
\newenvironment{thm}{\begin{prethm}}{\end{prethm}\addtocounter{mathcount}{1}}
\renewcommand{\theprethm}{7.3.\arabic{mathcount}}
\let\apfunc\mapfunc
\let\autoref\cref
\let\extendsmb\ext
\let\type\UU
\makeatother

\begin{document}
\indexsee{n-truncation@$n$-truncation}{truncation}%
\index{truncation!n-truncation@$n$-truncation|(defstyle}%

In \PMlinkname{\S 3.7}{37propositionaltruncation} we introduced the propositional truncation, which makes the ``best approximation'' of a type that is a mere
proposition, i.e.\ a $(-1)$-type.
In \PMlinkname{\S 6.9}{69truncations} we constructed this truncation as a higher inductive type, and gave one way to generalize it to a
0-truncation.
We now explain a better generalization of this, which truncates any type into an $n$-type for any $n\geq -2$; in classical homotopy theory this would be called its \define{$n^{\mathrm{th}}$ Postnikov section}.\index{Postnikov tower}

The idea is to make use of \PMlinkname{}{72uniquenessofidentityproofsandhedbergstheorem# utoref}, which states that $A$ is an $n$-type just when $\Omega^{n+1}(A,a)$
\index{loop space!iterated}%
is contractible for
all $a:A$, and \PMlinkname{Lemma 6.5.4}{65suspensions#Thmprelem3}, which implies that
\narrowequation{\Omega^{n+1}(A,a) \eqvsym \Map_{*}(\Sn^{n+1},(A,a)),} where $\Sn^{n+1}$ is equipp\-ed with some basepoint which we may as well call \base.
However, contractibility of $\Map_*(\Sn^{n+1},(A,a))$ is something that we can ensure directly by giving path constructors.

We might first of all try to define $\trunc nA$ to be generated by a function $\tprojf n : A \to \trunc n A$, together with for each
$r:\Sn^{n+1} \to \trunc n A$ and each $x:\Sn^{n+1}$, a path $s_r(x):r(x) = r(\base)$.
%
But this does not quite work, for the same reason that \PMlinkname{Remark 6.7.1}{67hubsandspokes#Thmprermk1} fails.
\index{hub and spoke}%
Instead, we use the full ``hub and spoke'' construction as in \PMlinkname{\S 6.7}{67hubsandspokes}.

Thus, for $n\ge -1$, we take $\trunc nA$ to be the higher inductive type generated by:
\begin{itemize}
\item a function $\tprojf n : A \to \trunc n A$,
\item for each $r:\Sn^{n+1} \to \trunc n A$, a \emph{hub} point $h(r):\trunc n A$, and
\item for each $r:\Sn^{n+1} \to \trunc n A$ and each $x:\Sn^{n+1}$, a \emph{spoke} path $s_r(x):r(x) = h(r)$.
\end{itemize}

\noindent
The existence of these constructors is now enough to show:

\begin{lem}
  $\trunc n A$ is an $n$-type.
\end{lem}
\begin{proof}
  By \PMlinkname{}{72uniquenessofidentityproofsandhedbergstheorem# utoref}, it suffices to show that $\Omega ^{n+1}(\trunc nA,b)$ is contractible for all $b:\trunc nA$, which by
\PMlinkname{Lemma 6.5.4}{65suspensions#Thmprelem3} is equivalent to \narrowequation{\Map_*(\Sn^{n+1},(\trunc nA,b)).}
  As center of contraction for the latter, we choose the function $c_b:\Sn^{n+1} \to \trunc nA$ which is constant at $b$, together with
$\refl b : c_b(\base) = b$.

  Now, an arbitrary element of $\Map_*(\Sn^{n+1},(\trunc nA,b))$ consists of a map $r:\Sn^{n+1} \to \trunc n A$ together with a path
$p:r(\base)=b$.
  By function extensionality, to show $r = c_b$ it suffices to give, for each $x:\Sn^{n+1}$, a path $r(x)=c_b(x) \jdeq b$.
  We choose this to be the composite $s_r(x) \ct \opp{s_r(\base)} \ct p$, where $s_r(x)$ is the spoke at $x$.


  Finally, we must show that when transported along this equality $r=c_b$, the path $p$ becomes $\refl b$.
  By transport in path types, this means we need
  \[\opp{(s_r(\base) \ct \opp{s_r(\base)} \ct p)} \ct p = \refl b.\]
  But this is immediate from path operations.
\end{proof}

(This construction fails for $n=-2$, but in that case we can simply define $\trunc{-2}{A}\defeq \unit$ for all $A$.
From now on we assume $n\ge -1$.)

\index{induction principle!for truncation}%
To show the desired universal property of the $n$-truncation, we need the induction principle.
We extract this from the constructors in the usual way; it says that given $P:\trunc nA\to\type$ together with
\begin{itemize}
\item For each $a:A$, an element $g(a) : P(\tproj na)$,
\item For each $r:\Sn^{n+1} \to \trunc n A$ and $r':\prd{x:\Sn^{n+1}} P(r(x))$, an element $h'(r,r'):P(h(r))$,
\item For each $r:\Sn^{n+1} \to \trunc n A$ and $r':\prd{x:\Sn^{n+1}} P(r(x))$, and each $x:\Sn^{n+1}$, a dependent path
$\dpath{P}{s_r(x)}{r'(x)}{h'(r,r')}$,
\end{itemize}
there exists a section $f:\prd{x:\trunc n A} P(x)$ with $f(\tproj n a) \jdeq g(a)$ for all $a:A$.
To make this more useful, we reformulate it as follows.

\begin{thm}\label{thm:truncn-ind}
  For any type family $P:\trunc n A \to \type$ such that each $P(x)$ is an $n$-type, and any function $g : \prd{a:A} P(\tproj n a)$, there
exists a section $f:\prd{x:\trunc n A} P(x)$ such that $f(\tproj n a)\defeq g(a)$ for all $a:A$.
\end{thm}
\begin{proof}
  It will suffice to construct the second and third data listed above, since $g$ has exactly the type of the first datum.
  Given $r:\Sn^{n+1} \to \trunc n A$ and $r':\prd{x:\Sn^{n+1}} P(r(x))$, we have $h(r):\trunc n A$ and $s_r :\prd{x:\Sn^{n+1}} (r(x) =
h(r))$.
  Define $t:\Sn^{n+1} \to P(h(r))$ by $t(x) \defeq \trans{s_r(x)}{r'(x)}$.
  Then since $P(h(r))$ is $n$-truncated, there exists a point $u:P(h(r))$ and a contraction $v:\prd{x:\Sn^{n+1}} (t(x) = u)$.
  Define $h'(r,r') \defeq u$, giving the second datum.
  Then (recalling the definition of dependent paths), $v$ has exactly the type required of the third datum.
\end{proof}

In particular, if $E$ is some $n$-type, we can consider the constant family of types equal to $E$ for every point of $A$.
\symlabel{extend}
\index{recursion principle!for truncation}%
Thus, every map $f:A\to{}E$ can be extended to a map $\extend{f}:\trunc nA\to{}E$ defined by $\extend{f}(\tproj na)\defeq f(a)$; this is the \emph{recursion principle} for $\trunc n A$.

The induction principle also implies a uniqueness principle for functions of this form.
\index{uniqueness!principle, propositional!for functions on a truncation}%
Namely, if $E$ is an $n$-type and $g,g':\trunc nA\to{}E$ are such
that $g(\tproj na)=g'(\tproj na)$ for every $a:A$, then $g(x)=g'(x)$ for all $x:\trunc nA$, since the type $g(x)=g'(x)$ is an $n$-type.
Thus, $g=g'$.
This yields the following universal property.

\begin{lem}[Universal property of truncations]\label{thm:trunc-reflective}
  \index{universal!property!of truncation}%
  Let $n\ge-2$, $A:\type$ and $B:\typele{n}$. The following map is an
  equivalence:
  \[\function{(\trunc nA\to{}B)}{(A\to{}B)}{g}{g\circ\tprojf n}\]
\end{lem}

\begin{proof}
  Given that $B$ is $n$-truncated, any $f:A\to{}B$ can be extended to a map $\extend{f}:\trunc nA\to{}B$.
  The map $\extend{f}\circ\tprojf n$ is equal to $f$, because for every $a:A$ we have $\extend{f}(\tproj na)=f(a)$ by definition.
  And the map $\extend{g\circ\tprojf n}$ is equal to $g$, because they both send $\tproj na$ to $g(\tproj na)$.
\end{proof}

In categorical language, this says that the $n$-types form a \emph{reflective subcategory} of the category of types.
\index{reflective!subcategory}%
(To state this fully precisely, one ought to use the language of $(\infty,1)$-categories.)
\index{.infinity1-category@$(\infty,1)$-category}%
In particular, this implies that the $n$-truncation is functorial:
given $f:A\to B$, applying the recursion principle to the composite $A\xrightarrow{f} B \to \trunc n B$ yields a map $\trunc n f: \trunc n A \to \trunc n B$.
By definition, we have a homotopy
\begin{myeqn}
  \mathsf{nat}^f_n : \prd{a:A} \trunc n f(\tproj n a) = \tproj n {f(a)},\label{eq:trunc-nat}
\end{myeqn}
expressing \emph{naturality} of the maps $\tprojf n$.

Uniqueness implies functoriality laws such as $\trunc n {g\circ f} = \trunc n g \circ \trunc n f$ and $\trunc n{\idfunc[A]} = \idfunc[\trunc n A]$, with attendant coherence laws.
We also have higher functoriality, for instance:

\begin{lem}\label{thm:trunc-htpy}
  Given $f,g:A\to B$ and a homotopy $h:f\htpy g$, there is an induced homotopy $\trunc n h : \trunc n f \htpy \trunc n g$ such that the composite
  \begin{myeqn}\label{eq:trunc-htpy}
  \includegraphics{HoTT_fig_7.3.1.png}
%    \xymatrix@C=3.6pc{\tproj n{f(a)} \ar@{=}[r]^-{\opp{\mathsf{nat}^f_n(a)}} &
%      \trunc n f(\tproj n a) \ar@{=}[r]^-{\trunc n h(\tproj na)} &
%      \trunc n g(\tproj n a) \ar@{=}[r]^-{\mathsf{nat}^g_n(a)} &
%      \tproj n{g(a)}}\label{eq:trunc-htpy}
  \end{myeqn}
  is equal to $\apfunc{\tprojf n}(h(a))$.
\end{lem}
\begin{proof}
  First, we indeed have a homotopy with components $\apfunc{\tprojf n}(h(a)) : \tproj n{f(a)} = \tproj n{g(a)}$.
  Composing on either sides with the paths $\tproj n{f(a)} = \trunc n f(\tproj n a)$ and $\tproj n{g(a)} = \trunc n g(\tproj n a)$, which arise from the definitions of $\trunc n f$ and $\trunc ng$, we obtain a homotopy $(\trunc n f \circ \tprojf n) \htpy (\trunc n g \circ \tprojf n)$, and hence an equality by function extensionality.
  But since $(\blank\circ \tprojf n)$ is an equivalence, there must be a path $\trunc nf = \trunc ng$ inducing it, and the coherence laws for function extensionality imply~\eqref{eq:trunc-htpy}.
\end{proof}

The following observation about reflective subcategories is also standard.

\begin{cor}
  A type $A$ is an $n$-type if and only if $\tprojf n : A \to \trunc n A$ is an equivalence.
\end{cor}
\begin{proof}
  ``If'' follows from closure of $n$-types under equivalence.
  On the other hand, if $A$ is an $n$-type, we can define $\ext(\idfunc[A]):\trunc n A\to{}A$.
  Then we have $\ext(\idfunc[A])\circ\tprojf n=\idfunc[A]:A\to{}A$ by
  definition.  In order to prove that
  $\tprojf n\circ\ext(\idfunc[A])=\idfunc[\trunc nA]$, we only need to prove
  that $\tprojf n\circ\ext(\idfunc[A])\circ\tprojf n=
  \idfunc[\trunc nA]\circ\tprojf n$.
  This is again true:
 \begin{figure}
  \centering
  \includegraphics{HoTT_fig_7.3.2.png}
 \end{figure}
 % \[\raisebox{\depth-\height+1em}{\xymatrix{
 %   A \ar^-{\tprojf n}[r] \ar_{\idfunc[A]}[rd] &
 %   \trunc nA \ar^>>>{\ext(\idfunc[A])}[d] \ar@/^40pt/^{\idfunc[\trunc nA]}[dd] \\
 %   & A \ar_{\tprojf n}[d] \\
 %   & \trunc nA}}
  \qedhere\]
\end{proof}

The category of $n$-types also has some special properties not possessed by all reflective subcategories.
For instance, the reflector $\trunc n-$ preserves finite products.

\begin{thm}\label{cor:trunc-prod}
  For any types $A$ and $B$, the induced map $\trunc n{A\times B} \to \trunc nA \times \trunc nB$ is an equivalence.
\end{thm}
\begin{proof}
  It suffices to show that $\trunc nA \times \trunc nB$ has the same universal property as $\trunc n{A\times B}$.
  Thus, let $C$ be an $n$-type; we have
  \begin{align*}
    (\trunc nA \times \trunc nB \to C)
    &= (\trunc nA \to (\trunc nB \to C))\\
    &= (\trunc nA \to (B \to C))\\
    &= (A \to (B \to C))\\
    &= (A \times B \to C)
  \end{align*}
  using the universal properties of $\trunc nB$ and $\trunc nA$, along with the fact that $B\to C$ is an $n$-type since $C$ is.
  It is straightforward to verify that this equivalence is given by composing with $\tprojf n \times \tprojf n$, as needed.
\end{proof}

The following related fact about dependent sums is often useful.

\begin{thm}\label{thm:trunc-in-truncated-sigma}
Let $P:A\to\type$ be a family of types. Then there is an equivalence
\begin{equation*}
\eqv{\Trunc n{\sm{x:A}\trunc n{P(x)}}}{\Trunc n{\sm{x:A}P(x)}}.
\end{equation*}
\end{thm}

\begin{proof}
We use the induction principle of $n$-truncation several times to construct
functions
\begin{align*}
\varphi & : \Trunc n{\sm{x:A}\trunc n{P(x)}}\to\Trunc n{\sm{x:A}P(x)}\\
\psi & : \Trunc n{\sm{x:A}P(x)}\to \Trunc n{\sm{x:A} \trunc n{P(x)}}
\end{align*}
and homotopies $H:\varphi\circ\psi\htpy \idfunc$ and $K:\psi\circ\varphi\htpy
\idfunc$ exhibiting them as quasi-inverses.
We define $\varphi$ by setting $\varphi(\tproj n{\pairr{x,\tproj nu}})\defeq\tproj n{\pairr{x,u}}$.
We define $\psi$ by setting $\psi(\tproj n{\pairr{x,u}})\defeq\tproj n{\pairr{x,\tproj nu}}$.
Then we define $H(\tproj n{\pairr{x,u}})\defeq \refl{\tproj n{\pairr{x,u}}}$ and
$K(\tproj n{\pairr{x,\tproj nu}})\defeq \refl{\tproj n{\pairr{x,\tproj nu}}}$.
\end{proof}

\begin{cor}\label{thm:refl-over-ntype-base}
  If $A$ is an $n$-type and $P:A\to\type$ is any type family, then
  \[ \eqv{\sm{a:A} \trunc n{P(a)}}{\Trunc n{\sm{a:A}P(a)}} \]
\end{cor}
\begin{proof}
  If $A$ is an $n$-type, then the left-hand type above is already an $n$-type, hence equivalent to its $n$-truncation; thus this follows from \PMlinkname{Theorem 7.3.9}{73truncations#Thmprethm3}.
\end{proof}

We can characterize the path spaces of a truncation using the same method that we used in \PMlinkname{\S 2.12}{212coproducts},\PMlinkname{\S 2.13}{213naturalnumbers} for
coproducts and natural numbers (and which we will use in \PMlinkexternal{Chapter 8}{http://planetmath.org/node/87582} to calculate homotopy groups).
Unsurprisingly, the path spaces in the $(n+1)$-truncation of $A$ are the $n$-truncations of the path spaces of $A$.
Indeed, for any $x,y:A$ there is a canonical map
\begin{myeqn}
  f:\ttrunc n{x=_Ay}\to \Big(\tproj {n+1}x=_{\trunc{n+1}A}\tproj {n+1}y\Big)\label{eq:path-trunc-map}
\end{myeqn}
defined by
\[f(\tproj n{p})\defeq \apfunc{\tprojf {n+1}}(p). \]
This definition uses the recursion principle for $\truncf n$, which is correct because $\trunc {n+1}A$ is $(n+1)$-truncated, so that the
codomain of $f$ is $n$-truncated.

\begin{thm} \label{thm:path-truncation}
  For any $A$ and $x,y:A$ and $n\ge -2$, the map~\eqref{eq:path-trunc-map} is an equivalence; thus we have
  \[ \eqv{\ttrunc n{x=_Ay}}{\Big(\tproj {n+1}x=_{\trunc{n+1}A}\tproj {n+1}y\Big)}. \]
\end{thm}

\begin{proof}
  The proof is a simple application of the encode-decode method: 
  As in previous situations, we cannot directly define a quasi-inverse to the map~\eqref{eq:path-trunc-map} because there is no way to induct on an
equality between $\tproj {n+1}x$ and $\tproj {n+1}y$.
  Thus, instead we generalize its type, in order to have general elements of the type $\trunc{n+1}A$ instead of $\tproj {n+1}x$ and $\tproj
{n+1}y$.
  Define $P:\trunc {n+1}A\to\trunc {n+1}A\to\typele{n}$ by
  \[P(\tproj {n+1}x,\tproj {n+1}y)\defeq \trunc n{x=_Ay}\]
  This definition is correct because $\trunc n{x=_Ay}$ is $n$-truncated, and $\typele{n}$ is $(n+1)$-truncated by
\PMlinkname{Theorem 7.1.11}{71definitionofntypes#Thmprethm7}.
  Now for every $u,v:\trunc{n+1}A$, there is a map
  \[\encode:P(u,v) \to \big(u=_{\trunc{n+1}A}v\big)\]
  defined for $u=\tproj {n+1}x$ and $v=\tproj {n+1}y$ and $p:x=y$ by
  \[\encode(\tproj n{p})\defeq \apfunc{\tproj{n+1}\nameless} (p).\]
  Since the codomain of $\encode$ is $n$-truncated, it suffices to define it only for $u$ and $v$ of this form, and then it's just the same
definition as before.
  We also define a function
  \[ r : \prd{u:\trunc{n+1} A} P(u,u) \]
  by induction on $u$, where $r(\tproj{n+1} x) \defeq \tproj n {\refl x}$.

  Now we can define an inverse map
  \[\decode: (u=_{\trunc{n+1}A}v) \to P(u,v)\]
  by
  \[\decode(p) \defeq \transfib{v\mapsto P(u,v)}{p}{r(u)}. \]
  To show that the composite
  \[ (u=_{\trunc{n+1}A}v) \xrightarrow{\decode} P(u,v) \xrightarrow{\encode} (u=_{\trunc{n+1}A}v) \]
  is the identity function, by path induction it suffices to check it for $\refl u : u=u$, in which case what we need to know is that
$\decode(r(u)) = \refl{u}$.
  But since this is an $n$-type, hence also an $(n+1)$-type, we may assume $u\jdeq \tproj {n+1} x$, in which case it follows by definition
of $r$ and $\decode$.
  Finally, to show that 
  \[ P(u,v) \xrightarrow{\encode} (u=_{\trunc{n+1}A}v) \xrightarrow{\decode} P(u,v) \]
  is the identity function, since this goal is again an $n$-type, we may assume that $u=\tproj {n+1}x$ and $v=\tproj {n+1}y$ and that we are
considering $\tproj n p:P(\tproj{n+1}x,\tproj{n+1}y)$ for some $p:x=y$.
  Then we have
  \begin{align*}
    \decode(\encode(\tproj n p)) &= \decode(\apfunc{\tproj{n+1}\nameless}(p))\\
    &= \transfib{v\mapsto P(\tproj{n+1}x,v)}{\apfunc{\tproj{n+1}\nameless}(p)}{\tproj n {\refl x}}\\
    &= \transfib{v\mapsto \trunc n{u=v}}{p}{\tproj n {\refl x}}\\
    &= \tproj n {\transfib{v \mapsto (u=v)}{p}{\refl x}}\\
    &= \tproj n p.
  \end{align*}
  This completes the proof that \encode and \decode are quasi-inverses.
  The stated result is then the special case where $u=\tproj {n+1}x$ and $v=\tproj {n+1}y$.
\end{proof}

\begin{cor}
  Let $n\ge-2$ and $(A,a)$ be a pointed type. Then
  \[\ttrunc n{\Omega(A,a)}=\Omega\mathopen{}\left(\trunc{n+1}{(A,a)}\right)\]
\end{cor}
\begin{proof}
  This is a special case of the previous lemma where $x=y=a$.
\end{proof}

\begin{cor}
  Let $n\ge -2$ and $k\ge 0$ and $(A,a)$ a pointed type.  Then
  \[\ttrunc n{\Omega^k(A,a)} = \Omega^k\mathopen{}\left(\trunc{n+k}{(A,a)}\right). \]
\end{cor}
\begin{proof}
  By induction on $k$, using the recursive definition of $\Omega^k$.
\end{proof}

We also observe that ``truncations are cumulative'': if we truncate to an $n$-type and then to a $k$-type with $k\le n$, then we might as
well have truncated directly to a $k$-type.

\begin{lem} \label{lem:truncation-le}
  Let $k,n\ge-2$ with $k\le{}n$ and $A:\type$. Then
  $\trunc k{\trunc nA}=\trunc kA$.
\end{lem}
\begin{proof}
  We define two maps $f:\trunc k{\trunc nA}\to\trunc kA$ and
  $g:\trunc kA\to\trunc k{\trunc nA}$ by
  %
  \[
   f(\tproj k{\tproj na}) \defeq \tproj ka
   \qquad\text{and}\qquad
   g(\tproj ka) \defeq \tproj k{\tproj na}.
  \]
  %
  The map $f$ is well-defined because $\trunc kA$ is $k$-truncated and also
  $n$-truncated (because $k\le{}n$), and the map $g$ is well-defined because
  $\trunc k{\trunc nA}$ is $k$-truncated.

  The composition $f\circ{}g:\trunc kA\to\trunc kA$ satisfies
  $(f\circ{}g)(\tproj ka)=\tproj ka$, hence $f\circ{}g=\idfunc[\trunc kA]$.
  Similarly, we have $(g\circ{}f)(\tproj k{\tproj na})=\tproj k{\tproj na}$ and hence $g\circ{}f=\idfunc[\trunc k{\trunc nA}]$.
\end{proof}

% \begin{lem}
%   We have $\trunc n{\unit}=\unit$.
% \end{lem}
% \begin{proof}
%   Indeed, $\unit$ is $n$-truncated for every $n$ hence $\trunc n{\unit}=\unit$ by
%   \autoref{reflectPequiv}.
% \end{proof}

\index{truncation!n-truncation@$n$-truncation|)}%


\end{document}
