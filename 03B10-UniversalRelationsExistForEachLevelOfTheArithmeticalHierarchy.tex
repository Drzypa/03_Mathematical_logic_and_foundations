\documentclass[12pt]{article}
\usepackage{pmmeta}
\pmcanonicalname{UniversalRelationsExistForEachLevelOfTheArithmeticalHierarchy}
\pmcreated{2013-03-22 12:58:40}
\pmmodified{2013-03-22 12:58:40}
\pmowner{Henry}{455}
\pmmodifier{Henry}{455}
\pmtitle{universal relations exist for each level of the arithmetical hierarchy}
\pmrecord{5}{33349}
\pmprivacy{1}
\pmauthor{Henry}{455}
\pmtype{Theorem}
\pmcomment{trigger rebuild}
\pmclassification{msc}{03B10}

% this is the default PlanetMath preamble.  as your knowledge
% of TeX increases, you will probably want to edit this, but
% it should be fine as is for beginners.

% almost certainly you want these
\usepackage{amssymb}
\usepackage{amsmath}
\usepackage{amsfonts}

% used for TeXing text within eps files
%\usepackage{psfrag}
% need this for including graphics (\includegraphics)
%\usepackage{graphicx}
% for neatly defining theorems and propositions
%\usepackage{amsthm}
% making logically defined graphics
%%%\usepackage{xypic}

% there are many more packages, add them here as you need them

% define commands here
%\PMlinkescapeword{theory}
\begin{document}
Let $L\in\{\Sigma_n,\Delta_n,\Pi_n\}$ and take any $k\in\mathbb{N}$.  Then there is a $k+1$-ary relation $U\in L$ such that $U$ is universal for the $k$-ary relations in $L$.

\subsection*{Proof}

First we prove the case where $L=\Delta_1$, the recursive relations.  We use the example of a G\"odel numbering.

Define $T$ to be a $k+2$-ary relation such that $T(e,\vec{x},a)$ if:

\begin{itemize}
\item $e=\ulcorner\phi\urcorner$ 

\item $a$ is a deduction of either $\phi(\vec{x})$ or $\neg\phi(\vec{x})$
\end{itemize}

Since \PMlinkname{deductions are }{DeductionsAreDelta1}$\Delta_1$, it follows that $T$ is $\Delta_1$.  Then define $U^\prime(e,\vec{x})$ to be the least $a$ such that $T(e,\vec{x},a)$ and $U(e,\vec{x})\leftrightarrow (U^\prime(e,\vec{x}))_{\operatorname{len}(U^\prime(e,\vec{x}))}=e$.  This is again $\Delta_1$ since the $\Delta_1$ functions are \PMlinkname{closed under minimization}{Delta_1Bootstrapping}.

If $f$ is any $k-ary$ $\Delta_1$ function then $f(\vec{x})=U(\ulcorner f\urcorner,\vec{x})$.

Now take $L$ to be the $k$-ary relatons in either $\Sigma_n$ or $\Pi_n$.  Call the universal relation for $k+n$-ary $\Delta_1$ relations $U_\Delta$.  Then any $\phi\in L$ is equivalent to a relation in the form $Q y_1 Q^\prime y_2 \cdots Q^* y_n \psi(\vec{x},\vec{y})$ where $g\in\Delta_1$, and so $U(\vec{x})=Q y_1 Q^\prime y_2 \cdots Q^* y_n U_\Delta(\ulcorner\psi\urcorner,\vec{x},\vec{y})$.  Then $U$ is universal for $L$.

Finally, if $L$ is the $k$-ary $\Delta_n$ relations and $\phi\in L$ then $\phi$ is equivalent to relations of the form $\exists y_1\forall y_2\cdots Q y_n \psi(\vec{x},\vec{y})$ and $\forall z_1\exists z_2\cdots Q z_n\eta(\vec{x},\vec{z})$.  If the $k$-ary universal relations for $\Sigma_n$ and $\Pi_n$ are $U_\Sigma$ and $U_\Pi$ respectively then $\phi(\vec{x})\leftrightarrow U_\Sigma(\ulcorner\psi\urcorner,\vec{x})\wedge U_\Pi(\ulcorner\eta\urcorner,\vec{x})$.
%%%%%
%%%%%
\end{document}
