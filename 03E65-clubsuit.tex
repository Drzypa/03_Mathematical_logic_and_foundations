\documentclass[12pt]{article}
\usepackage{pmmeta}
\pmcanonicalname{clubsuit}
\pmcreated{2013-03-22 12:53:52}
\pmmodified{2013-03-22 12:53:52}
\pmowner{Henry}{455}
\pmmodifier{Henry}{455}
\pmtitle{$\clubsuit$}
\pmrecord{4}{33246}
\pmprivacy{1}
\pmauthor{Henry}{455}
\pmtype{Definition}
\pmcomment{trigger rebuild}
\pmclassification{msc}{03E65}
\pmsynonym{clubsuit}{clubsuit}
\pmrelated{Diamond}
\pmrelated{DiamondIsEquivalentToClubsuitAndContinuumHypothesis}
\pmrelated{ProofOfDiamondIsEquivalentToClubsuitAndContinuumHypothesis}
\pmrelated{CombinatorialPrinciple}

% this is the default PlanetMath preamble.  as your knowledge
% of TeX increases, you will probably want to edit this, but
% it should be fine as is for beginners.

% almost certainly you want these
\usepackage{amssymb}
\usepackage{amsmath}
\usepackage{amsfonts}

% used for TeXing text within eps files
%\usepackage{psfrag}
% need this for including graphics (\includegraphics)
%\usepackage{graphicx}
% for neatly defining theorems and propositions
%\usepackage{amsthm}
% making logically defined graphics
%%%\usepackage{xypic}

% there are many more packages, add them here as you need them

% define commands here
%\PMlinkescapeword{theory}
\begin{document}
$\clubsuit_S$ is a combinatoric principle weaker than $\Diamond_S$.  It states that, for $S$ stationary in $\kappa$, there is a sequence $\langle A_\alpha\rangle_{\alpha\in S}$ such that $A_\alpha\subseteq\alpha$ and $\operatorname{sup}(A_\alpha)=\alpha$ and with the property that for each unbounded subset $T\subseteq\kappa$ there is some $A_\alpha\subseteq X$.

Any sequence satisfying $\Diamond_S$ can be adjusted so that $\operatorname{sup}(A_\alpha)=\alpha$, so this is indeed a weakened form of $\Diamond_S$.

Any such sequence actually contains a stationary set of $\alpha$ such that $A_\alpha\subseteq T$ for each $T$: given any club $C$ and any unbounded $T$, construct a $\kappa$ sequence, $C^*$ and $T^*$, from the elements of each, such that the $\alpha$-th member of $C^*$ is greater than the $\alpha$-th member of $T^*$, which is in turn greater than any earlier member of $C^*$.  Since both sets are unbounded, this construction is possible, and $T^*$ is a subset of $T$ still unbounded in $\kappa$.  So there is some $\alpha$ such that $A_\alpha\subseteq T^*$, and since $\operatorname{sup}(A_\alpha)=\alpha$, $\alpha$ is also the limit of a subsequence of $C^*$ and therefore an element of $C$.
%%%%%
%%%%%
\end{document}
