\documentclass[12pt]{article}
\usepackage{pmmeta}
\pmcanonicalname{Infinitesimal}
\pmcreated{2013-03-22 13:22:59}
\pmmodified{2013-03-22 13:22:59}
\pmowner{mps}{409}
\pmmodifier{mps}{409}
\pmtitle{infinitesimal}
\pmrecord{12}{33918}
\pmprivacy{1}
\pmauthor{mps}{409}
\pmtype{Definition}
\pmcomment{trigger rebuild}
\pmclassification{msc}{03H05}
\pmclassification{msc}{06F25}
\pmclassification{msc}{03C64}
\pmrelated{Hyperreal}
\pmdefines{infinitesimal}
\pmdefines{Archimedean}

\endmetadata

% this is the default PlanetMath preamble.  as your knowledge
% of TeX increases, you will probably want to edit this, but
% it should be fine as is for beginners.

% almost certainly you want these
\usepackage{amssymb}
\usepackage{amsmath}
\usepackage{amsfonts}

% used for TeXing text within eps files
%\usepackage{psfrag}
% need this for including graphics (\includegraphics)
%\usepackage{graphicx}
% for neatly defining theorems and propositions
%\usepackage{amsthm}
% making logically defined graphics
%%%\usepackage{xypic}

% there are many more packages, add them here as you need them

% define commands here
\newcommand{\fm}[1]{{\it #1}}
\begin{document}
Let \fm{R} be a real closed field, for example the reals thought of as a
structure in \fm{L}, the language of ordered rings.  Let \fm{B} be some set
of parameters from \fm{R}. Consider the following set of formulas in
\fm{L}(\fm{B}):
\[
  \{ x<b: b \in B \land b>0\}
\]
Then this set of formulas is finitely satisfied, so by compactness is
consistent.  In fact this set of formulas extends to a unique type \fm{p}
over \fm{B}, as it defines a Dedekind cut.  Thus there is some model \fm{M}
containing \fm{B} and some $a \in M$ so that the type of \fm{a} over \fm{B} is
\fm{p}.

Any such element will be called \emph{B-infinitesimal}.  In
particular, suppose $B=\emptyset$.  Then the definable closure of
\fm{B} is the intersection of the reals with the algebraic numbers.
Then a $\emptyset$-infinitesimal (or simply \emph{infinitesimal}) is
any element of any real closed field that is positive but smaller than
every real algebraic (positive) number.

As noted above such models exist, by compactness. One can construct
them using ultraproducts; see the entry ``\PMlinkname{Hyperreal}{Hyperreal}'' for more
details.  This is due to
Abraham Robinson, who used such fields to formulate nonstandard
analysis.

Let \fm{K} be any ordered ring.  Then \fm{K} contains $\mathbf{N}$.
We say $K$ is \emph{archimedean} if and only if for every $a \in K$
there is some $n \in \mathbf{N}$ so that $\fm{a} < \fm{n}$.  Otherwise
$K$ is \emph{non-archimedean}.

Real closed fields with infinitesimal elements are non-archimedean:
for any infinitesimal \fm{a} we have $a<1/n$ and thus $1/a>n$ for each
$n \in \mathbf{N}$.

\begin{thebibliography}{9}
\bibitem{R}
Robinson, A., \emph{Selected papers of Abraham
Robinson. Vol. II. Nonstandard analysis and philosophy}, New Haven,
Conn., 1979.
\end{thebibliography}

% more things will need to be added here
% and some things below might need to be added to the defines list
\PMlinkescapeword{archimedean}
\PMlinkescapeword{non-archimedean}

%%%%%
%%%%%
\end{document}
