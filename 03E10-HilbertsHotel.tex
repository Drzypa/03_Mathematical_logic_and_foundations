\documentclass[12pt]{article}
\usepackage{pmmeta}
\pmcanonicalname{HilbertsHotel}
\pmcreated{2013-03-22 14:18:55}
\pmmodified{2013-03-22 14:18:55}
\pmowner{Daume}{40}
\pmmodifier{Daume}{40}
\pmtitle{Hilbert's hotel}
\pmrecord{11}{35781}
\pmprivacy{1}
\pmauthor{Daume}{40}
\pmtype{Topic}
\pmcomment{trigger rebuild}
\pmclassification{msc}{03E10}

\endmetadata

% this is the default PlanetMath preamble.  as your knowledge
% of TeX increases, you will probably want to edit this, but
% it should be fine as is for beginners.

% almost certainly you want these
\usepackage{amssymb}
\usepackage{amsmath}
\usepackage{amsfonts}

% used for TeXing text within eps files
%\usepackage{psfrag}
% need this for including graphics (\includegraphics)
%\usepackage{graphicx}
% for neatly defining theorems and propositions
%\usepackage{amsthm}
% making logically defined graphics
%%%\usepackage{xypic}

% there are many more packages, add them here as you need them

% define commands here
\begin{document}
\PMlinkescapeword{free}
\PMlinkescapeword{time}
\PMlinkescapeword{times}
\PMlinkescapeword{walk}

The hotel manager David Hilbert had a very large hotel, in fact, it had infinitely many rooms numbered $1$, $2$, $3$, $\ldots$.  The hotel was very popular and every room was occupied. One day a new guest arrived.\\
-Is there any free room?\\
-No, Mr. Hilbert said.\\
-Oh, what a pity, the guest said and started to walk away.\\
-But wait, you can still get a room.\\

The new guest was very confused by this and asked how that could be possible.\\
-I'll just ask the guest in room number 1 to move to room number 2, the guest in number 2 move to room 3, the guest in room 3 move to number 4, and so on, and then you can have room number 1.

The guest was very happy with this and called his friends to tell them about this fantastic hotel. Then one day they all arrived at the same time.\\
-Hello, we are \PMlinkname{countably}{Countable} many people, and we want a room each.
Mr. Hilbert felt reluctant to ask each guest to move to a new room an infinite number of times. That would be very unpleasant, and they will never finish either. Luckily he got a brilliant idea.\\
-I'll let guest in number 1 move into number 2, guest in number 2 move into number 4, guest in number 3 move into number 6, number 4 to number 8 and so on.
Then you can move into any room with an odd number; they will all be free, and there will be rooms for all of you.

Now all the new guests got a room on their own.

\textbf{Historical note:}\\
David Hilbert is one of the great mathematicians in the history of mathematics. One of the things he studied was the foundations of mathematics and that includes the nature of infinity.
%%%%%
%%%%%
\end{document}
