\documentclass[12pt]{article}
\usepackage{pmmeta}
\pmcanonicalname{PrincipleOfFiniteInduction}
\pmcreated{2013-03-22 11:46:41}
\pmmodified{2013-03-22 11:46:41}
\pmowner{CWoo}{3771}
\pmmodifier{CWoo}{3771}
\pmtitle{principle of finite induction}
\pmrecord{24}{30245}
\pmprivacy{1}
\pmauthor{CWoo}{3771}
\pmtype{Theorem}
\pmcomment{trigger rebuild}
\pmclassification{msc}{03E25}
\pmclassification{msc}{00-02}
\pmrelated{TransfiniteInduction}
\pmrelated{AnExampleOfMathematicalInduction}
\pmrelated{Induction}
\pmrelated{WellFoundedInduction}
\pmdefines{induction hypothesis}
\pmdefines{inductive hypothesis}
\pmdefines{base case}
\pmdefines{base step}

\usepackage{amssymb}
\usepackage{amsmath}
\usepackage{amsfonts}
\usepackage{graphicx}
%%%%\usepackage{xypic}
\begin{document}
\PMlinkescapeword{moment's}
\PMlinkescapeword{complete}
\PMlinkescapeword{equivalent}
\PMlinkescapeword{term}
\PMlinkescapeword{base}
\PMlinkescapeword{hypothesis}
\PMlinkescapeword{range}


The principle of finite induction, also known as \emph{mathematical induction}, is commonly formulated in two ways. Both are equivalent. The first formulation is known as \emph{weak} induction. It asserts that if a statement $P(n)$ holds for $n=0$ and if $P(n)\Rightarrow P(n+1)$, then $P(n)$ holds for all natural numbers $n$. The case $n=0$ is called the \emph{base case} or \emph{base step} and the implication $P(n)\Rightarrow P(n+1)$ is called the \emph{inductive step}. In an inductive proof, one uses the term \emph{induction hypothesis} or \emph{inductive hypothesis} to refer back to the statement $P(n)$ when one is trying to prove $P(n+1)$ from it.

The second formulation is known as \emph{strong} or \emph{complete} induction. It asserts that if the implication $\forall n((\forall m < n P(m))\Rightarrow P(n))$ is true, then $P(n)$ is true for all natural numbers $n$. (Here, the quantifiers range over all natural numbers.) As we have formulated it, strong induction does not require a separate base case. Note that the implication $\forall n((\forall m < n P(m))\Rightarrow P(n)$ already entails $P(0)$ since the statement $\forall m<0 P(m)$ holds vacuously (there are no natural numbers less that zero). 

A moment's thought will show that the first formulation (weak induction) is equivalent to the following:
\begin{quote}
Let $S$ be a set natural numbers such that
\begin{enumerate}
\item $0$ belongs to $S$, and
\item if $n$ belongs to $S$, so does $n+1$.
\end{enumerate}
Then $S$ is the set of all natural numbers.
\end{quote}

Similarly, strong induction can be stated:
\begin{quote}
If $S$ is a set of natural numbers such that $n$ belongs to $S$ whenever all numbers less than $n$ belong to $S$, then $S$ is the set of all natural numbers.
\end{quote}

The principle of finite induction can be derived from the fact that every nonempty set of natural numbers has a smallest element. This fact is known as the \emph{well-ordering principle for natural numbers}. (Note that this is not the same thing as the \emph{well-ordering principle}, which is equivalent to the axiom of choice and has nothing to do with induction.)
%%%%%
%%%%%
%%%%%
%%%%%
\end{document}
