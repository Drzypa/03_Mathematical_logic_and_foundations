\documentclass[12pt]{article}
\usepackage{pmmeta}
\pmcanonicalname{TruthFunction}
\pmcreated{2013-03-22 11:53:03}
\pmmodified{2013-03-22 11:53:03}
\pmowner{akrowne}{2}
\pmmodifier{akrowne}{2}
\pmtitle{truth function}
\pmrecord{8}{30483}
\pmprivacy{1}
\pmauthor{akrowne}{2}
\pmtype{Definition}
\pmcomment{trigger rebuild}
\pmclassification{msc}{03B05}

\endmetadata

\usepackage{amssymb}
\usepackage{amsmath}
\usepackage{amsfonts}
\usepackage{graphicx}
%%%%\usepackage{xypic}
\begin{document}
A \emph{truth function} is a function that returns one of two values, one of which is interpreted as ``true,'' and the other which is interpreted as ``false''.  Typically either ``T'' and ``F'' are used, or ``1'' and ``0'', respectively.  Using the latter, we can write $$f : \{0,1\}^n \rightarrow \{0,1\}$$
defines a truth function $f$. That is, $f$ is a mapping from any number ($n$) of true/false (0 or 1) values to a single value, which is 0 or 1.
%%%%%
%%%%%
%%%%%
%%%%%
\end{document}
