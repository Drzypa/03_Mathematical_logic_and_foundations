\documentclass[12pt]{article}
\usepackage{pmmeta}
\pmcanonicalname{LogicalImplication}
\pmcreated{2013-03-22 17:55:15}
\pmmodified{2013-03-22 17:55:15}
\pmowner{Jon Awbrey}{15246}
\pmmodifier{Jon Awbrey}{15246}
\pmtitle{logical implication}
\pmrecord{24}{40413}
\pmprivacy{1}
\pmauthor{Jon Awbrey}{15246}
\pmtype{Definition}
\pmcomment{trigger rebuild}
\pmclassification{msc}{03F03}
\pmclassification{msc}{03C05}
\pmclassification{msc}{03B22}
\pmclassification{msc}{03B05}
\pmsynonym{material implication}{LogicalImplication}
\pmrelated{Implication}
\pmrelated{LogicalConnective}
\pmdefines{material conditional}
\pmdefines{antecedent}
\pmdefines{consequent}
\pmdefines{conditional}
\pmdefines{consequence}

\endmetadata

% this is the default PlanetMath preamble.  as your knowledge
% of TeX increases, you will probably want to edit this, but
% it should be fine as is for beginners.

% almost certainly you want these
\usepackage{amssymb}
\usepackage{amsmath}
\usepackage{amsfonts}

% used for TeXing text within eps files
%\usepackage{psfrag}
% need this for including graphics (\includegraphics)
%\usepackage{graphicx}
% for neatly defining theorems and propositions
%\usepackage{amsthm}
% making logically defined graphics
%%%\usepackage{xypic}

% there are many more packages, add them here as you need them

% define commands here

\begin{document}
\PMlinkescapephrase{adapted}
\PMlinkescapephrase{Adapted}
\PMlinkescapephrase{area}
\PMlinkescapephrase{Area}
\PMlinkescapephrase{complete}
\PMlinkescapephrase{Complete}
\PMlinkescapephrase{completes}
\PMlinkescapephrase{Completes}
\PMlinkescapephrase{connection}
\PMlinkescapephrase{Connection}
\PMlinkescapephrase{connections}
\PMlinkescapephrase{Connections}
\PMlinkescapephrase{derivation}
\PMlinkescapephrase{Derivation}
\PMlinkescapephrase{even}
\PMlinkescapephrase{Even}
\PMlinkescapephrase{field}
\PMlinkescapephrase{Field}
\PMlinkescapephrase{focus}
\PMlinkescapephrase{Focus}
\PMlinkescapephrase{holder}
\PMlinkescapephrase{Holder}
\PMlinkescapephrase{holders}
\PMlinkescapephrase{Holders}
\PMlinkescapephrase{interpretation}
\PMlinkescapephrase{Interpretation}
\PMlinkescapephrase{ma}
\PMlinkescapephrase{MA}
\PMlinkescapephrase{object}
\PMlinkescapephrase{Object}
\PMlinkescapephrase{objects}
\PMlinkescapephrase{Objects}
\PMlinkescapephrase{place}
\PMlinkescapephrase{Place}
\PMlinkescapephrase{places}
\PMlinkescapephrase{Places}
\PMlinkescapephrase{point}
\PMlinkescapephrase{Point}
\PMlinkescapephrase{potential}
\PMlinkescapephrase{Potential}
\PMlinkescapephrase{primary}
\PMlinkescapephrase{Primary}
\PMlinkescapephrase{series}
\PMlinkescapephrase{Series}
\PMlinkescapephrase{source}
\PMlinkescapephrase{Source}

\section{Short version}

\textbf{Logical implication} is an operation on two logical values, typically the values of two \PMlinkname{propositions}{PropositionalCalculus}, that produces a value of false just in case the first operand is true and the second operand is false.

The truth table for the logical implication operation that is written as $p \Rightarrow q$ and read as $``p\ \operatorname{implies}\ q",$ also written as $p \rightarrow q$ and read as $``\operatorname{if}\ p\ \operatorname{then}\ q",$ is as follows:

\begin{quote}\begin{tabular}{|c|c|c|}
\multicolumn{3}{c}{Logical Implication} \\
\hline
$p$ & $q$ & $p \Rightarrow q$ \\
\hline\hline
F & F & T \\
F & T & T \\
T & F & F \\
T & T & T \\
\hline
\end{tabular}\end{quote}

\section{Long version}

The mathematical objects that inform our capacity for logical reasoning are easier to describe in a straightforward way than it is to reconcile their traditional accounts.  Still, some discussion of the language that occurs in the literature cannot be avoided.

The concept of \textbf{logical implication} encompasses a specific logical function, a specific logical relation, and the various symbols that are used to denote this function and this relation.  In order to define the specific function, relation, and symbols in question it is first necessary to establish a few ideas about the connections among them.

Close approximations to the concept of logical implication are expressed in ordinary language by means of linguistic forms like the following:

\begin{itemize}
\item
$p$ implies $q$.
\item
If $p$ then $q$.
\end{itemize}

Here $p$ and $q$ are propositional variables that stand for any propositions in a given language.  In a statement of the form ``if $p$ then $q$", the first term, $p$, is called the \textit{antecedent} and the second term, $q$, is called the \textit{consequent}, while the statement as a whole is called either the \textit{conditional} or the \textit{consequence}.  Assuming that the conditional statement is true, then the truth of the antecedent is a \textit{sufficient condition} for the truth of the consequent, while the truth of the consequent is a \textit{necessary condition} for the truth of the antecedent.

Many writers draw a technical distinction between the form ``$p$ implies $q$" and the form ``if $p$ then $q$".  In this view, writing ``$p$ implies $q$" asserts the existence of a certain relation between the logical value of $p$ and the logical value of $q$ while writing ``if $p$ then $q$" simply forms a compound sentence whose logical value is a function of the logical values of $p$ and $q$.  Notice that a relation is a mathematical object while a sentence, whether open or closed, is a syntactic form that exists in the domain of \PMlinkname{signs}{SignRelation}.

Two factors enter at cross-purposes to our understanding at this point, and they have historically been the source of many divergent thinkers talking skew to each other on this score.

\begin{itemize}
\item
There is the contrast between ``object" and ``sign".  Objects include among their number abstract, formal, and mathematical objects.  Signs include the full variety of syntactic forms in general.  For some thinkers, objects are the primary concerns and language merely accessory to them.  For other thinkers, signs are the primary business and objects always mediated by them, with formal objects being perhaps the mere illusions that are produced in our minds by forms of words.
\item
There is the contrast between the relational aspect and the functional aspect of the mathematical objects in question.  This issue is less problematic within mathematics itself, it being axiomatic that a binary operation is a ternary relation and because there are standard ways of relating characteristic functions with the sets and relations that are their fibers.
\end{itemize}

\subsection{Definition}

The concept of logical implication is associated with an operation on two logical values, typically the values of two propositions, that produces a value of false just in case the first operand is true and the second operand is false.

In the interpretation where $0 = \mathrm{false}$ and $1 = \mathrm{true}$, the truth table associated with the statement ``$p$ implies $q$", symbolized as $p \Rightarrow q$, is as follows:

\begin{quote}\begin{tabular}{|c|c|c|}
\multicolumn{3}{c}{Logical Implication} \\
\hline
$p$ & $q$ & $p \Rightarrow q$ \\
\hline\hline
 0  &  0  &  1 \\
 0  &  1  &  1 \\
 1  &  0  &  0 \\
 1  &  1  &  1 \\
\hline
\end{tabular}\end{quote}

\subsection{Discussion}

The usage of the terms \textit{logical implication} and \textit{material conditional} varies from field to field and even across different contexts of discussion.  One way to minimize the potential confusion is to begin with a focus on the various types of formal objects that are being discussed, of which there are only a few, taking up the variations in language as a secondary matter.

The main formal object under discussion is a logical operation on two logical values, typically the values of two propositions, that produces a value of false just in case the first operand is true and the second operand is false.  By way of a temporary name, the logical operation in question may be written as $\operatorname{cond}(p, q)$, where $p$ and $q$ are logical values.  The truth table associated with this operation is as follows:

\begin{quote}\begin{tabular}{|c|c|c|}
\multicolumn{3}{c}{$\operatorname{cond} : \mathbb{B} \times \mathbb{B} \to \mathbb{B}$} \\
\hline
$p$ & $q$ & $\operatorname{cond}(p, q)$ \\
\hline\hline
 0  &  0  &  1 \\
 0  &  1  &  1 \\
 1  &  0  &  0 \\
 1  &  1  &  1 \\
\hline
\end{tabular}\end{quote}

Some writers draw a firm distinction between the \textit{conditional connective} (the syntactic sign $``\rightarrow"$) and the implication relation (the formal object denoted by the sign $``\Rightarrow"$).  These writers use the phrase ``if--then" for the conditional connective and the term ``implies" for the implication relation.  The difference is sometimes explained by saying that the conditional is the ``contemplated" relation while the implication is the ``asserted" relation.  In most areas of mathematics, the distinction is treated as a variation in the usage of the single sign $``\Rightarrow"$, not requiring two separate signs.  Not all of those who use the sign $``\rightarrow"$ for the conditional connective regard it as a sign that denotes any kind of formal object, but treat it as a so-called \textit{syncategorematic sign}, that is, a sign with a purely syntactic function.  For the sake of clarity and simplicity in the present introduction, it is convenient to use the two-sign notation, but allow the sign $``\rightarrow"$ to denote the boolean function that is associated with the truth table of the material conditional.  These considerations result in the following scheme of notation.

\begin{quote}$\begin{matrix}
p \rightarrow q                  & \quad & \quad & p \Rightarrow q        \\
\mathrm{if}\ p\ \mathrm{then}\ q & \quad & \quad & p\ \mathrm{implies}\ q \\
\end{matrix}$\end{quote}

Let $\mathbb{B} = \{ 0, 1 \}$, where 0 is interpreted as the logical value $\mathrm{false}$ and 1 is interpreted as the logical value $\mathrm{true}$.  The truth table shows the ordered triples of a triadic relation $L \subseteq \mathbb{B} \times \mathbb{B} \times \mathbb{B}$ that is defined as follows:

\begin{quote}
$L = \{ (p, q, r) \in \mathbb{B} \times \mathbb{B} \times \mathbb{B} : \operatorname{cond}(p, q) = r \}$.
\end{quote}

Regarded as a set, this triadic relation is the same thing as the binary operation:

\begin{quote}
$\operatorname{cond} : \mathbb{B} \times \mathbb{B} \to \mathbb{B}$.
\end{quote}

The relationship between $\operatorname{cond}$ and $L$ exemplifies the standard association that exists between any binary operation and its corresponding triadic relation.

The conditional sign $``\rightarrow"$ denotes the same formal object as the function name $\operatorname{cond}$, the only difference being that the first is written infix while the second is written prefix.  Thus we have the following equation:

\begin{quote}
$(p \rightarrow q) = \operatorname{cond}(p, q)$.
\end{quote}

Consider once again the triadic relation $L \subseteq \mathbb{B} \times \mathbb{B} \times \mathbb{B}$ that is defined in the following equivalent fashion:

\begin{quote}
$L = \{ (p, q, \operatorname{cond}(p, q)) : (p, q) \in \mathbb{B} \times \mathbb{B} \}$.
\end{quote}

Associated with the 3-adic relation $L$ is a 2-adic relation $L_{..1} \subseteq \mathbb{B} \times \mathbb{B}$ that is called the \textit{fiber} of $L$ with 1 in the third place.  This object is defined as follows:

\begin{quote}
$L_{..1} := \{ (p, q) \in \mathbb{B} \times \mathbb{B} : (p, q, 1) \in L \}$.
\end{quote}

The same object is achieved in the following way.  Begin with the 2-adic operation:

\begin{quote}
$\operatorname{cond} : \mathbb{B} \times \mathbb{B} \to \mathbb{B}$.
\end{quote}

Form the 2-adic relation that is called the \textit{fiber} of $\operatorname{cond}$ at 1, notated as follows:

\begin{quote}
$\operatorname{cond}^{-1}(1) \subseteq \mathbb{B} \times \mathbb{B}$.
\end{quote}

This object is defined as follows:

\begin{quote}
$\operatorname{cond}^{-1}(1) = \{ (p, q) \in \mathbb{B} \times \mathbb{B} : \operatorname{cond}(p, q) = 1 \}$.
\end{quote}

The implication sign $``\Rightarrow"$ denotes the same formal object as the relation names $``L_{..1}"$ and $``\operatorname{cond}^{-1}(1)"$, the only differences being purely syntactic.  Thus we have the following logical equivalence:

\begin{quote}
$(p \Rightarrow q) \iff (p, q) \in L_{..1} \iff (p, q) \in \operatorname{cond}^{-1}(1)$.
\end{quote}

This completes the derivation of the mathematical objects that are denoted by the signs $``\rightarrow"$ and $``\Rightarrow"$ in this discussion.  It needs to be remembered, though, that not all writers observe this distinction in every context.  Especially in mathematics, where the sign $``\rightarrow"$ is reserved for function notation, it is common to see the sign $``\Rightarrow"$ being used for both concepts.

\section{Bibliography}

\begin{itemize}
\item
Brown, Frank Markham (2003), \textit{Boolean Reasoning : The Logic of Boolean Equations}, 1st edition, Kluwer Academic Publishers, Norwell, MA, 1990.  2nd edition, Dover Publications, Mineola, NY.
\item
Church, Alonzo (1962), ``Logic, formal", pp. 170--181 in Runes (1962).
\item
Church, Alonzo (1996), \textit{Introduction to Mathematical Logic}.  Originally published, Annals of Mathemtics Studies, 1944.  Revised and enlarged edition, Princeton Mathematical Series, 1956.  10th printing, Princeton Landmarks in Mathematics and Physics, Princeton University Press, Princeton, NJ.
\item
van Heijenoort, Jean (1967, ed.), \textit{From Frege To G\"{o}del : A Source Book in Mathematical Logic, 1879--1931}, Harvard University Press, Cambridge, MA.
\item
Quine, Willard Van Orman (1982), \textit{Methods of Logic}, 1st ed. 1950, 2nd ed. 1959, 3rd ed. 1972.  4th edition, Harvard University Press, Cambridge, MA.
\item
Runes, Dagobert D. (1962, ed.), \textit{Dictionary of Philosophy}, Littlefield, Adams, and Company, Totowa, NJ.
\item
Styazhkin, N.I. (1969), \textit{History of Mathematical Logic from Leibniz to Peano}, MIT Press, Cambridge, MA.
\item
Tarski, Alfred (1983), \text{Logic, Semantics, Metamathematics : Papers from 1923 to 1938}, J.H. Woodger (trans.), Oxford University Press, Oxford, UK, 1956.  2nd edition, John Corcoran (ed.), Hackett Publishing, Indianapolis, IN, 1983.
\end{itemize}

\section{Document history}

Portions of the above article are adapted from the following sources under the GNU Free Documentation License, under other applicable licenses, or by permission of the copyright holders.

\begin{itemize}
\item
\PMlinkexternal{Logical implication}{http://www.mywikibiz.com/Logical_implication}, \PMlinkexternal{MyWikiBiz}{http://www.mywikibiz.com/}.
\item
\PMlinkexternal{Logical implication}{http://www.getwiki.net/-Logical_Implication}, \PMlinkexternal{GetWiki}{http://www.getwiki.net/}.
\item
\PMlinkexternal{Logical implication}{http://www.wikinfo.org/index.php/Logical_implication}, \PMlinkexternal{Wikinfo}{http://www.wikinfo.org/}.
\item
\PMlinkexternal{Logical implication}{http://en.wikipedia.org/w/index.php?title=Logical_implication&oldid=77109738}, \PMlinkexternal{Wikipedia}{http://en.wikipedia.org/}.
\end{itemize}

%%%%%
%%%%%
\end{document}
