\documentclass[12pt]{article}
\usepackage{pmmeta}
\pmcanonicalname{A211Definitions}
\pmcreated{2013-11-09 5:44:46}
\pmmodified{2013-11-09 5:44:46}
\pmowner{PMBookProject}{1000683}
\pmmodifier{PMBookProject}{1000683}
\pmtitle{A.2.11 Definitions}
\pmrecord{1}{}
\pmprivacy{1}
\pmauthor{PMBookProject}{1000683}
\pmtype{Feature}
\pmclassification{msc}{03B15}

\endmetadata

\usepackage{xspace}
\usepackage{amssyb}
\usepackage{amsmath}
\usepackage{amsfonts}
\usepackage{amsthm}
\makeatletter
\newcommand{\defeq}{\vcentcolon\equiv}  
\newcommand{\define}[1]{\textbf{#1}}
\newcommand{\isequiv}{\ensuremath{\mathsf{isequiv}}}
\def\lam#1{{\lambda}\@lamarg#1:\@endlamarg\@ifnextchar\bgroup{.\,\lam}{.\,}}
\def\@lamarg#1:#2\@endlamarg{\if\relax\detokenize{#2}\relax #1\else\@lamvar{\@lameatcolon#2},#1\@endlamvar\fi}
\def\@lameatcolon#1:{#1}
\def\@lamvar#1,#2\@endlamvar{(#2\,{:}\,#1)}
\newcommand{\narrowbreak}{}
\newcommand{\UU}{\ensuremath{\mathcal{U}}\xspace}
\newcommand{\vcentcolon}{:\!\!}
\newenvironment{narrowmultline*}{\csname equation*\endcsname}{\csname endequation*\endcsname}
\let\autoref\cref
\makeatother

\begin{document}
\index{definition}%

Although the rules we listed so far allows us to construct everything we need directly, we
would still like to be able to use named constants, such as $\isequiv$, as a matter of
convenience. Informally, we can think of these constants simply as
abbreviations, but the situation is a bit subtler in the formalization.

For example, consider function composition, which takes $f:A\to B$ and
$g:B\to C$ to $g\circ f:A\to C$. Somewhat unexpectedly, to make this work formally, $\circ$ must take as arguments not only $f$ and $g$, but also their types $A$, $B$, $C$:
%
\begin{narrowmultline*}
  {\circ} \defeq \lam{A:\UU_i}{B:\UU_i}{C:\UU_i}
  \narrowbreak
  \lam{g:B\to C}{f:A\to B}{x:A} g(f(x)).
\end{narrowmultline*}
%
From a practical perspective, we do not want to annotate each application of
$\circ$ with $A$, $B$ and $C$, as the are usually quite easily guessed from surrounding information. We would like to simply write $g\circ f$.
Then, strictly speaking, $g \circ f$ is not an abbreviation for $\lam{x : A} g(f(x))$,
because it involves additional \define{implicit arguments} which we want to suppress.
\index{implicit argument}

Inference of implicit arguments, typical ambiguity\index{typical ambiguity} (\PMlinkname{\S 1.3}{13universesandfamilies}),
ensuring that symbols are only defined once, etc., are collectively called
\define{elaboration}. \index{elaboration, in type theory}
Elaboration must take place prior to checking a derivation, and is
thus not usually presented as part of the core type theory. However, it is
essentially impossible to use any implementation of type theory which does not
perform elaboration; see \cite{Coq,norell2007towards} for further discussion.


\end{document}
