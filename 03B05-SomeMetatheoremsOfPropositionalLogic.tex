\documentclass[12pt]{article}
\usepackage{pmmeta}
\pmcanonicalname{SomeMetatheoremsOfPropositionalLogic}
\pmcreated{2013-03-22 19:34:29}
\pmmodified{2013-03-22 19:34:29}
\pmowner{CWoo}{3771}
\pmmodifier{CWoo}{3771}
\pmtitle{some meta-theorems of propositional logic}
\pmrecord{10}{42561}
\pmprivacy{1}
\pmauthor{CWoo}{3771}
\pmtype{Result}
\pmcomment{trigger rebuild}
\pmclassification{msc}{03B05}
\pmdefines{law of syllogism}

\endmetadata

\usepackage{amssymb,amscd}
\usepackage{amsmath}
\usepackage{amsfonts}
\usepackage{mathrsfs}

% used for TeXing text within eps files
%\usepackage{psfrag}
% need this for including graphics (\includegraphics)
%\usepackage{graphicx}
% for neatly defining theorems and propositions
\usepackage{amsthm}
% making logically defined graphics
%%\usepackage{xypic}
\usepackage{pst-plot}

% define commands here
\newcommand*{\abs}[1]{\left\lvert #1\right\rvert}
\newtheorem{prop}{Proposition}
\newtheorem{thm}{Theorem}
\newtheorem{ex}{Example}
\newcommand{\real}{\mathbb{R}}
\newcommand{\pdiff}[2]{\frac{\partial #1}{\partial #2}}
\newcommand{\mpdiff}[3]{\frac{\partial^#1 #2}{\partial #3^#1}}

\begin{document}
Based on the axiom system in \PMlinkname{this entry}{AxiomSystemForPropositionalLogic}, we will prove some meta-theorems of propositional logic.  In the discussion below, $\Delta$ and $\Gamma$ are sets of well-formed formulas (wff's), and $A,B,C,\ldots$ are wff's. 

\begin{enumerate}
\item (Deduction Theorem) $\Delta,A\vdash B$ iff $\Delta\vdash A\to B$.
\item (Proof by Contradiction) $\Delta,A\vdash \perp$ iff $\Delta \vdash \neg A$.
\item (Proof by Contrapositive) $\Delta,A\vdash \neg B$ iff $\Delta, B\vdash \neg A$.
\item (Law of Syllogism) If $\Delta \vdash A\to B$ and $\Gamma \vdash B\to C$, then $\Delta, \Gamma \vdash A\to C$.
\item $\Delta \vdash A$ and $\Delta \vdash B$ iff $\Delta \vdash A\land B$.
\item $\Delta \vdash A\leftrightarrow B$ iff $\Delta, A \vdash B$ and $\Delta, B \vdash A$.
\item If $\Delta \vdash A\leftrightarrow B$, then $\Delta \vdash B\leftrightarrow A$.
\item If $\Delta \vdash A\leftrightarrow B$ and $\Delta \vdash B\leftrightarrow C$, then $\Delta \vdash A \leftrightarrow C$.
\item $\Delta\vdash A\land B\to C$ iff $\Delta \vdash A \to (B\to C)$.
\item $\Delta \vdash A$ implies $\Delta \vdash B$ iff $\Delta \vdash A\to B$.  This is a useful restatement of the deduction theorem.
\item (Substitution Theorem) If $\vdash B_i \leftrightarrow C_i$, then $\vdash A[\overline{B}/\overline{p}] \leftrightarrow A[\overline{C}/\overline{p}]$.
\item $\Delta \vdash \perp$ iff there is a wff $A$ such that $\Delta \vdash A$ and $\Delta \vdash \neg A$.
\item If $\Delta,A\vdash B$ and $\Delta,\neg A\vdash B$, then $\Delta \vdash B$.
\end{enumerate}

\textbf{Remark}.  The theorem schema $A\to \neg \neg A$ is used in the proofs below.

\begin{proof}  The first three are proved \PMlinkname{here}{DeductionTheoremHoldsForClassicalPropositionalLogic}, and the last three are proved \PMlinkname{here}{SubstitutionTheoremForPropositionalLogic}.  We will prove the rest here, some of which relies on the deduction theorem.

\begin{enumerate}
\setcounter{enumi}{3}
\item
From $\Delta \vdash A\to B$, by the deduction theorem, we have $\Delta, A\vdash B$.  Let $\mathcal{E}_1$ be a deduction of $B$ from $\Delta \cup \lbrace A \rbrace$, and $\mathcal{E}_2$ be a deduction of $B\to C$ from $\Gamma$, then $$\mathcal{E}_1, \mathcal{E}_2, C$$ is a deduction of $C$ from $\Delta\cup \lbrace A\rbrace \cup \Gamma$, so $\Delta, A, \Gamma \vdash C$, and by the deduction theorem again, we get $\Delta, \Gamma \vdash A\to C$.
\item
$(\Rightarrow)$.  Since $A\land B$ is $\neg (A\to \neg B)$, by the deduction theorem, it is enough to show $\Delta, A\to \neg B \vdash \perp$.  Suppose $\mathcal{E}_1$ is a deduction of $A$ from $\Delta$ and $\mathcal{E}_2$ is a deduction of $B$ from $\Delta$, then $$\mathcal{E}_1, \mathcal{E}_2, A\to \neg B, \neg B, \perp$$ is a deduction of $\perp$ from $\Delta \cup \lbrace A\to \neg B\rbrace$.

$(\Leftarrow)$.  We first show that $\Delta \vdash B$.  Now, $\neg B \to (A\to \neg B)$ is an axiom and $\vdash (A\to \neg B) \to \neg \neg (A\to \neg B)$ is a theorem, $\vdash \neg B \to \neg \neg (A\to \neg B)$, so that by modus ponens, $\vdash \neg (A\to \neg B)\to B$, using axiom schema $(\neg C\to \neg D)\to (D\to C)$.  Since by assumption $\Delta \vdash \neg (A\to \neg B)$, by modus ponens again, we get $\Delta \vdash B$.

We next show that $\Delta \vdash A$.  From the deduction $A,A\to \perp, \perp$, we have $A,\neg A \vdash \perp$, so certainly $\Delta, \neg A, A, B \vdash \perp$.  By three applications of the deduction theorem, we get $\Delta \vdash \neg A \to (A\to \neg B)$.  By theorem $(A\to \neg B) \to \neg \neg (A\to \neg B)$, $\Delta \vdash \neg A \to \neg \neg (A\to \neg B)$.  By axiom schema $(\neg C \to \neg D)\to (D\to C)$ and modus ponens, we get $\Delta \vdash \neg (A\to \neg B)\to A$.  Since $\Delta \vdash \neg A\to \neg B$ by assumption, $\Delta \to A$ as a result.
\item
$\Delta \vdash A\leftrightarrow B$ iff $\Delta \vdash A\to B$ and $\Delta \vdash B\to A$ iff $\Delta, A \vdash B$ and $\Delta, B \vdash A$.
\item
Apply 6 to $\Delta \vdash A\to B$ and $\Delta \vdash B\to A$.
\item
Apply 5 and 6.
\item
Since $\Delta,A\vdash B\to A\land B$ by the theorem schema $\vdash A\to (B\to A\land B)$, together with 
$\Delta\vdash A\land B\to C$, we have $\Delta,A\vdash B\to C$ by law of syllogism, or equivalently $\Delta \vdash A\to (B\to C)$, by the deduction theorem.  Conversely, $\Delta, A\vdash B\to C$ and theorem schema $A\land B\to B$ result in $\Delta, A\vdash A\land B\to C$ by law of syllogism.  So $\Delta \vdash A\to (A\land B\to C)$ by the deduction theorem.  But $A\land B\to A$ is a theorem schema, $\Delta \vdash A\land B\to (A\land B\to C)$, and therefore $\Delta \vdash A\land B\to C$ by the theorem schema $(X\to (X\to Y)) \leftrightarrow (X\to Y)$.
\item
Assume the former.  Then a deduction of $B$ from $\Delta$ may or may not contain $A$.  In either case, $\Delta, A \vdash B$, so $\Delta \vdash A\to B$ by the deduction theorem.  Next, assume the later.  Let $\mathcal{E}_1$ be a deduction of $A\to B$ from $\Delta$.  Then if $\mathcal{E}_2$ is a deduction of $A$ from $\Delta$, then $\mathcal{E}_1, \mathcal{E}_2, B$ is a deduction of $B$ from $\Delta$, and therefore $\Delta \vdash B$.
\end{enumerate}
To see the last meta-theorem implies the deduction theorem, assume $\Delta, A \vdash B$.  Suppose $\Delta \vdash A$.  Let $\mathcal{E}_1$ be a deduction of $A$ from $\Delta$, and $\mathcal{E}_2$ a deduction of $B$ from $\Delta\cup \lbrace A \rbrace$.  Then $\mathcal{E}_1,\mathcal{E}_2$ is a deduction of $B$ from $\Delta$.  So $\Delta \vdash B$.  As a result $\Delta A\to B$ by the statement of the meta-theorem.
\end{proof}

\textbf{Remark}.  Meta-theorems 7 and 8, together with the theorem schema $A\leftrightarrow A$, show that $\leftrightarrow$ defines an equivalence relation on the set of all wff's of propositional logic.  Formally, for any two wff's $A,B$, if we define $A\sim B$ iff $\vdash A\leftrightarrow B$, then $\sim$ is an equivalence relation.

%%%%%
%%%%%
\end{document}
