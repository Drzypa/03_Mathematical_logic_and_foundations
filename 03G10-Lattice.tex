\documentclass[12pt]{article}
\usepackage{pmmeta}
\pmcanonicalname{Lattice}
\pmcreated{2013-03-22 12:27:16}
\pmmodified{2013-03-22 12:27:16}
\pmowner{mps}{409}
\pmmodifier{mps}{409}
\pmtitle{lattice}
\pmrecord{25}{32593}
\pmprivacy{1}
\pmauthor{mps}{409}
\pmtype{Definition}
\pmcomment{trigger rebuild}
\pmclassification{msc}{03G10}
\pmclassification{msc}{06B99}
\pmsynonym{absorptive}{Lattice}
\pmrelated{Semilattice}
\pmrelated{DistributiveLattice}
\pmrelated{ASemilatticeIsACommutativeBand}
\pmrelated{PartitionLattice}
\pmrelated{AlgebraicCategoryOfLMnLogicAlgebras}
\pmrelated{ClosedSublattice}
\pmrelated{GeneralizedToposesTopoiWithManyValuedLogicSubobjectClassifiers}
\pmrelated{HeytingAlgebra}
\pmdefines{sublattice}
\pmdefines{absorption}

% this is the default PlanetMath preamble.  as your knowledge
% of TeX increases, you will probably want to edit this, but
% it should be fine as is for beginners.

% almost certainly you want these
\usepackage{amssymb}
\usepackage{amsmath}
\usepackage{amsfonts}

% used for TeXing text within eps files
%\usepackage{psfrag}
% need this for including graphics (\includegraphics)
%\usepackage{graphicx}
% for neatly defining theorems and propositions
%\usepackage{amsthm}
% making logically defined graphics
%%\usepackage{xypic} 

% there are many more packages, add them here as you need them

% define commands here
\begin{document}
\PMlinkescapeword{properties}
\PMlinkescapeword{equivalent}
\PMlinkescapeword{classes}
\PMlinkescapeword{distributive}
A \emph{lattice} is any poset $L$ in which any two elements $x$ and $y$ have a least upper bound, $x\lor y$, and a greatest lower bound, $x\land y$.  The operation $\land$ is called \emph{meet}, and the operation $\lor$ is called \emph{join}.  In some literature, $L$ is required to be non-empty.

A \emph{sublattice} of $L$ is a subposet of $L$ which is a lattice, that is, which is closed under the operations $\land$ and $\lor$ as defined in $L$.

The operations of meet and join are idempotent, commutative, associative, and absorptive: 
\[x\land (y\lor x)=x\mbox{ and }x\lor (y\land x)=x. \] 
Thus a lattice is a commutative band with either operation.  The partial order relation can be recovered from meet and join by 
defining 
\[x\le y \text{\ if and only if\ } x\land y = x.\]
Once $\le$ is defined, it is not hard to see that $x\le y$ iff $x\lor y=y$ as well (one direction goes like: $x\lor y= (x\land y)\lor y=y\lor (x\land y)=y\lor (y\land x)=y$, while the other direction is the dual of the first).

Conspicuously absent from the above list of properties is \PMlinkname{distributivity}{DistributiveLattice}.  While many nice lattices, such as face lattices of polytopes, are distributive, there are also important classes of lattices, such as \PMlinkname{partition lattices}{PartitionLattice}, that are usually not distributive.

Lattices, like posets, can be visualized by diagrams called Hasse diagrams.  Below are two diagrams, both posets.  The one on the left is a lattice, while the one on the right is not:
$$
\entrymodifiers={[o]} \xymatrix @!=1pt {
& & \bullet \ar@{-}[ld] \ar@{-}[rd] & & \\
& \bullet \ar@{-}[ld] \ar@{-}[rd] & & \bullet \ar@{-}[ld] \ar@{-}[rd] & \\
\bullet \ar@{-}[rd] & & \bullet \ar@{-}[ld] \ar@{-}[rd] & & \bullet \ar@{-}[ld] \\
& \bullet \ar@{-}[rd] & & \bullet \ar@{-}[ld] & \\
& & \bullet & & }
\hspace{2cm}
\entrymodifiers={[o]} \xymatrix @!=1pt {
& & \bullet \ar@{-}[ld] \ar@{-}[rd] & & \\
& \bullet \ar@{-}[ld] \ar@{-}[rd] \ar@{-}[d] & & \bullet \ar@{-}[ld] \ar@{-}[rd] \ar@{-}[d] & \\
\bullet \ar@{-}[rd] & \ar@{-}[d] & \ar@{-}[ld] \ar@{-}[rd] & \ar@{-}[d] & \bullet \ar@{-}[ld] \\
& \bullet \ar@{-}[rd] & & \bullet \ar@{-}[ld] & \\
& & \bullet & & }
$$
The vertices of a lattice diagram can also be labelled, so the lattice diagram looks like
$$
\xymatrix @!=1pt {
& & a \ar@{-}[ld] \ar@{-}[rd] & & \\
& b \ar@{-}[ld] \ar@{-}[rd] & & c \ar@{-}[ld] \ar@{-}[rd] & \\
d \ar@{-}[rd] & & e \ar@{-}[ld] \ar@{-}[rd] & & f \ar@{-}[ld] \\
& g \ar@{-}[rd] & & h \ar@{-}[ld] & \\
& & i & & }
$$

\textbf{Remark}.  Alternatively, a lattice can be defined as an algebraic system.  Please see the link below for details.
%%%%%
%%%%%
\end{document}
