\documentclass[12pt]{article}
\usepackage{pmmeta}
\pmcanonicalname{PartialAlgebraicSystem}
\pmcreated{2013-03-22 18:42:10}
\pmmodified{2013-03-22 18:42:10}
\pmowner{CWoo}{3771}
\pmmodifier{CWoo}{3771}
\pmtitle{partial algebraic system}
\pmrecord{28}{41465}
\pmprivacy{1}
\pmauthor{CWoo}{3771}
\pmtype{Definition}
\pmcomment{trigger rebuild}
\pmclassification{msc}{03E99}
\pmclassification{msc}{08A55}
\pmclassification{msc}{08A62}
\pmsynonym{partial operator}{PartialAlgebraicSystem}
\pmsynonym{partial algebra}{PartialAlgebraicSystem}
\pmrelated{RelationalSystem}
\pmdefines{partial operation}
\pmdefines{partial groupoid}

\usepackage{amssymb,amscd}
\usepackage{amsmath}
\usepackage{amsfonts}
\usepackage{mathrsfs}

% used for TeXing text within eps files
%\usepackage{psfrag}
% need this for including graphics (\includegraphics)
%\usepackage{graphicx}
% for neatly defining theorems and propositions
\usepackage{amsthm}
% making logically defined graphics
%%\usepackage{xypic}
\usepackage{pst-plot}

% define commands here
\newcommand*{\abs}[1]{\left\lvert #1\right\rvert}
\newtheorem{prop}{Proposition}
\newtheorem{thm}{Theorem}
\newtheorem{ex}{Example}
\newcommand{\real}{\mathbb{R}}
\newcommand{\pdiff}[2]{\frac{\partial #1}{\partial #2}}
\newcommand{\mpdiff}[3]{\frac{\partial^#1 #2}{\partial #3^#1}}
\begin{document}
Let $\lambda$ be a cardinal.  A partial function $f: A^{\lambda} \to A$ is called a \emph{partial operation} on $A$.  $\lambda$ is called the arity of $f$.  When $\lambda$ is finite, $f$ is said to be \emph{finitary}.  Otherwise, it is \emph{infinitary}.  A nullary partial operation is an element of $A$ and is called a constant.

\textbf{Definition}.  A \emph{partial algebraic system} (or \emph{partial algebra} for short) is defined as a pair $(A,O)$, where $A$ is a set, usually non-empty, and called the underlying set of the algebra, and $O$ is a set of finitary partial operations on $A$.  The partial algebra $(A,O)$ is sometimes denoted by $\boldsymbol{A}$.

Partial algebraic systems sit between algebraic systems and relational systems; they are generalizations of algebraic systems, but special cases of relational systems.

The \emph{type} of a partial algebra is defined exactly the same way as that of an algebra.  When we speak of a partial algebra $\boldsymbol{A}$ of type $\tau$, we typically mean that $\boldsymbol{A}$ is \emph{proper}, meaning that the partial operation $f_{\boldsymbol{A}}$ is non-empty for every function symbol $f\in \tau$, and if $f$ is a constant symbol, $f_{\boldsymbol{A}} \in A$.

Below is a short list of partial algebras.
\begin{enumerate}
\item
Every algebraic system is automatically a partial algebraic system.
\item
A division ring $(D,\lbrace +\mbox{, }\cdot\mbox{, }-\mbox{, }^{-1}\mbox{, }0\mbox{, }1\rbrace)$ is a prototypical example of a partial algebra that is not an algebra.  It has type $\langle 2,2,1,1,0,0\rangle$.  It is not an algebra because the unary operation $^{-1}$ (multiplicative inverse) is only partial, not defined for $0$.
\item
Let $A$ be the set of all non-negative integers.  Let ``$-$'' be the ordinary subtraction.  Then $(A,\lbrace -\rbrace)$ is a partial algebra.
\item
A \emph{partial groupoid} is a partial algebra of type $\langle 2\rangle$.  In other words, it is a set with a partial binary operation (called the product) on it.  For example, a small category may be viewed as a partial algebra.  The product $ab$ is only defined when the source of $a$ matches with the target of $b$.  Special types of small categories are \PMlinkname{groupoids (category theoretic)}{GroupoidCategoryTheoretic}, and Brandt groupoids, all of which are partial.
\item
A small category can also be thought of as a partial algebra of type $\langle 2,1,1\rangle$, where the two (total) unary operators are the source and target operations.
\end{enumerate}

\textbf{Remark}.  Like algebraic systems, one can define subalgebras, direct products, homomorphisms, as well as congruences in partial algebras.

\begin{thebibliography}{7}
\bibitem{gg} G. Gr\"{a}tzer: {\em Universal Algebra}, 2nd Edition, Springer, New York (1978).
\end{thebibliography}
%%%%%
%%%%%
\end{document}
