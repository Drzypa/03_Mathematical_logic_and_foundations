\documentclass[12pt]{article}
\usepackage{pmmeta}
\pmcanonicalname{Sentence}
\pmcreated{2013-03-22 13:00:24}
\pmmodified{2013-03-22 13:00:24}
\pmowner{Henry}{455}
\pmmodifier{Henry}{455}
\pmtitle{sentence}
\pmrecord{7}{33387}
\pmprivacy{1}
\pmauthor{Henry}{455}
\pmtype{Definition}
\pmcomment{trigger rebuild}
\pmclassification{msc}{03B99}
\pmsynonym{closed formula}{Sentence}
\pmdefines{open formula}
\pmdefines{universal closure}
\pmdefines{existential closure}

\endmetadata

% this is the default PlanetMath preamble.  as your knowledge
% of TeX increases, you will probably want to edit this, but
% it should be fine as is for beginners.

% almost certainly you want these
\usepackage{amssymb}
\usepackage{amsmath}
\usepackage{amsfonts}

% used for TeXing text within eps files
%\usepackage{psfrag}
% need this for including graphics (\includegraphics)
%\usepackage{graphicx}
% for neatly defining theorems and propositions
%\usepackage{amsthm}
% making logically defined graphics
%%%\usepackage{xypic}

% there are many more packages, add them here as you need them

% define commands here
%\PMlinkescapeword{theory}
\begin{document}
A \emph{sentence} is a formula with no free variables.

Simple examples include:

\begin{itemize}
\item
$$\forall x\exists y [x<y]$$
\item
$$\exists z [z+7-43=0]$$
\item
$$1+2<2+3$$
\end{itemize}

Note that the last sentence contains no variables.

A sentence is also called a \emph{closed formula}.  A formula that is not a sentence is called an \emph{open formula}.

The following formula is open:

$$x+2=3$$

\textbf{Remark}.  In first-order logic, the main difference between a sentence and an open formula, semantically, is that a sentence has a definite truth value, whereas the truth value of an open formula may vary, depending on the interpretations of the free variables occurring in the formula.  In the open formula above, if $x$ were $1$, then the formula is true.  Otherwise, it is false.

Every open formula may be converted into a sentence by placing quantifiers in front of it.  Given a formula $\varphi$, the \emph{universal closure} of $\varphi$ is the sentence $$\forall x_1 \forall x_2 \cdots \forall x_n \varphi$$
where $\lbrace x_1,\ldots, x_n\rbrace$ is the set of all free variables occurring in $\varphi$.

The \emph{existential closure} of a formula $\varphi$ may be defined similarly.

For example, the universal closure of $x+2=3$ is $$\forall x [x+2=3],$$ and its existential closure is $$\exists x [x+2=3].$$  Note that the first sentence is false, while the second is true.
%%%%%
%%%%%
\end{document}
