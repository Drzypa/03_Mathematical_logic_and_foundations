\documentclass[12pt]{article}
\usepackage{pmmeta}
\pmcanonicalname{LimitAlongAFilter}
\pmcreated{2013-03-22 15:32:20}
\pmmodified{2013-03-22 15:32:20}
\pmowner{kompik}{10588}
\pmmodifier{kompik}{10588}
\pmtitle{limit along a filter}
\pmrecord{6}{37433}
\pmprivacy{1}
\pmauthor{kompik}{10588}
\pmtype{Definition}
\pmcomment{trigger rebuild}
\pmclassification{msc}{03E99}
\pmclassification{msc}{40A05}
\pmsynonym{limit along filter}{LimitAlongAFilter}
\pmsynonym{F-limit}{LimitAlongAFilter}
\pmrelated{Filter}
\pmdefines{limit along a filter}

% this is the default PlanetMath preamble. as your knowledge
% of TeX increases, you will probably want to edit this, but
% it should be fine as is for beginners.

% almost certainly you want these
\usepackage{amssymb}
\usepackage{amsmath}
\usepackage{amsfonts}
\usepackage{amsthm}

% used for TeXing text within eps files
%\usepackage{psfrag}
% need this for including graphics (\includegraphics)
%\usepackage{graphicx}
% for neatly defining theorems and propositions
%
% making logically defined graphics
%%%\usepackage{xypic}

% there are many more packages, add them here as you need them

% define commands here

\newcommand{\sR}[0]{\mathbb{R}}
\newcommand{\sC}[0]{\mathbb{C}}
\newcommand{\sN}[0]{\mathbb{N}}
\newcommand{\sZ}[0]{\mathbb{Z}}
\newcommand{\N}[0]{\mathbb{N}}


\usepackage{bbm}
\newcommand{\Z}{\mathbbmss{Z}}
\newcommand{\C}{\mathbbmss{C}}
\newcommand{\R}{\mathbbmss{R}}
\newcommand{\Q}{\mathbbmss{Q}}



\newcommand*{\norm}[1]{\lVert #1 \rVert}
\newcommand*{\abs}[1]{| #1 |}

\newcommand{\Map}[3]{#1:#2\to#3}
\newcommand{\Emb}[3]{#1:#2\hookrightarrow#3}
\newcommand{\Mor}[3]{#2\overset{#1}\to#3}

\newcommand{\Cat}[1]{\mathcal{#1}}
\newcommand{\Kat}[1]{\mathbf{#1}}
\newcommand{\Func}[3]{\Map{#1}{\Cat{#2}}{\Cat{#3}}}
\newcommand{\Funk}[3]{\Map{#1}{\Kat{#2}}{\Kat{#3}}}

\newcommand{\intrv}[2]{\langle #1,#2 \rangle}

\newcommand{\vp}{\varphi}
\newcommand{\ve}{\varepsilon}

\newcommand{\Invimg}[2]{\inv{#1}(#2)}
\newcommand{\Img}[2]{#1[#2]}
\newcommand{\ol}[1]{\overline{#1}}
\newcommand{\ul}[1]{\underline{#1}}
\newcommand{\inv}[1]{#1^{-1}}
\newcommand{\limti}[1]{\lim\limits_{#1\to\infty}}

\newcommand{\Ra}{\Rightarrow}

%fonts
\newcommand{\mc}{\mathcal}

%shortcuts
\newcommand{\Ob}{\mathrm{Ob}}
\newcommand{\Hom}{\mathrm{hom}}
\newcommand{\homs}[2]{\mathrm{hom(}{#1},{#2}\mathrm )}
\newcommand{\Eq}{\mathrm{Eq}}
\newcommand{\Coeq}{\mathrm{Coeq}}

%theorems
\newtheorem{THM}{Theorem}
\newtheorem{DEF}{Definition}
\newtheorem{PROP}{Proposition}
\newtheorem{LM}{Lemma}
\newtheorem{COR}{Corollary}
\newtheorem{EXA}{Example}

%categories
\newcommand{\Top}{\Kat{Top}}
\newcommand{\Haus}{\Kat{Haus}}
\newcommand{\Set}{\Kat{Set}}

%diagrams
\newcommand{\UnimorCD}[6]{
\xymatrix{ {#1} \ar[r]^{#2} \ar[rd]_{#4}& {#3} \ar@{-->}[d]^{#5} \\
& {#6} } }

\newcommand{\RovnostrCD}[6]{
\xymatrix@C=10pt@R=17pt{
& {#1} \ar[ld]_{#2} \ar[rd]^{#3} \\
{#4} \ar[rr]_{#5} && {#6} } }

\newcommand{\RovnostrCDii}[6]{
\xymatrix@C=10pt@R=17pt{
{#1} \ar[rr]^{#2} \ar[rd]_{#4}&& {#3} \ar[ld]^{#5} \\
& {#6} } }

\newcommand{\RovnostrCDiiop}[6]{
\xymatrix@C=10pt@R=17pt{
{#1}  && {#3} \ar[ll]_{#2}  \\
& {#6} \ar[lu]^{#4} \ar[ru]_{#5} } }

\newcommand{\StvorecCD}[8]{
\xymatrix{
{#1} \ar[r]^{#2} \ar[d]_{#4} & {#3} \ar[d]^{#5} \\
{#6} \ar[r]_{#7} & {#8}
}
}

\newcommand{\TriangCD}[6]{
\xymatrix{ {#1} \ar[r]^{#2} \ar[rd]_{#4}&
{#3} \ar[d]^{#5} \\
& {#6} } }

\newcommand{\F}{\mc F}
\newcommand{\Flim}{\operatorname{\F\text{-}\lim}}
\begin{document}
\begin{DEF}
Let $\F$ be a filter on $\N$ and $(x_n)$ be a sequence in a metric
space $(X,d)$. We say that $L$ is the \emph{$\F$-limit} of $(x_n)$
if
$$A(\ve)=\{n\in\N: d(x_n,L)<\ve\} \in \F$$
for every $\ve>0$.
\end{DEF}

The name \emph{\PMlinkescapetext{limit} along $\F$} is used as well.

In the usual definition of limit one requires all sets $A(\ve)$ to
be cofinite - i.e.~they have to be large. In the definition of
$\F$-limit we simply choose which sets are considered to be large
- namely the sets from the filter $\F$.

\subsection*{Remarks}

This notion shouldn't be confused with the notion of \PMlinkname{limit of a
filter}{filter} defined in general topology.

Let us note that the same notion is defined by some authors using
the dual notion of ideal instead of filter and, of course, all
results can be reformulated using ideals as well. For this
approach see e.g.~\cite{ksw}.

\subsection*{Examples}

Limit along the Fr\'echet filter, which consist of complements of
finite sets, is the usual limit of a sequence.

Limit of the sequence $(x_n)$ along the principal filter
$\F_k=\{A\subseteq\N; k\in A\}$ is $x_k$.

If we put $\F=\{\N\setminus A: d(A)=0\}$, where $d$ denotes the
asymptotic density, then it can be shown that $\F$ is a filter. In
this case $\F$-convergence is known as statistical convergence.

\begin{thebibliography}{1}

\bibitem{agg}
M.~A. Alekseev, L.~{Yu.} Glebsky, and E.~I. Gordon, \emph{On
approximations of
  groups, group actions {and Hopf} algebras}, Journal of Mathematical Sciences
  \textbf{107} (2001), no.~5, 4305--4332.

\bibitem{balste}
B.~Balcar and P.~{\v{S}}t\v{e}p\'anek, \emph{Teorie mno\v{z}in},
Academia,
  Praha, 1986 (Czech).

\bibitem{hrjech}
K.~Hrbacek and T.~Jech, \emph{{Introduction to set theory}},
{Marcel Dekker},
  New York, 1999.

\bibitem{ksw}
P.~Kostyrko, T.~{\v{S}}al{\'a}t, and W.~Wilczy{\'n}ski,
  \emph{{$\mathcal{I}$}-convergence}, Real Anal. Exchange \textbf{26}
  (2000-2001), 669--686.

\end{thebibliography}
%%%%%
%%%%%
\end{document}
