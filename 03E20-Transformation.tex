\documentclass[12pt]{article}
\usepackage{pmmeta}
\pmcanonicalname{Transformation}
\pmcreated{2013-03-22 12:20:19}
\pmmodified{2013-03-22 12:20:19}
\pmowner{rmilson}{146}
\pmmodifier{rmilson}{146}
\pmtitle{transformation}
\pmrecord{6}{31977}
\pmprivacy{1}
\pmauthor{rmilson}{146}
\pmtype{Definition}
\pmcomment{trigger rebuild}
\pmclassification{msc}{03E20}
\pmsynonym{mapping function}{Transformation}
\pmrelated{Invariant}
\pmrelated{Mapping}
\pmrelated{Fixed}

\usepackage{amsmath}
\usepackage{amsfonts}
\usepackage{amssymb}


\newtheorem{proposition}{Proposition}
\begin{document}
Synonym of {\em mapping} and {\em function}.  Often used to refer to mappings where the domain and 
codomain are the same set, i.e. one can compose a transformation with itself.  For example, when one 
speaks of {\em transformation of a space}, one refers to some deformation/rearrangement of that space.
%%%%%
%%%%%
\end{document}
