\documentclass[12pt]{article}
\usepackage{pmmeta}
\pmcanonicalname{DiscontinuityOfCharacteristicFunction}
\pmcreated{2015-02-03 21:23:33}
\pmmodified{2015-02-03 21:23:33}
\pmowner{pahio}{2872}
\pmmodifier{pahio}{2872}
\pmtitle{discontinuity of characteristic function}
\pmrecord{3}{88017}
\pmprivacy{1}
\pmauthor{pahio}{2872}
\pmtype{Theorem}
\pmclassification{msc}{03-00}
\pmclassification{msc}{26-00}
\pmclassification{msc}{26A09}

% this is the default PlanetMath preamble.  as your knowledge
% of TeX increases, you will probably want to edit this, but
% it should be fine as is for beginners.

% almost certainly you want these
\usepackage{amssymb}
\usepackage{amsmath}
\usepackage{amsfonts}

% need this for including graphics (\includegraphics)
\usepackage{graphicx}
% for neatly defining theorems and propositions
\usepackage{amsthm}

% making logically defined graphics
%\usepackage{xypic}
% used for TeXing text within eps files
%\usepackage{psfrag}

% there are many more packages, add them here as you need them

% define commands here

\begin{document}
\textbf{Theorem.}\; For a subset $A$ of $\mathbb{R}^n$, the set of the 
\PMlinkname{discontinuity}{Continuous} 
points of the characteristic function $\chi_A$ is the 
\PMlinkname{boundary}{BoundaryFrontier} of $A$.\\

{\it Proof.}\, Let $a$ be a discontinuity point of $\chi_A$.\, Then any 
\PMlinkname{neighborhood}{Neighborhood} of $a$ 
contains the points $b$ and $c$ such that\, $\chi_A(b) = 1$\, and\, $\chi_A(c) = 0$.\, Thus\, 
$b \in A$\, and\, $c \notin A$,\, whence $a$ is a boundary point of $A$.

If, on the contrary, $a$ is a boundary point of $A$ and $U(a)$ an arbitrary neighborhood of 
$a$, it follows that $U(a)$ contains both points belonging to $A$ and points not belonging to 
$A$.\, So we have in $U(a)$ the points $b$ and $c$ such that\, $\chi_A(b) = 1$\, and\, 
$\chi_A(c) = 0$.\, This means that $\chi_A$ cannot be continuous at the point $a$ (N.B. that
one does not need to know the value $\chi_A(a)$).


\end{document}
