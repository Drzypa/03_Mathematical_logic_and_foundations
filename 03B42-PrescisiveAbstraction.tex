\documentclass[12pt]{article}
\usepackage{pmmeta}
\pmcanonicalname{PrescisiveAbstraction}
\pmcreated{2013-03-22 17:54:03}
\pmmodified{2013-03-22 17:54:03}
\pmowner{Jon Awbrey}{15246}
\pmmodifier{Jon Awbrey}{15246}
\pmtitle{prescisive abstraction}
\pmrecord{5}{40392}
\pmprivacy{1}
\pmauthor{Jon Awbrey}{15246}
\pmtype{Definition}
\pmcomment{trigger rebuild}
\pmclassification{msc}{03B42}
\pmclassification{msc}{03B15}
\pmclassification{msc}{03B22}
\pmclassification{msc}{03B30}
\pmclassification{msc}{03A05}
\pmclassification{msc}{00A30}
\pmsynonym{simple abstraction}{PrescisiveAbstraction}
\pmrelated{HypostaticAbstraction}

% this is the default PlanetMath preamble.  as your knowledge
% of TeX increases, you will probably want to edit this, but
% it should be fine as is for beginners.

% almost certainly you want these
\usepackage{amssymb}
\usepackage{amsmath}
\usepackage{amsfonts}

% used for TeXing text within eps files
%\usepackage{psfrag}
% need this for including graphics (\includegraphics)
%\usepackage{graphicx}
% for neatly defining theorems and propositions
%\usepackage{amsthm}
% making logically defined graphics
%%%\usepackage{xypic}

% there are many more packages, add them here as you need them

% define commands here

\begin{document}
\textbf{Prescisive abstraction} or \textbf{prescision}, variously spelled as \textit{precisive abstraction} and \textit{prescission}, is a formal operation that marks, selects, or singles out one feature of a concrete experience to the disregard of others.

The above definition derives from one given by Charles Sanders Peirce (CP 4.235) in the context of distinguishing two kinds of abstraction, the other being hypostatic abstraction.

\section{References}

\begin{itemize}
\item
Peirce, Charles Sanders (1902), ``The Simplest Mathematics", CP 4.227--323 in \textit{Collected Papers of Charles Sanders Peirce}, vols. 1--6, Charles Hartshorne and Paul Weiss (eds.), vols. 7--8, Arthur W. Burks (ed.), Harvard University Press, Cambridge, MA, 1931--1935, 1958.  Cited as (CP volume.paragraph).
\end{itemize}

%%%%%
%%%%%
\end{document}
