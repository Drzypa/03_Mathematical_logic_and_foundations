\documentclass[12pt]{article}
\usepackage{pmmeta}
\pmcanonicalname{TarskisAxiom}
\pmcreated{2013-03-22 15:37:25}
\pmmodified{2013-03-22 15:37:25}
\pmowner{rspuzio}{6075}
\pmmodifier{rspuzio}{6075}
\pmtitle{Tarski's axiom}
\pmrecord{5}{37548}
\pmprivacy{1}
\pmauthor{rspuzio}{6075}
\pmtype{Definition}
\pmcomment{trigger rebuild}
\pmclassification{msc}{03E30}

% this is the default PlanetMath preamble.  as your knowledge
% of TeX increases, you will probably want to edit this, but
% it should be fine as is for beginners.

% almost certainly you want these
\usepackage{amssymb}
\usepackage{amsmath}
\usepackage{amsfonts}

% used for TeXing text within eps files
%\usepackage{psfrag}
% need this for including graphics (\includegraphics)
%\usepackage{graphicx}
% for neatly defining theorems and propositions
%\usepackage{amsthm}
% making logically defined graphics
%%%\usepackage{xypic}

% there are many more packages, add them here as you need them

% define commands here
\begin{document}
Tarski proposed the following axiom for set theory:

For every set $S$, there exists a set $U$ which enjoys the following properties:

\begin{itemize}
\item $S$ is an element of $U$
\item For every element $X \in U$, every subset of $X$ is also an element of $U$.
\item For every element $X \in U$, the power set of $X$ is also an element of $U$.
\item Every subset of $U$ whose cardinality is less than the cardinality of $U$ is  an element of $U$.
\end{itemize}

This axiom implies the axiom of choice.  It also implies the existence of inaccessible cardinal numbers.
%%%%%
%%%%%
\end{document}
