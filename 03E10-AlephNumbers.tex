\documentclass[12pt]{article}
\usepackage{pmmeta}
\pmcanonicalname{AlephNumbers}
\pmcreated{2013-03-22 14:15:39}
\pmmodified{2013-03-22 14:15:39}
\pmowner{yark}{2760}
\pmmodifier{yark}{2760}
\pmtitle{aleph numbers}
\pmrecord{6}{35710}
\pmprivacy{1}
\pmauthor{yark}{2760}
\pmtype{Definition}
\pmcomment{trigger rebuild}
\pmclassification{msc}{03E10}
\pmsynonym{alephs}{AlephNumbers}
\pmrelated{GeneralizedContinuumHypothesis}
\pmrelated{BethNumbers}

\usepackage{amssymb}
\usepackage{amsmath}
\usepackage{amsfonts}

\def\N{\mathbb{N}}
\def\Q{\mathbb{Q}}
\def\Z{\mathbb{Z}}
\begin{document}
\PMlinkescapeword{alphabet}

The \emph{aleph numbers} are infinite cardinal numbers 
defined by transfinite recursion, as described below.
They are written $\aleph_\alpha$, where $\aleph$ is aleph,
the first letter of the Hebrew alphabet,
and $\alpha$ is an ordinal number.
Sometimes we write $\omega_\alpha$ instead of $\aleph_\alpha$,
usually to emphasise that it is an ordinal.

To start the transfinite recursion, 
we define $\aleph_0$ to be the first infinite ordinal.
This is the cardinality of countably infinite sets, such as $\N$ and $\Q$.
For each ordinal $\alpha$,
the cardinal number $\aleph_{\alpha+1}$ is defined to be 
the least ordinal of cardinality greater than $\aleph_\alpha$.
For each limit ordinal $\delta$, 
we define $\aleph_\delta=\bigcup_{\alpha\in\delta}\aleph_\alpha$.

As a consequence of the \PMlinkname{Well-Ordering Principle}{ZermelosWellOrderingTheorem},
every infinite set is equinumerous with an aleph number.
Every infinite cardinal is therefore an aleph.
More precisely, for every infinite cardinal $\kappa$ there is exactly one ordinal $\alpha$ such that $\kappa=\aleph_\alpha$.
%%%%%
%%%%%
\end{document}
