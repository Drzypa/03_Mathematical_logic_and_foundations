\documentclass[12pt]{article}
\usepackage{pmmeta}
\pmcanonicalname{Invariant}
\pmcreated{2013-03-22 12:26:09}
\pmmodified{2013-03-22 12:26:09}
\pmowner{rmilson}{146}
\pmmodifier{rmilson}{146}
\pmtitle{invariant}
\pmrecord{8}{32504}
\pmprivacy{1}
\pmauthor{rmilson}{146}
\pmtype{Definition}
\pmcomment{trigger rebuild}
\pmclassification{msc}{03E20}
\pmrelated{Transformation}
\pmrelated{InvariantSubspace}
\pmrelated{Fixed}

\endmetadata

\usepackage{amsmath}
\usepackage{amsfonts}
\usepackage{amssymb}

\newcommand{\reals}{\mathbb{R}}
\newcommand{\natnums}{\mathbb{N}}
\newcommand{\cnums}{\mathbb{C}}

\newcommand{\lp}{\left(}
\newcommand{\rp}{\right)}
\newcommand{\lb}{\left[}
\newcommand{\rb}{\right]}

\newcommand{\supth}{^{\text{th}}}


\newtheorem{proposition}{Proposition}
\begin{document}
Let $A$ be a set, and   $T:A\rightarrow A$  a transformation of that
set.  We say that $x\in A$ is {\em an invariant} of $T$ whenever $x$ is
fixed by $T$:
$$T(x)=x.$$
We say that a subset $B\subset A$ is
{\em invariant with respect to $T$} whenever
$$T(B)\subset B.$$ If this is so, the restriction of $T$ 
is a well-defined transformation of the invariant subset:
$$T\Big|_B : B\rightarrow B.$$
The definition generalizes readily to a family of transformations with
common domain
$$T_i : A\rightarrow A,\quad i\in I$$
In this case we say that a subset is invariant, if it is invariant
with respect to all elements of the family.
%%%%%
%%%%%
\end{document}
