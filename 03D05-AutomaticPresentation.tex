\documentclass[12pt]{article}
\usepackage{pmmeta}
\pmcanonicalname{AutomaticPresentation}
\pmcreated{2013-03-22 14:16:57}
\pmmodified{2013-03-22 14:16:57}
\pmowner{mathcam}{2727}
\pmmodifier{mathcam}{2727}
\pmtitle{automatic presentation}
\pmrecord{8}{35736}
\pmprivacy{1}
\pmauthor{mathcam}{2727}
\pmtype{Definition}
\pmcomment{trigger rebuild}
\pmclassification{msc}{03D05}
\pmclassification{msc}{03C57}
\pmsynonym{automatic structure}{AutomaticPresentation}
\pmsynonym{FA presentation}{AutomaticPresentation}
%\pmkeywords{automata}
%\pmkeywords{structure}
%\pmkeywords{computability}
%\pmkeywords{decidable}
\pmrelated{AutomaticGroup}

% this is the default PlanetMath preamble.  as your knowledge
% of TeX increases, you will probably want to edit this, but
% it should be fine as is for beginners.

% almost certainly you want these
\usepackage{amssymb}
\usepackage{amsmath}
\usepackage{amsfonts}

% used for TeXing text within eps files
%\usepackage{psfrag}
% need this for including graphics (\includegraphics)
%\usepackage{graphicx}
% for neatly defining theorems and propositions
%\usepackage{amsthm}
% making logically defined graphics
%%%\usepackage{xypic}

% there are many more packages, add them here as you need them

% define commands here
\begin{document}
Let $S$ be a relational structure (for example, a graph).

$S$ has an \emph{automatic presentation} if (for some alphabet $\Sigma$) there is a language $L\subseteq \Sigma^*$ and a surjective map $f$ from $L$ onto the \PMlinkescapetext{domain (universe}) of $S$ such that
\begin{itemize}
\item $L$ can be checked by a \PMlinkname{finite automaton}{DeterministicFiniteAutomaton} (Membership)
\item The language of all convolutions of $x,y\in L$ where $f(x)=f(y)$ can be checked by a \PMlinkescapetext{finite automaton} (Equality)
\item For all $n$-ary \PMlinkname{relations}{Relation} $R_i$ in $S$, the language of all convolutions of $x_1,x_2,\ldots,x_n\in L$ where $R_i(f(x_1),f(x_2),\ldots,f(x_n))$ can be checked by a \PMlinkescapetext{finite automaton} (\PMlinkescapetext{Relations})
\end{itemize}

The class of languages accepted by finite automata, i.e. regular languages, is closed under operations like union, intersection, complementation etc, and it is decidable whether or not a finite \PMlinkescapetext{automaton} accepts the empty language.  This allows any first order sentence over the structure to be decided - using union for 'and', complementation for 'not' etc., and emptiness for dealing with 'there exists'.  As such, the first order theory of any structure with an automatic presentation is decidable.

Note that wrt group \PMlinkescapetext{theory} this is inspired by, but not \PMlinkescapetext{equivalent} to, the definition of automatic groups.
%%%%%
%%%%%
\end{document}
