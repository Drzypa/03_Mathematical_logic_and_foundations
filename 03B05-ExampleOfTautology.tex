\documentclass[12pt]{article}
\usepackage{pmmeta}
\pmcanonicalname{ExampleOfTautology}
\pmcreated{2013-03-22 15:27:38}
\pmmodified{2013-03-22 15:27:38}
\pmowner{bloftin}{6104}
\pmmodifier{bloftin}{6104}
\pmtitle{example of tautology}
\pmrecord{4}{37311}
\pmprivacy{1}
\pmauthor{bloftin}{6104}
\pmtype{Example}
\pmcomment{trigger rebuild}
\pmclassification{msc}{03B05}
\pmclassification{msc}{03B10}

% this is the default PlanetMath preamble.  as your knowledge
% of TeX increases, you will probably want to edit this, but
% it should be fine as is for beginners.

% almost certainly you want these
\usepackage{amssymb}
\usepackage{amsmath}
\usepackage{amsfonts}

% used for TeXing text within eps files
%\usepackage{psfrag}
% need this for including graphics (\includegraphics)
%\usepackage{graphicx}
% for neatly defining theorems and propositions
%\usepackage{amsthm}
% making logically defined graphics
%%%\usepackage{xypic}

% there are many more packages, add them here as you need them

% define commands here
\begin{document}
An example of a tautology and how to test it is given in the truth table below for the statement $(P \vee Q) \vee ( \neg P \wedge \neg Q) $


\begin{center}
\begin{tabular}{ccccc}
$P$ & $Q$ & $\neg P \wedge \neg Q$ & $P \vee Q$ & $(P \vee Q) \vee ( \neg P \wedge \neg Q)$ \\
\hline
F & F & T & F & T \\
F & T & F & T & T \\
T & F & F & T & T \\
T & T & F & T & T
\end{tabular}
\end{center}


Thus for whatever truth values P and Q take on, the statement always comes out true as shown in the last coloumn of the truth table.
%%%%%
%%%%%
\end{document}
