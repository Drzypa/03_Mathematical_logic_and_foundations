\documentclass[12pt]{article}
\usepackage{pmmeta}
\pmcanonicalname{BertrandsConjecture}
\pmcreated{2013-03-22 11:46:56}
\pmmodified{2013-03-22 11:46:56}
\pmowner{KimJ}{5}
\pmmodifier{KimJ}{5}
\pmtitle{Bertrand's conjecture}
\pmrecord{13}{30251}
\pmprivacy{1}
\pmauthor{KimJ}{5}
\pmtype{Theorem}
\pmcomment{trigger rebuild}
\pmclassification{msc}{03B05}
\pmclassification{msc}{03B10}
\pmclassification{msc}{11N05}
\pmsynonym{Bertrand's postulate}{BertrandsConjecture}
%\pmkeywords{number theory}

\usepackage{amssymb}
\usepackage{amsmath}
\usepackage{amsfonts}
\usepackage{graphicx}
%%%%\usepackage{xypic}
\begin{document}
Bertrand conjectured that for every positive integer $n > 1$, there exists at least one prime $p$ satisfying $n < p < 2n$. This result was proven in 1850 by Chebyshev, but the phrase \PMlinkescapeword{name} ``Bertrand's Conjecture'' remains in the literature.
%%%%%
%%%%%
%%%%%
%%%%%
\end{document}
