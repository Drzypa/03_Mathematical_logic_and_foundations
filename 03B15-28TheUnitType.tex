\documentclass[12pt]{article}
\usepackage{pmmeta}
\pmcanonicalname{28TheUnitType}
\pmcreated{2013-11-14 3:23:33}
\pmmodified{2013-11-14 3:23:33}
\pmowner{PMBookProject}{1000683}
\pmmodifier{rspuzio}{6075}
\pmtitle{2.8 The unit type}
\pmrecord{6}{87614}
\pmprivacy{1}
\pmauthor{PMBookProject}{6075}
\pmtype{Feature}
\pmclassification{msc}{03B15}

\endmetadata

\usepackage{xspace}
\usepackage{amssyb}
\usepackage{amsmath}
\usepackage{amsfonts}
\usepackage{amsthm}
\newcommand{\eqv}[2]{\ensuremath{#1 \simeq #2}\xspace}
\newcommand{\jdeq}{\equiv}      
\newcommand{\refl}[1]{\ensuremath{\mathsf{refl}_{#1}}\xspace}
\newcommand{\ttt}{\ensuremath{\star}\xspace}
\newcommand{\unit}{\ensuremath{\mathbf{1}}\xspace}
\newtheorem{thm}{Theorem}[section]
\let\autoref\cref
\setcounter{chapter}{2}
\setcounter{section}{8}
\renewcommand{\thethm}{2.8.\arabic{thm}}
\begin{document}
\index{type!unit|(}%
Trivial cases are sometimes important, so we mention briefly the case of the unit type~\unit.

\begin{thm}\label{thm:path-unit}
  For any $x,y:\unit$, we have $\eqv{(x=y)}{\unit}$.
\end{thm}
\begin{proof}
  A function $(x=y)\to\unit$ is easy to define by sending everything to \ttt.
  Conversely, for any $x,y:\unit$ we may assume by induction that $x\jdeq \ttt\jdeq y$.
  In this case we have $\refl{\ttt}:x=y$, yielding a constant function $\unit\to(x=y)$.

  To show that these are inverses, consider first an element $u:\unit$.
  We may assume that $u\jdeq\ttt$, but this is also the result of the composite $\unit \to (x=y)\to\unit$.

  On the other hand, suppose given $p:x=y$.
  By path induction, we may assume $x\jdeq y$ and $p$ is $\refl x$.
  We may then assume that $x$ is \ttt, in which case the composite $(x=y) \to \unit\to(x=y)$ takes $p$ to $\refl x$, i.e.\ to~$p$.
\end{proof}

In particular, any two elements of $\unit$ are equal.
We leave it to the reader to formulate this equivalence in terms of introduction, elimination, computation, and uniqueness rules.
\index{transport!in unit type}%
The transport lemma for \unit is simply the transport lemma for constant type families (\PMlinkname{Lemma 2.3.5}{23typefamiliesarefibrations#Thmlem4}).

\index{type!unit|)}%


\end{document}
