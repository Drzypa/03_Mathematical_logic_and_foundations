\documentclass[12pt]{article}
\usepackage{pmmeta}
\pmcanonicalname{ContinuumHypothesis}
\pmcreated{2013-03-22 12:05:29}
\pmmodified{2013-03-22 12:05:29}
\pmowner{rspuzio}{6075}
\pmmodifier{rspuzio}{6075}
\pmtitle{continuum hypothesis}
\pmrecord{14}{31183}
\pmprivacy{1}
\pmauthor{rspuzio}{6075}
\pmtype{Axiom}
\pmcomment{trigger rebuild}
\pmclassification{msc}{03E50}
\pmsynonym{CH}{ContinuumHypothesis}
\pmrelated{AxiomOfChoice}
\pmrelated{ZermeloFraenkelAxioms}
\pmrelated{GeneralizedContinuumHypothesis}
\pmdefines{continuum}

\usepackage{amssymb}
\usepackage{amsmath}
\usepackage{amsfonts}
\usepackage{graphicx}
%%%\usepackage{xypic}
\begin{document}
The \emph{continuum hypothesis} \PMlinkescapetext{states} that there is no cardinal number $\kappa$ such that $\aleph_0<\kappa <2^{\aleph_0}$.

An equivalent statement is that $\aleph_1 =2^{\aleph_0}$.

It is known to be \PMlinkescapetext{independent} of the axioms of ZFC.

The continuum hypothesis can also be stated as:  there is no subset of the real numbers which has cardinality strictly between that of the reals and that of the integers.  It is from this that the name comes, since the set of real numbers is also known as the continuum.
%%%%%
%%%%%
%%%%%
\end{document}
