\documentclass[12pt]{article}
\usepackage{pmmeta}
\pmcanonicalname{68Pushouts}
\pmcreated{2013-11-20 16:07:29}
\pmmodified{2013-11-20 16:07:29}
\pmowner{PMBookProject}{1000683}
\pmmodifier{rspuzio}{6075}
\pmtitle{6.8 Pushouts}
\pmrecord{5}{87690}
\pmprivacy{1}
\pmauthor{PMBookProject}{6075}
\pmtype{Feature}
\pmclassification{msc}{03B15}

\usepackage{xspace}
\usepackage{amssyb}
\usepackage{amsmath}
\usepackage{amsfonts}
\usepackage{amsthm}
\makeatletter
\newcommand{\cocone}[2]{\mathsf{cocone}_{#1}(#2)}
\newcommand{\composecocone}[2]{#1\circ#2}
\newcommand{\Ddiag}{\mathscr{D}}
\newcommand{\defeq}{\vcentcolon\equiv}  
\newcommand{\defid}{\coloneqq}
\newcommand{\define}[1]{\textbf{#1}}
\def\@dprd#1{\prod_{(#1)}\,}
\def\@dprd@noparens#1{\prod_{#1}\,}
\def\@dsm#1{\sum_{(#1)}\,}
\def\@dsm@noparens#1{\sum_{#1}\,}
\def\@eatprd\prd{\prd@parens}
\def\@eatsm\sm{\sm@parens}
\newcommand{\eqvsym}{\simeq}    
\newcommand{\function}[4]{\left\{\begin{array}{rcl}#1 &      \longrightarrow & #2 \\ #3 & \longmapsto & #4 \end{array}\right.}
\newcommand{\glue}{\mathsf{glue}}
\newcommand{\indexdef}[1]{\index{#1|defstyle}}   
\newcommand{\indexsee}[2]{\index{#1|see{#2}}}    
\newcommand{\inl}{\ensuremath\inlsym\xspace}
\newcommand{\inlsym}{{\mathsf{inl}}}
\newcommand{\inr}{\ensuremath\inrsym\xspace}
\newcommand{\inrsym}{{\mathsf{inr}}}
\newcommand{\jdeq}{\equiv}      
\def\lamu#1{{\lambda}\@lamuarg#1:\@endlamuarg\@ifnextchar\bgroup{.\,\lamu}{.\,}}
\def\@lamuarg#1:#2\@endlamuarg{#1}
\newcommand{\map}[2]{\ensuremath{{#1}\mathopen{}\left({#2}\right)\mathclose{}}\xspace}
\newcommand{\mapfunc}[1]{\ensuremath{\mathsf{ap}_{#1}}\xspace} 
\def\prd#1{\@ifnextchar\bgroup{\prd@parens{#1}}{\@ifnextchar\sm{\prd@parens{#1}\@eatsm}{\prd@noparens{#1}}}}
\def\prd@noparens#1{\mathchoice{\@dprd@noparens{#1}}{\@tprd{#1}}{\@tprd{#1}}{\@tprd{#1}}}
\def\prd@parens#1{\@ifnextchar\bgroup  {\mathchoice{\@dprd{#1}}{\@tprd{#1}}{\@tprd{#1}}{\@tprd{#1}}\prd@parens}  {\@ifnextchar\sm    {\mathchoice{\@dprd{#1}}{\@tprd{#1}}{\@tprd{#1}}{\@tprd{#1}}\@eatsm}    {\mathchoice{\@dprd{#1}}{\@tprd{#1}}{\@tprd{#1}}{\@tprd{#1}}}}}
\newcommand{\proj}[1]{\ensuremath{\mathsf{pr}_{#1}}\xspace}
\newcommand{\refl}[1]{\ensuremath{\mathsf{refl}_{#1}}\xspace}
\def\sm#1{\@ifnextchar\bgroup{\sm@parens{#1}}{\@ifnextchar\prd{\sm@parens{#1}\@eatprd}{\sm@noparens{#1}}}}
\def\sm@noparens#1{\mathchoice{\@dsm@noparens{#1}}{\@tsm{#1}}{\@tsm{#1}}{\@tsm{#1}}}
\def\sm@parens#1{\@ifnextchar\bgroup  {\mathchoice{\@dsm{#1}}{\@tsm{#1}}{\@tsm{#1}}{\@tsm{#1}}\sm@parens}  {\@ifnextchar\prd    {\mathchoice{\@dsm{#1}}{\@tsm{#1}}{\@tsm{#1}}{\@tsm{#1}}\@eatprd}    {\mathchoice{\@dsm{#1}}{\@tsm{#1}}{\@tsm{#1}}{\@tsm{#1}}}}}
\newcommand{\Sn}{\mathbb{S}}
\newcommand{\susp}{\Sigma}
\newcommand{\symlabel}[1]{\refstepcounter{symindex}\label{#1}}
\def\@tprd#1{\mathchoice{{\textstyle\prod_{(#1)}}}{\prod_{(#1)}}{\prod_{(#1)}}{\prod_{(#1)}}}
\def\@tsm#1{\mathchoice{{\textstyle\sum_{(#1)}}}{\sum_{(#1)}}{\sum_{(#1)}}{\sum_{(#1)}}}
\newcommand{\unit}{\ensuremath{\mathbf{1}}\xspace}
\newcommand{\UU}{\ensuremath{\mathcal{U}}\xspace}
\newcommand{\vcentcolon}{:\!\!}
\newcounter{mathcount}
\setcounter{mathcount}{1}
\newtheorem{predefn}{Definition}
\newenvironment{defn}{\begin{predefn}}{\end{predefn}\addtocounter{mathcount}{1}}
\renewcommand{\thepredefn}{6.8.\arabic{mathcount}}
\newtheorem{prelem}{Lemma}
\newenvironment{lem}{\begin{prelem}}{\end{prelem}\addtocounter{mathcount}{1}}
\renewcommand{\theprelem}{6.8.\arabic{mathcount}}
\newtheorem{prermk}{Remark}
\newenvironment{rmk}{\begin{prermk}}{\end{prermk}\addtocounter{mathcount}{1}}
\renewcommand{\theprermk}{6.8.\arabic{mathcount}}
\let\ap\map
\let\autoref\cref
\let\type\UU
\makeatother
\def\coloneqq{:=}
\begin{document}
\index{type!limit}%
\index{type!colimit}%
\index{limit!of types}%
\index{colimit!of types}%
From a category-theoretic point of view, one of the important aspects of any foundational system is the ability to construct limits and colimits.
In set-theoretic foundations, these are limits and colimits of sets, whereas in our case they are limits and colimits of \emph{types}.
We have seen in \PMlinkname{\S 2.15}{215universalproperties} that cartesian product types have the correct universal property of a categorical product of types, and in \PMlinkexternal{Exercise 2.9}{http://planetmath.org/node/87640} that coproduct types likewise have their expected universal property.

As remarked in \PMlinkname{\S 2.15}{215universalproperties}, more general limits can be constructed using identity types and $\Sigma$-types, e.g.\ the pullback\index{pullback} of $f:A\to C$ and $g:B\to C$ is $\sm{a:A}{b:B} (f(a)=g(b))$ (see \PMlinkexternal{Exercise 2.11}{http://planetmath.org/node/87642}).
However, more general \emph{colimits} require identifying elements coming from different types, for which higher inductives are well-adapted.
Since all our constructions are homotopy-invariant, all our colimits are necessarily \emph{homotopy colimits}, but we drop the ubiquitous adjective in the interests of concision.

In this section we discuss \emph{pushouts}, as perhaps the simplest and one of the most useful colimits.
Indeed, one expects all finite colimits (for a suitable homotopical definition of ``finite'') to be constructible from pushouts and finite coproducts.
It is also possible to give a direct construction of more general colimits using higher inductive types, but this is somewhat technical, and also not completely satisfactory since we do not yet have a good fully general notion of homotopy coherent diagrams.

\indexsee{type!pushout of}{pushout}%
\index{pushout|(defstyle}%
\index{span}%
Suppose given a span of types and functions:
\begin{figure}
 \centering
 \includegraphics{HoTT_fig_6.8.1.png}
\end{figure}
% \[\Ddiag=\;\vcenter{\xymatrix{C \ar^g[r] \ar_f[d] & B \\ A & }}\]
The \define{pushout} of this span is the higher inductive type $A\sqcup^CB$ presented by
\begin{itemize}
\item a function $\inl:A\to A\sqcup^CB$,
\item a function $\inr:B \to A\sqcup^CB$, and
\item for each $c:C$ a path $\glue(c):(\inl(f(c))=\inr(g(c)))$.
\end{itemize}
In other words, $A\sqcup^CB$ is the disjoint union of $A$ and $B$, together with for every $c:C$ a witness that $f(c)$ and $g(c)$ are equal.
The recursion principle says that if $D$ is another type, we can define a map $s:A\sqcup^CB\to{}D$ by defining
\begin{itemize}
\item for each $a:A$, the value of $s(\inl(a)):D$,
\item for each $b:B$, the value of $s(\inr(b)):D$, and
\item for each $c:C$, the value of $\mapfunc{s}(\glue(c)):s(\inl(f(c)))=s(\inr(g(c)))$.
\end{itemize}
We leave it to the reader to formulate the induction principle.
It also implies the uniqueness principle that if $s,s':A\sqcup^CB\to{}D$ are two maps such that
\index{uniqueness!principle, propositional!for functions on a pushout}%
\begin{align*}
  s(\inl(a))&=s'(\inl(a))\\
  s(\inr(b))&=s'(\inr(b))\\
  \mapfunc{s}(\glue(c))&=\mapfunc{s'}(\glue(c))
  \qquad\text{(modulo the previous two equalities)}
\end{align*}
for every $a,b,c$, then $s=s'$.

To formulate the universal property of a pushout, we introduce the following.

\begin{defn}\label{defn:cocone}
  Given a span $\Ddiag= (A \xleftarrow{f} C \xrightarrow{g} B)$ and a type $D$, a \define{cocone under $\Ddiag$ with vertex $D$}
  \indexdef{cocone}%
  \index{vertex of a cocone}%
  consists of functions $i:A\to{}D$ and $j:B\to{}D$ and a homotopy $h : \prd{c:C} (i(f(c))=j(g(c)))$:
\begin{figure}
 \centering
 \includegraphics{HoTT_fig_6.8.2.png}
\end{figure}
%\[\uppercurveobject{{ }}\lowercurveobject{{ }}\twocellhead{{ }}
%  \xymatrix{C \ar^g[r] \ar_f[d] \drtwocell{^h} & B \ar^j[d] \\ A \ar_i[r] & D
%  }\]
  We denote by $\cocone{\Ddiag}{D}$ the type of all such cocones, i.e.
  \[ \cocone{\Ddiag}{D} \defeq
  \sm{i:A\to D}{j:B\to D} \prd{c:C} (i(f(c))=j(g(c))).
  \]
\end{defn}

Of course, there is a canonical cocone under $\Ddiag$ with vertex $A\sqcup^C B$ consisting of $\inl$, $\inr$, and $\glue$.
\begin{figure}
 \centering
 \includegraphics{HoTT_fig_6.8.3.png}
\end{figure}
%\[\uppercurveobject{{ }}\lowercurveobject{{ }}\twocellhead{{ }}
%\xymatrix{C \ar^g[r] \ar_f[d] \drtwocell{^\glue\ \ } & B \ar^\inr[d] \\
%  A \ar_-\inl[r] & A\sqcup^CB }\]
The following lemma says that this is the universal such cocone.

\begin{lem}\label{thm:pushout-ump}
  \index{universal!property!of pushout}%
  For any type $E$, there is an equivalence
  \[ (A\sqcup^C B \to E) \;\eqvsym\; \cocone{\Ddiag}{E}. \]
\end{lem}
\begin{proof}
  Let's consider an arbitrary type $E:\type$.
  There is a canonical function
  \[\function{(A\sqcup^CB\to{}E)}{\cocone{\Ddiag}{E}}
  {t}{\composecocone{t}c_\sqcup}\]
  defined by sending $(i,j,h)$ to $(t\circ{}i,t\circ{}j,\mapfunc{t}\circ{}h)$.
  We show that this is an equivalence.

  Firstly, given a $c=(i,j,h):\cocone{\mathscr{D}}{E}$, we need to construct a
  map $\mathsf{s}(c)$ from $A\sqcup^CB$ to $E$.
\begin{figure}
 \centering
 \includegraphics{HoTT_fig_6.8.4.png}
\end{figure}
%  \[\uppercurveobject{{ }}\lowercurveobject{{ }}\twocellhead{{ }}
%  \xymatrix{C \ar^g[r] \ar_f[d] \drtwocell{^h} & B \ar^{j}[d] \\
%    A \ar_-{i}[r] & E }\]
 The map $\mathsf{s}(c)$ is defined in the following way
  \begin{align*}
    \mathsf{s}(c)(\inl(a))&\defeq i(a),\\
    \mathsf{s}(c)(\inr(b))&\defeq j(b),\\
    \mapfunc{\mathsf{s}(c)}(\glue(x))&\defid h(x).
  \end{align*}
We have defined a map
\[\function{\cocone{\Ddiag}{E}}{(A\sqcup^BC\to{}E)}{c}{\mathsf{s}(c)}\]
and we need to prove that this map is an inverse to
$t\mapsto{}\composecocone{t}c_\sqcup$.
On the one hand, if $c=(i,j,h):\cocone{\Ddiag}{E}$, we have
\begin{align*}
  \composecocone{\mathsf{s}(c)}c_\sqcup & =
  (\mathsf{s}(c)\circ\inl,\mathsf{s}(c)\circ\inr,
  \mapfunc{\mathsf{s}(c)}\circ\glue) \\
  & = (\lamu{a:A} \mathsf{s}(c)(\inl(a)),\;
  \lamu{b:B} \mathsf{s}(c)(\inr(b)),\;
  \lamu{x:C} \mapfunc{\mathsf{s}(c)}(\glue(x))) \\
  & = (\lamu{a:A} i(a),\;
  \lamu{b:B} j(b),\;
  \lamu{x:C} h(x)) \\
  & \jdeq (i, j, h) \\
  & = c.
\end{align*}
%
On the other hand, if $t:A\sqcup^BC\to{}E$, we want to prove that
$\mathsf{s}(\composecocone{t}c_\sqcup)=t$.
For $a:A$, we have
\[\mathsf{s}(\composecocone{t}c_\sqcup)(\inl(a))=t(\inl(a))\]
because the first component of $\composecocone{t}c_\sqcup$ is $t\circ\inl$. In
the same way, for $b:B$ we have
\[\mathsf{s}(\composecocone{t}c_\sqcup)(\inr(b))=t(\inr(b))\]
and for $x:C$ we have
\[\mapfunc{\mathsf{s}(\composecocone{t}c_\sqcup)}(\glue(x))
=\mapfunc{t}(\glue(x))\]
hence $\mathsf{s}(\composecocone{t}c_\sqcup)=t$.

This proves that $c\mapsto\mathsf{s}(c)$ is a quasi-inverse to $t\mapsto{}\composecocone{t}c_\sqcup$, as desired.
\end{proof}

A number of standard homotopy-theoretic constructions can be expressed as (homotopy) pushouts.
\begin{itemize}
\item The pushout of the span $\unit \leftarrow A \to \unit$ is the \define{suspension} $\susp A$ (see \PMlinkname{\S 6.5}{65suspensions}).%
  \index{suspension}
\symlabel{join}
\item The pushout of $A \xleftarrow{\proj1} A\times B \xrightarrow{\proj2} B$ is called the \define{join} of $A$ and $B$, written $A*B$.%
  \indexdef{join!of types}
\item The pushout of $\unit \leftarrow A \xrightarrow{f} B$ is the \define{cone} or \define{cofiber} of $f$.%
  \indexdef{cone!of a function}%
  \indexsee{mapping cone}{cone of a function}%
  \indexdef{cofiber of a function}%
\symlabel{wedge}
\item If $A$ and $B$ are equipped with basepoints $a_0:A$ and $b_0:B$, then the pushout of $A \xleftarrow{a_0} \unit \xrightarrow{b_0} B$ is the \define{wedge} $A\vee B$.%
  \indexdef{wedge}
\symlabel{smash}
\item If $A$ and $B$ are pointed as before, define $f:A\vee B \to A\times B$ by $f(\inl(a))\defeq (a,b_0)$ and $f(\inr(b))\defeq (a_0,b)$, with $\ap f \glue \defid \refl{(a_0,b_0)}$.
  Then the cone of $f$ is called the \define{smash product} $A\wedge B$.%
  \indexdef{smash product}
\end{itemize}
We will discuss pushouts further in \PMlinkexternal{Chapter 7}{http://planetmath.org/node/87580},\PMlinkexternal{Chapter 8}{http://planetmath.org/node/87582}.

\begin{rmk}
  As remarked in \PMlinkname{\S 3.7}{37propositionaltruncation}, the notations $\wedge$ and $\vee$ for the smash product and wedge of pointed spaces are also used in logic for ``and'' and ``or'', respectively.
  Since types in homotopy type theory can behave either like spaces or like propositions, there is technically a potential for conflict --- but since they rarely do both at once, context generally disambiguates.
  Furthermore, the smash product and wedge only apply to \emph{pointed} spaces, while the only pointed mere proposition is $\top\jdeq\unit$ --- and we have $\unit\wedge \unit = \unit$ and $\unit\vee\unit=\unit$ for either meaning of $\wedge$ and $\vee$.
\end{rmk}

\index{pushout|)}%

\begin{rmk}
  Note that colimits do not in general preserve truncatedness.
  For instance, $\Sn^0$ and \unit are both sets, but the pushout of $\unit \leftarrow \Sn^0 \to \unit$ is $\Sn^1$, which is not a set.
  If we are interested in colimits in the category of $n$-types, therefore (and, in particular, in the category of sets), we need to ``truncate'' the colimit somehow.
  We will return to this point in \PMlinkname{\S 6.9}{69truncations},\PMlinkexternal{Chapter 7}{http://planetmath.org/node/87580},\PMlinkexternal{Chapter 10}{http://planetmath.org/node/87584}.
\end{rmk}



\end{document}
