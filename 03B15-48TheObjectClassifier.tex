\documentclass[12pt]{article}
\usepackage{pmmeta}
\pmcanonicalname{48TheObjectClassifier}
\pmcreated{2013-11-20 2:12:44}
\pmmodified{2013-11-20 2:12:44}
\pmowner{PMBookProject}{1000683}
\pmmodifier{rspuzio}{6075}
\pmtitle{4.8 The object classifier}
\pmrecord{5}{87670}
\pmprivacy{1}
\pmauthor{PMBookProject}{6075}
\pmtype{Feature}
\pmclassification{msc}{03B15}

\endmetadata

\usepackage{xspace}
\usepackage{amssyb}
\usepackage{amsmath}
\usepackage{amsfonts}
\usepackage{amsthm}
\makeatletter
\newcommand{\defeq}{\vcentcolon\equiv}  
\def\@dprd#1{\prod_{(#1)}\,}
\def\@dprd@noparens#1{\prod_{#1}\,}
\def\@dsm#1{\sum_{(#1)}\,}
\def\@dsm@noparens#1{\sum_{#1}\,}
\def\@eatprd\prd{\prd@parens}
\def\@eatsm\sm{\sm@parens}
\newcommand{\eqv}[2]{\ensuremath{#1 \simeq #2}\xspace}
\newcommand{\eqvsym}{\simeq}    
\newcommand{\hfib}[2]{{\mathsf{fib}}_{#1}(#2)}
\newcommand{\htpy}{\sim}
\newcommand{\idfunc}[1][]{\ensuremath{\mathsf{id}_{#1}}\xspace}
\newcommand{\indexdef}[1]{\index{#1|defstyle}}   
\newcommand{\jdeq}{\equiv}      
\def\lam#1{{\lambda}\@lamarg#1:\@endlamarg\@ifnextchar\bgroup{.\,\lam}{.\,}}
\def\@lamarg#1:#2\@endlamarg{\if\relax\detokenize{#2}\relax #1\else\@lamvar{\@lameatcolon#2},#1\@endlamvar\fi}
\def\@lameatcolon#1:{#1}
\def\@lamvar#1,#2\@endlamvar{(#2\,{:}\,#1)}
\newcommand{\pairr}[1]{{\mathopen{}(#1)\mathclose{}}}
\newcommand{\Pairr}[1]{{\mathopen{}\left(#1\right)\mathclose{}}}
\newcommand{\Parens}[1]{\Bigl(#1\Bigr)}
\newcommand{\pointed}[1]{\ensuremath{#1_\bullet}}
\def\prd#1{\@ifnextchar\bgroup{\prd@parens{#1}}{\@ifnextchar\sm{\prd@parens{#1}\@eatsm}{\prd@noparens{#1}}}}
\def\prd@noparens#1{\mathchoice{\@dprd@noparens{#1}}{\@tprd{#1}}{\@tprd{#1}}{\@tprd{#1}}}
\def\prd@parens#1{\@ifnextchar\bgroup  {\mathchoice{\@dprd{#1}}{\@tprd{#1}}{\@tprd{#1}}{\@tprd{#1}}\prd@parens}  {\@ifnextchar\sm    {\mathchoice{\@dprd{#1}}{\@tprd{#1}}{\@tprd{#1}}{\@tprd{#1}}\@eatsm}    {\mathchoice{\@dprd{#1}}{\@tprd{#1}}{\@tprd{#1}}{\@tprd{#1}}}}}
\newcommand{\proj}[1]{\ensuremath{\mathsf{pr}_{#1}}\xspace}
\newcommand{\refl}[1]{\ensuremath{\mathsf{refl}_{#1}}\xspace}
\def\sm#1{\@ifnextchar\bgroup{\sm@parens{#1}}{\@ifnextchar\prd{\sm@parens{#1}\@eatprd}{\sm@noparens{#1}}}}
\def\sm@noparens#1{\mathchoice{\@dsm@noparens{#1}}{\@tsm{#1}}{\@tsm{#1}}{\@tsm{#1}}}
\def\sm@parens#1{\@ifnextchar\bgroup  {\mathchoice{\@dsm{#1}}{\@tsm{#1}}{\@tsm{#1}}{\@tsm{#1}}\sm@parens}  {\@ifnextchar\prd    {\mathchoice{\@dsm{#1}}{\@tsm{#1}}{\@tsm{#1}}{\@tsm{#1}}\@eatprd}    {\mathchoice{\@dsm{#1}}{\@tsm{#1}}{\@tsm{#1}}{\@tsm{#1}}}}}
\def\@tprd#1{\mathchoice{{\textstyle\prod_{(#1)}}}{\prod_{(#1)}}{\prod_{(#1)}}{\prod_{(#1)}}}
\newcommand{\trans}[2]{\ensuremath{{#1}_{*}\mathopen{}\left({#2}\right)\mathclose{}}\xspace}
\def\tsm#1{\@tsm{#1}\@ifnextchar\bgroup{\tsm}{}}
\def\@tsm#1{\mathchoice{{\textstyle\sum_{(#1)}}}{\sum_{(#1)}}{\sum_{(#1)}}{\sum_{(#1)}}}
\newcommand{\ua}{\ensuremath{\mathsf{ua}}\xspace} 
\newcommand{\UU}{\ensuremath{\mathcal{U}}\xspace}
\newcommand{\vcentcolon}{:\!\!}
\newcounter{mathcount}
\setcounter{mathcount}{1}
\newtheorem{prelem}{Lemma}
\newenvironment{lem}{\begin{prelem}}{\end{prelem}\addtocounter{mathcount}{1}}
\renewcommand{\theprelem}{4.8.\arabic{mathcount}}
\newtheorem{prethm}{Theorem}
\newenvironment{thm}{\begin{prethm}}{\end{prethm}\addtocounter{mathcount}{1}}
\renewcommand{\theprethm}{4.8.\arabic{mathcount}}
\let\autoref\cref
\let\hfiber\hfib
\let\type\UU
\makeatother

\begin{document}
In type theory we have a basic notion of \emph{family of types}, namely a function $B:A\to\type$.
We have seen that such families behave somewhat like \emph{fibrations} in homotopy theory, with the fibration being the projection $\proj1:\sm{a:A} B(a) \to A$.
A basic fact in homotopy theory is that every map is equivalent to a fibration.
With univalence at our disposal, we can prove the same thing in type theory.

\begin{lem}\label{thm:fiber-of-a-fibration}
  For any type family $B:A\to\type$, the fiber of $\proj1:\sm{x:A} B(x) \to A$ over $a:A$ is equivalent to $B(a)$:
  \[ \eqv{\hfib{\proj1}{a}}{B(a)} \]
\end{lem}
\begin{proof}
  We have
  \begin{align*}
    \hfib{\proj1}{a} &\defeq \sm{u:\sm{x:A} B(x)} \proj1(u)=a\\
    &\eqvsym \sm{x:A}{b:B(x)} (x=a)\\
    &\eqvsym \sm{x:A}{p:x=a} B(x)\\
    &\eqvsym B(a)
  \end{align*}
  using the left universal property of identity types.
\end{proof}

\begin{lem}\label{thm:total-space-of-the-fibers}
  For any function $f:A\to B$, we have $\eqv{A}{\sm{b:B}\hfib{f}{b}}$.
\end{lem}
\begin{proof}
  We have
  \begin{align*}
    \sm{b:B}\hfib{f}{b} &\defeq \sm{b:B}{a:A} (f(a)=b)\\
    &\eqvsym \sm{a:A}{b:B} (f(a)=b)\\
    &\eqvsym A
  \end{align*}
  using the fact that $\sm{b:B} (f(a)=b)$ is contractible.
\end{proof}

\begin{thm}\label{thm:nobject-classifier-appetizer}
For any type $B$ there is an equivalence
\begin{equation*}
\chi:\Parens{\sm{A:\type} (A\to B)}\eqvsym (B\to\type).
\end{equation*}
\end{thm}
\begin{proof}
We have to construct quasi-inverses
\begin{align*}
\chi & : \Parens{\sm{A:\type} (A\to B)}\to B\to\type\\
\psi & : (B\to\type)\to\Parens{\sm{A:\type} (A\to B)}.
\end{align*}
We define $\chi$ by $\chi((A,f),b)\defeq\hfiber{f}b$, and $\psi$ by $\psi(P)\defeq\Pairr{(\sm{b:B} P(b)),\proj1}$.
Now we have to verify that $\chi\circ\psi\htpy\idfunc{}$ and that $\psi\circ\chi \htpy\idfunc{}$.
\begin{enumerate}
\item Let $P:B\to\type$.
  By \PMlinkname{Lemma 4.8.1}{48theobjectclassifier#Thmprelem1},
$\hfiber{\proj1}{b}\eqvsym P(b)$ for any $b:B$, so it follows immediately
that $P\htpy\chi(\psi(P))$.
\item Let $f:A\to B$ be a function. We have to find a path
\begin{equation*}
\Pairr{\tsm{b:B} \hfiber{f}b,\,\proj1}=\pairr{A,f}.
\end{equation*}
First note that by \PMlinkname{Lemma 4.8.2}{48theobjectclassifier#Thmprelem2}, we have
$e:\sm{b:B} \hfiber{f}b\eqvsym A$ with $e(b,a,p)\defeq a$ and $e^{-1}(a)
\defeq(f(a),a,\refl{f(a)})$.
By \PMlinkname{Theorem 2.7.2}{27sigmatypes#Thmprethm1}, it remains to show $\trans{(\ua(e))}{\proj1} = f$.
But by the computation rule for univalence and~\PMlinkname{(2.9.4)}{29pitypesandthefunctionextensionalityaxiom#S0.E3}, we have $\trans{(\ua(e))}{\proj1} = \proj1\circ e^{-1}$, and the definition of $e^{-1}$ immediately yields $\proj1 \circ e^{-1} \jdeq f$.\qedhere
\end{enumerate}
\end{proof}

\noindent
\indexdef{object!classifier}%
\indexdef{classifier!object}%
\index{.infinity1-topos@$(\infty,1)$-topos}%
In particular, this implies that we have an \emph{object classifier} in the sense of higher topos theory.
Recall from \PMlinkname{Definition 2.1.7}{21typesarehighergroupoids#Thmpredefn1} that $\pointed\type$ denotes the type $\sm{A:\type} A$ of pointed types.

\begin{thm}\label{thm:object-classifier}
Let $f:A\to B$ be a function. Then the diagram
\begin{figure}
 \centering
 \includegraphics{HoTT_fig_4.8.1.png}
\end{figure}
%\begin{equation*}
%  \vcenter{\xymatrix{
%      A\ar[r]^-{\vartheta_f} \ar[d]_{f} &
%      \pointed{\type}\ar[d]^{\proj1}\\
%      B\ar[r]_{\chi_f} &
%      \type
%      }}
%\end{equation*}
is a pullback\index{pullback} square (see \PMlinkexternal{Exercise 2.11}{http://planetmath.org/node/87642}).
Here the function $\vartheta_f$ is defined by
\begin{equation*}
 \lam{a} \pairr{\hfiber{f}{f(a)},\pairr{a,\refl{f(a)}}}.
\end{equation*}
\end{thm}
\begin{proof}
Note that we have the equivalences
\begin{align*}
A & \eqvsym \sm{b:B} \hfiber{f}b\\
& \eqvsym \sm{b:B}{X:\type}{p:\hfiber{f}b= X} X\\
& \eqvsym \sm{b:B}{X:\type}{x:X} \hfiber{f}b= X\\
& \eqvsym \sm{b:B}{Y:\pointed{\type}} \hfiber{f}b = \proj1 Y\\
& \jdeq B\times_{\type}\pointed{\type}.
\end{align*}
which gives us a composite equivalence $e:A\eqvsym B\times_\type\pointed{\type}$. 
We may display the action of this composite equivalence step by step by
\begin{align*}
a & \mapsto \pairr{f(a),\; \pairr{a,\refl{f(a)}}}\\
& \mapsto \pairr{f(a), \; \hfiber{f}{f(a)}, \; \refl{\hfiber{f}{f(a)}}, \; \pairr{a,\refl{f(a)}}}\\
& \mapsto \pairr{f(a), \; \hfiber{f}{f(a)}, \; \pairr{a,\refl{f(a)}}, \; \refl{\hfiber{f}{f(a)}}}.
\end{align*}
Therefore, we get homotopies $f\htpy\proj1\circ e$ and $\vartheta_f\htpy \proj2\circ e$. 
\end{proof}


\end{document}
