\documentclass[12pt]{article}
\usepackage{pmmeta}
\pmcanonicalname{GeneralizedContinuumHypothesis}
\pmcreated{2013-03-22 12:05:31}
\pmmodified{2013-03-22 12:05:31}
\pmowner{yark}{2760}
\pmmodifier{yark}{2760}
\pmtitle{generalized continuum hypothesis}
\pmrecord{15}{31184}
\pmprivacy{1}
\pmauthor{yark}{2760}
\pmtype{Axiom}
\pmcomment{trigger rebuild}
\pmclassification{msc}{03E50}
\pmsynonym{generalised continuum hypothesis}{GeneralizedContinuumHypothesis}
\pmsynonym{GCH}{GeneralizedContinuumHypothesis}
%\pmkeywords{cardinality}
%\pmkeywords{cardinal}
\pmrelated{AlephNumbers}
\pmrelated{BethNumbers}
\pmrelated{ContinuumHypothesis}
\pmrelated{Cardinality}
\pmrelated{CardinalExponentiationUnderGCH}
\pmrelated{ZermeloFraenkelAxioms}

\endmetadata

\usepackage{amssymb}
\usepackage{amsmath}
\usepackage{amsfonts}
%\usepackage{graphicx}
%%%%\usepackage{xypic}
\begin{document}
\PMlinkescapeword{equivalent}
\PMlinkescapeword{independent}
\PMlinkescapeword{states}

The \emph{generalized continuum hypothesis} states that for any infinite cardinal $\lambda$ there is no cardinal $\kappa$ such that $\lambda <\kappa <2^{\lambda}$.

An equivalent condition is that $\aleph_{\alpha+1}=2^{\aleph_\alpha}$ for every ordinal $\alpha$.
Another equivalent condition is that $\aleph_\alpha=\beth_\alpha$ for every ordinal $\alpha$.

Like the continuum hypothesis, the generalized continuum hypothesis is known to be independent of the axioms of ZFC.
%%%%%
%%%%%
%%%%%
\end{document}
