\documentclass[12pt]{article}
\usepackage{pmmeta}
\pmcanonicalname{RelationAlgebra}
\pmcreated{2013-03-22 17:48:18}
\pmmodified{2013-03-22 17:48:18}
\pmowner{CWoo}{3771}
\pmmodifier{CWoo}{3771}
\pmtitle{relation algebra}
\pmrecord{11}{40267}
\pmprivacy{1}
\pmauthor{CWoo}{3771}
\pmtype{Definition}
\pmcomment{trigger rebuild}
\pmclassification{msc}{03G15}
\pmclassification{msc}{06E25}
\pmrelated{OperationsOnRelations}
\pmdefines{set relation algebra}

\endmetadata

\usepackage{amssymb,amscd}
\usepackage{amsmath}
\usepackage{amsfonts}
\usepackage{mathrsfs}

% used for TeXing text within eps files
%\usepackage{psfrag}
% need this for including graphics (\includegraphics)
%\usepackage{graphicx}
% for neatly defining theorems and propositions
\usepackage{amsthm}
% making logically defined graphics
%%\usepackage{xypic}
\usepackage{pst-plot}

% define commands here
\newcommand*{\abs}[1]{\left\lvert #1\right\rvert}
\newtheorem{prop}{Proposition}
\newtheorem{thm}{Theorem}
\newtheorem{ex}{Example}
\newcommand{\real}{\mathbb{R}}
\newcommand{\pdiff}[2]{\frac{\partial #1}{\partial #2}}
\newcommand{\mpdiff}[3]{\frac{\partial^#1 #2}{\partial #3^#1}}
\begin{document}
It is a well-known fact that if $A$ is a set, then $P(A)$ the power set of $A$, equipped with the intersection operation $\cap$, the union operation $\cup$, and the complement operation $'$ turns $P(A)$ into a Boolean algebra.  Indeed, a Boolean algebra can be viewed as an abstraction of the family of subsets of a set with the usual set-theoretic operations.  The algebraic abstraction of the set of binary relations on a set is what is known as a \emph{relation algebra}.

Before defining formally what a relation algebra is, let us recall the following facts about binary relations on some given set $A$:
\begin{itemize}
\item binary relations are closed under $\cap,\cup$ and $'$, and in fact, satisfy all the axioms of a Boolean algebra.  In short, the set $R(A)$ of binary relations on $A$ is a Boolean algebra, where $A\times A$ is the top and $\varnothing\times \varnothing (=\varnothing) $ is the bottom of $R(A)$;
\item if $R$ and $S$ are binary relations on $A$, then so are $R\circ S$, relation composition of $R$ and $S$, and $R^{-1}$, the inverse of $R$;
\item $I_A$, defined by $\lbrace (a,a)\mid a\in A\rbrace$, is a binary relation which is also the identity with respect to $\circ$, also known as the identity relation on $A$;
\item some familiar identities:
\begin{enumerate}
\item $R\circ (S\circ T) = (R\circ S)\circ T$
\item $(R^{-1})^{-1}=R$
\item $(R\circ S)^{-1} = S^{-1}\circ R^{-1}$
\item $(R\cup S)\circ T= (R\circ T)\cup (S\circ T)$
\item $(R\cup S)^{-1}=R^{-1}\cup S^{-1}$
\end{enumerate}
\end{itemize}

In addition, there is also a rule that is true for all binary relations on $A$: 
\begin{eqnarray}
(R\circ S)\cap T=\varnothing\quad \mbox{ iff } \quad (R^{-1}\circ T)\cap S = \varnothing
\end{eqnarray}
The verification of this rule is as follows: first note that sufficiency implies necessity, for if $(R^{-1}\circ T)\cap S = \varnothing$, then $(R\circ S)\cap T=((R^{-1})^{-1}\circ S)\cap T=\varnothing$.  To show sufficiency, suppose $(a,b)\in S$ is also an element of $R^{-1}\circ T$.  Then there is $c\in A$ such that $(a,c)\in R^{-1}$ and $(c,b)\in T$.  Therefore, $(c,a)\in R$ and $(c,b)=(c,a)\circ (a,b)\in R\circ S$ as well.  This shows that $(R\circ S)\cap T\ne \varnothing$.

It can be shown that Rule (1) is equivalent to the following inclusion 
\begin{eqnarray}
R^{-1}\circ(R\circ S)'\subseteq S'.
\end{eqnarray}

\textbf{Definition}.  A \emph{relation algebra} is a Boolean algebra $B$ with the usual Boolean operators $\vee, \wedge, '$, and additionally a binary operator $\ ;$, a unary operator $^-$, and a constant $i$ such that
\begin{enumerate}
\item $a\ ;i=a$
\item $a\ ;(b\ ; c)=(a\ ;b)\ ;c$
\item $a^{--}=a$
\item $(a\ ;b)^-=b^-\ ;a^-$
\item $(a \vee b)\ ; c=(a\ ; c)\vee (b\ ; c)$
\item $(a \vee b)^- = a^- \vee b^-$
\item $a^- \ ; (a\ ; b)'\le b'$
\end{enumerate}
where $\le$ is the induced partial order in the underlying Boolean algebra.

Clearly, the set of all binary relations $R(A)$ on a set $A$ is a relation algebra, as we have just demonstrated.  Specifically, in $R(A)$, $\ ;$ is the composition operator $\circ$, $^-$ is the inverse (or converse) operator $^{-1}$, and $i$ is $I_A$.  

A relation algebra is an algebraic system.  As an algebraic system, we can define the usual algebraic notions, such as subalgebras of an algebra, homomorphisms between two algebras, etc...  A relation algebra $B$ that is a subalgebra of $R(A)$, the set of all binary relations on a set $A$, is called a \emph{set relation algebra}.

\begin{thebibliography}{8}
\bibitem{sg} S. R. Givant, \emph{The Structure of Relation Algebras Generated by Relativizations}, American Mathematical Society (1994).
\end{thebibliography}
%%%%%
%%%%%
\end{document}
