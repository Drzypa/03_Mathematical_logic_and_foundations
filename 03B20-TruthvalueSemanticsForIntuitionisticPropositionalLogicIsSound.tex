\documentclass[12pt]{article}
\usepackage{pmmeta}
\pmcanonicalname{TruthvalueSemanticsForIntuitionisticPropositionalLogicIsSound}
\pmcreated{2013-03-22 19:31:38}
\pmmodified{2013-03-22 19:31:38}
\pmowner{CWoo}{3771}
\pmmodifier{CWoo}{3771}
\pmtitle{truth-value semantics for intuitionistic propositional logic is sound}
\pmrecord{21}{42504}
\pmprivacy{1}
\pmauthor{CWoo}{3771}
\pmtype{Definition}
\pmcomment{trigger rebuild}
\pmclassification{msc}{03B20}

\usepackage{amssymb,amscd}
\usepackage{amsmath}
\usepackage{amsfonts}
\usepackage{mathrsfs}

% used for TeXing text within eps files
%\usepackage{psfrag}
% need this for including graphics (\includegraphics)
%\usepackage{graphicx}
% for neatly defining theorems and propositions
\usepackage{amsthm}
% making logically defined graphics
%%\usepackage{xypic}
\usepackage{pst-plot}

% define commands here
\newcommand*{\abs}[1]{\left\lvert #1\right\rvert}
\newtheorem{prop}{Proposition}
\newtheorem{thm}{Theorem}
\newtheorem{lem}{Lemma}
\newtheorem{ex}{Example}
\newcommand{\real}{\mathbb{R}}
\newcommand{\pdiff}[2]{\frac{\partial #1}{\partial #2}}
\newcommand{\mpdiff}[3]{\frac{\partial^#1 #2}{\partial #3^#1}}

\begin{document}
\begin{prop} The truth-value semantics for intuitionistic propositional logic is sound. \end{prop}
\begin{proof}
We show that, for each positive integer $n$, every theorem of intuitionistic propositional logic is a tautology for $V_n$.  This amounts to showing that, for every interpretation $v$ on $V_n$,
\begin{itemize}
\item
each axiom is true, and
\item 
modus ponens preserves truth.
\end{itemize}
Let us take care of the second one first.  Suppose $v(A)=v(A\to B)=n$.  If $v(A)\le v(B)$, then $v(B)=n$.  Otherwise, $v(B)<v(A)$.  But this means that $n=v(A\to B)=v(B)$, forcing $v(B)=n$.  Therefore, $v(B)=n$.

Now, we verify that each of the axiom schemas below are true:

\begin{enumerate}
\item $(A \land B) \to A$ and $(A\land B)\to B$.

Since $v(A\land B)=\min\lbrace v(A),v(B)\rbrace \le v(A)$, we get $v((A\land B)\to A)=n$.  The other one is proved similarly.

\item $A \to (A \lor B)$ and $B \to (A\lor B)$.

Since $v(A) \le \max\lbrace v(A),v(B)\rbrace = v(A\lor B)$, we get $v(A\to (A\lor B))=n$.  The other one is proved similarly.

\item $A \to (B \to A)$.

If $v(B)\le v(A)$, $v(B\to A)=n$, so that $v(A\to (B\to A))=n$ as well.  If $v(A)<v(B)$, then $v(B\to A)=v(A)$, so that $v(A\to (B\to A))=n$.

\item $\neg A \to  (A \to B)$.

If $v(A)\le v(B)$, $v(A\to B)=n$, so that $v(\neg A\to (A\to B))=n$ as well.  If $v(B)<v(A)$, then $v(A\to B)=v(B)$.  Also, $v(B)<v(A)$ implies that $v(A)>0$, so that $v(\neg A)=0$, and $v(\neg A\to (A\to B))=n$ as a result.

\item $A \to (B \to (A \land B))$.

If $v(B)=v(A\land B)$, then $v(B)\le v(A)$ and $v(B\to (A\land B))=n$, so that $v(A\to (B\to (A\land B))=n$ also.  If on the other hand $v(A\land B)<v(B)$, then $$v(A)=v(A\land B) \quad \mbox{and} \quad v(B\to (A\land B))=v(A\land B),$$ so that 
$$v(A \to (B\to (A\land B)))= v(A\to (A\land B)) = n.$$

\item $(A \to C) \to ((B \to C) \to ((A \lor B) \to C))$.

If $v(B)=v(A\lor B)$, then $v(A)\le v(B)$, and $$v((B \to C) \to ((A \lor B) \to C))=v((B \to C) \to (B \to C)) = n,$$ so that $$v((A\to C) \to ((B \to C) \to ((A \lor B) \to C)))=n$$ as well.  Otherwise, $v(B)<v(A)=v(A\lor B)$.  This means that $$v((B\to C) \to ((A\lor B)\to C)))=v((B\to C)\to (A \to C)),$$ and therefore $$v((A \to C) \to ((B \to C) \to ((A \lor B) \to C))) = v((A\to C) \to ((B\to C)\to (A \to C))) =n$$ by 3 above.

\item $(A \to B) \to ((A \to (B \to C)) \to (A \to C))$.

It is clear that $v(C)\le v(B\to C)$.  If $v(C)=v(B\to C)$, then $$v((A \to (B \to C)) \to (A \to C)) = v((A \to C) \to (A \to C)) = n,$$ so that $$v((A \to B) \to ((A \to (B \to C)) \to (A \to C)))=n$$ too.  If $v(C)<v(B\to C)$, then $v(B\to C)=n$, which implies $v(B)\le v(C)$.  This in turn implies that $v(A\to B) \le v(A\to C)$, so that $$v((A \to C) \to ((A \to (B \to C)) \to (A \to C))) \le v((A \to B) \to ((A \to (B \to C)) \to (A \to C))).$$  But by 3 above, $$v((A \to C) \to ((A \to (B \to C)) \to (A \to C)))=n,$$ hence $$v((A \to B) \to ((A \to (B \to C)) \to (A \to C)))=n$$ as a result.

\item $(A\to B)\to ((A\to \neg B)\to \neg A)$.

Pick any $C$ such that $v(C)=0$, such as $D\land \neg D$.  Then $v(\neg B) = v(B\to C)$, so that $$v((A\to B)\to ((A\to \neg B)\to \neg A))= v((A\to B)\to ((A\to (B\to C))\to (A\to C)))=n$$ by 7.

\end{enumerate}
\end{proof}
Note that the proofs of the axioms employ some elementary facts, for any wff's $A,B,C$:
\begin{itemize}
\item If $v(B)=n$ or $v(A)=0$, then $v(A\to B)=n$.
\item if $v(B)=0$, then $v(A\to B)=v(\neg A)$.
\item if $v(A)=n$, then $v(A\to B)=v(B)$.
\item $v(B)\le v(A\to B)$.
\item if $v(B)\le v(C)$, then 
\begin{itemize}
\item $v(A\lor B)\le v(A\lor C)$,
\item $v(A\land B)\le v(A\land C)$,
\item $v(A\to B)\le v(A\to C)$,
\item $v(C\to A)\le v(B\to A)$.
\end{itemize}
\end{itemize}
From the facts above, one readily deduces:
\begin{itemize}
\item if $v(B)\le v(C)$, then $v(\neg C)\le v(\neg B)$,
\item if $v(B)=v(C)$, then
\begin{itemize}
\item $v(A\lor B)= v(A\lor C)$,
\item $v(A\land B)= v(A\land C)$,
\item $v(A\to B)= v(A\to C)$,
\item $v(C\to A)= v(B\to A)$,
\item $v(\neg B)=v(\neg C)$.
\end{itemize}
\end{itemize}

%%%%%
%%%%%
\end{document}
