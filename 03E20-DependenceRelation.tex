\documentclass[12pt]{article}
\usepackage{pmmeta}
\pmcanonicalname{DependenceRelation}
\pmcreated{2013-03-22 14:19:25}
\pmmodified{2013-03-22 14:19:25}
\pmowner{CWoo}{3771}
\pmmodifier{CWoo}{3771}
\pmtitle{dependence relation}
\pmrecord{9}{35792}
\pmprivacy{1}
\pmauthor{CWoo}{3771}
\pmtype{Definition}
\pmcomment{trigger rebuild}
\pmclassification{msc}{03E20}
\pmclassification{msc}{05B35}
\pmrelated{LinearIndependence}
\pmrelated{AlgebraicallyDependent}
\pmrelated{Matroid}
\pmrelated{AxiomatizationOfDependence}

% this is the default PlanetMath preamble.  as your knowledge
% of TeX increases, you will probably want to edit this, but
% it should be fine as is for beginners.

% almost certainly you want these
\usepackage{amssymb}
\usepackage{amsmath}
\usepackage{amsfonts}

% used for TeXing text within eps files
%\usepackage{psfrag}
% need this for including graphics (\includegraphics)
%\usepackage{graphicx}
% for neatly defining theorems and propositions
%\usepackage{amsthm}
% making logically defined graphics
%%%\usepackage{xypic}

% there are many more packages, add them here as you need them

% define commands here
\begin{document}
Let $X$ be a set. A (binary) relation $\prec$ between an element $a$ of $X$ and a subset $S$ of $X$ is called a \emph{dependence relation}, written $a \prec S$, when the following conditions are satisfied:
\begin{enumerate}
\item
if $a \in S$, then $a \prec S$;
\item
if $a \prec S$, then there is a finite subset $S_0$ of $S$, such that $a \prec S_0$;
\item
if $T$ is a subset of $X$ such that $b \in S$ implies $b \prec T$, then $a \prec S$ implies $a \prec T$;
\item
if $a \prec S$ but $a \not\prec S-\lbrace b \rbrace$ for some $b \in S$, then $b \prec (S-\lbrace b \rbrace)\cup\lbrace a \rbrace$.
\end{enumerate}

Given a \emph{dependence relation} $\prec$ on $X$,  a subset $S$ of $X$ is said to be \emph{independent} if $a \not\prec S - \lbrace a \rbrace$ for all $a \in S$.  If $S \subseteq T$, then $S$ is said to \emph{span} $T$ if $t \prec S$ for every $t \in T$.  $S$ is said to be a \emph{basis} of $X$ if $S$ is \emph{independent} and $S$ \emph{spans} $X$.

\vspace{8mm}

\textbf{Remark.}  If $X$ is a non-empty set with a \emph{dependence relation} $\prec$, then $X$ always has a \emph{basis} with respect to $\prec$.  Furthermore, any two {\em \PMlinkescapetext{bases}} of $X$ have the same cardinality.

\vspace{8mm}

\textbf{Examples:}
\begin{itemize}
\item
Let $V$ be a vector space over a field $F$.  The relation $\prec$, defined by $\upsilon \prec S$ if $\upsilon$ is in the subspace \PMlinkescapetext{spanned by} $S$, is a dependence relatoin.  This is equivalent to the definition of {\em \PMlinkname{linear dependence}{LinearIndependence}}.
\item
Let $K$ be a field extension of $F$.  Define $\prec$ by $\alpha \prec S$ if $\alpha$ is algebraic over $F(S)$. Then $\prec$ is a dependence relation.  This is equivalent to the definition of \emph{algebraic dependence}.  
\end{itemize}
%%%%%
%%%%%
\end{document}
