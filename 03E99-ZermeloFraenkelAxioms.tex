\documentclass[12pt]{article}
\usepackage{pmmeta}
\pmcanonicalname{ZermeloFraenkelAxioms}
\pmcreated{2013-03-22 11:47:51}
\pmmodified{2013-03-22 11:47:51}
\pmowner{mathcam}{2727}
\pmmodifier{mathcam}{2727}
\pmtitle{Zermelo-Fraenkel axioms}
\pmrecord{20}{30317}
\pmprivacy{1}
\pmauthor{mathcam}{2727}
\pmtype{Axiom}
\pmcomment{trigger rebuild}
\pmclassification{msc}{03E99}
\pmsynonym{Zermelo-Fraenkel set theory}{ZermeloFraenkelAxioms}
\pmsynonym{ZFC}{ZermeloFraenkelAxioms}
\pmsynonym{ZF}{ZermeloFraenkelAxioms}
%\pmkeywords{set theory}
\pmrelated{AxiomOfChoice}
\pmrelated{RussellsParadox}
\pmrelated{VonNeumannOrdinal}
\pmrelated{Axiom}
\pmrelated{ContinuumHypothesis}
\pmrelated{GeneralizedContinuumHypothesis}
\pmrelated{SetTheory}
\pmrelated{VonNeumannBernausGodelSetTheory}
\pmrelated{Set}

\endmetadata

\usepackage{amssymb}
\usepackage{amsmath}
\usepackage{amsfonts}
\begin{document}
Ernst Zermelo and Abraham Fraenkel proposed the following axioms as a \PMlinkescapetext{foundation} for what is now called Zermelo-Fraenkel set theory, or ZF.  If this set of axioms are accepted along with the Axiom of Choice, it is often denoted ZFC.

\begin{itemize}
\item \emph{Equality of sets}: If $X$ and $Y$ are sets, and $x \in X$ iff $x \in Y$, then $X = Y$.
\item \emph{Pair set}: If $X$ and $Y$ are sets, then there is a set $Z$ containing only $X$ and $Y$.
\item \emph{\PMlinkname{Union}{Union} over a set}: If $X$ is a set, then there exists a set that contains every element of each $x \in X$.
\item \emph{\PMlinkescapetext{Axiom of power set}}: If $X$ is a set, then there exists a set $\mathcal{P}(x)$ with the property that $Y \in \mathcal{P}(x)$ iff any element $y \in Y$ is also in $X$.
\item \emph{Replacement axiom}: Let $F(x,y)$ be some formula.  If, for all $x$, there is exactly one $y$ such that $F(x,y)$ is true, then for any set $A$ there exists a set $B$ with the property that $b \in B$ iff there exists some $a \in A$ such that $F(a,b)$ is true.
\item \emph{\PMlinkescapetext{Regularity axiom}}: Let $F(x)$ be some formula.  If there is some $x$ that makes $F(x)$ true, then there is a set $Y$ such that $F(Y)$ is true, but for no $y \in Y$ is $F(y)$ true.
\item \emph{Existence of an infinite set}: There exists a non-empty set $X$ with the property that, for any $x \in X$, there is some $y \in X$ such that $x \subseteq y$ but $x \neq y$.
\item \emph{\PMlinkescapetext{Separation axiom}}:  If $X$ is a set and $P$ is a condition on sets, there exists a set $Y$ whose members are precisely the members of $X$ satisfying $P$.  (This axiom is also occasionally referred to as the \emph{\PMlinkescapetext{subset axiom}}).
\end{itemize}
%%%%%
%%%%%
%%%%%
%%%%%
\end{document}
