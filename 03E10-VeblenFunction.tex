\documentclass[12pt]{article}
\usepackage{pmmeta}
\pmcanonicalname{VeblenFunction}
\pmcreated{2013-03-22 13:29:10}
\pmmodified{2013-03-22 13:29:10}
\pmowner{Henry}{455}
\pmmodifier{Henry}{455}
\pmtitle{Veblen function}
\pmrecord{4}{34058}
\pmprivacy{1}
\pmauthor{Henry}{455}
\pmtype{Definition}
\pmcomment{trigger rebuild}
\pmclassification{msc}{03E10}
\pmclassification{msc}{03F15}
\pmdefines{strongly critical}
\pmdefines{Feferman-Schutte ordinal}

% this is the default PlanetMath preamble.  as your knowledge
% of TeX increases, you will probably want to edit this, but
% it should be fine as is for beginners.

% almost certainly you want these
\usepackage{amssymb}
\usepackage{amsmath}
\usepackage{amsfonts}

% used for TeXing text within eps files
%\usepackage{psfrag}
% need this for including graphics (\includegraphics)
%\usepackage{graphicx}
% for neatly defining theorems and propositions
%\usepackage{amsthm}
% making logically defined graphics
%%%\usepackage{xypic}

% there are many more packages, add them here as you need them

% define commands here
%\PMlinkescapeword{theory}
\begin{document}
The \emph{Veblen function} is used to obtain larger ordinal numbers than those provided by exponentiation.  It builds on a hierarchy of closed and unbounded classes:
\begin{itemize}
\item $Cr(0)$ is the additively indecomposable numbers, $\mathbb{H}$
\item $Cr(Sn)=Cr(n)^\prime$ the set of fixed points of the enumerating function of $Cr(n)$
\item $Cr(\lambda)=\bigcap_{\alpha<\lambda} Cr(\alpha)$
\end{itemize}

The Veblen function $\varphi_\alpha\beta$ is defined by setting $\varphi_\alpha$ equal to the enumerating function of $Cr(\alpha)$.

We call a number $\alpha$ \emph{strongly critical} if $\alpha\in Cr(\alpha)$.  The class of strongly critical ordinals is written $\mathbf{SC}$, and the enumerating function is written $f_{\mathbf{SC}}(\alpha)=\Gamma_\alpha$.

$\Gamma_0$, the first strongly critical ordinal, is also called the \emph{Feferman-Schutte} ordinal.
%%%%%
%%%%%
\end{document}
