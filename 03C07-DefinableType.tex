\documentclass[12pt]{article}
\usepackage{pmmeta}
\pmcanonicalname{DefinableType}
\pmcreated{2013-03-22 13:29:26}
\pmmodified{2013-03-22 13:29:26}
\pmowner{Timmy}{1414}
\pmmodifier{Timmy}{1414}
\pmtitle{definable type}
\pmrecord{4}{34064}
\pmprivacy{1}
\pmauthor{Timmy}{1414}
\pmtype{Definition}
\pmcomment{trigger rebuild}
\pmclassification{msc}{03C07}
\pmclassification{msc}{03C45}
%\pmkeywords{definable}
%\pmkeywords{type}
\pmrelated{Type2}
\pmdefines{definable type}
\pmdefines{defining scheme}

\endmetadata

% this is the default PlanetMath preamble.  as your knowledge
% of TeX increases, you will probably want to edit this, but
% it should be fine as is for beginners.

% almost certainly you want these
\usepackage{amssymb}
\usepackage{amsmath}
\usepackage{amsfonts}

% used for TeXing text within eps files
%\usepackage{psfrag}
% need this for including graphics (\includegraphics)
%\usepackage{graphicx}
% for neatly defining theorems and propositions
%\usepackage{amsthm}
% making logically defined graphics
%%%\usepackage{xypic}

% there are many more packages, add them here as you need them

% define commands here
\begin{document}
Let $M$ be a first order structure. 
Let $A$ and $B$ be sets of parameters from $M$.
Let $p$ be a complete $n$-type over $B$. 
Then we say that $p$ is an {\em $A$-definable} type iff 
for every formula $\psi(\bar{x},\bar{y})$ with ln$(\bar{x})=n$, 
there is some formula $d\psi(\bar{y},\bar{z})$ and some parameters $\bar{a}$ from $A$ so that
for any $\bar{b}$ from $B$ we have $\psi(\bar{x},\bar{b}) \in p$ iff $M \models d\psi(\bar{b},\bar{a})$.

Note that if $p$ is a type over the model $M$ then this condition is equivalent to showing that $\{\bar{b} \in M:\psi(\bar{x},\bar{b}) \in M\}$ is an $A$-definable set.

\medskip

For $p$ a type over $B$, we say $p$ is {\em definable} if it is $B$-definable.

\medskip

If $p$ is definable, we call $d\psi$ the {\em defining formula} for $\psi$, and the function $\psi \mapsto d\psi$ a defining scheme for $p$.
%%%%%
%%%%%
\end{document}
