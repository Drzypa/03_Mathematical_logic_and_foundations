\documentclass[12pt]{article}
\usepackage{pmmeta}
\pmcanonicalname{AnInjectionBetweenTwoFiniteSetsOfTheSameCardinalityIsBijective}
\pmcreated{2013-03-22 15:10:20}
\pmmodified{2013-03-22 15:10:20}
\pmowner{alozano}{2414}
\pmmodifier{alozano}{2414}
\pmtitle{an injection between two finite sets of the same cardinality is bijective}
\pmrecord{6}{36923}
\pmprivacy{1}
\pmauthor{alozano}{2414}
\pmtype{Theorem}
\pmcomment{trigger rebuild}
\pmclassification{msc}{03-00}
%\pmkeywords{bijective}
%\pmkeywords{injective}
\pmrelated{SchroederBernsteinTheorem}
\pmrelated{ProofOfSchroederBernsteinTheorem}
\pmrelated{OneToOneFunctionFromOntoFunction}

\endmetadata

% this is the default PlanetMath preamble.  as your knowledge
% of TeX increases, you will probably want to edit this, but
% it should be fine as is for beginners.

% almost certainly you want these
\usepackage{amssymb}
\usepackage{amsmath}
\usepackage{amsthm}
\usepackage{amsfonts}

% used for TeXing text within eps files
%\usepackage{psfrag}
% need this for including graphics (\includegraphics)
%\usepackage{graphicx}
% for neatly defining theorems and propositions
%\usepackage{amsthm}
% making logically defined graphics
%%%\usepackage{xypic}

% there are many more packages, add them here as you need them

% define commands here

\newtheorem{thm}{Theorem}
\newtheorem{defn}{Definition}
\newtheorem{prop}{Proposition}
\newtheorem*{lemma}{Lemma}
\newtheorem{cor}{Corollary}

\theoremstyle{definition}
\newtheorem{exa}{Example}

% Some sets
\newcommand{\Nats}{\mathbb{N}}
\newcommand{\Ints}{\mathbb{Z}}
\newcommand{\Reals}{\mathbb{R}}
\newcommand{\Complex}{\mathbb{C}}
\newcommand{\Rats}{\mathbb{Q}}
\newcommand{\Gal}{\operatorname{Gal}}
\newcommand{\Cl}{\operatorname{Cl}}
\begin{document}
\begin{lemma}
Let $A,B$ be two finite sets of the same cardinality. If $f\colon A \to B$ is an injective function then $f$ is bijective.
\end{lemma}
\begin{proof}
In order to prove the lemma, it suffices to show that if $f$ is an injection then the cardinality of $f(A)$ and $A$ are equal. We prove this by induction on $n=\text{card}(A)$. The case $n=1$ is trivial. Assume that the lemma is true for sets of cardinality $n$ and let $A$ be a set of cardinality $n+1$. Let $a\in A$ so that $A_1=A-\{a\}$ has cardinality $n$. Thus, $f(A_1)$ has cardinality $n$ by the induction hypothesis. Moreover, $f(a)\notin f(A_1)$ because $a\notin A_1$ and $f$ is injective. Therefore:
$$f(A)=f(\{a\}\cup A_1)=\{f(a)\}\cup f(A_1)$$
and the set $\{f(a)\}\cup f(A_1)$ has cardinality $1+n$, as desired.
\end{proof}
%%%%%
%%%%%
\end{document}
