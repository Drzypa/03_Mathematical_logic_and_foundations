\documentclass[12pt]{article}
\usepackage{pmmeta}
\pmcanonicalname{ElementaryEmbedding}
\pmcreated{2013-03-22 13:00:29}
\pmmodified{2013-03-22 13:00:29}
\pmowner{CWoo}{3771}
\pmmodifier{CWoo}{3771}
\pmtitle{elementary embedding}
\pmrecord{5}{33389}
\pmprivacy{1}
\pmauthor{CWoo}{3771}
\pmtype{Definition}
\pmcomment{trigger rebuild}
\pmclassification{msc}{03C99}
\pmsynonym{elementary monomorphism}{ElementaryEmbedding}
\pmdefines{elementary substructure}
\pmdefines{elementary extension}
\pmdefines{elementary chain}

\endmetadata

% this is the default PlanetMath preamble.  as your knowledge
% of TeX increases, you will probably want to edit this, but
% it should be fine as is for beginners.

% almost certainly you want these
\usepackage{amssymb}
\usepackage{amsmath}
\usepackage{amsfonts}

% used for TeXing text within eps files
%\usepackage{psfrag}
% need this for including graphics (\includegraphics)
%\usepackage{graphicx}
% for neatly defining theorems and propositions
%\usepackage{amsthm}
% making logically defined graphics
%%%\usepackage{xypic}

% there are many more packages, add them here as you need them

% define commands here
%\PMlinkescapeword{theory}
\begin{document}
Let $\tau$ be a signature and $\mathcal{A}$ and $\mathcal{B}$ be two structures for $\tau$ such that $f:\mathcal{A}\to \mathcal{B}$ is an embedding.  Then $f$ is said to be \emph{elementary} if for every first-order formula $\phi \in F(\tau)$, we have $$\mathcal{A}\vDash\phi \quad \mbox{iff} \quad \mathcal{B}\vDash \phi.$$
In the expression above, $\mathcal{A}\vDash\phi$ means: if we write $\phi=\phi(x_1,\ldots,x_n)$ where the free variables of $\phi$ are all in $\lbrace x_1,\ldots,x_n\rbrace$, then $\phi(a_1,\ldots,a_n)$ holds in $\mathcal{A}$ for any $a_i\in \mathcal{A}$ (the underlying universe of $\mathcal{A}$).

If $\mathcal{A}$ is a substructure of $\mathcal{B}$ such that the inclusion homomorphism is an elementary embedding, then we say that $\mathcal{A}$ is an \emph{elementary substructure} of $\mathcal{B}$, or that $\mathcal{B}$ is an elementary extension of $\mathcal{A}$.

\textbf{Remark}.  A chain $\mathcal{A}_1\subseteq \mathcal{A}_2\subseteq \cdots \subseteq \mathcal{A}_n \subseteq \cdots$ of $\tau$-structures is called an \emph{elementary chain} if $\mathcal{A}_i$ is an elementary substructure of $\mathcal{A}_{i+1}$ for each $i=1,2,\ldots$.  It can be shown (Tarski and Vaught) that $$\bigcup_{i<\omega} \mathcal{A}_i$$ is a $\tau$-structure that is an elementary extension of $\mathcal{A}_i$ for every $i$.

%If $\mathcal{A}$ and $\mathcal{B}$ are models of $\mathcal{L}$ such that for each $t\in T$, $A_t\subseteq B_t$, then we say $\mathcal{B}$ is an \emph{elementary extension} of $\mathcal{A}$, or, equivalently, $\mathcal{A}$ is an \emph{elementary substructure} of $\mathcal{B}$ if, whenever $\phi$ is a formula of $\mathcal{L}$ with free variables included in $x_1,\ldots,x_n$ (of types $t_1,\ldots,t_n$) and $a_1,\ldots,a_n$ are such that $a_i\in t_i$ for each $i\leq n$ then:

%$$\mathcal{A}\vDash \phi(a_1,\ldots,a_n)\text{iff}\mathcal{B}\vDash \phi(a_1,\ldots,a_n)$$

%If $\mathcal{A}$ and $\mathcal{B}$ are models of $\mathcal{L}$ then a collection of one-to-one functions $f_t:A_t\rightarrow B_t$ for each $t\in T$ is an \emph{elementary embedding} of $\mathcal{A}$ if whenever $\phi$ is a formula of type $\mathcal{L}$ with free variables included in $x_1,\ldots,x_n$ (of types $t_1,\ldots,t_n$) and $a_1,\ldots,a_n$ are such that $a_i\in t_i$ for each $i\leq n$ then:

%$$\mathcal{A}\vDash \phi(a_1,\ldots,a_n)\text{iff}\mathcal{B}\vDash \phi(f_{t_1}(a_1),\ldots,f_{t_n}(a_n))$$
%%%%%
%%%%%
\end{document}
