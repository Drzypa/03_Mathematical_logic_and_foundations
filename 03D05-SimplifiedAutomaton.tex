\documentclass[12pt]{article}
\usepackage{pmmeta}
\pmcanonicalname{SimplifiedAutomaton}
\pmcreated{2013-03-22 19:01:51}
\pmmodified{2013-03-22 19:01:51}
\pmowner{CWoo}{3771}
\pmmodifier{CWoo}{3771}
\pmtitle{simplified automaton}
\pmrecord{8}{41903}
\pmprivacy{1}
\pmauthor{CWoo}{3771}
\pmtype{Definition}
\pmcomment{trigger rebuild}
\pmclassification{msc}{03D05}
\pmclassification{msc}{68Q45}
\pmsynonym{trim automaton}{SimplifiedAutomaton}
\pmsynonym{coaccessible}{SimplifiedAutomaton}
\pmrelated{ReducedAutomaton}
\pmdefines{reachable}
\pmdefines{accessible}
\pmdefines{live}
\pmdefines{useful}

\endmetadata

\usepackage{amssymb,amscd}
\usepackage{amsmath}
\usepackage{amsfonts}
\usepackage{mathrsfs}

% used for TeXing text within eps files
%\usepackage{psfrag}
% need this for including graphics (\includegraphics)
\usepackage{graphicx}
% for neatly defining theorems and propositions
\usepackage{amsthm}
% making logically defined graphics
%%\usepackage{xypic}
\usepackage{pst-plot}

% define commands here
\newcommand*{\abs}[1]{\left\lvert #1\right\rvert}
\newtheorem{prop}{Proposition}
\newtheorem{thm}{Theorem}
\newtheorem{ex}{Example}
\newcommand{\real}{\mathbb{R}}
\newcommand{\pdiff}[2]{\frac{\partial #1}{\partial #2}}
\newcommand{\mpdiff}[3]{\frac{\partial^#1 #2}{\partial #3^#1}}
\begin{document}
Word acceptance is the primary function of an automaton.  When a word is fed into an automaton at an initial state, it is accepted exactly when a final state is reached upon reading the last symbol of the word.  So when a state does not contribute in anyway to word acceptance, it may be tossed out without changing the accepting power of the automaton.

\textbf{Definition}.  Let $A=(S,\Sigma,\delta,I,F)$ be an automaton.  A state $s\in S$ is said to be \emph{reachable} from a state $t\in S$, if for some word $u$ over $\Sigma$, $t\in \delta(s,u)$.  A state is said to be \emph{accessible} if it is reachable from a start state.  A state is \emph{live} or \emph{coaccessible} if a final state can be reached from it.  If a state is both accessible and live, then it is called \emph{useful}.

Clearly, every start state is accessible, and every final state is live.  The following is the state diagram of an automaton that illustrates the concepts defined above.

\begin{figure}[!h]
\centering
\scalebox{1.0}{\includegraphics{automaton.eps}}
\end{figure}

States $\sigma,p,r,s$ are accessible, while states $\sigma,p,q$ are live.  So the useful states are $\sigma$ and $p$.

The automaton $A$ is said to be \emph{accessible} if every state in $S$ is accessible, and is \emph{live} if every state is live.  $A$ is called \emph{simplified} or \emph{trim} if it is both accessible and live (every state is useful).

\begin{prop} An automaton is \PMlinkname{equivalent}{EquivalentAutomata} to a simplified automaton iff it accepts at least one word.  \end{prop}
\begin{proof}  Suppose an automaton $A$ is equivalent to a simplified automaton $A'$.  Then the set of start states of $A'$ is non-empty, say containing a state $q$.  So $q$ is live, which means there is a word $u$ such that $\delta(p,u)$ contains a final state, or that $u\in L(A')=L(A)$.

Conversely, suppose an automaton $A=(S,\Sigma,\delta,I,F)$ accepts a word $u$.  This means that there is a start state $q$ and a final state $r$ such that $r\in \delta(q,u)$, so that both $q$ and $r$ are useful.  Now, let $S'$ be the subset (of $S$) all useful states in $A$, and let $I'=I\cap S'$, $F'=F\cap S'$.  Then neither $I'$ nor $F'$ is empty.  Finally, let $\delta'$ be the restriction of $\delta$ on the set $S'\times \Sigma$, such that $$\delta'(s,a) := \delta(s,a)\cap S'.$$  Then the five-tuple $A':=(S',\Sigma,\delta',I',F')$ is an automaton.  Obviously, $L(A')\subseteq L(A)$.  On the other hand, the word $u$ above is accepted by $A'$.  Since $u$ is arbitrary, we see that $L(A')=L(A)$ as a result.
\end{proof}

\textbf{Remark}.  If we allow the start state set of an automaton to be empty, then every automaton can be simplified, whether or not it accepts any words.

In the example above, the automaton is equivalent to the following simplified automaton:

\begin{figure}[!h]
\centering
\scalebox{1.0}{\includegraphics{automaton2.eps}}
\end{figure}

However, the construction in the proof shows that, given a deterministic automaton $A$, the simplified automaton $A'$ may not be determinstic.  In the example above, $A'$ is not deterministic because $\delta(\sigma,b)=\varnothing$.  With respect to deterministic automata, the best we can do is the following:

\begin{prop} A deterministic automaton is \PMlinkname{equivalent}{EquivalentAutomata} to a deterministic accessible automaton.  \end{prop}
The proof of this is very similar to the one given above, except $S',I'$, and $F'$ now consist of all accessible states, and $\delta'$ is just a restriction on the domain of $\delta$, since if a state $s$ is accessible, so is any state in $\delta(s,a)$ for any $a\in \Sigma$.

Again, referring the example above, $A$ is deterministic, and it is equivalent to the following deterministic accessible automaton:

\begin{figure}[!h]
\centering
\scalebox{1.0}{\includegraphics{automaton1.eps}}
\end{figure}

The language accepted by the automaton is represented by the regular expression $a(ab)^*$.
%%%%%
%%%%%
\end{document}
