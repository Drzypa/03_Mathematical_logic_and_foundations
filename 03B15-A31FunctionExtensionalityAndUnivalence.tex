\documentclass[12pt]{article}
\usepackage{pmmeta}
\pmcanonicalname{A31FunctionExtensionalityAndUnivalence}
\pmcreated{2013-11-09 5:56:49}
\pmmodified{2013-11-09 5:56:49}
\pmowner{PMBookProject}{1000683}
\pmmodifier{PMBookProject}{1000683}
\pmtitle{A.3.1 Function extensionality and univalence}
\pmrecord{1}{}
\pmprivacy{1}
\pmauthor{PMBookProject}{1000683}
\pmtype{Feature}
\pmclassification{msc}{03B15}

\endmetadata

\usepackage{xspace}
\usepackage{amssyb}
\usepackage{amsmath}
\usepackage{amsfonts}
\usepackage{amsthm}
\makeatletter
\newcommand{\funext}{\mathsf{funext}}
\newcommand{\happly}{\mathsf{happly}}
\newcommand{\idtoeqv}{\ensuremath{\mathsf{idtoeqv}}\xspace}
\newcommand{\isequiv}{\ensuremath{\mathsf{isequiv}}}
\newcommand{\oftp}[3]{#1 \vdash #2 : #3}
\def\tprd#1{\@tprd{#1}\@ifnextchar\bgroup{\tprd}{}}
\def\@tprd#1{\mathchoice{{\textstyle\prod_{(#1)}}}{\prod_{(#1)}}{\prod_{(#1)}}{\prod_{(#1)}}}
\newcommand{\univalence}{\ensuremath{\mathsf{univalence}}\xspace} 
\newcommand{\UU}{\ensuremath{\mathcal{U}}\xspace}
\let\autoref\cref
\makeatother

\begin{document}
There are two basic ways of introducing axioms which do not introduce new syntax or judgmental equalities (function extensionality and univalence are of this form):
either add a primitive constant to inhabit the axiom, or prove all theorems which depend on the axiom by hypothesizing a variable that inhabits the axiom, cf.\ \PMlinkname{\S 1.1}{11typetheoryversussettheory}.
While these are essentially equivalent, we opt for the former approach because we feel that the axioms of homotopy type theory are an essential part of the core theory.

\index{function extensionality}%
\autoref{axiom:funext} is formalized by introduction of a constant $\funext$ which
asserts that $\happly$ is an equivalence:
%
\begin{mathparpagebreakable}
  \inferrule*[right=$\Pi$-\textsc{ext}]
  {\oftp\Gamma{f}{\tprd{x:A} B} \\
   \oftp\Gamma{g}{\tprd{x:A} B}}
  {\oftp\Gamma{\funext(f,g)}{\isequiv(\happly_{f,g})}}
\end{mathparpagebreakable}
%
The definitions of $\happly$ and $\isequiv$ can be found in~\eqref{eq:happly} and
\PMlinkname{\S 4.5}{45onthedefinitionofequivalences}, respectively.

\index{univalence axiom}%
\autoref{axiom:univalence} is formalized in a similar fashion, too:
%
\begin{mathparpagebreakable}
  \inferrule*[right=$\UU_i$-\textsc{univ}]
  {\oftp\Gamma{A}{\UU_i} \\
   \oftp\Gamma{B}{\UU_i}}
  {\oftp\Gamma{\univalence(A,B)}{\isequiv(\idtoeqv_{A,B})}}
\end{mathparpagebreakable}
%
The definition of $\idtoeqv$ can be found in~\eqref{eq:uidtoeqv}.


\end{document}
