\documentclass[12pt]{article}
\usepackage{pmmeta}
\pmcanonicalname{DerivationsInNaturalDeduction}
\pmcreated{2013-03-22 19:32:08}
\pmmodified{2013-03-22 19:32:08}
\pmowner{CWoo}{3771}
\pmmodifier{CWoo}{3771}
\pmtitle{derivations in natural deduction}
\pmrecord{36}{42513}
\pmprivacy{1}
\pmauthor{CWoo}{3771}
\pmtype{Definition}
\pmcomment{trigger rebuild}
\pmclassification{msc}{03F03}
\pmsynonym{derivation}{DerivationsInNaturalDeduction}
\pmdefines{deduction}
\pmdefines{assumption}
\pmdefines{discharged assumption}
\pmdefines{derivable}
\pmdefines{derivability relation}

\endmetadata

\usepackage{amssymb,amscd}
\usepackage{amsmath}
\usepackage{amsfonts}
\usepackage{mathrsfs}
\usepackage{proof}
\usepackage{bussproofs}
\usepackage{pstricks,pst-tree}

% used for TeXing text within eps files
%\usepackage{psfrag}
% need this for including graphics (\includegraphics)
%\usepackage{graphicx}
% for neatly defining theorems and propositions
\usepackage{amsthm}
% making logically defined graphics
%%\usepackage{xypic}
\usepackage{pst-plot}

% define commands here
\newcommand*{\abs}[1]{\left\lvert #1\right\rvert}
\newtheorem{prop}{Proposition}
\newtheorem{thm}{Theorem}
\newtheorem{ex}{Example}
\newcommand{\real}{\mathbb{R}}
\newcommand{\pdiff}[2]{\frac{\partial #1}{\partial #2}}
\newcommand{\mpdiff}[3]{\frac{\partial^#1 #2}{\partial #3^#1}}

\begin{document}
Derivations, or deductions, for natural deduction systems, unlike the axiom systems, which are finite sequences of well-formed formulas (wff's) in a logic, are finite tress of wff's.  More precisely, they are rooted tress whose vertices are labeled by wff's of the logic.  For example,
$$
\pstree[treemode=U,radius=3pt,levelsep=7ex]{\Tc{3pt}~[tnpos=b,tndepth=0pt,radius=4pt]{A\to A}}
{
\pstree{\TC~[tnpos=r]{A}}
{\pstree{\TC~[tnpos=r]{A\land A}}{\TC~[tnpos=l]{A}\TC~[tnpos=r]{A}}}
}
$$
is a labeled rooted tree, whose root is labeled by the wff $A \to A$.  In natural deduction, the trees are presented a little bit differently, replacing each edge with a horizontal bar separating the neighboring vertices.  Thus, the tree above looks like
\begin{prooftree}
\AxiomC{$A$}
\AxiomC{$A$}
\BinaryInfC{$A \land A$}
\UnaryInfC{$A$}
\UnaryInfC{$A \to A$}
\end{prooftree}

Derivations are defined recursively: from atomic formulas as single-vertex trees via rules of inference of the system.  To see how this is done exactly, recall that a rule of inference generally takes the following form:
$$ \infer[R]{Y}{X_1 \quad \cdots \quad X_i \quad \cdots \quad \infer*{X_j}{[A_j]} \quad \cdots \quad X_n} $$
where $R$ is the name of the rule, and the $X_k$'s, $A_k$'s, and $Y$ are wff's of the logic.  Each $X_k$ is a \emph{premise}, and $Y$ is the \emph{conclusion}.  Some of the premises (like $X_j$ above) have optional \emph{assumptions} ($A_j$) which can be \emph{cancelled} or \emph{discharged} once the conclusion is reached.  Just exactly what ``cancelled'' or ``discharged'' means will be clear shortly.

Now we are ready for the formal definition of derivations.

\textbf{Definition}.  A labeled rooted tree over a natural deduction system is a rooted tree whose labels are of the form $A$ or $[A]$, where $A$ is a wff of the logical system.

\textbf{Definition}.  The set $\mathbb{D}$ of \emph{derivations} of a natural deduction system the smallest set of labeled rooted trees over the wff's of system, such that
\begin{enumerate}
\item for each propositional variable $p$, a one-vertex tree with label $p$ is in $\mathbb{D}$
\item for every inference rule $R$ where 
\begin{itemize}
\item $X_1,\ldots, X_n$ are premises
\item $Y$ is the conclusion
\item a subset $\mathcal{X}$ (may be empty) of the premises where each $X\in \mathcal{X}$ has an optional assumption $A_X$ that can be discharged when conclusion $Y$ is reached
\end{itemize}
if $T_1,\ldots, T_n$ are in $\mathdd{D}$ where each $X_i$ is the label of the root of $T_i$, then the rooted tree $T$, obtained by
\begin{enumerate}
\item transforming each of the trees $T$ whose root has label $X\in \mathcal{X}$ to a tree such that some (or none) of the leaves in $T$ with label $A_X$ are converted to leaves with label $[A_X]$,
\item then combining the resulting trees, as well as the rest of the other trees by joining their roots to a new vertex as the root of $T$, with label is $Y$
\end{enumerate}
is also in $\mathbb{D}$.  Graphically, $T$ can be represented
\begin{prooftree}
\alwaysNoLine
\AxiomC{$T_1$}
\UnaryInfC{$X_1$}
\AxiomC{$\cdots$}
\AxiomC{[$A_{X_n}$]}
\UnaryInfC{$T_n$}
\UnaryInfC{$X_n$}
\alwaysSingleLine
\TrinaryInfC{$Y$}
\end{prooftree}
\end{enumerate}

For example, using the following rule of inference in classical propositional logic, called $\to I$
$$ \infer[\to I]{A\to B}{\infer*{B}{[A]}} $$
we have the following derivations
$$\infer{p\to q}{q}\qquad \qquad \infer{q\to (p\to q)}{\infer{p\to q}{[q]}}$$
The first one is a derivation because $q$ is a propositional variable by $(a)$, and since the rule does not require the presence of an assumption (it is optional), the whole tree is a derivation by $(b)$.  As a result, by $(b)$ once more, the second one is also a derivation, where the label $q$ of the leaf (top vertex) has been changed to $[q]$.

From now on, if no confusion arises, the vertices of a derivation are identified with their labels.

\textbf{Definition}.  Given a derivation $T$ in a natural deduction logical system, the \emph{assumptions} of $T$ are leaves of the form $A$, and \emph{discharged assumptions} of $T$ if they are of the form $[A]$.  The \emph{conclusion} of $T$ is the root of $T$.  If $\Sigma$ is a set of wff's and $A$ a wff, we write $$\Sigma \vdash A$$ if there is a derivation $T$ whose assumptions are elements of $\Sigma$, and whose conclusion is $A$.  We say that $A$ is \emph{derivable from} $\Sigma$.  The symbol $\vdash$ is the \emph{derivability relation} between sets of wff's and wff's.  We say that $A$ is a \emph{theorem} (of the logical system) if it is derivable from the empty set, and write $$\vdash A.$$
In other words, $A$ is a theorem if there is a derivation of $A$ whose assumptions are all discharged.  For example, from the two derivations above, we see that $q \vdash p\to q$ and $\vdash q\to (p\to q)$.

%%%%%
%%%%%
\end{document}
