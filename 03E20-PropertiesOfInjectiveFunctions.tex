\documentclass[12pt]{article}
\usepackage{pmmeta}
\pmcanonicalname{PropertiesOfInjectiveFunctions}
\pmcreated{2013-03-22 16:40:20}
\pmmodified{2013-03-22 16:40:20}
\pmowner{rspuzio}{6075}
\pmmodifier{rspuzio}{6075}
\pmtitle{properties of injective functions}
\pmrecord{22}{38879}
\pmprivacy{1}
\pmauthor{rspuzio}{6075}
\pmtype{Theorem}
\pmcomment{trigger rebuild}
\pmclassification{msc}{03E20}
\pmclassification{msc}{03E99}

% this is the default PlanetMath preamble.  as your knowledge
% of TeX increases, you will probably want to edit this, but
% it should be fine as is for beginners.

% almost certainly you want these
\usepackage{amssymb}
\usepackage{amsmath}
\usepackage{amsfonts}

% used for TeXing text within eps files
%\usepackage{psfrag}
% need this for including graphics (\includegraphics)
%\usepackage{graphicx}
% for neatly defining theorems and propositions
\usepackage{amsthm}
% making logically defined graphics
%%%\usepackage{xypic}

% there are many more packages, add them here as you need them

% define commands here
\newtheorem{theorem}{Theorem}
\begin{document}
\begin{theorem}
Suppose $A,B,C$ are sets and $f\colon A\to B$, $g\colon B\to C$
are injective functions. Then the composition $g\circ f$ is an injection.
\end{theorem}

\begin{proof}
Suppose that $(g \circ f) (x) = (g \circ f) (y)$ for some $x, y \in A$.
By definition of composition, $g(f(x)) = g(f(y))$.  Since $g$, is 
assumed injective, $f(x) = f(y)$.  Since $f$ is also assumed injective,
$x = y$.  Therefore, $(g \circ f) (x) = (g \circ f) (y)$ implies 
$x=y$, so $g \circ f$ is injective.
\end{proof}

\begin{theorem}
Suppose $f\colon A\to B$ is an injection, and $C\subseteq A$. Then
the restriction  $f|_C \colon C\to B$ is an injection.
\end{theorem}

\begin{proof}
Suppose $(f|_C) (x) = (f|_C) (y)$ for some $x,y \in C$. By definition
of restriction, $f(x) = f(y)$.  Since $f$ is assumed injective this,
in turn, implies that $x = y$.  Thus, $f|_C$ is also injective.
\end{proof}

\begin{theorem}
Suppose $A,B,C$ are sets and that the functions $f\colon A\to B$ and 
$g\colon B\to C$ are such that $g \circ f$ is injective. Then $f$ is 
injective.
\end{theorem}

\begin{proof} ({\em direct proof})
Let $x, y \in A$ be such that $f(x) = f(y)$. Then $g(f(x)) = g(f(y))$. 
But as $g \circ f$ is injective, this implies that $x = y$, hence
$f$ is also injective.
\end{proof}

\begin{proof} ({\em proof by contradiction})
Suppose that $f$ were not injective.  Then there would exist $x,y \in A$
such that $f(x) = f(y)$ but $x \neq y$.  Composing with $g$, we would
then have $g(f(x)) = g(f(y))$.  However, since $g \circ f$ is assumed
injective, this would imply that $x = y$, which contradicts a previous
statement.  Hence $f$ must be injective.
\end{proof}

\begin{theorem}
Suppose $f\colon A\to B$ is an injection.  Then, for all $C\subseteq A$, it is the case that
$f^{-1} (f (C)) = C$.\footnote{In this equation, the symbols ``$f$'' and 
``$f^{-1}$'' as applied to sets denote the direct image and the inverse 
image, respectively}
\end{theorem}

\begin{proof}
It follows from the definition of $f^{-1}$ that $C \subseteq f^{-1}  
(f (C))$, whether or not $f$ happens to be injective.  Hence, all that
need to be shown is that $f^{-1} (f (C)) \subseteq C$.  Assume the
contrary.  Then there would exist $x \in f^{-1} (f (C))$ such that
$x \notin C$.  By defintion, $x \in f^{-1} (f (C))$ means $f(x) \in 
f(C)$, so there exists $y \in A$ such that $f(x) = f(y)$.  Since $f$
is injective, one would have $x = y$, which is impossible because
$y$ is supposed to belong to $C$ but $x$ is not supposed to belong to $C$.
\end{proof}

\begin{theorem}
Suppose $f\colon A\to B$ is an injection.  Then, for all $C, D\subseteq A$, 
it is the case that $f(C \cap D) = f(C) \cap f(D)$.
\end{theorem}

\begin{proof}
Whether or not $f$ is injective, one has $f(C \cap D) \subseteq f(C) 
\cap f(D)$; if $x$ belongs to both $C$ and $D$, then $f(x)$ will clearly
belong to both $f(C)$ and $f(D)$.  Hence, all that needs to be shown is
that $f(C) \cap f(D) \subseteq f(C \cap D)$.  Let $x$ be an element of
$B$ which belongs to both $f(C)$ and $f(D)$.  Then, there exists $y \in C$
such that $f(y) = x$ and $z \in D$ such that $f(z) = x$.  Since $f(y) = 
f(z)$ and $f$ is injective, $y = z$, so $y \in C \cap D$, hence $x \in
f(C \cap D)$.
\end{proof}
%%%%%
%%%%%
\end{document}
