\documentclass[12pt]{article}
\usepackage{pmmeta}
\pmcanonicalname{OrderedTuplet}
\pmcreated{2013-03-22 14:55:44}
\pmmodified{2013-03-22 14:55:44}
\pmowner{rspuzio}{6075}
\pmmodifier{rspuzio}{6075}
\pmtitle{ordered tuplet}
\pmrecord{16}{36617}
\pmprivacy{1}
\pmauthor{rspuzio}{6075}
\pmtype{Definition}
\pmcomment{trigger rebuild}
\pmclassification{msc}{03-00}
\pmsynonym{tuplet}{OrderedTuplet}
\pmsynonym{$n$-tuplet}{OrderedTuplet}
\pmsynonym{$n$-tuplets}{OrderedTuplet}
\pmsynonym{ordered $n$-tuplet}{OrderedTuplet}
\pmsynonym{-tuplet}{OrderedTuplet}
\pmsynonym{-tuplets}{OrderedTuplet}
\pmsynonym{tuple}{OrderedTuplet}
\pmsynonym{$n$-tuple}{OrderedTuplet}
\pmsynonym{ordered $n$-tupule}{OrderedTuplet}
\pmsynonym{-tuple}{OrderedTuplet}
\pmsynonym{finite sequence}{OrderedTuplet}
\pmrelated{OrderedPair}
\pmrelated{GeneralizedCartesianProduct}
\pmdefines{triplet}
\pmdefines{quadruplet}
\pmdefines{quintuplet}
\pmdefines{sextuplet}
\pmdefines{septuplet}
\pmdefines{octuplet}
\pmdefines{nonuplet}
\pmdefines{decuplet}

\endmetadata

% this is the default PlanetMath preamble.  as your knowledge
% of TeX increases, you will probably want to edit this, but
% it should be fine as is for beginners.

% almost certainly you want these
\usepackage{amssymb}
\usepackage{amsmath}
\usepackage{amsfonts}

% used for TeXing text within eps files
%\usepackage{psfrag}
% need this for including graphics (\includegraphics)
%\usepackage{graphicx}
% for neatly defining theorems and propositions
%\usepackage{amsthm}
% making logically defined graphics
%%%\usepackage{xypic}

% there are many more packages, add them here as you need them

% define commands here
\begin{document}
The concept of ordered $n$-tuplet is the generalization of ordered pair to $n$ items.  For small values of $n$, the following \PMlinkescapetext{terms} are used:
$$\begin{matrix}
n & \hbox{\sl name} \hfill & \hbox{\sl example} \hfill \\
3 & \hbox{triplet} \hfill & (a,b,c) \hfill \\ 
4 & \hbox{quadruplet} \hfill & (a,b,c,d) \hfill \\
5 & \hbox{quintuplet} \hfill & (a,b,c,d,e) \hfill \\
6 & \hbox{sextuplet} \hfill & (a,b,c,d,e,f) \hfill \\
7 & \hbox{septuplet} \hfill & (a,b,c,d,e,f,g) \hfill \\
8 & \hbox{octuplet} \hfill & (a,b,c,d,e,f,g,h) \hfill \\
9 & \hbox{nonuplet} \hfill & (a,b,c,d,e,f,g,h,i) \hfill \\
10 & \hbox{decuplet} \hfill & (a,b,c,d,e,f,g,h,i,j) \hfill \\
\end{matrix}$$

This notion can be defined set-theoretically in a number of ways.  For convenience, we shall express two of these definitions for quintuplets --- it is perfectly easy to generalize them to any other value of $n$.

One possibility is to build $n$-tuplets out of nested ordered pairs.  In the case of our example $(a,b,c,d,e)$, the \PMlinkescapetext{representation} as a nested ordered pair looks like
 $$(a,(b,(c,(d,e)))).$$
This form of \PMlinkescapetext{representation} is used in the programming language LISP.

Another possibility is to define $n$-tuplets as maps.  In this way of thinking, a quintuplet is a function whose domain is the set $\{ 1, 2, 3, 4, 5 \}$.  In the case of our example, the function $f$ in question is defined as
\[ \begin{array}{ccc}
f(1) & = & a \\
f(2) & = & b \\
f(3) & = & c \\
f(4) & = & d \\
f(5) & = & e \\
\end{array} \]

Especially with the second interpretation, one sees that a synonym for "ordered tuplet" is "finite sequence" or "list".  For instance, a quintuplet can also be regarded as a sequence of five items or a list of five items.

%%%%%
%%%%%
\end{document}
