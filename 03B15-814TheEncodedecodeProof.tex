\documentclass[12pt]{article}
\usepackage{pmmeta}
\pmcanonicalname{814TheEncodedecodeProof}
\pmcreated{2013-11-06 2:35:21}
\pmmodified{2013-11-06 2:35:21}
\pmowner{PMBookProject}{1000683}
\pmmodifier{rspuzio}{6075}
\pmtitle{8.1.4 The encode-decode proof}
\pmrecord{1}{}
\pmprivacy{1}
\pmauthor{PMBookProject}{6075}
\pmtype{Feature}
\pmclassification{msc}{03B15}

\endmetadata

\usepackage{xspace}
\usepackage{amssyb}
\usepackage{amsmath}
\usepackage{amsfonts}
\usepackage{amsthm}
\makeatletter
\newcommand{\base}{\ensuremath{\mathsf{base}}\xspace}
\newcommand{\blank}{\mathord{\hspace{1pt}\text{--}\hspace{1pt}}}
\newcommand{\code}{\ensuremath{\mathsf{code}}\xspace}
\newcommand{\ct}{  \mathchoice{\mathbin{\raisebox{0.5ex}{$\displaystyle\centerdot$}}}             {\mathbin{\raisebox{0.5ex}{$\centerdot$}}}             {\mathbin{\raisebox{0.25ex}{$\scriptstyle\,\centerdot\,$}}}             {\mathbin{\raisebox{0.1ex}{$\scriptscriptstyle\,\centerdot\,$}}}}
\newcommand{\decode}{\ensuremath{\mathsf{decode}}\xspace}
\newcommand{\defeq}{\vcentcolon\equiv}  
\def\@dprd#1{\prod_{(#1)}\,}
\def\@dprd@noparens#1{\prod_{#1}\,}
\def\@dsm#1{\sum_{(#1)}\,}
\def\@dsm@noparens#1{\sum_{#1}\,}
\def\@eatprd\prd{\prd@parens}
\def\@eatsm\sm{\sm@parens}
\newcommand{\encode}{\ensuremath{\mathsf{encode}}\xspace}
\newcommand{\eqv}[2]{\ensuremath{#1 \simeq #2}\xspace}
\newcommand{\id}[3][]{\ensuremath{#2 =_{#1} #3}\xspace}
\newcommand{\indexsee}[2]{\index{#1|see{#2}}}    
\newcommand{\jdeq}{\equiv}      
\newcommand{\lloop}{\ensuremath{\mathsf{loop}}\xspace}
\newcommand{\narrowequation}[1]{$#1$}
\newcommand{\opp}[1]{\mathord{{#1}^{-1}}}
\def\prd#1{\@ifnextchar\bgroup{\prd@parens{#1}}{\@ifnextchar\sm{\prd@parens{#1}\@eatsm}{\prd@noparens{#1}}}}
\def\prd@noparens#1{\mathchoice{\@dprd@noparens{#1}}{\@tprd{#1}}{\@tprd{#1}}{\@tprd{#1}}}
\def\prd@parens#1{\@ifnextchar\bgroup  {\mathchoice{\@dprd{#1}}{\@tprd{#1}}{\@tprd{#1}}{\@tprd{#1}}\prd@parens}  {\@ifnextchar\sm    {\mathchoice{\@dprd{#1}}{\@tprd{#1}}{\@tprd{#1}}{\@tprd{#1}}\@eatsm}    {\mathchoice{\@dprd{#1}}{\@tprd{#1}}{\@tprd{#1}}{\@tprd{#1}}}}}
\newcommand{\refl}[1]{\ensuremath{\mathsf{refl}_{#1}}\xspace}
\def\sm#1{\@ifnextchar\bgroup{\sm@parens{#1}}{\@ifnextchar\prd{\sm@parens{#1}\@eatprd}{\sm@noparens{#1}}}}
\def\sm@noparens#1{\mathchoice{\@dsm@noparens{#1}}{\@tsm{#1}}{\@tsm{#1}}{\@tsm{#1}}}
\def\sm@parens#1{\@ifnextchar\bgroup  {\mathchoice{\@dsm{#1}}{\@tsm{#1}}{\@tsm{#1}}{\@tsm{#1}}\sm@parens}  {\@ifnextchar\prd    {\mathchoice{\@dsm{#1}}{\@tsm{#1}}{\@tsm{#1}}{\@tsm{#1}}\@eatprd}    {\mathchoice{\@dsm{#1}}{\@tsm{#1}}{\@tsm{#1}}{\@tsm{#1}}}}}
\newcommand{\Sn}{\mathbb{S}}
\def\@tprd#1{\mathchoice{{\textstyle\prod_{(#1)}}}{\prod_{(#1)}}{\prod_{(#1)}}{\prod_{(#1)}}}
\newcommand{\transfib}[3]{\ensuremath{\mathsf{transport}^{#1}(#2,#3)\xspace}}
\newcommand{\transfibf}[1]{\ensuremath{\mathsf{transport}^{#1}\xspace}}
\newcommand{\trunc}[2]{\mathopen{}\left\Vert #2\right\Vert_{#1}\mathclose{}}
\def\@tsm#1{\mathchoice{{\textstyle\sum_{(#1)}}}{\sum_{(#1)}}{\sum_{(#1)}}{\sum_{(#1)}}}
\newcommand{\vcentcolon}{:\!\!}
\newcommand{\Z}{\ensuremath{\mathbb{Z}}\xspace}
\newcommand{\Zpred}{\mathsf{pred}}
\newcommand{\Zsuc}{\mathsf{succ}}
\newcounter{mathcount}
\setcounter{mathcount}{1}
\newtheorem{precor}{Corollary}
\newenvironment{cor}{\begin{precor}}{\end{precor}\addtocounter{mathcount}{1}}
\renewcommand{\theprecor}{8.1.\arabic{mathcount}}
\newtheorem{predefn}{Definition}
\newenvironment{defn}{\begin{predefn}}{\end{predefn}\addtocounter{mathcount}{1}}
\renewcommand{\thepredefn}{8.1.\arabic{mathcount}}
\newtheorem{prelem}{Lemma}
\newenvironment{lem}{\begin{prelem}}{\end{prelem}\addtocounter{mathcount}{1}}
\renewcommand{\theprelem}{8.1.\arabic{mathcount}}
\newtheorem{prethm}{Theorem}
\newenvironment{thm}{\begin{prethm}}{\end{prethm}\addtocounter{mathcount}{1}}
\renewcommand{\theprethm}{8.1.\arabic{mathcount}}
\let\autoref\cref
\makeatother

\begin{document}

\index{encode-decode method|(}%
\indexsee{encode}{encode-decode method}%
\indexsee{decode}{encode-decode method}%
We begin with the function~\eqref{eq:pi1s1-encode} that maps paths to codes:
\begin{defn}
Define $\encode : \prd{x : \Sn ^1} (\base=x) \rightarrow  \code(x)$ by 
\[
\encode \: p \defeq \transfib{\code} p 0
\]
(we leave the argument $x$ implicit).  
\end{defn}
Encode is defined by lifting a path into the universal cover, which
determines an equivalence, and then applying the resulting equivalence
to $0$.  
The interesting thing about this function is that it computes a concrete
number from a loop on the circle, when this loop is represented using
the abstract groupoidal framework of homotopy type theory.  To gain an
intuition for how it does this, observe that by the above lemmas,
$\transfib \code \lloop x$ is the successor map and $\transfib \code {\opp
  \lloop} x$ is the predecessor map.
Further, $\mathsf{transport}$ is functorial (\autoref{cha:basics}), so
$\transfib{\code} {\lloop \ct \lloop}{\blank}$ is
\[(\transfib \code \lloop-) \circ (\transfib \code \lloop-)\]
and so on.
Thus, when $p$ is a composition like 
\[
\lloop \ct \opp \lloop \ct \lloop \ct \cdots
\]
$\transfib{\code}{p}{\blank}$ will compute a composition of functions like
\[
\Zsuc \circ \Zpred \circ \Zsuc \circ \cdots 
\]
Applying this composition of functions to 0 will compute the
\index{winding!number}%
\emph{winding number} of the path---how many times it goes around the
circle, with orientation marked by whether it is positive or negative,
after inverses have been canceled.  Thus, the computational behavior of
$\encode$ follows from the reduction rules for higher-inductive types and
univalence, and the action of $\mathsf{transport}$ on compositions and inverses.

Note that the instance $\encode' \defeq \encode_{\base}$ has type 
$(\id \base \base) \rightarrow \Z$.
This will be one half of our desired equivalence; indeed, it is exactly the function $g$ defined in \autoref{sec:pi1s1-initial-thoughts}.

Similarly, the function~\eqref{eq:pi1s1-decode} is a generalization of the function $\lloop^{\blank}$ from \autoref{sec:pi1s1-initial-thoughts}.

\begin{defn}\label{thm:pi1s1-decode}
Define $\decode : \prd{x : \Sn ^1}\code(x) \rightarrow (\base=x)$ by 
circle induction on $x$.  It suffices to give a function 
${\code(\base) \rightarrow (\base=\base)}$, for which we use $\lloop^{\blank}$, and 
to show that $\lloop^{\blank}$ respects the loop.  
\end{defn}

\begin{proof}
To show that $\lloop^{\blank}$ respects the loop, it suffices to give a path
from $\lloop^{\blank}$ to itself that lies over $\lloop$. 
By the definition of dependent paths, this means a path from
\[\transfib {(x' \mapsto \code(x') \rightarrow (\base=x'))} {\lloop} {\lloop^{\blank}}\]
to $\lloop^{\blank}$.  We define such a
path as follows:
\begin{align*}
  \MoveEqLeft \transfib {(x' \mapsto \code(x') \rightarrow (\base=x'))} \lloop {\lloop^{\blank}}\\
&= \transfibf {x'\mapsto (\base=x')}(\lloop) \circ {\lloop^{\blank}} \circ \transfibf \code ({\opp \lloop}) \\
&= (- \ct \lloop) \circ (\lloop^{\blank}) \circ \transfibf \code ({\opp \lloop}) \\
&= (- \ct \lloop) \circ (\lloop^{\blank}) \circ \Zpred \\
&= (n \mapsto \lloop^{n - 1} \ct \lloop).
\end{align*}
On the first line, we apply the characterization of $\mathsf{transport}$
when the outer connective of the fibration is $\rightarrow$, which
reduces the $\mathsf{transport}$ to pre- and post-composition with
$\mathsf{transport}$ at the domain and codomain types.  On the second line,
we apply the characterization of $\mathsf{transport}$ when the type family
is $x\mapsto \id{\base}{x}$, which is post-composition of paths.  On the third line,
we use the action of $\code$ on $\opp \lloop$ from
\autoref{lem:transport-s1-code}.  And on the fourth line, we simply
reduce the function composition.  Thus, it suffices to show that for all
$n$, \id{\lloop^{n - 1} \ct \lloop}{\lloop^{n}}, which is an easy
induction, using the groupoid laws.  
\end{proof}

We can now show that $\encode$ and $\decode$ are quasi-inverses.
What used to be the difficult direction is now easy!

\begin{lem} \label{lem:s1-decode-encode}  For all 
for all $x: \Sn ^1$ and $p : \id \base x$, $\id
{\decode_x({{\encode_x(p)}})} p$.  
\end{lem}

\begin{proof}
By path induction, it suffices to show that 
\narrowequation{\id {\decode_{\base}({{\encode_{\base}(\refl{\base})}})} {\refl{\base}}.}
But
\narrowequation{\encode_{\base}(\refl{\base}) \jdeq \transfib{\code}{\refl{\base}} 0 \jdeq 0,}
and $\decode_{\base}(0) \jdeq \lloop^ 0 \jdeq \refl{\base}$.  
\end{proof}

The other direction is not much harder.

\begin{lem} \label{lem:s1-encode-decode} For all 
$x: \Sn ^1$ and $c : \code(x)$, we have $\id
{\encode_x({{\decode_x(c)}})} c$.  
\end{lem}

\begin{proof}
The proof is by circle induction.  It suffices to show the case for
\base, because the case for \lloop is a path between paths in
$\Z$, which is immediate because $\Z$ is a set.  

Thus, it suffices to show, for all $n : \Z$, that
\[
\id {\encode'(\lloop^n)} {n}
\]
The proof is by induction, using \cref{thm:looptothe}.
%
\begin{itemize}

\item In the case for $0$, the result is true by definition.

\item In the case for $n+1$, 
\begin{align}
 {\encode'(\lloop^{n+1})}
&= {\encode'(\lloop^{n} \ct \lloop)} \tag{by definition of $\lloop^{\blank}$} \\
&= \transfib{\code}{(\lloop^{n} \ct \lloop)}{0} \tag{by definition of $\encode$}\\
&= \transfib{\code}{\lloop}{(\transfib{\code}{\lloop^n}{0})} \tag{by functoriality}\\
&= {(\transfib{\code}{\lloop^n}{0})} + 1 \tag{by \autoref{lem:transport-s1-code}}\\
&= n + 1. \tag{by the inductive hypothesis}
\end{align}

\item The case for negatives is analogous.  \qedhere
\end{itemize}
\end{proof}

Finally, we conclude the theorem.

\begin{thm} 
There is a family of equivalences $\prd{x : \Sn ^1} (\eqv {(\base=x)} {\code(x)})$.
\end{thm}
\begin{proof}
The maps $\encode$ and $\decode$ are quasi-inverses by
\autoref{lem:s1-decode-encode},\autoref{lem:s1-decode-encode}.
\end{proof}

Instantiating at {\base} gives
\begin{cor}\label{cor:omega-s1}
$\eqv {\Omega(\Sn^1,\base)} {\Z}$.
\end{cor}

A simple induction shows that this equivalence takes addition to
composition, so that $\Omega(\Sn ^1) = \Z$ as groups.

\begin{cor} \label{cor:pi1s1}
$\id{\pi_1(\Sn ^1)} {\Z}$, while $\id{\pi_n(\Sn ^1)}0$ for $n>1$.
\end{cor}
\begin{proof}
For $n=1$, we sketched the proof from \autoref{cor:omega-s1} above.
For $n > 1$, we have $\trunc 0 {\Omega^{n}(\Sn ^1)} = \trunc 0 {\Omega^{n-1}(\Omega{\Sn ^1})} = \trunc 0 {\Omega^{n-1}(\Z)}$.
And since $\Z$ is a set, $\Omega^{n-1}(\Z)$ is contractible, so this is trivial.
\end{proof}

\index{encode-decode method|)}%



\end{document}
