\documentclass[12pt]{article}
\usepackage{pmmeta}
\pmcanonicalname{ConsequenceOperator}
\pmcreated{2013-03-22 16:28:48}
\pmmodified{2013-03-22 16:28:48}
\pmowner{rspuzio}{6075}
\pmmodifier{rspuzio}{6075}
\pmtitle{consequence operator}
\pmrecord{40}{38646}
\pmprivacy{1}
\pmauthor{rspuzio}{6075}
\pmtype{Definition}
\pmcomment{trigger rebuild}
\pmclassification{msc}{03G25}
\pmclassification{msc}{03G10}
\pmclassification{msc}{03B22}
\pmsynonym{closure operator}{ConsequenceOperator}
\pmdefines{finitary consequence operator}
\pmdefines{finite consequence operator}
\pmdefines{algebraic consequence operator}
\pmdefines{axiomatic consequence operator}
\pmdefines{axiomless consequence operator}

\endmetadata

% this is the default PlanetMath preamble.  as your knowledge
% of TeX increases, you will probably want to edit this, but
% it should be fine as is for beginners.

% almost certainly you want these
\usepackage{amssymb}
\usepackage{amsmath}
\usepackage{amsfonts}

% used for TeXing text within eps files
%\usepackage{psfrag}
% need this for including graphics (\includegraphics)
%\usepackage{graphicx}
% for neatly defining theorems and propositions
%\usepackage{amsthm}
% making logically defined graphics
%%%\usepackage{xypic}

% there are many more packages, add them here as you need them

% define commands here

\begin{document}
\section{Definition}

Let $L$ be a set.  A \emph{consequence operator} on $L$ is a
mapping\footnote{Here, ``$\mathcal{P}$'' denotes the power set and 
$\mathcal{F}$ denotes the finite power set.} $C \colon \mathcal{P}(L) \to \mathcal{P}(L)$
which satisfies the following three properties:
\begin{enumerate}
\item For all $X \subseteq L$, it happens that $X \subseteq C(X)$.
\item $C \circ C = C$
\item For all $X, Y \subseteq L$, if $X \subseteq Y$, then $C(X) \subseteq C(Y)$
\end{enumerate}

If, in addition, the following condition is satisfied, then a consequence operator $C$ is known as \emph{finitary }.  (Synonyms are ``finite consequence operator'' and ``algebraic consequence operator''.)
\begin{itemize}
\item For all $X \in L$, it happens that $C(X) = \bigcup\limits_{Y \in \mathcal{F} (X)} C(Y)$.
\end{itemize}
It is worth noting that, if the above condition is satisfied, then the third
condition of last paragraph becomes superfluous --- as shown in 
\PMlinkid{an attachment}{ 8678}, it automatically follows 
from conitions 1 and 2 of last paragraph and the condition stated above.

A consequence operator $C$ such that $C(\emptyset) = \emptyset$ is called \emph{axiomless}.  A consequence operator $C$ such that $C(\emptyset) \not= \emptyset$ is called \emph{axiomatic}.

\section{Motivation}

Alfred Tarksi introduced consequence operators as a way of discussing the 
notion of conclusions following from premises in a general fashion.  Suppose
that the set $L$ consists of statements in some language.  Then, given a 
set of statements $X$, let $C(X)$ be the set of all statements which can be
inferred form statements of $X$.

The defining properties of ``consequence operator'' given above then express
some fundamental facts about the process of inferring conclusions from 
premises:  Any statement can be concluded from itself.  If a statement $s$ 
follows from a set of premises $X$ and $Y$ is a superset of $X$, then
$s$ also follows from $Y$.  If one augments a set of premises by conclusions
derived from those premises, then one can only draw conclusions from the 
larger set which could have been drawn from the original set of premises.
Note that these conditions hold for a large class of logics, not just 
classical logic of  Aristotle, Boole, and Frege.  However, they do not hold 
for all logics ---  in particular, there are the so-called nonmonotonic 
logics in which it is not always the case that, if $X \subset Y$, then 
$C(X) \subseteq C(Y)$.

In terms of this usage in logic, it is easy to understand the origin of the
terms "axiomatic" and "axiomless".   An axiom in a logical theory is a 
statement which is assumed true without having to prove it from any other
statement.  Hence, an axiom is a consequence of the empty set, so we call
consequence operators which allow one to deduce conclusions from an empty
set of premises axiomatic.

The distinction of finitary consequence operators has to do with whether one
is permitted to draw a conclusion from an infinite set of premises which
could not be drawn from any finite subset thereof.  As for why one might
want to do this, consider the following example.  Suppose $X$ consists of
the following statements:
\begin{itemize}
\item $1 = 1^2$
\item $2 = 1^2 + 1^2$
\item $3 = 1^2 + 1^2 + 1^2$
\item $4 = 2^2$
\item $5 = 2^2 + 1^2$
\item $6 = 2^2 + 1^2 + 1^2$
\item $7 = 2^2 + 1^2 + 1^2 + 1^2$
\item $8 = 2^2 + 2^2$
\item $9 = 3^2$
\item $10 = 3^2 + 1^2$
\item $11 = 3^2 + 1^2 + 1^2$
\item $12 = 2^2 + 2^2 + 2^2$
\item $13 = 3^2 + 2^2$
\item $14 = 3^2 + 2^2 + 1^2$
\item $\cdots$
\end{itemize}
From $X$ one would like to be able to draw the conclusion ``Any positive
integer can be expressed as the sum of at most four squares.''.  This 
conclusion, however, cannot be inferred from any proper subset of $X$, in
particular, from any finite subset of $X$.  To make this conclusion would
require a consequence operator which is not finitary.

\section{Examples}

\begin{enumerate}
\item
To begin, there are two trivial consequence operators defined on any set.
One is the identity operator $I \colon \mathcal{P}(L) \to \mathcal{P}(L)$ defined as $I(X) = X$.  The other is the constant operator $U \colon \mathcal{P}(L) \to \mathcal{P}(L)$ defined as $U(X) = L$.  It is perfectly
straightforward to check that these two operators satisfy the defining 
properties of consequence operator and, furthermore, that they are both
finitary consequence operators and that $I$ is axiomless whilst $U$ is
axiomatic.  Trivial though they may be, these operators play an important 
role as exrtremal elements in the lattice of all consequence operators
over a given set.
\item
Next, we consider some less trivial consequence operators which can be 
defined over an arbitrary set.  Let $X$ and $Y$ be any two subsets of $L$.  
Then we may define operators $C_\cap (X,Y) \colon \mathcal{P}(L) \to \mathcal{P}(L)$
and $C_\cup (X,Y) \colon \mathcal{P}(L) \to \mathcal{P}(L)$ as follows:
\begin{eqnarray*} C_\cap (X,Y)(Z) &=& \begin{cases} X \cup Z & Y \cap Z 
\not= \emptyset \\ Z & Y \cap Z = \emptyset \end{cases} \\
C_\cup (X,Y)(Z) &=& \begin{cases} X \cup Z & Y \cup Z = Z \\ Z & Y \cup Z 
\not= Z \end{cases} \end{eqnarray*}
It is shown that these are indeed consequence operators in an attachment to
this entry.
\item
A much larger class of consequence operators may be defined as follows.
Let $K$ be a subset of $\mathcal{P}(L)$ which includes $L$.  Then, as
shown in an \PMlinkid{attachment}{8671}, the
map $C \colon \mathcal{P}(L) \to \mathcal{P}(L)$, defined as
\[C(X) = \cap \{ Y \in K \mid X \subseteq Y \},\]
is a consequence operator.  As we shall see, all consequence operators
can be obtained by this construction.  In particular, the examples discussed
above can be obtained as follows:  To obtain $I$, set $K = \mathbf{P}(L)$; to
obtain $U$, set $K = \{ L \}$; to obtain $C_\cap (X,Y)$, set
\[ K = \{ Z \subseteq L \mid Y \cap Z = \emptyset \} \cup
\{ X \cup Z \mid Z \subseteq L \,\land\, Y \cap Z \not= \emptyset \};\]
to obtain $C_\cup (X,Y)$, set
\[ K = \{ Z \subseteq L \mid Y \cup Z \not= \emptyset \} \cup
\{ X \cup Z \mid Z \subseteq L \,\land\, Y \cup Z = \emptyset \}.\]
\item
Turning to more specific examples, we have the example which inspired the 
definintion in the first place.  Let $L$ be a set of logical expressions
constructed from some set of sentence letters and predicate letters and
the usual connectives and quantifiers.  Given a subset $X \subseteq L$,
let $C(X)$ be the set of all expressions $\psi$ for which there exists
a finite set of expressions $\phi_1, \ldots, \phi_n$ such that
$\ulcorner \phi_1 \land \cdots \land \phi_n \Rightarrow \psi \urcorner$
is a tautology.  Note that this is a finitary consequence operator --- it
does not enable one to make the sort of deductions from infinite sets
of premises described above.
\item
This notion of consequence operator also applies to areas of mathematics
other than logic.  For instance, suppose that $L$ is a vector space.  Then
the operator which assigns to a subset of $L$ the vector subspace which it 
spans is a consequence operator.  This particular consequence operator is
finitary because if a vector $v$ belongs to the span of a set $X$, then
$v$ can be expressed as a linear combination of a finite number of elements
of $X$. 
\item
The closure operator in topology is a
consequence operator.  It is worth pointing out that not every consequence
operator can be expressed as the closure operator for some topology because
the closure operator satisfies some extra conditions beyond those which 
define consequence operators.  Typically, the closure operator is not 
finitary because infinite subsets of topological spaces may have limit
points.
\end{enumerate}

\section{Alternative Definition and Generalization}

A consequence operator can be characterized by its fixed points.  Given 
a consequence operator $C \colon \mathcal{P}(L) \to \mathcal{P}(L)$,
set $K = \{ X \subseteq L \mid C(X) = X\}$.  By the second defining 
property of consequence operator, we have  $K = \{ C(X) \mid X \in L\}$.
One can show that 
 \[C(X) = \cap \{ Y \in K \mid X \subseteq Y \}.\]

Conversely, suppose that $K$ is a subset of $L$ with the following minimum property:  
\begin{itemize} \item For every $X \in L$, there exists a $Y \in K$ such 
that $X \subseteq Y$ and if, for any $Z \in K$, if $X \subseteq Z$, then 
$Y \subseteq Z$. \end{itemize}
Then the operator $C$ defined as
 \[C(X) = \cap \{ Y \in K \mid X \subseteq Y \}\]
is a consequence operator with $K$ as its set of fixed points.

One may also define consequence operators in the more general context of
a partially ordered set which may not be the power set of any set.  Suppose
that $\langle S, \le \rangle$ is a partially ordered set.  Then we may define
a consequence operator on this ordered set to be a map $C \colon S \to S$
which satisfies the following three properties:
\begin{enumerate}
\item For all $X \in S$, it happens that $X \le C(X)$.
\item $C \circ C = C$
\item For all $X, Y \in S$, if $X \le Y$, then $C(X) \le C(Y)$
\end{enumerate}

Such more general consequence operators arise frequently when we restrict
attention to distinguished subsets of a set.  As an example, we may consider 
the following situation.  Let $S$ be the set of linear subspaces of a Banach
space, ordered by inclusion.  Then the operator $C \colon S \to S$ which
assigns to each subspace its Cauchy completion is a consequence operator.

As an example which does not arise this way, let $S = \mathbb{R}$ with the
usual order.  Then the ceiling function $\lceil \cdot \rceil \colon
\mathbb{R} \to \mathbb{R}$ is a consequence operator.

For another example, let $S$ be the set of all fields
with a countable number of elements.  This set may be ordered as follows: 
$E \le F$ if and only if there exists a non-trivial morphism of $E$ into $F$.  Then the operator which sends each field to its algebraic closure is a 
consequence operator.
%%%%%
%%%%%
\end{document}
