\documentclass[12pt]{article}
\usepackage{pmmeta}
\pmcanonicalname{QuantifierFree}
\pmcreated{2013-03-22 13:27:49}
\pmmodified{2013-03-22 13:27:49}
\pmowner{mathcam}{2727}
\pmmodifier{mathcam}{2727}
\pmtitle{quantifier free}
\pmrecord{7}{34031}
\pmprivacy{1}
\pmauthor{mathcam}{2727}
\pmtype{Definition}
\pmcomment{trigger rebuild}
\pmclassification{msc}{03C10}
\pmclassification{msc}{03C07}
\pmclassification{msc}{03B10}
\pmrelated{Quantifier}
\pmrelated{LogicalLanguage}
\pmdefines{quantifier free formula}
\pmdefines{quantifier elimination}
\pmdefines{elimination set}

\endmetadata

% this is the default PlanetMath preamble.  as your knowledge
% of TeX increases, you will probably want to edit this, but
% it should be fine as is for beginners.

% almost certainly you want these
\usepackage{amssymb}
\usepackage{amsmath}
\usepackage{amsfonts}

% used for TeXing text within eps files
%\usepackage{psfrag}
% need this for including graphics (\includegraphics)
%\usepackage{graphicx}
% for neatly defining theorems and propositions
%\usepackage{amsthm}
% making logically defined graphics
%%%\usepackage{xypic}

% there are many more packages, add them here as you need them

% define commands here
\begin{document}
Let $L$ be a first order language. 
A formula $\psi$ is {\em quantifier free} iff it contains no quantifiers.

\medskip

Let $T$ be a complete $L$-theory. Let $S \subseteq L$. Then $S$ is an {\em elimination set} for $T$ iff 
for every $\psi(\bar{x}) \in L$ there is some $\phi(\bar{x}) \in S$ so that 
$T \vdash \forall \bar{x} (\psi(\bar{x})) \leftrightarrow \phi(\bar{x})$.

\medskip

In particular, $T$ has {\em quantifier elimination} iff the set of quantifier free formulas is an elimination set for $T$. 
In other \PMlinkescapetext{words} $T$ has {\em quantifier elimination} iff
for every $\psi(\bar{x}) \in L$ there is some quantifier free $\phi(\bar{x}) \in L$ so that 
$T \vdash \forall \bar{x} (\psi(\bar{x})) \leftrightarrow \phi(\bar{x})$.
%%%%%
%%%%%
\end{document}
