\documentclass[12pt]{article}
\usepackage{pmmeta}
\pmcanonicalname{ProductOfAutomata}
\pmcreated{2013-03-22 18:03:06}
\pmmodified{2013-03-22 18:03:06}
\pmowner{CWoo}{3771}
\pmmodifier{CWoo}{3771}
\pmtitle{product of automata}
\pmrecord{12}{40577}
\pmprivacy{1}
\pmauthor{CWoo}{3771}
\pmtype{Definition}
\pmcomment{trigger rebuild}
\pmclassification{msc}{03D05}
\pmclassification{msc}{68Q45}

\usepackage{amssymb,amscd}
\usepackage{amsmath}
\usepackage{amsfonts}
\usepackage{mathrsfs}

% used for TeXing text within eps files
%\usepackage{psfrag}
% need this for including graphics (\includegraphics)
%\usepackage{graphicx}
% for neatly defining theorems and propositions
\usepackage{amsthm}
% making logically defined graphics
%%\usepackage{xypic}
\usepackage{pst-plot}

% define commands here
\newcommand*{\abs}[1]{\left\lvert #1\right\rvert}
\newtheorem{prop}{Proposition}
\newtheorem{thm}{Theorem}
\newtheorem{ex}{Example}
\newcommand{\real}{\mathbb{R}}
\newcommand{\pdiff}[2]{\frac{\partial #1}{\partial #2}}
\newcommand{\mpdiff}[3]{\frac{\partial^#1 #2}{\partial #3^#1}}
\begin{document}
One way to manufacture an automaton out of existing automata is by taking products.

\subsubsection*{Products of Two Automata}

Let $A_1=(S_1,\Sigma_1,\delta_1,I_1,F_1)$ and $A_2=(S_2,\Sigma_2,\delta_2,I_2,F_2)$ be two automata.  We define the product $A$ of $A_1$ and $A_2$, written $A_1\times A_2$, as the quituple 
$$(S,\Sigma,\delta,I,F):=(S_1\times S_2,\Sigma_1\times \Sigma_2,\delta_1\times \delta_2, I_1\times I_2, F_1\times F_2),$$
where $\delta$ is a function from $S\times \Sigma$ to $P(S_1)\times P(S_2)\subseteq P(S)$, given by $$\delta((s_1,s_2),(\alpha_1,\alpha_2)):=\delta_1(s_1,\alpha_1)\times \delta_2(s_2,\alpha_2).$$

Since $S,\Sigma,I,F$ are non-empty, $A$ is an automaton.  The automaton $A$ can be thought of as a machine that runs automata $A_1$ and $A_2$ simultaneously.  A pair $(\alpha_1,\alpha_2)$ of symbols being fed into $A$ at start state $(q_1,q_2)\in I$ is the same as $A_1$ reading $\alpha_1$ at state $q_1$ and $A_2$ reading $\alpha_2$ at state $q_2$.  The set of all possible next states for the configuration $((s_1,s_2),(\alpha_1,\alpha_2))$ in $A$ is the same as the set of all possible combinations $(t_1,t_2)$, where $t_1$ is a next state for the configuration $(s_1,\alpha_1)$ in $A_1$ and $t_2$ is a next state for the configuration $(s_2,\alpha_2)$ in $A_2$.

If $A_1$ and $A_2$ are FSA, so is $A$.  In addition, if both $A_1$ and $A_2$ are deterministic, so is $A$, because $$\delta((s_1,s_2),(\alpha_1,\alpha_2))=(\delta_1(s_1,\alpha_1),\delta_2(s_2, \alpha_2)),$$ and $I$ is a singleton.

As usual, $\delta$ can be extended to read words over $\Sigma$, and it is easy to see that  $$\delta((s_1,s_2),(a_1,a_2))=\delta_1(s_1,a_1)\times \delta_2(s_2,a_2),$$ where $a_1$ and $a_2$ are words over $\Sigma_1$ and $\Sigma_2$ respectively.  A word $(a_1,a_2)$ is accepted by $A$ iff $a_1$ is accepted by $A_1$ and $a_2$ is accepted by $A_2$.

\subsubsection*{Intersection of Two Automata}

Again, we assume $A_1$ and $A_2$ are automata specified above.  Now, suppose $\Sigma_1=\Sigma_2=\Delta$.  Then $\Delta$ can be identified as the diagonal in $\Sigma=\Sigma_1\times \Sigma_2=\Delta^2$.  We are then led to an automaton $$A_1\cap A_2:=(S,\Delta, \delta,I,F),$$ where $S,I$, and $F$ are defined previously, and $\delta$ is given by $$\delta((s_1,s_2),\alpha)=\delta_1(s_1,\alpha)\times \delta_2(s_2,\alpha).$$  
Suppose in addition that $\Delta$ is finite.  From the discussion in the previous section, it is evident that the language accepted by $A_1\cap A_2$ is the same as the intersection of the language accepted by $A_1$ and the language accepted by $A_2$: $$L(A_1\cap A_2) = L(A_1)\cap L(A_2).$$
%%%%%
%%%%%
\end{document}
