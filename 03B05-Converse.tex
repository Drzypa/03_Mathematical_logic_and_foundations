\documentclass[12pt]{article}
\usepackage{pmmeta}
\pmcanonicalname{Converse}
\pmcreated{2013-03-22 17:13:37}
\pmmodified{2013-03-22 17:13:37}
\pmowner{pahio}{2872}
\pmmodifier{pahio}{2872}
\pmtitle{converse}
\pmrecord{24}{39554}
\pmprivacy{1}
\pmauthor{pahio}{2872}
\pmtype{Definition}
\pmcomment{trigger rebuild}
\pmclassification{msc}{03B05}
\pmclassification{msc}{03F07}
\pmrelated{ExamplesOfContrapositive}
\pmrelated{DifferntiableFunction}
\pmrelated{Inverse6}
\pmrelated{ConverseOfEulersHomogeneousFunctionTheorem}
\pmdefines{converse theorem}
\pmdefines{conversely}

% this is the default PlanetMath preamble.  as your knowledge
% of TeX increases, you will probably want to edit this, but
% it should be fine as is for beginners.

% almost certainly you want these
\usepackage{amssymb}
\usepackage{amsmath}
\usepackage{amsfonts}

% used for TeXing text within eps files
%\usepackage{psfrag}
% need this for including graphics (\includegraphics)
%\usepackage{graphicx}
% for neatly defining theorems and propositions
 \usepackage{amsthm}
% making logically defined graphics
%%%\usepackage{xypic}

% there are many more packages, add them here as you need them

% define commands here

\theoremstyle{definition}
\newtheorem*{thmplain}{Theorem}

\begin{document}
\PMlinkescapeword{congruent}
\PMlinkescapeword{contains}
\PMlinkescapeword{words}

Let a statement be of the form of an implication

\begin{center}
If $p$ then $q$
\end{center}

\PMlinkname{i.e.}{Ie} it has a certain premise $p$ and a conclusion $q$.\, The statement in which one has interchanged the conclusion and the premise,

\begin{center}
If $q$ then $p$
\end{center}

is the \emph{converse} of the first.\, In other words, from the former one concludes that $q$ is necessary for $p$, and from the latter that $p$ is necessary for $q$.

Note that the converse of an implication and the inverse of the same implication are contrapositives of each other and thus are logically equivalent.

If there is originally a statement which is a (true) theorem and if its converse also is true, then the latter can be called the \emph{converse theorem} of the original one.\, Note that, if the converse of a true theorem ``If $p$ then $q$'' is also true, then ``$p$ iff $q$'' is a true theorem. \\

For example, we know the theorem on isosceles triangles:

\emph{If a triangle contains two \PMlinkname{congruent}{Congruent2} sides, then it has two congruent angles.}

There is also its converse theorem:

\emph{If a triangle contains two congruent angles, then it has two congruent sides.}

Both of these propositions are true, thus being theorems (see the entries angles of an isosceles triangle and determining from angles that a triangle is isosceles).\, But there are many (true) theorems whose converses are not true, \PMlinkname{e.g.}{Eg}:

\emph{If a function is differentiable on an interval $I$, then it is \PMlinkname{continuous}{ContinuousFunction} on $I$.}
%%%%%
%%%%%
\end{document}
