\documentclass[12pt]{article}
\usepackage{pmmeta}
\pmcanonicalname{BerrysParadox}
\pmcreated{2013-03-22 18:06:03}
\pmmodified{2013-03-22 18:06:03}
\pmowner{yesitis}{13730}
\pmmodifier{yesitis}{13730}
\pmtitle{Berry's paradox}
\pmrecord{7}{40643}
\pmprivacy{1}
\pmauthor{yesitis}{13730}
\pmtype{Topic}
\pmcomment{trigger rebuild}
\pmclassification{msc}{03B99}

\endmetadata

% this is the default PlanetMath preamble.  as your knowledge
% of TeX increases, you will probably want to edit this, but
% it should be fine as is for beginners.

% almost certainly you want these
\usepackage{amssymb}
\usepackage{amsmath}
\usepackage{amsfonts}

% used for TeXing text within eps files
%\usepackage{psfrag}
% need this for including graphics (\includegraphics)
%\usepackage{graphicx}
% for neatly defining theorems and propositions
%\usepackage{amsthm}
% making logically defined graphics
%%%\usepackage{xypic}

% there are many more packages, add them here as you need them

% define commands here

\begin{document}
We begin with calling a positive integer \emph{curious} if it can be defined
in the English language using no more than 1234 words. Since there
are finitely many English words, we see that there are only finitely
many curious positive integers.

Define $n_0$ to be: \emph{the least positive integer that is not
curious}.

$n_0$ has just been described in $8\leq 1234$ words, therefore, it
is curious after all!

The paradox above is called Berry's Paradox. Berry's 
Paradox suggests the advantage of separating the language used to formulate 
mathematical statements or theory (the object language) from the language 
used to discuss those statements or the theory (the metalanguage).

Berry's Paradox can be avoided by the following reformulation:

\begin{enumerate}
    \item fix the object language, called $\mathbf{E^\ast}$;
    \item declare $\mathbf{E^\ast}$ to be different from our
    metalanguage, which is English here;
    \item define a curious positive integer to be one which can be
    described in $\mathbf{E^\ast}$ using no more than 1234 words of the
    language;
    \item define $n_0$ to be the least positive integer that is not
    curious.
\end{enumerate}

In the reformulation, we have defined curious positive integers and
$n_0$ in English, which is not $\mathbf{E^\ast}$. Thus, we have no
basis to conclude that $n_0$ is curious, hence no contradiction
arises.

Commonly, $\mathbf{E^\ast}$ is the first order logic. However, it is
not often necessarily the case, and $\mathbf{E^\ast}$ above could
have been English anyway. We only need to formally distinguish the
statements formulating the mathematics from the statements
discussing those formulations, i.e., declaring the two classes of
statements to be disjunct, perhaps by italicizing the former.
Nevertheless, such approach evidently involves more work and is
understandably hard to follow.

\begin{thebibliography}{1}
\bibitem{Sc1997}
Schechter, E., \emph{Handbook of Analysis and Its Foundations}, 1st ed., Academic Press, 1997.
\end{thebibliography}
%%%%%
%%%%%
\end{document}
