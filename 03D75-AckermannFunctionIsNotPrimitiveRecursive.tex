\documentclass[12pt]{article}
\usepackage{pmmeta}
\pmcanonicalname{AckermannFunctionIsNotPrimitiveRecursive}
\pmcreated{2013-03-22 19:07:29}
\pmmodified{2013-03-22 19:07:29}
\pmowner{CWoo}{3771}
\pmmodifier{CWoo}{3771}
\pmtitle{Ackermann function is not primitive recursive}
\pmrecord{11}{42019}
\pmprivacy{1}
\pmauthor{CWoo}{3771}
\pmtype{Theorem}
\pmcomment{trigger rebuild}
\pmclassification{msc}{03D75}
\pmrelated{AckermannFunctionIsTotalRecursive}

\usepackage{amssymb,amscd}
\usepackage{amsmath}
\usepackage{amsfonts}
\usepackage{mathrsfs}

% used for TeXing text within eps files
%\usepackage{psfrag}
% need this for including graphics (\includegraphics)
%\usepackage{graphicx}
% for neatly defining theorems and propositions
\usepackage{amsthm}
% making logically defined graphics
%%\usepackage{xypic}
\usepackage{pst-plot}

% define commands here
\newcommand*{\abs}[1]{\left\lvert #1\right\rvert}
\newtheorem{prop}{Proposition}
\newtheorem{thm}{Theorem}
\newtheorem{cor}{Corollary}
\newtheorem{ex}{Example}
\newcommand{\real}{\mathbb{R}}
\newcommand{\pdiff}[2]{\frac{\partial #1}{\partial #2}}
\newcommand{\mpdiff}[3]{\frac{\partial^#1 #2}{\partial #3^#1}}
\begin{document}
In this entry, we show that the Ackermann function $A(x,y)$, given by $$A(0,y)=y+1, \qquad A(x+1,0)=A(x,1), \qquad A(x+1,y+1) = A(x,A(x+1,y))$$
is not primitive recursive.  We will utilize the properties of $A$ listed in \PMlinkname{this entry}{PropertiesOfAckermannFunction}.

The key to showing that $A$ is not primitive recursive, is to find a properties shared by all primitive recursive functions, but not by $A$.  One such property is in showing that $A$ in some way ``grows'' faster than any primitive recursive function.  This is formalized by the notion of ``majorization'', which is explained \PMlinkname{here}{SuperexponentiationIsNotElementary}.

\begin{prop} Let $\mathcal{A}$ be the set of all functions majorized by $A$.  Then $\mathcal{PR} \subseteq \mathcal{A}$.  \end{prop}
\begin{proof} We break this up into three parts: show all initial functions are in $\mathcal{A}$, show $\mathcal{A}$ is closed under functional composition, and show $\mathcal{A}$ is closed under primitive recursion.  The proof is completed by realizing that $\mathcal{PR}$ is the smallest set satisfying the three conditions.

In the proofs below, $\boldsymbol{x}$ denotes some tuple of non-negative integers $(x_1,\ldots,x_n)$ for some $n$, and $x = \max\lbrace x_1,\ldots,x_n\rbrace$.  Likewise for $\boldsymbol{y}$ and $y$.
\begin{enumerate}
\item We show that the zero function, the successor function, and the projection functions are in $\mathcal{A}$.
\begin{itemize}
\item $z(n)=0 < n+1=A(0,n)$, so $z\in \mathcal{A}$.
\item $s(n)=n+1 < n+2 = A(1,n)$, so $s\in \mathcal{A}$.
\item $p_m^k(x_1,\ldots, x_k)=x_m \le x < x+1 = A(0,x)$, so $p_m^k \in \mathcal{A}$.
\end{itemize}
\item Next, suppose $g_1, \ldots, g_m$ are $k$-ary, and $h$ is $m$-ary, and that each $g_i$, and $h$ are in $\mathcal{A}$.  This means that $g_i(\boldsymbol{x}) < A(r_i,x)$, and $h(\boldsymbol{y}) < A(s,y)$.  Let $$f=h(g_1,\ldots,g_m), \quad\mbox{and}\quad g_j(\boldsymbol{x}) = \max \lbrace g_i(\boldsymbol{x}) \mid i=1,\ldots, m \rbrace.$$  Then $f(\boldsymbol{x}) < A(s,g_j(\boldsymbol{x})) < A(s, A(r_j, x)) < A(s+ r_j+2,x)$, showing that $f\in \mathcal{A}$.
\item Finally, suppose $g$ is $k$-ary and $h$ is $(k+2)$-ary, and that $g,h\in \mathcal{A}$.  This means that $g(\boldsymbol{x})< A(r,x)$ and $h(\boldsymbol{y}) < A(s,y)$.   We want to show that $f$, defined by primitive recursion via functions $g$ and $h$, is in $\mathcal{A}$.

We first prove the following claim: $$f(\boldsymbol{x},n)<A(q,n+x), \quad \mbox{for some }q\mbox{ not depending on }x\mbox{ and }n.$$

Pick $q=1+\max \lbrace r,s\rbrace$, and induct on $n$.  First, $f(\boldsymbol{x},0) = g(\boldsymbol{x}) < A(r,x) <  A(q,x)$.  Next, suppose $f(\boldsymbol{x},n)< A(q,n+x)$.  Then $f(\boldsymbol{x},n+1) = h(\boldsymbol{x},n, f(\boldsymbol{x},n)) < A(s, z)$, where $z=\max \lbrace x,n, f(\boldsymbol{x},n)\rbrace$.  By the induction hypothesis, together with the fact that $\max\lbrace x,n\rbrace \le n+x < A(q,n+x)$, we see that $z<A(q,n+x)$.  Thus, $f(\boldsymbol{x},n+1) < A(s,z) < A(s,A(q,n+x)) \le A(q-1,A(q,n+x)) = A(q,n+1+x)$, proving the claim.

To finish the proof, let $z=\max\lbrace x,y\rbrace$.  Then, by the claim, $f(\boldsymbol{x},y) < A(q,x+y) \le A(q,2z) < A(q,2z+3) = A(q,A(2,z))=A(q+4,z)$, showing that $f\in \mathcal{A}$.
\end{enumerate}
Since $\mathcal{PR}$ is by definition the smallest set containing the initial functions, and closed under composition and primitive recursion, $\mathcal{PR}\subseteq \mathcal{A}$.
\end{proof}

As a corollary, we have
\begin{cor} The Ackermann function $A$ is not primitive recursive. \end{cor}
\begin{proof} Otherwise, $A\in \mathcal{A}$, which is impossible.  \end{proof}
%%%%%
%%%%%
\end{document}
