\documentclass[12pt]{article}
\usepackage{pmmeta}
\pmcanonicalname{InverseFunctionTheorem}
\pmcreated{2013-03-22 12:58:30}
\pmmodified{2013-03-22 12:58:30}
\pmowner{azdbacks4234}{14155}
\pmmodifier{azdbacks4234}{14155}
\pmtitle{inverse function theorem}
\pmrecord{9}{33346}
\pmprivacy{1}
\pmauthor{azdbacks4234}{14155}
\pmtype{Theorem}
\pmcomment{trigger rebuild}
\pmclassification{msc}{03E20}
\pmrelated{DerivativeOfInverseFunction}
\pmrelated{LegendreTransform}
\pmrelated{DerivativeAsParameterForSolvingDifferentialEquations}
\pmrelated{TheoryForSeparationOfVariables}

\endmetadata

% this is the default PlanetMath preamble.  as your knowledge
% of TeX increases, you will probably want to edit this, but
% it should be fine as is for beginners.

% almost certainly you want these
\usepackage{amssymb}
\usepackage{amsmath}
\usepackage{amsfonts}

% used for TeXing text within eps files
%\usepackage{psfrag}
% need this for including graphics (\includegraphics)
%\usepackage{graphicx}
% for neatly defining theorems and propositions
%\usepackage{amsthm}
% making logically defined graphics
%%%\usepackage{xypic}

% there are many more packages, add them here as you need them

% define commands here
\begin{document}
Let $\mathbf{f}$ be a continuously differentiable, vector-valued function mapping the open set $E \subset \mathbb{R}^{n}$ to $\mathbb{R}^{n}$ and let $S = \mathbf{f}(E)$.  If, for some point $\mathbf{a} \in E$, the Jacobian, $| J_{\mathbf{f}}(\mathbf{a}) |$, is non-zero, then there is a uniquely defined function $\mathbf{g}$ and two open sets $X \subset E$ and $Y \subset S$ such that
\begin{enumerate}
\item $\mathbf{a} \in X$, $\mathbf{f}(\mathbf{a}) \in Y$;
\item $Y = \mathbf{f}(X)$;
\item $\mathbf{f}:X \to Y$ is one-one;
\item $\mathbf{g}$ is continuously differentiable on $Y$ and $\mathbf{g}(\mathbf{f}(\mathbf{x})) = \mathbf{x}$ for all $\mathbf{x} \in X$.
\end{enumerate}

\subsubsection{Simplest case} When $n = 1$, this theorem becomes:  Let $f$ be a continuously differentiable, real-valued function defined on the open interval $I$.  If for some point $a \in I$, $f'(a) \neq 0$, then there is a neighbourhood $[\alpha, \beta]$ of $a$ in which $f$ is strictly monotonic.  Then $y \to f^{-1}(y)$ is a continuously differentiable, strictly monotonic function from $[f(\alpha), f(\beta)]$ to $[\alpha, \beta]$.  If $f$ is increasing (or decreasing) on $[\alpha, \beta]$, then so is $f^{-1}$ on $[f(\alpha), f(\beta)]$.

\subsubsection{Note} The inverse function theorem is a special case of the implicit function theorem where the dimension of each variable is the same.
%%%%%
%%%%%
\end{document}
