\documentclass[12pt]{article}
\usepackage{pmmeta}
\pmcanonicalname{JanLukasiewicz}
\pmcreated{2013-03-22 16:10:12}
\pmmodified{2013-03-22 16:10:12}
\pmowner{Mravinci}{12996}
\pmmodifier{Mravinci}{12996}
\pmtitle{Jan \L{}ukasiewicz}
\pmrecord{17}{38255}
\pmprivacy{1}
\pmauthor{Mravinci}{12996}
\pmtype{Definition}
\pmcomment{trigger rebuild}
\pmclassification{msc}{03G12}
\pmclassification{msc}{03G30}
\pmclassification{msc}{03G10}
\pmclassification{msc}{92B05}
\pmclassification{msc}{01A60}
\pmclassification{msc}{03G20}
\pmsynonym{Jan \L{ }ukasiewicz}{JanLukasiewicz}
\pmsynonym{Jan \L ukasiewicz}{JanLukasiewicz}
\pmsynonym{Jan Lukasiewicz}{JanLukasiewicz}
%\pmkeywords{many-valued logics}
%\pmkeywords{logic algebra}
%\pmkeywords{Chrysippean logic}
%\pmkeywords{Boolean logic}
%\pmkeywords{\L{}ukasiewicz and Post logic algebras}
\pmrelated{GeneticNetsOrNetworks}
\pmrelated{AlgebraicCategoryOfLMnLogicAlgebras}
\pmrelated{AnalyticsAndOntologyFormalLogics}
\pmrelated{FormalLogicsAndMetaMathematics}

% this is the default PlanetMath preamble.  as your knowledge
% of TeX increases, you will probably want to edit this, but
% it should be fine as is for beginners.

% almost certainly you want these
\usepackage{amssymb}
\usepackage{amsmath}
\usepackage{amsfonts}

% used for TeXing text within eps files
%\usepackage{psfrag}
% need this for including graphics (\includegraphics)
%\usepackage{graphicx}
% for neatly defining theorems and propositions
%\usepackage{amsthm}
% making logically defined graphics
%%%\usepackage{xypic}

% there are many more packages, add them here as you need them

% define commands here

\begin{document}
\emph{Jan \L{}ukasiewicz} (1878 - 1956) Polish mathematician and logician mainly concerned with logic in mathematics probably best known for the three-valued logics, the Polish notation, the law of excluded middle and the axiomatizations of much classical propositional logic. He studied at the University of Lw\'ow, earning doctorates in mathematics and philosophy in 1902. In 1919 he was Minister of Education in Poland. After WWII, he was exiled to Belgium. Arguably, the 3-valued logics named after him was the first published report of a non-Boolean, or non-Chrysippean logic (with only two logic values, `true' or `false', in the Chrysippean case). Subsequently, \L ukasiewicz logic algebras were constructed by Grigore Moisil in 1940-1945 to define `nuances' in logics, or many-valued logics, as well as 3-state control logic (electronic) circuits. \L ukasiewicz-Moisil ($LM_n$) logic algebras were defined axiomatically after 1969 as n-valued logic algebra representations and extensions of the \L ukasiewcz (3-valued) logics; then, the universal properties of 
\PMlinkname{categories of $LM_n$ -logic algebras}{AlgebraicCategoryOfLMnLogicAlgebras} were also investigated and reported in a series of recent publications.  Recently, several modifications of {\em $LM_n$-logic algebras} are under consideration as valid candidates for representations of {\em quantum logics}, as well as for modeling non-linear biodynamics in genetic `nets' or networks, as well as in single-cell organisms, or in tumor growth.



%%%%%
%%%%%
\end{document}
