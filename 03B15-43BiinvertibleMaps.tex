\documentclass[12pt]{article}
\usepackage{pmmeta}
\pmcanonicalname{43BiinvertibleMaps}
\pmcreated{2013-11-17 23:10:56}
\pmmodified{2013-11-17 23:10:56}
\pmowner{PMBookProject}{1000683}
\pmmodifier{rspuzio}{6075}
\pmtitle{4.3 Bi-invertible maps}
\pmrecord{2}{87665}
\pmprivacy{1}
\pmauthor{PMBookProject}{6075}
\pmtype{Feature}
\pmclassification{msc}{03B15}

\endmetadata

\usepackage{xspace}
\usepackage{amssyb}
\usepackage{amsmath}
\usepackage{amsfonts}
\usepackage{amsthm}
\newcommand{\biinv}{\ensuremath{\mathsf{biinv}}}
\newcommand{\defeq}{\vcentcolon\equiv}  
\newcommand{\define}[1]{\textbf{#1}}
\newcommand{\eqv}[2]{\ensuremath{#1 \simeq #2}\xspace}
\newcommand{\ishae}{\ensuremath{\mathsf{ishae}}}
\newcommand{\linv}{\ensuremath{\mathsf{linv}}}
\newcommand{\qinv}{\ensuremath{\mathsf{qinv}}}
\newcommand{\rinv}{\ensuremath{\mathsf{rinv}}}
\newcommand{\vcentcolon}{:\!\!}
\newcounter{mathcount}
\setcounter{mathcount}{1}
\newtheorem{precor}{Corollary}
\newenvironment{cor}{\begin{precor}}{\end{precor}\addtocounter{mathcount}{1}}
\renewcommand{\theprecor}{4.3.\arabic{mathcount}}
\newtheorem{predefn}{Definition}
\newenvironment{defn}{\begin{predefn}}{\end{predefn}\addtocounter{mathcount}{1}}
\renewcommand{\thepredefn}{4.3.\arabic{mathcount}}
\newtheorem{prethm}{Theorem}
\newenvironment{thm}{\begin{prethm}}{\end{prethm}\addtocounter{mathcount}{1}}
\renewcommand{\theprethm}{4.3.\arabic{mathcount}}
\let\autoref\cref

\begin{document}

\index{function!bi-invertible|(defstyle}%
\index{bi-invertible function|(defstyle}%
\index{equivalence!as bi-invertible function|(defstyle}%

Using the language introduced in \PMlinkname{\S 4.2}{42halfadjointequivalences}, we can restate the definition proposed in \PMlinkname{\S 2.4}{24homotopiesandequivalences} as follows.

\begin{defn}\label{defn:biinv}
  We say $f:A\to B$ is \define{bi-invertible}
  if it has both a left inverse and a right inverse:
  \[ \biinv (f) \defeq \linv(f) \times \rinv(f). \]
\end{defn}

In \PMlinkname{\S 2.4}{24homotopiesandequivalences} we proved that $\qinv(f)\to\biinv(f)$ and $\biinv(f)\to\qinv(f)$.
What remains is the following.

\begin{thm}\label{thm:isprop-biinv}
  For any $f:A\to B$, the type $\biinv(f)$ is a mere proposition.
\end{thm}
\begin{proof}
  We may suppose $f$ to be bi-invertible and show that $\biinv(f)$ is contractible.
  But since $\biinv(f)\to\qinv(f)$, by \PMlinkname{Lemma 4.2.9}{42halfadjointequivalences#Thmprelem4} in this case both $\linv(f)$ and $\rinv(f)$ are contractible, and the product of contractible types is contractible.
\end{proof}

Note that this also fits the proposal made at the beginning of \PMlinkname{\S 4.2}{42halfadjointequivalences}: we combine $g$ and $\eta$ into a contractible type and add an additional datum which combines with $\epsilon$ into a contractible type.
The difference is that instead of adding a \emph{higher} datum (a 2-dimensional path) to combine with $\epsilon$, we add a \emph{lower} one (a right inverse that is separate from the left inverse).

\begin{cor}\label{thm:equiv-biinv-isequiv}
  For any $f:A\to B$ we have $\eqv{\biinv(f)}{\ishae(f)}$.
\end{cor}
\begin{proof}
  We have $\biinv(f) \to \qinv(f) \to \ishae(f)$ and $\ishae(f) \to \qinv(f) \to \biinv(f)$.
  Since both $\ishae(f)$ and $\biinv(f)$ are mere propositions, the equivalence follows from \PMlinkname{Lemma 3.3.3}{33merepropositions#Thmprelem2}.
\end{proof}

\index{function!bi-invertible|)}%
\index{bi-invertible function|)}%
\index{equivalence!as bi-invertible function|)}%


\end{document}
