\documentclass[12pt]{article}
\usepackage{pmmeta}
\pmcanonicalname{ArithmeticalHierarchyIsAProperHierarchy}
\pmcreated{2013-03-22 12:55:14}
\pmmodified{2013-03-22 12:55:14}
\pmowner{Henry}{455}
\pmmodifier{Henry}{455}
\pmtitle{arithmetical hierarchy is a proper hierarchy}
\pmrecord{6}{33273}
\pmprivacy{1}
\pmauthor{Henry}{455}
\pmtype{Result}
\pmcomment{trigger rebuild}
\pmclassification{msc}{03B10}

\endmetadata

% this is the default PlanetMath preamble.  as your knowledge
% of TeX increases, you will probably want to edit this, but
% it should be fine as is for beginners.

% almost certainly you want these
\usepackage{amssymb}
\usepackage{amsmath}
\usepackage{amsfonts}

% used for TeXing text within eps files
%\usepackage{psfrag}
% need this for including graphics (\includegraphics)
%\usepackage{graphicx}
% for neatly defining theorems and propositions
%\usepackage{amsthm}
% making logically defined graphics
%%%\usepackage{xypic}

% there are many more packages, add them here as you need them

% define commands here
%\PMlinkescapeword{theory}
\begin{document}
By definition, we have $\Delta_n=\Pi_n\cap \Sigma_n$.  In addition, $\Sigma_n\cup\Pi_n\subseteq \Delta_{n+1}$.

This is proved by vacuous quantification.  If $R$ is equivalent to $\phi(\vec{n})$ then $R$ is equivalent to $\forall x\phi(\vec{n})$ and $\exists x\phi(\vec{n})$, where $x$ is some variable that does not occur free in $\phi$.

More significant is the proof that all containments are proper.  First, let $n\geq 1$ and $U$ be universal for $2$-ary $\Sigma_n$ relations.  Then $D(x)\leftrightarrow U(x,x)$ is obviously $\Sigma_n$.  But suppose $D\in \Delta_n$.  Then $D\in Pi_n$, so $\neg D\in\Sigma_n$. Since $U$ is universal, ther is some $e$ such that $\neg D(x)\leftrightarrow U(e,x)$, and therefore $\neg D(e)\leftrightarrow U(e,e)\leftrightarrow \neg U(e,e)$.  This is clearly a contradiction, so $D\in\Sigma_n\setminus\Delta_n$ and $\neg D\in\Pi_n\setminus\Delta_n$.

In addition the recursive join of $D$ and $\neg D$, defined by 
$$D\oplus\neg D(x)\leftrightarrow (\exists y<x[x=2\cdot y]\wedge D(x)) \vee (\neg\exists y<x[x=2\cdot y]\wedge \neg D(x))$$

Clearly both $D$ and $\neg D$ can be recovered from $D\oplus\neg D$, so it is contained in neither $\Sigma_n$ nor $\Pi_n$.  However the definition above has only unbounded quantifiers except for those in $D$ and $\neg D$, so $D\oplus\neg D(x)\in \Delta_{n+1}\setminus\Sigma_n\cup\Pi_n$
%%%%%
%%%%%
\end{document}
