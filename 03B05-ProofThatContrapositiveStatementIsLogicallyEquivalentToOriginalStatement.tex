\documentclass[12pt]{article}
\usepackage{pmmeta}
\pmcanonicalname{ProofThatContrapositiveStatementIsLogicallyEquivalentToOriginalStatement}
\pmcreated{2013-03-22 13:42:10}
\pmmodified{2013-03-22 13:42:10}
\pmowner{sprocketboy}{2515}
\pmmodifier{sprocketboy}{2515}
\pmtitle{proof that contrapositive statement is logically equivalent to original statement}
\pmrecord{10}{34379}
\pmprivacy{1}
\pmauthor{sprocketboy}{2515}
\pmtype{Proof}
\pmcomment{trigger rebuild}
\pmclassification{msc}{03B05}
\pmrelated{Inverse7}
\pmrelated{Inverse6}

% this is the default PlanetMath preamble.  as your knowledge
% of TeX increases, you will probably want to edit this, but
% it should be fine as is for beginners.

% almost certainly you want these
\usepackage{amssymb}
\usepackage{amsmath}
\usepackage{amsfonts}

% used for TeXing text within eps files
%\usepackage{psfrag}
% need this for including graphics (\includegraphics)
%\usepackage{graphicx}
% for neatly defining theorems and propositions
%\usepackage{amsthm}
% making logically defined graphics
%%%\usepackage{xypic}

% there are many more packages, add them here as you need them


% define commands here
\begin{document}
You can see that the contrapositive of an implication is true by considering
the following:

The statement $p\Rightarrow q$ is logically equivalent to $\neg p\vee q$ which can also be written as $\overline{p}\vee q$.

By the same token, the contrapositive statement $\overline{q}\Rightarrow \overline{p}$ is logically equivalent to $\neg \overline{q}\vee \overline{p}$ which, using double negation on $q$, becomes $q\vee \overline{p}$.

This, of course, is the same logical statement.
%%%%%
%%%%%
\end{document}
