\documentclass[12pt]{article}
\usepackage{pmmeta}
\pmcanonicalname{TransitiveRelation}
\pmcreated{2013-03-22 12:15:52}
\pmmodified{2013-03-22 12:15:52}
\pmowner{yark}{2760}
\pmmodifier{yark}{2760}
\pmtitle{transitive relation}
\pmrecord{14}{31669}
\pmprivacy{1}
\pmauthor{yark}{2760}
\pmtype{Definition}
\pmcomment{trigger rebuild}
\pmclassification{msc}{03E20}
\pmrelated{Reflexive}
\pmrelated{Symmetric}
\pmrelated{Antisymmetric}
\pmdefines{transitivity}
\pmdefines{transitive}

\usepackage{amssymb}
\usepackage{amsmath}
\usepackage{amsfonts}
\begin{document}
A relation $\mathcal{R}$ on a set $A$ is \emph{transitive} if and only if
$\forall x,y,z \in A$, $(x\mathcal{R}y \land y\mathcal{R}z) \rightarrow (x\mathcal{R}z)$.

For example, the ``is a subset of'' relation $\subseteq$
on any set of sets is transitive.
The ``less than'' relation $<$ on the set of real numbers
is also transitive.

The ``is not equal to'' relation $\neq$
on the set of integers is not transitive,
because $1\neq 2$ and $2\neq 1$ does not imply $1\neq 1$.

%%%%%
%%%%%
%%%%%
\end{document}
