\documentclass[12pt]{article}
\usepackage{pmmeta}
\pmcanonicalname{ProofOfTheoremsInAdditivelyIndecomposable}
\pmcreated{2013-03-22 13:29:07}
\pmmodified{2013-03-22 13:29:07}
\pmowner{mathcam}{2727}
\pmmodifier{mathcam}{2727}
\pmtitle{proof of theorems in additively indecomposable}
\pmrecord{9}{34057}
\pmprivacy{1}
\pmauthor{mathcam}{2727}
\pmtype{Proof}
\pmcomment{trigger rebuild}
\pmclassification{msc}{03E10}
\pmclassification{msc}{03F15}

\endmetadata

% this is the default PlanetMath preamble.  as your knowledge
% of TeX increases, you will probably want to edit this, but
% it should be fine as is for beginners.

% almost certainly you want these
\usepackage{amssymb}
\usepackage{amsmath}
\usepackage{amsfonts}

% used for TeXing text within eps files
%\usepackage{psfrag}
% need this for including graphics (\includegraphics)
%\usepackage{graphicx}
% for neatly defining theorems and propositions
%\usepackage{amsthm}
% making logically defined graphics
%%%\usepackage{xypic}

% there are many more packages, add them here as you need them

% define commands here
%\PMlinkescapeword{theory}
\begin{document}
\PMlinkescapeword{decomposable}

\begin{itemize} 

\item $\mathbb{H}$ is closed.\\

Let $\{\alpha_i\mid i<\kappa\}$ be some increasing sequence of elements of $\mathbb{H}$ and let $\alpha=\sup \{\alpha_i\mid i<\kappa\}$.  Then for any $x,y<\alpha$,  it must be that $x<\alpha_i$ and $y<\alpha_j$ for some $i,j<\kappa$.  But then $x+y<\alpha_{\max\{i,j\}}<\alpha$.

\item $\mathbb{H}$ is unbounded.\\

Consider any $\alpha$, and define a sequence by $\alpha_0=S\alpha$ and $\alpha_{n+1}=\alpha_n+\alpha_n$.  Let $\alpha_\omega=\sup_{n<\omega}\alpha_n$ be the limit of this sequence.  If $x,y<\alpha_\omega$ then it must be that $x<\alpha_i$ and $y<\alpha_j$ for some $i,j<\omega$, and therefore $x+y<\alpha_{\max\{i,j\}+1}$.  Note that $\alpha_\omega$ is, in fact, the next element of $\mathbb{H}$, since every element in the sequence is clearly additively decomposable.

\item $f_\mathbb{H}(\alpha)=\omega^\alpha$.\\

Since $0$ is not in $\mathbb{H}$, we have $f_\mathbb{H}(0)=1$.

For any $\alpha+1$, we have $f_\mathbb{H}(\alpha+1)$ is the least additively indecomposable number greater than $f_\mathbb{H}(\alpha)$.  Let $\alpha_0=Sf_\mathbb{H}(\alpha)$ and $\alpha_{n+1}=\alpha_n+\alpha_n=\alpha_n\cdot 2$.  Then $f_\mathbb{H}(\alpha+1)=\sup_{n<\omega} \alpha_n=\sup_{n<\omega}S\alpha\cdot 2\cdots 2=f_\mathbb{H}(\alpha)\cdot\omega$.  The limit case is trivial since $\mathbb{H}$ is closed and unbounded, so $f_\mathbb{H}$ is continuous.
\end{itemize}
%%%%%
%%%%%
\end{document}
