\documentclass[12pt]{article}
\usepackage{pmmeta}
\pmcanonicalname{FreelyGeneratedInductiveSet}
\pmcreated{2013-03-22 18:51:24}
\pmmodified{2013-03-22 18:51:24}
\pmowner{CWoo}{3771}
\pmmodifier{CWoo}{3771}
\pmtitle{freely generated inductive set}
\pmrecord{11}{41666}
\pmprivacy{1}
\pmauthor{CWoo}{3771}
\pmtype{Definition}
\pmcomment{trigger rebuild}
\pmclassification{msc}{03B99}
\pmclassification{msc}{03E20}
\pmrelated{ClosureOfSetsClosedUnderAFinitaryOperation}
\pmdefines{inductive closure}

\usepackage{amssymb,amscd}
\usepackage{amsmath}
\usepackage{amsfonts}
\usepackage{mathrsfs}

% used for TeXing text within eps files
%\usepackage{psfrag}
% need this for including graphics (\includegraphics)
%\usepackage{graphicx}
% for neatly defining theorems and propositions
\usepackage{amsthm}
% making logically defined graphics
%%\usepackage{xypic}
\usepackage{pst-plot}

% define commands here
\newcommand*{\abs}[1]{\left\lvert #1\right\rvert}
\newtheorem{prop}{Proposition}
\newtheorem{thm}{Theorem}
\newtheorem{ex}{Example}
\newcommand{\real}{\mathbb{R}}
\newcommand{\pdiff}[2]{\frac{\partial #1}{\partial #2}}
\newcommand{\mpdiff}[3]{\frac{\partial^#1 #2}{\partial #3^#1}}
\begin{document}
In the parent entry, we see that an \emph{inductive set} is a set that is closed under the successor operator.  If $A$ is a non-empty inductive set, then $\mathbb{N}$ can be embedded in $A$.  

More generally, fix a non-empty set $U$ and a set $F$ of finitary operations on $U$.  A set $A\subseteq U$ is said to be \emph{inductive} (with respect to $F$) if $A$ is closed under each $f\in F$.  This means, for example, if $f$ is a binary operation on $U$ and if $x,y\in A$, then $f(x,y)\in A$.  $A$ is said to be inductive over $X$ if $X\subseteq A$.  The intersection of inductive sets is clearly inductive.  Given a set $X\subseteq U$, the intersection of all inductive sets over $X$ is said to be the \emph{inductive closure} of $X$.  The inductive closure of $X$ is written $\langle X\rangle$.  We also say that $X$ generates $\langle X\rangle$.  

Another way of defining $\langle X\rangle$ is as follows: start with $$X_0 :=X.$$  Next, we ``inductively'' define each $X_{i+1}$ from $X_i$, so that $$X_{i+1}:= X_i \cup \bigcup \lbrace f(X_i^n)\mid f\in F, f \mbox{ is } n\mbox{-ary}\rbrace.$$  Finally, we set $$\overline{X}: = \bigcup_{i=0}^{\infty} X_i.$$  It is not hard to see that $\overline{X}=\langle X\rangle$.
\begin{proof}  By definition, $X\subseteq \overline{X}$.  Suppose $f\in F$ is $n$-ary, and $a_1,\ldots, a_n\in \overline{X}$, then each $a_i \in X_{m(i)}$.  Take the maximum $m$ of the integers $m(i)$, then $a_i \in X_m$ for each $i$.  Therefore $f(a_1,\ldots, a_n) \in X_{m+1} \subseteq \overline{X}$.  This shows that $\overline{X}$ is inductive over $X$, so $\langle X\rangle \subseteq \overline{X}$, since $\langle X\rangle$ is minimal.  On the other hand, suppose $a\in \overline{X}$.  We prove by induction that $a\in \langle X\rangle$.  If $a\in X$, this is clear.  Suppose now that $X_i\subseteq \langle X\rangle$, and $a\in X_{i+1}$.  If $a\in X_i$, then we are done.  Suppose now $a\in X_{i+1}-X_i$.  Then there is some $n$-ary operation $f\in F$, such that $a=f(a_1,\ldots, a_n)$, where each $a_j\in X_i$.  So $a_j\in \langle X\rangle$ by hypothesis.  Since $\langle X\rangle$ is inductive, $f(a_1,\ldots, a_n) \in \langle X\rangle$, and hence $a\in \langle X\rangle$ as well.  This shows that $X_{i+1}\subseteq A$, and consequently $\overline{X} \subseteq \langle X\rangle$.
\end{proof}

The inductive set $A$ is said to be freely generated by $X$ (with respect to $F$), if the following conditions are satisfied:
\begin{enumerate}
\item $A=\langle X\rangle$,
\item for each $n$-ary $f\in F$, the restriction of $f$ to $A^n$ is one-to-one;
\item for each $n$-ary $f\in F$, $f(A^n)\cap X=\varnothing$;
\item if $f,g\in F$ are $n,m$-ary, then $f(A^n)\cap g(A^m)=\varnothing$.
\end{enumerate}

For example, the set $\overline{V}$ of well-formed formulas (wffs) in the classical proposition logic is inductive over the set of $V$ propositional variables with respect to the logical connectives (say, $\neg$ and $\vee$) provided.  In fact, by unique readability of wffs, $\overline{V}$ is freely generated over $V$.  We may readily interpret the above ``freeness'' conditions as follows:
\begin{enumerate}
\item $\overline{V}$ is generated by $V$,
\item for distinct wffs $p,q$, the wffs $\neg p$ and $\neg q$ are distinct; for distinct pairs $(p,q)$ and $(r,s)$ of wffs, $p\vee q$ and $r\vee s$ are distinct also
\item for no wffs $p,q$ are $\neg p$ and $p\vee q$ propositional variables
\item for wffs $p,q$, the wffs $\neg p$ and $p\vee q$ are never the same
\end{enumerate}

A characterization of free generation is the following:
\begin{prop} The following are equivalent:
\begin{enumerate}
\item $A$ is freely generated by $X$ (with respect to $F$)
\item if $V\ne \varnothing$ is a set, and $G$ is a set of finitary operations on $V$ such that there is a function $\phi: F\to G$ taking every $n$-ary $f\in F$ to an $n$-ary $\phi(f)\in G$, then every function $h: X \to B$ has a unique extension $\overline{h}: A\to B$ such that $$\overline{h}(f(a_1,\ldots, a_n))=\phi(f)(\overline{h}(a_1),\ldots, \overline{h}(a_n)),$$ where $f$ is an $n$-ary operation in $F$, and $a_i\in A$.
\end{enumerate}
\end{prop}

\begin{thebibliography}{7}
\bibitem{he} H. Enderton: {\em A Mathematical Introduction to Logic}, Academic Press, San Diego (1972).
\end{thebibliography}
%%%%%
%%%%%
\end{document}
