\documentclass[12pt]{article}
\usepackage{pmmeta}
\pmcanonicalname{PairingFunction}
\pmcreated{2013-03-22 14:34:46}
\pmmodified{2013-03-22 14:34:46}
\pmowner{rspuzio}{6075}
\pmmodifier{rspuzio}{6075}
\pmtitle{pairing function}
\pmrecord{18}{36140}
\pmprivacy{1}
\pmauthor{rspuzio}{6075}
\pmtype{Definition}
\pmcomment{trigger rebuild}
\pmclassification{msc}{03D20}
\pmrelated{ExampleOfBijection}

% this is the default PlanetMath preamble.  as your knowledge
% of TeX increases, you will probably want to edit this, but
% it should be fine as is for beginners.

% almost certainly you want these
\usepackage{amssymb}
\usepackage{amsmath}
\usepackage{amsfonts}

% used for TeXing text within eps files
%\usepackage{psfrag}
% need this for including graphics (\includegraphics)
%\usepackage{graphicx}
% for neatly defining theorems and propositions
%\usepackage{amsthm}
% making logically defined graphics
%%%\usepackage{xypic}

% there are many more packages, add them here as you need them

% define commands here

\begin{document}
A pairing function is a function $P \colon \mathbb{Z}_+^2 \to \mathbb{Z}_+$ which establishes a one-to-one correspondence between $\mathbb{Z}_+^2$ and $\mathbb{Z}_+$.  Such functions are useful in the theory of recursive functions because they allow one to express recursive functions of $m$ variables in terms of recursive functions of $n$ variables with $m \ne n$.

Two examples of pairing functions are the following;
 $$P_1 (x,y) = (x+y)(x+y+1)/2 + y$$
 $$P_2 (x,y) = 2^x (2y + 1) - 1$$

It is not hard to see that these functions are recursive (actually, primitive recursive).  For instance, one could use the recursion relations and initial conditions
 $$P_1 (x+1,y) = P_1(x,y) + x + y + 1$$
 $$P_1 (0,y) = T(y) + y$$
 $$T(y+1) = T(y) + y + 1$$
 $$T(0) = 0$$
where $T(n)$ is the $n$-th triangular number to show that $P_1$ is recursive. Likewise, one could use the recursions
 $$P_2 (x+1,y) = P_2 (x,y) + P_2 (x,y)$$
 $$P_2 (0,y) = y + y$$
to show that $P_2$ is recursive.

An easy way to see that $P_1$ effects a one-to-one correspondence between $\mathbb{Z}_+^2$ and $\mathbb{Z}_+$ is as follows:  Define the ``successor'' of a pair $(x,y) \in \mathbb{Z}_+^2$ to be the pair $(x-1,y+1)$ when $x\neq0$; otherwise, when $x=0$, the successor is $(y+1,0)$.  It is easy to see that every pair has a successor and that every pair except $(0,0)$ is the successor of exactly one other pair.  With this definition of successor, the set of pairs of positive integers satisfies the Peano axioms and, hence, is isomorphic to the integers.  From the definition of $P_1$ it follows that, if $(x',y')$ is the successor of $(x,y)$, then $P_1 (x',y') = P(x,y) + 1$ and that $P_1(0,0) = 0$.  This means that $P_1$ is the isomorphism described two sentences ago.

That $P_2$ effects a one-to-one correspondence between positive integers and pairs of positive integers follows readily from uniqueness of factorization of integers.  On the one hand, for any number $z$, one can find numbers $x$ and $y$ such that $z = P_2 (x,y)$ by factoring $z+1$ and letting $x$ be the power of $2$ which appears in the factorization.  On the other hand, this is the only solution of $z = P_2 (x,y)$ because prime factorization is unique.

Since a pairing function $P$ sets up a 1-1 correspondence between $\mathbb{Z}_+$ and $\mathbb{Z}_+^n$, there exist uniquely defined unpairing functions $R$ and $L$ such that
 $$P(L(x),R(x)) = x$$
It is not hard to show that, if $P$ is recursive, $R$ and $L$ will also be recursive.

Once one has a pairing function $P^{(2)}$, one can use it to set up 1-1 correspondences between $\mathbb{Z}_+$ and $\mathbb{Z}_+^n$ for any $n$.  For instance, one could define
 $$P^{(3)} (x,y,z) = P^{(2)} (x, P^{(2)}(y,z))$$
 $$P^{(4)} (x,y,z,w) = P^{(2)} (x, P^{(3)}(y,z,w)) = P^{(2)} (x,P^{(2)}(y, P^{(2)}(z,w)))$$
In general, 
 $$P^{(n+1)} (x_1, \ldots, x_{n+1}) = P^{(2)} (x_1, P^{(n)} (x_2, \ldots, x_n))$$
(This manner of encoding a list one pair at a time will be familiar to anyone who has programmed a computer in LISP.  In fact, LISP was designed to be serve as a mathematical definition of computability equivalent to Turing machines or recursive functions.  A fun exercise is to write a compiler which translates LISP programs into recursive functions using the representation of lists by single integers defined above.)

An important consequence of the fact noted above is that there is a 1-1 correspondence between recursive functions of $n$ variables and recursive functions of a single variable.  If we have a function $F \colon \mathbb{Z}_+^n \to \mathbb{Z}_+$, we can associate to it the function $G \colon \mathbb{Z}_+ \to \mathbb{Z}_+$ by the formula
 $$F(x_1, \ldots, x_n) = G( P^{(n)} (x_1, \ldots, x_n))$$
Doing this can often save work by allowing one to draw conclusions about recursive functions of
several variables from the special case of functions of one variable.

%%%%%
%%%%%
\end{document}
