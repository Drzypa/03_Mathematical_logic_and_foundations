\documentclass[12pt]{article}
\usepackage{pmmeta}
\pmcanonicalname{IndexSet}
\pmcreated{2013-03-22 18:09:48}
\pmmodified{2013-03-22 18:09:48}
\pmowner{yesitis}{13730}
\pmmodifier{yesitis}{13730}
\pmtitle{index set}
\pmrecord{5}{40723}
\pmprivacy{1}
\pmauthor{yesitis}{13730}
\pmtype{Definition}
\pmcomment{trigger rebuild}
\pmclassification{msc}{03D25}

% this is the default PlanetMath preamble.  as your knowledge
% of TeX increases, you will probably want to edit this, but
% it should be fine as is for beginners.

% almost certainly you want these
\usepackage{amssymb}
\usepackage{amsmath}
\usepackage{amsfonts}

% used for TeXing text within eps files
%\usepackage{psfrag}
% need this for including graphics (\includegraphics)
%\usepackage{graphicx}
% for neatly defining theorems and propositions
%\usepackage{amsthm}
% making logically defined graphics
%%%\usepackage{xypic}

% there are many more packages, add them here as you need them

% define commands here

\begin{document}
In computability theory, a set $A\subseteq\omega$ is called an \emph{index set} if for all $x, y$,

\begin{equation*}
x\in A, \varphi_x=\varphi_y \Longrightarrow y\in A.
\end{equation*} 

$\varphi_x$ stands for the partial function with G\"odel number (or index) $x$.

Thus, if $A$ is an index set and $\varphi_x=\varphi_y$, then either $x, y\in A$ or $x, y\not\in A$. Intuitively, if $A$ contains the G\"odel index $x$ of a partial function $\varphi$, then $A$ contains all indices for the partial function. (Recall that there are $\aleph_0$ G\"odel numbers for each partial function.)

It is instructive to compare the notion of an index set in computability theory with that of an \emph{indexing} set.
%%%%%
%%%%%
\end{document}
