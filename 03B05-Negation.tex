\documentclass[12pt]{article}
\usepackage{pmmeta}
\pmcanonicalname{Negation}
\pmcreated{2015-04-25 17:44:13}
\pmmodified{2015-04-25 17:44:13}
\pmowner{pahio}{2872}
\pmmodifier{pahio}{2872}
\pmtitle{negation}
\pmrecord{8}{38619}
\pmprivacy{1}
\pmauthor{pahio}{2872}
\pmtype{Definition}
\pmcomment{trigger rebuild}
\pmclassification{msc}{03B05}
\pmsynonym{logical not}{Negation}
\pmrelated{SetMembership}

% this is the default PlanetMath preamble.  as your knowledge
% of TeX increases, you will probably want to edit this, but
% it should be fine as is for beginners.

% almost certainly you want these
\usepackage{amssymb}
\usepackage{amsmath}
\usepackage{amsfonts}

% used for TeXing text within eps files
%\usepackage{psfrag}
% need this for including graphics (\includegraphics)
%\usepackage{graphicx}
% for neatly defining theorems and propositions
 \usepackage{amsthm}
% making logically defined graphics
%%%\usepackage{xypic}

% there are many more packages, add them here as you need them

% define commands here

\theoremstyle{definition}
\newtheorem*{thmplain}{Theorem}

\begin{document}
\PMlinkescapeword{symbol} \PMlinkescapeword{line}

In logics and mathematics, {\em negation} (from Latin {\em negare} `to deny') is the unary operation ``$\lnot$'' which swaps the truth value of any operand to the \PMlinkescapetext{opposite} truth value.\, So, if the statement $P$ is true then its negated statement $\lnot P$ is false, and vice versa.

\textbf{Note 1.}\, The negated statement $\lnot P$ (by Heyting) has been denoted also with $-P$ (Peano), $\sim\! P$ (Russell), $\overline{P}$ (Hilbert) and $NP$ (by the Polish notation).

\textbf{Note 2.}\, $\lnot P$ may be expressed by implication as
$$P\to\curlywedge$$
where $\curlywedge$ means any contradictory statement.

\textbf{Note 3.}\, The negation of logical or and logical and give the results
$$\lnot(P\lor Q) \equiv \lnot P \land \lnot Q, \qquad 
  \lnot(P\land Q) \equiv \lnot P \lor \lnot Q.$$
Analogical results concern the quantifier statements:
$$\lnot (\exists x)P(x) \equiv (\forall x)\lnot P(x), \qquad
  \lnot (\forall x)P(x) \equiv (\exists x)\lnot P(x).$$
These all are known as de Morgan's laws.

\textbf{Note 4.}\, Many mathematical relation statements, expressed with such special relation symbols as\, $=,\, \subseteq,\, \in,\, \cong,\, \parallel,\, \mid$,\, are negated by using in the symbol an additional cross line:\,\,
 $\neq,\, \nsubseteq,\, \notin,\, \ncong,\, \nparallel,\, \nmid$.
%%%%%
%%%%%
\end{document}
