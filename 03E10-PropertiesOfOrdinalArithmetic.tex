\documentclass[12pt]{article}
\usepackage{pmmeta}
\pmcanonicalname{PropertiesOfOrdinalArithmetic}
\pmcreated{2013-03-22 17:51:05}
\pmmodified{2013-03-22 17:51:05}
\pmowner{CWoo}{3771}
\pmmodifier{CWoo}{3771}
\pmtitle{properties of ordinal arithmetic}
\pmrecord{8}{40326}
\pmprivacy{1}
\pmauthor{CWoo}{3771}
\pmtype{Result}
\pmcomment{trigger rebuild}
\pmclassification{msc}{03E10}
\pmrelated{OrdinalExponentiation}

\usepackage{amssymb,amscd}
\usepackage{amsmath}
\usepackage{amsfonts}
\usepackage{mathrsfs}

% used for TeXing text within eps files
%\usepackage{psfrag}
% need this for including graphics (\includegraphics)
%\usepackage{graphicx}
% for neatly defining theorems and propositions
\usepackage{amsthm}
% making logically defined graphics
%%\usepackage{xypic}
\usepackage{pst-plot}

% define commands here
\newcommand*{\abs}[1]{\left\lvert #1\right\rvert}
\newtheorem{prop}{Proposition}
\newtheorem{thm}{Theorem}
\newtheorem{ex}{Example}
\newcommand{\real}{\mathbb{R}}
\newcommand{\pdiff}[2]{\frac{\partial #1}{\partial #2}}
\newcommand{\mpdiff}[3]{\frac{\partial^#1 #2}{\partial #3^#1}}
\begin{document}
Let $\textbf{On}$ be the class of ordinals, and $\alpha,\beta,\gamma,\delta\in \textbf{On}$.  Then the following properties are satisfied:
\begin{enumerate}
\item (additive identity): $\alpha+0=0+\alpha=\alpha$ (\PMlinkname{proof}{ExampleOfTransfiniteInduction})
\item (associativity of addition): $\alpha+(\beta+\gamma)=(\alpha+\beta)+\gamma$
\item (multiplicative identity): $\alpha\cdot 1=1\cdot \alpha=\alpha$
\item (multiplicative zero): $\alpha\cdot 0 = 0\cdot \alpha=0$
\item (associativity of multiplication): $\alpha\cdot (\beta\cdot \gamma)=(\alpha\cdot \beta)\cdot \gamma$
\item (left distributivity): $\alpha\cdot(\beta+\gamma)=\alpha\cdot \beta+\alpha\cdot \gamma$
\item (existence and uniqueness of subtraction): if $\alpha\le \beta$, then there is a unique $\gamma$ such that $\alpha+\gamma=\beta$
\item (existence and uniqueness of division): for any $\alpha,\beta$ with $\beta\ne 0$, there exists a unique pair of ordinals $\gamma,\delta$ such that $\alpha=\beta\cdot \delta+\gamma$ and $\gamma<\beta$.
\end{enumerate}

Conspicuously absent from the above list of properties are the commutativity laws, as well as right distributivity of multiplication over addition.  Below are some \PMlinkescapetext{simple} counterexamples:
\begin{itemize}
\item $\omega+1\ne 1+\omega=\omega$, for the former has a top element and the latter does not.
\item $\omega\cdot 2\ne 2\cdot \omega$, for the former is $\omega+\omega$, which consists an element $\alpha$ such that $\beta<\alpha$ for all $\beta<\omega$, and the latter is $2\cdot \sup \lbrace n\mid n<\omega\rbrace = \sup \lbrace 2\cdot n\mid n<\omega \rbrace =\sup \lbrace n\mid n<\omega\rbrace$, which is just $\omega$, and which does not consist such an element $\alpha$
\item $(1+1)\cdot \omega\ne 1\cdot \omega+1\cdot \omega$, for the former is $2\cdot \omega$ and the latter is $\omega\cdot 2$, and the rest of the \PMlinkescapetext{argument} follows from the previous counterexample.
\end{itemize}

All of the properties above can be proved using transfinite induction.  For a proof of the first property, please see \PMlinkname{this link}{ExampleOfTransfiniteInduction}.

For properties of the arithmetic regarding exponentiation of ordinals, please refer to \PMlinkname{this link}{OrdinalExponentiation}.
%%%%%
%%%%%
\end{document}
