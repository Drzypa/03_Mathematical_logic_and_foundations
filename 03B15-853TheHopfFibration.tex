\documentclass[12pt]{article}
\usepackage{pmmeta}
\pmcanonicalname{853TheHopfFibration}
\pmcreated{2013-11-06 14:29:35}
\pmmodified{2013-11-06 14:29:35}
\pmowner{PMBookProject}{1000683}
\pmmodifier{rspuzio}{6075}
\pmtitle{8.5.3 The Hopf fibration}
\pmrecord{2}{87720}
\pmprivacy{1}
\pmauthor{PMBookProject}{6075}
\pmtype{Feature}
\pmclassification{msc}{03B15}

\endmetadata

\usepackage{xspace}
\usepackage{amssyb}
\usepackage{amsmath}
\usepackage{amsfonts}
\usepackage{amsthm}
\makeatletter
\newcommand{\base}{\ensuremath{\mathsf{base}}\xspace}
\newcommand{\bool}{\ensuremath{\mathbf{2}}\xspace}
\newcommand{\ct}{  \mathchoice{\mathbin{\raisebox{0.5ex}{$\displaystyle\centerdot$}}}             {\mathbin{\raisebox{0.5ex}{$\centerdot$}}}             {\mathbin{\raisebox{0.25ex}{$\scriptstyle\,\centerdot\,$}}}             {\mathbin{\raisebox{0.1ex}{$\scriptscriptstyle\,\centerdot\,$}}}}
\newcommand{\defeq}{\vcentcolon\equiv}  
\newcommand{\defid}{\coloneqq}
\def\@dprd#1{\prod_{(#1)}\,}
\def\@dprd@noparens#1{\prod_{#1}\,}
\def\@dsm#1{\sum_{(#1)}\,}
\def\@dsm@noparens#1{\sum_{#1}\,}
\def\@eatprd\prd{\prd@parens}
\def\@eatsm\sm{\sm@parens}
\newcommand{\eqv}[2]{\ensuremath{#1 \simeq #2}\xspace}
\newcommand{\fst}{\ensuremath{\proj1}\xspace}
\newcommand{\funext}{\mathsf{funext}}
\newcommand{\glue}{\mathsf{glue}}
\newcommand{\happly}{\mathsf{happly}}
\newcommand{\idfunc}[1][]{\ensuremath{\mathsf{id}_{#1}}\xspace}
\newcommand{\indexsee}[2]{\index{#1|see{#2}}}    
\newcommand{\inl}{\ensuremath\inlsym\xspace}
\newcommand{\inlsym}{{\mathsf{inl}}}
\newcommand{\inr}{\ensuremath\inrsym\xspace}
\newcommand{\inrsym}{{\mathsf{inr}}}
\def\lam#1{{\lambda}\@lamarg#1:\@endlamarg\@ifnextchar\bgroup{.\,\lam}{.\,}}
\def\@lamarg#1:#2\@endlamarg{\if\relax\detokenize{#2}\relax #1\else\@lamvar{\@lameatcolon#2},#1\@endlamvar\fi}
\def\@lameatcolon#1:{#1}
\def\@lamvar#1,#2\@endlamvar{(#2\,{:}\,#1)}
\newcommand{\lloop}{\ensuremath{\mathsf{loop}}\xspace}
\newcommand{\map}[2]{\ensuremath{{#1}\mathopen{}\left({#2}\right)\mathclose{}}\xspace}
\newcommand{\mapdepfunc}[1]{\ensuremath{\mathsf{apd}_{#1}}\xspace} 
\newcommand{\mapfunc}[1]{\ensuremath{\mathsf{ap}_{#1}}\xspace} 
\newcommand{\narrowbreak}{}
\def\prd#1{\@ifnextchar\bgroup{\prd@parens{#1}}{\@ifnextchar\sm{\prd@parens{#1}\@eatsm}{\prd@noparens{#1}}}}
\def\prd@noparens#1{\mathchoice{\@dprd@noparens{#1}}{\@tprd{#1}}{\@tprd{#1}}{\@tprd{#1}}}
\def\prd@parens#1{\@ifnextchar\bgroup  {\mathchoice{\@dprd{#1}}{\@tprd{#1}}{\@tprd{#1}}{\@tprd{#1}}\prd@parens}  {\@ifnextchar\sm    {\mathchoice{\@dprd{#1}}{\@tprd{#1}}{\@tprd{#1}}{\@tprd{#1}}\@eatsm}    {\mathchoice{\@dprd{#1}}{\@tprd{#1}}{\@tprd{#1}}{\@tprd{#1}}}}}
\newcommand{\proj}[1]{\ensuremath{\mathsf{pr}_{#1}}\xspace}
\newcommand{\refl}[1]{\ensuremath{\mathsf{refl}_{#1}}\xspace}
\def\sm#1{\@ifnextchar\bgroup{\sm@parens{#1}}{\@ifnextchar\prd{\sm@parens{#1}\@eatprd}{\sm@noparens{#1}}}}
\def\sm@noparens#1{\mathchoice{\@dsm@noparens{#1}}{\@tsm{#1}}{\@tsm{#1}}{\@tsm{#1}}}
\def\sm@parens#1{\@ifnextchar\bgroup  {\mathchoice{\@dsm{#1}}{\@tsm{#1}}{\@tsm{#1}}{\@tsm{#1}}\sm@parens}  {\@ifnextchar\prd    {\mathchoice{\@dsm{#1}}{\@tsm{#1}}{\@tsm{#1}}{\@tsm{#1}}\@eatprd}    {\mathchoice{\@dsm{#1}}{\@tsm{#1}}{\@tsm{#1}}{\@tsm{#1}}}}}
\newcommand{\Sn}{\mathbb{S}}
\newcommand{\snd}{\ensuremath{\proj2}\xspace}
\newcommand{\susp}{\Sigma}
\def\@tprd#1{\mathchoice{{\textstyle\prod_{(#1)}}}{\prod_{(#1)}}{\prod_{(#1)}}{\prod_{(#1)}}}
\newcommand{\transfib}[3]{\ensuremath{\mathsf{transport}^{#1}(#2,#3)\xspace}}
\def\@tsm#1{\mathchoice{{\textstyle\sum_{(#1)}}}{\sum_{(#1)}}{\sum_{(#1)}}{\sum_{(#1)}}}
\newcommand{\vcentcolon}{:\!\!}
\newcounter{mathcount}
\setcounter{mathcount}{1}
\newtheorem{prelem}{Lemma}
\newenvironment{lem}{\begin{prelem}}{\end{prelem}\addtocounter{mathcount}{1}}
\renewcommand{\theprelem}{8.5.\arabic{mathcount}}
\newenvironment{narrowmultline*}{\csname equation*\endcsname}{\csname endequation*\endcsname}
\newtheorem{prethm}{Theorem}
\newenvironment{thm}{\begin{prethm}}{\end{prethm}\addtocounter{mathcount}{1}}
\renewcommand{\theprethm}{8.5.\arabic{mathcount}}
\let\ap\map
\let\apdfunc\mapdepfunc
\let\apfunc\mapfunc
\let\autoref\cref
\makeatother
\def\coloneqq{:=}
\begin{document}
%
\index{Hopf fibration|(}
\indexsee{fibration!Hopf}{Hopf fibration}
We will first construct a structure of H-space on the circle $\Sn^1$, hence by
\autoref{lem:hopf-construction} we will get a fibration over $\Sn^2$ with fiber
$\Sn^1$ and total space $\Sn^1*\Sn^1$. We will then prove that this join is
equivalent to $\Sn^3$.

\begin{lem}\label{lem:hspace-S1}
  There is an H-space structure on the circle $\Sn^1$.
\end{lem}
\begin{proof}
  For the base point of the H-space structure we choose $\base$.
  %
  Now we need to define the multiplication operation
  $\mu:\Sn^1\times\Sn^1\to\Sn^1$.
  We will define the curried form $\widetilde\mu:\Sn^1\to(\Sn^1\to\Sn^1)$ of $\mu$
  by recursion on $\Sn^1$:
  \begin{equation*}
    \widetilde\mu(\base) \defeq\idfunc[\Sn^1],
    \qquad\text{and}\qquad
    \ap{\widetilde\mu}{\lloop} \defid\funext(h).
  \end{equation*}
  where $h:\prd{x:\Sn^1}(x=x)$ is the function defined in \autoref{thm:S1-autohtpy},
  which has the property that $h(\base) \defeq\lloop$.

  Now we just have to prove that $\mu(x,\base)=\mu(\base,x)=x$ for every
  $x:\Sn^1$.
  %
  By definition, if $x:\Sn^1$ we have
  $\mu(\base,x)=\widetilde\mu(\base)(x)=\idfunc[\Sn^1](x)=x$. For the equality
  $\mu(x,\base)=x$ we do it by induction on $x:\Sn^1$:
  \begin{itemize}
  \item If $x$ is $\base$ then $\mu(\base,\base)=\base$ by definition, so we
    have $\refl\base:\mu(\base,\base)=\base$.
  \item When $x$ varies along $\lloop$, we need to prove that
    \[\refl\base\ct\apfunc{\lam{x}x}({\lloop})=
    \apfunc{\lam{x}\mu(x,\base)}({\lloop})\ct\refl\base.\]
    The left-hand side is equal to $\lloop$, and for the right-hand side we have:
    \begin{align*}
      \apfunc{\lam{x}\mu(x,\base)}({\lloop})\ct\refl\base &=
      \apfunc{\lam{x}(\widetilde\mu(x))(\base)}({\lloop})\\
      &=\happly(\apfunc{\lam{x}(\widetilde\mu(x))}({\lloop}),\base)\\
      &=\happly(\funext(h),\base)\\
      &=h(\base)\\
      &=\lloop. \qedhere
    \end{align*}
  \end{itemize}
\end{proof}

Now recall from \autoref{sec:colimits} that the \emph{join} $A*B$ of types $A$ and $B$ is the pushout of the diagram
\index{join!of types}%
\[A \xleftarrow{\proj1}A\times B \xrightarrow{\proj2} B. \]

\begin{lem}
  \index{associativity!of join}%
  The operation of join is associative: if $A$, $B$ and $C$ are three types
  then we have an equivalence $\eqv{(A*B)*C}{A*(B*C)}$.
\end{lem}

\begin{proof}
  We define a map $f:(A*B)*C\to A*(B*C)$ by induction. We first need to define
  $f\circ\inl:A*B\to A*(B*C)$ which will be done by induction, then
  $f\circ\inr:C\to A*(B*C)$, and then $\apfunc{f}\circ\glue:\prd{t:(A*B)\times
    C}f(\inl(\fst(t)))=f(\inr(\snd(t)))$ which will be done by induction on the
  first component of $t$:
  \begin{align*}
    (f\circ\inl)(\inl(a))) &\defeq \inl(a), \\
    (f\circ\inl)(\inr(b))) &\defeq \inr(\inl(b)), \\
    \apfunc{f\circ\inl}(\glue(a,b)) &\defid \glue(a,\inl(b)), \\
    f(\inr(c)) &\defeq \inr(\inr(c)),\\
    \apfunc{f}(\glue(\inl(a),c)) &\defid\glue(a,\inr(c)),\\
    \apfunc{f}(\glue(\inr(b),c)) &\defid\apfunc{\inr}(\glue(b,c)),\\
    \apdfunc{\lam{x}\apfunc{f}(\glue(x,c))}(\glue(a,b)) &\defid
    ``\apdfunc{\lam{x}\glue(a,x)}(\glue(b,c))''.
  \end{align*}
  %
  For the last equation, note that the right-hand side is of type
  \[\transfib{\lam{x}\inl(a)=\inr(x)}{\glue(b,c)}{\glue(a,\inl(b))}=
  \glue(a,\inr(c))\]
  whereas it is supposed to be of type
  %
  \begin{narrowmultline*}
    \transfib{\lam{x}f(\inl(x))=f(\inr(c))}{\glue(a,b)}{\apfunc{f}(\glue(\inl(a),c))}
    = \narrowbreak
    \apfunc{f}(\glue(\inr(b),c)).
  \end{narrowmultline*}
  %
  But by the previous clauses in the definition, both of these types are equivalent to the following type:
  \[\glue(a,\inr(c))=\glue(a,\inl(b))\ct\apfunc\inr(\glue(b,c)),\]
  and so we can coerce by an equivalence to obtain the necessary element.
  %
  Similarly, we can define a map $g:A*(B*C)\to(A*B)*C$, and checking that $f$ and
  $g$ are inverse to each other is a long and tedious but essentially
  straightforward computation.
\end{proof}

A more conceptual proof sketch is as follows.

\begin{proof}
  Let us consider the following diagram where the maps are the obvious
  projections:
  \[\xymatrix{
    A & A\times C \ar[l] \ar[r] & A\times C\\
    A\times B \ar[u] \ar[d] & A\times B\times C \ar[l]\ar[u]\ar[r]\ar[d] &
    A\times C \ar[u] \ar[d] \\
    B & B\times C \ar[l] \ar[r] & C}\]
  %
  Taking the colimit of the columns gives the following
  diagram, whose colimit is $(A*B)*C$:
  \[\xymatrix{A*B & (A*B)\times C \ar[l]\ar[r] & C}\]
  %
  On the other hand, taking the colimit of the lines gives a diagram whose
  colimit is $A*(B*C)$.

  Hence using a Fubini-like theorem for colimits (that we haven’t proved) we
  have an equivalence $\eqv{(A*B)*C}{A*(B*C)}$. The proof of this Fubini theorem
  \index{Fubini theorem for colimits}
  for colimits still requires the long and tedious computation, though.
\end{proof}

\begin{lem}
  For any type $A$, there is an equivalence $\eqv{\susp A}{\bool*A}$.
\end{lem}

\begin{proof}
  It is easy to define the two maps back and forth and to prove that they are
  inverse to each other. The details are left as an exercise to the reader.
\end{proof}

We can now construct the Hopf fibration:

\begin{thm}
  There is a fibration over $\Sn^2$ of fiber $\Sn^1$ and total space $\Sn^3$.
  \index{total!space}%
\end{thm}
\begin{proof}
  We proved that $\Sn^1$ has a structure of H-space (cf \autoref{lem:hspace-S1})
  hence by \autoref{lem:hopf-construction} there is a fibration over $\Sn^2$ of
  fiber $\Sn^1$ and total space $\Sn^1*\Sn^1$. But by the two previous results
  and \autoref{thm:suspbool} we have:
  \begin{equation*}
    \Sn^1*\Sn^1 = (\susp\bool)*\Sn^1
    =(\bool*\bool)*\Sn^1
    =\bool*(\bool*\Sn^1)
    =\susp(\susp\Sn^1)
    =\Sn^3. \qedhere
  \end{equation*}
\end{proof}
\index{Hopf fibration|)}


\end{document}
