\documentclass[12pt]{article}
\usepackage{pmmeta}
\pmcanonicalname{MoreExamplesOfReversePolishNotation}
\pmcreated{2013-03-22 16:10:19}
\pmmodified{2013-03-22 16:10:19}
\pmowner{Mravinci}{12996}
\pmmodifier{Mravinci}{12996}
\pmtitle{more examples of reverse Polish notation}
\pmrecord{5}{38257}
\pmprivacy{1}
\pmauthor{Mravinci}{12996}
\pmtype{Example}
\pmcomment{trigger rebuild}
\pmclassification{msc}{03B70}
\pmclassification{msc}{68N17}

\endmetadata

% this is the default PlanetMath preamble.  as your knowledge
% of TeX increases, you will probably want to edit this, but
% it should be fine as is for beginners.

% almost certainly you want these
\usepackage{amssymb}
\usepackage{amsmath}
\usepackage{amsfonts}

% used for TeXing text within eps files
%\usepackage{psfrag}
% need this for including graphics (\includegraphics)
%\usepackage{graphicx}
% for neatly defining theorems and propositions
%\usepackage{amsthm}
% making logically defined graphics
%%%\usepackage{xypic}

% there are many more packages, add them here as you need them

% define commands here

\begin{document}
The following examples, presented first in standard infix notation, converted to reverse Polish notation by using the shunting yard algorithm, all use the same four operands but combined with different operators and parentheses. Operators are assumed to be binary.

$(1 + 2) \times (3 + 4)$ in standard infix notation becomes $1 \quad 2 + 3 \quad 4 + \quad \times$ in reverse Polish notation. Both expressions should evaluate to 21, each in the appropriate calculator.

$1 + 2 \times 3 + 4$ standard becomes $1 \quad 2 \quad 3 \times 4 +$ RPN. Both evaluate to 11.


$1 + 2 \times (3 + 4)$ turns to $1 \quad 2 \quad 3 \quad 4 + \quad \times \quad +$. Evaluate to 15.

$(1 + 2) \times 3 + 4$ is $1 \quad 2 + 3 \times 4 +$. Evaluate to 13.
%%%%%
%%%%%
\end{document}
