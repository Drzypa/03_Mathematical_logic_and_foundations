\documentclass[12pt]{article}
\usepackage{pmmeta}
\pmcanonicalname{Partition}
\pmcreated{2013-03-22 11:49:05}
\pmmodified{2013-03-22 11:49:05}
\pmowner{Wkbj79}{1863}
\pmmodifier{Wkbj79}{1863}
\pmtitle{partition}
\pmrecord{11}{30362}
\pmprivacy{1}
\pmauthor{Wkbj79}{1863}
\pmtype{Definition}
\pmcomment{trigger rebuild}
\pmclassification{msc}{03-00}
\pmclassification{msc}{45D05}
\pmsynonym{set partition}{Partition}
\pmrelated{EquivalenceRelation}
\pmrelated{EquivalenceClass}
\pmrelated{BeattysTheorem}
\pmrelated{Coloring}

\usepackage{amssymb}
\usepackage{amsmath}
\usepackage{amsfonts}
\usepackage{graphicx}
%%%%\usepackage{xypic}

\begin{document}
A \emph{partition} $P$ of a set $S$ is a collection of pairwise disjoint nonempty sets such that $\cup P = S$.

Any partition $P$ of a set $S$ introduces an equivalence relation on $S$, where each $A \in P$ is an equivalence class.  Similarly, given an equivalence relation on $S$, the collection of distinct equivalence classes is a partition of $S$.
%%%%%
%%%%%
%%%%%
%%%%%
\end{document}
