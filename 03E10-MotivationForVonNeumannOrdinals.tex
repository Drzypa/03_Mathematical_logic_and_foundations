\documentclass[12pt]{article}
\usepackage{pmmeta}
\pmcanonicalname{MotivationForVonNeumannOrdinals}
\pmcreated{2013-03-22 16:50:40}
\pmmodified{2013-03-22 16:50:40}
\pmowner{yark}{2760}
\pmmodifier{yark}{2760}
\pmtitle{motivation for von Neumann ordinals}
\pmrecord{13}{39089}
\pmprivacy{1}
\pmauthor{yark}{2760}
\pmtype{Derivation}
\pmcomment{trigger rebuild}
\pmclassification{msc}{03E10}

\endmetadata

% this is the default PlanetMath preamble.  as your knowledge
% of TeX increases, you will probably want to edit this, but
% it should be fine as is for beginners.

% almost certainly you want these
\usepackage{amssymb}
\usepackage{amsmath}
\usepackage{amsfonts}

% used for TeXing text within eps files
%\usepackage{psfrag}
% need this for including graphics (\includegraphics)
%\usepackage{graphicx}
% for neatly defining theorems and propositions
%\usepackage{amsthm}
% making logically defined graphics
%%%\usepackage{xypic}

% there are many more packages, add them here as you need them

% define commands here

\begin{document}
The idea of the von Neumann ordinal can be traced back to the following
well-known fact: for any natural number $n$, there are exactly $n$ 
natural numbers which are less than $n$.  For instance, the set
\[
\{ 0, 1, 2, 3, 4 \}
\]
has $5$ elements, the set
\[
\{0, 1, 2, 3, 4, 5, 6 \}
\]
has $7$ elements, etc.

To obtain von Neumann ordinals, we turn this idea around.  Instead of
taking it for granted that numbers exist (and have certain properties),
we want to start with the more primitive notion of set and define
numbers (and derive their properties).  The way to define a number is
as a set of objects which have that number of elements.  For instance,
consider counting on fingers --- in that case, a set of fingers stands
for a number.  We will apply the same idea here in a more sophisticated
form --- the counters we will use are not going to be fingers or beads 
on an abacus, but abstract elements of an abstract set.

To do this, we turn the observation made earlier around and
\emph{define} a natural number to be the set of all natural numbers
less than it.  At first sight, this definition appears circular, but
upon closer examination, we see that it is legitimate.  The reason
is that to define a particular number, we only need to make use
of the numbers smaller than it as counters, so cn use our definition
repeatedly to express numbers as sets.

To begin, we notice that, since there are no natural numbers smaller
than zero, we represent zero by the empty set.  Next, since the only
number smaller than $1$, is zero, which corresponds to the empty set,
we see that $1$ corresponds to the set whose only element is the
empty set, i.e. $1 = \{ 0 \} = \{ \emptyset \}$.  Then we can go on to
express all other numbers in terms of the empty set in a manner which
may be explained with a typical example:
\begin{align*}
4 &= \{ 0, 1, 2, 3 \} \\
&= \{ \emptyset, \{ 0 \}, \{ 0, 1 \}, \{ 0, 1, 2 \} \} \\
& = \{ \emptyset,
    \{ \emptyset \},
    \{ \emptyset, \{ \emptyset \} \},
    \{ 0, 1, \{ 0, 1 \} \} \\
&= \{ \emptyset, 
   \{ \emptyset \},
   \{ \emptyset, \{ \emptyset \} \},
   \{ \emptyset, \{ \emptyset \} , \{ \emptyset, \{ \emptyset \} \} \}.
\end{align*}
As we already see in this example, this representation of integers 
in terms of the empty set is extremely clumsy.  While it is of little
use in practical application (even tally marks or Roman numerals are
more concise) it is of use theoretically because it is easy to 
define the basic operations on numbers in terms of set-theoretical
operations.

As an example of such a definition, we note that the ordering relation ---
given two numbers $m$ and $n$, we have $m < n$ exactly when $m \subset n$
as sets.  As it turns out, our numbers are totally ordered, in fact
well-ordered under this relation, so our numbers are ordinal numbers,
hence the name ``von Neumann ordinals''.

Another important example is the successor function.  Thinking for a
minute about how we define a number as the set of numbers smaller
than itself, we see that the next number is gotten by adding the 
set denoting the previous element to itself as an element, in symbols,
$n + 1 = n \cup \{ n \}$.  For example, we have
\[
4 + 1 =
4 \cup \{ 4 \} =
\{ 0, 1, 2, 3 \} \cup \{ 4 \} =
\{ 0, 1, 2, 3, 4 \} =
5.
\]
Using this definition, one may do things like derive the Peano
axioms from the axioms of set theory.  From a foundational point
of view, that derivation is important because it shows that it is
not necessary to separately postulate natural numbers, but that
they arise naturally from set theory.

Finally, this definition applies equally well to transfinite numbers.
For instance, consider the first transfinite ordinal $\omega$.  By 
definition, this is the ordinal number of the ordered set of natural
numbers.  In our scheme, we simply define $\omega$ to be the set of
all natural numbers.  Furthermore, given \emph{any} well-ordered set,
one can show by transfinite induction that it is isomorphic to some
von Neumann ordinal, so all ordinal numbers are represented.
%%%%%
%%%%%
\end{document}
