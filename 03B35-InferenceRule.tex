\documentclass[12pt]{article}
\usepackage{pmmeta}
\pmcanonicalname{InferenceRule}
\pmcreated{2013-03-22 16:50:51}
\pmmodified{2013-03-22 16:50:51}
\pmowner{rspuzio}{6075}
\pmmodifier{rspuzio}{6075}
\pmtitle{inference rule}
\pmrecord{8}{39093}
\pmprivacy{1}
\pmauthor{rspuzio}{6075}
\pmtype{Definition}
\pmcomment{trigger rebuild}
\pmclassification{msc}{03B35}
\pmclassification{msc}{03B22}
\pmclassification{msc}{03B05}
\pmsynonym{rule of inference}{InferenceRule}
\pmrelated{ModusPonens}
\pmrelated{LogicalAxiom}
\pmrelated{DeductiveSystem}

% this is the default PlanetMath preamble.  as your knowledge
% of TeX increases, you will probably want to edit this, but
% it should be fine as is for beginners.

% almost certainly you want these
\usepackage{amssymb}
\usepackage{amsmath}
\usepackage{amsfonts}

% used for TeXing text within eps files
%\usepackage{psfrag}
% need this for including graphics (\includegraphics)
%\usepackage{graphicx}
% for neatly defining theorems and propositions
%\usepackage{amsthm}
% making logically defined graphics
%%%\usepackage{xypic}

% there are many more packages, add them here as you need them

% define commands here

\begin{document}
In logic, an \emph{inference rule} is a rule whereby one may correctly
draw a conclusion from one or more premises.  For example, the law of
the contrapositive allows one to conclude a statement of the form
\[ \neg Q \Rightarrow \neg P \]
from a premise of the form
\[ P \Rightarrow Q. \]
Here, `$P$' and `$Q$' are propositional variables, which can stand for
arbitrary propositions.  A popular way to indicate applications of rules
of inference is to list the premises above a line and write the
conclusions below the line.  For instance, we might indicate the law
of the contrapositive thus:
\[
{P \Rightarrow Q \over \neg Q \Rightarrow \neg P}
\]


A typical application of the law of
contrapositive would be to conclude "If my clothes are dry, then it is not
raining", from "If it rains, then my clothes will be wet." which could be
expressed as follows using the notation described above:
\[
{\hbox{If it rains, then my clothes will be wet.} \over
\hbox{If my clothes are dry, then it is not raining.}}
\]
(In this 
instance, $P$ is ``It is raining'' and $Q$ is ``My clothes are dry''.

An important feature of rules of inference is that they are purely formal, 
which means that all that matters is the form of the expression;
meaning is not a consideration in applying a rule of inference.
Thus, the following are equally valid applications of the rule of
the contrapositive:
\[
{\hbox{If the jabberwocky is mimsy, then the toves blithe.} \over
\hbox{If the toves are not blithing, then the jabberwocky is not mimsy.}}
\]
\medskip
\[
{\hbox{If my cat has a tail, then my cat is a dog.} \over
\hbox{If my cat is not a dog, then my cat does not have a tail.}}
\]

In the first example, the statements are nonsense and in the second
example, the statements are false, but this doesn't matter --- both
examples constitute valid apllications of the rule of the contrapositive.
Of course, in order to draw valid conclusions, we need to start with
valid premises, but the point of these examples is clarify the
distinction between valid statements and valid applications of
rules of inference.
%%%%%
%%%%%
\end{document}
