\documentclass[12pt]{article}
\usepackage{pmmeta}
\pmcanonicalname{ProofOfTukeysLemma}
\pmcreated{2013-03-22 13:54:58}
\pmmodified{2013-03-22 13:54:58}
\pmowner{Koro}{127}
\pmmodifier{Koro}{127}
\pmtitle{proof of Tukey's lemma}
\pmrecord{4}{34672}
\pmprivacy{1}
\pmauthor{Koro}{127}
\pmtype{Proof}
\pmcomment{trigger rebuild}
\pmclassification{msc}{03E25}

\usepackage{amssymb}
\usepackage{amsmath}
\usepackage{amsfonts}
\begin{document}
Let $S$ be a set and $F$ a set of subsets of $S$ such that $F$ is
of finite character. By Zorn's lemma, it is enough to show that
$F$ is inductive. For that, it will be enough to show that if
$(F_i)_{i\in I}$ is a family of elements of $F$ which is totally ordered
by inclusion, then the union $U$ of the $F_i$ is an element of $F$
as well (since $U$ is an upper bound on the family $(F_i)$).
So, let $K$ be a finite subset of $U$. Each element of
$U$ is in $F_i$ for some $i\in I$. Since $K$ is finite and
the $F_i$ are totally ordered by inclusion, there is some $j\in I$
such that all elements of $K$ are in $F_j$. That is, $K\subset F_j$.
Since $F$ is of finite character, we get $K\in F$, QED.
%%%%%
%%%%%
\end{document}
