\documentclass[12pt]{article}
\usepackage{pmmeta}
\pmcanonicalname{Infinite}
\pmcreated{2013-03-22 11:59:03}
\pmmodified{2013-03-22 11:59:03}
\pmowner{yark}{2760}
\pmmodifier{yark}{2760}
\pmtitle{infinite}
\pmrecord{18}{30881}
\pmprivacy{1}
\pmauthor{yark}{2760}
\pmtype{Definition}
\pmcomment{trigger rebuild}
\pmclassification{msc}{03E99}
\pmsynonym{infinite set}{Infinite}
\pmsynonym{infinite subset}{Infinite}
%\pmkeywords{infinite}
\pmrelated{Finite}
\pmrelated{AlephNumbers}

\endmetadata

\usepackage{amssymb}
\usepackage{amsmath}
\usepackage{amsfonts}
\begin{document}
\PMlinkescapeword{between}
\PMlinkescapeword{finite}
\PMlinkescapeword{equivalent}

A set $S$ is \emph{infinite} if it is not \PMlinkname{finite}{Finite}; that is, there is no $n \in \mathbb{N}$ for which there is a bijection between $n$ and $S$.  

Assuming the \PMlinkname{Axiom of Choice}{AxiomOfChoice} (or the Axiom of Countable Choice), this definition of infinite sets is equivalent to that of \PMlinkname{Dedekind-infinite sets}{DedekindInfinite}.

Some examples of finite sets:

\begin{itemize}
\item The empty set: $\{\}$.
\item $\{0, 1\}$
\item $\{1, 2, 3, 4 , 5\}$
\item $\{1,1.5, e, \pi\}$
\end{itemize}

Some examples of infinite sets:

\begin{itemize}
\item $\{1, 2, 3, 4, \ldots\}$.
\item The primes: $\{2, 3, 5, 7, 11, \ldots\}$.
\item The rational numbers: $\mathbb{Q}$.
\item An interval of the reals: $(0, 1)$.
\end{itemize}

The first three examples are countable, but the last is uncountable.
%%%%%
%%%%%
%%%%%
\end{document}
