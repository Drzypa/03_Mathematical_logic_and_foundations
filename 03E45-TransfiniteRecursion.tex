\documentclass[12pt]{article}
\usepackage{pmmeta}
\pmcanonicalname{TransfiniteRecursion}
\pmcreated{2013-03-22 17:53:51}
\pmmodified{2013-03-22 17:53:51}
\pmowner{CWoo}{3771}
\pmmodifier{CWoo}{3771}
\pmtitle{transfinite recursion}
\pmrecord{12}{40387}
\pmprivacy{1}
\pmauthor{CWoo}{3771}
\pmtype{Theorem}
\pmcomment{trigger rebuild}
\pmclassification{msc}{03E45}
\pmclassification{msc}{03E10}
\pmrelated{WellFoundedRecursion}
\pmrelated{TransfiniteInduction}

\usepackage{amssymb,amscd}
\usepackage{amsmath}
\usepackage{amsfonts}
\usepackage{mathrsfs}

% used for TeXing text within eps files
%\usepackage{psfrag}
% need this for including graphics (\includegraphics)
%\usepackage{graphicx}
% for neatly defining theorems and propositions
\usepackage{amsthm}
% making logically defined graphics
%%\usepackage{xypic}
\usepackage{pst-plot}

% define commands here
\newcommand*{\abs}[1]{\left\lvert #1\right\rvert}
\newtheorem{prop}{Proposition}
\newtheorem{thm}{Theorem}
\newtheorem{ex}{Example}
\newcommand{\real}{\mathbb{R}}
\newcommand{\pdiff}[2]{\frac{\partial #1}{\partial #2}}
\newcommand{\mpdiff}[3]{\frac{\partial^#1 #2}{\partial #3^#1}}
\begin{document}
Transfinite recursion, roughly speaking, is a statement about the ability to define a function recursively using transfinite induction.  In its most general and intuitive form, it says

\begin{thm}  Let $G$ be a (class) function on $V$, the class of all sets.  Then there exists a unique (class) function $F$ on $\mathbf{On}$, the class of ordinals, such that $$F(\alpha)=G(F|\alpha)$$ where $F|\alpha$ is the function whose domain is $\operatorname{seg}(\alpha):=\lbrace \beta\mid \beta<\alpha\rbrace$ and whose values coincide with $F$ on every $\beta\in \operatorname{seg}(\alpha)$.  In other words, $F|\alpha$ is the restriction of $F$ to $\operatorname{\alpha}$.
\end{thm}

Notice that the theorem above is not provable in ZF set theory, as $G$ and $F$ are both classes, not sets.  In order to prove this statement, one way of getting around this difficulty is to convert both $G$ and $F$ into formulas, and modify the statement, as follows:

Let $\varphi(x,y)$ be a formula such that $$\forall x\exists y \forall z (\varphi(x,z)\leftrightarrow z=y).$$  Think of $G=\lbrace (x,y)\mid \varphi(x,y)\rbrace$.  Then there is a unique formula $\psi(\alpha,z)$ (think of $F$ as $\lbrace (\alpha,z)\mid \psi(\alpha,z)\rbrace$) such that the following two sentences are derivable using ZF axioms:
\begin{enumerate}
\item $\forall x\exists y \forall z \big(\mathbf{On}(x)\wedge (\psi(x,z)\leftrightarrow z=y) \big),$ where $\mathbf{On}(x)$ means ``$x$ is an ordinal'',
\item $\forall x\forall y \Big(\mathbf{On}(x)\wedge \big(\psi(x,y)\leftrightarrow \exists f (A\wedge B \wedge C \wedge D) \big) \Big)$, where 
\begin{itemize}
\item $A$ is the formula ``$f$ is a function'', 
\item $B$ is the formula ``$\operatorname{dom}(f)=x$'', 
\item $C$ is the formula $\forall z \big(z\in x \wedge \varphi(f|z,f(z))\big)$, and 
\item $D$ is the formula $\varphi(f,y)$.
\end{itemize}
\end{enumerate}

A stronger form of the transfinite recursion theorem says:
\begin{thm}  Let $\varphi(x,y)$ be any formula (in the language of set theory).  Then the following is a theorem: assume that $\varphi$ satisfies property that, for every $x$, there is a unique $y$ such that $\varphi(x,y)$.  If $A$ be a well-ordered set (well-ordered by $\le$), then there is a unique function $f$ defined on $A$ such that $$\varphi(f|\operatorname{seg}(s),f(s))$$ for every $s\in A$.  Here, $\operatorname{seg}(s):=\lbrace t\in A\mid t<s\rbrace$, the initial segment of $s$ in $A$.
\end{thm}

The above theorem is actually a collection of theorems, or known as a theorem schema, where each theorem corresponds to a formula.  The other difference between this and the previous theorem is that this theorem is provable in ZF, because the domain of the function $f$ is now a set.
%%%%%
%%%%%
\end{document}
