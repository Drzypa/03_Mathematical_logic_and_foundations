\documentclass[12pt]{article}
\usepackage{pmmeta}
\pmcanonicalname{AFormalTypeTheory}
\pmcreated{2013-11-09 2:39:26}
\pmmodified{2013-11-09 2:39:26}
\pmowner{PMBookProject}{1000683}
\pmmodifier{PMBookProject}{1000683}
\pmtitle{A. Formal type theory}
\pmrecord{2}{87882}
\pmprivacy{1}
\pmauthor{PMBookProject}{1000683}
\pmtype{Feature}
\pmclassification{msc}{03B15}

\endmetadata

\usepackage{xspace}
\usepackage{amssyb}
\usepackage{amsmath}
\usepackage{amsfonts}
\usepackage{amsthm}
\newcommand{\Agda}{\textsc{Agda}\xspace}
\newcommand{\Coq}{\textsc{Coq}\xspace}
\newcommand{\emptyt}{\ensuremath{\mathbf{0}}\xspace}
\newcommand{\eqv}[2]{\ensuremath{#1 \simeq #2}\xspace}
\newcommand{\jdeq}{\equiv}      
\newcommand{\Sn}{\mathbb{S}}
\newcommand{\unit}{\ensuremath{\mathbf{1}}\xspace}
\let\autoref\cref

\begin{document}


\index{formal!type theory|(}%
\index{type theory!formal|(}%
\index{rules of type theory|(}%

Just as one can develop mathematics in set theory without explicitly using the axioms of Zermelo--Fraenkel set theory, 
in this book we have developed mathematics in univalent foundations without explicitly referring to a formal
system of homotopy type theory. Nevertheless, it is important to \emph{have} a
precise description of homotopy type theory as a formal system in order to, for example,
%
\begin{itemize}
\item state and prove its metatheoretic properties, including logical
consistency,
\item construct models, e.g.\  in simplicial sets, model categories, higher toposes,
etc., and
\item implement it in proof assistants like \Coq or \Agda.
  \index{proof!assistant}
\end{itemize}
%
Even the logical consistency\index{consistency} of homotopy type theory, namely that in the empty context there is no term $a:\emptyt$, is not obvious: if we had erroneously
chosen a definition of equivalence for which $\eqv{\emptyt}{\unit}$, then
univalence would imply that $\emptyt$ has an element, since $\unit$ does.
Nor is it obvious that, for example, our definition of $\Sn^1$ as a higher
inductive type yields a type which behaves like the ordinary circle.

There are two aspects of type theory which we must pin down before addressing
such questions. Recall from the Introduction that type theory
comprises a set of rules specifying when the judgments $a:A$ and $a\jdeq a':A$
hold---for example, products are characterized by the rule that whenever $a:A$
and $b:B$, $(a,b):A\times B$. To make this precise, we must first define
precisely the syntax of terms---the objects $a,a',A,\dots$ which these judgments
relate; then, we must define precisely the judgments and their rules of
inference---the manner in which judgments can be derived from other judgments.

In this appendix, we present two formulations of Martin-L\"{o}f type
theory, and of the extensions that constitute homotopy type theory.
The first presentation (\autoref{sec:syntax-informally}) describes the syntax of
terms and the forms of judgments as an extension of the untyped
$\lambda$-calculus, while leaving the rules of inference informal.
The second (\autoref{sec:syntax-more-formally}) defines the terms, judgments,
and rules of inference inductively in the style of natural deduction, as
is customary in much type-theoretic literature.


\end{document}
