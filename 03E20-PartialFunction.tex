\documentclass[12pt]{article}
\usepackage{pmmeta}
\pmcanonicalname{PartialFunction}
\pmcreated{2013-03-22 12:58:15}
\pmmodified{2013-03-22 12:58:15}
\pmowner{Henry}{455}
\pmmodifier{Henry}{455}
\pmtitle{partial function}
\pmrecord{11}{33341}
\pmprivacy{1}
\pmauthor{Henry}{455}
\pmtype{Definition}
\pmcomment{trigger rebuild}
\pmclassification{msc}{03E20}
\pmdefines{total function}
\pmdefines{domain of definition}

\endmetadata

% this is the default PlanetMath preamble.  as your knowledge
% of TeX increases, you will probably want to edit this, but
% it should be fine as is for beginners.

% almost certainly you want these
\usepackage{amssymb}
\usepackage{amsmath}
\usepackage{amsfonts}

% used for TeXing text within eps files
%\usepackage{psfrag}
% need this for including graphics (\includegraphics)
%\usepackage{graphicx}
% for neatly defining theorems and propositions
%\usepackage{amsthm}
% making logically defined graphics
%%%\usepackage{xypic}

% there are many more packages, add them here as you need them

% define commands here
%\PMlinkescapeword{theory}
\begin{document}
A function $f:A\rightarrow B$ is sometimes called a \emph{total function}, to signify that $f(a)$ is defined for every $a\in A$.  If $C$ is any set such that $C\supseteq A$ then $f$ is also a \emph{partial function} from $C$ to $B$.

Clearly if $f$ is a function from $A$ to $B$ then it is a partial function from $A$ to $B$, but a partial function need not be defined for every element of its domain.  The set of elements of $A$ for which $f$ is defined is sometimes called the \emph{domain of definition}.
%%%%%
%%%%%
\end{document}
