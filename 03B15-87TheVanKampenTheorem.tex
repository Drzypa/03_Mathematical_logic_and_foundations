\documentclass[12pt]{article}
\usepackage{pmmeta}
\pmcanonicalname{87TheVanKampenTheorem}
\pmcreated{2013-11-06 15:22:41}
\pmmodified{2013-11-06 15:22:41}
\pmowner{PMBookProject}{1000683}
\pmmodifier{rspuzio}{6075}
\pmtitle{8.7 The van Kampen theorem}
\pmrecord{1}{}
\pmprivacy{1}
\pmauthor{PMBookProject}{6075}
\pmtype{Feature}
\pmclassification{msc}{03B15}

\usepackage{xspace}
\usepackage{amssyb}
\usepackage{amsmath}
\usepackage{amsfonts}
\usepackage{amsthm}
\newcommand{\blank}{\mathord{\hspace{1pt}\text{--}\hspace{1pt}}}
\newcommand{\ct}{  \mathchoice{\mathbin{\raisebox{0.5ex}{$\displaystyle\centerdot$}}}             {\mathbin{\raisebox{0.5ex}{$\centerdot$}}}             {\mathbin{\raisebox{0.25ex}{$\scriptstyle\,\centerdot\,$}}}             {\mathbin{\raisebox{0.1ex}{$\scriptscriptstyle\,\centerdot\,$}}}}
\newcommand{\defeq}{\vcentcolon\equiv}  
\newcommand{\mapfunc}[1]{\ensuremath{\mathsf{ap}_{#1}}\xspace} 
\newcommand{\opp}[1]{\mathord{{#1}^{-1}}}
\newcommand{\refl}[1]{\ensuremath{\mathsf{refl}_{#1}}\xspace}
\newcommand{\Sn}{\mathbb{S}}
\newcommand{\trunc}[2]{\mathopen{}\left\Vert #2\right\Vert_{#1}\mathclose{}}
\newcommand{\UU}{\ensuremath{\mathcal{U}}\xspace}
\newcommand{\vcentcolon}{:\!\!}
\let\apfunc\mapfunc
\let\autoref\cref
\let\type\UU

\begin{document}

\index{van Kampen theorem|(}%
\index{theorem!van Kampen|(}%

\index{fundamental!group}%
The van Kampen theorem calculates the fundamental group $\pi_1$ of a (homotopy) pushout of spaces.
It is traditionally stated for a topological space $X$ which is the union of two open subspaces $U$ and $V$, but in homotopy-theoretic terms this is just a convenient way of ensuring that $X$ is the pushout of $U$ and $V$ over their intersection.
Thus, we will prove a version of the van Kampen theorem for arbitrary pushouts.

In this section we will describe a proof of the van Kampen theorem which uses the same encode-decode method that we used for $\pi_1(\Sn^1)$ in \autoref{sec:pi1-s1-intro}.
There is also a more homotopy-theoretic approach; see \autoref{ex:rezk-vankampen}.

We need a more refined version of the encode-decode method.
In \autoref{sec:pi1-s1-intro} (as well as in \autoref{sec:compute-coprod},\autoref{sec:compute-nat}) we used it to characterize the path space of a (higher) inductive type $W$ --- deriving as a consequence a characterization of the loop space $\Omega(W)$, and thereby also of its 0-truncation $\pi_1(W)$.
In the van Kampen theorem, our goal is only to characterize the fundamental group $\pi_1(W)$, and we do not have any explicit description of the loop spaces or the path spaces to use.

It turns out that we can use the same technique directly for a truncated version of the path fibration, thereby characterizing not only the fundamental \emph{group} $\pi_1(W)$, but also the whole fundamental \emph{groupoid}.
\index{fundamental!pregroupoid}%
Spe\-cif\-ical\-ly, for a type $X$, write $\Pi_1 X: X\to X\to \type$ for the $0$-truncation of its identity type, i.e.\ $\Pi_1 X(x,y) \defeq \trunc0{x=y}$.
Note that we have induced groupoid operations
\begin{align*}
  (\blank\ct\blank) &\;:\; \Pi_1X(x,y) \to \Pi_1X(y,z) \to \Pi_1X(x,z)\\
  \opp{(\blank)} &\;:\; \Pi_1X(x,y) \to \Pi_1X(y,x)\\
  \refl{x} &\;:\; \Pi_1X(x,x)\\
  \apfunc{f} &\;:\; \Pi_1X(x,y) \to \Pi_1Y(fx,fy)
\end{align*}
for which we use the same notation as the corresponding operations on paths.


\end{document}
