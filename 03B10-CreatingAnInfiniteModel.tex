\documentclass[12pt]{article}
\usepackage{pmmeta}
\pmcanonicalname{CreatingAnInfiniteModel}
\pmcreated{2013-03-22 12:44:29}
\pmmodified{2013-03-22 12:44:29}
\pmowner{CWoo}{3771}
\pmmodifier{CWoo}{3771}
\pmtitle{creating an infinite model}
\pmrecord{9}{33043}
\pmprivacy{1}
\pmauthor{CWoo}{3771}
\pmtype{Example}
\pmcomment{trigger rebuild}
\pmclassification{msc}{03B10}
\pmclassification{msc}{03C07}
%\pmkeywords{Skolem-L\"owenheim}
\pmrelated{CompactnessTheoremForFirstOrderLogic}
\pmrelated{GettingModelsIModelsConstructedFromConstants}

\endmetadata

% this is the default PlanetMath preamble.  as your knowledge
% of TeX increases, you will probably want to edit this, but
% it should be fine as is for beginners.

% almost certainly you want these
\usepackage{amssymb}
\usepackage{amsmath}
\usepackage{amsfonts}
\usepackage{amsthm}

% used for TeXing text within eps files
%\usepackage{psfrag}
% need this for including graphics (\includegraphics)
%\usepackage{graphicx}
% for neatly defining theorems and propositions
%\usepackage{amsthm}
% making logically defined graphics
%%%\usepackage{xypic}

% there are many more packages, add them here as you need them

% define commands here

\newcommand{\Prob}[2]{\mathbb{P}_{#1}\left\{#2\right\}}
\newcommand{\Expect}{\mathbb{E}}
\newcommand{\norm}[1]{\left\|#1\right\|}

% Some sets
\newcommand{\Nats}{\mathbb{N}}
\newcommand{\Ints}{\mathbb{Z}}
\newcommand{\Reals}{\mathbb{R}}
\newcommand{\Complex}{\mathbb{C}}



%%%%%% END OF SAVED PREAMBLE %%%%%%
\newcommand{\T}{\textbf{T}}
\renewcommand{\L}{\textbf{L}}
\newcommand{\M}{\mathcal{M}}
\begin{document}
From the syntactic compactness theorem for first order logic, we get this nice (and useful) result:

Let $\T$ be a theory of first-order logic.  If $\T$ has finite models of unboundedly large sizes, then $\T$ also has an infinite model.

\begin{proof}
Define the propositions
$$
\Phi_n \equiv \underline{\exists x_1 \cdots \exists x_n . (x_1\ne x_2)\wedge \cdots \wedge(x_1\ne x_n)
\wedge(x_2\ne x_3)\wedge \cdots \wedge(x_{n-1}\ne x_n)}
$$
($\Phi_n$ says ``\textit{there exist (at least) $n$ \emph{different} elements in the world}'').  Note that
$$\cdots \vdash \Phi_n \vdash \cdots \vdash \Phi_2 \vdash \Phi_1.$$
Define a new theory
$$
\T_\infty = \T \cup \left\{\Phi_1, \Phi_2, \ldots \right\}.
$$
For any \emph{finite} subset $\T'\subset \T_\infty$, we claim that $\T'$ is consistent:  Indeed, $\T'$ contains axioms of $\T$, along with finitely many of $\left\{\Phi_n\right\}_{n\ge 1}$.  Let $\Phi_m$ correspond to the largest index appearing in $\T'$.  If $\M_m\models\T$ is a model of $\T$ with at least $m$ elements (and by hypothesis, such as model exists), then $\M_m\models \T\cup\{\Phi_m\}\vdash\T'$.

So every finite subset of $\T_\infty$ is consistent; by the compactness theorem for first-order logic, $\T_\infty$ is consistent, and by G\"odel's completeness theorem for first-order logic it has a model $\M$.  Then $\M\models\T_\infty\vdash\T$, so $\M$ is a model of $\T$ with infinitely many elements ($\M\models \Phi_n$ for any $n$, so $\M$ has at least $\ge n$ elements for all $n$).
\end{proof}
%%%%%
%%%%%
\end{document}
