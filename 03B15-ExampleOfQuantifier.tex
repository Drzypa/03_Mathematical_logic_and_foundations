\documentclass[12pt]{article}
\usepackage{pmmeta}
\pmcanonicalname{ExampleOfQuantifier}
\pmcreated{2013-05-23 19:14:17}
\pmmodified{2013-05-23 19:14:17}
\pmowner{hkkass}{6035}
\pmmodifier{hkkass}{6035}
\pmtitle{example of quantifier}
\pmrecord{19}{40177}
\pmprivacy{1}
\pmauthor{hkkass}{6035}
\pmtype{Example}
\pmcomment{trigger rebuild}
\pmclassification{msc}{03B15}
\pmclassification{msc}{03B10}

\endmetadata

% this is the default PlanetMath preamble.  as your knowledge
% of TeX increases, you will probably want to edit this, but
% it should be fine as is for beginners.

% almost certainly you want these
\usepackage{amssymb}
\usepackage{amsmath}
\usepackage{amsfonts}

% need this for including graphics (\includegraphics)
\usepackage{graphicx}
% for neatly defining theorems and propositions
\usepackage{amsthm}

% making logically defined graphics
%\usepackage{xypic}
% used for TeXing text within eps files
%\usepackage{psfrag}

% there are many more packages, add them here as you need them

% define commands here

\begin{document}
there are some examples and theorems about logical quantifiers in the Word Document below .
you can download it:

http://www.freewebs.com/hkkass

or

http://www.hkkass.blogspot.com/


I include extracts of this Document below:

Definition: a property is something like $x >0$ or $x=0$ in which $x$ is a variable in some set. Such a formula is shown by $p(x)$, $q(x)$ ,etc. if x is fixed then $p(x)$ is a proposition, i.e. it is a true or a false sentence.
 
Example 1: let $p(x)$ be the property $0 < x$ where x is a real number. $p(1)$ is true and $p(0)$ is false.
 
Example 2: a property can have two or more variables. Let $p(x,y)$ be $x=y$. in this case $p(1,1)$ is true but $p(0,1)$ is false because $0$ is not equal to $1$.

Definition: let $p(x)$ be a property on the set $X$, i.e. $p(x)$ is a property and $x$ varies in the set $X$. 
a) The symbol $(\forall x \in X)(p(x))$ means for every $x$ in the set $X$ the proposition $p(x)$ is true.
b) The symbol $(\exists x \in X)(p(x))$ means there is some $x$ in the set $X$ for which the proposition $p(x)$ is true.
If $X=\emptyset$ , i.e. if the set $X$ is empty, $(\forall x \in X)(p(x))$ is defined to be true and $(\exists x \in X)(p(x))$ is defined to be false. 
 
 
Example 1: $(\forall x \in \Bbb R)(x=0\text{ or }x > 0\text{ or }x < 0)$ is a true proposition. 
 
Example 2: $(\exists x \in \Bbb R)(x^2 +1=0)$ is false, because no real number satisfies $x^2+10=0$. 
 
Example 3: $(\forall x \in \Bbb R)(x < y)$ is a property. $y$ varies in $\Bbb R$. As a result $(\forall x \in \Bbb R)(\forall y \in \Bbb R) (x < y)$ is a proposition, i.e. it is a true or a false sentence. In fact 
$(\forall x \in \Bbb R)(\forall y \in \Bbb R)(x < y)$ is false but $(\forall x \in \Bbb R)(\forall y \in (x,\infty)(x < y)$ is true; here $(x,\infty)$ is the interval containing real numbers greater than $x$. 
 

some theorems:
 
for proofs of the following theorems see the address above

Theorem 1: if $(\forall x \in A)(p(x))$ and $(\forall x \in A)(p(x) \to q(x))$ then $(\forall x \in A)(q(x))$. 

Theorem 2: suppose $\{a\}$ is a singleton, i.e. a set with only one element. We have 
"$(\forall x \in \{a\})(p(x))$" is equivalent to $p(a)$. 


Theorem 22: if $(\exists y \in B)(\forall x \in A)(r(x,y))$ then $(\forall x \in A)(\exists y \in B)(r(x,y))$.

here $r(x,y)$ is a property on $A \times B$.

\end{document}
