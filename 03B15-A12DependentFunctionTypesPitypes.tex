\documentclass[12pt]{article}
\usepackage{pmmeta}
\pmcanonicalname{A12DependentFunctionTypesPitypes}
\pmcreated{2013-11-09 4:45:53}
\pmmodified{2013-11-09 4:45:53}
\pmowner{PMBookProject}{1000683}
\pmmodifier{PMBookProject}{1000683}
\pmtitle{A.1.2 Dependent function types ($\Pi$-types)}
\pmrecord{1}{}
\pmprivacy{1}
\pmauthor{PMBookProject}{1000683}
\pmtype{Feature}
\pmclassification{msc}{03B15}

\endmetadata

\usepackage{xspace}
\usepackage{amssyb}
\usepackage{amsmath}
\usepackage{amsfonts}
\usepackage{amsthm}
\makeatletter
\def\lam#1{{\lambda}\@lamarg#1:\@endlamarg\@ifnextchar\bgroup{.\,\lam}{.\,}}
\def\@lamarg#1:#2\@endlamarg{\if\relax\detokenize{#2}\relax #1\else\@lamvar{\@lameatcolon#2},#1\@endlamvar\fi}
\def\@lameatcolon#1:{#1}
\def\@lamvar#1,#2\@endlamvar{(#2\,{:}\,#1)}
\def\tprd#1{\@tprd{#1}\@ifnextchar\bgroup{\tprd}{}}
\def\@tprd#1{\mathchoice{{\textstyle\prod_{(#1)}}}{\prod_{(#1)}}{\prod_{(#1)}}{\prod_{(#1)}}}
\newcommand{\UU}{\ensuremath{\mathcal{U}}\xspace}
\makeatother
\begin{document}
We introduce a primitive constant $c_\Pi$, but write
$c_\Pi(A,\lam{x} B)$ as $\tprd{x:A}B$. Judgments concerning
such expressions and expressions of the form $\lam{x} b$ are introduced by the following rules:
%
\begin{itemize}
\item if $\Gamma \vdash A:\UU_n$ and $\Gamma,x:A \vdash B:\UU_n$, then $\Gamma \vdash \tprd{x:A}B : \UU_n$
\item if $\Gamma, x:A \vdash b:B$ then $\Gamma \vdash (\lam{x} b) : (\tprd{x:A} B)$
\item if $\Gamma\vdash g:\tprd{x:A} B$ and $\Gamma\vdash t:A$ then $\Gamma\vdash g(t):B[t/x]$
\end{itemize}
%
If $x$ does not occur freely in $B$, we abbreviate $\tprd{x:A} B$ as the non-dependent function type 
$A\rightarrow B$ and derive the following rule:
%
\begin{itemize}
\item if $\Gamma\vdash g:A \rightarrow B$ and $\Gamma\vdash t:A$ then $\Gamma\vdash g(t):B$
\end{itemize}
Using non-dependent function types and leaving implicit the context $\Gamma$, the rules above can be written in the following alternative style that we use in the rest of this section of the appendix.
%
\begin{itemize}
\item if $A:\UU_n$ and $B:A\to\UU_n$, then $\tprd{x:A}B(x) : \UU_n$
\item if $x:A \vdash b:B$ then $ \lam{x} b : \tprd{x:A} B(x)$
\item if $g:\tprd{x:A} B(x)$ and $t:A$ then $g(t):B(t)$
\end{itemize}
%


\end{document}
