\documentclass[12pt]{article}
\usepackage{pmmeta}
\pmcanonicalname{LabelledStateTransitionSystem}
\pmcreated{2013-03-22 19:23:20}
\pmmodified{2013-03-22 19:23:20}
\pmowner{CWoo}{3771}
\pmmodifier{CWoo}{3771}
\pmtitle{labelled state transition system}
\pmrecord{17}{42344}
\pmprivacy{1}
\pmauthor{CWoo}{3771}
\pmtype{Definition}
\pmcomment{trigger rebuild}
\pmclassification{msc}{03B45}
\pmclassification{msc}{68Q85}
\pmclassification{msc}{68Q10}
\pmsynonym{labeled state transition system}{LabelledStateTransitionSystem}
\pmsynonym{TS}{LabelledStateTransitionSystem}
\pmsynonym{LTS}{LabelledStateTransitionSystem}
\pmsynonym{STS}{LabelledStateTransitionSystem}
\pmsynonym{LSTS}{LabelledStateTransitionSystem}
\pmrelated{SemiThueSystem}
\pmdefines{state transition system}
\pmdefines{reachability}
\pmdefines{reachable}
\pmdefines{inaccessible state}

\endmetadata

\usepackage{amssymb,amscd}
\usepackage{amsmath}
\usepackage{amsfonts}
\usepackage{mathrsfs}

% used for TeXing text within eps files
%\usepackage{psfrag}
% need this for including graphics (\includegraphics)
%\usepackage{graphicx}
% for neatly defining theorems and propositions
\usepackage{amsthm}
% making logically defined graphics
%%\usepackage{xypic}
\usepackage{pst-plot}

% define commands here
\newcommand*{\abs}[1]{\left\lvert #1\right\rvert}
\newtheorem{prop}{Proposition}
\newtheorem{thm}{Theorem}
\newtheorem{ex}{Example}
\newcommand{\real}{\mathbb{R}}
\newcommand{\pdiff}[2]{\frac{\partial #1}{\partial #2}}
\newcommand{\mpdiff}[3]{\frac{\partial^#1 #2}{\partial #3^#1}}

\begin{document}
A \emph{state transition system} (TS) is a pair $(S,\rightarrow)$, where $\rightarrow$ is a binary relation on $S$.  Elements of $S$ are called \emph{states} and elements of $\rightarrow$ are called \emph{transitions}.  When $(p,q)\in \rightarrow$, we write $p\rightarrow q$.  

A \emph{labelled state transition system} (LTS) is a triple $(S,\Sigma,\rightarrow)$ where $\rightarrow \subseteq S\times \Sigma \times S$.  Elements of $\Sigma$ are called \emph{labels} of the system.  If $(p,\alpha,q)\in \rightarrow$, we write $p\stackrel{\alpha}{\rightarrow} q$.  In this way, an LTS can be thought of as an STS where each transition is labelled (possibly by more than one label).  On the other hand, any TS is just a LTS with one label and every transition has the same label.

Examples of LTS are mostly found in theoretical computer science, such as automata and Turing machines.  An automaton, for example, is just an LTS $(S,\rightarrow)$ with two distinguished mutually disjoint sets (subsets of $S$) of states: a set of \emph{start states}, and a set of \emph{final} or \emph{accept states}.

Given an LTS $(S,\rightarrow)$, if we take the reflexive transitive closure of $\rightarrow$, then we get a binary relation $\Rightarrow$ on $S$ called the \emph{reachability relation}.  In other words, for any $p\in S$, $p\Rightarrow p$, and for any $q\in S$, if $p\Rightarrow q$, then there is a finite sequence $p=p_1,\ldots p_{n+1}=q$, such that $p_i\rightarrow p_{i+1}$ for $i=1,\ldots n$.  When $p\Rightarrow q$, we say that $q$ is \emph{reachable} from $p$.

For an LTS $(S,\rightarrow)$ with a set $I$ of start states, a state is said to be \emph{inaccessible} if it is not reachable from any start state.  If $S$ is finite, one can use the one of the search techniques to find all inaccessible states of $S$.  Here's a basic breadth-first search algorithm:
\begin{enumerate}
\item Initially, circle all start states as accessible.  If there are no start states, skip to step 5.
\item Then for each circled $p$, circle all $q$ such that $p\rightarrow q$.  
\item Then cross out each $p$ after all $q$ such $p\rightarrow q$ have been circled.  
\item Repeat the last two steps for each circled $q$.  Do this until there are no more circled states.  
\item Finally, the states in $S$ are either crossed out or unmarked, and inaccessible states are precisely the unmarked states.
\end{enumerate}

An LTS is said to be \emph{deterministic} if $\rightarrow$ is a partial function.  Otherwise, it is \emph{non-deterministic}.

%%%%%
%%%%%
\end{document}
