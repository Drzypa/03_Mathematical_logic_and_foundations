\documentclass[12pt]{article}
\usepackage{pmmeta}
\pmcanonicalname{DeductionTheoremHoldsForIntuitionisticPropositionalLogic}
\pmcreated{2013-03-22 19:31:32}
\pmmodified{2013-03-22 19:31:32}
\pmowner{CWoo}{3771}
\pmmodifier{CWoo}{3771}
\pmtitle{deduction theorem holds for intuitionistic propositional logic}
\pmrecord{21}{42502}
\pmprivacy{1}
\pmauthor{CWoo}{3771}
\pmtype{Theorem}
\pmcomment{trigger rebuild}
\pmclassification{msc}{03F55}
\pmclassification{msc}{03B20}
\pmrelated{AxiomSystemForIntuitionisticLogic}

\endmetadata

\usepackage{amssymb,amscd}
\usepackage{amsmath}
\usepackage{amsfonts}
\usepackage{mathrsfs}

% used for TeXing text within eps files
%\usepackage{psfrag}
% need this for including graphics (\includegraphics)
%\usepackage{graphicx}
% for neatly defining theorems and propositions
\usepackage{amsthm}
% making logically defined graphics
%%\usepackage{xypic}
\usepackage{pst-plot}

% define commands here
\newcommand*{\abs}[1]{\left\lvert #1\right\rvert}
\newtheorem{prop}{Proposition}
\newtheorem{thm}{Theorem}
\newtheorem{ex}{Example}
\newcommand{\real}{\mathbb{R}}
\newcommand{\pdiff}[2]{\frac{\partial #1}{\partial #2}}
\newcommand{\mpdiff}[3]{\frac{\partial^#1 #2}{\partial #3^#1}}

\begin{document}
In this entry, we show that the deduction theorem below holds for intuitionistic propositional logic.  We use the axiom system provided in \PMlinkname{this entry}{AxiomSystemForIntuitionisticLogic}.

\begin{thm} If $\Delta, A \vdash_i B$, where $\Delta$ is a set of wff's of the intuitionistic propositional logic, then $\Delta \vdash_i A\to B$. \end{thm}

The proof is very similar to that of the classical propositional logic, given \PMlinkname{here}{DeductionTheoremHoldsForClassicalPropositionalLogic}, in that it uses induction on the length of the deduction of $B$.  In fact, the proof is simpler as only two axiom schemas are used: $A\to (B\to A)$ and $(A\to B) \to ((A\to (B\to C)) \to (A\to C))$. 

\begin{proof}  There are two main cases to consider:
\begin{itemize}
\item
If $B$ is an axiom or in $\Delta\cup \lbrace A\rbrace$, then $$B, B\to (A\to B), A\to B$$ is a deduction of $A\to B$ from $\Delta$, where $A\to B$ is obtained by modus ponens applied to $B$ and the axiom $B\to (A\to B)$.  So $\Delta \vdash_i A\to B$.
\item
Now, suppose that $$A_1,\ldots, A_n$$ is a deduction of $B$ from $\Delta\cup \lbrace A \rbrace$, with $B$ obtained from earlier formulas by modus ponens.

We use induction on the length $n$ of deduction of $B$.  Note that $n\ge 3$.  If $n=3$, then $C$ and $C\to B$ are either axioms or in $\Delta\cup \lbrace A\rbrace$.  
\begin{itemize}
\item If $C$ is $A$, then $C\to B$ is either an axiom or in $\Delta$.  So $\Delta \vdash_i A\to B$.
\item If $C\to B$ is $A$, then $C$ is either an axiom or in $\Delta$.  Then 
\begin{eqnarray*}
&&\mathcal{E}_0, C\to (A\to C), C, A\to C, (A\to C)\to ((A\to (C\to B))\to (A\to B)), \\
&&(A\to (C\to B))\to (A\to B), A\to B
\end{eqnarray*}
is a deduction of $A\to B$ from $\Delta$, where $\mathcal{E}_0$ is a deduction of the theorem $A\to A$, followed by an axiom instance, then $C$, then the result of modus ponens, then an axiom instance, and finally two applications of modus ponens.  Note the second to the last formula is just $(A\to A)\to (A\to B)$.
\item If $C$ and $C\to B$ are axioms or in $\Delta$, then $\Delta \vdash_i A\to B$ based on the deduction $C, C\to B, B, B\to (A\to B), A\to B$.
\end{itemize}

Next, assume there is a deduction $\mathcal{E}$ of $B$ of length $n>3$.  So one of the earlier formulas is $A_k \to B$, and a subsequence of $\mathcal{E}$ is a deduction of $A_k \to B$, which has length less than $n$, and therefore by induction, $\Delta \vdash_i A \to (A_k \to B)$.  Likewise, a subsequence of $\mathcal{E}$ is a deduction of $A_k$, so by induction, $\Delta \vdash_i A \to A_k$.  With the axiom instance $(A\to A_k)\to ((A\to (A_k\to B))\to (A\to B))$, and two applications of modus ponens, we get $\Delta\vdash_i A\to B$ as required.
\end{itemize}
In both cases, $\Delta \vdash_i A\to B$, and the proof is complete.
\end{proof}

\textbf{Remark}  The deduction theorem can be used to prove the deduction theorem for the first and second order intuitionistic predicate logic.

\begin{thebibliography}{7}
\bibitem{JR} J. W. Robbin, {\it Mathematical Logic, A First Course}, Dover Publication (2006)
\end{thebibliography}

%%%%%
%%%%%
\end{document}
