\documentclass[12pt]{article}
\usepackage{pmmeta}
\pmcanonicalname{RecursiveFunctionIsURMcomputable}
\pmcreated{2013-03-22 19:04:51}
\pmmodified{2013-03-22 19:04:51}
\pmowner{CWoo}{3771}
\pmmodifier{CWoo}{3771}
\pmtitle{recursive function is URM-computable}
\pmrecord{7}{41969}
\pmprivacy{1}
\pmauthor{CWoo}{3771}
\pmtype{Result}
\pmcomment{trigger rebuild}
\pmclassification{msc}{03D10}
\pmclassification{msc}{68Q05}
\pmrelated{RecursiveFunction}

\endmetadata

\usepackage{amssymb,amscd}
\usepackage{amsmath}
\usepackage{amsfonts}
\usepackage{mathrsfs}

% used for TeXing text within eps files
%\usepackage{psfrag}
% need this for including graphics (\includegraphics)
%\usepackage{graphicx}
% for neatly defining theorems and propositions
\usepackage{amsthm}
% making logically defined graphics
%%\usepackage{xypic}
\usepackage{pst-plot}

% define commands here
\newcommand*{\abs}[1]{\left\lvert #1\right\rvert}
\newtheorem{prop}{Proposition}
\newtheorem{thm}{Theorem}
\newtheorem{ex}{Example}
\newcommand{\real}{\mathbb{R}}
\newcommand{\pdiff}[2]{\frac{\partial #1}{\partial #2}}
\newcommand{\mpdiff}[3]{\frac{\partial^#1 #2}{\partial #3^#1}}
\begin{document}
\begin{prop}  Every recursive function is URM-computable function. \end{prop}

The proof can be broken down in several simple steps.

\begin{prop} The zero function, the successor function, and the projection functions are URM-computable.  \end{prop}
\begin{proof}  The zero function is computed by $Z(1)$, the successor function is computed by $S(1)$, and for any $n>0$, the $i$-th projection function $p_i^n(x_1,\ldots,x_n)=x_i$ is computed by $T(i,1)$.
\end{proof}

\begin{prop} URM-computability is closed under functional composition. \end{prop}
\begin{proof} This is proved in the entry on combining URMs. \end{proof}

\begin{prop} URM-computability is closed under primitive recursion. \end{prop}
\begin{proof}
Suppose $f:\mathbb{N}^m\to \mathbb{N},g:\mathbb{N}^{m+2}\to \mathbb{N}$ are computed by $M,N$ respectively.  Let $h:\mathbb{N}^{m+1}\to \mathbb{N}$ be obtained from $f,g$ by primitive recursion, namely, 
\begin{eqnarray*}
h(0,x_1,\ldots,x_m)&:=&f(x_1,\ldots,x_m) \\
h(n+1,x_1,\ldots,x_m)&:=& g(h(x_1,\ldots,x_m,n),n,x_1,\ldots,x_m).
\end{eqnarray*}
Let $P$ be the following URM:
\begin{eqnarray*}
&& T(1,p+1),T(2,p+2),\cdots, T(m+1,p+m+1),\\
&& M[p+2,\ldots,p+m+1;p+m+2],J(p+1,p+m+3,q),S(p+m+3),\\
&& N[p+2,\ldots,p+m+1,p+m+3,p+m+2;p+m+2],\\
&& J(1,1,m+3),T(p+m+2,1).
\end{eqnarray*}
where $p=\max(m+2,\rho(M),\rho(N))$ and $q=m+7$.  The program works as follows:
\begin{description}
\item[$\boldsymbol{I_1,\ldots,I_{m+1}}$:] transfer the first $m+1$ registers to another are on the tape:
$$T(1,p+1),T(2,p+2),\cdots, T(m+1,p+m+1)$$
\item[$\boldsymbol{I_{m+2}}$:] compute $h(0,x_1,\ldots,x_m)$ using $M[p+2,\ldots,p+m+1;p+m+2]$
\item[$\boldsymbol{I_{m+3}}$:] if the content of register $p+1$ (formerly the content of register $1$) is the same as the content of $p+m+3$ (initially set to $0$), go to the last instruction whose index is $q(=m+7)$; otherwise continue to the next instruction: $J(p+1,p+m+3,q)$
\item[$\boldsymbol{I_{m+4}}$:] increment register $p+m+3$ by $1$: $S(p+m+3)$
\item[$\boldsymbol{I_{m+5}}$:] compute $h(i,x_1,\ldots,x_m)$, where $i$ is the content of register $p+m+3$, using $$N[p+2,\ldots,p+m+1,p+m+3,p+m+2;p+m+2]$$
\item[$\boldsymbol{I_{m+6}}$:] go to instruction $m+3$: $J(1,1,m+3)$
\item[$\boldsymbol{I_{m+7}}$:] transfer result back to register $1$: $T(p+m+2,1)$.
\end{description}
Note that if $(x_1,\ldots,x_m,n)\in \operatorname{dom}(h)$, then $P(x_1,\ldots,x_m,n)\downarrow \! h(x_1,\ldots,x_m,n)$.  Otherwise, $h(x_1,\ldots,x_m,n)$ is undefined.  This can happen either $f(x_1,\ldots, x_m)$ is undefined, in which case $M$ diverges, $g(x_1,\ldots, x_m,i,h(x_1,\ldots,x_m,i))$ is undefined, in which case $N$ diverges, or $h(x_1,\ldots,x_m,i)\ne 0$ for all $i\in \mathbb{N}$, in which case $P$ loops indefinitely.  In all cases, $P$ diverges.  This shows that $P$ computes $h$.
\end{proof}
\begin{prop} URM-computability is closed under minimization. \end{prop}
\begin{proof}  Suppose $f:\mathbb{N}^{m+1}\to \mathbb{N}$ is computed by $M$.  Let $g:\mathbb{N}^{m} \to \mathbb{N}$ be obtained from $f$ by minimization.  In other words, for any $\boldsymbol{x}=(x_1,\ldots,x_m)\in \mathbb{N}^m$, set
$$A(\boldsymbol{x}):=\lbrace y\in \mathbb{N}\mid (z,\boldsymbol{x}) \in \operatorname{dom}(f)\mbox{ for all }z\le y \mbox{ and }f(y,\boldsymbol{x})=0 \rbrace$$
and define 
\begin{displaymath}
g(\boldsymbol{x}):= \left\{
\begin{array}{ll}
\min A(\boldsymbol{x}) & \textrm{if } A(\boldsymbol{x})\ne \varnothing, \\
\textrm{undefined} & \textrm{otherwise.}
\end{array}
\right.
\end{displaymath}
Let $Q$ be the following URM:
\begin{eqnarray*}
&& T(1,p+1),T(2,p+2),\cdots, T(m,p+m),M[p+m+1,p+1,\ldots,p+m;1],\\
&& J(1,p+m+2,q),S(p+m+1),J(1,1,m+1),T(p+m+1,1)
\end{eqnarray*}
where $p=\max(m+1,\rho(M))$ and $q=m+5$.  The program works as follows:
\begin{description}
\item[$\boldsymbol{I_1,\ldots,I_m}$:] transfer the first $m$ registers to another are on the tape:
$$T(1,p+1),T(2,p+2),\cdots, T(m,p+m)$$
\item[$\boldsymbol{I_{m+1}}$:] compute $f(0,x_1,\ldots,x_m)$ using $M[p+m+1,p+1,\ldots,p+m;1]$, where the content of register $p+m+1$ is set to $0$ initially.
\item[$\boldsymbol{I_{m+2}}$:] if the content of register $p+m+2$ (which is always $0$) is the same as the content of register $1$ (value of $f(0,x_1,\ldots,x_m)$, if defined), go to the last instruction whose index is $q(=m+5)$; otherwise continue to the next instruction: $J(1,p+m+2,q)$
\item[$\boldsymbol{I_{m+3}}$:] increment register $p+m+1$ by $1$ (counter): $S(p+m+1)$
\item[$\boldsymbol{I_{m+4}}$:] go to instruction $m+1$: $J(1,1,m+1)$
\item[$\boldsymbol{I_{m+5}}$:] transfer content of register $p+m+1$ to register $1$: $T(p+m+1,1)$.
\end{description}
If $(x_1,\ldots,x_m)\in \operatorname{dom}(g)$, then $Q(x_1,\ldots,x_m)\downarrow \! g(x_1,\ldots,x_m)$.  Otherwise, $g(x_1,\ldots,x_m)$ is undefined, which can happen either when $f(i,x_1,\ldots, x_m)\ne 0$ for all $i\in \mathbb{N}$, in which case $Q$ loops indefinitely, or $f(i,x_1,\ldots, x_m)$ is undefined, while $f(j,x_1,\ldots,x_m)$ are defined and non-zero, for all $j<i$, in which case $M$ diverges.  In both cases, $Q$ diverges.  Hence $Q$ computes $g$.
\end{proof}

Since a recursive function is obtained by a finite application of functional operations specified in Propositions 3,4,5 on the basic arithmetic functions specified in Proposition 2, every recursive function is URM computable as result, proving Proposition 1.
%%%%%
%%%%%
\end{document}
