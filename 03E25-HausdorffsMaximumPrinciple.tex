\documentclass[12pt]{article}
\usepackage{pmmeta}
\pmcanonicalname{HausdorffsMaximumPrinciple}
\pmcreated{2013-03-22 13:04:42}
\pmmodified{2013-03-22 13:04:42}
\pmowner{CWoo}{3771}
\pmmodifier{CWoo}{3771}
\pmtitle{Hausdorff's maximum principle}
\pmrecord{12}{33491}
\pmprivacy{1}
\pmauthor{CWoo}{3771}
\pmtype{Theorem}
\pmcomment{trigger rebuild}
\pmclassification{msc}{03E25}
\pmsynonym{maximum principle}{HausdorffsMaximumPrinciple}
\pmsynonym{Hausdorff maximality theorem}{HausdorffsMaximumPrinciple}
\pmrelated{ZornsLemma}
\pmrelated{AxiomOfChoice}
\pmrelated{ZermelosWellOrderingTheorem}
\pmrelated{ZornsLemmaAndTheWellOrderingTheoremEquivalenceOfHaudorffsMaximumPrinciple}
\pmrelated{EveryVectorSpaceHasABasis}
\pmrelated{MaximalityPrinciple}

% this is the default PlanetMath preamble.  as your knowledge
% of TeX increases, you will probably want to edit this, but
% it should be fine as is for beginners.

% almost certainly you want these
\usepackage{amssymb}
\usepackage{amsmath}
\usepackage{amsfonts}

% used for TeXing text within eps files
%\usepackage{psfrag}
% need this for including graphics (\includegraphics)
%\usepackage{graphicx}
% for neatly defining theorems and propositions
\usepackage{amsthm}
% making logically defined graphics
%%%\usepackage{xypic}

% there are many more packages, add them here as you need them

% define commands here

\newcommand{\sR}[0]{\mathbb{R}}
\newcommand{\sC}[0]{\mathbb{C}}
\newcommand{\sN}[0]{\mathbb{N}}
\newcommand{\sZ}[0]{\mathbb{Z}}

% The below lines should work as the command
% \renewcommand{\bibname}{References}
% without creating havoc when rendering an entry in 
% the page-image mode.
\makeatletter
\@ifundefined{bibname}{}{\renewcommand{\bibname}{References}}
\makeatother
\begin{document}
{\bf Theorem} 
Let $X$ be a partially ordered set. Then there exists a maximal totally 
ordered subset of $X$. 

The Hausdorff's maximum principle is one of the many theorems equivalent
to the 
\PMlinkname{axiom of choice}{AxiomOfChoice}. 
The below proof uses Zorn's lemma, which
is also equivalent to the 
\PMlinkescapetext{axiom of choice}. 

\begin{proof}
Let $S$ be the set of all totally ordered subsets of $X$.  $S$ is not empty, since the empty set is an element of $S$.  Partial order $S$ by inclusion.  Let $\tau$ be a chain (of elements) in $S$.  Being each totally ordered, the union of all these elements of $\tau$ is again a totally ordered subset of $X$, and hence an element of $S$, as is easily verified. This shows that $S$, ordered by inclusion, is inductive. The result now follows from Zorn's lemma.
\end{proof}
%%%%%
%%%%%
\end{document}
