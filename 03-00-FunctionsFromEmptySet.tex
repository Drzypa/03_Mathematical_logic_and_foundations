\documentclass[12pt]{article}
\usepackage{pmmeta}
\pmcanonicalname{FunctionsFromEmptySet}
\pmcreated{2013-03-22 18:08:20}
\pmmodified{2013-03-22 18:08:20}
\pmowner{rspuzio}{6075}
\pmmodifier{rspuzio}{6075}
\pmtitle{functions from empty set}
\pmrecord{5}{40693}
\pmprivacy{1}
\pmauthor{rspuzio}{6075}
\pmtype{Definition}
\pmcomment{trigger rebuild}
\pmclassification{msc}{03-00}

\endmetadata

% this is the default PlanetMath preamble.  as your knowledge
% of TeX increases, you will probably want to edit this, but
% it should be fine as is for beginners.

% almost certainly you want these
\usepackage{amssymb}
\usepackage{amsmath}
\usepackage{amsfonts}

% used for TeXing text within eps files
%\usepackage{psfrag}
% need this for including graphics (\includegraphics)
%\usepackage{graphicx}
% for neatly defining theorems and propositions
%\usepackage{amsthm}
% making logically defined graphics
%%%\usepackage{xypic}

% there are many more packages, add them here as you need them

% define commands here

\begin{document}
Sometimes, it is useful to consider functions whose domain is the 
empty set.  Given a set, there exists exactly one function from 
the the empty set to that set.  The rationale for this comes from 
carefully examining the definition of function in this degenerate 
case.  Recall that, in set theory, a function from a set $D$ to a 
set $R$ is a set of ordered pairs whose first element lies in $D$ 
and whose second element lies in $R$ such that every element of $D$
appears as the first element of exactly one ordered pair.  If we 
take $D$ to be the empty set, we see that this definition is satisfied
if we take our function to be set of no ordered pairs --- since there
are no elements in the empty set, it is technically correct to say
that every element of the empty set appears as a first element of
an ordered pair which is an element of the empty set!

This observation turns out to be more than just an exercise in 
logic, being useful in several contexts.  Given a set $S$ and a positive
integer $n$, we may define $S^n$ as the set of all functions from 
$\{1, \ldots, n\}$ to $S$.   If we choose $n=0$, then $S^0$ consists
of all maps from the empty set to $S$, hence consists of exactly one
element --- see the entry on empty products for a discussion of the
usefulness of this convention.  In category theory, it turns out that 
functions from the empty set are important because they make the empty 
set be an initial object in this category.

%%%%%
%%%%%
\end{document}
