\documentclass[12pt]{article}
\usepackage{pmmeta}
\pmcanonicalname{CantorNormalForm}
\pmcreated{2013-03-22 15:33:01}
\pmmodified{2013-03-22 15:33:01}
\pmowner{rspuzio}{6075}
\pmmodifier{rspuzio}{6075}
\pmtitle{Cantor normal form}
\pmrecord{9}{37448}
\pmprivacy{1}
\pmauthor{rspuzio}{6075}
\pmtype{Theorem}
\pmcomment{trigger rebuild}
\pmclassification{msc}{03E10}
%\pmkeywords{ordinal}
%\pmkeywords{normal}
%\pmkeywords{Cantor}
%\pmkeywords{basis}

% this is the default PlanetMath preamble.  as your knowledge
% of TeX increases, you will probably want to edit this, but
% it should be fine as is for beginners.

% almost certainly you want these
\usepackage{amssymb}
\usepackage{amsmath}
\usepackage{amsfonts}

% used for TeXing text within eps files
%\usepackage{psfrag}
% need this for including graphics (\includegraphics)
%\usepackage{graphicx}
% for neatly defining theorems and propositions
\usepackage{amsthm}
% making logically defined graphics
%%%\usepackage{xypic}

% there are many more packages, add them here as you need them

% define commands here
\begin{document}
\newtheorem*{thm}{Ordinal Normal Form}
\begin{thm}[Cantor]
For ordinal numbers $\alpha\geq 2$ and $\gamma\geq 1$ there is a unique $n$ such that there exist unique $\beta_0>\cdots>\beta_n$ and $0<\delta_0<\alpha,\ldots,0<\delta_n<\alpha$ such that $\gamma=\alpha^{\beta_0}\cdot\delta_0+\cdots+\alpha^{\beta_n}\cdot\delta_n$.
\end{thm}

This theorem is often referred to as the \emph{Cantor Normal Form of $\gamma$ in the base of $\alpha$}.
%%%%%
%%%%%
\end{document}
