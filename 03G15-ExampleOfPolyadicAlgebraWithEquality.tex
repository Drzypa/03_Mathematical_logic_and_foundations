\documentclass[12pt]{article}
\usepackage{pmmeta}
\pmcanonicalname{ExampleOfPolyadicAlgebraWithEquality}
\pmcreated{2013-03-22 17:55:24}
\pmmodified{2013-03-22 17:55:24}
\pmowner{CWoo}{3771}
\pmmodifier{CWoo}{3771}
\pmtitle{example of polyadic algebra with equality}
\pmrecord{5}{40417}
\pmprivacy{1}
\pmauthor{CWoo}{3771}
\pmtype{Example}
\pmcomment{trigger rebuild}
\pmclassification{msc}{03G15}
\pmdefines{functional equality algebra}
\pmdefines{functional equality}
\pmdefines{functional polyadic algebra with equality}

\endmetadata

\usepackage{amssymb,amscd}
\usepackage{amsmath}
\usepackage{amsfonts}
\usepackage{mathrsfs}

% used for TeXing text within eps files
%\usepackage{psfrag}
% need this for including graphics (\includegraphics)
%\usepackage{graphicx}
% for neatly defining theorems and propositions
\usepackage{amsthm}
% making logically defined graphics
%%\usepackage{xypic}
\usepackage{pst-plot}

% define commands here
\newcommand*{\abs}[1]{\left\lvert #1\right\rvert}
\newtheorem{prop}{Proposition}
\newtheorem{thm}{Theorem}
\newtheorem{ex}{Example}
\newcommand{\real}{\mathbb{R}}
\newcommand{\pdiff}[2]{\frac{\partial #1}{\partial #2}}
\newcommand{\mpdiff}[3]{\frac{\partial^#1 #2}{\partial #3^#1}}
\begin{document}
Recall that given a triple $(A,I,X)$ where $A$ is a Boolean algebra, $I$ and $X\neq \varnothing$ are sets.  we can construct a polyadic algebra $(B,I,\exists,S)$ called the functional polyadic algebra for $(A,I,X)$.  In this entry, we will construct an example of a polyadic algebra with equality called the \emph{functional polyadic algebra with equality} from $(B,I,\exists,S)$.

We start with a simpler structure.  Let $B$ be an arbitrary Boolean algebra, $I$ and $X\neq \varnothing$ are sets.  Let $Y=X^I$, the set of all $I$-indexed $X$-valued sequences, and $Z=B^Y$, the set of all functions from $Y$ to $B$.  Call the function $e:I\times I\to Z$ the \emph{functional equality associated with} $(B,I,X)$, if for each $i,j\in I$, $e(i,j)$ is the function defined by
\begin{displaymath}
e(i,j)(x):=\left\{
\begin{array}{ll}
1 & \textrm{if }x_i=x_j, \\
0 & \textrm{otherwise.}
\end{array}
\right.
\end{displaymath}
The quadruple $(B,I,X,e)$ is called a \emph{functional equality algebra}.

Now, $B$ will have the additional structure of being a polyadic algebra.  Start with a Boolean algebra $A$, and let $I$ and $X$ be defined as in the last paragraph.  Then, as stated above in the first paragraph, and illustrated in \PMlinkname{here}{ExampleOfPolyadicAlgebra}, $(B,I,\exists,S)$ is a polyadic algebra (called the functional polyadic algebra for $(A,I,X)$).  Using the $B$ just constructed, the quadruple $(B,I,X,e)$ is a functional equality algebra, and is called the \emph{functional polyadic algebra with equality} for $(A,I,X)$.  

It is not hard to show that $e$ is an equality predicate on $C=(B,I,\exists,S)$, and as a result $(C,e)$ is a polyadic algebra with equality.

\begin{thebibliography}{8}
\bibitem{ph} P. Halmos, \emph{Algebraic Logic}, Chelsea Publishing Co. New York (1962).
\end{thebibliography}
%%%%%
%%%%%
\end{document}
