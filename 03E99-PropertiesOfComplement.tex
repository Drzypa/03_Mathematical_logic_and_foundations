\documentclass[12pt]{article}
\usepackage{pmmeta}
\pmcanonicalname{PropertiesOfComplement}
\pmcreated{2013-03-22 17:55:32}
\pmmodified{2013-03-22 17:55:32}
\pmowner{CWoo}{3771}
\pmmodifier{CWoo}{3771}
\pmtitle{properties of complement}
\pmrecord{5}{40419}
\pmprivacy{1}
\pmauthor{CWoo}{3771}
\pmtype{Derivation}
\pmcomment{trigger rebuild}
\pmclassification{msc}{03E99}

\usepackage{amssymb,amscd}
\usepackage{amsmath}
\usepackage{amsfonts}
\usepackage{mathrsfs}

% used for TeXing text within eps files
%\usepackage{psfrag}
% need this for including graphics (\includegraphics)
%\usepackage{graphicx}
% for neatly defining theorems and propositions
\usepackage{amsthm}
% making logically defined graphics
%%\usepackage{xypic}
\usepackage{pst-plot}

% define commands here
\newcommand*{\abs}[1]{\left\lvert #1\right\rvert}
\newtheorem{prop}{Proposition}
\newtheorem{thm}{Theorem}
\newtheorem{ex}{Example}
\newcommand{\real}{\mathbb{R}}
\newcommand{\pdiff}[2]{\frac{\partial #1}{\partial #2}}
\newcommand{\mpdiff}[3]{\frac{\partial^#1 #2}{\partial #3^#1}}
\begin{document}
Let $X$ be a set and $A,B$ are subsets of $X$.

\begin{enumerate}
\item $(A^{\complement})^\complement=A$.
\begin{proof}  $a\in (A^{\complement})^\complement$ iff $a\notin A^{\complement}$ iff $a\in A$.
\end{proof}
\item $\emptyset^\complement = X$.
\begin{proof}
$a\in \emptyset^\complement$ iff $a\notin \emptyset$ iff $a\in X$.
\end{proof}
\item $X^\complement = \emptyset$.
\begin{proof}
$a\in X^\complement$ iff $a\notin X$ iff $a\in \emptyset$.
\end{proof}
\item $A\cup A^\complement = X$.
\begin{proof}
$a\in A\cup A^\complement$ iff $a\in A$ or $a\in A^\complement$ iff $a\in A$ or $a\notin A$ iff $a\in X$.
\end{proof}
\item $A\cap A^\complement =\emptyset$.
\begin{proof}
$a\in A\cap A^\complement$ iff $a\in A$ and $a\in A^\complement$ iff $a\in A$ and $a\notin A$ iff $a\in \emptyset$.
\end{proof}
\item $A\subseteq B$ iff $B^\complement\subseteq A^\complement$.
\begin{proof}
Suppose $A\subseteq B$.  If $a\in B^\complement$, then $a\notin B$, so $a\notin A$, or $a\in A^\complement$.  This shows that $B^\complement\subseteq A^\complement$.  On the other hand, if $B^\complement\subseteq A^\complement$, then by applying what's just been proved, $A=(A^\complement)^\complement \subseteq (B^\complement)^\complement =B$.
\end{proof}
\item $A\cap B=\emptyset$ iff $A\subseteq B^\complement$.
\begin{proof}  Suppose $A\cap B=\emptyset$.  If $a\in A$, then $a\in B^\complement$, or $a\notin B$, which implies that $A\cap B=\emptyset$.  Suppose next that $A\subseteq B^\complement$.  If there is $a\in A\cap B$, then $a\in B$ and $a\in A$.  But the second containment implies that $a\in B^\complement$, which contradicts the first containment.
\end{proof}
\item $A\setminus B = A\cap B^\complement$, where the complement is taken in $X$.
\begin{proof}  $a\in A\setminus B$ iff $a\in A$ and $a\notin B$ iff $a\in A$ and $a\in B^\complement$ iff $a\in A\cap B^\complement$.
\end{proof}
\item (de Morgan's laws) $(A \cup B)^\complement = A^\complement \cap B^\complement$ and $(A \cap B)^\complement = A^\complement \cup B^\complement$.
\begin{proof}  See \PMlinkname{here}{DeMorgansLawsProof}.
\end{proof}
\end{enumerate}
%%%%%
%%%%%
\end{document}
